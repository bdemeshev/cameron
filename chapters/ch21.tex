


\part{Модели анализа панельных данных}


Модели анализа данных пространственного типа имеют ряд неотъемлемых ограничений.  Главным образом, речь идет о равновесных моделях, которые не проливают свет на межвременную зависимость событий. Кроме того, они не могут разрешить фундаменталные вопросы о неизменных поведенческих процессах. Такие неизменные процессы могут быть поведенческими, т.е. являющимися результатом истинной зависимости, или могут быть кажущимися, т.е. являющимися результатом невозможности учитывать гетерогенное поведение. Вследствие того, что панельные данные, также называемые лонгитюдными, содержат периодически повторяющиеся наблюдения одних и тех же субъектов, они обладают большим потенциалом в разрешении таких вопросов, которые модели анализа пространственных данных не в состоянии решить.
Главы с 21 по 23 описывают методы анализа панельных данных. Мы систематически двигаемся от линейных моделей для непрерывных переменных в главе 21 к нелинейным моделям панельных данных для ограниченных зависимых переменных в главе 23, рассматривая фиксированные и случайные эффекты. На протяжении этих трех глав 
мы постоянно возвращаемся к вопросу о важности использования робастных для панельных данных методов получения статистических выводов.

Глава 21, в которой содержится обзор основных результатов для моделей линейной регрессии для панельных данных, легко осваивается на базе знаний о линейных регрессионных моделях; она не требует знания материала частей 2-4. Даже тем, кто заинтересован исключительно в продвинутом материале, мы советуем в первую очередь внимательно ознакомится  с содержанием этой главы, чтобы получить представление об основных концепциях и определениях, касающихся анализа панельных данных.

В главе 22 рассматриваются расширенные модели, в особенности динамические модели анализа панельных данных, которые позволяют анализировать марковскую зависимость текущих переменных. Анализ проводится в рамках ММП, который широко используется практиками этой области. Анализ довольно-таки трудный и влечет за собой множество деталей, которые необходимо учитывать. Хорошее знание ММП облегчает понимание основных результатов данной главы.

Нелинейные модели панельных данных главы 23 в общем не являются продолжением результатов глав 21 и 22. Там уже представлено меньше результатов для панельных данных с ограниченными зависимыми переменными. Несмотря на это, глава 23 начинается с анализа общих вопросов и подходов. Дальнейшие разделы главы содержат результаты для аналогичных моделей пространственных данных, изученных в главе 4. В этих частях описывается анализ четырех категорий моделей: бинарных, количественных, цензурированных и моделей длительности. Они должны быть доступны для среднего читателя, имеющего знания о соответствующих моделях  данных пространственного типа.

\chapter{Линейные модели панельных данных: основы}
\section{Вступление}
\textbf{Панельные данные} представляют собой одни и те же повторяющиеся наблюдения, относящиеся к разным периодам времени. В микроэкономических приложениях это обычно фирмы или индивидуумы. По-другому, эти данные называют \textbf{лонгитюдными данными} или повторяющимися измерениями (\textbf{repeated measures}). Мы будем рассматривать главным образом \textbf{короткие панели}, соответствующие большим выборкам индивидуумов. Они наблюдаются в течение короткого периода времени, в то время как длинные панели, к примеру, небольшие группы стран, наблюдаются в течение многих периодов.

Основное преимущество панельных данных  --- это увеличение точности оценивания. Это результат увеличения количества наблюдений вследствие комбинирования (\textbf{pooling}) нескольких периодов для каждого индивидуального наблюдения. Однако, для получения верных статистических выводов для данного наблюдения необходимо учитывать возможную корреляцию во времени ошибок регрессионной модели. В частности, обычная формула для нахождения стандартных ошибок в модели сквозной регрессии переоценивает выигрыш от увеличения точности, приводя к недооцененным стандартным ошибкам и  резко увеличенным t-статистикам.

Во-вторых, панельные данные привлекают возможностью получить состоятельную оценку параметров в модели с \textbf{фиксированными эффектами}, которые учитывают коррелируемую с регрессорами ненаблюдаемую гетерогенность между индивидуами. Такая ненаблюдаемая гетерогенность приводит к \textbf{смещению в связи с опущенной переменной} (omitted variable bias), которое можно скорректировать с помощью метода инструментальных переменных с использованием только одной группы наблюдений. Это однако сложно осуществить на практике, так как найти годные инструменты достаточно трудно. Данные короткой, к примеру, даже двухпериодной панели позволяют преодолеть эту проблему альтернативным способом в предположении, что ненаблюдаемые индивидуальные эффекты аддитивны и не изменяются со временем.

Большинство дисциплин в прикладной статистике, кроме микроэконометрики, предполагают, что ненаблюдаемые индивидуальные эффекты распределены независимо от регрессоров. В таком случае они называются \textbf{случайными эффектами}, хотя лучшим определением было бы {\it чисто} случайные эффекты. Это более строгое предположение дает преимущество  по сравнению с фиксированными эффектами. Оно позволяет получить состоятельные оценки всех параметров, включая коэффициенты регрессоров, не изменяющихся во времени. Однако, случайные эффекты и оценки коэффициентов не состоятельны, если истинной моделью является модель с фиксированными эффектами. Экономисты считают, что часто данное предположение модели со случайными эффектами не выполняется на реальных данных.

Третье преимущество панельных данных состоит в том, что они позволяют более детально изучить \textbf{динамику} индивидуального поведения, чем при использовании пространственных данных. Так анализ пространственных данных может выявить уровень бедности в 20\%, однако для выявления того, входят ли в эти 20 \% те же самые индивидуумы или нет, необходим анализ панельных данных. Также панельные данные позволяют определить, например, является ли  причиной высокой автокорреляции индивидуальных доходов или продолжительности безработицы
специфическая индивидуальная тенденция к получению высоких доходов или к длительному безработному периоду, или же это является следствием того, что индивидуумы в прошлом имели высокие доходы или были безработными. Это рассматривается в главе 22.

Линейные модели панельных данных и связанные с ними оценки просты для понимания в отличие от фундаментальных вопросов о необходимости использования фиксированных эффектов. Рассматриваемые алгебраические методы, используемые для выведения свойств оценок в моделях панельных данных, не должны уводить от понимания основ: статистические свойства оценок в моделях панельных данных меняются в зависимости от предполагаемой модели и от предположений о ненаблюдаемых эффектах. Кроме того, значительная доля алгебраических выводов не распространяются на нелинейные модели панельных данных.

Текущая глава описывает основные виды оценок линейных моделей панельных данных. В подробном введении в разделе 21.2 и 21.3 представлены часто используемые модели и оценки, а также объяснен практический пример зависимости годовых часов работы от зарплат. Важное различие между моделями с фиксированными и случайными эффектами объясняется в разделе 21.4. Разделы 21.5-21.7 содержат дополнительные детали, связанные с оцениванием соответственно модели сквозной регрессии, модели с индивидуальными фиксированными эффектами и модели с индивидуальными случайными эффектами. Раздел 21.8 рассматривает другие основные аспекты, такие как получение статистических выводов и построение прогнозов в линейных моделях панельных данных.

\section{Обзор моделей и оценок}
Панельные данные содержат информацию о поведении индивидуумов как среди индивидуумов, так и во времени. 

Даже в случае линейной регрессии стандартный анализ панельных данных предполагает использование гораздо большего ряда моделей и оценок, чем в случае с пространственными данными. Несколько стандартных моделей представлены в разделе 21.2.1, после чего в разделе 21.2.2 описаны некоторые оценки. В таблице \ref{Tab:21.1}, обобщены основные модели и оценки. В таблице указано, что некоторые оценки несостоятельны, если процесс, порождающий данные, является моделью с фиксированными индивидуальными эффектами.

В случае с панельными данными получить верные стандартные ошибки для оценок сложнее, чем в случае с пространственными данными. Необходимо учитывать корреляцию во времени для данного индивидуума, а также возможную гетероскедастичность. Эта тема затрагивается в разделе 21.2.3.

\begin{table}[ht]
\centering
\caption[]{\itЛинейные модели панельных данных: основные оценки и модели ${}^a$}
\begin{tabular}{p{4cm} p{3.2cm} p{3.2cm} p{3.2cm}}
\hline \hline
				& \multicolumn{3}{c}{\bf{Предполагаемая модель}} \\
\bf{Оценка} $\beta$ 	& \bf{Модель сквозной регрессии (21.1)}	& \bf{Модель со случайными эффектами (21.3)  и (21.5)} &  \bf{Модель с фиксированными эффектами (21.3)} \\
МНК оценка модели сквозной регрессии (21.1)	&	Состоятельная	 &	Состоятельная	&	Несостоятельная	 \\

Between (21.7)	&	Состоятельная	 &	Состоятельная	&	Несостоятельная	 \\
Within (или оценка модели с фиксированными эффектами) (21.8)	&	Состоятельная	 &	Состоятельная	 &	Состоятельная	 \\
Модель в первых разностях & Состоятельная	 &	Состоятельная	 &	Состоятельная	 \\
Оценка модели со случайными эффектами	& Состоятельная	 &	Состоятельная	 &	Несостоятельная	 \\
\hline \hline
\multicolumn{4}{p{15cm}}{${}^a$ В этой таблице рассматриваются только состоятельность оценок $\beta$.  Для корректного вычисления стандартных ошибок см. раздел 21.2.3.}
\end{tabular}
\label{Tab:21.1}
\end{table}

\subsection{Модели анализа панельных данных}

Самая общая линейная модель для панельных данных предполагает, что свобоный член и коэффциенты наклона могут варьироваться по индивидуальным наблюдениям и во времени:
\begin{align}
& y_{it} = \alpha _{it} + x'_{it} \beta_{it}+u_{it},
& i= 1, \dots, N, &
& t=1, \dots, T,
\nonumber
\end{align}
где $y_{it}$ --- это скалярная зависимая переменная, $x_{it}$ --- вектор независимых переменных размерности $K \times 1$, $u_{it}$ --- ошибки модели, $i$ --- индивидуальный индекс (индивида, фирмы или страны), $t$ --- индекс временного периода.

Эта модель слишком общая, и оценить ее не представляется возможным, так как количество параметров для оценки больше, чем количество наблюдений. Необходимо наложить ограничения на степень, в которой $\alpha$ и $\beta$ изменяются в зависимости от $i$ и $t$, а также на поведение ошибок.

\vspace{0.8cm}

{\centering
Объединенная регрессия\\}


Модель, накладывающая наибольшее ограничение, --- это \textbf{модель сквозной регрессии}, которая базируется на предположении о \textbf{постоянных коэффициентах}, как это обычно предполагается при анализе пространственных данных:
\begin{align}
y_{it}= \alpha + \x'_{it} \bm\beta + u_{it}.
\label{Eq:21.1}
\end{align}
Если эта модель правильно специфицирована, и регрессоры некоррелированы с ошибками, тогда оценивание с помощью МНК сквозной регрессии дает состоятельные оценки. Ошибки могут быть коррелированы во времени для данного индивидуального наблюдения; в этом случае не следует пользоваться стандартными ошибками, так как они могут быть смещены вниз. Более того, МНК оценка сквозной регрессии будет несостоятельна, если моделью, описывающей данные, является модель с фиксированными эффектами, описанная ниже.

{\centering
Индивидуальные и временные эффекты\\}

Простой вариант модели \ref{Eq:21.1} предполагает, что свободный член может изменяться по индивидуальным наблюдениям и во времени, в то время как коэффициенты наклона остаются неизменными. Тогда $y_{it}= \alpha_{i} + \gamma_t + \mathbf x'_{it} \bm\beta+ u_{it}$, или
\begin{align}
y_{it}=\sum \limits_{j=1}^{N} \alpha_j d_{j,it} + \sum \limits_{s=2}^{T} \gamma_s d_{s,it} +  \mathbf x'_{it} \bm\beta,
\label{Eq:21.2}
\end{align}
где N индивидуальных фиктивных переменных $d_{j,it}$, равные единице, если $i=j$ и равны 0 в противном случае, $(T-1)$ временных фиктивных переменных $d_{s,it}$, равные единице, если $t=s$ и нулю в противном случае; предполагается, что $\mathbf x'_{it}$ не включает свободный член (Если свободный член включается, тогда необходимо исключить одну из индивидуальных фиктивных переменных).

Эта модель включает $N+(T-1)+\dim[\x]$ параметров, для которых могут быть получены состоятельные оценки, если $N\rightarrow \infty$ и $T \rightarrow \infty$. Мы главным образом рассматриваем \textbf{короткие панели}, где $N\rightarrow \infty$, но не $T$. Тогда для $\gamma_s$ может быть получена состоятельная оценка, так что $(T-1)$ временных фиктивных переменных просто включены в регрессоры $\mathbf x'_{it}$. В таком случае трудность состоит в оценивании параметров $\bm\beta$ при учете $N$ свободных членов для каждого индивидуума. Одно из решений состоит в том, чтобы вместо дамми для каждого индивидуального наблюдения, использовать дамми для групп, для чего можно использовать кластерный анализ, описанный в главе 24. В текущей главе мы используем спецификацию с полным набором $N$ свободных членов, что приводит к проблеме увеличения параметров $N \rightarrow \infty$.


{\centering
Модели с фиксированными и случайными эффектами\\}
Модели с \textbf{индивидуальными эффектами} предполагают, что каждая пространственная единица имеет свой свободный член, в то время как коэффициенты наклона одинаковы, так что 
\begin{align}
y_{it}=\alpha_i + \mathbf x'_{it} \bm \beta + \e_{it},
\label{Eq:21.3}
\end{align}
где $\e_{it}$ независимы и одинаково распределены по $i$ и $t$.  Это более компактный вид модели \ref{Eq:21.2} с фиктивными переменными, включенными в регрессоры $\mathbf x'_{it}$. $\alpha_i$ --- случайные переменные, которые охватывают \textbf{ненаблюдаемую гетерогенность}, которая обсуждалась в разделах 18.2-18.5 и 20.4.

В данной главе мы делаем предположение о \textbf{сильной экзогенности} (или \textbf{строгой экзогенности}) :
\begin{align}
E[\e_{it} | \alpha_i, \mathbf x'_{i1}, \dots, \mathbf x'_{iT}]=0, &
& t=1, \dots, T,
\label{Eq:21.4}
\end{align}
так что предполагается, что ошибки имеют нулевое условное математическое ожидание при прошлых, настоящих и будущих значениях регрессоров. Чемберлин (1980) детально обсуждает предположения об экзогенности и тесты для проверки экзогенности для панельных данных. Строгая экзогенность исключает использование моделей с лаговыми переменными или с эндогенными переменными в регрессорах, такие модели мы отложим до главы 22.

Один вариант модели \ref{Eq:21.3} предполагает, что $\alpha_i$ --- это ненаблюдаемые случайные величины, которые могут быть коррелированы с наблюдаемыми регрессорами. Такая модель называется \textbf{моделью с фиксированными эффектами}, так как в первую очередь эффекты моделировались как параметры $\alpha_i \dots \alpha_N$ для оценки. Если фиксированные эффекты присутствуют и коррелированы с $\mathbf x'_{it}$, то многие оценки, такие как МНК оценки модели сквозной регрессии будут несостоятельны. Вместо этого, для получения состоятельных оценок $\beta$ в коротких панелях должны быть использованы альтернативные методы оценки, которые элиминируют $\alpha_i$.

Другой вариант модели \ref{Eq:21.3} предполагает, что ненаблюдаемые индивидуальные эффекты $\alpha_i$ --- это случайные переменные, которые распределены независимо от регрессоров. Такая модель называется \textbf{моделью со случайными эффектами}, которая обычно накладывает дополнительные предположения:
\begin{align} \label{Eq:21.5}
 & \alpha_i  \thicksim [\alpha, \sigma^2_{\alpha}],  \\ 
& \epsilon_{it}  \thicksim [0, \sigma^2_{\epsilon}], %\nonumber
\end{align}
т.е. и случайные эффекты, и ошибки в \ref{Eq:21.3} независимы и одинаково распределены по предположению. Заметим, что в  \ref{Eq:21.5} не специфицировано никакое конкретное распределение. Для того, чтобы отличать эту модель от более общих моделей со случайными эффектами, таких как \textbf{смешанных линейных моделей} (mixed linear models), представленных в разделе 22.8, такой модели дано более точное название  --- \textbf{модель с индивидуальными случайными эффектами} (one-way individual-specific random effects model), или, что проще, \textbf{модель со случайными свободными членами} (random intercept model). Кроме того, другое название модели --- \textbf{модель со случайными компонентами} (random components model). 

Термин фиксированные эффекты может вводить в заблуждение, а вместо термина случайные эффекты более корректно было бы употреблять термин чисто случайные эффекты. Чтобы избежать путаницы, М.-Дж. Ли (2002)  называет фиксированные эффекты <<связанными эффектами>>, а случайные эффекты --- <<несвязанными эффектами>>. Мы используем традиционные названия и терминологию, но необходимо помнить, что $\alpha_i$  --- это случайная переменная как в моделях с фиксированными, так и со случайными эффектами.


{\centering
Равнокоррелированная модель\\}
Модель со случайными эффектами может рассматриваться как частный случай модели сквозной регрессии, так как $\alpha_i$ можно отнести к ошибке. Тогда \ref{Eq:21.3} может рассматриваться как регрессия $y_{it}$ на $\mathbf x'_{it}$ с составной ошибкой $u_{it}=\alpha_i+\e_{it}$. В \ref{Eq:21.5} указано предположение, что 
\begin{align}
Cov[(\alpha_i+\e_{it}), (\alpha_i+\e_{is})]= 
	\begin{cases}
		\sigma^2_{\alpha}, & t\neq s, \\
		\sigma^2_{\alpha} + \sigma^2_{\e}, & t=s.
	\end{cases}
\label{Eq:21.6}
\end{align}

Модель со случайными эффектами накладывает ограничение, что составная ошибка $u_{it}$ \textbf{равнокоррелирована} (equicorrelated), так как $Cor[u_{it},u_{is}]=\sigma^2_{\alpha}/[\sigma^2_{\alpha} + \sigma^2_{\epsilon}]$ для $t\neq s$ не меняется в зависимости от $t-s$. Очевидно, что МНК оценка модели сквозной регрессии будет состоятельна, но неэффективна в модели со случайными эффектами. Модель со случайными эффектами также называется \textbf{равнокоррелированной моделью} (equicorrelated model) или \textbf{моделью взаимозаменяемых ошибок} (exchangeable errors model).

{\centering
Модель с фиксированными эффектами против модели со случайными эффектами\\}

Модели с и без фиксированных эффектов фундаментально различаются между собой. Современная эконометрическая литература фокусируется на фиксированных эффектах, но мы также остановимся и на модели со случайными эффектами.

Некоторые авторы, включая Чемберлина (1980, 1984) и  Вулдриджа (2002), используют запись
\begin{align}
y_{it} = c_i + \mathbf x'_{it} \bm{\beta} + \e_{it}
\nonumber
\end{align}
в \ref{Eq:21.3}, чтобы было ясно, что индивидуальный эффект --- это случайная величина как в модели с фиксированными, так и со случайными эффектами. Обе модели предполагают, что
\begin{align}
E[y_{it} | c_i,  \mathbf x'_{it}] = c_i +  \mathbf x'_{it} \bm{\beta}.
\nonumber
\end{align}

Индивидуальный эффект $c_i$ неизвестен, и  в коротких панелях для него не может быть получена состоятельная оценка, поэтому мы не можем оценить $E[y_{it} | \mathbf x'_{it}]$. Вместо этого мы можем элиминировать $c_i$ взятием математического ожидания по $c_i$, 
\begin{align}
E[y_{it} | \mathbf x'_{it}] = E [c_i | \mathbf x'_{it}] + \mathbf x'_{it} \bm{\beta}\nonumber.
\end{align}

В модели со случайными эффектами предполагается, что $E [c_i | \mathbf x'_{it}] = \alpha$, и как следствие $E[y_{it} | \mathbf x'_{it}] = \alpha + \mathbf x'_{it} \bm{\beta}$ и поэтому    
$E[y_{it} | \mathbf x'_{it}]$ возможно идентифицировать. В модели с фиксированными эффектами однако $E [c_i | \mathbf x'_{it}]$ изменяется с $\mathbf x'_{it}$. Каким образом изменяется $\mathbf x'_{it}$, неизвестно. Поэтому идентифицировать $E[y_{it} | \mathbf x'_{it}]$ мы не можем. И тем не менее, возможность получить состоятельную оценку $\bm{\beta}$ в модели с фиксированным эффектом в короткой панели сохраняется (это будет обсуждаться далее). Таким образом, в модели с фиксированными эффектами можно идентифицировать предельный эффект
\begin{align}
\bm{\beta} = \partial E[y_{it} | c_i, \mathbf x'_{it}]/ \partial \mathbf x'_{it},
\nonumber
\end{align}
но даже в этом случае условное математическое ожидание не может быть идентифицировано. Можно, например, идентифицировать эффект на отдачу от дополнительного года образования, учитывая индивидуальный эффект, но даже в таком случае индивидуальные эффекты и условное математическое ожидание не идентифицируемы.

В коротких панелях модели с фиксированными эффектами позволяют только идентифицировать  предельные эффекты $\partial E[y_{it} | c_i, \mathbf x'_{it}]/ \partial \mathbf x'_{it}$, но даже в этом случае только для регрессоров, меняющихся во времени,. Поэтому предельные эффекты расы или пола, к примеру, не идентифицируемы. Модель со случайными эффектами позволяет идентифицировать все компоненты $\bm{\beta}$ и $E[y_{it}| \mathbf x'_{it}]$, но ключевое предположение модели со случайными эффектами заключается в том, что $E[c_{it}| \mathbf x'_{it}]$ является константой, что не выполняется во многих микроэконометрических приложениях.

\subsection{Оценки параметров в моделях панельных данных}

Мы сейчас представим несколько часто используемых оценок параметра $\bm{\beta}$ в моделях панельных данных, детальное описание которых представлено в разделах 21.5 --- 21.7. Оценки различаются в зависимости от степени использования изменчивости пространственных и временных данных, а их свойства меняются в соответствии с тем, является ли модель с фиксированными эффектами подходящей.

Регрессор $x_{it}$ может быть либо \textbf{неизменным}, с $x_{it}=x_i$ при $t=1, \dots , T$, или \textbf{изменяющимся во времени}. Для некоторых оценок, в особенности для межгрупповой оценки и оценки в первых разностях, описанных далее, идентифицированы только коэффициенты перед регрессорами, меняющимися во времени.

{\centering
 Модель сквозной регрессии \\}

Оценка \textbf{модели сквозной регрессии} может быть получена с помощью оценивания регрессии МНК, полученной с помощью составления всех имеющихся данных в одну группу длинного формата с $NT$ наблюдениями
\begin{align}
& y_{it} = \alpha +  \mathbf x'_{it} \bm{\beta} + u_it, 
& i= 1, \dots, N, &
& t=1, \dots, T.
\nonumber
\end{align}
Если $Cov[u_{it}, \mathbf x_{it} ] = \mathbf 0$, тогда для состоятельности оценок достаточно либо $ N \rightarrow  \infty$ , либо $T\rightarrow \infty$

Очевидно, что оценка модели сквозной регрессии состоятельна, если модель сквозной регрессии \ref{Eq:21.1} наилучшим образом описывает данные и регрессоры некоррелированы с ошибкой. Обычная ковариационая матрица МНК основывается на независимых и одинаково распределенных ошибках, однако не является подходящей, так как ошибки для данного $i$ почти наверняка коррелируют  во времени. $NT$ коррелированных наблюдений вмешают в себя меньше информации, чем $NT$ независимых наблюдений.

Для понимания этой корреляции заметим, что для данного индивидуального наблюдения мы ожидаем существенную корреляцию $y$  во времени, так что $Cor[y_{it}, y_{is}]$ велика. Даже после включения регрессоров $Cor[u_{it}, u_{is}]$ может по прежнему не равняться нулю, и быть все еще достаточно высокой. Например, если модель завышает оценку будущих индивидуальных доходов в одном году, она может также завысить оценку будущих доходов того же индивидуума в другой год. Модель со случайными эффектами учитывает такую корреляцию $Cor[u_{it}, u_{is}] = \sigma^2_\alpha / [ \sigma^2_\alpha + \sigma^2_\epsilon]$ при $t \neq s$ (см. \ref{Eq:21.6}).

Обычный МНК использует $T$ лет как независимые частицы информации, однако при наличии положительной корреляции ошибок информации в этих частицах содержится меньше, чем предполагается МНК. Это приводит к переоцениванию точности оценок, которое может быть довольно большим, как показано в разделе 21.3.2 и формально продемонстрировано в разделе 21.5.4. Поэтому необходимо использовать скорректированные (panel-corrected) стандартные ошибки (см. раздел 21.2.3)  всегда, когда для оценивания панельных данных применяется МНК. Можно применить различные корректировки для стандартных ошибок в зависимости от предполагаемой структуры корреляции и гетероскедастичности. Корректировки могут различаться в зависимости от того, какая панель, короткая или длинная, используется (см. раздел 21.5).

Оценка МНК модели сквозной регрессии несостоятельна, если истинная модель --- это модель с фиксированными эффектами. Чтобы это обнаружить, необходимо переписать модель (\ref{Eq:21.3}) как 
\begin{align}
y_{it} = \alpha +  \mathbf x'_{it} \bm{\beta} + (\alpha_i --- \alpha + \e_{it}).
\nonumber
\end{align}
Сквозная регрессия  $y_{it}$ на $\mathbf x_{it}$ с включением константы, оцениваемая с помощью МНК, дает несостоятельные оценки $\bm{\beta}$, если индивидуальный эффект $\alpha_i$ коррелирован с регрессорами $\mathbf x'_{it}$, вследствие того, что такая корреляция подразумевает, что составная ошибка $(\alpha_i --- \alpha + \epsilon_{it})$ коррелирована с регрессорами.

Таким образом, МНК оценка модели сквозной регрессии  
является подходящей, если моделями, хорошо описывающими данные, являются модели с постоянными коэффициентами или модели со случайными эффектами. Однако, для статистических выводов необходимо использовать скорректированные стандартные ошибки и t-статистики. Модель объединенной регрессии МНК не будет состоятельна, если наилучшим образом описывает данные модель с фиксированными эффектами.

{\centering
Оценка between \\}
При оценке коэффициента $\be$ с помощью объединенной модели регрессии используется изменчивость и во времени, и в пространстве.

Оценка between использует только пространственную изменчивость в коротких панелях. Начнем с модели с фиксированными индивидуальными эффектами \ref{Eq:21.7}. Усредняя по годам, получаем $\bar{y}_{it} = \alpha_i +  \bar{\mathbf x}'_{it} \bm{\beta} +\bar{\varepsilon}_{it}$ и переписываем в виде  \textbf{модели between}
\begin{align}
& \bar{y}_{it} = \alpha +  \bar{\mathbf x}'_{it} \bm{\beta} + (\alpha_i --- \alpha + \bar{\varepsilon}_{it}),
& i=1, \dots, N,
\label{Eq:21.7}
\end{align}
где $\bar{y}_{it} =T^{-1}\sum_t {y}_{it}$, $\bar{\varepsilon}_{it}=\sum_t {\varepsilon}_{it}$ и $\bar{\mathbf x}'_{i}=T^{-1}\sum_t {\mathbf x}_{it}$.

\textbf{Оценка between} представляет собой МНК оценку регрессии $\bar{y}_{it}$ на константу и $\bar{\mathbf x}'_{i}$. Эта оценка использует изменчивость между индивидуальными наблюдениями и является аналогом обычной регрессионной модели пространственных данных. Регрессионная модель пространственных данных является частным случаем модели between, когда $T=1$.

Оценка between состоятельна, если регрессоры $\bar\x$ не зависят от составной ошибки  $(\alpha_i --- \alpha + \bar{\varepsilon}_{it})$ в \ref{Eq:21.7}. Это выполняется для модели с постоянными коэффициентами и модели со случайными эффектами. Для модели с фиксированными эффектами оценка between несостоятельна, так как предполагается, что $\alpha_i$ коррелирует с $\x_{it}$, а следовательно коррелирует и с $\bar\x_{it}$.

{\centering 
Оценка within или оценка с фиксированным эффектом\\}

Оценка within в отличие от оценки модели сквозной регрессии или оценки between использует особые качества панельных данных. В короткой панели она измеряет связь между индивидуальными отклонениями регрессоров от своих средних по времени значений и индивидуальными отклонениями зависимой переменной от своего среднего по времени значения. Это возможно благодаря использованию вариации данных во времени.

Начнем с модели с индивидуальными эффектами \ref{Eq:21.3}, которая является частными случаем модели \ref{Eq:21.1} при $\alpha_i=\alpha$. Усредняя по времени, получаем $\bar{y}_i=\alpha_i+\bar{\x}'_{i} \bm{\beta}+\bar{\e}_i$. Вычитаем полученное из $y_{it}$ в  \ref{Eq:21.3} и получаем \textbf{модель within}
\begin{align}
& y_{it} --- \bar{y}_{it} = (\x_{it} --- \bar{\mathbf x}'_{it}) \be + (\e_{it}-\bar\e_i),
& i=1, \dots, N, &
& t=1, \dots, T,
\label{Eq:21.8}
\end{align}
так как $\alpha_i$ взаимно уничтожаются.

\textbf{Оценка within} представляет собой МНК оценку модели \ref{Eq:21.8}. Особенностью этой модели явялется то, что она дает состоятельные оценки $\be$ в модели с фиксированными эффектами в отличие от МНК оценки в модели сквозной регрессии и оценки between.

Оценка within имеет несколько интерпретаций (см. раздел 21.6). Она называется \textbf{оценкой с фиксированным эффектом}, так как это эффективная оценка $\be$ в модели \ref{Eq:21.3}, где $\alpha_i$ --- это фиксированные эффекты, а ошибки $\e_{it}$ независимы и одинаково распределены. В этой главе внимание фокусируется на фиксированных эффектах как \textbf{вспомогательных параметрах (nuisance parameters)}, которые могут быть проигнорированы, ведь интерес заключается только в оценке параметра $\be$. В случае необходимости фиксированные эффекты также могут быть оценены. В коротких панелях  оценки индивидуальных эффектов $\alpha$ несостоятельны, хотя их распределение или их ковариация с зависимой переменной могут быть информативными. Если $N$ не слишком велико, то альтернативным и более простым путем получить оценку within будет \textbf{МНК оценка с фиктивными переменными} (Least Squares Dummy Variable estimator, LSDV). Она получается путем оценивания \ref{Eq:21.2} МНК регрессии $y_{it}$ на $x_{it}$ и  N индивидуальных фиктивных переменных. В результате получается  оценка within коэффициента $\be$ наряду с оценками N фиксированных эффектов (см. раздел 21.6.4). Другая интерпретация оценки within --- оценка ковариации. Наконец, взятие отклонений средних индивидуальных значений эквивалентно взятию остатков из вспомогательной регрессии $y_{it}$ и $\x_{it}$ на индивидуальные фиктивные переменные и последующему использованию этих остатков.

Основное ограничение оценки within состоит в том, что коэффициенты регрессоров, которые не меняются во времени, не идентифицируемы в данной модели, так как если $x_{it}=x_i$, то $\bar{x}_i=x_i$ и $(x_{it}-\bar{x}_i)=0$. Было осуществлено множество попыток оценить влияние регрессоров, не меняющихся во времени. Например, в модели заработной платы мы можем быть заинтересованы в эффекте пола или расы. По этой причине многие не используют оценку within. МНК оценка модели сквозной регрессии или оценка со случайным эффектом позволяют оценивать коэффициенты перед переменными, которые не меняются во времени, но эти оценки несостоятельны в случае, если модель с фиксированным эффектом оказывается истинной моделью.


{\centering
Оценка модели в первых разностях\\}

Оценка модели в первых разностях также использует особенности панельных данных. Она измеряет связь между индивидуальными однопериодными изменениями регрессоров и индивидуальными однопериодными изменениями зависимой переменной в короткой панели.

Начнем с модели с индивидуальными эффектами \ref{Eq:21.3}. Вычитая однопериодное запаздывание $y_{i,t-1}=\alpha_i+\x'_{i,t-1} \be + \e_{i,t-1}$ из $y_{it}$ в  \ref{Eq:21.3}, получаем \textbf{модель в первых разностях}
\begin{align}
& y_{it} --- {y}_{i,t-1} = (\x_{it} --- \mathbf x'_{i,t-1})' \be + (\e_{it}-\bar\e_{i,t-1}),
& i=1, \dots, N, &
& t=2, \dots, T,
\label{Eq:21.9}
\end{align}
так как $\alpha_i$ взаимно уничтожаются.

Оценка модели в первых разностях --- это МНК оценка в \ref{Eq:21.9}. Как и в случае оценки within, это оценивание приводит к состоятельным оценкам коэффициентов $\be$  в модели с фиксированным эффектом, хотя коэффициенты не меняющихся во времени регрессоров не идентифицированы. Оценка модели в первых разностях менее эффективна, чем внутригрупповая оценка для $T>2$, если $\e_{it}$ независимы и одинаково распределены.

{\centering
Оценка со случайным эффектом\\}
Оценка со случайным эффектом также использует особенности панельных данных.

Начнем с модели с индивидуальными эффектами \ref{Eq:21.3}, но будем предполагать модель со случайным эффектом, в которой $\alpha_{i}$ и $\e_{it}$ независимы и одинаково распределены, как в \ref{Eq:21.5}. МНК оценка модели сквозной регрессии состоятельна, но ОМНК оценка будет более эффективна. \textbf{Доступная ОМНК-оценка} (см. раздел 4.5.1) модели со случайным эффектом, называемая \textbf{оценкой со случайным эффектом}, может быть посчитана посредством МНК оценивания преобразованной модели 
\begin{align}
y_{it} --- \hat{\lambda}\bar{y}_{i} = (1-\hat{\lambda})\mu+(\x_{it} --- \hat{\lambda}\bar{\x}_{i})' \be +v_{it},
\label{Eq:21.10}
\end{align}
где $v_{it}=(1-\hat{\lambda})\alpha_{i}+(\e_{it}-\hat{\lambda}{\bar{\e}}_{i})$ асимптотически независимы и одинаково распределены, и $\hat{\lambda}$ состоятельна для
\begin{align}
\lambda=1-\frac{\sigma_{\e}}
				{\sqrt{\sigma^2_{\e}+T\sigma^2_\alpha}}.
\label{Eq:21.11}
\end{align}
Вывод \ref{Eq:21.10} и способы оценки $\sigma^2_{\alpha}$  и $\sigma^2_{\e}$, а следовательно и $\lambda$ представлены в разделе 21.7. Заметим, что $\hat{\lambda}=0$ соответствует МНК оценке в модели сквозной регрессии, $\hat{\lambda}=1$ соответствует оценке within, и $\hat{\lambda} \rightarrow 1$ при $T \rightarrow \infty$. Это двухшаговая оценка $\be$.

Оценка со случайным эффектом является эффективной, если модель со случайными эффектами является истинной, хотя выигрыш в эффективности в сравнении с МНК не должен быть большим. Однако, если истинная модель --- это модель с фиксированными эффектами, то оценка несостоятельна.

\subsection{Статистические выводы робастные для панельных данных}
Различные модели панельных данных включают ошибки, обозначаемые $u_{it}$, $\e_{it}$ и $\alpha$. Во многих микроэконометрических приложениях стоит использовать предположение о независимости по $i$. Однако, ошибки могут быть (1) \textbf{коррелированы} (например, во времени для данного $i$) и/или (2) \textbf{гетероскедастичны}. Для получения надежных статистических выводов необходимо учитывать оба этих фактора.

Состоятельная оценка при наличии гетероскедастичности в форме Уайта, рассмотренная в разделе 4.4.5, может применяться и к коротким панелям, так как для i-го наблюдения ковариационная матрица ошибок имеет конечную размерность $T\times T$ при $N \rightarrow \infty$. Таким образом, робастные к гетероскедастичности стандартные ошибки могут быть получены без предположения о специальной функциональной форме для внутригрупповой корреляции ошибок или гетероскедастичности. Более эффективные оценки полученные с помощью ОММ будут рассматриваться позже в разделе 22.2.3. 

Важно заметить, что часто команды для работы с панельными данными во многих статистических пакетах по умолчанию считают стандартные ошибки в предположении, что ошибки модели независимы и одинаково распределены, что приводит к ошибочным выводам. В частности, для модели сквозной регрессии $y_{it}$ на $\x_{it}$ без какого-либо учета индивидуальных эффектов очень вероятно, что $Cov[u_{it}, u_{is}] > 0$ для $t \neq s$. Игнорирование этой корреляции может привести к сильно завышенным стандартным ошибкам и переоцененным $t$-статистикам, что продемонстрировано на примере в разделе 21.3 и алгебраически показано в разделе 21.5.4. Как только в модель включаются фиксированные или индивидуальные случайные эффекты, корреляция в ошибках значительно снижается, но все же не может быть устранена полностью.
К тому же, вероятно, необходимо учитывать и возможную гетероскедастичность как это обычно делается для пространственных данных.
 
{\centering
Робастные для панельных данных стандартные ошибки в сэндвич форме\\}

Оценки применимые к панельным данным, описанные в разделе 21.2.2 могут быть получены с помощью МНК оценивания $\bm\theta$ в сквозной регрессионной модели
\begin{align}
\tilde{y}_{it}=\tilde{\mathbf w}'_{it} \bm\theta+\tilde{u}_{it},
\label{Eq:21.12}
\end{align}
где разные оценки панельных данных соответствуют разным преобразованиям $\tilde{y}_{it}$,
 $\tilde{w}'_{it}$ и $\tilde{u}_{it}$ для $y_{it}$, $\tilde{w}'_{it}=[1\; \x'_{it}]$, и $u_{it}$. Дело в том, что $\tilde{y}_{it}$ --- известная функция только для $y_{i1}, \dots, y_{iT}$, что аналогично для  $\tilde{w}'_{it}$ и $\tilde{u}_{it}$.

В самом простом случае модели сквозной регрессии нет необходимости в преобразовании и $\bm\theta=[\alpha \; \be']'$. Для оценки within $\tilde{y}_{it}=y_{it}-\bar{y}_{it}$, $\tilde{\mathbf w}'_{it}=(\x_{it}-\bar\x_i)$, где присутствуют только регрессоры, меняющиеся во времени, и $\bm\theta$ равно коэффициентам этих регрессоров. Для оценивания уравнения в первых разностях $\tilde{y}_{it}=y_{it}-{y}_{i,t-1}$, $\tilde{\mathbf w}'_{it}=(\x_{it}-\x_{i,t-1})$ и снова идентифицированы коэффициенты только регрессоров, меняющихся во времени. В случае со случайными эффектами $\tilde{y}_{it}=y_{it}-\hat{\lambda}\bar{y}_{it}$, $\tilde{\mathbf w}'_{it}=(\mathbf w_{it}-\hat{\lambda} \bar{\mathbf w}_i)$ и $\bm\theta=[\bm\alpha \; \be']'$. Такие преобразования могут стать причиной корреляции даже если исходные ошибки не были коррелированы.

Для записи удобно соединять наблюдения по временным периодам для данного индивида:
\begin{align}
\widetilde{\mathbf y}_{i}=\widetilde{\mathbf W}'_{i} \bm\theta+\tilde{\mathbf u}_{i},
\nonumber
\end{align}
где $\tilde{\mathbf{y}}_{i}$ --- это вектор размерности $T \times 1$, как в предыдущих примерах (кроме модели в первых разностях, где размерность $(T-1) \times 1$), и $\tilde{\mathbf{W}}_i$  --- это матрица размерности $T \times q$ (или $(T-1) \times q$ --- для модели в первых разностях). Если далее соединить наблюдения по индивидуумам, то запись модели будет выглядеть следующим образом:
 \begin{align}
\tilde{\mathbf y}=\widetilde{\mathbf W} \bm\theta+\tilde{\mathbf u}.
\nonumber
\end{align}
Поэтому МНК оценку можно записать тремя разными способами:
 \begin{align}
\hat{\theta}_{OLS}
&=[\tilde{\mathbf W}'\tilde{\mathbf W}]^{-1}\tilde{\mathbf W}'\tilde{\mathbf y} \nonumber \\
& = \left[\sum^{N}_{i=1} \tilde{\mathbf W}_i'\tilde{\mathbf W}_i\right]^{-1}\sum_i \tilde{\mathbf W}_i'\tilde{\mathbf y}_i \nonumber \\
& =\left[\sum^{N}_{i=1} \sum^{T}_{t=1}\tilde{\mathbf w}_{it}'\tilde{\mathbf w}_{it}\right]^{-1}\sum_{i=1}^{N} \sum_{t=1}^{T} \tilde{\mathbf w}_{it}'\tilde{\mathbf y}_{it}
\nonumber
\end{align}
где в случае модели в первых разностях пределы суммирования по времени для третьего выражения меняются на от $t=2$ до $T$. В зависимости от контекста используется наиболее удобная форма записи.

В целях проверки состоятельности, заметим, что если модель правильно специфицирована, то с помощью стандартных алгебраических преобразований получаем $\hat{\bm \theta}_{OLS}=\bm\theta+[\tilde{\mathbf W}'\tilde{\mathbf W}]^{-1}\tilde{\mathbf W}'\tilde{\mathbf u}$ или 
 \begin{align}
\hat{\theta}_{OLS}
& = \bm\theta + \left[\sum^{N}_{i=1} \tilde{\mathbf W}_i'\tilde{\mathbf W}_i\right]^{-1}\sum_i^N \tilde{\mathbf W}_i'\tilde{\mathbf u}_i.
\nonumber
\end{align}
При условии независимости индивидуальных наблюдений необходимое условие состоятельности --- $E[\tilde{\mathbf W}_i'\tilde{\mathbf u}_i]=0$, что влечет за собой более строгое предположение $E[u_{it}|\mathbf w_{it}]=0$. Существенным условием также является условие строгой экзогенности, данное в \ref{Eq:21.4}. Оценка, использующая более слабые предположения, чем строгая экзогенность, описана в главе 22. Она позволит, к примеру, включить лаговые переменные.

Асимптотическая дисперсия $\hat{\theta}_{OLS}$ будет равна
 \begin{align}
V[\hat{\theta}_{OLS}]= \left[\sum^{N}_{i=1} \tilde{\mathbf W}_i'\tilde{\mathbf W}_i\right]^{-1} \sum^{N}_{i=1} \tilde{\mathbf W}_i' \mathrm E[\tilde{\mathbf u}_i \tilde{\mathbf u}'_i|\tilde{\mathbf W}_i]\tilde{\mathbf W_i} \left[\sum^{N}_{i=1} \tilde{\mathbf W}_i'\tilde{\mathbf W}_i\right]^{-1},
\nonumber
\end{align}
при условии независимости ошибок по индивидуальным наблюдениям. Получение состоятельной оценки $V[\hat\theta_{OLS}]$ в данном случае аналогично проблеме получения состоятельных оценок $V[\hat{\theta}_{OLS}]$ робастных к гетероскедастичности в случае пространственных данных. Единственное усложнение --- это вектор $\mathbf u_i$ вместо скалярной $u_i$, что не представляет особой проблемы в случае короткой панели, так как в таком случае размерность $\mathbf u_i$ конечна.

Таким образом, мы получаем \textbf{робастную для панельных данных оценку} асимптотической ковариационной матрицы МНК оценки в  модели сквозной регрессии, которая учитывает и корреляцию, и гетероскедастичность:
 \begin{align}
V[\hat{\theta}_{OLS}]= \left[\sum^{N}_{i=1} \tilde{\mathbf W}_i'\tilde{\mathbf W}_i\right]^{-1} \sum^{N}_{i=1} \tilde{\mathbf W}_i' \hat{\mathbf u}_i \hat{\mathbf u}'_i\tilde{\mathbf W}_i]\tilde{\mathbf W_i} \left[\sum^{N}_{i=1} \tilde{\mathbf W}_i'\tilde{\mathbf W}_i\right]^{-1},
\label{Eq:21.13}
\end{align}
где $\hat{\mathbf u}_i=\hat{\tilde{\mathbf u}}_i=\tilde{\mathbf y}_i-\tilde{\mathbf W}_i \tilde{\theta}$. Оценка \ref{Eq:21.13} предполагает независимость индивидуальных наблюдений и $N \rightarrow \infty$ в случае короткой панели, в противном случае $V[u_it]$  и  $Cov[u_it, u_is]$ могут меняться в зависимости от $i, t$ и $s$. По-другому выражение \ref{Eq:21.13} можно записать следующим образом:
\begin{align}
V[\hat{\theta}_{OLS}]=\left[\sum^{N}_{i=1} \sum^{T}_{t=1} \tilde{\mathbf w}_{it}\tilde{\mathbf w}'_{it}\right]^{-1} \sum_{i=1}^{N} \sum_{t=1}^{T} \sum_{s=1}^T \tilde{\mathbf w}_{it}\tilde{\mathbf w}'_{is} \hat{u}_{it} \hat{ u}_{is} \left[\sum^{N}_{i=1} \sum^{T}_{t=1}\tilde{\mathbf w}_{it}\tilde{\mathbf w}'_{it}\right]^{-1}, 
\nonumber
\end{align}
где $\hat{u}_{it}=\hat{\tilde{y}}_{it}-\tilde{\mathbf w}_{it} \hat{\bm\theta}$. Эта оценка была предложена Ареллано (1987) для оценки с фиксированным эффектом.

Робастные стандартные ошибки, основанные на \ref{Eq:21.13} могут быть вычислены с помощью стандартной команды для вычисления стандартных ошибок в МНК-регрессии, если в команде есть опция вычисления \textbf{кластерно-робастных} стандартных ошибок (см. раздел 24.5.2). Так как кластеризация здесь проводится на уровне индивидуумов, индивидулаьный индекс $i$ используется как \textbf{кластерная переменная}. Этот метод был использован для получения робастных стандартных ошибок, представленных в таблице 24.1. %\ref{Tab:24.1}.

Термин <<робастные>> стандартные ошибки может ввести читателя в заблуждение. Распространенная ошибка в оценивании модели сквозной регрессии --- использование обычных робастных стандартных ошибок, используемых при оценивании регрессии МНК \ref{Eq:21.12}  (см. раздел 4.4.5). Однако, обычные робастные стандартные ошибки решают только проблемы гетероскедастиности, а на практике при работе с панельными данными намного более важно устранить проблему корреляции ошибок  индивидуальных наблюдений. Другая распространенная ошибка, хоть и не влекущая за собой серьезных последствий, --- использование кластерных робастных стандартных ошибок, которые предполагают гомоскедастичность, т.е. что $\mathrm E[\mathbf u_i \mathbf u'_i]$ постоянно для каждого $i$.


{\centering
Робастные для панельных данных стандартные ошибки, полученные методом бутстрэп \\}

Альтернативный способ получения робастных для панельных данных стандартных ошибок --- \textbf{бутстрэп метод}. Ключевым предположением метода является независимость индивидуальных наблюдений, так что ресэмплинг проводится \textbf{с заменой по $i$}, и используются все наблюдаемые периоды для данного индивидуального наблюдения.
Таким образом, мы получаем $B$ псевдо-выборок для данных $\{(\mathbf y_i, \mathbf X_i), i = 1, \dots, N\}$, и для каждой \textbf{псевдо-выборки} оценивается МНК регрессия $\tilde{y}_{it}$ на $\tilde{\mathbf w}_{it}$. В результате оценивания получается $B$ оценок $\hat{\bm \theta}_b, b=1, \dots, B$.

\textbf{Оценка ковариационной матрицы для панельных данных, полученная методом бутстрэп:}
\begin{align}
\hat{\mathrm V}_{\mathrm {Boot}}[\hat{\bm\theta}]=\frac{1}{B-1} \sum_{b=1}^B \left(\hat{\bm\theta}_b --- \bar{\hat{\bm\theta}} \right) \left( \hat{\bm\theta}_b --- \bar{\hat{\bm\theta}} \right)',
\label{Eq:21.14}
\end{align}
где $\bar{\bm{\hat{\theta}}}=B^{-1} \sum\nolimits_b \bm{\hat{\theta}}$. Бутстрэпирование не дает \textbf{асимптотических улучшений} (см. гл. 11.2.2). При условии независимости индивидуальных наблюдений оценка состоятельна при $N \rightarrow \infty$. Оценка \ref{Eq:21.14} асимптотически эквивалетна оценке \ref{Eq:21.3}, как в случае с данными пространственного типа бутстрэп пары асимптотически эквивалентны состоятельной в случае гетероскедастичности оценке Уайта. Этот бутстрэп не дает асимптотического улучшения, хотя оно возможно (см. раздел 11.6.2).

Такой бутстрэп метод  можно применить к любой оценке, которая основывается на независимости индивидуальных наблюдений и на том, что $N \rightarrow \infty$, в том числе и к доступной ОМНК-оценке модели сквозной регрессии для коротких панелей (см. раздел 21.5.2). Идея заключается в том, что нужно проводить ресэмплинг только по индивидуальным наблюдениям, а не по $i$ и $t$.

 
{\centering
Обсуждение\\}

Невозможно преувеличить важность корректирования стандартных ошибок на корреляцию в остатках на уровне индивидуальных наблюдений. Статистические пакеты еще не решают эту проблему автоматически. Бертран, Дуфло, и Маллайнатан (2004) иллюстрируют смещение вниз вычисляемых стандартных ошибок в контексте оценивания методом <<разность разностей>> (см. раздел 22.6). Они выяснили, что робастные методы для панельных данных и бутстрэп методы успешно справляются с проблемой, хоть и количество индивидуальных наблюдений $N$ (количество штатов) в их выборке достаточно мало, в то время как асимптотическая теория предполагает $N \rightarrow \infty$.

Следующий пример (см. таблицу 21.2) также показывает значимость корректирования стандартных ошибок при наличии любой корреляции или автокорреляции ошибок.

\section{Пример линейной модели панельных данных: количество часов работы и заработная плата}

Один из важных вопросов экономики рынка труда  --- чувствительность предложения труда к заработной плате. Стандартная модель предложения труда предполагает, что для уже работающих людей эффект увеличения заработной платы на предложение труда неоднозначен. Эффект дохода, стимулирующий меньше работать, компенсирует эффект замещения, стимулирующий работать больше.

Анализ данных пространственного типа о взрослых мужчинах выявил слабую положительную зависимость заработной платы и количества часов работы. Однако, вполне верятно, что это взаимосвязь ложная, так как она просто отображает ненаблюдаемое желание больше работать, связанное с более высокой оплатой труда. Анализ панельных данных позволяет учитывать такой феномен в предположении, что это ненаблюдаемое желание больше работать не меняется во времени. Например, оценка within учитывает этот феномен, измеряя насколько больше (меньше), чем среднее количество часов, работает сотрудник тогда, когда заработная плата выше (ниже) среднего.

Допустим, имеются данные о 532 мужчинах для каждого из 10 лет с 1979 по 1988 гг. (Зилиак, 1997). Исследуемая переменная --- lnhrs, натуральный логарифм ежегодного количества часов работы. Единственная объясняющая переменная --- lnwg, натуральный логарифм почасовой заработной платы. Рассмотрим регрессионную модель
\begin{align}
\mathrm {lnhrs}_{it}=\alpha_i +\beta \mathrm {lnwg}_{it}+\e_{it},
\nonumber
\end{align}
где индивидуальный эффект $\alpha_i$ в некоторых моделях упрощается до $\alpha$, и $\beta$ измеряет эластичность предложения труда. Предполагается, что ошибки $\e_{it}$ независимы по $i$, но могут быть коррелированы по $t$ для данного $i$. Как уже было упомянуто, ожидается слабая положительная эластичность предложения труда $\beta$.

Зилиак (1997) дополнительно включал в модели квадрат возраста, количество детей, индикатор плохого здоровья. Эти регрессоры и фиктивные переменные для каждого года в незначительной степени меняют оценку $\beta$ и стандартных ошибок. Для краткости они здесь пропущены. В главе 22 мы рассмотрим более общие модели, которые не исключают эндогенность lnwg и позволяют использовать лаги lnhrs в качестве регрессоров.

\subsection{Результаты оценивания}
Для 5320 наблюдений выборочные средние значения lnhrs и lnwg --- 7.66 и 2.61 соответственно, геометрическое среднее значение 2,120 часов в год и 13.60\$  в час. Выборочные стандартные отклонения  --- 0.29 и 0.43 соответственно, что говорит о более высокой дисперсии заработной платы, чем часов работы.

При анализе панельных данных полезно знать, где главным образом присутствует дисперсия --- среди индивидуальных наблюдений или же во времени. Общая дисперсия ряда $x_{it}$ может быть представлена в виде декомпозиции
\begin{align}
\sum_{i=1}^N \sum_{t=1}^T (x_{it}-\bar x)^2
& =\sum_{i=1}^N \sum_{t=1}^T [(x_{it}-\bar x_i)+(\bar x_i-\bar x)]^2 \nonumber \\
& =\sum_{i=1}^N \sum_{t=1}^T(x_{it}-\bar x_i)^2+\sum_{i=1}^N \sum_{t=1}^T(\bar x_i --- \bar x)^2,
\nonumber
\end{align}
так как $2 \sum_{i=1}^N \sum_{t=1}^T (x_{it}-\bar x_i)(\bar x_i-\bar x)=0$. Другими словами, общая сумма квадратов равна сумме \textbf{внутригрупповой суммы квадратов} и \textbf{межгрупповой суммы квадратов}. Такая декомпозиция позволяет определить \textbf{внутригрупповое стандартное отклонение} $s_W$ и \textbf{межгрупповое стандартное отклонение} $s_B$, где
\begin{align}
s^2_W =\frac{1}{NT-N} \sum_{i=1}^N \sum_{t=1}^T (x_{it}-\bar x_i)^2
\nonumber
\end{align}
и
\begin{align}
s^2_B =\frac{1}{N-1} (\bar x_i-\bar x)^2.
\nonumber
\end{align}

\begin{table}[ht]
\caption{{\it Заработная плата и количество часов работы: Стандартные оценки линейных моделей панельных данных} ${}^a$} 
\centering
\begin{tabular}{p{2cm} p{1.5cm} p{2cm} p{1.6cm} p{2cm} p{2cm} p{2cm}}
\hline \hline
				& \textbf{POLS} &  \textbf{Between} & \textbf{Within} & \textbf{First Diff} & \textbf{RE-GLS} & \textbf{RE-MLE}\\
\hline
$\alpha$ 	& 7.442	& 7.483  & 7.220& .001&  7.346&7.346\\
 $\beta$& .083&.067 &.168 &.109 & .119& .120\\
 Робастные ст.ош. ${}^b$ & (.030)&(.024) &(.085) &(.084) & (.051)&(.052) \\
 Ст.ош. (бутстрэп) & [.030] & [.019]& [.084]& [.083]& [.056]&[.058] \\
 Дефолтные ст.ош.&{.009} &{.020} & {.019}& {.021}& {.014}& {.014}\\
\hline
$R^2$& .015 &.021 & .016& .008&.014& .014\\
RMSE & .283& .177&.233&.296 & .233&.233 \\
RSS & 427.225& 0.363& 259.398& 417.944& 288.860 & 288.612\\
TSS & 433.831& 17.015& 263.677& 420.223&293.023 & 292.773\\
$\sigma_{\alpha} $&.000 & & .181& &.161 & .162\\
$\sigma_{\e}$ & .283& & .232& & .233& .233\\
$\lambda$ & 0.000& -&1.000 &- &.585 &.586 \\
$N$ &5320 & 532& 5320& 4788& 5320& 5320\\
\hline \hline
\multicolumn{7}{p{15cm}}{${}^a$ Представленные оценки: объединенная МНК-оценка (POLS), внутригрупповая (between), межгрупповая (within), в первых разностях (first diff), ОМНК-оценка со случайным эффектом (RE-GLS) и ММП-оценка линейной регрессии lnhrs на lnwg. В круглых скобках представлены стандартные ошибки для наклонов коэффициентов робастные к гетероскедастичности, в квадратных скобках --- стандартные ошибки, полученные бутстрэп методом, в фигурных скобках --- стандартные ошибки, вычисленные по умолчанию в предположении, что ошибки независимы и одинаково распределены. $R^2$, корень из среднеквадратичной ошибки (RMSE), сумма квадратов остатков (RSS), общая сума квадратов (TSS) и количество наблюдений в выборке вычислены для соответствующих моделей, описанных в разделе 21.2. Параметр $\lambda$ определен после \ref{Eq:21.11}.} \\
\multicolumn{7}{l}{${}^b$ ст.ош. --- стандартные ошибки}
\end{tabular}
\label{Tab:21.2}
\end{table}

Внутригрупповые и межгрупповые выборочные стандартные отклонения --- 0.22 и 0.18 соответственно для lnhrs и 0.19 и 0.39 для lnwg. Таким образом, большая изменчивость заработной платы по сравнению с часами работы объясняется более высокой межгрупповой изменчивостью для заработной платы. Заметим, что внутригрупповая изменчивость несколько меньше для заработной платы, чем для количества рабочих часов.

\subsection{Сравнение оценок, используемых при анализе для панельных данных}
Таблица \ref{Tab:21.2} обобщает результаты применения стандартных оценок панельных данных, представленных в разделе 21.2.2, наряду с тремя разными оценками стандартных ошибок. Как показано ниже, для надежных статистических выводов необходимо использовать робастные для панельных данных стандартные ошибки  или робастные стандартные ошибки, полученные бутстрэп методом.

{\centering
Оценки коэффициентов наклона\\}

Оценка коэффициента наклона $\beta$ различается в зависимости от метода оценивания.  Оценка between, которая использует только пространственную изменчивость, меньше, чем МНК оценка модели сквозной регрессии. Оценка within или оценка с фиксированным эффектом, равная 0.168, значительно выше, чем МНК оценка модели сквозной регрессии, равная 0.083. Оценки статистически значимы, как показывает двусторонний тест на 5\% уровне значимости, оценки стандартных ошибок равны 0.084 и 0.085. %?? 
Оценка в первых разностях, равная 0.109, также превышает МНК оценку модели сквозной регрессии, но оказывается значительно ниже оценки within, которая тоже использует только изменчивость во времени. Оценка со случайным эффектом, равная 0.119 или 0.120, лежит между оценками between и within. Это вполне ожидаемо, так как оценки со случайным эффектом могут быть представлены как \textbf{взвешенное среднее within и between оценок}. Две оценки со случайным эффектом очень близки друг к другу, так как здесь оценки дисперсий $\sigma^2_{\alpha}$ и $\sigma^2_{\e}$ одинаковы для обеих оценок, что является причиной близких значений $\hat{\alpha}=0.585$ и $\hat{\alpha}=0.586$ в регрессии \ref{Eq:21.10}. Удивительно, что оценки со случайным эффектом оказались менее эффективными, чем МНК оценки сквозной регрессии. Это говорит о том, что модель со случайным эффектом плохо описывает данные с корреляцией в ошибках. 

Какая оценка более предпочтительна? Оценка within и оценка модели в первых разностях состоятельны при любой истинной модели (модели сквозной регрессии, модели со случайным эффектом, модели с фиксированным эффектом), в то время как другие оценки несостоятельны в случае истинности модели с фиксированным эффектом. Вследствие этого  самыми робастными оценками являются оценка within и оценка модели в первых разностях, равные 0.168 и 0.109.

Однако, используя более робастные оценки, мы теряем в эффективности. Стандартные ошибки, равные 0.83 и 0.85, намного выше тех, которые получены при получении МНК оценивании сквозной регрессии и оценки со случайным эффектом. Можно использовать формальный тест Хаусмана (для более подробного описания и обсуждения см. раздел 21.4.3), чтобы определить, являются ли индивидуальные эффекты фиксированными. При относительной неточности оценивания в нашем примере, тест Хаусмана не отвергает нулевую гипотезу о случайных эффектах, несмотря на большую разницу между оценками со случайными и фиксированными эффектами. Так, более эффективная оценка со случайным эффектом могла бы быть использована здесь. Другое преимущество оценки со случайным эффектом состоит в том, что она позволяет оценивать коэффициенты регрессоров, меняющихся во времени.

{\centering
Оценивание стандартных ошибок\\}

Перейдем к сравнению оценок стандартных ошибок. Как отмечено в разделе 21.2.3, статистические выводы должны быть основаны на робастных стандарных ошибках, которые учитывают корреляцию ошибок во времени для данного индивидуального наблюдения и разные дисперсии и ковариации индивидуумов. Также, как подробнее описано в последующих разделах, стандартные ошибки, основанные на отклонениях от средних значений, таких как \ref{Eq:21.8} и \ref{Eq:21.10},  должны учитывать потерю не $K$, а $N+K$ степеней свободы.

Первая оценка стандартной ошибки вычислена с помощью робастного метода, описанного в  \ref{Eq:21.13}, а вторая вычислена с помощью бутстрэп метода (\ref{Eq:21.14}) с 500 симуляциями. Для краткости эти оценки называются робастными для панельных данных, хотя они еще робастны к гетероскедастичности. Эти две оценки очень близки друг к другу и далеки от оценок модели со случайным эффектом, где робастные для панельных данных стандартные ошибки недооценены вследствие того, что вычислены для регрессии \ref{Eq:21.10}, которые игнорируют ошибки в оценивании $\hat{\lambda}$.

Третья оценка стандартной ошибки  --- значение, вычисляемое компьютером по умолчанию в предположении, что ошибки независимы и одинаково распределены. В нашем примере правильно специфицированные стандартные ошибки выше, чем стандартные ошибки по умолчанию в три-четыре раза. Исключение составляет оценка between, оценка со стандартными ошибками, которые необходимо скорректировать только на гетероскедастичность, так как она использует только изменчивость между кросс-секциями. 

Например, для МНК оценки сквозной регрессии коэффициента $\beta$ стандартная ошибка по умолчанию равна 0.09, что приводит к неверной $t$-статистике 9.07. Робастная для панельных данных стандартная ошибка намного выше --- 0.30, поэтому верные $t$-статистики намного меньше --- 2.83. Стандартные ошибки, вычисляемые по умолчанию, предполагают независимость ошибок в модели по времени для данного $i$, в то время как на практике они часто положительно коррелированы. Из-за этого ошибочного предположения переоценивается выгода от дополнительных временных периодов, что приводит к смещению стандартных ошибок вниз (см. раздел 21.5.4). К тому же, игнорирование гетероскедастичности в ошибках также приводит к смещению, хотя это смещение могло иметь любое направление. В случае с нашими данными неудавшаяся попытка учесть гетероскедастичность также приводит к большому смещению вниз: стандартная ошибка $\hat{\beta}_{POLS}$, учитывающая гетероскедастичность, но не учитывающая корреляцию во времени для данного $i$, равна 0.020. Для любых других данных коррекция на гетероскедастичность обычно менее важна, чем коррекция на корреляцию во времени. 

Включение $\alpha_i$ для оценок between и within позволяет учитывать некоторую корреляцию ошибок во времени для данного индивидуального наблюдения. В случае с нашими данными разница между робастными для панельных данных и неробастными стандартными ошибками остается значительной, частично из-за неудавшейся попытки дополнительно учесть гетероскедастичность.

Очевидно, необходимо использовать робастные для панельных данных стандартные ошибки. 

\subsection{Графический анализ}

Проведение графического сравнения сквозной, between регрессий и регрессий с фиксированным эффектом (within или в первых разностях) очень полезно. Такие графики редко представлены в регрессиях для панельных данных, но их легко использовать в данном случае, так как у нас имеется только один регрессор.

Все графики включают линии непараметрической регрессии с использованием локально взвешенного сглаживания (Lowess, см. раздел 9.6.2) и линию линейной регрессии, которая соотносится с оценками, представленными в таблице \ref{Tab:21.2}.

Рисунок \ref{Fig:21.2} изображет зависимость lnhrs от lnwg для всех фирм за все годы (5320 наблюдений). На графике показана положительная взаимосвязь, почти линейная за исключением крайних значений. Наклон равен 0.083, а $R^2=0.015$ достаточно низок (см. табл. \ref{Tab:21.2}).

Оценка between (\ref{Eq:21.7}) вычисляется из регрессии $\bar{y}_i$  на $\bar{x}_i$. Соответствующий график представлен на рисунке \ref{Fig:21.2}, где снова замечается положительная взаимосвязь. 

Оценка within и оценка с фиксированным эффектом (21.8) вычисляется из регрессии $(y_{it}-\bar{y}_i)$  на $(x_{it}-\bar{x}_i)$. На рисунке \ref{Fig:21.3} представлен связанный график $(y_{it}-\bar{y}_i+\bar{y})$  на $(x_{it}-\bar{x}_i+\bar{x})$, где $\bar{y}=N^{-1}\sum_{i} \bar{y}_i$ и $\bar{x}=N^{-1} \sum_{i} \bar{x}_i$ и общих средних по $y$ и $x$. Сравнение с рисунком \ref{Fig:21.1} показывает, что вычитание индивидуального среднего приводит к значительному снижению изменчивости в lnwg и менее значительному снижению изменчивости в lnhrs. Наклон оказывается более крутой, чем для сквозной регрессии, и по сравнению с таблицей \ref{Tab:21.2} наклон увелиился с 0.083 до 0.168.

Оценка модели в первых разностях \ref{Eq:21.9} получается посредством построения регрессии $(y_{it}-y_{i,t-1})$ на $(x_{it} --- x_{i,t-1})$. Соответствующий график для lnhrs --- lnwg данных представлен на рис \ref{Fig:21.4}. Рисунок качественно не отличается от рисунка \ref{Fig:21.3}.
 
Вывод предшествующего анализа состоит в том, что количество часов работы чувствительно к изменению заработной платы в большей степени ввиду изменчивости во времени, а не изменчивости внутри кросс-секций.

  \begin{figure}[ht]
                \begin{center}
                    \includegraphics[scale=1.2]{chapters/fig211}
                    \caption{Количество часов работы и заработная плата: объединенная регрессия. По оси ординат --- натуральный логарифм ежегодных часов работы, по оси абсцисс --- натуральный логарифм заработной платы за час. Данные для 532 мужчин, проживающих в США, для каждого из 1979-88 гг. }
                    \label{Fig:21.1}
                \end{center}
     \end{figure}

 \begin{figure}[ht]
                \begin{center}
                    \includegraphics[scale=1.2]{chapters/fig212}
                    \caption{Количество часов работы и заработная плата: межгрупповая регрессия. По оси ординат --- среднее за 10 лет логарифма рабочих часов, по оси абсцисс  --- среднее за 10 лет логарифма заработной платы для 532 мужчин. Та же выборка, что и на рисунке \ref{Fig:21.1}}
                    \label{Fig:21.2}
                \end{center}
     \end{figure}

\subsection{Анализ остатков}

Было бы полезно рассмотреть также автокорреляцию данных и остатков. Например, для остатков $\hat{u}_{it}=y_{it} --- \hat{y}_{it}$ автокорреляция между периодом $s$ и периодом $t$ рассчитывается как $\hat{\rho}_{st} = c_{st}/\sqrt{c_{ss} c_{tt}}, s, t = 1, \dots, T$, где оценка ковариации  $c_{st}=(N-1)^{-1} \sum_i (\hat{u}_{it} = \bar{\hat{u}}_t)(\hat{u}_{is} = \bar{\hat{u}}_s)$ и $\bar{\hat{u}}_t= N^{-1} \sum_i \hat{u}_{it}$.

 \begin{figure}[ht]
                \begin{center}
                    \includegraphics[scale=1.2]{chapters/fig213}
                    \caption{Количество часов работы и заработная плата: внутригрупповая (с фиксированным эффектом) регрессия. По оси ординат --- отклонение логарифма рабочих часов от среднего за 10 лет , по оси абсцисс --- отклонение логарифма заработной платы от среднего за 10 лет  по данным о 532 мужчинах. Та же выборка, что и на рисунке \ref{Fig:21.1}}
                    \label{Fig:21.3}
                \end{center}
     \end{figure}

 \begin{figure}[ht]
                \begin{center}
                    \includegraphics[scale=1.2]{chapters/fig214}
                    \caption{Количество часов работы и заработная плата: оценка модели в первых разностях. По оси ординат --- первая разность логарифма рабочих часов, по оси абсцисс --- первая разность логарифма заработной платы по данным о 532 мужчинах. Та же выборка, что и на рисунке \ref{Fig:21.1}}
                    \label{Fig:21.4}
                \end{center}
     \end{figure}

В таблице \ref{Tab:21.3} представлена автокорреляция остатков после оценивания модели сквозной регрессии lnhrs на lnwg. Автокорреляция обычно лежит в пределах от 0.2 до 0.4 для наблюдений, находящихся в двух-девяти годах друг от друга. Степень угасания очень мала, и автокорреляция, скорее, тяготит к модели со случайным эффектом, которая предполагает, что $Cor[u_{it}, u_{is}]$  постоянная для $t \neq s$, чем к экспоненциально угасающей модели AR(1).

Автокорреляции lnhrs для регрессии очень близки к тем, что даны в таблице \ref{Tab:21.3}, так как $\hat{u}_{it} \simeq y_{it}$, что ожидаемо исходя из слабой объяснительной силы сквозной регрессии с $R^2=0.015$. Автокорреляция регрессора lnwg, не представленная здесь, значительно больше, она принимает значения от 0.9 для одного лага до 0.7 для девяти лагов.

Корреляции остатков из внутригрупповой регрессии даны в таблице \ref{Tab:21.4}. Если исходные ошибки $\e_{it}$ в \ref{Eq:21.3} независимы и одинаково распределены, тогда можно показать, что автокорреляция преобразованных ошибок $\e_{it}-\bar{\e}_i$ для любого лага равна $-1/(T-1)=-0.11$. Исключение может составлять только первый лаг, для которого корреляция всегда положительна.


\begin{table}[ht]
\caption{{\it Заработная плата и количество часов работы: Автокорреляции объединенной МНК-регрессии} ${}^a$} 
\centering
\begin{tabular}{ccccccccccc}
\hline \hline
	&	\textbf{u79} & \textbf{u80} & \textbf{u81} & \textbf{u82} & \textbf{u83} & \textbf{u84} & \textbf{u85} & \textbf{u86} & \textbf{u87} & \textbf{u88}  \\
\hline
upols79 & 1.00 & & & & & & & & & \\
upols80 & .33 	& 1.00 & & & & & & & & \\
upols81 & .44	& .40 & 1.00 & & & & & & & \\
upols82 & .30	& .31 & .57 & 1.00 & & & & & & \\
upols83 & .21	& .23 & .37 & .47 & 1.00 & & & & & \\
upols84 & .20	& .23 & .32 & .34 & .64 & 1.00 & & & & \\
upols85 & .24	& .32 & .41 & .35 & .39 & .58 & 1.00 & & & \\
upols86 & .20	& .19 & .28 & .25 & .31 & .35 & .40 & 1.00 & & \\
upols87 & .20	& .32 & .33 & .29 & .31 & .34 & .39 & .35 & 1.00 & \\
upols88 & .16	& .25 & .30 & .26 & .21 & .25 & .34 & .55 & 0.53 &  1.00\\
\hline \hline
\multicolumn{11}{p{15cm}}{${}^a$ Прим.: Автокорреляции остатков взяты из объединенной МНК-регрессии lnhrs на lnwg для 532 мужчин за 10 лет. Автокорреляции медленно угасают.}
\end{tabular}
\label{Tab:21.3}
\end{table}

\begin{table}[ht]
\caption{{\it Заработная плата и количество часов работы: Автокорреляции остатков внутригрупповой регрессии} ${}^a$} 
\centering
\begin{tabular}{ccccccccccc}
\hline \hline
	&	\textbf{u79} & \textbf{u80} & \textbf{u81} & \textbf{u82} & \textbf{u83} & \textbf{u84} & \textbf{u85} & \textbf{u86} & \textbf{u87} & \textbf{u88}  \\
\hline
upols79 & 1.00 & & & & & & & & & \\
upols80 & .10 	& 1.00 & & & & & & & & \\
upols81 & .21	& .08 & 1.00 & & & & & & & \\
upols82 & .00	& -.04 & .26 & 1.00 & & & & & & \\
upols83 & -.26& -.27 & -.21 & .01 & 1.00 & & & & & \\
upols84 & -.26& -.27 & -.30 & -.20 & .32 & 1.00 & & & & \\
upols85 & -.18& -.10 & .41 & .35 & .39 & .58 & 1.00 & & & \\
upols86 & -.19& -.25 & -.26 & -.27 & -.17 & -.14 & -.08 & 1.00 & & \\
upols87 & -.15& -.05 & -.16 & -.20 & -.24 & -.21 & -.09 & -.09 & 1.00 & \\
upols88 & -.17& -.11 & -.14 & -.18 & -.38 & -.31 & .13 & .24 & .24 &  1.00\\
\hline \hline
\multicolumn{11}{p{15cm}}{${}^a$ Прим.: Автокорреляции остатков взяты из внутригрупповой (с фиксированным эффектом) регрессии lnhrs на lnwg для 532 мужчин за 10 лет.}
\end{tabular}
\label{Tab:21.4}
\end{table}

Корреляции остатков регрессии со случайным эффектом довольно похожи на корреляции остатков регрессии с фиксированным эффектом, представленные в таблице \ref{Tab:21.4}. Корреляции остатков регрессии в первых разностях качественно близки к теоретическому результату, что если исходные ошибки $\e_{it}$ в \ref{Eq:21.3} независимы и одинаково распределены, то преобразованные ошибки $\e_{it} --- \e_{it-1}$ имеют автокорреляцию от 0.5 для первого лага до 0 для остальных лагов.

\section{Модели с фиксированным эффектом против моделей со случайным эффектом}

Модель с фиксированным эффектом привлекательна тем, что она позволяет использовать панельные данные для выявления взаимосвязей, основываясь на более слабых предположениях (представленных в разделе 21.4.1), чем те, которые необходимы при использовании данных пространственного типа или панельных данных без фиксированных эффектов, таких как объединенные модели или модели со случайным эффектом.

Для некоторых данных зависимость очевидна, поэтому модель со случайным эффектом может адекватно описывать данные, как, например, в контролируемом эксперименте об уровне урожайности в зависимости от различного количества удобрений. В других случаях использование модели со случайным эффектом может оказаться целесообразным для измерения степени корреляции.  При этом для определение причинности нужны другие подходы. Хороший пример --- связь курения и рака легких. Однако, экономисты обычно предпочитают модели со случайным эффектом модель с фиксированным эффектом, так как желают измерить \textbf{причинность}, несмотря на наблюдаемые данные.

У модели с фиксированным эффектом есть некоторые практические слабые стороны. Оценивание коэффициента любого регрессора, не меняющегося во времени, как, например, индикаторной переменной {\it пол} невозможна, так как ее влияние поглощается в индивидуальном эффекте. Коэффициенты регрессоров, меняющихся во времени, могут быть оценены, но эти коэффициенты могут быть очень неточными, если большая часть изменчивости регрессоров отражается не во времени, а  внутри кросс-секций. Прогнозирование условного среднего невозможно. Вместо этого могут быть прогнозированы только изменения условного среднего, обусловленного изменениями регрессоров, меняющихся во времени. Даже коэффициенты регрессоров, меняющихся во времени, может оказаться сложно или теоретически невозможно идентифицировать в нелинейных моделях с фиксированных эффектах (см. главу 23).  Ввиду этих причин экономисты также используют модели со случайным эффектом, даже если причинная интерпретация может быть необоснованна.

\subsection{Пример применения модели с фиксированными эффектами}

Рассмотрим влияние использования компьютера на заработную плату.  В некоторых исследованиях, использющих данные пространственного типа, (например, Крюгер (1993), ДиНардо и Пишке (1997)) выявляется, что использование компьютера в работе связано с существенно более высокими заработными платами, даже при учете многих детерминант заработной платы, таких как образование, возраст, пол, отрасль и занятость. Как было отмечено в работе ДиНардо и Пишке (1997), это необязательно говорит о причинности. Если регрессоры коррелированы с ошибкой, это может идентифицировать проблему эндогенности или пропущенных переменных.

 А именно, рассмотрим пространственную модель:
\begin{align}
y_i=\x'_i \bm\beta+\alpha_i +\e_i,
\nonumber
\end{align}
где $y$ --- это натуральный логарифм заработной платы, $\x$ --- вектор индивидуальных характеристик, включая фиктивную переменную для использования компьютера в работе, и $\e$ --- ошибка, которая по предположению независима от $\x$. Трудность состоит в добавлении ненаблюдаемой переменной $\alpha$, которая по предположению коррелирована с переменной <<использование компьютера на работе>>, а поэтому и с наблюдаемым регрессором $\x$, хотя компоненты $\x$ такие, как занятость и образование, могут частично учитывать влияние переменной {\it использование компьютера на работе}. Оценивание регрессии $y$ на $\x$ приводит к проблеме \textbf{смещения в связи с пропущенной переменной}, что в свою очередь является причиной появления несостоятельных оценок $\beta$, так как комбинированная ошибка $(\alpha + \e)$ коррелирована с $\x$.

Благодаря панельным данным можно обойти эту проблему, если предположить, что ненаблюдаемая переменная $\alpha_i$ не меняется во времени. Тогда 
\begin{align}
y_{it}=\x'_{it} \bm\beta+\alpha_i +\e_{it},
\nonumber
\end{align}
где снова $\e$  некоррелирована с $\x$ и $\alpha$ коррелирована с $\x$. Путем взятия разностей мы избавляемся от $\alpha_i$ (см. раздел 21.2.2), что позволяет получить состоятельную оценку $\beta$. В примере с компьютером, влияние использования компьютера на заработную плату измеряется взаимосвязью между индивидуальными изменениями заработной плате и индивидуальными изменениями в использовании компьютера в работе. Хайскен-Денью и Шмидт (1999) не находят подтверждение этому эффекту на панельных немецких данных.

Подход к оценке панельных данных с использованием фиксированных эффектов позволяет определить причинность при более слабых предпосылках, чем при анализе данных пространственного типа. Однако, сделать предположения все же необходимо. Ключевое предположение состоит в том, что ненаблюдаемая $\alpha_i$ неизменна во времени (иначе она содержала бы индекс $t$: $\alpha_{it}$). В нашем примере предполагается, что индивидуальная предрасположенность к использованию компьютера на работе может быть эндогенна, но $\alpha_i$, ненаблюдаемая часть эффекта этой склонности к использованию компьютера  на заработную плату, постоянна во времени, как только мы учтем наблюдаемые  $\x_{it}$. По существу, когда мы учитываем влияние ненаблюдаемой $\alpha_i$ и наблюдаемого $\x_{it}$, отдельные временные периоды, в которые работа индивидуума предполагает или не предполагает использование компьютера, предполагаются чисто случайными.

Метод оценивания объединенной регрессии и метод оценивания с использованием случайных эффектов не имеют таких свойств. Метод оценивания с использованием случайных эффектов не предполагает коррелированность $\alpha$ и $\x$, напротив, он предполагает, что 
 $\alpha$ независимы и одинаково распределены с параметрами $[0,\sigma^2]$ и поэтому некоррелированы с $\x$. Если на самом деле $\alpha$ коррелирована с $\x$, метод оценивания с использованием случайных эффектов дает несостоятельные оценки, в то время как оценка с фиксированным эффектом состоятельна, даже если $\alpha$ коррелирует с $\x$ при условии неизменности  $\alpha$ во времени. 

\subsection{Условный анализ против предельного анализа}

Оценивание модели с фиксированным эффектом  --- это \textbf{условный анализ}, исследующий влияние $\x_{it}$ на $y_{it}$, учитывая индивидуальный эффект $\alpha_i$. Построение проноза возможно только для индивидуумов той же самой выборки, и даже тогда нужна длинная панель для получения состоятельной оценки $\alpha_i$. Оценивание модели со случайным эффектом, напротив, является примером \textbf{предельного анализа} или \textbf{усредненного (population-averaged) анализа}, так как индивидуальные эффекты трактуются как независимые и одинаково распределенные случайные переменные. Выводы оценивания моделей со случайным эффектом могут распространяться не только на выборку, но и на генеральную совокупность.

Если истинная модель  --- это модель со случайным эффектом, тогда тип анализа (условный или предельный) выбирается в зависимости от конкретных данных. Если анализ проводится для случайной выборки стран, тогда необходимо использовать модель со случайным эффектом. Однако, если исследователю в действительности интересны конкретные страны, входящие в выборку, тогда применяется оценивание с фиксированным эффектом, хотя это может повлечь за собой потерю эффективности. 

Однако, если истинная модель вместо этого содержит индивидуальные эффекты, коррелированные с регрессорами, тогда использование оценки со случайным эффектом уже не имеет смысла, так как оценка со случайным эффектом несостоятельна. Вместо этого в таком случае необходимо использовать альтернативные оценки, такие как оценка с фиксированным эффектом и оценки модели в первых разностях. В целях определить причинные взаимосвязи в микроэкономических приложениях чаще используются последние методы оценки.

\subsection{Тест Хаусмана}

Если индивидуальные эффекты фиксированны, то оценка within $\hat{\beta}_W$ состоятельна, в то время как оценка со случайным эффектом $\tilde{\beta}_{RE}$ несостоятельна. Здесь $\beta$ относится к вектору коэффициентов только тех регрессоров, которые не меняются во времени. С помощью теста Хаусмана можно проверить, имеют ли место фиксированные эффекты. Тест проверяет, есть ли статистически значимая разница между этими оценками. В качестве альтернативы можно протестировать любую другую пару оценок с похожими свойствами, например, оценку модели в первых разностях и МНК оценку модели сквозной регрессии.

При большом значении статистики теста Хаусмана нулевая гипотеза о том, что индивидуальные эффекты некоррелированы с регрессорами, отвергается, и делается вывод о том, что модель с фиксированным эффектом лучше описывает данные. При этом все еще можно избежать использования модели с фиксированным эффектом. Если регрессоры коррелированы с индивидуальными эффектами, вызванными пропущенными переменными, тогда можно добавлять дополнительные регрессоры, меняющиеся или неменяющиеся во времени, и снова провести тест Хаусмана в этой расширенной модели для повторной проверки значимости фиксированных эффектов. Даже если такая корреляция сохранится, то можно оценить модель со случайным эффектом, используя метод инструментальных переменных (см. раздел 22.4.3-22.4.4). 

{\centering
Ситуация эффективности модели со  случайным эффектом\\}

Мы начнем с предположения о том, что истинная модель  --- это модель со случайным эффектом \ref{Eq:21.3}, где $\alpha_i$ независимы и одинаково распределены с параметрами $[0,\sigma^2_{\alpha}]$ и некоррелированы с регрессорами и ошибками $\e_{it}$, которые в свою очередь независимы и одинаково распределены с параметрами $[0,\sigma^2_{\e}]$.

Тогда оценка $\tilde{\beta}_{RE}$ эффективна, и из раздела 8.3 статистика \textbf{теста Хаусмана} упрощается до
\begin{align}
H=\left(\tilde{\beta}_{1,RE} --- \hat{\beta}_{1,W}\right)' \left[\hat{V}[\hat{\beta}_{1,W}]-\hat{V}[\tilde{\beta}_{1,RE}]\right]^{-1} \left(\tilde{\beta}_{1,RE} --- \hat{\beta}_{1,W}\right),
\nonumber
\end{align}
где $\beta_1$ обозначает часть вектора $\beta$, соответствующую регрессорам, меняющимся во времени, так как только эта часть может быть оценена с помощью оценки within. При нулевой гипотезе эта статистика асимптотически распределена как $\chi^2(\dim[\bm\beta_1])$.

Хаусман (1978) показал, что асимптотически эквивалентной версией этого теста является тест Вальда о $\gamma=0$ во вспомогательной регрессии,
\begin{align}
y_{it}-\hat{\lambda}\bar{y}_i=(1-\hat{\lambda})\mu + (\x_{1it}-\hat{\lambda}\bar{\x}_{1i})'\beta_1+(\x_{1it}-\bar{\x}_{1i})'\gamma+v_{it},
\label{Eq:21.15}
\end{align}
где $\x_{1it}$ обозначает регрессоры, меняющиеся во времени, и $\hat{\lambda}$ определена в \ref{Eq:21.11}. Используются только регрессоры, меняющиеся во времени. Этот результат может быть интерпретирован следующим образом. Из модели с индивидуальными эффектами \ref{Eq:21.10} следует, что $v_{it}=(1-\hat{\lambda})\alpha_i+(\e_{it}-\hat{\lambda}\bar{\e}_i)$. Оценка со случайным эффектом в действительности получена МНК-оцениванием модели \ref{Eq:21.15} c $\gamma=0$ (cм. \ref{Eq:21.10}). Если же более подходящей является спецификация с фиксированным эффектом, тогда ошибка $v_{it}$ будет коррелирована с регрессорами из-за корреляции $\alpha_i$ с регрессорами. Эта корреляция приводит к тому, что такие регрессоры, как $(\x_{it}-\bar{\x_i})$, становятся статистическими значимыми переменными в \ref{Eq:21.15}.

{\centering
Ситуация неэффективности модели со случайным эффектом\\}

Простая форма теста Хаусмана недействительна, когда $\alpha_i$ либо $\e_{it}$ не являются независимыми и одинаково распределенными, что очень часто встречается при гетероскедастичности в микроэкономических данных. Тогда оценка со случайным эффектом не эффективна при нулевой гипотезе, поэтому выражение $\hat{V}[\hat{\beta}_W]-\hat{V}[\tilde{\beta}_{RE}]$ необходимо заменить на более общее $\hat{V}[\tilde{\beta}_W-\tilde{\beta}_{RE}]$ (см. раздел 8.3).

В случае коротких панелей эта ковариационная матрица может быть состоятельно оценена с помощью бутстрэп по $i$ (см. раздел 21.2.3). Тогда робастной для панельных данных статистика Хаусмана будет
\begin{align}
\hat{V}_{Boot}[\tilde{\beta}_{1,W}]=\frac{1}{B-1} \sum^B_{b=1} 
\left(\hat{\delta}_b-\bar{\hat{\delta}}\right)
\left(\hat{\delta}_b-\bar{\hat{\delta}}\right)',
\end{align}
где $b$ обозначает $b$-ю репликацию (из $B$) (см. раздел 21.2.3), и $\hat{\delta}=\tilde{\beta}_{1,RE}-\hat{\beta}_{1,W}$. Эта статистика может быть применена к компоненте $\beta_1$, а вместо $\hat{\beta}_{1,RE}$ может использоваться  оценка $\tilde{\beta}_{1,POLS}$, и вместо  $\hat{\beta}_{1,W}$ --- оценка $\hat{\beta}_{1,FD}$.

Как альтернативу Вулдридж (2002) предлагает оценивание вспомогательной МНК регрессии \ref{Eq:21.15} и тестирование $\gamma=0$ с помощью робастных для панельных данных стандартных ошибок. Если эффекты случайны, хотя и необязательно, чтобы $\alpha_i$ и $\e_{it}$ были независимыми и одинаково распределенными, тогда $v_{it}=(1-\hat{\lambda})\alpha_i+(\e_{it}-\hat{\lambda}\bar{\e}_i)$ все еще будут коррелированы с регрессорами, даже не будучи асимптотически независимыми и одинаково распределенными. Таким образом, нужно использовать \textbf{кластерные робастные стандартные ошибки}. Если эффекты фиксированы, тогда ошибка  $v_{it}$ коррелирована с регрессорами, и такие регрессоры, как $(\x_{it}-\bar{\x_i})$, становятся статистическими значимыми переменными. Эта робастная версия вспомогательной регрессии для теста Хаусмана предпочтительнее той, которая предполагает, что  $v_{it}$ асимптотически независимы и одинаково распределены, основываясь на минимальных предположениях о распределении. Однако, не совсем ясно, совпадает ли этот тест с тестом Хаусмана, когда оценка со случайным эффектом неэффективна.


{\centering
Пример теста Хаусмана\\}

Для примера lnhrs-lnwg, представленном в таблице \ref{Tab:21.2}, сравнение оценок с фиксированным и случайным эффектом с использованием стандартных ошибок, вычисляемых по умолчанию, дает значение статистики: $H \simeq (0.168 --- 0.119)^2/(0.019^2-0.014^2)$. Так $H=14 > \chi^2_{.05}(1)=3.84$, т.е. модель со случайным эффектом отвергается.

Однако, в нашем случае тест неприменим. Статистика $H$ завышена, так как обычные стандартные ошибки в этом примере сильно смещены вниз (см. раздел 21.3.2). Более того, это смещение  --- признак того, что оценка со случайным эффектом не эффективна при нулевой гипотезе, поэтому необходимо использовать более общую форму теста Хаусмана.

Использование вспомогательной регрессии \ref{Eq:21.15}  дает робастную для панельных данных $t$-статистику  $\gamma$, равную 1.28, и как следствие $H*=1.28^2=1.65$. Так, модель со случайными эффектами не отвергается на 5\% уровне значимости. Хотя оценки эластичности заработной платы отличаются на 0.049, оценки недостаточно точные, поэтому разница статистически незначима. Заметим, что если вместо этого использовать неробастную $t$-статистику для $\hat{\gamma}$, то $t^2=13.69$, что близко к предыдущей неверной статистике теста Хаусмана.

\subsection{Более сложные модели для случайных эффектов}

Модель со случайными эффектами предполагает, что случайный эффект $\alpha_i$ распределен независимо от регрессоров. Более сложные модели, которые ближе к моделям с фиксированными эффектам, ослабляют это предположение. 

Мундлак (1978) предполагает, что в модели панельных данных \ref{Eq:21.3} индивидуальные эффекты определены \textbf{средними по времени} значениями регрессоров: $\alpha_i=\bar{\x}'_i\pi+w_i$, где $w_i$ независимы и одинаково распределены. Тогда оценивание $\beta$ и $\pi$ методом ОМНК в этой расширенной модели дает оценку $\beta$, равную оценке с фиксированным эффектом в модели \ref{Eq:21.3}. Обычная оценка $\beta$ со случайным эффектом в модели \ref{Eq:21.3}, которая ошибочно предполагает независимые и одинаково распределенные случайные эффекты, будет несостоятельна.

Чемберлин (1982, 1984) рассматривал еще более сложную модель для случайных эффектов, в которой $\alpha_i$ --- это \textbf{взвешенная сумма} регрессоров $\alpha_i=\x'_{1i}\pi_1+ \dots + \x'_{Ti} \pi_T+w_i$. Он предложил производить оценку с помощью метода минимальных расстояний (см. раздел 22.2.7), который дает оценку  $\beta$, равную оценке с фиксированным эффектом.

Смешанные линейные модели  и иерархические линейные модели раздела 24.6 позволяют применять вполне общие модели для случайных свободных членов, а также для случайных параметров наклона. Байесовский анализ панельных данных также использует такую систему (см. раздел 22.8).

В линейных моделях подход оценивания с фиксированным эффектом используется, если ненаблюдаемый индивидуальный эффект коррелирован с регрессорами. В более сложных моделях, таких как нелинейные модели, модели с фиксированными эффектами не всегда могут быть оценены. В таком случае в качестве альтернативного подхода можно использовать модели с более сложными случайными эффектами.

\section{Модели сквозной регрессии}

\textbf{Модель сквозной регрессии} или \textbf{модель с постоянными коэффициентами}:
\begin{align}
y_{it}=\alpha+\x'_{it}\bm\beta + u_{it}.
\label{Eq:21.17}
\end{align}
В статистической литература модель называется \textbf{усредненной моделью}, так как в ней явном виде не присутствует зависимость $y_{it}$ от индивидуальных эффектов. Вместо этого, индивидуальные эффекты были неявным образом усреднены. Модель со случайным эффектом  --- это частный случай такой модели, в котором $u_{it}$ равнокоррелирована по $t$ для данного $i$ (см. раздел 21.2.1).

Основная сложность предположения об отсутствии фиксированных эффектов для статистических выводов состоит в том, что распределение МНК-оценок меняется в зависимости от распределения $u_{it}$. В коротких панелях, робастные для панельных данных стандартные ошибки могут быть получены с помощью \ref{Eq:21.13}.

Однако, здесь мы фокусируемся на ОМНК оценивании, используя множество различных спецификаций, делая предположение о равнокоррелированности для  ковариационной матрицы $u_{it}$  по времени и среди индивидуальных наблюдений, как это было предложено в литературе.

Хотя мы рассматриваем преимущественно оценивание сквозной ОМНК регрессии \ref{Eq:21.17}, модели без индивидуальных фиксированных эффектов, методы, описанные в данном разделе, могут быть применены в более общих случаях для оценивания сквозной модифицированной ОМНК модели \ref{Eq:21.12} раздела 21.2.3.

\subsection{МНК, доступная ОМНК и взвешенная МНК оценки модели сквозной регрессии}

Здесь будет удобно использовать матричные обозначения. Комбинируя наблюдения по времени для данного индивидуума, определим
\begin{align}
\mathbf y_{i}=\mathbf W_{i}\bm\delta+\mathbf u_{i},
\label{Eq:21.18}
\end{align}
где $\delta=[\alpha \; \beta']'$ --- вектор размерности  $(K+1)\times 1$, $\mathbf y_i$ и $\mathbf u_i$ --- вектора размерности $T \times 1$ c $t$-ми элементами $y_{it}$ и $u_{it}$ соответственно и $\mathbf W_i$ --- матрица размерности $T \times (K+1)$, $t$-ая строка которой равна $\mathbf w'_{it}=[1 \; \x_{it}]'$. Соединяя все индивидуальные наблюдения, получаем
\begin{align}
\mathbf y=\mathbf W\bm\delta+ \mathbf u,
\label{Eq:21.19}
\end{align}
где $\mathbf y$  и $\mathbf u$ --- вектора размерности $NT \times 1$, например, $\mathbf y=[\mathbf y'_1 \dots \mathbf y'_N]'$ и $\mathbf W$ --- матрица регрессоров размерности $NT \times (K + 1)$, первым столбцом которой является единичный вектор. Мы предполагаем, что $E[\mathbf u| \mathbf W]=0$, таким образом ошибки строго экзогенны. $\Omega=E[\mathbf u \mathbf u'|\mathbf W]$.

Есть несколько возможных МНК оценок для данной модели. Они обобщены в таблице \ref{Tab:21.5}.

Во-первых, \textbf{МНК оценка модели сквозной регрессии} состоятельна и асимптотически нормальна. Однако, мала вероятность, что $\Omega=\sigma^2 \mathbf I_{NT}$, поэтому МНК оценка состоятельна за исключением некоторых особых случаев, например, когда все регрессоры не меняются во времени.  Следует использовать не обычную МНК оценку дисперсии  $\sigma^2 (\mathbf W' \mathbf W)^{-1}$, а робастную для панельных данных оценку, как в \ref{Eq:21.13}. 

Во-вторых, \textbf{доступная ОМНК оценка сквозной модели регрессии} (PFGLS) состоятельна и эффективна, если $\Omega$ правильно специфицирована и $\hat{\Omega}$ является состоятельной оценкой для $\Omega$. Некоторые из огромного количества структур $u_{it}$, а следовательно и $\bm\Omega$, которые были предложены в литературе, посвященной анализу панельных данных, и реализованы в статистических пакетах, представлены в разделах 21.5.2 и 21.5.3 для коротких и длинных панелей соответственно.

В-третьих, \textbf{взвешенная МНК оценка модели сквозной регрессии} (PWLS) является хорошим инструментом защиты от неправильной спецификации $\Omega$. Она кладет в основу ковариационной матрицы ошибок $\bm\Omega$ \textbf{вспомогательную матрицу $\bm\Sigma$}, но затем дает статистические выводы, которые достоверны даже при $\bm\Sigma \neq \bm\Omega$. Обычный МНК --- это частный случай с $\bm\Sigma = \sigma^2 \mathbf I_{NT}$. Другие выборы $\bm\Sigma$ могут увеличить эффективность.

Оценивание ковариационной матрицы МНК оценки модели сквозной регрессии требует такую $\hat{\bm\Omega}$, что $(NT)^{-1} \mathbf W' \hat{\bm\Omega} \mathbf W$ будет состоятельной оценкой для $(NT)^{-1} \mathbf W' \bm\Omega \mathbf W$.

Для коротких панелей это возможно посредством прямого применения результатов разделов 21.2.3. Для оценивания ковариационной матрицы взвешенной МНК оценки модели сквозной регрессии необходима такая $\hat{\bm\Omega}$, что $(NT)^{-1} \mathbf W' \bm\Sigma^{-1} \hat{\bm\Omega}\bm\Sigma^{-1} \mathbf W$ является состоятельной оценкой для $(NT)^{-1} \mathbf W' \bm\Sigma^{-1} \bm\Omega \bm\Sigma^{-1} \mathbf W$. Робастная для панельных данных оценка для МНК, данная в \ref{Eq:21.13}, может быть приспособлена к взвешенному МНК для оценивания сквозной регрессии путем замены $\mathbf W' \bm\Sigma^{-1} \bm\Omega \bm\Sigma^{-1} \mathbf W$ (или эквивалентного $\sum_i \mathbf W_i' \hat{\bm\Sigma}^{-1}_i \mathrm E[\mathbf u_i \mathbf u_i' | \mathbf W_i] \hat{\bm\Sigma}^{-1}_i \mathbf W_i$ при условии независимости по $i$) на  $\sum_i \mathbf W_i' \bm\Sigma^{-1}_i \hat{\mathbf u}_i \hat{\mathbf u}_i'  \hat{\bm\Sigma}^{-1}_i \mathbf W_i$, где $\hat{\mathbf u}_i=\mathbf y_i --- \mathbf W_i \hat{\delta}$. В качестве альтернативы можно использовать панельный бутстрэп.

\begin{table}[ht]
\caption{{\it Заработная плата и количество часов: Автокорреляции объединенной МНК-регрессии}} 
\centering
\begin{tabular}{p{4.5cm} c p{4cm}}
\hline \hline
	Оценка & Формула ${}^a$ & Ковариационная матрица	${}^b$ \\
\hline
МНК оценка: $\hat{\delta}_{POLS}$ & $(\mathbf W' \mathbf W)^{-1} \mathbf W' \mathbf y$ & $(\mathbf W' \mathbf W)^{-1} \mathbf W' \hat{\bm\Omega} \mathbf W (\mathbf W' \mathbf W)^{-1}$ \\
доступная ОМНК оценка: $\hat{\delta}_{PFGLS}$& $(\mathbf W' \hat{\bm\Omega}^{-1} \mathbf W)^{-1} \mathbf W' \hat{\bm\Omega}^{-1} \mathbf y$ &  $(\mathbf W' \hat{\bm\Omega}^{-1} \mathbf W)^{-1}$\\
взвешеная МНК оценка:  $\hat{\delta}_{PWLS}$ & $(\mathbf W' \hat{\bm\Sigma}^{-1} \mathbf W)^{-1} \mathbf W' \hat{\bm\Sigma}^{-1} \mathbf y$ & $(\mathbf W' \hat{\bm\Sigma}^{-1} \mathbf W)^{-1} \mathbf W' \hat{\bm\Sigma}^{-1} \hat{\bm\Omega} \mathbf W \times (\mathbf W' \hat{\bm\Sigma}^{-1} \mathbf W)^{-1}$  \\
\hline \hline
\multicolumn{3}{p{15cm}}{${}^a$ Формулы для модели $\mathbf y =\mathbf W \sigma + \mathbf u$ определены в \ref{Eq:21.19} и матрица ошибок $\Omega$.} \\
\multicolumn{3}{p{15cm}}{${}^b$ Для вычисления $\hat{\Omega}$ для коварационных матриц POLS и PWLS см. текст; в этих случаях $\hat{\Omega}$ не обязательно должна быть состоятельна для $\Omega$. Для 
доступной ОММ оценки модели сквозной регрессии предполагается, что $\hat{\Omega}$ состоятельна для $\Omega$.}
\end{tabular}
\label{Tab:21.5}
\end{table}

\subsection{Ковариационная матрица ошибок для коротких панелей}

В коротких панелях мы обладаем всего лишь несколькими временными периодами и большим количеством индивидуальных наблюдений (обычно это индивидуумы или фирмы). Предполагается, что ошибки независимы по индивидуальным наблюдениям, т.е. $Cov[u_{it}, u_{js}]=0, i \neq j$. В таких случаях удобно перейти к обозначениям суммирования. Например, PFGLS оценку, представленную в таблице \ref{Tab:21.5}, можно записать в виде
\begin{align}
\hat{\bm\beta}_{PFGLS}=\left[\sum^N_{i=1} \mathbf W'_i \hat{\bm\Omega}^{-1}_i \mathbf W_i \right]^{-1} \sum^N_{i=1} \mathbf W'_i \hat{\bm\Omega}^{-1}_i \mathbf y_i,
\label{Eq:21.20}
\end{align}
где $\hat{\bm\Omega}_i$ является состоятельной оценкой для 
\begin{align}
\bm\Omega_i =\mathrm E[\mathbf u_i \mathbf u'_i | \mathbf W_i],
\label{Eq:21.21}
\end{align}
и $\bm\Omega_i$ является недиагональной матрицей, так как ошибки для данного индивидуального наблюдения, скорее всего, будут коррелированы по времени. Заметим, что $\hat{\bm\Omega}_i$ должна быть получена из оценивания модели, специфицированной специально для $\bm\Omega_i$. Мы не можем использовать $\hat{\bm\Omega}_i=\hat{\mathbf u}_i\hat{\mathbf u}'_i$ (см. относящееся к этому обсуждение после уравнения 5.88. %\ref{Eq:5.88}).

{\centering
Равнокоррелированные ошибки\\}

Наиболее часто используемая структура ошибок описана в рамках модели со случайными эффектами в разделе 21.2.1. Из (21.6) видно, что $\bm\Omega_i$ имеет одинаковые диагональные элементы $\sigma^2_{\alpha}+\sigma^2_{\e}$ и одинаковые недиагональные элементы $\sigma^2_{\alpha}$. Другими словами, ошибки \textbf{равнокоррелированы}, когда $\bm\Omega_i$ имеет одинаковые диагональные элементы $\sigma^2$ и одинаковые недиагональные элементы $\rho\sigma^2$. При использовании доступной ОМНК оценки требуется оценить только $\sigma^2_{\alpha}$ и $\sigma^2_{\e}$ или $\sigma^2$ и $\rho$ (см. разделы 21.2.2 и 21.7).

{\centering
Ошибки, имеющие ARMA структуру\\}

Альтернативно можно предположить, что ошибки имеют структуру ARMA модели. Например, модель ошибок AR(1) предполагает, что $u_{it}=\rho u_{i,t-1}+\e_{it}$, где $e_{it}$ независимы и одинаково распределены. Тогда  $Cov[u_{it}, u_{is}]=\rho^{|t-s|}\sigma^2$. В этом случае ковариация ошибок падает по мере увеличения временных промежутков между ними. Модель со случайными эффектами и модель AR(1) сравниваются в разделе 21.5.4.

Бальтаджи и Ли (1991) комбинируют эти две модели и рассматривают модель со случайным эффектом с ошибками вида AR(1). Эту модель можно без труда обобщить до случая AR(p), а методы, использующие ошибки вида MA и ARMA (см. раздел 5.8.7) в моделях со случайными эффектами, были разработаны относительно недавно. Краткий обзор этих моделей представлен у Бальтаджи (2001, глава 5).  

{\centering
Гомоскедастичные ошибки и неструктурированная автокорреляция \\}

При использовании доступной ОМНК оценки в коротких панелях в действительности нет необходимости налагать на ошибки такие структуры, как в модели со случайным эффектом или AR(1) модели, если сделано предположение о неизменности матрицы $\bm\Omega_i$ размерности $T \times T$ для индивидуальных наблюдений. В таком случае остается оценить только $T(T+1)/2$ ковариационных параметров. Тогда состоятельной оценкой $\bm\Omega_i$ будет $\hat{\bm\Omega}_i$ с $(t,s)$-м элементом $\hat{\sigma}_{ts}=N^{-1} \sum^N_{i=1} \hat{u}_{it} \hat{u}_{is}$. Предшествующие модели также предполагают гомоскедастичность, но накладывают дополнительный структурированный вид на $\bm\Omega_i$.

{\centering
Робастные статистические выводы\\}

Как и в предыдущих спецификациях предположим, что ковариации ошибок одинаковы для индивидуальных наблюдений, что решает проблему гетероскедастичности. В случае с короткой панелью можно тем не менее использовать предшествующие модели матрицы ошибок как основу для взвешенной МНК оценки сквозной регрессии, но тогда следует использовать робастные стандартные ошибки, как это обсуждалось после таблицы \ref{Tab:21.5}. Вместо этого также можно использовать  более сложные  смешанные модели, представленные в главе 22.

В главах 21-23 выполняется предположение о независимости по $i$. На самом деле оно может быть ослаблено даже для малых $T$, если на корреляцию наложить определенную структуру. Хорошим примером может послужить модель для пространственной корреляции для панельных данных по регионам, таким как штаты или страны. Корреляции снижаются по мере увеличения физического расстояния между индивидуальными наблюдениями. 

\subsection{Ковариационная матрица ошибок для длинных панелей}

В {\bf длинных панелях} мы располагаем множеством временных периодов, но относительно небольшим количеством индивидуальных наблюдений. Такие данные имеют место быть в микроэконометрическом анализе, если индивидуальными наблюдениями являются небольшое количество регионов, таких как штаты, страны, или фирмы. Однако эти объекты наблюдаются в течение достаточного количества периодов, чтобы статистические выводы основывались на предположении $T \rightarrow \infty$. 

Корреляция во времени для данного индивидуального наблюдения может быть описана с помощью ARMA  модели ошибок, в которых параметры ARMA модели меняются только по $i$, так как $N$ сейчас фиксировано, а  $T \rightarrow \infty$. Например, рассмотрим ошибку вида AR(1): $u_{it}=\rho u_{i,t-1}+\e_{it}$, где ошибка $\e_{it} \thicksim [0,\sigma^2_i]$ гетероскедастична и $\rho_i$ также различается по индивидуальным наблюдениям. Отдельные регрессии $y_{it}$ на $\mathbf w_{it}$
для каждого индивидуального наблюдения с ошибками вида AR(1) с использованием $T$ временных периодов дают состоятельные оценки $\hat \rho_i$ и $\hat \sigma^2_i$, так как  $T \rightarrow \infty$. Эти оценки могут быть использованы для получения доступной ОМНК оценки $\delta$ при использовании уже всех $NT$ наблюдений. Более подробно информацию можно найти у Кмента (1986). Эта модель позволяет учитывать как гетероскедастичность по индивидуальным наблюдениям, так и корреляцию во времени для данного индивидуального наблюдения. Песаран (2004) предлагает более расширенную модель, оцениваемую при помощи ОМНК.

При оценивании длинных моделей можно представить корреляцию индивидуальных наблюдений, т.е. $Cov[u_{it}, u_{jt}] \neq 0$ для $i \neq j$, так как $N$ фиксировано и асимптотически результаты основываются на $T \rightarrow \infty$. В частности, можно применить ОМНК оценку сквозной регрессии, как это предлагалось ранее, предполагая независимость по $i$, а затем посчитать стандартные ошибки с использованием упомянутого в разделе 6.4.4 метода Ньюи и Веста (1987b), который позволяет учитывать случайную зависимость между индивидуальными наблюдениями  и во времени, если зависимость во времени нивелируется достаточно быстро. Более подробно см. у Ареллано (2003, p.19). 

Аспекты анализа временных рядов для панельных данных обсуждаются более подробно в разделе 22.5 для моделей с использованием лагов зависимых переменных в качестве регрессоров.

\subsection{Влияние автокоррелированных ошибок}

Ошибки регрессионных моделей панельных данных обычно автокоррелированы во времени для данного индивидуального наблюдения. Если фиксированные эффекты отсутствуют, тогда модель сквозной регрессии в этом случае дает состоятельные оценки параметров. Однако \textbf{корреляция ошибок} может привести к \textbf{большому смещению} в стандартных ошибках сквозной регрессии, если автокорреляция не учитывается, и относительно меньшему увеличению эффективности в связи с увеличением длины панели.

Анализ достаточно прост для оценивания среднего $y$ на основе $T$ наблюдений для одного индивидуального наблюдения ($N=1$) c равнокоррелированными ошибками. Тогда $y_t=\beta + u_t$, и МНК оценка --- это выборочное среднее: $\hat{\beta}=\bar{y}=T^{-1} \sum_t y_t$. Истинная дисперсия МНК оценки равна $V[\hat{\beta}]=V[\bar{y}]=T^{-2}\sum_t \sum_s Cov[u_{t},u_{s}]$. В предположении о равнокоррелированности двойная сумма состоит из $T$ дисперсий, равных $\sigma^2$, и $T(T-1)$ ковариаций, равных $\rho \sigma^2$. Тогда $V[\bar{y}]=T^{-1}\sigma^2(1+\rho (T-1))$. Получается, что $V[\bar{y}]=T^{-1}\sigma^2$ должно быть домножено на $(1+\rho (T-1))$. В частности, $V[\bar{y}]$ стремится к $\sigma^2$ по мере того, как $\rho \rightarrow \infty$.

\begin{table}[ht]
\caption{{\it Дисперсии оценок сквозной регрессии с равнокоррелированными ошибками$^a$}} 
\centering
\begin{tabular}{ccccccc}
\hline \hline
	\textbf{T} & $\rho=\mathbf{0.0}$& $\rho=\mathbf{0.2}$	& $\rho=\mathbf{0.4}$	& $\rho=\mathbf{0.6}$	& $\rho=\mathbf{0.8}$	& $\rho=\mathbf{1.0}$ \\
\hline
1	&1.00	&1.00	&1.00	&1.00	&1.00	&1.00\\
2	&0.50	&0.60	&0.70	&0.80	&0.90	&1.00\\
5	&0.20	&0.36	&0.52	&0.68	&0.84	&1.00\\
10	&0.10	&0.28	&0.46	&0.64	&0.82	&1.00\\
\hline \hline
\multicolumn{7}{p{14cm}}{$^a$Представлены диспресии МНК оценки сквозной регрессии при растущей корреляции $\rho$ равнокоррелированных ошибок для модели, включающей только свободный член, с дисперсией ошибок, пронормированной к единице. Предполагается, что ошибки коррелированы, хотя гомоскедастичны.} \\
\end{tabular}
\label{Tab:21.6}
\end{table}

Таблица \ref{Tab:21.6} показывает влияние корреляции на дисперсию $\bar{y}$ для разных значений $T$ и $\rho$, где для простоты мы положим, что $\sigma^2=1$. Точность оценивания значительно падает с увеличением $\rho$, и оценка $V[\bar{y}]$ в предположении независимости (предполагая для простоты, что $\sigma^2$ известна) может сильно недооценить истинную дисперсию. Более того, для $\rho > 0$ увеличение точности в связи с увеличением количества временных периодов намного меньше, чем в случае с независимыми данными, когда увеличение временного интервала вдвое уменьшит дисперсию оценки в два раза. Например, если $\rho=0.4$, тогда в случае использования пяти временных периодов дисперсия оценки только в два раза меньше ($1/0.52$), чем в случае оценивания одномоментной выборки. В случае же с независимыми данными уменьшение дисперсии оценки меньше в целых пять раз ($1/0.2$). Кроме того, удвоение количества периодов с 5 до 10 приводит лишь к небольшому снижению дисперсии оценки с 0.52 до 0.46.

Этот результат применим в более общем виде для сбалансированных панелей с равнокоррелированными ошибками и регрессорами, не изменяющимися во времени, где истинная дисперсия МНК оценки равна $(1+\rho(T-1))$, умноженное на дисперсию, в предположении о независимости ошибок (см. Kloek, 1981). На практике регрессоры, меняющиеся во времени, тоже включены в регрессию, что делает более трудным получение каких-либо точных аналитических результатов. Скотт и Холт (1982) показали, что для регрессий со свободным членом и одним регрессором, меняющимся во времени, дисперсия коэффициента наклона увеличивается на множитель $(1+\hat{\rho}_x\rho(T-1))$, где $\hat{\rho}_x$ может рассматриваться как оценка индивидуальной автокорреляции в $x$. В случае с панельными данными  $\hat{\rho}_x$ обычно высок, так что увеличение достаточно значительное. Эти результаты также применимы к другим формам кластеризованных  данных и более подробно представлены в разделе 24.5.2.

Вышеописанный анализ предполагает равнокоррелированные ошибки. Это свойство модели со случайным эффектом. Если вместо этого ошибки описываются процессом AR(1), то выгода от увеличения длины панели будет значительно больше. В этом случае $\mathrm{Cov}[u_{t}, u_{s}]=\rho^{|t-s|}\sigma^2$, а $V[\bar{y}]=T^{-2}\sigma^2[T+2\sum^{t-1}_s=1(T-s)\rho^s]$. Например, если $\rho=0.8$, тогда $V[\bar{y}]=0.72\sigma^2$ для $T=5$ и $0.54\sigma^2$ для $T=10$, что ниже, чем соответствующие значения из таблицы \ref{Tab:21.6} $0.84\sigma^2$ и $0.82\sigma^2$ для равнокоррелированных ошибок с $\rho=0.8$, но тем не менее выше значений $0.2\sigma^2$ и $0.1\sigma^2$ для $\rho=0.0$.

Микроэконометристы для коротких панелей чаще всего используют модель со случайным эффектом или модели с равнокоррелированными ошибками Например, рассмотрим данные для большого количества семей о разных близнецах в семье. Разумно предполагать, что корреляции ненаблюдаемых характеристик близнецов в одной и той же семье одни и те же для разных семей. Например, корреляция между первыми и вторыми близнецами равна корреляции между первыми и третьими близнецами. Те, кто используют длинные панели, зачастую обладают исторической информацией и предполагают, что корреляция снижается со временем. Поэтому для ошибок они используют модель AR(1). 

Какая модель корреляции ошибок во времени будет более оправданной, зависит от данных. Во множестве коротких панелей, используемых в микроэкономических приложениях, при МНК оценивании сквозной регрессии присутствуют автокорреляции ошибок. Эти автокорреляции качественно близки к тем, что даны в таблице \ref{Tab:21.3}. Они более точно описываются моделью со случайным эффектом, чем AR(1), хотя может также хорошо подходить и ARMA(1,1). Однако модель со случайным эффектом с ошибками вида AR(1) может быть лучше. Во всех случаях корреляция ошибок приводит к потере информации и к тому, что обычные стандартные ошибки МНК недооценивают истинные стандартные ошибки. Для коротких панелей статистические выводы можно основывать на робастных для панельных данных стандартных ошибках (см. раздел 21.2.3), которые не требуют спецификации модели для корреляции ошибок.

\subsection{Количество часов работы и заработная плата. Пример сквозной ОМНК регрессии}

ОМНК оценки сквозной регрессии и соответствующие им робастные стандартные ошибки, а также стандартные ошибки, вычисляемые по умолчанию, модели $y_{it}=\alpha_i+\beta x_{it} + u_{it}$ для регрессии lnhrs на lnwg представлены в таблице 21.7. Предполагается, что ошибки $u_{it}$ независимы и одинаково распределены по $i$. Предположения относительно корреляции $u_{it}$ по $t$ различны.

Первая колонка таблицы \ref{Tab:21.7}, предназначенная для МНК оценки сквозной регрессии, повторяет первую колонку таблицы \ref{Tab:21.2}.

ОМНК оценки сквозной регрессии, предполагающие равнокоррелированные ошибки, записаны во второй колонке таблицы \ref{Tab:21.7}. Они совпадают с колонкой RE-GLS в таблице \ref{Tab:21.2}, так как модель со случайным эффектом предполагает равнокоррелированные ошибки (см. (21.6)).

ОМНК оценки сквозной регрессии, предполагающие ошибки вида AR(1): $u_{it}=\rho u_{it-1}+\e_{it}$, где $\e_{it}$ независимы и одинаково распределены, представлены в третей колонке таблицы \ref{Tab:21.7}. Оценка коэффициента наклона близка к МНК оценке сквозной регрессии.

ОМНК оценки сквозной регрессии, не накладывающие никакой структуры на корреляцию ошибок, кроме гомоскедастичности, т.е. $Cov[u_{it}, u_{is}]=\sigma_{ts}$, даны в четвертой колонке таблицы \ref{Tab:21.7}. Тогда оценка $\sigma_{ts}$ будет состоятельна при малом $T$ для всех $t$ и $s$: $\hat{\sigma}_{ts}=N^{-1}\sum^N_{i=1}\hat{u}_{it}\hat{u}_{is}$. Эти оценки также близки к МНК оценкам сквозной регрессии.

Из таблицы \ref{Tab:21.7} видно, что робастные для панельных данных стандатные ошибки следует предпочесть стандартным ошибкам, вычисленным по умолчанию, и предполагающим гомоскедастичность и правильно специфицированную модель для корреляции во времени.


 \begin{table}[ht]
\caption{{\it Количество часов работы и заработная плата: МНК и ОМНК оценки сквозной регрессии$^a$}} 
\centering
\begin{tabular}{cc p{2cm} cc}
\hline \hline
	Оценка & POLS & \multicolumn{3}{c}{PFGLS}\\
 Корреляция ошибок & --- & Equi & AR(1) & General\\
\hline
$\alpha$ & 7.442 & 7.346 & 7.440 & 7.426\\
$\beta$	& .083	& .120 	&.084 & .091 \\
Робастные станд. ош.		& (.029) &(.052) & (.037) & (.050) \\
Бутстрэп станд. ош.			& [.032] & [.060] & [.050] &[-] \\
Станд. ош. по умолчанию	& \{.009\} & \{.014\}& \{.012\}  & \{.014\} \\
\hline \hline
\multicolumn{5}{p{14cm}}{$^a$МНК и ОМНК модели линейной сквозной регреccии lnhrs на lnwg для короткой панели. Предполагается независимость и одинаковая распределенность по $i$ и отсутствие фиксированных эффектов.  ОМНК оценка сквозной регрессии предполагает равнокоррелированность  ошибки или ошибки со случайными эффектами, ошибки вида AR(1), или отсутствие какой-либо структуры корреляций. Стандартные ошибки для коэффициента наклона робастные для панельных данных выписаны в скобках, полученные с помощью бутстрэп --- в квадратных скобках, и оценки, вычисляемые по умолчанию, предполагающие, что ошибки независимы и одинаково распределены --- в фигурных скобках.} \\
\end{tabular}
\label{Tab:21.7}
\end{table}

\section{Модели с фиксированными эффектами}
 

\textbf{Модель с фиксированными эффектами} предполагает 
\begin{align}
y_{it}=\alpha_i +\x'_{it}\bm\beta +\e_{it},
\label{Eq:21.22}
\end{align}
где индивидуальные эффекты $\alpha_1, \dots , \alpha_N$ измеряют ненаблюдаемую гетерогенность, которая, вероятно, коррелирует с регрессорами, $\x_{it}$ и $\bm\beta$ --- вектора размерности $K \times 1$. Для начала ошибки $\e_{it}$ предполагаются независимыми и одинаково распределенными с параметрами $[0, \sigma^2]$.

Трудность оценивания составляет наличие $N$ индивидуальных эффектов, которые увеличиваются в количестве по мере того, как $N \rightarrow \infty$. С практической точки зрения мы больше заинтересованы в $K$ коэффициентах наклона $\bm\beta$, которые отображают  предельный эффект изменения регрессоров $\delta \mathrm E[y_{it}]/ \partial \x_{it}=\bm\beta$. Параметры $\alpha_1, \dots, \alpha_N$ являются \textbf{вспомогательными параметрами} или \textbf{второстепенными параметрами}, они не представляют существенного интереса. Тем не менее, их присутствие мешают оцениванию параметров $\bm\beta$, которые являются предметом изучения.

Есть несколько способов получить состоятельные оценки $\bm\beta$ для линейной модели, несмотря на присутствие вспомогательных параметров. Это (1) МНК в модели within \ref{Eq:21.8}; (2) прямое МНК оценивание модели \ref{Eq:21.2} с фиктивными переменными для каждого из $N$ фиксированных эффектов; (3) ОМНК модели within \ref{Eq:21.8}; (4) оценка ММП при условии индивидуальных средних $\bar{y}_i$, $i=1, \dots, N$; и (5) МНК оценка модели в первых разностях \ref{Eq:21.9}.

Первые два метода всегда приводят к одинаковой оценке $\bm\beta$. К этой же оценке приводят третий метод, если дополнительно предположить, что $\e_{it}$ в модели \ref{Eq:21.22} независимы и одинаково распределены, и четвертый метод, если $\e_{it} \thicksim \mathcal N[0, \sigma^2]$. Последний метод отличается от остальных для случая  $T>2$. Для нелинейных моделей, рассматриваемых в главе 23, такие эквивалентные соотношения не характерны. 

Важный результат для оценки within представлен в следующем разделе. Оценка в первых разностях, описанная в разделе 21.6.2, используется и в главе 22, когда регрессоры уже не являются строго экзогенными. Другие оценки представлены в приложении раздела 21.6, которое некоторые читатели могут предпочесть пропустить.

\subsection{Оценка within или оценка с фиксированным эффектом}

Модель within получается посредством вычитания усредненной по времени модели $\bar{y}_i=\alpha_i+\bar{\x}'_i\bm\beta +\bar{\e}_i$ из первоначальной модели:
\begin{align}
y_{it}-\bar{y}_i=(\x'_{it}-\bar{\x}_i)'\bm\beta +(\e_{it}-\bar{\e}_i).
\label{Eq:21.23}
\end{align}
Таким образом, фиксированный эффект $\alpha_i$ уничтожается наряду с регрессорами, не меняющимися во времени, так как $\x_{it}-\bar{\x}'_i=\mathbf 0$, если $\x_{it}=\x_i$ для всех $t$.

Использование МНК дает \textbf{оценку within} или \textbf{оценку с фиксированным эффектом} $\hat{\bm\beta}_W$, где 
\begin{align}
\hat{\bm\beta}_W=\left[ \sum^N_{i=1} \sum^T_{t=1} (\x_{it}-\bar{\x}_i)(\x_{it}-\bar{\x}_i)'\right]^{-1} \sum^N_{i=1} \sum^T_{t=1} (\x_{it}-\bar{\x}_i)(y_{it}-\bar{y}_i).
\label{Eq:21.24}
\end{align}

 Индивидуальные эффекты $\alpha_i$ могут быть оценены следующим образом:
\begin{align}
&\hat{\alpha}_i=\bar{y}_i-\x'_{it}\hat{\bm\beta}_W, & i=1, \dots, N.
\label{Eq:21.25}
\end{align}

Оценка $\hat{\alpha}_i$ является несмещенной оценкой $\alpha_i$. Она также является состоятельной при $T \rightarrow \infty$, так как $\hat{\alpha}_i$ усредняет $T$  наблюдений. В коротких панелях оценки $\hat{\alpha}_i$ несостоятельны, но $\hat{\bm\beta}_W$ остается состоятельной оценкой $\bm\beta$. Параметры  $\alpha_i$ являются \textbf{вспомогательными}. Их, к счастью, не требуется оценивать состоятельно в целях получения состоятельных оценок более существенных параметров $\beta$. Этот удивительный результат не обязательно верен  в более сложных моделях с фиксированными эффектами, например, в нелинейных моделях.

{\centering
Состоятельность оценки within\\}

Оценка within параметра $\bm\beta$ состоятельна, если  $\mathrm{plim}(NT)^{-1}\sum_i\sum_t(\x_{it}-\bar{\x_i})(\e_{it}-\bar{\e}_i)=\mathbf 0$. Это выполняется, если $N \rightarrow \infty$ или $T \rightarrow \infty$ и 
\begin{align}
\mathrm E[\e_{it}-\bar{\e}_i|\x_{it}-\bar{\x}_i]=0.
\label{Eq:21.26}
\end{align}
 
В связи с присутствием средних $\bar{\x}_i=T^{-1}\sum_i\x_{it}$ и $\bar{\e}_i$ это условие сильнее, чем $\mathrm E[\e_{it}|\x_{it}]=0$. Существенное условие для \ref{Eq:21.26} --- строгая экзогенность: $\mathrm E[\e_{it}|\x_{i1}, \dots, \x_{iT}]=0$. Это условие мешает использовать оценки within при включении лаговых эндогенных переменных как регрессоров (см. раздел 22.5).

{\centering
Асимптотическое распределение и оценка within\\}

Распределение $\hat{\bm\beta}_W$ может быть достаточно сложным, так как ошибка $(\e_{it}-\bar{\e}_i)$ в модели within \ref{Eq:21.8} коррелирована по $t$ для данного $i$. Ниже показывается, что обычный МНК тем не менее применим. При строгом предположении, что $\e_{it}$ независимы и одинаково распределены,
\begin{align}
\mathrm V[\hat{\bm\beta}_W]=\sigma^2_{\e} \left[ \sum^N_{i=1} \sum^T_{t=1} \ddot{\x}_{it}\ddot{\x}'_{it} \right]^{-1},
\label{Eq:21.27}
\end{align}
где $\ddot{\x}_{it}=\x_{it}-\bar{\x}_i$. Состоятельной и несмещенной оценкой $\sigma^2_{\e}$ будет $\hat{\sigma}^2_{\e}=[N(T-1)-K]^{-1}\sum_i\sum_t \hat{\e}^2_{it}$, где количество степеней свободы равно размеру выборки $NT$ за вычетом числа параметров модели $K$ и количества $N$ индивидуальных эффектов. Заметим, что если регрессия \ref{Eq:21.23} оценивается с помощью стандартного подхода МНК, то необходимо увеличить вычисленную дисперсию на $[N(T-1)-K]^{-1}[NT-K]$.

Для коротких панелей формула \ref{Eq:21.13} дает робастную оценку асимптотической дисперсии
\begin{align}
\mathrm V[\hat{\bm\beta}_W]= \left[ \sum^N_{i=1} \sum^T_{t=1} \ddot{\x}_{it}\ddot{\x}'_{it} \right]^{-1}
\sum^N_{i=1} \sum^T_{t=1} \sum^T_{s=1} \ddot{\x}_{it}\ddot{\x}'_{is} \hat{\ddot{\e}}_{it} \hat{\ddot{\e}}_{is}
\left[ \sum^N_{i=1} \sum^T_{t=1} \ddot{\x}_{it}\ddot{\x}'_{it} \right]^{-1},
\label{Eq:21.28}
\end{align}
где $\ddot{\e}_{it}=\e_{it}-\bar{\e}_i$. Эта оценка допускает любую автокорреляцию $\e_{it}$ и гетероскедастичность, поэтому более предпочтительна. 


{\centering
Вывод дисперсии оценки within\\}

Здесь мы  выведем оценку дисперсии \ref{Eq:21.27} и \ref{Eq:21.28} оценки within при помощи матричной алгебры. Начнем с модели для $i$-го наблюдения
\begin{align}
y_{it}=\alpha_i+\x'_{it}\bm\beta+\e_{it},
\nonumber
\end{align}
где $\x_{it}$ и $\bm\beta$ --- вектора размерности $K \times 1$. Для $i$-го индивидуального наблюдения, запишем все $T$ наблюдений:
\begin{align}
&\begin{bmatrix}
 y_{i1} \\
 \vdots \\
 y_{iT}
\end{bmatrix}
=
\begin{bmatrix}
1 \\ \vdots  \\ 1
\end{bmatrix}
\alpha_i + 
\begin{bmatrix}
\x'_{i1} \\ \vdots \\  \x'_{iT} 
\end{bmatrix}
 \bm\beta + 
\begin{bmatrix}
 \e_{i1} \\ \vdots \\ \e_{iT} 
\end{bmatrix}
, & i=1, \dots N,
\nonumber
\end{align}
или
\begin{align}
& \mathbf y_{it}=\mathbf e \alpha_i+\mathbf X'_{it}\bm\beta+\bm\e_{it},
& i=1, \dots, N,
\label{Eq:21.29}
\end{align}
где $\mathbf e=(1,1, \dots, 1)'$  --- единичный  вектор размерности $T \times 1$, $\mathbf X_i$  --- матрица размерности $T \times K$, и $\mathbf y_i$ и $\bm\e_i$ --- вектора размерности $T \times 1$.

Для преобразования модели \ref{Eq:21.29} в модель within, которая получается посредством вычитания среднего по индивидуальным наблюдениям, введем матрицу размерности $T \times T$
\begin{align}
\mathbf Q=\mathbf I_T --- T^{-1} \mathbf e \mathbf e'.
\label{Eq:21.30}
\end{align}
Умножение на матрицу $\mathbf Q$ дает отклонения от средних, так как
\begin{align}
\mathbf Q \mathbf W_i= \mathbf  W_i --- \mathbf e \bar{\mathbf w}'_i,
\label{Eq:21.31}
\end{align}
где $\mathbf W_i$ --- матрица размерности $T \times m$ с $t$-ой строкой $\mathbf w'_{it}$ и $\bar{\mathbf w}'_i=T^{-1} \sum^T_{t=1} \mathbf w_{it}$ --- вектор средних размерности $m \times 1$. Для получения результата \ref{Eq:21.31} используется равенство $\mathbf e' \mathbf W_i=T\bar{\mathbf w}'_i$. Заметим также, что $\mathbf Q\mathbf Q'=\mathbf Q$, кроме того $\mathbf e \mathbf e'=T$ и $\mathbf Q \mathbf e =\mathbf 0$,  поэтому $\mathbf Q$ идемпотентна.

Умножая на $\mathbf Q$ модель с фиксированными эффектами \ref{Eq:21.29} для $i$-го индивидуального наблюдения, используя $\mathbf Q \mathbf e= \mathbf 0$, получаем
\begin{align}
&\mathbf Q \mathbf y_i= \mathbf Q\mathbf  X_i \bm\beta + \mathbf Q \bm\e_i,
&i=1, \dots, N,
\label{Eq:21.32}
\end{align}
Это модель within \ref{Eq:21.23}, так как это эквивалентно выражению $\mathbf y_i --- \mathbf e \bar{y}'_i = (\mathbf X_i --- \mathbf e \bar{\mathbf x}'_i ) \bm\beta + (\e_i- \mathbf e \bar{\e}_i)$, полученному с помощью \ref{Eq:21.31}. Умножение на $\mathbf Q$ дает модель within. МНК оценивание \ref{Eq:21.32} дает $\hat{\bm\beta}_W$ с ковариационной матрицей, предполагающей независимость по $i$, равной
\begin{align}
V[\hat{\bm\beta}]=\left[ \sum_{i=1}^N \mathbf X'_i \mathbf Q' \mathbf Q \mathbf X_i \right]^{-1} 
\sum_{i=1}^N \mathbf X'_i \mathbf Q' \mathrm V[\mathbf Q \bm\e_i | \mathbf X_i] \mathbf Q \mathbf X_i 
\left[ \sum_{i=1}^N \mathbf X'_i \mathbf Q' \mathbf Q \mathbf X_i \right]^{-1}.
\label{Eq:21.33}
\end{align}
Начнем с сильного предположения, что $\e_{it}$ независимы и одинаково распределены с параметрами $[0, \sigma^2_{\e}]$, и $\bm\e_i$ независимы и одинаково распределены с параметрами $[\mathbf 0, \sigma^2_{\e} \mathbf I]$. Тогда вектор ошибок $\mathbf Q \mathbf \e_i$ размерности $T \times 1$ независим по $i$ с нулевым ожиданием и дисперсией $\mathrm V[\mathbf Q \bm\e_i]=\mathbf Q \mathrm V[\bm\e_i]\mathbf Q'=\sigma^2_\e \mathbf Q \mathbf Q' = \sigma^2_\e \mathbf Q$. Тогда
\begin{align}
\sum_{i=1}^N \mathbf X'_i \mathbf Q' \mathrm V[\mathbf Q \e_i | \mathbf X_i] \mathbf Q \mathbf X_i=
&\sum_{i=1}^N \mathbf X'_i \mathbf Q' \mathrm \sigma^2_\e \mathbf Q \mathbf Q \mathbf X_i  \nonumber \\
&=\sigma^2_\e \sum_{i=1}^N \mathbf X'_i \mathbf Q'  \mathbf Q \mathbf X_i.
\nonumber
\end{align}
Таким образом \ref{Eq:21.33} можем упростить до оценки, данной в \ref{Eq:21.27}, используя
\begin{align}
(\mathbf Q \mathbf X_i)'(\mathbf Q \mathbf X_i)=\sum^T_{t=1}(\x_{it}-\bar{\x}_i)(\x_{it}-\bar{\x}_i)'.
\nonumber
\end{align}
В настоящее время большинство пакетов вычисляют дисперсию по формуле \ref{Eq:21.27}, однако следует помнить, что альтернативные оценки могут быть лучше. В частности, достаточно просто можно ослабить предположение о некоррелированности ошибок $\e_{it}$ во времени. Если $\bm\e_i$ независимы и одинаково распределены с параметрами $[\mathbf 0, \sum_i]$, то используется более общая формула ковариационной матрицы \ref{Eq:21.33} с $\mathrm{Cov}[\mathbf Q \bm\e_i, \mathbf Q\bm \e_i]= \mathbf 0$ для $i \neq j$, и заменой $\mathrm V[\mathbf Q \mathbf \e_i]$ на $(\mathbf Q  \hat{\bm\e}_i) (\mathbf Q  \hat{\bm\e}_i)'$, где $\hat{\bm\e}_i=\mathbf y_i --- \mathbf X_i \hat{\bm\beta}_W$, что дает нам дисперсию оценки вида \ref{Eq:21.28}.

Оценка $\hat{\bm\beta}_W$ также является состоятельной в модели со случайными эффектами, хотя, как показано в разделе 21.7, она менее эффективна, чем оценка со случайным эффектом в случае, если модель со случайными эффектами более адекватно описывает данные.

{\centering
ОМНК оценивание модели within\\}

Модель within \ref{Eq:21.32} может быть оценена с помощью доступного ОМНК.

Но если в действительности $\e_{it}$  независимы и одинаково распределены с параметрами $[0, \sigma^2_\e]$, то применение ОМНК не дает преимуществ. Продемонстрируем это. Заметим, что $\mathbf Q \bm\e_i$ не зависит от $\mathbf Q \bm\e_j$ при $i \neq j$, $\mathrm V[\mathbf Q \bm\e_i]=\sigma^2_\e \mathbf Q$.     \textbf{ОМНК оценка} будет меть вид
\begin{align}
\hat{\bm\beta}_{W,GLS}=\left[ \sum_{i=1}^N \mathbf X'_i \mathbf Q' \mathbf Q^{-} \mathbf Q \mathbf X_i \right]^{-1} 
\sum_{i=1}^N \mathbf X'_i \mathbf Q' \mathbf Q^{-} \mathbf Q \mathbf y_i,
\nonumber
\end{align}
где используется квазиобратная матрица $\mathbf Q^-$ из-за того, что матрица  $\mathbf Q$ имеет неполный ранг. Однако $\mathbf Q' \mathbf Q^{-} \mathbf Q =\mathbf Q' \mathbf Q$, так как $\mathbf Q' \mathbf Q^{-} \mathbf Q = \mathbf Q$ для квазиобратной матрицы, и $\mathbf Q= \mathbf Q \mathbf Q'$, так как $\mathbf Q$ идемпотентна. Заменяя  $\mathbf Q' \mathbf Q^{-} \mathbf Q $ на $\mathbf Q' \mathbf Q$ в формуле для $\hat{\bm\beta}_{W,GLS}$, получаем МНК оценку \ref{Eq:21.32}.

Преимущества от использования ОМНК могут иметь место, если для $\e_{it}$ предполагаются другие модели. По существу это тот же подход, что и в разделе 21.5.2 для ОМНК сквозной регрессии без фиксированный эффектов, за исключением того, что предварительно фиксированные эффекты должны быть элиминированы. Это приводит к ошибке $\mathbf Q\bm\e_i$, имеющей неполный ранг. Поэтому вначале необходимо выбросить один временной период и применять ОМНК лишь к $(T-1)$ временным периодам. Проще, и зачастую без большой потери эффективности, использовать обычную оценку within c фиксированным эффектом, а затем вычислить робастные для панельных данных стандратные ошибки, используя \ref{Eq:21.28}.

МакКарди (1928b) предлагает анализ типа Бокса-Дженкинса для идентификации и оценки процессов ARMA для $\e_{it}$ в модели с фиксированными эффектами в случае короткой панели. При использовании коротких панелей нет необходимости предполагать, что $\e_{it}$ описывается ARMA процессом, или даже, что $\e_{it}$ стационарна, так как для $N \rightarrow \infty$ всегда можно состоятельно оценить $\mathrm Cov[u_{it}, u_{is}]$ с помощью $N^{-1}\sum_i \hat{u}_{it}\hat{u}_{it}$. Тем не менее, определение ARMA процесса для ошибок может представлять интерес.


\subsection{Оценка в первых разностях}

Модель within получается путем вычитания усредненной по времени модели $\bar{y}_i=\alpha_i+\bar{\x}'_i\bm\beta +\bar{\e}_i$ из первоначальной модели. Альтернативно можно вычесть первый временной лаг модели. Тогда
\begin{align}
&(y_{it}-y_{i,t-1})=(\x_{it}-\x_{i,t-1})'\bm\beta+(\e_{it}-\e_{i,t-1}), &t=2, \dots, T.
\label{Eq:21.34}
\end{align}
Фиксированные эффекты $\alpha_i$ элиминированы. МНК оценивание дает \textbf{оценку в первых разностях}
\begin{align}
\hat{\bm\beta}_{FD}=\left[\sum^N_{i=1}\sum^T_{t=2}(\x_{it}-\x_{i,t-1})(\x_{it}-\x_{i,t-1})'\right]^{-1}
\sum^N_{i=1}\sum^T_{t=2}(\x_{it}-\x_{i,t-1})(y_{it}-y_{i,t-1}).
\label{Eq:21.35}
\end{align}
Заметим, что в регрессии присутствует только $N(T-1)$ наблюдений. Распространенная ошибка в применении данного метода заключается в том, что сначала составляются все $NT$ наблюдений в один вектор, а потом вычитается первый лаг. В таком случае удаляется лишь одно наблюдение $(1, 1)$, в то время, как необходимо выбросить все $T$ наблюдений  первого временного периода $(i, 1), i=1, \dots, N$.

 {\centering
Состоятельность оценки в первых разностях\\}

Для состоятельности оценки в первых разностях необходимо выполнение условия $\mathrm E[\e_{it}-\e_{i,t-1}| \x_{it}-\x_{i, t-1}]$. Это более сильное условие, чем  $\mathrm E[\e_{it}|\x_{it}]=0$, но слабее, чем условие строгой экзогенности, требуемое для состоятельности оценки within.


 {\centering
Асимптотическое распределение оценки в первых разностях\\}

Для  статистических выводов необходима корректировка стандартных ошибок обычного МНК, учитывающая корреляцию во времени ошибок $\e_{it}-\e_{i,t-1}$. С целью получить асимптотическую дисперсию $\hat{\bm\beta}_{FD}$, запишем модель для $i$-го индивидуального наблюдения как
\begin{align}
\Delta \mathbf y_i=\Delta \mathbf X'_i \bm\beta + \Delta \bm\e_i,
\nonumber
\end{align}
где $\Delta \mathbf y_i$ --- вектор размерности $(T-1)\times 1$, элементами которого являются $(y_{i2}-y{i1}), \dots, (y_{iT}-y_{i,T-1})$, и $\Delta \mathbf X_i$ --- вектор размерности $(T-1)\times K$, рядами которого являются $(\x_{i2}-\x_{i1})', \dots, (\x_{iT}-\x_{i,T-1})'$. Тогда оценка
\begin{align}
\hat{\bm\beta}_{FD}=\left[\sum^N_{i=1} (\Delta \mathbf X_i)'(\Delta \mathbf X_i)\right]^{-1}
\sum^N_{i=1} (\Delta \mathbf X_i)'(\Delta \mathbf y_i)
\label{Eq:21.36}
\end{align}
в предположении о независимости по $i$ имеет ковариационную матрицу
\begin{align}
\mathrm V[\hat{\bm\beta}_{FD}]=\left[\sum^N_{i=1} (\Delta \mathbf X_i)'(\Delta \mathbf X_i)\right]^{-1}
\left[\sum^N_{i=1} (\Delta \mathbf X_i)' 
\mathrm V[\Delta \bm\e| \Delta \mathbf X_i]
(\Delta \mathbf X_i)\right]
\left[\sum^N_{i=1} (\Delta \mathbf X_i)'(\Delta \mathbf X_i)\right]^{-1}.
\label{Eq:21.37}
\end{align}

Самое простое предположение состоит в том, что $\e_{it}$ независимы и одинаково распределены с параметрами $[0,\sigma^2_\e]$. Тогда ошибка $\e_{it}-\e_{i,t-1}$ описывается процессом MA(1) с дисперсией $2\sigma^2_\e$ и ковариацией с предыдущим периодом $\sigma^2_\e$ для индивидуального наблюдения $i$. Из этого следует, что $\mathrm V[\Delta \bm\e_i]$  равна $\sigma_\e^2$, умноженной на матрицу размерности $(T-1)\times (T-1)$, состоящую из двоек на диагонали, единиц --- на диагоналях, соседних с главной, и нулей на остальных местах.

Более реалистично предположить, что $\e_{it}$ коррелированы во времени для данного $i$, т.е. $\mathrm{Cov}[\e_{it}, \e_{is}] \neq 0$ для $t \neq s$, но независимы по $i$. Из \ref{Eq:21.13} следует, что для коротких панелей оценкой, которая робастна к общим видам автокорреляции и гетероскедастичности, является \ref{Eq:21.37} с $\mathrm V[\Delta \bm\e_i]$, замененной на $(\widehat{\Delta \bm\e_i})'(\widehat{\Delta \bm\e_i})$. Никогда не следует использовать обычные стандартные ошибки из МНК регрессии модели в первых разностях \ref{Eq:21.37}, так как они являются верными только в том редком случае, когда $(\e_{it}$) описывается процессом случайного блуждания и, как следствие, $(\e_{it}-\e_{i,t-1})$ независимы и одинаково распределены.

Для $T=2$ оценка within и оценка в первых разностях равны, так как $\bar{y}=(y_1+y_2)/2$, что $(y_1-\bar{y})=(y_1-y_2)/2$ и $(y_2-\bar{y})=-(y_1-y_2)/2$ (аналогично для $\x$). В случае $T>2$ эти две оценки различаются. В самом простом предположении, что $\e_{it}$ независимы и одинаково распределены, может быть показано, что ОМНК оценка модели в первых разностях \ref{Eq:21.34} равна оценке within. Вместо этого оценка $\hat{\bm\beta}_{FD}$ получается посредством оценивания  \ref{Eq:21.34} МНК  и является менее эффективной, чем $\hat{\bm\beta}$. По этой причине оценка в первых разностях больше не описывается в большинстве вводных курсов. Однако она широко применяется, когда в модели присутствуют лаги зависимых переменных (см. главу 22). В таком случае оценка within состоятельна. Оценка в первых разностях также состоятельна, но она основывается на более слабых предположениях об экзогенности, что позволяет использовать состоятельную IV оценку.

\subsection{Оценка условного ММП}

Условный ММП заключается в максимизации совместного правдоподобия $y_{11}, \dots, y_{NT}$ при условии индивидуальных средних $\bar{y}_1, \dots, \bar{y}_T$. Этот метод привлекателен тем, что для линейной модели панельных данных в предположении нормальности фиксированные эффекты $\alpha_i$ устраняются, так что максимизация функции правдоподобия осуществляется только по $\bm\beta$.

Предположим, что $y_{it}$ при условии регрессоров $\x_{it}$ и параметров $\bm\alpha_i, \bm\beta$, и  $\sigma^2$ независимы и одинаково нормально распределены $\mathcal N [\bm\alpha_i+\x'_{it}\bm\beta, \sigma^2]$. Тогда \textbf{условная функция правдоподобия} равна
\begin{align}
\mathbf L(\bm\beta, \sigma^2, \bm\alpha)=
& \prod^N_{i=1} f(y_{i1}, \dots, y_{iT}|\bar{y}_i) =  \label{Eq:21.38} \\
& \prod^N_{i=1} \frac{f(y_{i1}, \dots, y_{iT}, \bar{y}_i)} {f(\bar{y}_i)} = \nonumber \\
& \prod^N_{i=1} \frac{(2\pi\sigma^2)^{-T/2}}{(2\pi\sigma^2/T)^{-1/2}}\mathrm{exp}
\left\{ \sum^{T}_{t=1} -[(y_{it}-\x'_{it}\bm\beta)^2+(\bar{y}_i-\bar{\x}'_i\bm\beta)^2]/2\sigma^2 \right\}. \nonumber
\end{align}
Первое равенство определяет условную функцию правдоподобия в предположении о независимости по $i$. Второе равенство выполняется всегда, так как $f(y_1, \dots, y_T|\bar{y})=f(y_i, \dots, y_T, \bar{y})/f(\bar{y})$ и $f(y_1, \dots, y_T, \bar{y})=f(y_1, \dots, y_T)$, а знание о том, что $\bar{y}=T^{-1}\sum_i y_i$, не приносит дополнительной информации при том, что уже известны $y_1, \dots, y_T$ (индекс $i$ здесь опущен для простоты). Третье равенство получается при условии нормальности посредством алгебраических преобразований, которые оставлены для читателя в качестве упражнения.

Ключевой результат заключается в том, что фиксированные эффекты $\bm\alpha$ отсутствуют в последнем выражении в \ref{Eq:21.38}, так $\mathbf L_{COND}(\bm\beta, \sigma^2, \bm\alpha)$ в действительности равно $\mathbf L_{COND}(\bm\beta, \sigma^2)$. Таким образом, необходимо максимизировать условную функцию максимального правдоподобия \ref{Eq:21.38} только по $\bm\beta$ и $\sigma^2$. \textbf{Оценка условного ММП} $\hat{\bm\beta}_{CML}$ получается в результате решения условий первого порядка
\begin{align}
\frac{1}{\sigma^2} \sum^T_{t=1} \sum^N_{i=1} [(y_{it}-\x'_{it}\bm\beta)\x_{it}-(\bar{y}_i-\bar{\x}'_i\bm\beta)\bar{\x}_i]=\mathbf 0,
\nonumber
\end{align}
или, что эквивалентно,
 \begin{align}
 \sum^T_{t=1} \sum^N_{i=1} [(y_{it}-\bar{y}_i)-(\x_{it}-\bar{\x)}'_i\bm\beta)]
(\x_{it}-\bar{\x}_i)=\mathbf 0.
\nonumber
\end{align}
Однако это всего лишь условия первого порядка из МНК регрессии $(y_{it}-\bar{y}_i)$ на $(\x_{it}-\bar{\x}_i)$.

Поэтому оценка условного ММП $\hat{\bm\beta}_{CML}$ равна оценке within $\hat{\bm\beta}_W$.

То, что метод дает состоятельную оценку, интуитивно объясняется тем, что условие $\bar{y}_i$ в \ref{Eq:21.38} уничтожило фиксированные эффекты. Более формально, $\bar{y}_i$ --- достаточная статистика для $\alpha_i$, а условие на достаточную статистику дает состоятельную оценку $\bm\beta$ (см. раздел 23.2.2).


\subsection{МНК оценка с фиктивными переменными}

Рассмотрим стандартную модель с фиксированными эффектами \ref{Eq:21.22} до взятия разностей. МНК анализ можно применить к модели напрямую, одновременно оценивая параметры $\bm\alpha$ и $\bm\beta$.

В принципе не требуется никакого специального программного обеспечения. Просто оценивается МНК регрессия $y_{it}$ на $\x_{it}$ и набор $N$ фиктивных переменных $d_{1,it}, \dots, d_{N,it}$, где $d_{j,it}$ равно единице, если $j=i$, и нулю в противном случае. Однако по мере того, как растет $N$, регрессия обрастает слишком большим числом регрессоров, чтобы обращать матрицу регрессоров размерности $(N+K) \times (N + K)$. Некоторые приемы матричной алгебры снижают проблему обращения матрицы размерности $(K \times K)$.

Конечная оценка $\bm\beta$ будет равна оценке within. Это частный случай теоремы Фриша-Вау. Строится регрессия всех переменных на фиктивные переменные, и если остатки из этой регрессии используются на втором шаге оценивания регрессии, то мы получим те же оценки, что и в полной регрессии. Но эти остатки являются просто отклонениями от своих средних, т.е. это регрессия within. Для полноты описания представим соответствующую матричную алгебру.

Запишем векторы размерности $T \times 1$ в \ref{Eq:21.29} всех $N$ индивидуальных наблюдений для получения \textbf{модели с фиксированными эффектами в виде фиктивных переменных}
\begin{align}
\begin{bmatrix}
 \mathbf y_{1} \\
 \vdots \\
 \mathbf y_{N}
\end{bmatrix}
=
\begin{bmatrix}
\mathbf e & \mathbf 0 & \mathbf 0 \\
\mathbf 0 & \ddots & \mathbf 0\\
\mathbf 0 & \mathbf 0 & \mathbf e
\end{bmatrix}
\begin{bmatrix}
\alpha_i \\ \vdots \\  \alpha_N 
\end{bmatrix}
+
\begin{bmatrix}
 \mathbf X_{1} \\ \vdots \\ \mathbf X_N
\end{bmatrix}
\bm\beta
+
&\begin{bmatrix}
 \bm\e_{1} \\
 \vdots \\
 \bm\e_{N}
\end{bmatrix}
\nonumber
\end{align}
или
\begin{align}
\mathbf y=[(\mathbf I_N \otimes \mathbf e) \mathbf X] 
\begin{bmatrix}
 \bm\alpha \\ \bm\beta
\end{bmatrix}
+\bm\e,
\label{Eq:21.39}
\end{align}
где $\mathbf y$ --- вектор размерности $NT \times 1$, кронекерово произведение $(\mathbf I_N \otimes \mathbf e)$ --- блочная диагональная матрица размерности $NT \times K$, а $ \mathbf X$ --- матрица регрессоров, не являющихся константами, размерности $NT \times K$.

Применение МНК к данной модели дает \textbf{МНК оценку с фиктивными переменными (LSDV)}
 \begin{align}
\begin{bmatrix}
 \hat{\bm\alpha}_{LSDV} \\ \hat{\bm\beta}_{LSDV}
\end{bmatrix}
&=
\begin{bmatrix}
(\mathbf I_N \otimes \mathbf e)'(\mathbf I_N \otimes \mathbf e) & (\mathbf I_N \otimes \mathbf e)'\mathbf X\\
\mathbf X'(\mathbf I_N \otimes \mathbf e) & \mathbf X' \mathbf X
\end{bmatrix}
^{-1} \times
\begin{bmatrix}
(\mathbf I_N \otimes \mathbf e)'\mathbf y\\
\mathbf X'\mathbf y
\end{bmatrix} 
\nonumber \\
&=
\begin{bmatrix}
T \mathbf I_N & T\bar{\mathbf X} \\
T \bar{\mathbf X}' & \mathbf X' \mathbf X
\end{bmatrix}
^{-1} \times
\begin{bmatrix}
 \bar{\mathbf y} \\ \mathbf X' \mathbf y,
\end{bmatrix}
\nonumber
\end{align}
где матрица выборочных средних $\bar{\mathbf X}=[\bar{\x}'_1 \dots \bar{\x}'_N]', \bar{\x}_i=T^{-1} \sum^T_{t=1} \x_{it}, \bar{\mathbf y}=[\bar{y} \dots \bar{y}_N]'$, и $\bar{y}_i=T^{-1} \sum^T_{t=1} y_{it}$. Используя формулу обращения разделенной на блоки матрицы, выполняем последующие алгебраические преобразования, получаем
 \begin{align}
\begin{bmatrix}
 \hat{\bm\alpha}_{LSDV} \\ \hat{\bm\beta}_{LSDV}
\end{bmatrix}
=
\begin{bmatrix}
\bar{\mathbf y}-\bar{\mathbf X} \bm\beta_W\\
[\mathbf X'\mathbf X-\bar{\mathbf X}'\bar{\mathbf X}]^{-1}(\mathbf X' \mathbf y --- \bar{\mathbf X}'\bar{\mathbf y}) 
\end{bmatrix}.
\label{Eq:21.40}
\end{align}
Перезаписав это в обозначениях сумм, имеем $\hat{\bm\beta}_{LSDV}=\hat{\bm\beta}_{W}$, определенное в \ref{Eq:21.24}, и $\hat{\bm\alpha}_{LSDV}=\hat{\bm\alpha}_{FE}$, определенное в \ref{Eq:21.25}, так что LSDV оценка равна оценке within или оценке с фиксированным эффектом. 

Для коротких панелей потенциальная очевидная проблема состоит в том, что получение состоятельной оценки для $\bm\beta$ и $\bm\alpha$ не гарантировано, так как необходимо оценивать $N+K$ параметров, а $N \rightarrow \infty$. Однако состоятельное оценивание $\bm\beta$  возможно при $T \rightarrow \infty$, даже если для $\bm\alpha$ получены несостоятельные оценки.

Эта оценка 'эффективна, если $\e_{it}$ независимы и одинаково распределены с параметрами $[0, \sigma^2]$. Из этого следует, что оценка within для $\bm\beta$ более эффективна, чем альтернативные оценки в разностях, например, такие, как вычитание первого наблюдения или наблюдения предыдущего периода, которые также элиминируют $\bm\alpha_i$. Если дополнительно ошибки нормально распределены, LSDV оценка равна оценке ММП ввиду обычной эквивалентности МНК и ММП в линейной модели со сферичными нормальными ошибками.

\subsection{Оценка ковариации}

Предположим, что данные принадлежат одному из $N$ классов, с $y_{it}$, обозначающим $t$-е наблюдение в $i$-м классе. \textbf{Анализ дисперсии} представляет общую изменчивость $y_{it}$ вокруг общего среднего $\bar{y}$, $\sum_i \sum_t (y_{it}-\bar{y})^2$, в виде декомпозиции на \textbf{внутригрупповую} дисперсию $\sum_i \sum_t (y_{it}-\bar{y}_i+\bar{y})^2$ и \textbf{межгрупповую} дисперсию $\sum_i(\bar{y}_i-\bar{y})^2$, где $\bar{y}_i$ --- среднее $i$-й группы. Принадлежность к группе становится более важной по мере увеличения межгрупповой дисперсии. \textbf{Анализ ковариации} расширяет этот подход и использует регрессоры, в этом случае сумма квадратов остатков подвергается декомпозиции похожим образом. Этот прием часто используется в в прикладной статистике.

В коротких панелях каждое индивидуальное наблюдение рассматривается как класс, наблюдаемый в течение нескольких временных периодов. Модель \ref{Eq:21.3} называется \textbf{моделью анализа ковариации}, так как она позволяет средним остаткам в $i$-м классе меняться в зависимости от класса. Оценка этой модели, оценка within, соответственно называется \textbf{оценкой ковариации}.

\section{Модель со случайными эффектами}

\textbf{Модель со случайными эффектами} \ref{Eq:21.3} может быть переписана в виде
 \begin{align}
&y_{it}=\mu + \x'_{it} \bm\beta +\alpha_i + \e_{it},
&i=1, \dots, N,
& t=1, \dots, T,
\label{Eq:21.41}
\end{align}
или
 \begin{align}
y_{it}=\mathbf w'_{it} \bm\delta+\alpha_i + \e_{it},
\label{Eq:21.42}
\end{align}
где $\mathbf w_{it} = [1  \; \x_{it}]$ и $\bm\delta=[\mu \; \bm\beta']'$. Предполагается, что индивидуальные эффекты $\alpha_i$ --- независимы и одинаково распределены  с параметрами $[0, \sigma^2_{\alpha}]$, а ошибки $\e_{it}$ независимы и одинаково распределены с параметрами $[0, \sigma^2_\e]$. Неслучайный скалярный свободный член $\mu$ добавляется таким образом, что в отличие от \ref{Eq:21.5} случайные эффекты могут быть нормированы и будут иметь нулевое среднее.

С другой стороны модель может рассматриваться как частный случай \textbf{модели со случайным коэффициентом} или \textbf{модели с изменяющимся коэффициентом}, где случаен только свободный член. Модель может быть переписана как $y_{it}=\mu+\x'_{it}\beta+u_{it}$, где ошибка состоит их двух компонент $u_{it}=\alpha_i+\e_{it}$. По этой причине модель со случайным эффектом также называется \textbf{моделью с компонентами ошибок}. Еще более точно эту модель можно было бы назвать \textbf{моделью со случайным свободным членом}. Более общие модели также могут содержать случайные коэффициенты наклона, см. главу 22.

Существует множество состоятельных оценок модели со случайными эффектами, включая (1) ОМНК оценивание в модели \ref{Eq:21.42}; (2) ММП в модели \ref{Eq:21.42} в предположении, что $\alpha_i$ и $\e_{it}$ нормально распределены; (3) МНК оценивание в модели \ref{Eq:21.42}; и (4) оценки с фиксированными эффектами, такие как оценка within или оценка в первых разностях, хотя эти оценки применимы лишь для оценивания коэффициентов, не меняющихся во времени. Первые две оценки асимптотически эквивалентны, но могут различаться в ограниченных выборках в зависимости от выбора оценок для $\sigma^2_{\alpha}$ и $\sigma^2_\e$. Оставшиеся оценки состоятельны, хотя неэффективны в том случае, если в действительности $\alpha_i$ и $\e_{it}$ независимы и одинаково распределены.

\subsection{ОМНК оценка}

\textbf{Оценка со случайным эффектом} $\mu$ и $\bm\beta$ --- это доступная ОМНК оценка модели \ref{Eq:21.42}. Ниже в этом разделе будет показано, что эта оценка может быть получена с помощью МНК оценивания измененного уравнения
 \begin{align}
y_{it}-\hat{\lambda}\bar{y}_i=(1-\hat{\lambda})\mu+(\x_{it}-\hat{\lambda}\bar{\x}_i)'\bm\beta+v_{it},
\label{Eq:21.43}
\end{align}
где $v_{it}=(1-\hat{\lambda})\alpha_i+(\e_{it}-\hat{\lambda}\bar{\e}_i)$ и $\hat{\lambda}$ состоятельна для 
 \begin{align}
\lambda=1-\sigma_\e/(T\sigma^2_{\alpha}+\sigma^2_\e)^{1/2}.
\label{Eq:21.44}
\end{align}
Эквивалентно,
 \begin{align}
\hat{\delta}_{RE}=
\begin{bmatrix}
 \hat{\mu}_{RE} \\ \hat{\bm\beta}_{RE}
\end{bmatrix}
\left[ \sum^N_{i=1} \sum^T_{t=1} (\mathbf w_{it}  -\hat{\lambda} \bar{\mathbf w}_i)
 (\mathbf w_{it}  -\hat{\lambda} \bar{\mathbf w}_i)' \right]^{-1}
\sum^N_{i=1} \sum^T_{t=1} (\mathbf w_{it}  -\hat{\lambda} \bar{\mathbf w}_i)(y_{it}-\hat{\lambda}\bar{y}_i),
\label{Eq:21.45}
\end{align}
где $\mathbf w_{it}=[1 \; \x_{it}]$ и $\bar{\mathbf w}_i=[1\; \bar{\x}_i]$. Для состоятельности необходимо условие $NT \rightarrow \infty$, для этого будет достаточно условия $N \rightarrow \infty$, или условия $T \rightarrow \infty$.

Предполагая, что $\e_{it}$ и $\alpha_i$ независимы и одинаково распределены, результаты оценивания регрессии \ref{Eq:21.43} обычным МНК могут быть использованы для получения оценки ковариационной матрицы,
 \begin{align}
\mathrm V
\begin{bmatrix}
 \hat{\mu}_{RE} \\ \hat{\bm\beta}_{RE}
\end{bmatrix}
=
\sigma^2_\e
\left[ \sum^N_{i=1} \sum^T_{t=1} (\mathbf w_{it} --- \hat{\lambda} \bar{\mathbf w}_i)
 (\mathbf w_{it} -\hat{\lambda} \bar{\mathbf w}_i)' \right]^{-1}.
\label{Eq:21.46}
\end{align}
В случае коротких панелей робастная оценка дисперсии, которая допускает более общую структуру для $\alpha_i+\e_{it}$, может быть получена, используя \ref{Eq:21.13}, из чего следует
 \begin{align}
\mathrm V
\begin{bmatrix}
 \hat{\mu}_{RE} \\ \hat{\bm\beta}_{RE}
\end{bmatrix}
=
\left[ \sum^N_{i=1} \sum^T_{t=1} \tilde{\mathbf w}_{it} \tilde{\mathbf w}_{it}' \right]^{-1}
\sum^N_{i=1} \sum^T_{t=1} \sum^T_{s=1} \tilde{\mathbf w}_{it} \tilde{\mathbf w}_{it}' \hat{\tilde{\e}}_{it} \hat{\tilde{\e}}_{is}
\left[ \sum^N_{i=1} \sum^T_{t=1} \tilde{\mathbf w}_{it} \tilde{\mathbf w}_{it}' \right]^{-1},
\label{Eq:21.47}
\end{align}
где $\tilde{\mathbf w}_{it}=\mathbf w_{it} -\hat{\lambda} \bar{\mathbf w}_{it}$ и $\tilde{\e}_{it}=\hat{\e}_{it}-\hat{\lambda}\bar{\hat{\e}}_i$, где $\hat{\e}_{it}$ --- остаток в модели со случайным эффектом. Эта оценка допускает произвольную автокорреляцию $\e_{it}$ и произвольную гетероскедастичность.

Для уравнения \ref{Eq:21.46} необходимы состоятельные оценки \textbf{компонент дисперсии} $\sigma^2_\e$ и $\sigma^2_\alpha$. Из регрессии within (или регрессии с фиксированными эффектами) $(y_{it}-\bar{y}_i)$ на $(\x_{it}-\bar{\x}_i)$ получаем
 \begin{align}
\hat{\sigma}^2_\e=\frac{1}{N(T-1)-K}
\sum_i \sum_t ((y_{it}-\bar{y}_i)-((\x_{it}-\bar{\x}_i)'\hat{\bm\beta}_W)^2.
\label{Eq:21.48}
\end{align}
Из регрессии between $\bar{y}_i$ на свободный член и $\bar{\x}_i$, уравнение, содержащее ошибку с дисперсией $\sigma^2_\alpha+\sigma^2_\e/T$, получаем
 \begin{align}
\hat{\sigma}^2_\alpha=\frac{1}{N-(K+1)}
\sum_i  (\bar{y}_{i}-\hat{\mu}_B-\bar{\x}'_i\hat{\bm\beta}_B)^2-\frac{1}{T}\hat{\sigma}^2_\e.
\label{Eq:21.49}
\end{align}
Возможны более эффективные оценки компонент дисперсии $\sigma^2_\e$ и $\sigma^2_\alpha$ (см., например, работу Амэмия, 1985), но такие оценки не обязательно будут приводит к увеличению эффективности $\hat{\bm\beta}_{RE}$. Выбор оценок достаточно широк. Оценка дисперсии \ref{Eq:21.43} может быть отрицательной. В таких случаях статистические пакеты зачастую присваивают нулевое значение дисперсии $\alpha$: $\hat{\sigma}^2_{\alpha}=0$. Тогда $\hat{\lambda}=0$, и оценивание производится посредством МНК для сквозной регрессии.

В целях проверки того, что доступный ОМНК упрощается до МНК оценивания \ref{Eq:21.43}, запишем \ref{Eq:21.42} по наблюдениям из всех $T$ временных периодов для данного $i$ таким же образом, как для модели с фиксированным эффектом. Тогда
 \begin{align}
\mathbf y_i=\mathbf W_i \bm\delta + (\mathbf e \alpha_i + \bm\e_i),
\label{Eq:21.50}
\end{align}
где $\mathbf y_i, \mathbf e, \bm\e_i$ и $\mathbf X_i$ определены так же, как и после \ref{Eq:21.29}, $\mathbf W'_i = [\mathbf e \; \mathbf X'_i]$. Для оценивания с помощью ОМНК необходимо получить ковариационную матрицу $\bm\Omega$ вектора ошибок $(\mathbf e \alpha_i +\e_i)$ размерности $T \times 1$. Учитывая, что $\alpha_i$ и $\e_{it}$ независимы, $\mathrm E[( \mathbf e \alpha_i+\bm\e_{i})( \mathbf e \alpha_i+\bm\e_{i})']=\mathrm E[\bm\e_i \bm\e'_i]+\mathrm E[\alpha^2_i]\mathbf e \mathbf e'$.

Так как $\e_{it}$ независимы и одинаково распределены с параметрами $[0, \sigma^2_\e]$ и $\alpha_i$ независимы и одинаково распределены с параметрами $[0,\sigma^2_\alpha]$ получаем
 \begin{align}
\bm\Omega=\sigma^2_\e \mathbf I_T + \sigma^2_{\alpha} \mathbf e \mathbf e'=\sigma^2_\e
\left[\mathbf Q+ \frac{1}{\psi^2}(\mathbf I_T-\mathbf Q) \right],
\nonumber
\end{align}
где $\mathbf Q= \mathbf I_T-T^{-1} \mathbf e \mathbf e'$ было преставлено в \ref{Eq:21.30} и $\psi^2=\sigma^2_\e/[\sigma^2_\e+T \sigma^2_\alpha]$. Используя 
$\mathbf Q \mathbf Q'= \mathbf Q$, можем легко проверить, что $\bm\Omega^{-1}=\sigma^{-2}_\e[\mathbf Q + \psi^2(\mathbf I_T-\mathbf Q)]$ и 
 \begin{align}
\bm\Omega^{-1/2}=\frac{1}{\sigma_\e}[\mathbf Q +\psi (\mathbf Q +\psi (\mathbf I_T- \mathbf Q)].
\label{Eq:21.51}
\end{align}
ОМНК оценка получается путем домножения \ref{Eq:21.50} на скалярный множитель $\bm\Omega^{-1/2}$. Сейчас
 \begin{align}
[\bm\Omega+\psi (\mathbf I_T --- \mathbf Q)] \mathbf y_i
&=\mathbf y_i --- \mathbf e \bar{y}'_i + \psi(\mathbf y_i --- (\mathbf y_i --- \mathbf e \bar{y}'_i)) \nonumber \\
&=\mathbf y_i --- \lambda \mathbf e \bar{y}'_i,
\nonumber
\end{align}
где $\lambda=(1-\psi)$. Применяя похожие алгебраические преобразования для $\mathbf W_i, \mathbf e \alpha_i$, и $\bm\e_i$ в \ref{Eq:21.50}, получаем следующую модель:
 \begin{align}
\mathbf y_i --- \lambda \mathbf e \bar{y}'_i=(\mathbf W_i --- \lambda \bar{\mathbf W})'\bm\delta+(1-\lambda)\alpha_i+(\bm\e_i-\lambda \mathbf e \bar{\e}'_i),
\label{Eq:21.52}
\end{align}
где преобразованная ошибка в \ref{Eq:21.52} имеет ковариационную матрицу $\sigma^2_\e \mathbf I_T$. ОМНК оценка --- это МНК оценка в \ref{Eq:21.52}, но \ref{Eq:21.52} --- это та же версия \ref{Eq:21.43}, записанная в векторной форме, со скалярной величиной $\lambda$, замененной на состоятельную оценку.

Оценка со случайным эффектом $\hat{\bm\beta}_{RE}$ параметра наклона сходится к оценке within при $T \rightarrow \infty$, так как в таком случае $\lambda \rightarrow 1$. В противном случае может быть показано, что $\hat{\bm\beta}_{RE}$ равна \textbf{матрично-взвешенной комбинации} оценки within и оценки between. Если модель со случайными эффектами адекватно описывает данные, это взвешенное среднее предпочитается простой оценке within. Однако если данные описываются  моделью с фиксированными эффектами, то взвешенное среднее не является состоятельной оценкой, так как оценка between в этом случае будет несостоятельна. Можно показать, что оценка свободного члена упрощается до $\hat{\mu}_{RE}=\bar{y}-\bar{\mathbf X}\hat{\bm\beta}_{RE}$. Более подробно см., например, работу Хсяо (2003, с.36) или Грина (2003).


\subsection{Оценка ММП}

В предыдущем разделе в ходе вывода не предполагается нормальность ошибок. Если они действительно являются \textbf{нормальными}, мы можем максимизировать логарифмическую функцию правдоподобия по $\bm\beta, \mu, \sigma^2_\e$, и $\sigma^2_{\alpha}$. Для данных  $\sigma^2_\e$ и $\sigma^2_{\alpha}$ оценка ММП для $\bm\beta$  и $\mu$ совпадает с оценкой ОМНК, но оценки ММП $\tilde{\sigma}^2_\e$ и $\tilde{\sigma}^2_{\alpha}$ отличаются от тех, что даны в \ref{Eq:21.48} и \ref{Eq:21.49}. 

Таким образом, оценки ММП для $\bm\beta$ и $\mu$ даны в \ref{Eq:21.45} с $\hat{\lambda}$, замененной на другую состоятельную оценку $\tilde{\lambda}=1-\tilde{\sigma}_\e/(T\tilde{\sigma}^2_\alpha+\tilde{\sigma}^2_\e)^{1/2}$. Асимптотически, оценки ММП и ОМНК модели со случайными эффектами эквивалентны, однако в ограниченных выборках они будут различаться.

Для оценки ММП возможны два локальных максимума функции правдоподобия при $0 < \psi^2 \leq 1$, а не один. Поэтому необходимо позаботиться о том, чтобы получить глобальный максимум.


\subsection{Другие оценки}

Множество других оценок $\bm\beta$ состоятельны, если модель со случайными эффектами является истинной моделью. В частности, оценки МНК сквозной регрессии, within, в первых разностях, и between будут состоятельны. Однако они не будут эффективными, если $\alpha_i$ и $\e_{it}$ независимы и одинаково распределены, а оценки within и в первых разностях могут оценивать коэффициенты только  регрессоров, меняющихся во времени.

\section{Особенности моделирования}

В этом разделе мы рассмотрим некоторые практические особенности, которые имеют место при оцениваниии линейных панельных данных, даже в отсутствие таких трудностей как эндогенность или лаговые зависимые переменные и других сложностей, которые будут рассмотрены в главе 22.


\subsection{Тесты на объединение}

В модели со случайными эффектами все параметры регрессии одинаковы для разных кросс-секций и временных периодов, в то время как в моделях с фиксированными эффектами параметры постоянны за исключением свободного члена, который может изменяться в зависимости от индивидуальных наблюдений. \textbf{Тесты на  объединение} проверяют адекватность данных ограничений.

Эти тесты обычно реализуются при помощи теста Чоу (см. работу Грина, 2003, c. 130), основанного на тестах на равенство регрессоров в двух линейных регрессиях в предположении об одинаковых дисперсиях. В зависимости от предположений относительно ошибок, тест Чоу может быть применен к моделям, оцененным с помощью МНК или ОМНК. Бальтаджи (2001, глава 4) и Хсяо (2003, глава 2) подробно это описывают.

В случае с короткими панелями невозможно предполагать, что параметры наклона будут различаться по индивидуальным наблюдениям, так как в таком случае количество параметров возрастает слишком сильно. Однако параметры наклона могут меняться во времени. Тестируется модель $y_{it}=\gamma+\x'_{it} \bm\beta+u_{it}$ против модели $y_{it}=\gamma_t+\x'_{it} \bm\beta+u_{it}$. Самым очевидным будет предположить случайные эффекты вида $u_{it}=\e_{it}+\alpha_i$, оценить модель с ограничениями ($\gamma_t=\gamma$ и $\bm\beta_t=\bm\beta$), используя ОМНК оценку со случайным эффектом, и сравнить сумму квадратов остатков моделей с ограничениями и без в преобразованной модели. Если необходимо получить более робастные статистичесике выводы, тогда необходимо вычислить робастные для панельных данных стандартные ошибки и провести тест Вальда. Чаще всего для коротких панелей специфицируется модель с постоянными параметрами наклона $\bm\beta$, хотя свободный член $\gamma_t$ может меняться во времени посредством включения временных фиктивных переменных в качестве дополнительных регрессоров.

\subsection{Тесты на индивидуальные эффекты}

Бройш и Паган (1980) вывели тест множителей Лагранжа, проверяющий гипотезу наличия индивидуальных случайных эффектов против нулевой гипотезы о независимых и одинаково распределенных ошибках. Преимущество состоит в том, что тест легко осуществляется благодаря вспомогательной регрессии, для которой необходимы лишь остатки из МНК оценивания сквозной регрессии. Альтернативно можно предположить нормальность и воспользоваться тестом отношения правдоподобия для  оценки ММП со случайным эффектом против оценки ММП модели с постоянными коэффициентами, или тестом Вальда для гипотезы $\sigma_{\alpha}=0$ в модели со случайными эффектами.

На практике нулевая гипотеза о том, что ошибки в модели с постоянными коэффициентами независимы и одинаково распределены, отвергается. Самый простой способ --- оценить модель с помощью МНК сквозной регрессии с вычислением робастных для панельных данных стандартных ошибок или с помощью ОМНК со случайным эффектом.

Для коротких панелей тест на присутствие индивидуальных фиксированных эффектов невозможен в том числе из-за проблемы с параметрами. Невозможно протестировать, будут ли $N$ параметров равны нулю, когда в наличии только $NT$ наблюдений и $T$ достаточно мало. Вместо этого, тест Хаусмана, описанный в разделе 21.4.3, используется для тестирования нулевой гипотезы со случайными эффектами против альтернативной гипотезы о фиксированных эффектах.

\subsection{Прогнозирование}

Прогнозирование в моделях без индивидуальных эффектов довольно простое: $\hat{y}_{js}=\x'_{js}\hat{\bm\beta}$. Это прогнозирование среднего генеральной совокупности $\mathrm E[y_{js}|\x_{js}]$.

Прогнозирование для данного индивидуального наблюдения при условии индивидуального эффекта менее тривиально. Это прогнозирование $\mathrm E[y_{js}|\x_{js}, \alpha_i]$. Мы рассмотрим прогноз наблюдения, не входящего в выборку, для $i$-го индивидуального наблюдения, используя модель со случайными эффектами \ref{Eq:21.42}. Тогда $y_{i,t+s}=\mathbf w'_{it} \bm\delta + u_{i,t+s}$, где $u_{i,t+s}=\alpha_i+\e_{i,t+s}$. При очевидном прогнозировании $\bm\delta$ заменяется на $\hat{\delta}_{RE}$ и $u_{i,t+s}$ --- на 0 или $\bar{\hat{u}}_i$, где $\bar{\hat{u}}_i=\bar{y}_i-\mathbf w'_i\hat{\bm\delta}_{RE}$ --- выборочное среднее остатков модели within для $i$-го наблюдения. Однако такое прогнозирование не будет эффективным, так как оно игнорирует корреляцию между $u_{i,t+s}$ и ошибками, вызванную индивидуальным случайным эффектом $\alpha_i$. Эта проблема является, скорее, примером более общей проблемы прогнозирования в рамках ОМНК, а не МНК. Лучшее линейное несмещенное прогнозирование для этого специального случая (см. раздел 22.8.3) --- это $\hat{y}_{i,t+s}=\x'_{it}\hat{\delta}_{RE}+(T\sigma^2_\alpha/(T\sigma^2_\alpha+\sigma^2_\e))\bar{\hat{u}}_i$.
Аналогичное очевидное прогнозирование для модели с фиксированными эффектами: $\hat{y}_{i,t+s}=\x'_{it}\hat{\bm\beta}_W+\hat{\alpha}_{i,FE}$. Опять же это прогнозирование не будет состоятельным в случае короткой панели.

\subsection{Модели с двусторонними  эффектами}

До этого момента анализ был сфокусирован на модели только с индивидуальными эффектами \ref{Eq:21.1}, где $u_{it}=\alpha_i+\e_{it}$. Более общая модель --- \textbf{модель с двусторонними (two-way) эффектами}, где $u_{it}=\alpha_i+\gamma_t+\e_{it}$, т.е. которая предполагает также наличие временных эффектов. Тогда
 \begin{align}
& y_{it}=\alpha_i+\gamma_t+\x'_{it}\bm\beta+\e_{it}
& i=1, \dots, N, &
& t=1, \dots, T.
\label{Eq:21.53}
\end{align}
Эта модель была представлена в \ref{Eq:21.2}.

Как уже было отмечено, обычным подходом для коротких панелей было предположение о том, что временные эффекты фиксированы, и оценивание их как коэффициентов фиктивных переменных, включенных в состав регрессоров, наряду с анализом о природе индивидуальных эффектов.

Если $\alpha_i$ и $\gamma_t$ фиксированы, то оценка МНК коэффициентов $\bm\beta$ в \ref{Eq:21.53} эквивалентна регрессии $y_{it}-\bar{y}_i-\bar{y}_t+\overline{\bar{y}}$ на $\x_{it}-\bar{\x}_i-\bar{\x}_t+\overline{\bar{\x}}$, где $\bar{y}_i=T^{-1}\sum^T_{t=1} y_{it}, \bar{y}_t=N^{-1}\sum^N_{i=1}y_{it}$, и $\overline{\bar{y}}=(NT)^{-1}\sum^N_{i=1} \sum^T_{i=1}y_{it}$. $\bar{\x}_i, \bar{\x}_t,$ и $\overline{\bar{\x}}$ определены аналогично. Этот метод дает состоятельные оценки, если $T$ достаточно большое.

Если вместо этого $\alpha_i$ и $\gamma_t$  случайны, то ошибки будут содержать компоненту $\gamma_t$, которая вызывает корреляцию ошибок по индивидуальным наблюдениям, в то время как мы предполагали независимость по $i$. Можно показать, что ОМНК оценка может быть посчитана посредством МНК оценивания $y^{\star}_{it}$ на константу и $\x_{it}^{\star}$,
 \begin{align}
 y_{it}^{\star}=y_{it}-\lambda_1\bar{y}_i-\lambda_2\bar{y}_t+\lambda_3\overline{\bar{y}},
\nonumber
\end{align}
где $\bar{y}_i$, $\bar{y}_t$ и $\overline{\bar{y}}$ уже были определены, а $\x^{\star}_{it}$ определяется аналогично $y^{\star}_{it}$. Этот и другие результаты для модели с двусторонними эффектами представлены у Хсяо (2003) или Бальтаджи (2001). 

\subsection{Несбалансированные панельные данные}

До этого момента предполагалось, что панель сбалансирована, что означает доступность данных для каждого индивидуального наблюдения в каждый из временных периодов. Для региональных панельных данных это предположение часто выполняется. Для панельных исследований индивидуумов, напротив, зачастую характерно сокращение или \textbf{истощение} со временем пропорции индивидуумов, которые продолжают отвечать на вопросы исследования. Более того,  в некоторых случаях некоторые индивидуумы могут пропустить один или более периодов, а затем снова вернуться в выборку, согласно  дизайну \textbf{чередующихся панелей}, таких как CPS, где домохозяйства опрашиваются в течение четырех месяцев, затем восемь месяцев не опрашиваются, а потом снова наблюдаются четыре месяца. Такие панели, где разные индивидуумы появляются в разные годы, называются \textbf{несбалансированными или неполными панелями}.

Пусть $d_{it}$ будет индикаторной переменной, равной 1, если $i$-е наблюдение наблюдается, и равной нулю в противном случае. Тогда для модели с индивидуальными эффектами \ref{Eq:21.3} оценка с фиксированным эффектом будет состоятельной, если  предположение \ref{Eq:21.4} о строгой эндогенности будет выглядеть следующим образом:
 \begin{align}
\mathrm E[u_{it}|\alpha_i, \x_{i1}, \dots, \x_{iT}, d_{i1}, \dots, d_{iT}]=0,
\label{Eq:21.54}
\end{align}
и оценка со случайным эффектом состоятельна, если дополнительно $\alpha_i$ не зависит от остальных переменных условия. Оценки с фиксированным и случайным эффектом могут применяться к несбалансированным данным с относительно небольшой корректировкой. Это должно быть понятно из изначального представления оценок как МНК оценок в различных моделях, представленных в разделе 22.2.2. Например, замена $\hat{\lambda}$ в \ref{Eq:21.10} для модели со случайными эффектами на $\hat{\lambda}_i=1-\sigma_\e/(T_i\sigma^2_\alpha+\sigma^2_\e)^{1/2}$, где $T_i$ --- количество наблюдений для индивидуального наблюдения $i$ (см. работы Бальтаджи, 1985, Вансбика и Каптейна, 1989). Дэвис (200) рассматривает модель со случайными многосторонними (multi-way) эффектами. Индивидуальное наблюдение для модели с фиксированными эффектами должно наблюдаться по крайней мере дважды в выборке, а число степеней свободы должно быть скорректировано. Бальтаджи (2001) довольно широко описывает, как нужно обращаться с несбалансированными панелями. Эконометрические пакеты, которые оценивают основные  модели панельных данных, описанные в главах 21-23, обычно автоматически справляются с пропущенными наблюдениями.

Иногда удобно конвертировать несбалансированную панель в сбалансированную, включая в выборку только тех индивидуумов, данные по которым известны для всех периодов. Очевидно, это сильно снизит эффективность в связи с потерей большого количества наблюдений. Более того, если данные пропущены неслучайно, это может усилить потенциальную проблему нерепрезентативности выборки.

Одной из причин отсутствия данных может быть ненаблюдаемость по меньшей мере одной переменной, в то время как все остальные переменные наблюдаемы. Например, \textbf{уровень отсутствия ответов} на вопросы о доходах может быть вполне высоким. Чем выбрасывать целое наблюдение в связи с отсутствием данных об одном регрессоре, доходе, может оказаться выгодным использование методов восстановления данных, описанных в главе 27.

Необходимо использовать особые методы для анализа несбалансированных панелей, если причина выпадания индивидуумов из выборки коррелирована с ошибкой, так что \ref{Eq:21.52} не выполняется. Например, индивидуумы с необычно низкими заработными платами (после учета наблюдаемых характеристик) могут выпадать из выборки с большей вероятностью. В результате нерепрезентативная панель приведет к \textbf{смещению в связи с истощением} (attrition bias), если заработная плата выступает в качестве зависимой переменной. Для получения состоятельной оценки нужно использовать методы формирования выборки, расширенные для панельных данных (см. раздел 23.5.2).

\subsection{Ошибки измерения}

Ошибки измерения в регрессорах приводят к несостоятельным оценкам параметров в моделях с данными пространственного типа. Если используются методы панельных данных, включающие в себя взятие разностей, проблема несостоятельности оценок может в результате только усугубиться.  Несостоятельность будет  вызвана ошибкой измерения и будет зависеть от предположений, сделанных относительно процесса, порождающего данные.

\subsection{Практические соображения}

Различные оценки, представленные в данной главе просты в применении. Самый проверенный метод --- использование команд для панельных данных, доступных в таких эконометрических пакетах, как LIMDEP, STATA и TSP, каждый из которых можно использовать для  несбалансированных панелей. Большинство оценок могут быть получены с использованием адекватной сквозной МНК регрессии на преобразованных данных, для которой необходим лишь стандартный алгоритм для анализа данных пространственного типа. Однако стандартные ошибки при этом могут отличаться от тех, что получаются при использовании алгоритма для панельных данных, так как пространственный алгоритм может игнорировать автокорреляцию, вызванную преобразованием данных, и алгоритмы могут использовать разное количество степеней свободы.

Недостаток команд статистических пакетов состоит в том, что в настоящий момент они вычисляют стандартные ошибки, которые основываются на  предположениях о распределениях: в моделях с фиксированными эффектами ошибки независимы и одинаково распределены,  индивидуальные эффекты и ошибки независимы и одинаково распределены в модели со случайными эффектами. Для расчета более робастных оценок стандартных ошибок, описанных в данной главе, требуется оценивание панельных данных с панельным бутстрэпом или МНК оценивание сквозной регрессии с опцией расчета кластерных робастных стандартных ошибок.

В микроэконометрическом анализе модели с и без фиксированных эффектов существенно различаются. Если выбрана модель без фиксированных эффектов, то это должно быть подтверждено тестом Хаусмана. Если этот тест отвергает модель со случайными эффектами, тогда возможно состоятельно оценить коэффициенты  регрессоров, не меняющихся во времени,  используя инструментальные переменные, представленные в следующей главе.

\subsection{Библиографические заметки}

Большинство учебников, таких как учебник Грина (2003), содержат по меньшей мере главу, посвященную анализу панельных данных. В книге Вулдриджа (2002) содержится несколько глав, которые посвящены линейным и нелинейным моделям панельных данных. Эконометричеcкие монографии о панельных данных  --- Хсяо (1986, 2003), Бальтаджи (1995, 2001), Матиас и Севестр (1995), М.-Дж. Ли (2002) и Ареллано (2003). Последние три книги уделяют большое внимание методам, представленным в главах 22 и 23. Диггл, Лианг и Цегер (1994, 2002) представляет собой стандартный статистический справочник.

\textbf{21.4} Мундлак (1978) написал классическую статью о выборе между моделями с фиксированными и случайными эффектами. Хаусман (1978) для иллюстрации своего подхода использовал тесты для выбора между этими двумя моделями.

\textbf{21.6} Кух (1959) и Хох (1962) были авторами одного из самых первых исследований на панельных данных, в котором оценивали инвестиционные функции и производственные функции Кобба-Дугласа. Эти работы сравнивают использование оценок within с использованием изменчивости во времени и оценками between с использованием изменчивости внутри кросс-секций.


 {\centering
{\bf Упражнения}\\}

\textbf{21-1} (Бальтаджи, 1999) Рассмотрим модель анализа панельных данных $y_{it}=\alpha + \beta \x_{it}+u_{it}$, где $\alpha$ и $\beta$ --- скалярные величины.
\begin{itemize}
\item[{\bf (a)}] Покажите вычитанием, что из этой модели следует
 \begin{align}
y_{it}-\bar{y}=\beta(\x_{it}-\bar{\x}_i) + \beta(\bar{\x}_{i}-\bar{\x}) + (u_{it}-\bar{u}),
\nonumber
\end{align}
где $\bar{y}=(NT)^{-1}\sum_{i,t} y_{it}$, $\bar{x}=(NT)^{-1}\sum_{i,t} x_{it}$ и $\bar{\x}_i=T^{-1}\sum_t\x_{it}$.

\item[{\bf (b)}] Для соответствующей неограниченной  МНК регрессии
\begin{align}
y_{it}-\bar{y}=\beta(\x_{it}-\bar{\x}_i) + \beta(\bar{\x}_{i}-\bar{\x}) + (u_{it}-\bar{u})
\nonumber
\end{align}
покажите, что МНК оценка $\beta_1$ --- это оценка within, а $\beta_2$ --- оценка between.

\item[{\bf (с)}] Покажите, что если $u_{it}=\mu_i+v_{it}$, где $\mu_i$ независимы и одинаково распределены с параметрами $[0,\sigma^2_\mu]$ и $v_{it}$ независимы и одинаково распределены с параметрами $[0,\sigma^2_v]$, и они независимы по  $i$ и $t$, то МНК и ОМНК оценки эквивалентны.
\end{itemize}

\textbf{21-2} Рассмотрите оценивание фиксированных эффектов линейной регрессионной модели $y_{it}=\alpha_i+\x'_{it}\bm\beta+\e_{it}$, где $\alpha_i$ --- фиксированные эффекты, возможно коррелированные с $\x_{it}$. Объединение всех  $T$ наблюдений в один вектор для индивидуума $i$ дает $\mathbf y_i=\alpha_i \mathbf e + \mathbf X_i \beta + \e_i$ (см. \ref{Eq:21.29}). Рассмотрим оценку $\hat{\beta}=\sum^N_{i=1}[ \mathbf X'_i \mathbf J' \mathbf J \mathbf X_i]^{-1} \times \sum^N_{i=1}\mathbf X'_i \mathbf J' \mathbf J \mathbf y_i$, где $\mathbf J$ --- это матрица размерности $T \times T$ известных констант. Эти константы таковы, что $\mathbf J \mathbf e =\mathbf 0$. [Заметим, что примером $\mathbf J$ может служить $\mathbf Q = \mathbf I_T --- T^{-1} \mathbf e \mathbf e'$.]

\begin{itemize}
\item[{\bf (a)}] Объясните, почему необходима такая оценка $\hat{\beta}$.

\item[{\bf (b)}] Найдите $\mathrm E[\hat{\beta}]$. Для простоты используйте предположение, что $\mathbf X_i$ --- фиксированные регрессоры, и что $\e_{it} \thicksim [0, \sigma^2]$. Является ли $\hat{\beta}$ несмещенной оценкой для $\beta$?

\item[{\bf (с)}] Найдите $V[\hat{\beta}]$. Для простоты используйте предположение, что $\mathbf X_i$ --- фиксированные регрессоры, и что $\e_{it} \thicksim [0, \sigma^2]$.

\item[{\bf (d)}] Теперь предположите, что $\e_{it}$ независимы по $i$, но коррелированы по $t$ с $V[\e_i]=\Omega_i$. Найдите $V[\hat{\beta}]$.

\item[{\bf (e)}] Предположим, что $\alpha_i$ --- случайные эффекты с $(0, \sigma^2_\alpha)$, а не фиксированные. Будет ли в этом случае оценка состоятельна?

\end{itemize}

\textbf{21-3} (Бальтаджи, 1998) Рассмотрим фиксированные эффекты, модель панельных данных с временными и индивидуальными эффектами
 \begin{align}
y_{it}=\alpha + \x'_{it} \bm\beta +\bm\mu_i+\lambda _t+\e_{it},
\nonumber
\end{align}
где $\alpha$  это скалярная величина, $\x_{it}$ --- вектор экзогенных переменных размерности $k \times 1$, $\bm\beta$ --- вектор размерности $K \times 1$, $\mu$ и $\lambda$ обозначают фиксированные индивидуальные и временные эффекты соответственно, и $\e_{it}$ независимы и одинаково распределены с параметрами $[0, \sigma^2]$.

\begin{itemize}
\item[{\bf (a)}] Покажите, что оценка within параметра $\bm\beta$, которая является лучшей линейной несмещенной оценкой, может быть получена посредством двух преобразований within (one-way). Первое преобразование --- преобразование within, игнорирующее фиксированные эффекты. За ним следует преобразование within, игнорирующее индивидуальные эффекты.

\item[{\bf (b)}]  Покажите, что порядок этих двух преобразований неважен. Дайте интуитивное объяснение этому результату.

\end{itemize}

\textbf{21-4} Используйте 50\%-ю случайную подвыборку данных о заработной плате и количестве часов работы в разделе 21.3.
\begin{itemize}
\item[{\bf (a)}] Может ли $\beta$ интерпретироваться как эластичность предложения труда? Объясните.

\item[{\bf (b)}] Для следующих оценок: (1) МНК сквозной регрессии, (2)  between, (3) within, (4)  в первых разностях, (5) ОМНК со случайным эффектом, (6) оценка ММП со случайным эффектом запишите (i) $\hat{\beta}$ (оцененный коэффициент lnwg), (ii) стандартные ошибки, вычисляемые по умолчанию и (iii) робастные для панельных данных стандартные ошибки, полученные методом бутстрэп, используя 200 репликаций.

\item[{\bf (c)}] Схожи ли оценки $\beta$?

\item[{\bf (d)}] Есть ли систематическая разница между стандартными ошибками, вычисляемыми по умолчанию, и робастными для панельных данных стандартными ошибками.

\item[{\bf (e)}] Будет ли МНК оценка сквозной регрессии в части (b)  состоятельной оценкой $\beta$ в модели с фиксированными эффектами? Будет ли МНК оценка сквозной регрессии состоятельна для $\beta$ в модели со случайными эффектами?

\item[{\bf (f)}] Выполните тест Хаусмана и сделайте выбор между ОМНК оценкой со случайным эффектом и оценкой с фиксированным эффектом параметра $\beta$ в этой модели. Проведите тест вручную, используя результаты предыдущей регрессии со стандартными ошибками, вычисленными по умолчанию. Какой вывод Вы сделаете и какую модель предпочтете?

\item[{\bf (g)}] Верите ли Вы, что кривая предложения труда имеет положительный наклон? Объясните.

\end{itemize}

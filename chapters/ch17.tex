
                %%%% Comments and Terms %%%%
                %%%%%%%%%   "Транзитные данные"
                % \url{http://finbiz.spb.ru/download/4_2008_nivorog.pdf}
                %%%%%%%%%   Ениколопов о Treatment groups
                % \url{http://quantile.ru/06/06-RE.pdf}
                %%%%%%%%%   Ненаблюдаемая гетерогенность
                %\url{http://books.google.de/books?id=B-6hxCay0PkC&pg=PA551&lpg=PA551&dq=%22%D0%BD%D0%B5%D0%BD%D0%B0%D0%B1%D0%BB%D1%8E%D0%B4%D0%B0%D0%B5%D0%BC%D0%B0%D1%8F+%D0%B3%D0%B5%D1%82%D0%B5%D1%80%D0%BE%D0%B3%D0%B5%D0%BD%D0%BD%D0%BE%D1%81%D1%82%D1%8C%22&source=bl&ots=V1NuZfl839&sig=LlpEFlSPrgVIc5kT102FqB32lbA&hl=en&sa=X&ei=uw55UpPYI8bHswaYz4HACQ&redir_esc=y#v=onepage&q=%22%D0%BD%D0%B5%D0%BD%D0%B0%D0%B1%D0%BB%D1%8E%D0%B4%D0%B0%D0%B5%D0%BC%D0%B0%D1%8F%20%D0%B3%D0%B5%D1%82%D0%B5%D1%80%D0%BE%D0%B3%D0%B5%D0%BD%D0%BD%D0%BE%D1%81%D1%82%D1%8C%22&f=false} % \url{http://quantile.ru/01/01-Articles.pdf}
                % \url{http://www.quantile.ru/01/01-SA2.pdf}
                %%%%%%%%%   Модели с преобразованием
                % \url{http://quantile.ru/05/05-Issue.pdf}
                %%%%%%%%%   Набор идентичных данных
                % \url{http://www.proz.com/kudoz/english_to_russian/medical%3A_pharmaceuticals/1501766-tied_data.html}
                %%%% %%%%%%%% %%%%

\chapter*{Глава 17. Транзитные данные: Анализ выживаемости}
% В отечественной литературе нет единой терминологии в отношении моделей выживаемости. Поэтому многие понятия, которые мы будем употреблять в этой и последующих главах можно встретить под другими названиями. Например, анализ выживаемости также называют анализом дожития. Чтобы избежать путаницы, в скобках мы будем указывать англоязычные термины, и иногда приводить альтернативные варианты перевода на русский.
% \textbf{Идея: в конце книги сделать приложение-табличку с англ.терминами и вариантами перевода?}


\section{Введение}
\label{sec:17.1}

\noindent
Эконометрическими моделями времени жизни \emph{(models of durations)} называются модели, которые позволяют анализировать продолжительность пребывания в каком-либо состоянии до перехода к другому. Например, это может быть безработное состояние, состояние жизни или отсутствие страхового полиса. В биостатистике такую продолжительность называют \textbf{временем жизни}, а момент перехода --- \textbf{смертью}. В операционных исследованиях занимаются изучением времени жизни таких объектов, как лампы, приборы или машины, поэтому предметом анализа будет являться срок службы прибора, а моментом перехода --- \textbf{время отказа} \emph{(failure time)}. В эконометрике \textbf{состояние} определяют как класс исследуемого объекта в определенный момент времени, а \textbf{момент перехода} как перемещение объекта из одного состояния в другое. \textbf{Длительность состояния} \emph{(spell length)} или \textbf{время жизни} \emph{(duration)}, называют временем, которое объект находился в данном состоянии. В качестве типовой задачи регрессионого анализа можно рассматривать оценку влияния размера пособия по безработице на длительность безработного состояния или же вероятность перехода в другое.

Литература, посвященная анализу времени жизни, может путать и сбивать читателя с толку по ряду причин.
Во-первых, объектом анализа может выступать как продолжительность пребывания в некотором состоянии, так и вероятность перехода в другое; при этом может использоваться несколько различных функций распределения.
Во-вторых, результаты зависят не только от выбранной модели, но и от метода построения выборки. Например, для данных по безработице можно построить выборку типа поток \emph{(flow sampling)} (мы знаем момент времени, когда индивид получил статус безработного), запас \emph{(stock sampling)} (мы знаем только статус индивида) или же выборку всего населения, независимо от статуса.
В-третьих, данные зачастую являются цензурированными. Это главная причина, по которой следует моделировать переходы, а не среднюю длительность состояния, как обычно принято в регрессионном анализе, поскольку не требуется вводить строгие предположения относительно распределения ошибок для состоятельности оценок.
% ALT: предпочитается моделировать?
В-четвертых, транзитные данные могут включать не одно, а несколько состояний. Например, индивид может быть безработным, работать на пол-ставки, работать полный рабочий день, или вовсе быть исключен из рабочей силы. Более того, данные по конкретному индивиду могут содержать несколько переходов между этими состояниями.
Наконец, методы анализа выживаемости применяются во многих прикладных разделах статистики, в каждом из которых существуют свои принятые названия и обозначения. Так,
    \textbf{анализ времени жизни} или \textbf{перехода} из одного состояния в другое, также называется \textbf{анализом выживаемости} или \textbf{дожития} \emph{(survival analysis)} (т.е. времени, в течение которого объект оставался живым) в биостатистике,
    \textbf{анализом времени отказа} \emph{(failure time analysis)} (т.е. времени до отказа лампочки, детали и т.д.) в операционных исследованиях,
    \textbf{анализом таблицы выживания} \emph{(life table analysis)} в демографии и актуарной математике (где переход в другое состояние означает смерть) и
    \textbf{анализом рисков} \emph{(hazard analysis)} в страховании.
    В прикладных социальных науках также занимаются изучением рецидивов, продолжительности браков, временем до очередных выборов.

Результаты, представленные в данной главе, основаны на данных \textbf{типа поток} с \textbf{единственным переходом}. В качестве классического примера можно рассматривать моделирование момента смерти, известное из анализа выживаемости, или анализа таблицы выживания. Поскольку это является наиболее популярным примером в статистике, соответствующие методы анализа встроены в большинство статистических пакетов. В начале этой главы мы представим эмпирический пример для того, чтобы подчеркнуть некоторые отличительные черты, характерные для данных о выживаемости.

В силу такой специфичности данных мы начнем анализ с определения базовых концепций и описательных характеристик при отсутствии каких-либо объясняющих переменных. В частности, в разделах \ref{sec:17.3}-\ref{sec:17.5} мы рассмотрим такие основные понятия, как риск, кумулятивный риск и функция выживания. В разделе \ref{sec:17.4} мы определим различные типы цензурирования, необходимые для учета ненаблюдаемых переходов. Например, клинические испытания обычно завершаются до того, как умрет последний объект; значит, момент смерти последнего объекта будет неизвестен, или цензурирован. В разделе \ref{sec:17.5} мы представим непараметрические оценки риска, кумулятивного риска (оценка Нельсон-Аалена) и функции выживания (оценка Каплан-Мейера), которые являются состоятельными оценками при предпосылке о независимом цензурировании.

Сохранив эту предпосылку, мы расширим анализ с помощью введения регрессоров. Оценивание полностью параметрических моделей, в частности, модели Вейбулла, представлено в разделе \ref{sec:17.6}. Роль цензурирования аналогична её роли в модели тобит. Некоторые важные модели времени жизни можно найти в разделе \ref{sec:17.7}. Альтернативный полупараметрический подход, предложенный Коксом (1972), заключается в моделировании функции риска, или вероятности смерти при условии, что она не наступила раньше, при относительно слабых предположениях о распределениях параметров. Модель Кокса, ставшая с тех пор стандартной для анализа выживаемости, представлена в разделе \ref{sec:17.8}. В отличие от большинства пространственных моделей, модели времени жизни допускают изменения регрессоров во времени (например, размера пособия по безработице). Соответствующие модели с регрессорами, меняющимися со временем, описаны в разделе \ref{sec:17.9}. Модели рисков в дискретном случае рассмотрены в разделе \ref{sec:17.10}. Наконец, эмпирический пример в разделе \ref{sec:17.11} является иллюстрацией к описанным методам.

В следующих двух главах мы рассмотрим более сложные аспекты моделирования переходов, редко встречающиеся в учебниках. В частности, речь пойдет о ненаблюдаемой гетерогенности и моделях с множественными состояниями.




\section{Пример: Длительность забастовок}\label{sec:17.2}

\begin{figure}[ht!]\caption{Длительность забастовок: оценка Каплан-Мейера функции выживания. Данные по 566 завершенным забастовкам в США в 1968-76 гг.}\label{fig:17.1}
\centering
%\includegraphics[scale=0.7]{fig.png}
\end{figure}

\noindent
Возьмем данные о длительности забастовок, на которых были основаны % которые были использованы в таких работах, как
такие работы как Kennan (1985), Джаггиа (1991c) и другие. Интересующая нас переменная --- это длительность забастовок в обрабатывающей промышленности США, измеренная в днях с начала забастовки. Выборка включает 566 завершенных (нецензурированных) наблюдений. Средняя продолжительность забастовки ($dur$) составляет 43.6 дней, медиана --- около 28 дней. Однако 90 дней спустя после начала наблюдения все еще продолжаются 88 забастовок.

Графически информацию о длительности забастовок можно представить в виде эмпирической \textbf{функции выживания}. На вертикальной оси рисунка \ref{fig:17.1} отображена доля забастовок, которые продолжаются по прошествии соответствующего количества дней на горизонтальной оси. Поскольку нас интересует только длительность забастовки, а не дата ее начала, на рисунке отсутствуют даты. Как и ожидалось, функция равна единице в нуле и затем постепенно падает до нуля, предполагая, что все забастовки должны рано или поздно закончиться.

Пусть переменная ($z$) показывает отклонение фактического выпуска от тренда, т.е. является индикатором состояния экономики. Тогда положительные значения $z$ означают, что экономика находится выше тренда, отрицательные --- ниже. Узнать, является ли средняя продолжительность забастовки проциклической (т.е. $\partial (dur) / \partial z > 0$) или же контрциклической (т.е. $\partial (dur) / \partial z < 0$) величиной, можно с помощью простой линейной регрессии переменной $\ln (dur)$ на $z$, оценив условное мат.ожидание $\ln (dur)$.

Однако предположим, что мы также хотим узнать вероятность того, что забастовка, которая длится $t$ дней, закончится в $t+1$ день, или же условную вероятность завершения забастовки как функцию от ее длительности, с учетом влияния $z$, построив биномиальную регрессию зависимой переменной, принимающей значения 0 или 1, на независимую переменную z. Тогда анализ выживаемости окажется более прямым и эффективным методом, который также позволяет работать с цензурированными данными. В следующем разделе мы рассмотрим статистические методы, применяемые в анализе выживаемости.




\section{Основные понятия}\label{sec:17.3}

\noindent
Длительность состояния является неотрицательной случайной величиной ($T$), зачастую дискретной в экономических данных. Для определения основных понятий мы будем рассматривать непрерывный случай, с последующим переходом к дискретному.


\subsection{Функции выживания, риска и кумулятивная функция риска}\label{sec:17.3.1}

\noindent
Обозначим \textbf{функцию распределения} величины $T$ как $F(t)$, \textbf{функцию плотности}  --- как $f(t) = dF(t) / dt$. Тогда вероятность того, что время жизни, или длительность состояния, окажется меньше t, равна
    \begin{align}
    \label{eq:17.1}
    F(t) &= \Pr[T \le t]\\
    &= \int^{t}_{0} f(s)ds. \notag
    \end{align}

Вероятность того, что время жизни окажется больше или равно $t$, называется \textbf{функцией выживания}, задаваемой следующим образом
    \begin{align}
    \label{eq:17.2}
    S(t) &= \Pr[T > t]\\
    &= 1 - F(t). \notag
    \end{align}

Согласно Kalbfleisch и Prentice (2002), определение функции распределения в (\ref{sec:17.1}) соответствует стандартному. Некоторые авторы (например, Lancaster (1990)) дают иное определение функции распределения: $F(t) = \Pr[T < t]$, и, следовательно, функции выживания: $S(t) = \Pr[T\ge t]$, поскольку функции риска заданы при условии $T\ge t$, а не при $T>t$.
Такие различия в спецификации будут иметь место в дискретном случае (\ref{sec:17.3.2}), непосредственно в момент перехода к другому состоянию. 

Функция выживания монотонно убывает от единицы до нуля, поскольку функция распределения монотонно возрастает с нуля. Если все индивиды, которые подвержены риску перейти в другое состояние, в итоге перейдут в него (т.е. исследуемое событие рано или поздно произойдет), то $S(\infty) = 0$, в противном же случае $S(\infty) > 0$, и распределение длительностей называется несобственным. Выборочное среднее наступившего события является интегралом от функции выживания: $\int^{\infty}_{0} S(u)du$. Чтобы получить такой результат, необходимо проинтегрировать по частям
    $$\int^{\infty}_{0}uf(u)du = \int^{\infty}_{0}udF(u) = uF(u)|^{\infty}_{0} - \int^{\infty}_{0}F(u)du.$$
Поскольку $F(\infty) = 1$, а $F(0) = 0$, то
    \begin{align}
    \label{eq:17.3}
    \mathrm{E}[T] = \int^{\infty}_{0} (1-F(u))du = \int^{\infty}_{0} S(u)du.
    \end{align}
То есть, средняя длительность равна площади под кривой выживания.

\textbf{Функция риска} % в учебнике МКП она называется интенсивностью отказов или коэффициентом смертности
также является одним из наиболее важных понятий в анализе времени жизни. Буквально, она означает мгновенную вероятность перехода к другому состоянию, при условии что объект дожил до некоторого момента времени $t$
    \begin{align}
    \label{eq:17.4}
    \lambda(t) &= \lim_{\Delta t \rightarrow 0} \frac{\Pr [t \le T \le t+ \Delta t | T \ge t] }{\Delta t}\\
    &= \frac{f(t)}{S(t)}. \notag
    \end{align}

    \begin{table}[!ht]\caption{\textit{Анализ выживаемости: Основные определения}}\label{tab:17.1}
    \begin{center}
\begin{tabular}{lccc}
\hline \hline
Функция             & Символ        & Определение                                                    & Отношение\\
\hline
Плотности           & $f(t)$        &                                                                & $f(t)=dF(t)/dt$\\
Распределения       & $F(t)$        & $\Pr[T \le t]$                                                 & $F(t)=\int_{0}^{t} f(s)ds$ \\
Выживания           & $S(t)$        & $\Pr[T>t]$                                                     & $S(t)=1-F(t)$ \\
Риска               & $\lambda(t)$  & $\lim_{h \rightarrow 0} \frac{\Pr[t \le T < t+h|T \ge t]}{h}$  & $\lambda(t)=f(t)/S(t)$ \\
Кумулятивного риска & $\Lambda(t)$  & $\int_{0}^{t}\lambda(s)ds$                                     & $\Lambda(t)=-\ln S(t)$ \\
\hline \hline
    \end{tabular}
    \end{center}
    \end{table}
Легко проверить, что функция риска равна производной от логарифма функции выживания,
    $$\lambda(t) = - \frac{d \ln(S(t))}{dt}.$$
Риск $\lambda(t)$ описывает распределение длительности $T$. В частности, интегрируя $\lambda(t)$ и зная $S(0)=1$, можно показать, что
    \begin{align}
    \label{eq:17.5}
    S(t) = \exp{ \left( -\int^{t}_{0}\lambda(u)du \right)}.
    \end{align}
В регрессионном анализе выживаемости объектом для исследования является условный коэффициент риска $\lambda(t|\mathbf{x})$, в то время как стандартный подход предполагает анализ условного среднего, $\mathrm{E}(T|\mathbf{x})$. Однако он не подходит для моделирования длительностей, если наблюдения являются цензурированными.

Наконец, еще одной важной характеристикой распределения является \textbf{кумулятивная}, или \textbf{интегральная функция риска}
    \begin{align}
    \label{eq:17.6}
    \Lambda(t) &= \int^{t}_{0} \lambda(s)ds \\
    &= - \ln(S(t)), \notag
    \end{align}
где для последнего преобразования используется уравнение \ref{eq:17.5}. Если $S(\infty)=0$, то $\Lambda(\infty)=\infty$. Кумулятивный риск представляет больший интерес, чем обычная функция риска, в силу более высокой точности оценивания.

Для любого $T$ можно показать, что преобразование $\Lambda(T)$ имеет нормированное экспоненциальное распределение, а $\ln\Lambda(T)$ --- распределение экстремальных значений, что позволяет проводить тесты на спецификацию модели (см. раздел 18.7.2
% UNCOMMENT AT THE END OF CHAPTER 18
%\ref{sec:18.7.2.}
).

Основные функции для неотрицательной с.в. $T$ описаны в Таблице \ref{tab:17.1}.

Иногда используются и другие функции, например, \textbf{преобразование Лапласа} $L(s)=\mathrm{E}[\exp{(-sT)}], s>0$, что является примером производящей функции моментов для с.в. $T$ при условии, что она положительна.


\subsection{Дискретные данные}\label{sec:17.3.2}

\noindent
Очень часто длительность события измеряется интервалом. Например, данные могут содержать информацию о неделе, когда наступил переход, но конкретные день и время могут быть неизвестны. В таких случаях говорят, что моменты перехода являются сгруппированными, и предполагается, что риск внутри интервала остается постоянным. К таким данным применяются модели риска в дискретном времени.

Для начала необходимо определить \textbf{функцию риска в дискретном времени} как вероятность перехода в момент $t_j,j=1,2,....$, при условии, что объект дожил до некоторого времени $t$:
    \begin{align}
    \label{eq:17.7}
    \lambda_j &= \Pr[T = t_j|T \ge t_j]\\
    &= f^{\mathrm{d}} (t_j)/S^{\mathrm{d}}(t_{j-}), \notag
    \end{align}
где надстрочный индекс d обозначает дискретный случай, и $S^\mathrm{d}(a\_)=\lim_{s\rightarrow a\_}{S^d(t_j)}$, поскольку формально $S^\mathrm{d}(t)$ равна $\Pr{[T>t]}$, а не $\Pr{[T\ge t]}$.
% ALT: adjustment made --- скорректирована?

\textbf{Функцию выживания в дискретном времени} можно рекурсивно вывести из функции риска как
    \begin{align}
    \label{eq:17.8}
    S^{\mathrm{d}} &= \Pr[T \ge t]\\
    &= \prod_{j|t_j \le t} (1 - \lambda_j). \notag
    \end{align}
Например, $\Pr{[T>t_2]}$ равна произведению вероятности, что переход не наступит в момент $t_1$, и вероятности, что он не наступит в $t_2$, при условии что объект дожил до момента $t_2$, то есть $\Pr{[T>t_2]}=(1-\lambda_1)\times(1-\lambda_2)$. Функция $S^\mathrm{d}(t)$ является убывающей ступенчатой функцией со скачками в моментах $t_j,j=1,2,....$
% ALT: разрывами?
% ENG: step function with jumps

\textbf{Кумулятивная функция риска в дискретном времени} равна
    \begin{align}
    \label{eq:17.9}
    \Lambda^{\mathrm{d}}(t) = \sum_{j|t_j \le t} \lambda_j.
    \end{align}

Используя \ref{eq:17.7}, получим, что вероятность наступления перехода в момент $t_j$ в дискретном случае равна $\lambda_j S^\mathrm{d}(t_j).$

Можно обобщить непрерывный и дискретный случаи. Функция выживания определена \textbf{произведением интегралов}, которое сокращается до регулярного произведения (\ref{eq:17.8}) в дискретном случае и экспоненты от регулярного интеграла (\ref{eq:17.5}) в непрерывном. См. Kalbfleisch и Prentice (2002, с. 10), или Lancaster (1990, с. 10-12).

Заметим, что дискретные данные могут возникать потому, что сам процесс, порождающий переходы, оказывается дискретным. Однако чаще мы все же встречаем непрерывные процессы с дискретными наблюдениями. Например, мы можем знать неделю или месяц наступления перехода, но не конкретный день или час. Такие данные иногда называют \textbf{сгруппированными}. К дискретному случаю можно перейти следующим образом. Разделим период наблюдения на $k + 1$ интервалов $[a_0,a_1), [a_1,a_2), ..., [a_{k-1},a_{k}), [a_{k},a_{\infty})$. Длительность $T=t_j$ обозначает переход к интервалу $[a_{j-1},a_{j})$, т.е. переход в момент $a_{j-1}$ или позже. Таким образом, дискретные данные часто возникают как результат группировки данных, где переходы моделируются в непрерывном времени, а затем делаются необходимые поправки для сгруппированных данных. Дальнейшее обсуждение представлено в Главе \ref{sec:17.10}.




\section{Цензурирование}\label{sec:17.4}

\noindent
Поскольку некоторые объекты невозможно наблюдать в течение всего времени их жизни, данные о выживаемости оказываются цензурированными. Для таких наблюдений известно лишь то, что время их жизни лежит в определенном интервале. Например, исследование о текущем состоянии безработицы может содержать информацию только о незавершенных событиях, то есть о людях, которые еще не нашли работу, поэтому мы не сможем понять, сколько времени требуется людям для поиска работы. Вместо этого, мы сможем узнать лишь, как долго тот или иной индивид уже является безработным.


\subsection{Механизмы цензурирования}\label{sec:17.4.1}

\noindent
На практики данные могут быть цензурированными справа, слева или в интервале. При \textbf{цензурировании справа}, или цензурировании сверху, мы можем наблюдать время жизни объекта с начального, или нулевого, момента вплоть до момента цензурирования $c$. К этому времени некоторые объекты уже прекратят существование, но некоторые нет, и о них известно лишь то, что момент смерти находится где-то в промежутке $(c,\infty)$. \textbf{Цензурирование слева}, или цензурирование снизу, предполагает, что объекты прекратили существование в промежутке $(0,c)$, но конкретный момент смерти неизвестен. Примером является классическая модель тобит, в которой информация о времени жизни некоторых объектов и времени их цензурирования неизвестна. \textbf{Интервальное цензурирование} возникает в том случае, когда время жизни объектов можно наблюдать только в определенном временном интервале, $[t^*_1,t^*_2)$.

В литературе по анализу выживаемости рассматривается преимущественно цензурирование справа. Однако даже при таком ограничении возможно применение различных видов цензурирования, таких как случайное цензурирование, цензурирование типа I и цензурирование типа II. %???

\textbf{Случайное}, или экзогенное, \textbf{цензурирование} означает, что у каждого объекта в выборке есть независимые друг от друга момент перехода $T^*_i$ и момент цензурирования $C^*_i$. Мы наблюдаем момент перехода $T^*_i$, если он произошел до цензурирования, и момент цензурирования $C^*_i$, если после. При этом известно, было ли время жизни объекта цензурировано или нет. Тогда наблюдения $(t_1,\delta_1), (t_2,\delta_2), ..., (t_N,\delta_N)$ являются реализациями случайных величин
        \begin{align}
        \label{eq:17.10}
        T_i&=\min(T^*_i, C^*_i),\\
        \delta_i&=\mathbf{1}[T^*_i<C^*_i], \notag
        \end{align}
где функция-индикатор $\mathbf{1}[A]$ равна 1, если событие $A$ происходит, и 0, если нет. То есть, $\delta_i$ равна 1, если момент перехода известен, и 0, если нет.\footnote{Здесь и далее: \textit{censoring indicator}, хотя по смыслу переменная является индикатором \textit{отсутствия} цензурирования}
Случайное цензурирование может возникать в ряде случаев, например, при случайно неудавшейся попытке завершить наблюдение, случайном прекращении участия некоторых объектов в исследовании, или завершении самого исследования. % ???

\textbf{Цензурирование типа I} применяется, когда время жизни превышает определенную фиксированную величину $t_{c_i}$. Например, можно проводить испытание выборки электроламп на надежность в течение 5000 часов. По ходу испытания некоторые лампы погаснут, и, таким образом, мы узнаем их моменты отказа, но некоторые продолжат гореть и по окончании испытания. В таком случае, их срок службы окажется цензурирован справа. Это особый случай экзогенного цензурирования со значением $C^*_i=t_{c_i}$. Классическая модель тобит является примером цензурирования типа I снизу для непрерывной с.в. на интервале $(-\infty,\infty)$.


\subsection{Независимое (неинформативное) цензурирование}\label{sec:17.4.2}

\noindent
Для того чтобы стандартные методы анализа выживаемости были применимы к цензурированным данным, механизм цензурирования должен быть \textbf{независимым (неинформативным)}. Это означает, что параметры распределения $C^*$ не должны содержать какой-либо информации о параметрах распределения длительности $T^*$. Тогда можно принять индикатор цензурирования $\delta$ как экзогенный, и в таком случае сам механизм можно не моделировать.

В цензурированных данных $(t,\delta)$ нецензурированные наблюдения встречаются с вероятностью
    $$\Pr{[T=t,\delta=1]}=\Pr{[T=t|\delta=1]}\times\Pr{[\delta=1]}.$$
Если механизм цензурирования является независимым, то $\Pr{[T=t|\delta=1]}=\Pr{[T=t]}$. Если же цензурирование неинформативно, то можно также избавиться от множителя $\Pr{[\delta=1]}$, поскольку он не содержит информации о параметрах распределения $T$. Аналогично, цензурированные наблюдения встречаются с вероятностью
    $$\Pr{[T=t,\delta=0]}=\Pr{[T\ge t|\delta=0]}\times\Pr{[\delta=0]},$$
где $\Pr{[T\ge t|\delta=0]}=\Pr{[T\ge t]}$, если цензурирование независимо, и можно убрать множитель $\Pr{[\delta=0]}$, если неинформативно. Объединив вышеперечисленное, получим, что интересующая нас плотность равна $\Pr{[T=t]}$, когда $\delta=1$, и $\Pr{[T\ge t]}$, когда $\delta=0$.

Если добавить регрессоры $\mathbf{x}$, может случиться так, что $T^*$ и $C^*$ будут изменяться вместе с одними и теми же переменными. Но даже в таком случае, важно лишь, чтобы параметры $C^*$ были неинформативны по отношению к параметрам $T^*$. Проще говоря, цензурирование не должно происходить потому, что объект подвержен большему или меньшему риску перехода, при условии что $\mathbf{x}$ неизменны.

\textbf{Цензурирование типа II} происходит, если из выборки размера $N$ совершает переход требуемое количество объектов $p$. В таком случае известны лишь первые $p$ наиболее коротких длительностей, а остальные $N-p$ оказываются цензурированы в момент $C^*_i=t_{(p)}$. Например, клиническое испытание может быть завершено после того, как умер $p$-ый пациент.

Случайное цензурирование, цензурирование типа I и II являются примерами независимого цензурирования. Более формальное описание представлено в Kalbfleisch и Prentice (2002, с. 194-196).


\section{Непараметрические модели}\label{sec:17.5}

\noindent
Методы непараметрического оценивания хорошо подходят для описания поведения функций выживания. Как показано в примере с длительностями забастовок, полезно знать безусловное матожидание функции риска или выживания до включения регрессоров в анализ.

В данном разделе мы представим оценки функций выживания, риска и кумулятивного риска в условиях независимого цензурирования. Мы не будем рассматривать оценку плотности распределения из-за трудностей, возникающих при цензурировании; более того, функции риска и выживания проще интерпретировать, чем плотность.

Регрессоры отсутствуют. Для того чтобы выяснить, как ведет себя ключевая переменная в различных условиях эксперимента или при разных уровнях воздействия,
% ENG: treatment regimes or levels of treatment
можно по отдельности построить непараметрические оценки для каждого значения объясняющей переменной и затем сравнить их. Однако в экономике обычно требуются более сложные структурные модели с регрессорами, представленные в разделах \ref{sec:17.6}-\ref{sec:17.10}.

Мы будем работать с дискретными данными, такими как данные таблицы выживания, используя для этого формулы из раздела \ref{sec:17.3.2}.
% Опечатка в Cameron: 17.3.3
Рассмотрим, например, $N_0$ индивидов определенного пола и возраста, которые последовательно наблюдаются в течение нескольких лет. Через год в группе останется $N_1$ индивидов, поскольку $N_1-N_0$ индивидов либо умрут, либо данные о них окажутся утеряны (цензурированы). Через два года в группе останется $N_2-N_1$ индивидов и т.д. Таким образом, на основе таких данных мы сможем построить дискретную функцию выживания без каких-либо предпосылок о распределениях параметров.


\subsection{Непараметрическое оценивание}\label{sec:17.5.1}

\noindent
Очевидно, оценкой функции выживания при отсутствии цензурирования будет единица за вычетом функции распределения. Тогда $\hat{S}(t)$ будет равна количеству объектов длительностью больше $t$, деленной на размер выборки $N$. Это ступенчатая функция со скачками в моментах перехода (отказа), см. рисунок \ref{fig:17.1}.
% ENG: step function with jumps
Альтернативная оценка, которая будет представлена в уравнении \ref{eq:17.13}, также сохраняет состоятельность при отсутствии цензурирования.

Обозначим $t_1<t_2<...<t_j<...<t_k$ как \textbf{дискретные моменты отказа} объектов в выборке размера $N$, $N\ge k$. Пусть $d_j$ равняется количеству длительностей, завершившихся в момент $t_j$. Поскольку данные дискретны, $d_j$ может превышать единицу. Время жизни некоторых объектов невозможно наблюдать полностью. Пусть $m_j$ равняется количеству наблюдений, цензурированных справа в момент $[t_j,t_{j+1})$. Предположим также, что механизм цензурирования является независимым, то есть все, что мы знаем о наблюдении, цензурированном в $[t_j,t_{j+1})$, это только то, что момент отказа для него наступил позже $t_j$. Объект находится под риском отказа, если он до сих пор не был завершен или цензурирован. Пусть $r_j$ равняется количеству наблюдений, находящихся под риском в момент $t_{j-}$, то есть непосредственно перед $t_j$. Тогда $r_j=(d_j+m_j)+...+(d_k+m_k)=\sum_{l|l\ge j} (d_l+m_l)$. Заметим, что $r_1=N$. Таким образом,
        \begin{align}
        \label{eq:17.11}
        d_j&=\# \textrm{ объектов, завершенных в момент } t_j,\\
        m_j&=\# \textrm{ объектов, цензурированных в момент } [t_j,t_{j+1}), \notag \\
        r_j&=\# \textrm{ объектов под риском в момент } t_{j-}=\sum_{l|l\ge j} (d_l+m_l).\notag
        \end{align}

Очевидно, оценкой функции риска будет количество наблюдений, завершенных в момент $t_j$, деленное на количество наблюдений с риском отказа в момент $t_{j-}$, поскольку в дискретном случае $\lambda_j=\Pr{[T=t_j|T\ge t_j]}$
        \begin{align}
        \label{eq:17.12}
        \hat{\lambda_j}=\frac{d_j}{r_j}.
        \end{align}

Дискретная функция выживания определена в уравнении \ref{eq:17.8}. \textbf{Оценка Каплан-Мейера}, или множительная оценка функции выживания, является выборочным аналогом
        \begin{align}
        \label{eq:17.13}
        \hat{S}(t)=\prod_{j|t_j\le t} (1-\hat{\lambda}_j) = \prod_{j|t_j\le t} \frac{r_j-d_j}{r_j}.
        \end{align}

    \begin{table}[!htbp]\caption{\textit{Оценка коэффициента отказов и функции выживания: пример}${}^a$}\label{tab:17.2}
    \begin{tabularx}{\textwidth}{X X X X c c c}
    \hline \hline
$j$ &$r_j$  &$d_j$  &$m_j$  &$\hat{\lambda}_j=d_j/r_j$ &$\hat{\Lambda}(t_j)$   &$\hat{S}(t_j)$\\
    \hline
1   &80     &6      &4      &$6/80$                    &$6/80$                  &$1-6/80$\\
2   &70     &5      &3      &$5/70$                    &$6/80+5/70$             &$(1-5/70)\times(1-6/80)$\\
3   &62     &2      &1      &$2/62$                    &$\hat{\lambda}_2+2/62$  &$\hat{S}(t_2)\times(1-2/62)$\\
4   &-      &-      &-      &-                                  &               &\\
    \hline \hline
% FOOTNOTE --- CHANGE TO MINIPAGE?
\multicolumn{7}{l}{${}^a$ \scriptsize{В момент $t_j$, $r_j$ равно количеству наблюдений с риском, $d_j$ --- количеству смертей (отказов), $m_j$ --- количество}}\\[-0.15cm]
\multicolumn{7}{l}{\hspace{0.3cm}\scriptsize{пропущенных (цензурированных) наблюдений; $\hat{\lambda}_j$ является оценкой функции риска, $\hat{\Lambda}_j$ --- оценкой функции}}\\[-0.15cm]
\multicolumn{7}{l}{\hspace{0.3cm}\scriptsize{кумулятивного риска, $\hat{S}(t_j)$ --- оценкой функции выживания.}}
    \end{tabularx}
    \end{table}
Это убывающая ступенчатая функция со скачками в каждом моменте отказа. Оценка Каплан-Мейера может быть представлена как непараметрическая оценка ММП (см. Kalbfleisch и Prentice, 2002, стр. 14-16).

При отсутствии цензурирования $\hat{S}(t)$ упрощается до $\hat{S}(t)=r/N$, то есть количеству наблюдений с риском, деленному на размер выборки, что равняется единице за вычетом функции распределения. Чтобы показать это, заметим, что $r_j-d_j=r_{j+1}$, если $m_j=0$, то есть количество наблюдений с риском за вычетом количества смертей в момент $j$ равно количеству наблюдений с риском в момент $j+1$. Тогда можно преобразовать выражение \ref{eq:17.13} как $\hat{S}(t)=\prod_{j|t_j\le t} r_{j+1}/r_j$ и затем упростить его до $r/r_1$, где $r_1=N$.

Дискретная функция кумулятивного риска определена в уравнении \ref{eq:17.9}. \textbf{Оценка Нельсон-Аалена} функции кумулятивного риска, является выборочным аналогом
        \begin{align}
        \label{eq:17.14}
        \hat{\Lambda}(t)=\sum_{j|t_j\le t} \hat{\lambda}_j=\sum_{j|t_j\le t} \frac{d_j}{r_j}.
        \end{align}
С помощью данной оценки можно также получить оценку функции выживания $\tilde{S}(t_j)=\exp{(-\hat{\Lambda}(t))}$, используя равенство $S(t)=\exp{(-\Lambda(t))}$ из непрерывного случая.

В качестве примера, представим, что изначально было 80 наблюдений, из которых 6 завершились в момент $t_1$,  4 были цензурированы в момент $[t_1,t_2)$, 5 завершились в $t_2$, 3 были цензурированы в $[t_2,t_3)$, 2 завершились в $t_3$, 1 было цензурировано в $[t_3,t_4)$ и т.д. Тогда можно найти оценки функций кумулятивного риска и выживания для $t\le t_3$, которые представлены в таблице \ref{tab:17.2}.

\textbf{Набор идентичных данных}
% ENG: Tied Data
возникает, когда в определенный момент времени происходит сразу несколько отказов. Принято считать, что такие данные являются результатом группировки данных, а не самого процесса, порождающего их. Оценка риска $\hat{\lambda}_j=d_j/r_j$ предполагает, что все смерти наступают одновременно в момент $t_j$. На самом деле, смерти могут происходить постепенно в интервале $[t_j,t_{j+1})$, равно как и цензурирование. Тогда $r_j$ будет в среднем переоценивать количество объектов, находящихся под риском в интервале $[t_j,t_{j+1})$. В качестве стандартной поправки в анализе таблицы выживания $\hat{\lambda}_j=d_j/r_j$ заменяют на $d_j/(r_j-m_j/2)$. Аналогичные изменения делаются и для $\hat{S}(t)$, $\hat{\Lambda}(t)$ и т.д. Возможны также и другие поправки.

Основные таблицы и графики оценки Каплан-Мейера встроены в большинство статистических программ для анализа выживаемости. Пример подобных расчетов для данных о забастовках представлен в таблице \ref{tab:17.3}, которая дополняет, таким образом, рисунок \ref{fig:17.1}.

    \begin{table}[!htbp]\caption{\textit{Продолжительность забастовок: оценки Каплан-Мейера}}\label{tab:17.3}
    \begin{center}
\begin{tabular}{ccccc}
    \hline \hline
                &\textbf{Всего}        &\textbf{Количество} &\textbf{Функция}    &\textbf{Стандартная}\\
\textbf{День}   &\textbf{на начало}    &\textbf{отказов}    &\textbf{выживания}  &\textbf{ошибка}\\
    \hline
1               &566                        &10                 &0.9823                     &0.0055\\
2               &556                        &21                 &0.9452                     &0.0096\\
3               &535                        &16                 &0.9170                     &0.0116\\
4               &519                        &17                 &0.8869                     &0.0133\\
5               &502                        &18                 &0.8551                     &0.0148\\
6               &484                        &9                  &0.8392                     &0.0154\\
7               &475                        &12                 &0.8180                     &0.0162\\
8               &463                        &12                 &0.7968                     &0.0169\\
\vdots          &\vdots                     &\vdots             &\vdots                     &\vdots\\
13              &411                        &11                 &0.7067                     &0.0191\\
14              &400                        &11                 &0.6873                     &0.0195\\
    \hline \hline
\end{tabular}
    \end{center}
    \end{table}


\subsection{Доверительные интервалы для непараметрических оценок}\label{sec:17.5.2}

\noindent
Оценка функции риска $\hat{\lambda}_j=\frac{d_j}{r_j}$ не является непрерывной, при этом разрывы становятся тем больше, чем меньше $r_j$ по отношению к $d_j/r_j$. Поэтому перед построением графиков зависимости от времени иногда полезно проводить сглаживание оценок риска с помощью непараметрических методов, рассмотренных в разделе 9.5% \ref{sec:9.5}
.

Функции выживания и кумулятивного риска являются более гладкими, и зависимость от времени принято изображать на графике вместе с доверительными интервалами, которые отражают изменчивость выборки. Существует несколько способов оценки доверительных интервалов. Мы будем использовать формулы, которые применяются при расчетах в STATA.

Для оценки Каплан-Мейера функции выживания обычно используют оценку дисперсии Greenwood
    $$\widehat{\Var}[\hat{S}(t)]=\hat{S}(t)^2\sum_{j|t_j\le t}\frac{d_j}{r_j(r_j-d_j)}.$$
Полученные доверительные интервалы зачастую рассчитываются на основе $\ln(-\ln\hat{S}(t))$, а не $\hat{S}(t)$, поскольку такое преобразование гарантирует, что они находятся в пределах допустимых значений
% ALT: области определения?
функции выживания, то есть между нулем и единицей. Получаем $100(1-\alpha)\%$ доверительный интервал
        \begin{align}
        \label{eq:17.15}
        S^{\mathrm{d}}(t)\in (\hat{S}(t)\exp{}^{(-z_{\alpha/2} \hat{\sigma}(t))},\hat{S}(t)\exp{}^{(z_{\alpha/2} \hat{\sigma}(t))}),
        \end{align}
где $\sigma(t)$ обозначает стандартное отклонение $\ln(-\ln\hat{S}(t))$, которое рассчитывается по формуле
    $$\hat{\sigma}^2_s(t)=\frac{\sum_{j|t_j\le t} d_j/(r_j(r_j-d_j))}{\left[ \sum_{j|t_j\le t} \ln ((r_j-d_j)/d_j)\right] ^2}.$$

    \begin{table}[!htbp]\caption{\textit{Распределения Вейбулла и экспоненциальное: плотность, функция распределения, функция выживания, риска, кумулятивного риска, среднее и дисперсия}}\label{tab:17.4}
    \begin{center}
\begin{tabular}{lll}
\hline \hline
\textbf{Функция}&\textbf{Экспоненциальное}&\textbf{Weibull}\\
\hline
$f(t)$          &$\gamma\exp{(-\gamma t)}$  &$\gamma \alpha t^{\alpha-1}\exp{(-\gamma t^{\alpha})}$\\
$F(t)$          &$1-\exp{(-\gamma t)}$      &$1-\exp{(-\gamma t^{\alpha})}$\\
$S(t)$          &$\exp{(-\gamma t)}$        &$\exp{(-\gamma t^{\alpha})}$\\
$\lambda(t)$    &$\gamma$                   &$\gamma\alpha t^{\alpha-1}$\\
$\Lambda(t)$    &$\gamma t$                 &$\gamma t^{\alpha}$\\
$\E(T)$ &$\gamma^{-1}$              &$\gamma^{-1/\alpha}\Gamma(\alpha^{-1}+1)$\\
$\Var(T)$ &$\gamma^{-2}$              &$\gamma^{-2/\alpha}[\Gamma(2\alpha^{-1}+1)-[\Gamma(\alpha^{-1}+1)]^2]$\\
$\gamma,\alpha$ &$\gamma>0$                 &$\gamma>0,\alpha>0$\\
\hline \hline
\end{tabular}
    \end{center}
    \end{table}

Для оценки кумулятивного риска Нельсон-Аалена одной из оценок дисперсии будет
    $$\widehat{\Var}[\hat{\Lambda}(t)]=\sum_{j|t_j\le t} \frac{d_j}{r^2_j}.$$
С помощью преобразования $\ln\hat{\Lambda}(t)$ получаем $100(1-\alpha)\%$ доверительный интервал для кумулятивного риска
        \begin{align}
        \label{eq:17.16}
        \Lambda(t)\in [\hat{\Lambda}(t)\exp{(-z_{\alpha/2}\hat{\sigma}_{\Lambda}(t))}, \hat{\Lambda}(t)\exp{(z_{\alpha/2}\hat{\sigma}_{\Lambda}(t))}],
        \end{align}
где $\hat{\sigma}_{\Lambda}(t)$ обозначает стандартное отклонение для $\ln\hat{\Lambda}(t)$, которое рассчитывается по формуле
$$\hat{\sigma}^2_{\Lambda}(t)=\widehat{\Var}[\hat{\Lambda}(t)]/[\hat{\Lambda}(t)^2].$$




\section{Параметрические модели регрессии}\label{sec:17.6}

\noindent
В данном разделе мы опишем свойства двух распределений, составляющих основу параметрических моделей. Также мы рассмотрим некоторые стандартные модели регрессии для анализа времени жизни.


\subsection{Экспоненциальное распределение и распределение Вейбулла}\label{sec:17.6.1}

\noindent
Исходной точкой для параметрического анализа является экспоненциальное распределение, поскольку чистый точечный процесс Пуассона генерирует длительности, распределенные экспоненциально, см. Lancaster (1990 стр. 86). В силу свойства отсутствия памяти \textbf{экспоненциальное распределение длительностей} предполагает постоянный во времени коэффициент риска $\gamma$. Из \ref{eq:17.5} следует, что $S(t)=\exp{(-\int^t_0\gamma d u)}=\exp{(-\gamma t)}$. Плотность равна $f(t)=-S'(t)=\gamma\exp{(-\gamma t)}$, и кумулятивный риск $\Lambda(t)=-\ln S(t)=\gamma t$ линеен по времени.

Применение экспоненциального распределения ограничено наличием единственного параметра, поэтому в эконометрике часто используют обобщение, называемое распределением Вейбулла. В таблице \ref{tab:17.4} представлены моменты, плотность и другие функции для экспоненциального распределения и распределения Вейбулла, где первое является частным случаем второго при $\alpha=1$. Функция $\Gamma(\cdot)$ в таблице \ref{tab:17.5} является гамма функцией.

    \begin{table}[!htbp]\caption{\textit{Стандартные параметрические модели и их функции риска и выживания}${}^a$}\label{tab:17.5}
    \begin{center}
\begin{tabular}{llll}
\hline \hline
\textbf{Параметрическая модель} &\textbf{Функция риска}     &\textbf{Функция выживания}                 &\textbf{Тип}\\
\hline
Экспоненциальное    &$\gamma$                               &$\exp{(-\gamma t)}$                        &PH, AFT\\
Вейбулла            &$\gamma\alpha t^{\alpha-1}$            &$\exp{(-\gamma t^{\alpha})}$               &PH, AFT\\
Вейбулла обобщенное &$\gamma\alpha t^{\alpha-1}S(t)^{-\mu}$ &$[1-\mu\gamma t^{\alpha}]^{1/\mu}$         &PH\\
Гомперца            &$\gamma\exp{(\alpha t)}$               &$\exp{(-(\gamma/\alpha)(e^{\alpha t}-1))}$ &PH\\
Лог-нормальное      &$\frac{\exp{(-(\ln t-\mu)^2/2\sigma^2)}}{t\sigma\sqrt{2\pi[1-\Phi((\ln t -\mu)/\sigma)]}}$ &$1-\Phi((\ln t-\mu)/\sigma)$    &AFT\\
Лог-логистическое   &$\alpha\gamma^{\alpha}t^{\alpha-1}/[(1+(\gamma t)^{\alpha})]$                              &$1/[1+(\gamma t)^{\alpha}]$    &AFT\\
Гамма               &$\frac{\gamma(\gamma t)^{\alpha-1}\exp{[-(\gamma t)]}}{\Gamma(\alpha)[1-I(\alpha,\gamma t)]}$&$1-I(\alpha,\gamma t)$   &AFT\\
\hline \hline
\multicolumn{4}{c}{${}^a$ \scriptsize{Все параметры положительны, за исключением $-\infty<\alpha<\infty$ для модели Гомперца.}}
\end{tabular}
    \end{center}
    \end{table}

В модели Вейбулла функция риска $\lambda(t)=\gamma\alpha t^{\alpha-1}$ монотонно возрастает при $\alpha>1$ и убывает при $\alpha<1$. Это является частным случаем класса моделей пропорциональных рисков (PH),
% ALT: семейства
см. раздел \ref{sec:17.7.1}, где $\lambda(t)$ может быть разложена на две компоненты: базовую, зависящую только от $\lambda_0(t)$ и $t$, и относительную (например, $\gamma$), которая зависит только от набора ковариат (независимых переменных).

На рисунке \ref{fig:17.2} представлены свойства распределения Вейбулла при $\gamma=0.01$ и $\alpha=1.5$. Функция плотности скошена вправо, что вполне характерно для данных по длительностям. Форма функции выживания одинакова для большого количества распределений, что затрудняет сравнительный визуальный анализ различных оценок функции выживания. Функция риска является возрастающей, поскольку для данного примера $\alpha>1$. При этом для разных параметрических моделей могут быть разные формы функции риска, включая монотонно убывающие и возрастающие, прямые и обратные U-образные формы.

\begin{figure}[ht!]\caption{Распределение Вейбулла: плотность, функции выживания, риска и кумулятивного риска при $\gamma=0.01$ и $\alpha=0.5$}\label{fig:17.2}
\centering
%\includegraphics[scale=0.7]{fig.png}
\end{figure}

На практике оценка функции риска часто оказывается неточной, в особенности в правом хвосте. Оценка кумулятивного риска $\Lambda(t)$ является более точной и, таким образом, позволяет определять некоторые различия между моделями. Еще точнее можно оценить зависимость $\ln\Lambda(t)$ от времени, поскольку в модели Вейбулла $\ln\Lambda(t)=\ln\gamma+\alpha\ln t$ линеен по $\ln t$ с наклоном $\alpha$.


\subsection{Некоторые параметрические модели}\label{sec:17.6.2}

\noindent
Наиболее популярные параметрические модели основаны на распределениях Вейбулла, Гомперца, экспоненциальном, лог-нормальном, лог-логистическом и гамма. Их функции риска и выживания представлены в таблице \ref{tab:17.5}.

Для гамма распределения $\Gamma(\alpha)=\int^{\infty}_{0}e^{-t}t^{\alpha-1}dt$ называется \textbf{гамма функцией}, а $I(\alpha,\gamma t)$ --- \textbf{неполной гамма функцией}, где $I(\alpha,x)=\int^{x}_{0}e^{-t}t^{\alpha-1}dt/\Gamma(\alpha)$, $0<I(\alpha,x)<1$.

Обобщенная модель Вейбулла была предложена авторами Mudholkar, Srivastava и Kollia (1996). За счет добавления параметра $\mu$ функция риска становится более гибкой, что делает модель менее ограничительной. При этом модель Вейбулла является предельным случаем обобщенного варианта при $\mu\rightarrow 0$. Из таблицы \ref{tab:17.5} следует, что
    $$\ln\lambda(t)=\ln(\gamma\alpha)+(\alpha-1)\ln t-\mu\ln S(t).$$
Поскольку $\partial S(t)/\partial t<0$, правая часть уравнения возрастает по $t$ при $\mu>0$ и $\alpha>1$. Если $\alpha\le1$ и $\mu<0$, то функция риска монотонно убывает. Если $\alpha>1$ и $\mu<0$, то риск состоит из двух частей, одна из которых является возрастающей функцией, а другая --- убывающей, что в сумме позволяет кривой риска иметь одномодальную, или U-образную, форму.

Модель Гомперца похожа на вейбулловскую тем, что риск может как возрастать (при $\alpha>0$), так и убывать (при $\alpha<0$), или же быть постоянным, как в экспоненциальной модели (при $\alpha=0$). Распределение Гомперца хорошо подходит для анализа данных смертности и поэтому используется больше в биостатистике, чем в эконометрике.

Для лог-нормального распределения риск имеет обратную U-образную форму и сперва возрастает, а затем убывает по $t$. Аналогично, для лог-логистического распределения при $\alpha>1$. В силу данного свойства эти модели больше подходят для анализа данных по длительностям, чем экспоненциальная модель или модели Вейбулла и Гомперца.

Другие параметрические модели включают распределения Rayleigh и Makeham, обратную гауссовскую кусочно-непрерывную модель рисков и обобщенную гамма модель (Lawless, 1982), частными случаями которой являются гамма и вейбулловская модели. Детали представлены в Kalbfleisch и Prentice (1980, глава 3) и Lancaster (1990, глава 3).

В основном, распределения содержат два параметра. % Описаны/описываются двумя параметрами
\textbf{Регрессоры} включены как $\gamma=\exp{(\mathbf{x}'\beta)}$ с константой $\alpha$, или $\mu=\mathbf{x}'\beta$ с константой $\sigma^2$ для лог-нормального распределения.

Основные трудности в параметрическом моделировании связаны с тем, что состоятельность оценок зависит от правильной спецификации модели, выбор которой достаточно широк. Большинство моделей относятся либо к PH моделям (первые четыре в таблице \ref{tab:17.5}), либо к моделям ускоренной жизни (первые две и последние три в таблице \ref{tab:17.5}). Модель Вейбулла принадлежит обоим классам и поэтому часто применяется в экономических приложениях. Другой широко используемой моделью, в частности, для анализа большого количества наблюдений, является кусочно-постоянная модель рисков, которая представляет собой частный случай модели PH.


\subsection{Оценивание ММП}\label{sec:17.6.3}

\noindent
Мы будем изучать полностью параметрический анализ методами максимального правдоподобия и наименьших квадратов при независимом или неинформативном цензурировании, используя при этом формулы для непрерывного случая, поскольку параметрические модели основаны на непрерывных распределениях. Предположим, что регрессоры постоянны во времени, оставив случай с меняющимися регрессорами для рассмотрения в разделе \ref{sec:17.9}.

Пусть $T^*$ обозначает нецензурированные длительности, распределенные условно с плотностью $f(t|\mathbf{x},\theta)$, где $\theta$ является вектором размерности $q\times1$, а $\mathbf{x}$ --- набор регрессоров, которые могут меняться между объектами, но постоянны между состояниями для каждого объекта. При наличии цензурированных наблюдений анализ становится несколько сложнее. Так, длительности незавершенного события $t$ теперь соответствует новая переменная-индикатор цензурирования, которое неинформативно по предположению.

Из раздела \ref{sec:17.4.2} следует, что воздействие аналогично воздействию в модели тобит. Вклад нецензурированных наблюдений в функцию правдоподобия будет составлять $f(t|\mathbf{x},\mathbf{\theta})$. О наблюдениях, цензурированных справа, мы знаем только то, что их время жизни превышает $t$, а значит их вклад будет равен
        \begin{align}
        \Pr[T>t]&=\int^{\infty}_{t}f(u|\mathbf{x},\mathbf{\theta})du\notag\\
                &=1-F(t|\mathbf{x},\theta)=S(t|\mathbf{x},\mathbf{\theta}),\notag
        \end{align}
где $S(\cdot)$ --- функция выживания. Функция плотности для $i$-го наблюдения может быть записана как
    $$f(t_i|\mathbf{x}_i,\mathbf{\theta})^{\delta_i}S(t_i|\mathbf{x}_i,\mathbf{\theta})^{1-\delta_i},$$
где $\delta_i$ --- индикатор цензурирования справа, равный
        $$\delta_i=\begin{cases}
            1 \textrm{ (нет цензурирования)},\\
            0 \textrm{ (цензурирование справа)}.
                   \end{cases}$$
Взяв логарифмы и просуммировав, получим, что  оценка ММП $\hat{\theta}$ максимизирует логарифм правдоподобия
        \begin{align}
        \label{eq:17.17}
        \ln \mathrm{L}(\mathbf{\theta})=\sum^{N}_{i=1}[\delta_i\ln f(t_i|\mathbf{x}_i,\mathbf{\theta})+(1-\delta_i)\ln S(t_i|\mathbf{x}_i,\mathbf{\theta}a)],
        \end{align}
где предполагается, что объекты $i$ независимы. Первое слагаемое относится к завершенным событиям, а второе --- к цензурированным справа. Поскольку $\ln S(t) = \Lambda(t)$, а $\ln f(t) = \ln(\lambda(t)S(t)) = \ln \lambda(t) + \ln S(t)$, логарифм функции правдоподобия можно записать в терминах простой и интегральной функций риска:
        \begin{align}
        \label{eq:17.18}
        \ln \mathrm{L}(\mathbf{\theta})=\sum^{N}_{i=1}[\delta_i\ln\lambda(t_i|\mathbf{x}_i,\mathbf{\theta})+\Lambda(t_i|\mathbf{x}_i,
        \mathbf{\theta})].
        \end{align}
Результат является полезным в случае, если параметрическая модель изначально задана через функцию риска, а не через функцию плотности.

Далее применяется стандартная теория. Оценка ММП распределена как
$$\bm{\hat{\theta}}\thicksim\!\!\!\!\!^{{}^{a}}\hspace{0.1cm}\mathcal{N}\left[\bm{\theta},(-\E[\partial^2\ln\mathrm{L}/\partial\bm{\theta}\partial\bm{\theta}'])^{-1} \right],$$
% в книжке не по центру
если функция плотности специфицирована верно, см. раздел 5.7.3. % \ref{sec:5.7.3}
Если спецификация неверна, то оценка ММП не является состоятельной, независимо от наличия цензурированных наблюдений. Исключение составляет экспоненциальная модель при отсутствии цензурирования, для состоятельности оценок которой требуется лишь правильная спецификация условного матожидания, см. раздел 5.7.3. % \ref{sec:5.7.3}
Однако при наличии цензурирования неверная спецификация приводит к несостоятельности оценок даже в этой модели. Неустойчивость является главным недостатком параметрического подхода, как и в модели тобит.

Метод максимального правдоподобия позволяет анализировать и другие типы цензурированных данных. При \textbf{цензурировании слева} максимальная длительность состояния составляет $t$, и вклад таких наблюдений равен $\Pr[T^*<t]=\int^{t}_{0}f(s|\mathbf{x},\mathbf{\theta})ds=F(t|\mathbf{x},\mathbf{\theta}).$

При \textbf{интервальном цензурировании} длительность наблюдений лежит в интервале $[t_a,t_b)$, и их вклад в функцию правдоподобия составляет $\Pr[t_a\le T^*<t_b]=\int^{t_b}_{t_a}f(s|\mathbf{x},\mathbf{\theta})ds=S(t_a|\mathbf{x},\mathbf{\theta})-S(t_b|\mathbf{x},\mathbf{\theta}).$

В экономике данные по длительностям зачастую цензурированы интервалом. Например, данные по безработицы могут быть сгруппированы по неделям или месяцам при непрерывном распределении, например, распределении Вейбулла. Обычно предполагается, что эффект интервального цензурирования достаточно незначителен, что им можно пренебречь. Например, для индивида, являющегося безработным в течение двух месяцев, но нашедшего работу на третьем месяце, длительность состояния будет составлять ровно три месяца вместо интервала от двух до трех.


\subsection{Компоненты ММП}\label{sec:17.6.4}
% ALT: элементы

\noindent
Учитывая многообразие типов данных, в частности, то, что наблюдения могут быть завершенными, урезанными или цензурированными, для ММП параметрически специфицированной модели требуется правильная запись функции правдоподобия. (Lancaster (1979) показывает различные варианты выражений правдоподобия, соответствующие трем типам данных по длительностям безработицы). Каждый тип наблюдений несет определенный вклад в функцию правдоподобия так, что итоговая функция представляет собой произведение соответствующих элементов, таких как (см. Klein и Moeschberger, 1997, стр. 66):
        \begin{align}
        &\textrm{завершенные длительности: }                                &&f(t),\notag\\
        &\textrm{урезанные слева в момент } t_L \textrm{ } (t\ge t_L):      &&f(t)/S(t_L),\notag\\
        &\textrm{цензурированные слева в момент } t_{C_L}:                  &&1-S(t_{C_L}),\notag\\
        &\textrm{цензурированные справа в момент } t_{C_R}:                 &&S(t_{C_R}),\notag\\
        &\textrm{урезанные справа в момент } t_{C_R} \textrm{ } (t\le t_R): &&f(t_R)/[1-S(t_R)],\notag\\
        &\textrm{цензурированные в интервале в моментах } t_{C_L}, t_{C_R}: &&S(t_{C_L})-S(t_{C_R}).\notag
        \end{align}


\subsection{Пример ММП Вейбулла}\label{sec:17.6.5}

\noindent
Распределение Вейбулла детально представлено в \ref{sec:17.6.1}. Функция риска равна $\lambda(t)=\gamma\alpha t^{\alpha-1}$, где $\alpha>0$ и $\gamma>0$.

Существуют различные способы добавления регрессоров в модель. Обычно предполагают, что $\gamma=\exp{(\mathbf{x}'\be)}$, что гарантирует, что $\gamma>0$, при этом $\alpha$ не зависит от переменных. (Некоторые программы предполагают, что $\gamma=\exp{(-\mathbf{x}'\be)}$, что приводит к противоположным знакам оценок $\be$). Тогда
        \begin{align}
        \ln f(t|\mathbf{x},\be,\alpha)&=\ln[\exp{(\mathbf{x}'\be)}\alpha t^{\alpha-1}\exp{(-\exp{(\mathbf{x}'\be)}t^{\alpha})}]\notag\\
                                        &=\mathbf{x}'\be+\ln\alpha+(\alpha-1)\ln t-\exp{(\mathbf{x}'\be)t^{\alpha}}\notag
        \end{align}
и
        \begin{align}
        \ln S(t|\mathbf{x},\be,\alpha)&=\ln[\exp{(-\exp{(\mathbf{x}'\be)}t^{\alpha})}]\notag\\
                                        &=-\exp{(\mathbf{x}'\be)t^{\alpha}}.\notag
        \end{align}
Функция правдоподобия (\ref{eq:17.17}) принимает вид
        \begin{align}
        \label{eq:17.19}
        \ln \mathrm{L}=\sum_i[\delta_i\{\mathbf{x}_i'\beta+\ln\alpha+(\alpha-1)\ln t_i-\exp{(\mathbf{x}_i'\be)t_i^{\alpha}}\}-(1-\delta_i)\exp{(\mathbf{x}_i'\be)t_i^{\alpha}}].
        \end{align}
Условия первого порядка для $\be$ и $\alpha$ равны
        \begin{align}
        \frac{\partial\ln\mathrm{L}}{\partial\be}&=\sum_i(\delta_i-\exp{(\mathbf{x}_i'\be)t_i^{\alpha}})\mathbf{x}_i=\mathbf{0},\notag\\
        \frac{\partial\ln\mathrm{L}}{\partial\alpha}&=\sum_i\delta_i(1/\alpha+\ln t_i)-\ln t_i\exp{(\mathbf{x}_i'\be)t_i^{\alpha}}=0.\notag
        \end{align}
При этом для состоятельности требуются достаточно строгие предположения. Например, даже при отсутствии цензурирования для $\E[\partial\ln\mathrm{L}/\partial\be]=0$ необходимо, чтобы $\E[T^{\alpha}|\mathbf{x}]=\exp{(\mathbf{x}'\be)}.$


\subsection{Интерпретация оценок модели}\label{sec:17.6.6}

\noindent
Стандартный способ интерпретации коэффициентов в нелинейных моделях регрессии заключается в оценке их влияния на условное матожидание. Если $\gamma=\exp{(\mathbf{x}'\be)}$, то из таблицы \ref{tab:17.4} следует, что матожидание завершенных наблюдений в модели Вейбулла равно $\E[T^*|\mathbf{x}]=\exp{(-\mathbf{x}'\be/\alpha)}\Gamma(\alpha^{-1}+1)=\exp{(-\mathbf{x}'\be/\alpha)}\Gamma(\alpha^{-1})/\alpha.$ Значит, можно найти ожидаемую длительность при разных значениях $\mathbf{x}$. Например, можно предсказать продолжительность (завершенного) состояния безработицы индивида определенного возраста, пола и уровня образования.

Однако параметрические модели регрессии позволяют предсказывать и другие характеристики, кроме матожидания. Например, можно оценить, какую долю в общей продолжительности безработицы занимают длительности тех объектов, которые находились в безработном состоянии дольше определенного времени, или же узнать долю длительностей индивидов определенного социально-экономического статуса. % ALT: суммарной? Язык свернешь!!
При этом модели времени жизни построены не только на анализе независимых переменных, но и на предположении о форме функции риска, поскольку экономическая теория может заранее предсказывать ее поведение.

Несмотря на множество различных вариантов интерпретации, обычно останавливаются % рассматривают?
на вейбулловском коэффициенте риска, $\lambda(t)=\gamma\alpha t^{\alpha-1}$, его изменениях во времени и с регрессорами. Как уже упоминалось в разделе \ref{sec:17.3.2}, коэффициент возрастает при $\alpha>1$ и убывает при $\alpha<1$, поэтому основной интерес представляет тестирование гипотезы, что $\alpha=1$. Изменение регрессоров имеет мультипликативный эффект на функцию риска так, что
        $$d\lambda(t)/d\mathbf{x}=\exp{(\mathbf{x}'\be)}\alpha t^{\alpha-1}\be=\lambda(t)\be.$$
Таким образом, при $\beta_j>0$ рост $x_j$ приводит к повышению риска и снижению ожидаемой длительности. % Кэмерон два раза одно и то же пишет.. (положительный коэффициент $\beta_j$ предполагает рост коэффициента риска с ростом значений регрессоров $\mathbf{x}$.)


\subsection{Оценивание с помощью МНК}\label{sec:17.6.7}

\noindent
Помимо ММП оценивание полностью параметрических моделей можно проводить с помощью МНК, аналогично оцениванию цензурированной модели тобит. Заметим, что метод редко применяется на практике, так как по-прежнему требует корректной спецификации функции плотности, но при этом оценки коэффициентов менее эффективны, чем в ММП.

Рассмотрим экспоненциальную модель времени жизни. Поскольку $\E[T|\mathbf{x}]=1/\gamma=\exp{(-\mathbf{x}'\be)}$, то регрессия НМНК $t_i$ на $\exp{(-\mathbf{x}_i'\be)}$ дает состоятельную оценку $\be$. Экспоненциальная модель времени жизни может быть записана иначе, в виде линейной регрессии $\ln t=\mathbf{x}'\be+u$, где $u$ имеет распределение экстремальных значений (см. раздел \ref{sec:17.7.2}). Тогда $\E[\ln T|\mathbf{x}]=\mathbf{x}'\be-c$, где $c \simeq 0.5722$ --- константа Эйлера. Таким образом, $\be$ является состоятельной оценкой коэффициентов линейной регрессии $\ln t_i$ на $\mathbf{x_i}$. При цензурировании справа требуется найти сами моменты цензурирования, что также возможно для экспоненциальной модели.

Более общие результаты представлены в работе Kiefer (1988, стр. 665). Kiefer рассматривает модель PH (\ref{eq:17.21}) с $\phi(\mathbf{x}'\be)=\exp{(\mathbf{x}'\be)}$, то есть,
        $$\lambda(t|\mathbf{x})=\lambda_0(t,\alpha)\exp{(\mathbf{x}'\be)}.$$
Интегральную функцию базового риска можно получить следующим образом:
        \begin{align}
        \label{eq:17.20}
        \int^{t}_{0}\lambda(s|\mathbf{x})ds&=\int^{t}_{0}\lambda_0(s,\alpha)\exp(\mathbf{x}'\be)ds,\\
        \Lambda(t|\mathbf{x})&=\Lambda_0(t,\alpha)\exp(\mathbf{x}'\be),\notag\\
        \ln\Lambda(t|\mathbf{x})&=\ln\Lambda_0(t,\alpha)+\mathbf{x}'\be,\notag\\
        -\ln\Lambda_0(t,\alpha)&=\mathbf{x}'\be-\ln\Lambda(t|\mathbf{x})\notag\\
        &=\mathbf{x}'\be+u,\notag
        \end{align}

где ошибка $u=-\ln\Lambda(t|\mathbf{x})$ имеет распределение экстремальных значений.


Результат не зависит от конкретного выбора базового риска, поскольку для определенного $\lambda_0(t,\alpha)$ зависимую переменную $t$ можно преобразовать как $-\ln\Lambda_0(t,\alpha)$, так, чтобы получить линейную модель регрессии с ошибками, распределенными по закону экстремальных значений. % 1 тип?
Для экспоненциальной модели $\ln\Lambda_0(t,\alpha)=\ln t$, для модели Вейбулла $\ln\Lambda_0(t,\alpha)=\alpha\ln t$. Для цензурированных данных типа I нужно найти $\E[\ln\Lambda_0(T,\alpha)|T>t^*]$, используя распределение экстремальных значений, а затем применить двухшаговую процедуру Хекмана. Эти результаты могут быть использованы в качестве основы для простых диагностических тестов, которые будут рассмотрены в следующей главе.





\section{Некоторые важные модели времени жизни}\label{sec:17.7}

\noindent
Пожалуй, наиболее часто применяемой моделью в анализе времени жизни является модель пропорциональных рисков. Однако знакомство с некоторыми ее вариантами, % ALT: модификациями?
а также с моделями ускоренной жизни (AFT), представленными в разделе \ref{sec:17.7.2}, будет полезно.


\subsection{Модель пропорциональных рисков}\label{sec:17.7.1}

\noindent
Как уже упоминалось, в \textbf{модели пропорциональных рисков} (PH) условный коэффициент риска $\lambda(t|\mathbf{x})$ может быть представлен в виде двух независимых функций:
        \begin{align}
        \label{eq:17.21}
        \lambda(t|\mathbf{x})=\lambda_0(t,\mathbf{\bm{\alpha}})\phi(\mathbf{x},\bm{\beta}),
        \end{align}
где $\lambda_0(t,\bm{\alpha})$ называется \textbf{базовым риском} и является функцией только от $t$, а $\phi(\mathbf{x},\bm{\beta})$ является функцией только от регрессоров $\mathbf{x}$. Обычно $\phi(\mathbf{x},\bm{\beta})=\exp(\mathbf{x}'\bm{\beta})$. В литературе часто встречаются \textbf{полиномиальные базовые риски}.

Все функции риска $\lambda(t|\mathbf{x})$ типа \ref{eq:17.21} пропорциональны по отношению к базовому риску с коэффициентом $\phi(\mathbf{x},\bm{\beta})$, который не является функцией от $t$. Популярность моделей PH объясняется тем, что для состоятельности оценок $\be$ не требуется спецификации функциональной формы для $\lambda_0(\cdot)$ (см. раздел \ref{sec:17.8}).

Экспоненциальная модель, а также модели регрессии Вейбулла и Гомперца являются моделями PH, поскольку их риски равны $\exp(\mathbf{x}'\bm{\beta})$, $\exp(\mathbf{x}'\bm{\beta})\alpha t^{\alpha-1}$, $\exp(\mathbf{x}'\bm{\beta})\exp(\alpha t)$, соответственно.

Еще одним примером модели PH, используемой преимущественно в анализе продолжительности безработицы, является \textbf{модель кусочно-постоянных рисков}, которая разбивает $\lambda_0(t,\bm{\alpha})$ на $k$ сегментов так, что
        \begin{align}
        \label{eq:17.22}
        \lambda_0(t,\bm{\alpha})=e^{\bm{\alpha}_j}, \hspace{0.2cm} c_{j-1}\le t\le c_j, \hspace{0.2cm} j=1,...,k,
        \end{align}
где $c_0=0$, $c_k=\infty$ и другие пороговые значения $c_1,...,c_{k-1}$ определены, а параметры $\alpha_1,...,\alpha_k$ подлежат оценке. Параметры возведены в степень, что гарантирует, что $\lambda_0(t,\bm{\alpha})>0.$ Эта модель оценивает больше базовых параметров, чем, например, модель Вейбулла, которая содержит только один базовый риск. Тем не менее, она может быть практична в применении для анализа достаточно больших объемов данных.

Индентифицируемость модели PH при ненаблюдаемой гетерогенности
% неоднородности?
будет рассмотрена в разделе 18.3. % \ref{sec:18.3} # UNCOMMENT AFTER 18 CHAPTER


\subsection{Модель ускоренной жизни}\label{sec:17.7.2}

\noindent
Модель AFT специфицирована в виде
        \begin{align}
        \label{eq:17.23}
        \ln t=\mathbf{x}'\bm{\beta}+u,
        \end{align}
где различные предположения о распределении $u$ приводят к различным моделям AFT. Поскольку $\ln t$ принимает значения в интервале $(-\infty,\infty)$, то $u$ может относиться к любому распределению, непрерывному на $(-\infty,\infty)$.

Название \textbf{ускоренной жизни} объясняется тем, что $t=\exp(\mathbf{x}'\bm{\beta})\nu$, где $\nu=e^u$, имеет коэффициент риска $\lambda(t|\mathbf{x})=\lambda_0(\nu)\exp(\mathbf{x}'\bm{\beta})$, где базовый риск $\lambda_0(\nu)$ не зависит от времени. Произведя замену $\nu=t\exp(-\mathbf{x}'\bm{\beta})$, получим функцию риска
        \begin{align}
            \label{eq:17.24}
            \lambda(t|\mathbf{x})=\lambda_0(t\exp(-\mathbf{x}'\bm{\beta}))\exp(\mathbf{x}'\bm{\beta}).
        \end{align}
При $\exp(-\mathbf{x}'\bm{\beta})>1$ происходит ускорение базового риска $\lambda_0(t)$, а при $\exp(-\mathbf{x}'\bm{\beta})<1$ --- замедление.

Если $u\thicksim\mathcal{N}[0,\sigma^2]$, то модель является лог-нормальной; лог-логистическая модель получается, если $u$ распределена логистически. Гамма модель также может быть представлена как AFT, если предположить, что функция плотности $u$ равна $f(u)=\exp(\alpha u-e^u)/\Gamma(\alpha)$.

Единственными моделями, которые одновременно относятся к классам PH и AFT, являются экспоненциальная и вейбулловская. Модель AFT, в частности, получается, если $u$ равно $\alpha\omega$, где $\omega$ имеет распределение экстремальных значений с функцией плотности $f(\omega)=e^{\omega}\exp(-e^\omega)$.

Модели времени жизни $g(t)=\mathbf{x}'\bm{\beta}+u$ не обязательно должны иметь логлинейную функциональную форму, $g(t)=\ln t$. За счет изменения функциональной формы можно получить целый класс моделей с преобразованием, к которому принадлежит, например, модель регрессии Бокса-Кокса.


\subsection{Гибкие модели рисков}\label{sec:17.7.3} % Адаптивный / приспосабливающийся ?

\noindent
Некоторые модели изначально задаются через функцию риска, а не через функцию плотности. Например, риск может быть квадратичен по отношению к $t$ так, что $\lambda(t)=\mathbf{x}'\bm{\beta}+a_1t+a_2t^2$, что приводит к U-образной форме функции риска. Соответствующий кумулятивный риск равен $\Lambda(t)=(\mathbf{x}'\bm{\beta})t+(a_1/2)t^2+(a_2/3)t^3$. Зная $\lambda(t)$ и $\Lambda(t)$, можно записать логарифм правдоподобия, используя полученные ранее результаты.

Недостаток этого подхода заключается в том, что $\lambda$ и $\Lambda$ могут принимать отрицательные значения и функция риска может оказаться несобственной, поскольку функция плотности не обязательно будет интегрироваться к единице.




\section{Модель пропорциональных рисков Кокса}\label{sec:17.8}

\noindent
Для цензурированных данных с единственным переходом, или двумя возможными состояниями, оценивать полностью параметрические модели относительно легче, однако неверная спецификация любого из параметров модели приводит к несостоятельным оценкам. Эту проблему можно отчасти решить за счет выбора более гибкой параметрической функциональной формы. Но несмотря на аргументированность % ALT: обоснованность
подхода, идентификация и оценка таких функциональных форм часто оказывается затруднена. Примером является обобщенная гамма модель.

Другим решением проблемы является применение полупараметрических методов, которые не требуют спецификации распределений всех параметров и, таким образом, позволяют снизить зависимость оценок от выбора функциональной формы. Модели тобит для цензурированных данных также оказываются недостаточно робастны. Однако методы, применяемые при их оценивании, существенно отличаются от тех, которые используются при анализе выживаемости, где объектом моделирования является коэффициент риска, который не имеет содержательной интерпретации в случае тобит. Подход, рассматриваемый в данном разделе, оказался настолько успешен на практике, что стал стандартным для анализа данных о выживаемости.


\subsection{Модель пропорциональных рисков}\label{sec:17.8.1}

\noindent
В первую очередь, необходимо выбрать функциональную форму для коэффициента риска в модели PH (см. раздел \ref{sec:17.7.1}) таким образом, чтобы условный коэффициент риска $\lambda(t|\mathbf{x},\bm{\beta})$ состоял из двух независимых функций
        \begin{align}
        \label{eq:17.25}
        \lambda(t|\mathbf{x},\bm{\beta})=\lambda_0(t)\phi(\mathbf{x},\bm{\beta}),
        \end{align}
где $\lambda_0(t)$ называется базовым риском и является функцией только от $t$, а функция $\phi(\mathbf{x},\bm{\beta})$ зависит только от регрессоров $\mathbf{x}$. В начале мы будем предполагать, что $\mathbf{x}$ не зависят от времени, но позже опустим эту предпосылку. Модель является полупараметрической, так как в ней отсутствует спецификация функциональной формы для $\lambda_0(t)$, в то время как функциональная форма для $\phi(\mathbf{x},\bm{\beta})$ полностью определена. Чаще всего предполагают, что $\phi(\mathbf{x},\bm{\beta})$ принимает экспоненциальную форму
        \begin{align}
        \label{eq:17.26}
        \phi(\mathbf{x},\bm{\beta})=\exp(\mathbf{x}'\bm{\beta}),
        \end{align}
поскольку она позволяет легко интерпретировать коэффициенты, гарантируя при этом, что $\phi(\mathbf{x},\bm{\beta})>0$. Например, увеличив $j$ый регрессор $x_j$ на единицу, считая, что все остальные регрессоры неизменны, получим
        \begin{align}
        \label{eq:17.27}
        \lambda(t|\mathbf{x}_{\textrm{new}},\bm{\beta})&=\lambda_0(t)\exp(\mathbf{x}'\bm{\beta}+\bm{\beta}_j)\\
        &=\exp(\beta_j)\lambda(t|\mathbf{x},\bm{\beta}).\notag
        \end{align}
То есть, новый риск равен произведению исходного риска на $\exp(\beta_j)$, а изменение риска --- его произведению на $1-\exp(\beta_j)$. Взяв производную, получим, что изменение риска равно произведению исходного риска на коэффициент $\beta_j$
        \begin{align}
        \label{eq:17.28}
        \partial\lambda(t|\mathbf{x},\bm{\beta})/\partial x_j=\lambda_0(t)\exp(\mathbf{x}'\bm{\beta})\beta_j=\beta_j\lambda(t|\mathbf{x},\bm{\beta}),
        \end{align}
что соответствует предыдущему результату, поскольку $\exp(\beta_j)\simeq1+\beta_j$. Статистические пакеты обычно предоставляют оценки с соответствующими доверительными интервалами для обоих показателей, $\beta_j$ и $\exp(\beta_j)$.

В общем виде изменения регрессоров оказывают мультипликативный эффект на исходный риск
        \begin{align}
        \label{eq:17.29}
        \partial\lambda(t|\mathbf{x},\bm{\beta})/\partial\mathbf{x}&=\lambda_0(t)\partial\phi(\mathbf{x},\bm{\beta})/\partial x_j\\
        &=\lambda(t|\mathbf{x},\bm{\beta})\times[\partial\phi(\mathbf{x},\bm{\beta})/\partial x_j]/\phi(\mathbf{x},\bm{\beta}), \notag
        \end{align}
для оценки которого требуется лишь знание коэффициентов $\bm{\beta}$, но не базового риска $\lambda_0(t)$. Идентификация моделей PH обсуждается в следующей главе, где рассматривается более общий случай, допускающий наличие ненаблюдаемой гетерогенности.


\subsection{Оценивание методом частичного правдоподобия}\label{sec:17.8.2}

Кокс (1972, 1975) предложил модель PH, которая не требует одновременного оценивания коэффициентов $\bm{\beta}$ и базового риска $\lambda_0(t)$. При желании, базовый риск может быть восстановлен после оценивания $\bm{\beta}$. Результаты представлены с учетом независимого цензурирования и наличия идентичных данных.

Аналогично \ref{sec:17.5}, данные состоят из завершенных наблюдений и наблюдений, находящихся под риском. Обозначим $t_1<t_2<...<t_j<...<t_k$ как \textbf{дискретные моменты отказа} объектов в выборке размера $N$, $N\ge k$. \textbf{Множество объектов под риском} $R(t_j)$ включает в себя наблюдения, находящиеся под риском непосредственно перед моментом $t_j$, $D(t_j)$ --- наблюдения, завершенные в момент $t_j$, а $d_j$ равно количеству завершенных наблюдений. Таким образом,
        \begin{align}
        \label{eq:17.30}
        R(t_j) &=\{l:t_l\ge t_j\}           &&=\textrm{множество объектов под риском в момент } t_j,\\
        D(t_j) &=\{l:t_l=t_j\}              &&=\textrm{множество объектов, завершенных в момент } t_j,\notag\\
        d_j    &=\sum_l \mathbf{1}(t_l=t_j) &&=\textrm{количество объектов, завершенных в момент } t_j.\notag
        \end{align}
Множество объектов под риском содержит все незавершенные и нецензурированные на момент $t_j$ наблюдения. Если $d_j>1$, то имеет место набор идентичных данных.

Вероятность того, что определенный объект, находящийся под риском, будет завершен в момент $t_j$ равна условной вероятности отказа объекта $j$, деленной на условную вероятность отказа любого из объектов из множества $R(t_j)$, или сумму вероятностей отказа каждого объекта в множестве $R(t_j)$. То есть,
        \begin{align}
        \Pr\left[T_j=t_j|R(t_j)\right]&=\frac{\Pr\left[T_j=t_j|T_j\ge t_j\right]}{\sum_{l\in R(t_j)}\Pr\left[T_l=t_l|T_l\ge t_j\right]}\notag\\
        &=\frac{\lambda_j(t_j|\mathbf{x}_j,\bm{\beta})}{\sum_{l\in R(t_j)}\lambda_l(t_j|\mathbf{x}_l,\bm{\beta})}\notag\\
        &=\frac{\phi(\mathbf{x}_j,\bm{\beta})}{\sum_{l\in R(t_j)}\phi(\mathbf{x}_l,\bm{\beta})}. \notag
        \end{align}
Поскольку риски пропорциональны, базовый риск в последней строке сокращается. Как следствие, свободный член в модели неидентифицируем, и его отсутствие упрощает оценивание коэффициентов $\bm{\beta}$. Однако формула требует корректировки, если возникают идентичные данные, характерные для сгруппированных длительностей (то есть, если несколько наблюдений завершились в одном интервале группировки). Предположим, например, что известны два наблюдения $j_1$ и $j_2$, идентичных в момент $t_j$, с набором регрессоров $\mathbf{x}_{j1}$ и $\mathbf{x}_{j2}$, соответственно. Если $j_1$ завершается раньше $j_2$, то формула вероятности равна
        $$\phi(\mathbf{x}_{j1},\bm{\beta})/\sum_{l\in R(t_j)}\phi(\mathbf{x}_l,\bm{\beta})+\phi(\mathbf{x}_{j2},\bm{\beta})/\sum_{l\in R_1(t_j)}\phi(\mathbf{x}_l,\bm{\beta}),$$
где множество $R_1(t_j)$ равно $R(t_j)$, за исключением объекта $j_1$. Выражение аналогично, если $j_2$ завершается раньше $j_1$. Вклад в функцию правдоподобия будет равен сумме этих вероятностей. Однако формула правдоподобия становится громоздкой, если в выборке содержится достаточно большой набор идентичных данных, поэтому часто используют приближение Breslow и Peto, см. Кокс и Oakes (1984)
        \begin{align}
        \label{eq:17.31}
        \Pr\left[T_j=t_j|j\in R(t_j)\right]\simeq\frac{\prod_{m\in D(t_j)}\phi(\mathbf{x}_m,\bm{\beta})}{\left[\sum_{l\in R(t_j)} \phi(\mathbf{x}_l,\bm{\beta})\right]^{d_j}},
        \end{align}
где $D(t_j)$ является множеством наблюдений, завершенных в момент $t_j$, а $d_j$ обозначает их количество. Приближение верно, если количество отказов ($d_j$) в момент $t_j$ относительно мало по сравнению с количеством объектов под риском.

Произведение вероятностей отказа $\Pr\left[T_j=t_j|j\in R(t_j)\right]$ для первых k наблюдений называется \textbf{частичная функцией  правдоподобия}, предложенной Коксом для оценивания коэффициентов $\bm{\beta}$. То есть,
        \begin{align}
        \label{eq:17.32}
        \mathrm{L}_p(\bm{\beta})=\prod^{k}_{j=1}\frac{\prod_{m\in D(t_j)}\phi(\mathbf{x}_m,\bm{\beta})}{\left[\sum_{l\in R(t_j)} \phi(\mathbf{x}_l,\bm{\beta})\right]^{d_j}}.
        \end{align}
Суть метода заключается в максимизации логарифма частичной функции правдоподобия % Опечатка в Кэмероне?? Минимизация?
        \begin{align}
        \label{eq:17.33}
        \ln\mathrm{L_p}=\sum^{k}_{j=1}\left[\sum_{m\in D(t_j)}\ln\phi(\mathbf{x}_m,\bm{\beta})-d_j\ln\left(\sum_{l\in R(t_j)}\phi(\mathbf{x}_l,\bm{\beta})\right)\right],
        \end{align}
где цензурированные наблюдения содержатся лишь во втором члене функции $\mathrm{L_p}$, поскольку составляют множество объектов под риском до тех пор, пока не оказываются цензурированы.

Пусть $\delta_i = 1$ является индикатором отсутствия цензурирования. Тогда уравнение \ref{eq:17.33} может быть перезаписано в виде
        \begin{align}
        \label{eq:17.34}
        \ln\mathrm{L_p}(\bm{\beta})=\sum^{N}_{i=1}\delta_i\left[\ln\phi(\mathbf{x}_i,\bm{\beta})-\ln\left(\sum_{l\in R(t_i)}\phi(\mathbf{x}_l,\bm{\beta})\right)\right].
        \end{align}

Для стандартной спецификации $\phi(\mathbf{x},\bm{\beta})=\exp(\mathbf{x}'\bm{\beta})$, или $\ln\phi(\mathbf{x},\bm{\beta})=\mathbf{x}'\bm{\beta}$, условия первого порядка выглядят следующим образом
        $$\frac{\partial\ln\mathrm{L_p}(\bm{\beta})}{\bm{\beta}}=\sum^{N}_{i=1}\delta_i[\mathbf{x}_i-\mathbf{x}^*_i(\bm{\beta})]=\mathbf{0},$$
где $\mathbf{x}^*_i (\beta)=\sum_{l\in R(t_i)}\mathbf{x}_l\exp(\mathbf{x}_l'\bm{\beta})/\sum_{l\in R(t_i)}\exp(\mathbf{x}_l'\bm{\beta})$ представляет собой взвешенное среднее по регрессорам $\mathbf{x}_l$ из множества объектов под риском в момент отказа $t_i$.

Данный метод является методом максимального правдоподобия с ограниченной информацией, поскольку оцениваемая функция не содержит информации о базовом риске $\lambda_0(t)$, но при этом не относится к моделям условного или маргинального правдоподобия. В литературе представлено широкое обсуждение того, можно ли считать $\mathrm{L_p}(\bm{\beta})$ функцией правдоподобия. Можно показать (Andersen et al., 1993), что, хотя $\ln \mathrm{L_p}$ не является настоящей функцией правдоподобия, полученные оценки, при которых $\mathrm{L_p}$ максимально, тем не менее, состоятельны. См. также Kalbfleisch и Prentice (2002, стр. 99-101) и Lancaster (1990, глава 9).

Используя результаты, представленные в главе 5, % \ref{ch:5} # UNCOMMENT AFTER THE END OF CHAPTER 5
и предположив для упрощения, что $\mathbf{A}(\bm{\beta})=-\mathbf{B}(\bm{\beta})$, получим
        \begin{align}
        \label{eq:17.35}
        \bm{\hat{\beta}}\thicksim\!\!\!\!\!^{{}^{a}}\hspace{0.1cm}\mathcal{N} \left[ \bm{\beta},\left(-\E\left[\frac{\partial^2\ln\mathrm{L_p}(\bm{\beta})}{\partial\bm{\beta}\partial\bm{\beta}'}\right]\right)^{-1} \right].
        \end{align}
Оценка не является эффективной, хотя по сравнению с полностью параметрическими моделями PH, например, моделью Вейбулла, разница в дисперсии оценки небольшая.


\subsection{Функция выживания в модели Кокса}\label{sec:17.8.3}

Большинство исследований ограничивается оцениванием и интерпретацией коэффициентов $\bm{\beta}$, используя для этого формулы \ref{eq:17.28} или \ref{eq:17.29}, в то время как представленные методы позволяют проводить более содержательный анализ. Например, некоторые работы дополнительно исследуют форму функции базового риска. Зная же оценки коэффициентов $\bm{\beta}$, полученные методом частичного правдоподобия в модели PH, можно вывести вывести функцию выживания, оценка которой аналогична оценке Каплан-Мейреа в разделе \ref{sec:17.5.1}.

Функция выживания соответствует функции риска следующим образом: $S(t|\mathbf{x},\bm{\beta})=\exp\left[-\int^{t}_{0}\lambda_0(s)\phi(\mathbf{x},\bm{\beta})ds\right]$. Также мы можем определить $S_0(t)=\exp\left[-\int^{t}_{0}\lambda_0(s)ds\right]$. Тогда
        $$S(t|\mathbf{x},\bm{\beta})=S_0(t)^{\phi(\mathbf{x},\bm{\beta})}.$$

Рассмотрим дискретный случай, где базовый риск равен $1-\alpha_j$ в каждый момент времени $t_j$, $j=1,...,k$. Оценка $\hat{\alpha}_j$ является решением уравнения, рассмотренного более подробно в следующем разделе, \ref{sec:17.8.4}
        \begin{align}
        \label{eq:17.36}
        \sum^{k}_{l\in D(t_j)}\frac{\phi(\mathbf{x}_l,\bm{\hat{\beta}})}{1-\hat{\alpha}_j^{\phi(\mathbf{x}_l,\bm{\hat{\beta}})}}=\sum_{m\in R(t_j)}\phi(\mathbf{x}_m,\bm{\hat{\beta}}),\hspace{0.3cm}j=1,...,k,
        \end{align}
где $\bm{\hat{\beta}}$ получена методом частичного правдоподобия, $D(t_j)$ обозначает множество объектов, умерших в $t_j$, а $R(t_j)$ --- множество объектов под риском в момент $t_j$. Из раздела \ref{sec:17.3.2} % Опечатка в Cameron: 17.3.3
мы знаем, что функция выживания в дискретном случае равна произведению мгновенных условных вероятностей дожития, $S_0(t)=\prod_{j|t_j\le t}\alpha_j$. Значит, оценка функции выживания будет равна
        \begin{align}
        \label{eq:17.37}
        \hat{S}_0(t)=\prod_{j|t_j\le t}\hat{\alpha}_j.
        \end{align}
При отсутствии регрессоров $\hat{S}_0(t)$ упрощается до оценки Каплан-Мейера. То есть, полагая $\phi(\mathbf{x}_j,\bm{\beta})=1$, получим коэффициент базового риска $1-\hat{\alpha_j}=d_j/r_j$. При наличии регрессоров, но отсутствии идентичных наблюдений, коэффициент базового риска будет равен $1-\hat{\alpha}_j=\phi(\mathbf{x}_j,\bm{\hat{\beta}})/\sum_{m\in R(t_j)}\phi(\mathbf{x}_j,\bm{\hat{\beta}})$.

Для наблюдений со взвешенными регрессорами $\mathbf{x}=\mathbf{x}^*$ функция выживания оценивается с помощью
        $$\hat{S}(t|\mathbf{x}^*,\bm{\beta})=\hat{S}_0(t)^{\phi(\mathbf{x}^*,\hat{\bm{\beta}})}.$$
Линейные преобразования регрессоров не влияют на оценки $\bm{\beta}$, но влияют на функцию базового риска. Например,
        \begin{align}
        \lambda(t|\mathbf{x},\bm{\beta})&=\lambda_0\exp(\mathbf{x}'\bm{\beta})\notag\\
        &=\lambda_0(t)\exp(\mathbf{\bar{x}}'\bm{\beta})\exp((\mathbf{x}-\mathbf{\bar{x}})'\bm{\beta})\notag\\
        &=\lambda^*_0(t)\exp((\mathbf{x}-\mathbf{\bar{x}})'\bm{\beta}),\notag
        \end{align}
где новый базовый риск равен $\lambda^*_0(t)\exp((\mathbf{x}-\mathbf{\bar{x}})'\bm{\beta})$. % опечатка
То есть, вычитание выборочного среднего приведет к изменению базового риска, и в таком случае нужно быть аккуратным при интерпретации базового риска и функции выживания.

Несмотря на то, что оценка базового риска играет важную роль при сравнении коэффициентов риска между различными группами наблюдений, часто для ее интерпретации целесообразно проводить сглаживание, поскольку из-за разрывов сравнивать коэффициенты может оказаться непросто.


\subsection{Вывод функции выживания}\label{sec:17.8.4}

\noindent
Уравнение \ref{eq:17.36} для оценивания $\alpha_j$ получено согласно Kalbfleisch и Prentice (2002, стр. 114-118).

Вклад объекта с длительностью $t_j$ в функцию правдоподобия равен вероятности дожить до момента $t>t_{j-1}$ за вычетом вероятности дожития до момента $t>t_j$. То есть,
        \begin{align}
        S(t_j|\mathbf{x},\bm{\beta})-S(t_{j+1}|\mathbf{x},\bm{\beta})&=S_0(t_j)^{\phi(\mathbf{x},\bm{\beta})}-S_0(t_{j+1})^{\phi(\mathbf{x},\bm{\beta})}\notag\\
        &=(\alpha^{-1}_jS_0(t_{j+1}))^{\phi(\mathbf{x},\bm{\beta})}-S_0(t_{j+1})^{\phi(\mathbf{x},\bm{\beta})}\notag\\
        &=(\alpha^{-\phi(\mathbf{x},\bm{\beta})}_j-1)S_0(t_{j+1})^{\phi(\mathbf{x},\bm{\beta})}\notag
        \end{align}
где $S_0(t_{j+1})=\prod^{j}_{l=1}\alpha_l=\alpha_jS_0(t_j)$. 

Вклад объектов, цензурированных в момент $t_j$, равен вероятности дожить до $t > t_j$, или $S_0(t_{j+1})^{\phi(\mathbf{x},\bm{\beta})}$. Таким образом, умершие или цензурированные объекты за период $[t_j,t_{j-1})$ входят в вероятность следующим образом $S_0(t_{j+1})^{\phi(\mathbf{x},\bm{\beta})}=\prod^{j}_{l=1}\alpha^{\phi(\mathbf{x},\bm{\beta})}_l$, а для умерших есть ещё дополнительный множитель $\left(\alpha^{-\phi(\mathbf{x},\bm{\beta})}_j-1\right)$. Следовательно, можно записать функцию правдоподобия по всем объектам

        $$\mathrm{L}(\bm{\alpha},\bm{\beta})=\prod^{k}_{j=1}\left[\prod_{l\in D(t_j)}(\alpha^{-\phi(\mathbf{x}_l,\bm{\beta})}_j-1)\prod_{m\in R(t_j)}(\alpha^{-\phi(\mathbf{x}_m,\bm{\beta})}_j)\right].$$
Логарифм правдоподобия равен
        $$\ln\mathrm{L}(\bm{\alpha},\bm{\beta})=\sum^{k}_{j=1}\left[\sum_{l\in D(t_j)}\ln(\alpha^{-\phi(\mathbf{x}_l,\bm{\beta})}_j-1)+\sum_{m\in R(t_j)}-\phi(\mathbf{x}_m,\bm{\beta})\ln\alpha_j \right].$$
Условие первого порядка $\partial\ln\mathrm{L}(\bm{\alpha},\bm{\hat{\beta}})/\partial\alpha_j=0$ --- это и есть выражение \ref{eq:17.36}.
% ALT: Используя условие первого порядка $\partial\ln\mathrm{L}(\bm{\alpha},\bm{\hat{\beta}})/\partial\alpha_j=0$, получим выражение \ref{eq:17.36}.




\section{Регрессоры, меняющиеся со временем}\label{sec:17.9}

\noindent
До сих пор мы рассматривали регрессоры, которые могут меняться между объектами, но постоянны во времени для каждого объекта, например, пол. Такая структура данных типична для пространственных моделей, логит-моделей или моделей тобит. В моделях выживаемости, однако, характеристики объектов могут меняться в пределах одного состояния, в связи с чем соответствующие регрессоры будут варьироваться во времени. Например, в ходе лечения может меняться доза лекарственных препаратов, а в течение периода поиска работы могут поменяться размеры пособия по безработице. Или же индивид может жениться, оставаясь при этом безработным, вследствие чего изменится значение переменной, отвечающей за семейный статус.

При работе с регрессорами, меняющимися со временем, возникает два типа проблем. Во-первых, полная история изменений переменной может предсказывать коэффициент риска, в связи с чем потребуется использовать лаговые переменные в качестве объясняющих. Их же отсутствие приведет к неверной спецификации модели. Во-вторых, такие регрессоры могут испытывать \textbf{эффект обратной связи} и поэтому не будут являться экзогенными, как обычно предполагается в моделях времени жизни. Например, длительность безработного состояния может зависеть от стратегии поиска работы, однако сама стратегия может также меняться в зависимости от количества прошедшего времени. Аналогично, доза лекарственного препарата может определяться состоянием пациента.

Поскольку детерминированные изменения во времени учитывать проще, стандартный анализ справляется с проблемами первого типа при условии, что регрессоры слабо экзогенны. То есть, независимо от того, является ли процесс, определяющий изменчивость во времени, стохастичным или детерминированным, нет необходимости знать параметры этого процесса. Некоторые авторы (например, Kalbfleisch и Prentice, 2002, стр. 196-200) классифицируют такую изменчивость во времени как \textbf{внешнюю} \textit{(external time variation)}. Эндогенные регрессоры же определяют \textbf{внутреннюю} изменчивость \textit{(internal time variation)}.

Простым решением первого рода проблем (особенно, если программное обеспечение не поддерживает оценку регрессоров, меняющихся со временем) будет являться замена изменчивой во времени переменной на ее среднее по всем наблюдениям значение за весь период длительности состояния. Качественное ПО, тем не менее, допускает изменчивость регрессоров.

Рассмотрим некий объект, например, безработного индивида, который находится в таком состоянии в течение времени $T$, а затем совершает переход, то есть, находит работу. Пусть $0<t_1<t_2<t_3<T$, где $t_1$, $t_2$ и $t_3$ являются промежуточными моментами наблюдения. Предположим далее, что существует два объясняющих фактора $x_1$ и $x_2(t)$, первый из которых постоянен во времени, а второй изменчив. Для простоты допустим, что $x_1$ является бинарной переменной, а $x_2$ принимает значения $x_2(t_1)$, $x_2(t_2)$ и $x_3(t_3)$ в соответствующих интервалах $[0,t_1)$, $[t_1,t_2)$ и $[t_2,T)$. Наконец, предположим, что меняющийся со временем регрессор экзогенен, а процесс, определяющий его изменчивость детерминирован. Тогда эти данные можно записать в виде следующей таблицы
    \begin{table}[!htbp]
    \begin{center}
\begin{tabular}{ccccc}
\hline \hline
\textbf{Наблюдение}&\textbf{Длительность}&$\bm{x_1}$&$\bm{x_2(t)}$&\textbf{Индикатор цензурирования}\\
\hline
1   &$t_1$  &1  &$x_2(t_1)$ &0\\
1   &$t_2$  &1  &$x_2(t_2)$ &0\\
1   &$T$    &1  &$x_2(T)$   &1\\
\hline \hline
\end{tabular}
    \end{center}
    \end{table}

Смысл заключается в том, что мы можем разбить одну длительность на три части в соответствии с изменениями регрессоров. Таким образом, значения регрессоров будут равны $(1,x_2(t_1))$ и $(1,x_2(t_2))$ для первых двух частей, и $(1,x_2(T))$ для третьей. Интуитивно, это аналогично тому, как если бы у нас было три наблюдения, два из которых были бы цензурированы ($\delta_i = 0$) и одно завершено ($\delta_i = 1$).

Предположим теперь, что коэффициент риска может также зависеть от предыдущих значений переменной, меняющейся со временем, $x_2(t)$. Тогда структура данных будет выглядеть следующим образом
    \begin{table}[!htbp]
    \begin{center}
\begin{tabular}{cccccc}
\hline \hline
\textbf{Наблюдение}&\textbf{Длительность}&$\bm{x_1}$&$\bm{x_2(t)}$&$x_2(t-1)$&\textbf{Индикатор цензурирования}\\
\hline
1   &$t_1$  &1  &$x_2(t_1)$ &0            &0\\
1   &$t_2$  &1  &$x_2(t_2)$ &$x_2(t_1)$   &0\\
1   &$T$    &1  &$x_2(T)$   &$x_2(t_2)$   &1\\
\hline \hline
\end{tabular}
    \end{center}
    \end{table}

Мы предположили, что до начала изучаемого состояния значение переменной $x_2(t)$ равнялось 0. Заметим также, что в обоих примерах $x_2(t)$ изменяется дискретно.

Несмотря на то, что запись наблюдений, состоящих из нескольких строк, возможна, для больших объемов данных она может оказаться громоздкой и неудобной. В особенности, трудности возникнут, если ПО начнет распознавать строки как отдельные наблюдения. К счастью, статистические пакеты обычно предлагают опцию задать переменные как меняющиеся со временем.
Можно также подогнать ступенчатые или непрерывные функции на основе прошедшего в определенном состоянии количества времени. % не очень понятно это предложение


\subsection{Расширенная модель Кокса}\label{sec:17.9.1}

Представленную в разделе \ref{sec:17.8} модель Кокса с постоянными регрессорами можно легко обобщить на случай регрессоров, меняющихся со временем.

В общем виде функция риска зависит от всей траектории регрессоров $\mathbf{x}(t)$ следующим образом
        $$\lambda(t|\mathbf{x}(t))=\lim_{\Delta t\rightarrow0}\frac{\Pr[t\le T<t+\Delta t|\mathbf{x}(t),T\ge t]}{\Delta t}.$$
В форме PH функция риска зависит лишь от текущих значений $\mathbf{x}(t)$
        $$\lambda(t|\mathbf{x}(t))=\lambda_0(t,\bm{\alpha})\phi(\mathbf{x}(t)),\bm{\beta}).$$

Из раздела \ref{sec:17.8.2} мы знаем, что метод частичного правдоподобия основывается на оценивании регрессоров $\mathbf{x}(t_j)$ для объектов, принадлежащих множеству под риском $R(t_j)$. Значит, в каждый момент отказа $t_j$ необходимо заменить $\mathbf{x}_i$ на $\mathbf{x}_i(t_j)$. Функция правдоподобия в таком случае примет вид
    $$\ln\mathrm{L_p}=\sum^{k}_{j=1}\left[\sum_{m\in D(t_j)}\ln\phi(\mathbf{x}_m(t_j),\bm{\beta})-d_j\ln\left(\sum_{l\in R(t_j)}\phi(\mathbf{x}_l(t_j),\bm{\beta})\right)\right].$$

Заметим, что данные теперь представляют собой более сложную структуру, предполагающую наличие нескольких наблюдений для каждого объекта. В качестве иллюстрации предположим, что время дискретно, длительность наблюдения равна 25 и присутствует только один регрессор $x_1$, который принимает значение 50 в интервале $[0,5]$, 100 в $[6,15]$ и 200 в $[16,25]$. Пусть моменты отказа упорядочены и равны, соответственно, 3, 8, 13,18 и 25. Тогда $x_1(t_1)=50$, $x_1(t_2)=100$, $x_1(t_3)=100$, $x_1(t_4)=200$, и $x_1(t_5)=200$.




\section{Пропорциональные риски в дискретном времени}\label{sec:17.10}

\noindent
Если наблюдаются только агрегированные интервалы (такие как неделя или месяц), в которых происходят отказы, то для анализа подходят модели сгруппированных длительностей.

Простая идея заключается в том, чтобы сформировать панель данных и затем оценить ее с помощью составной \textit{(stacked)} логит или пробит модели для вероятности отказа с индивидуальными константами в каждом периоде. Этот подход продемонстрирован в разделе \ref{sec:17.10.3}. Но сначала мы представим модель PH в дискретном виде, которая была рассмотрена во многих работах, включая Kalbfleisch и Prentice (1980), Fahrmeir и Tutz (1994), Kiefer (1988) и Meyer (1990). Мы будем использовать те же обозначения, что и Blake, Lunde Timmermann (1999).


\subsection{Пропорциональные риски в дискретном времени}\label{sec:17.10.1}

\noindent
Для сгруппированных данных с моментами группировки $t_a$, $a=1,...,A,$ функция риска в дискретном времени задается как
        $$\lambda^{\mathrm{d}}(t_a|\mathbf{x})=\Pr[t_{a-1}\le T<t_a|T\ge t_{a-1},\mathbf{x}(t_{a-1})], \hspace{0.3cm} a=1,...,A.$$
Допускается, что регрессоры могут меняться во времени. Соответствующая функция выживания в дискретном времени равна
        $$S^{\mathrm{d}}(t_a|\mathbf{x})=\Pr[T\ge t_{a-1}|\mathbf{x}]=\prod^{a-1}_{s=1}\left(1-\lambda^{\mathrm{d}}(t_s|\mathbf{x}(t_s))\right).$$
Сначала определим общую зависимость между коэффициентами риска в дискретном и непрерывном времени. В дискретном случае риск отказа равен вероятности того, что отказ произойдет в интервале $[t_{a-1},t_a)$, деленной на вероятность дожить, по крайней мере, до момента $t_{a-1}$, то есть
        \begin{align}
        \label{eq:17.38}
        \lambda^{\mathrm{d}}(t_a|\mathbf{x})=\frac{S(t_{a-1}|\mathbf{x})-S(t_{a}|\mathbf{x})}{S(t_{a-1}|\mathbf{x})},
        \end{align}
где $S(t|\mathbf{x})$ является функцией выживания. Зная, что $S(t|\mathbf{x}) = \exp(-\int^{t}_{0}\lambda(s)ds)$ в непрерывном случае, после некоторых преобразований получим
% ALT: уравнение \ref{eq:17.38} можно перезаписать как
        \begin{align}
        \label{eq:17.39}
        \lambda^{\mathrm{d}}(t_a|\mathbf{x})=1-\exp(-\int^{t_a}_{t_{a-1}}\lambda(s)ds).
        \end{align}
В непрерывном случае дискретный риск можно определить как
        $$\lambda(t)=\lambda_0(t)\exp(\mathbf{x}(t_{a-1})'\bm{\beta})$$
для $t$ из промежутка $[t_{a-1},t_a)$. Заметим, что регрессоры постоянны внутри интервалов, но могут меняться между интервалами, а $\lambda_0(t)$ может изменяться и внутри интервала. Тогда уравнение \ref{eq:17.39} принимает вид
        \begin{align}
        \label{eq:17.40}
        \lambda^{\mathrm{d}}(t_a|\mathbf{x})&=1-\exp(-\exp(\mathbf{x}(t_{a-1})'\bm{\beta})\times\int^{t_a}_{t_{a-1}}\lambda(s)ds)\\
        &=1-\exp(-\lambda_{0a}\exp(\mathbf{x}(t_{a-1})'\bm{\beta}))\notag\\
        &=1-\exp(-\exp(\lambda_{0a}+\mathbf{x}(t_{a-1})'\bm{\beta})),\notag
        \end{align}
где $\lambda_{0a}=\int^{t_a}_{t_{a-1}}\lambda_{0}(s)ds$. Соответствующая функция выживания равна
        \begin{align}
        \label{eq:17.41}
        S^{\mathrm{d}}(t_a|\mathbf{x})=\prod^{a-1}_{s=1}\exp\left(-\exp(\ln\lambda_{0s}+\mathbf{x}(t_{s-1})'\bm{\beta})\right).
        \end{align}

Функция плотности для $i$-го объекта равна произведению функции выживания в каждом периоде, в котором он оставался живым, и функции риска в каждом моменте отказа. Тогда, на основе \ref{eq:17.40} и \ref{eq:17.41} можно записать функцию правдоподобия
        \begin{align}
        \label{eq:17.42}
        \mathrm{L}(\bm{\beta},\lambda_{01},...,\lambda_{0A})=&\prod^{N}_{i=1}\left[\prod^{a_i-1}_{s=1}\exp(-\exp(\ln\lambda_{0s}+\mathbf{x}_i(t_{s-1})'\bm{\beta}))\right]\\
        &\times(1-\exp(-\exp(\ln\lambda_{0a_i}+\mathbf{x}_i(t_{a-1})'\bm{\beta}))).\notag
        \end{align}
Для упрощения мы не рассматриваем цензурированные наблюдения, а отказ для $i$-го наблюдения наступает в момент $t_{a_i}$. Предполагается, что в промежутке $[t_{a-1},t_a)$ происходит по крайней мере один отказ.

ММП предполагает максимизацию уравнения \ref{eq:17.42} по отношению к $\bm{\beta}$ и $\lambda_{01},...,\lambda_{0A}$. В общем случае ММП отличается от метода частичного правдоподобия, хотя в некоторых случаях они могут быть эквивалентны. Модели с ограниченным числом параметров допускают наличие структуры в параметрах $\lambda_{01},...,\lambda_{0A}$, например, полиномиальную зависимость от времени. Еще более ограниченными являются полностью параметрические модели, такие как модель Вейбулла, которая подразумевает, что $\lambda_{0s}=\int^{t_a}_{t_{a-1}}\alpha s^{\alpha-1}ds$.


\subsection{Подход Han и Hausman}\label{sec:17.10.2}

\noindent
Han и Hausman предложили относительно простой и удобный подход для оценивания базового риска, который предшествует работе Blake et al. (1999) и имеет некоторое сходство с работами Meyer (1990) и Sueyoshi (1992). Этот подход допускает значительную свободу в плане спецификации базового риска $\lambda^{d}_{0}(t)$, но в то же время сохраняет параметрическую форму для объясняющих переменных (например, $\exp(\mathbf{x}'\bm{\beta})$). К преимуществам можно отнести также то, что он в явном виде предназначен для работы с дискретными данными и позволяет проще учитывать такие их особенности, как идентичность и ненаблюдаемую гетерогенность. Идентичные данные однако могут представлять основную проблему при работе с дискретными данными, так как, например, длительность безработного состояния может часто совпадать с окончанием выплат пособия по безработице (которое в США составляет 26 недель).

Начнем с того, что для наблюдения $i$ определим коэффициент риска $\lambda_i(\tau)$ в форме PH, равный условной вероятности того, что объект будет завершен в промежутке $(\tau,\tau+\Delta)$
        $$\lambda_i(\tau)=\lambda_0(\tau)\exp(-\mathbf{x}_i'\bm{\beta}),$$
где $\lambda_0(\tau)$ обозначает базовый риск. Взяв логарифмы и поменяв переменные местами, как уже было показано в \ref{eq:17.20}, получим
        \begin{align}
        \label{eq:17.43}
        \Lambda_0(t)-\mathbf{x}_i'\bm{\beta}=\varepsilon_i
        \end{align}
где $\Lambda_0(t)=\ln\int^{t}_{0}\lambda_0(\tau)d\tau$ обозначает логарифм интегрального риска и $\varepsilon_i=\ln\int^{t}_{0}\lambda_i(\tau)d\tau$. Тогда можно записать вероятность отказа как
        $$\Pr[\textrm{отказа в периоде t}]=\int^{\Lambda_0(t)-\mathbf{x}_i'\bm{\beta}}_{\Lambda_0(t-1)-\mathbf{x}_i'\bm{\beta}}f(\varepsilon)d\varepsilon$$
Пусть $y_{it}=1$, если отказ $i$-го объекта происходит в момент $t$, и $y_{it}=0$ в обратном случае. Тогда функция правдоподобия для $N$ объектов будет равна
        \begin{align}
        \label{eq:17.44}
        \ln\Lambda(\bm{\beta},\Lambda_0(1),...,\Lambda_0(T))=\sum^{N}_{i=1}\sum^{T}_{t=1}y_{it}\ln\left[\int^{\Lambda_0(t)-\mathbf{x}_i'\bm{\beta}}_{\Lambda_0(t-1)-\mathbf{x}_i'\bm{\beta}}f(\varepsilon)d\varepsilon\right]
        \end{align}
где $\Lambda_0(1),...,\Lambda_0(T)$ оцениваются одновременно с $\bm{\beta}$. Спецификация функциональной формы при этом не важна.

Интеграл в функции правдоподобия равен разнице функция распределения $[\Lambda_0(t-1)-\mathbf{x}_i'\bm{\beta},\Lambda_0(t)-\mathbf{x}_i'\bm{\beta}]$. Конкретная форма зависит от предположения о функциональной форме функции распределения. Если случайная ошибка $\varepsilon_i$ подчиняется стандартному нормальному распределению, то логарифм правдоподобия преобразуется в порядковую пробит модель; если же предполагается, что ошибка имеет распределение экстремальных значений, то выражение принимает вид порядковой логит модели. В частности, предполагая, что ошибка распределена нормально, интеграл можно записать как
    $$\Pr[\Lambda_0(t)<\mathbf{x}_i'\bm{\beta}+\varepsilon_i\le\Lambda_0(t+1)]=\Phi(\Lambda_0(t+1)-\mathbf{x}_i'\bm{\beta})-\Phi(\Lambda_0(t)-\mathbf{x}_i'\bm{\beta}).$$

По сравнению с методом частичного правдоподобия, который игнорирует базовый риск, подход, предложенный Han и Hausman (1990), позволяет одновременно оценивать все неизвестные параметры при относительно невысоких издержках расчета. Результаты моделирования по методу Монте-Карло показывают, что подход хорошо справляется с моделированием случайных функций риска, без каких-либо предположений об их функциональной форме.

\subsection{Бинарный выбор в дискретном времени}\label{sec:17.10.3}

\noindent
Альтернативным способом анализа дискретных данных по длительностям являются модели бинарного выбора, поскольку в каждый момент времени возможен лишь один из исходов --- отказ или наступает, или же нет.

В дискретном времени переходы моделируются как
        \begin{align}
        \label{eq:17.45}
        \Pr[t_{a-1}\le T<t_a|T\ge t_{a-1}|\mathbf{x}]=F(\lambda_a+\mathbf{x}'(t_{a-1})\bm{\beta}), \hspace{0.3cm} a=1,...,A,
        \end{align}
где коэффициенты при объясняющих переменных постоянны во времени, а константа $\lambda_a$, $a=1,...,A$ может принимать различные значения. В качестве функции $F$ обычно выбирают нормальное или логистическое распределение. В таком случае, коэффициенты $\lambda_a$ и $\bm{\beta}$ можно оценить с помощью составной логит или пробит модели, в которой константа изменяется со временем. Из-за простоты в применении этот метод является довольно привлекательным.

Итоговая функция правдоподобия имеет вид
        $$\mathrm{L}(\bm{\beta},\lambda_1,...,\lambda_A)=\prod^{N}_{i=1}\left[\prod^{a_i-1}_{s=1}(1-F\left(\lambda_s+\mathbf{x}'_i(t_{s-1})\bm{\beta}\right))\right]\times F\left(\lambda_{a_i}+\mathbf{x}'(t_{a_i-1})\bm{\beta}\right).$$
За исключением выбора функции $F$, выражение аналогично \ref{eq:17.42}. Так как риск \ref{eq:17.40} имеет распределение экстремальных значений с $\ln\lambda_{0a}+\mathbf{x}(t_{a-1})'\bm{\beta}$, то выражение \ref{eq:17.40} воспроизводит скорее сопряженную лог-логистическую модель бинарного выбора % ENG: "Complementary log-log" Термин должен совпадать с термином из главы 14, таблица 14.3 !!!
(см. таблицу 14.3), % \ref{tab:14.3} # UNCOMMENT AFTER CHAPTER 14
чем обычную логит или пробит модель.


\section{Пример: длительность состояния безработицы}\label{sec:17.11}

\noindent
Настоящее эмпирическое приложение основано на данных из работы McCall (1996), % любезно
предоставленных ее автором, Brian McCall. База данных получена из Приложений об уволенных работниках (\textit{Displaced Workers Supplements}, DWS) % ALT: смещенных / замещенных работниках
к Текущему обследованию населения \textit{(Current Population Survey)} за 1986, 1988, 1990 и 1992 гг. Будем называть мерой длительности длительность безработицы, хотя формально мы наблюдаем лишь длительность нахождения индивида без работы, независимо от того, находился ли он или она в поиске работы, или же нет.

Для анализа требуется выработать критерий, на основе которого можно будет относить индивидов, устроившихся на новую после увольнения работу, к работающим на полную ставку или по совместительству. Будем называть индивида частично занятым, если он работал менее 35 часов в неделю, предшествующую моменту опроса, и занятым полный рабочий день в обратном случае.

В таблице \ref{tab:17.6} представлены ключевые экономические факторы, объясняющие продолжительность периода отсутствия работы. Из соображений краткости мы не будем рассматривать весь набор переменных, который был использован в статье McCall (1996), а возьмем лишь его часть.
    \begin{table}[!htbp]\caption{\textit{Длительность безработицы: описание переменных}}\label{tab:17.6}
    \begin{center}
\begin{tabular}{llc}
\hline \hline
\textbf{Название переменной}&\textbf{Описание переменной}           &\textbf{Среднее}\\
\hline
spell       &периоды безработицы: двухнедельный интервал            &6.248\\
CENSOR1     &1, если занят полный рабочий день                      &0.321\\
CENSOR2     &1, если является частично занятым                      &0.102\\
CENSOR3     &1, если уволился с новой работы:                       &0.172\\
            &тип занятости неизвестен                               &\\
CENSOR4     &1, если все еще безработный                            &0.375\\
UI          &1, если оформил страховой случай                       &0.553\\
RR          &коэффициент «замещения»                                &0.454\\
DR          &коэффициент «безразличия»                              &0.109\\
TENURE      &количество лет на старой работе                        &4.114\\
LOGWAGE     &логарифм недельной зарплаты                            &5.693\\
\hline \hline
\end{tabular}
    \end{center}
    \end{table}

\begin{figure}[ht!]\caption{Длительность безработицы: Оценки Каплан-Мейера функции выживания. Данные за 1986-92 гг. по 3343 наблюдениям, некоторые из которых не были завершены.}\label{fig:17.3}
\centering
%\includegraphics[scale=0.7]{fig.png}
\end{figure}

Длительности безработицы измерялись с двухнедельными интервалами. Переменные CENSOR1, CENSOR2, CENSOR3 и CENSOR4 являются индикаторами типа новой работы (или ее отсутствия). Для анализа, представленного в данной главе, мы будем использовать лишь первую переменную, CENSOR1. Таким образом, будем считать, что наблюдение завершено, если индивид устроился на новую работу на полный рабочий день. Другая фиктивная переменная UI \textit{(unemployment insurance)} равняется единице, если индивид оформил страховой случай \textit{(UI claim)} и получал, как следствие, пособие по безработице. Переменная RR обозначает коэффициент «замещения» \textit{(replacement rate)}, который равен недельному пособию по безработице, деленному на недельную заработную плату, которую индивид получал на старой работе. Переменная DR обозначает коэффициент «безразличия» \textit{(disregard rate)} и равна максимальному потенциальному заработку за неделю, который индивид мог бы получать, работая на полставки без какого-либо сокращения пособия по безработице, деленному на недельную заработную плату, которую индивид получал на старой работе. Смысл остальных переменных понятен из названий.

Начнем с описательного анализа данных по длительностям безработицы. Сперва, на рисунке \ref{fig:17.3} изобразим кривую выживания Каплан-Мейера. Серыми линиями обозначены 95\% доверительные интервалы, рассчитанные по формулам, представленным в разделе \ref{sec:17.5.2}. Как и ожидалось, кривая выживания быстро убывает в начале, но падение замедляется с течением времени.

Из таблицы \ref{tab:17.7} можно увидеть, что вероятность выжить после первого периода равна 0.91, что означает, что примерно 9\% индивидов нашли работу в течение первых двух недель.

Кривые выживания можно также изобразить в зависимости от значений переменной UI, то есть, по отдельности для тех, кто получал пособие по безработице, и тех, кто его не получал (см. рисунок \ref{fig:17.4}). Как и следовало ожидать, вероятность оставаться безработным по истечении соответствующего периода времени выше для индивидов, получавших страховые выплаты.

Оценка кумулятивного риска Нельсон-Аалена, изображенная на рисунке \ref{fig:17.5}, выглядит практически линейной, что говорит о слабой изменчивости коэффициента риска. Если бы он менялся значительными скачками, то кумулятивный риск имел бы нелинейную форму.

    \begin{table}[!htbp]\caption{\textit{Длительность безработицы: Функция выживания Каплан-Мейера и функция кумулятивного риска Нельсон-Аалена}}\label{tab:17.7}
    \begin{center}
\begin{tabular}{lcc}
\hline \hline
\textbf{Период}&\textbf{Функция выживания}&\textbf{Кумулятивный риск}\\
\hline
1   &0.9121 &0.0879\\
2   &0.8541 &0.1514\\
3   &0.8103 &0.2027\\
4   &0.7864 &0.2322\\
5   &0.7376 &0.2943\\
\vdots&\vdots&\vdots\\
12  &0.5974 &0.5005\\
13  &0.5680 &0.5496\\
14  &0.5270 &0.6219\\
\vdots&\vdots&\vdots\\
26  &0.3651 &0.9809\\
27  &0.3098 &1.1325\\
28  &0.3098 &1.1325\\
\hline \hline
\end{tabular}
    \end{center}
    \end{table}

Функции кумулятивного риска в зависимости от переменной UI представлены на рисунке \ref{fig:17.6}. Как и предполагалось, риск растет быстрее для тех, кто не получал страховых выплат.

Мы рассмотрим четыре варианта параметрических моделей регрессии с объясняющими переменными UI, RR, DR, LOGWAGE и RRUI, DRUI, где $RRUI = RR \times UI$, а $DRUI = DR \times UI$. В качестве спецификации выберем экспоненциальную модель, модели Вейбулла, Гомперца и PH Кокса, риск для которых выглядит следующим образом
        $$\lambda(t|\mathbf{x})=\lambda_0(t,\alpha)\phi(\mathbf{x},\bm{\beta})=\lambda_0(t,\alpha)\exp(\mathbf{x}'\bm{\beta}),$$
    \begin{figure}[ht!]\caption{Длительность безработицы: Оценки функции выживания в зависимости от значений переменной UI. Данные те же, что и на рисунке \ref{fig:17.3}.}\label{fig:17.4}
    \centering
%    \includegraphics[scale=0.7]{fig.png}
    \end{figure}
    % % % % %
    \begin{figure}[ht!]\caption{Длительность безработицы: оценка кумулятивного риска Нельсон-Аалена. Данные те же, что и на рисунке \ref{fig:17.3}.}\label{fig:17.5}
    \centering
%    \includegraphics[scale=0.7]{fig.png}
    \end{figure}
где $\lambda_0(t,\alpha)=\textrm{constant}=\exp(a)$ для некоторой константы $a$ в экспоненциальной модели, $\lambda_0(t,\alpha)=\exp(a)\alpha t^{\alpha-1}$ в модели Вейбулла (риски монотонны), $\lambda_0(t,\alpha)=\exp(a)\exp(\gamma t)$ в модели Гомперца, и предположения о форме базового риска $\lambda_0$ отсутствуют в модели PH Кокса. Заметим, что такая формулировка подразумевает модель пропорциональных рисков, которую с тем же успехом можно переформулировать как параметрическую модель, или же модель AFT. В такой форме функция правдоподобия позволяет найти оценки параметров $(\alpha,\bm{\beta})$, которые вместе с соответствующими $t-$статистиками представлены в таблице \ref{tab:17.8}. Здесь же указан логарифм правдоподобия с минусом, где для модели Кокса существует логарифм лишь частничного правдоподобия. Можно увидеть, что качество экспоненциальной модели и модели Гомперца одинаково. Модель Вейбулла соответствует данным лучше остальных. Из таблицы \ref{tab:17.8} можно также заметить, что с течением времени вероятность того, что наблюдение будет завершено, увеличивается (так как $\alpha=1.129>1$).

\begin{figure}[ht!]\caption{Длительность безработицы: оценка кумулятивного риска в зависимости от значений переменной UI. Данные те же, что и на рисунке \ref{fig:17.3}.}\label{fig:17.6}
\centering
%\includegraphics[scale=0.7]{fig.png}
\end{figure}

    \begin{table}[!htbp]\caption{\textit{Длительность безработицы: оценки параметров для четырех параметрических моделей}}\label{tab:17.8}
    \begin{center}
\begin{tabular}{lcccccccc}
\hline \hline
&\multicolumn{2}{c}{\textbf{Экспоненциальное}}&\multicolumn{2}{c}{\textbf{Вейбулла}}&\multicolumn{2}{c}{\textbf{Гомперца}}&\multicolumn{2}{c}{\textbf{Кокса PH}}\\
\cmidrule(r){2-3}\cmidrule(r){4-5}\cmidrule(r){6-7}\cmidrule(r){8-9}
\textbf{Переменная} &коэфф. &t      &коэфф. &t      &коэфф. &t      &коэфф. &t\\
\hline
RR                  &0.472  &0.79   &0.448  &0.70   &0.472  &0.78   &0.522  &0.91\\
DR                  &-0.576 &-0.75  &-0.427 &-0.53  &-0.563 &-0.74  &-0.753 &-1.04\\
UI                  &-1.425 &-5.71  &-1.496 &-5.67  &-1.428 &-5.69  &-1.317 &-5.55\\
RRUI                &0.966  &0.92   &1.105  &1.57   &0.969  &1.58   &0.882  &1.52\\
DRUI                &-0.199 &-0.20  &-0.299 &-0.28  &-0.211 &-0.21  &-0.095 &-0.10\\
LOGWAGE             &0.35   &3.03   &0.37   &2.99   &0.35   &3.03   &0.34   &3.03\\
CONS                &-4.079 &-4.65  &-4.358 &-4.74  &-4.097 &-4.65  &-      &-\\
$\alpha$            &       &       &1.129  &&&&&\\
$-\ln\textrm{L}$    &\multicolumn{2}{c}{2700.7}&\multicolumn{2}{c}{2687.6}&\multicolumn{2}{c}{2700.6}&\multicolumn{2}{c}{-}\\
\hline \hline
\end{tabular}
    \end{center}
    \end{table}

Для всех моделей значимыми являются лишь коэффициенты при переменных UI и LOGWAGE. Коэффициент при UI отрицателен, что указывает на то, что для тех, кто получал пособие по безработице, длительность безработного состояния завершается медленней. При этом различия в оценках коэффициента UI между моделями незначительны: по сравнению с оценками экспоненциальной модели в абсолютных значениях данный коэффициент выше лишь на 5\% и 0.2\% для моделей Гомперца и Вейбулла, соответственно, и на 8\% ниже для модели Кокса. Аналогично, коэффициент при LOGWAGE положителен и различается между моделями несильно.

Заметим, что в то время, как в эконометрической литературе оценки коэффициентов $(\alpha,\bm{\beta})$ принято записывать в метрике AFT, в биостатистике обычно используют метрику, основанную на PH. Отношение риска равно $\lambda(t|\mathbf{x})/\lambda_0(t,\alpha)=\phi(\mathbf{x},\bm{\beta})=\exp(\mathbf{x}'\bm{\beta})$. То есть, для фиктивной переменной $x$, принимающей значения $0/1$, эффект от ее изменения от 0 до 1 может быть измерен как $\exp(\beta)-1$, что отражает изменения по отношению к базовому риску. Большинство статистических пакетов предоставляют пользователю право выбора, в какой метрике результаты будут отображены. Сравнительные преимущества использования обеих метрик можно найти в работе Cleves, Gould и Guitirrez (2002).

Рассмотрим экспоненциальную модель, результаты оценивания которой представлены в таблице \ref{tab:17.9}, где соответствующий коэффициент является экспонентой, возведенной в степень коэффициента из таблицы \ref{tab:17.8}. Можно увидеть, что отношение риска при переменной UI равно 0.241. Это означает, что для индивида, оформившего страховой случай, «риск найти работу» снижается на 76\% по отношению к базовому риску. Для моделей Вейбулла, Гомперца и Кокса аналогичный риск снижается, соответственно, на 78\%, 76\% и 73\%.

Поскольку в данном примере мы предполагали, что данные цензурированы справа, и игнорировали наличие ненаблюдаемой гетерогенности, все три модели представляют качественно схожие результаты. Однако относительно низкое число объясняющих переменных, которые при этом сильно значимы, указывает на то, что значительная часть дисперсии может быть не объяснена (например, по причине ненаблюдаемой гетерогенности). Этот вопрос и будет рассмотрен в следующей главе.


    \begin{table}[!htbp]\caption{\textit{Длительность безработицы: оценки коэффициентов риска для четырех параметрических моделей}}\label{tab:17.9}
    \begin{center}
\begin{tabular}{lcccccccc}
\hline \hline
&\multicolumn{2}{c}{\textbf{Экспоненциальное}}&\multicolumn{2}{c}{\textbf{Вейбулла}}&\multicolumn{2}{c}{\textbf{Гомперца}}&\multicolumn{2}{c}{\textbf{PH Кокса}}\\
\cmidrule(r){2-3}\cmidrule(r){4-5}\cmidrule(r){6-7}\cmidrule(r){8-9}
\textbf{Переменная} &$\bm{\beta}$ &t      &$\bm{\beta}$ &t      &$\bm{\beta}$ &t      &$\bm{\beta}$ &t\\
\hline
RR                  &1.603  &0.63   &1.565  &0.57   &1.604  &0.62   &1.686  &0.71\\
DR                  &0.562  &-1.02  &0.653  &-0.66  &0.570  &-0.99  &0.471  &-1.55\\
UI                  &0.241  &-12.65 &0.224  &-13.12 &0.240  &-12.65 &0.268  &-11.53\\
RRUI                &2.626  &1.01   &2.760  &0.99   &2.635  &1.01   &2.416  &1.01\\
DRUI                &0.819  &-0.22  &0.742  &-0.33  &0.810  &-0.23  &0.909  &-0.10\\
LOGWAGE             &1.420  &2.56   &1.441  &0.08   &1.42   &2.55   &1.40   &2.57\\
$\alpha$            &       &       &1.129  &&&&&\\
$-\ln\textrm{L}$    &\multicolumn{2}{c}{2700.7}&\multicolumn{2}{c}{2687.6}&\multicolumn{2}{c}{2700.6}&\multicolumn{2}{c}{-}\\
\hline \hline
\end{tabular}
    \end{center}
    \end{table}



\section{Практические соображения}\label{sec:17.12}

\noindent
Параметрический анализ выживаемости включен в большинство статистических пакетов, в том числе в виде дополнительных модулей или расширений, которые предоставляют возможность выполнять различные стандартные операции. Например, можно получить непараметрическую оценку Каплан-Мейера функции выживания, включая расчет доверительных интервалов и многочисленные графические приложения. Расширения обычно сопровождаются методическими рекомендациями, зачастую настолько детальными, что их можно использовать как дополнительные справочники. Например, Allison (1995) предлагает практические рекомендации по работе в SAS; Cleves et al. (2002) представляет собой руководство по работе в STATA. Зачастую в таких справочниках можно найти не только базовые принципы и алгоритмы выполнения определенных команд, но и советы по анализу специфических типов данных, использования альтернативных спецификаций и интерпретации результатов. Поэтому выполнение практических заданий в таких статистических пакетах, как LIMDEP, STATA, SAS S-plus, является удобным способом изучения анализа времени жизни и закрепления знаний. При этом руководства представляют собой идеальные источники информации по стандартным вопросам.




\section{Литература}\label{sec:17.13}

\begin{itemize}

    \item[\textbf{17.3-17.7}]
Классическими работами по анализу выживаемости, с акцентом на модель Кокса, являются Kalbfleisch и Prentice (1980, 2002). Lawless (1982), Кокс и Oakes (1984) также представляют собой полезные источники информации; помимо этого, существует значительное количество текстов, посвященных анализу выживаемости. Байесовские методы оценивания можно найти в работах Ibrahim, Chen и Sinha (2001). В последнее время все больше исследований основываются на анализе счетных процессов. Данный подход подробно представлен в Fleming и Harrington (1991) и Andersen et al. (1993).

Следует отметить, что перечисленные работы (в особенности две последние) представляют собой довольно перспективные исследования. Lancaster (1990) предлагает детальное, хотя и техническое, описание анализа выживаемости, уделяя при этом больше внимания общим вопросам, представленным в следующих двух главах. По социальным исследованиям информацию можно найти в работе Allison (1984), которая, как и Lancaster (1990), охватывает гораздо больше, чем базовые модели с единственным переходом. По вопросам практического применения моделей полезно начать с Kiefer (1988).

    \item[\textbf{17.8}]
Подробное обсуждение метода частичного правдоподобия можно также найти в Lancaster (1990).

    \item[\textbf{17.10}]
В работах Meyer (1990), Han и Hausman (1990) и Blake et al. можно найти полезную информацию по дискретным моделям риска с учетом ненаблюдаемой гетерогенности, которая будет рассмотрена в следующей главе.

    \item[\textbf{17.11}]
Экономические приложения представлены в Kiefer (1988) и Greene (2003). Примеры параметрических моделей времени жизни в приведенной форме можно найти в статьях Lancaster (1979), Narendranathan, Nickell и Stern (1985), Джаггиа (1991c) и Gritz (1993). Хотя анализ выживаемости представлен, по большей мере, в приведенной форме, в последнее время интерес сместился в сторону более сложных структурных моделей (см., например, Van den Berg (1990) и Ferall (1997)). По вопросам структурного анализа можно обращаться к работам Lancaster (1990) и Van den Berg (2001). Van den Berg также представляет довольно интересное обоснование модели РH с точки зрения экономической теории. В качестве времени жизни допускается использование различных концепций. Например, Tunali и Pritchett (1997) предлагают три варианта: календарное время, возраст и длительности.
\end{itemize}


\section{Упражнения}\label{sec:17.ex}
\begin{itemize}

    \item[\textbf{17--1}]
(Sapra, 1998) Покажите, что модель времени жизни с плотностью распределения Парето первого рода $f(t) = \alpha k^\alpha /t^{\alpha + 1}$, $\alpha > 0$, $t \ge k \ge 0$ является моделью ускоренной жизни, но не является моделью пропорциональных рисков. [Подсказка: покажите, что $t$ может быть представлено в виде линейной регрессии $k = \exp(\mathbf{x}'\bm{\beta})$ с аддитивными гомоскедастичными ошибками.]

    \item[\textbf{17--2}] (Lancaster, 1979) Для каждого из пунктов выпишите соответствующие выражения функции правдоподобия в терминах функции плотности $f(t|\mathbf{x},\bm{\theta})$ и функции выживания $S(t|\mathbf{x},\bm{\theta})$.
        \item[\textbf{(a)}] Дана выборка независимых завершенных длительностей $t_i$, $i = 1, ..., N$.
        \item[\textbf{(b)}] Выборка получена на основе опроса безработных индивидов, проводившегося в течение $h$ периодов. Между началом и окончанием опроса часть респондентов нашли работу, а часть --- нет. Для тех, кто нашел работу, известен конкретный момент наступления этого события.
        \item[\textbf{(c)}] Структура выборки аналогична пункту (b), за исключением того, что момент наступления события неизвестен.

    \item[\textbf{17--3}]
        \item[\textbf{(a)}] Используя 50\% случайной выборки по данным McCall, рассчитайте непараметрическую оценку Каплан-Мейера функции выживания и оценку интегральной функции риска по отдельности для каждого типа цензурирования. Отличается ли функция выживания для наблюдений, завершенных в связи с переходом на полставки и на полную?
        \item[\textbf{(b)}] Игнорируя типы цензурирования, оцените следующие модели риска: (i) экспоненциальная, (ii) Вейбулла, (iii) лог-логистическая, (iv) PH Кокса. Используйте тот же набор регрессоров, который был представлен в этой главе.
        \item[\textbf{(c)}] Сравните (i)--(iii) модели и обсудите, какая из них лучше всего отражает структуру данных. Какие можно сделать выводы относительно формы функции риска для каждой из этих моделей?
\end{itemize}


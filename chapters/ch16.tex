
\chapter {Тобит-модели и модели выбора}

\section{Введение}

В этой главе рассматриваются две связанные между собой темы: модели с частично наблюдаемой зависимой переменной и модели с полностью наблюдаемой зависимой переменной для совокупности отобранных данных. К выше обозначенным моделям относятся модели с ограниченной зависимой переменной, модели со скрытой переменной, обобщенные тобит-модели и модели самоотбора. 

Все эти модели имеют общую характеристику, а именно даже в самом простом случае, когда условное среднее генеральной совокупности линейно по параметрам, МНК-оценки несостоятельны, поскольку выборка нерепрезентативна. Альтернативные способы оценки, в которых используются сильные предпосылки о распределении, необходимы для получения состоятельных оценок параметров.

К основным причинам неполноты данных относят усечение и цензурирование. При усечении как зависимая, так и независимая переменные наблюдаемы частично. Например, в качестве зависимой переменной используется доход, и в выборку входят индивиды только с низким уровнем доходов. В то же время, для цензурированых данных пропуски в данных могут быть только для зависимой переменной. Например, люди всех уровней дохода могут быть включены в выборку, но в целях конфиденциальности может быть установлено пороговое значение дохода, скажем 10 000 долл., выше которого, значение дохода принимается равным 10 000 долл. Усечение влечет за собой большую потерю информации, чем цензурирование. Основным примером усечения и цензурирования является тобит-модель, названная в честь Тобина (1958). Тобин изучал линейные регрессионные модели, опираясь на предпосылку, что ошибки нормально распределены. Проблемы характерные для оценки усеченных и цензурированных моделей рассмотрены в последующих главах, например, для цензурированных данных по длительности (глава 17). В целом, усечение и цензурирование -- это примеры пропущенных данных, см. главу 27.

Применения методов оценки первого поколения требует строгих предпосылок о распределении. Малейшие отклонения от изначальных предпосылок, например, нарушение предпосылки о гомоскедастичности ошибок, могут привести к несостоятельным оценкам параметров. По данной причине в этой главе рассматриваются полупараметрические регрессионные методы. Полупараметрические методы успешно применяются для простых моделей, например для данных цензурированных сверху. Вместе с тем, в настоящее время, отсутствует общепринятый способ оценивания для случаев, когда происходит самоотбор по ненаблюдаемым переменным.

В разделе 16.2 дана общая теория цензурированных и усеченных нелинейных регрессий; подробное описание тобит-модели дано в разделе 16.3. Альтернативная модель оценки цензурированных данных -- двухчастная модель -- вводится в разделе 16.4. В разделе 16.5 представлена модели самоотбора выборки. В разделе 6.6. на примере оценки затрат на здравоохранение сравниваются двухчастная модель Хекмана и модели самоотбора выборки. В разделе 16.7 рассматривается модель Роя. В разделе 16.8 рассматриваются полные структурные модели с угловыми решениями, полученные путем максимизации полезности или через обобщение системы одновременных уравнений для случая самоотбора выборки. В разделе 16.9 дается анализ полупараметрических методов.


\section{Эконометрические модели с цензурированными и усеченными данными}

Рассмотрим общие методы оценки полностью параметрических моделей с цензурированными или усеченными данными. Эти методы могут быть также использованы для оценки моделей, представленных в последующих главах, например, для счетной модели и модель длительности. Использование на практике тобит-модели для анализа цензурированных и усеченных данных продемонстрировано в разделах 16.2.1 и 16.3.

\subsection{Пример цензурированной и усеченной модели}

Пусть $y^* $ частично наблюдаемая переменная. При усечении снизу $y^* $ наблюдаемы, если $y^* $ выше порогового значения. Допустим, что пороговое значение равно нулю. Тогда, $y=y^* $, если $y^* >0$. Поскольку в выборке отсутствуют отрицательные значения, среднее усеченных данных будет выше среднего $y^* $. Для цензурированных снизу данных $y^* $ ненаблюдаемо полностью при $y^* <0$, но известно, что выполняется неравенство $y^* <0$, при этом значение $y^* $ приравнивается для простоты к нулю. В связи с этим, среднее цензурированных данных будет превышать среднее $y^* $. Очевидно, что среднее значение усеченных или цензурированных выборок не может использоваться без корректировки для оценки среднего исходной модели.

В этой главе рассматриваются аналогичные вопросы при оценки регрессионных моделей. Казалось бы, усечение или цензурирование приводит к изменению константы, не оказывая влияния на коэффициент наклона, но это не так. Например, если в исходной модели $\E[y^* |x]=x'\beta$, для усеченной или цензурированой модели $\E[y|x]$ нелинейно зависит от $x$ и $\beta$, следовательно, оценки МНК параметра $\beta$, а также оценка предельного эффекта последнего будут несостоятельны.

В качестве примера проанализируем модель предложения труда на симулированных данных. Предположим что зависимость ожидаемого количества отработанных часов в год, $y^* $ от почасовой заработная плата, $w$, описывается логарифмической функцией, и процесс порождающий данные представлен тремя уравнениями:

\begin{equation}
\begin{array}{l}
y^* =-2500+1000 \ln w +\varepsilon\\
\varepsilon\sim N[0,1000^2],\\
\ln w \sim N[2.75,0.60^2].
\end{array}
\end{equation}

% Рисунок figure 16.1
Рисунок 16.1
Tobit: Censored and Truncated Means --- Тобит: усеченные и цензурированные средние

Different Conditional Means --- Различные условные средние
Natural Logarithm of Wage --- Логарифм зарплаты
Actual Latent Variable --- Значение скрытой переменной
Truncated Mean --- Усеченное среднее
Censored Mean --- Цензурированное среднее
Uncensored Mean --- Нецензурированное среднее

Рисунок 16.1. Тобит регрессия количества часов работы от логарифма зарплаты: нецензурированное условное среднее (внизу), цензурированное условное среднее (посередине) и усеченное условное среднее (сверху) при усечении/цензурировании часов работы ниже нуля. Данные порождены простой линейной моделью.

Эта тобит-модель подробно рассмотрена в разделе 16.3. Согласно модели, эластичность заработной платы равна $1000/y^* $, следовательно, при полной занятости (2,000 часов) эластичность равна $0,5$. При увеличении заработной платы на $1\%$, количество часов работы увеличивается в среднем на $10$.

На рисунке 16.1 изображен график рассеивания в осях $y$ и $\ln{ w}$ для сгенерированной выборки из двухсот наблюдений. График безусловного среднего $y^* $ равного $-2500+1000 \ln w $ -- нижняя прямая линия.

При цензурировании в нуле отрицательные значения $y^* $ обнуляются, поскольку отрицательное значение ожидаемого количества рабочих часов, означает что индивиды предпочитают отдых. Как показывает практика, примерно $35\%$ индивидов предпочитают отдых. Замена отрицательных значений $y^* $ нулевыми, приводит к завышению среднего значения низкой заработной платы и практически не влияет на среднее высокой заработной платы, поскольку незначительный процент $y^* $ принимает нулевые значения. На средней кривой графика изображено итоговое значение цензурированного среднего, рассчитанное по формуле (16.23).


При усечении в нуле $35\%$ данных с отрицательными $y^*$ отбрасываются. Следовательно, для усеченных данных среднее выше, чем для цензурированных данных, поскольку отрицательные значения не учитываются. График усеченного среднего отображает верхняя линия на рисунке 16.1.

Очевидно, что среднее цензурированных и усеченных данных нелинейно по $x$, даже если линейно по $x$ среднее генеральной совокупности. Оценка цензурированной или усеченной регрессии МНК даст несостоятельные оценки коэффициентов наклона. Из графика видно, что график линейно аппроксимированных усеченного и цензурированного средних имеет более пологий наклон, чем график неусеченного среднего. Таким образом, среднее цензурированных или усеченных данных должно рассчитываться по специальным формулам. К сожалению, как мы увидим, эти формулы основаны на сильных предположениях о законе распределения.

\subsection{Механизмы цензурирования и усечения}

Как правило, в регрессионном анализе $y$ обозначает наблюдаемое значение зависимой переменной.В отличие от классического подхода, предполагается, что $y$ --- это частично наблюдаемая скрытая переменная $y^* $ и зависимость можно описать следующим уравнением:

\[
y=g(y^* ),
\]

где $g(\cdot )$ обозначает некоторую функцию. Варианты функциональной зависимости рассмотрены ниже. 


\subsubsection*{Цензурирование}


При цензурировании мы полностью наблюдаем регрессоры $x$, полностью наблюдаем $y^* $ из определенного подмножества значений $y^*$, а частично наблюдаем $y^*$ для оставшихся значений $y^*$. При цензурировании снизу (или слева):

\begin{equation}
y=
\begin{cases}
y^* , \text{ если } y^* >L\\
L, \text{ если }y^*  \leq L.
\end{cases}
\end{equation}

Например, всех потребителей можно разбить на две группы: к первой группе относятся индивиды с ненулевыми затратами на товары длительного пользования, т.е. $y^* >0$, а ко второй группе с нулевыми затратами, т.е. $y^*  \leq 0$. При цензурировании сверху (или справа):

\begin{equation}
y=
	\begin{cases}
	y^* ,	 \text{ если }y^* <U\\
	U,		\text{ если }y^*  \geq U.
	\end{cases}
\end{equation}

Например, может быть определена верхняя граница для данных о доходе, к примеру, $U=100,000$. Такое цензурирование в литературе по методам дюрации, получило название цензурирование первого типа (см. раздел 17.4.1). 


\subsubsection*{Усечение}

Усечение приводит к потере информации, поскольку теряются все данные за порогом теряются. При усечении снизу мы наблюдаем только:

\begin{equation}
y=y^* , \text{ если } y^* >L
\end{equation}

К примеру, выборка может включать данные только по индивидам, которые купили товары длительного пользования ($L=0$). При усечении сверху

\begin{equation}
y=y^* , \text{ если } y^* <U.
\end{equation}

Например, в выборку могут быть включены только данные по индивидам с низким уровнем дохода.

\subsubsection*{Интервальные данные}

Интервальные данные записываются с помощью интервалов. Как правило, именно в этой форме записываются данные исследования для обеспечения анонимности персональных данных. Например, доход может лежать в промежутке от 10,000 долл. до 100,000 долл. Для таких данных может быть установлено несколько границ цензурирования, тогда наблюдаемой переменной $y$ является интервал, в котором может находиться ненаблюдаемое значение $y^* $. 

\subsection{Цензурироваанная и усеченная оценка с помощью ММП}

Проблему цензурирования и усечения легко преодолеть, если исследователь использует полностью параметрический подход. Это может хорошо сработать для интервальных или ограниченных сверху данных, например, где целесообразно предположить логнормальное распределение дохода или использовать отрицательную биномиальную модель для количества визитов к доктору.

Если условное по параметрам распределение $y^* $ специфицировано, тогда возможно рассчитать эффективные и состоятельные ММП-оценки, учитывая цензурированное или усеченное распределение $y$. Предположим, что $f^{*}(y^* |x)$ и $F^{*}(y^* |x)$ обозначает функцию плотности условного распределения (или условную вероятность) и функцию распределения скрытой переменной $y^* $. Тогда всегда можно найти функции $f(y|x)$ и $F(y|x)$, соответствующие условной функции плотности и функции распределения наблюдаемой зависимой переменной $y$, поскольку $y=g(y^* )$ --- это преобразование $y^* $.

Ограниченность полностью параметрического подхода обусловлена сильными предпосылками о распределении. Например, ММП-оценка линейной регрессионной модели остается состоятельной даже при нарушении предпосылки о нормальности распределения ошибок, тогда как ММП-оценка цензурированной регрессии будет несостоятельной (см. раздел 16.3.2). Более гибкие модели и полупараметрические методы будут рассмотрены в последующих разделах. 


\subsubsection*{Цензурированный метод максимального правдоподобия}


Цензурирование и усечение оказывают влияние на условное среднее и условную плотность. В первую очередь рассмотрим влияние на плотность.


В качестве примера рассмотрим ММП оценку при цензурировании снизу. Для $y>L$ плотность $y$ принимает такое же значение как плотность $y^* $, следовательно, $f(y|x)=f^{*}(y|x)$. Для нижней границы $y=L$ плотность вырождена, и мы полагаем её равной вероятности наблюдения $y^* <L$ или $F^{*}(L|x)$. Таким образом, при цензурировании снизу 

\[
f(y|x)=
\begin{cases}
f^{*}(y|x),& \text{если $y>L$},\\
F^{*}(L|x),& \text{если $y=L$}
\end{cases}
\]


Ранее было отмечено, что при $y^* <L$ не обязательно брать $y=L$. Даже если все значения $y$ ненаблюдаемы, при $y^*  \leq L$, значение плотности равно $F^{*}(L|x)$. По аналогии с бинарными моделями, введем индикаторную переменную 

\begin{equation}
d=
\begin{cases}
1, & \text{если $y>L$}, \\
0, & \text{если $y=L$}.
\end{cases}
\end{equation}


Тогда условную плотность цензурированных снизу данных можно записать:

\begin{equation}
f(y|x)=f^{*}(y|x)^{d}F^{*}(L|x)^{1-d}.
\end{equation}



Например, для $N$ наблюдений цензурированная ММП-оценка максимизирует значение выражения:

\begin{equation}
\ln{ L_N{\theta}}=\sum_{i=1}^{N} \lbrace{d_i}\ln{ f^{*}(y_i|x_i,\theta)-d_i)\ln{ F^{*}}(L_i|x_i,\theta)\rbrace} ,
\end{equation}


где $\theta$ является параметрами распределения для $y^* $. В общем случае, при цензурировании нижняя граница может быть для каждого индивида своя, хотя, как правило, $L_i=L$. ММП-оценка цензурированной регрессии состоятельна и асимптотически нормально распределена, при условии, что правильно специфицирована исходная плотность до цензурирования, $f^{*}(y^* |x,\theta)$.

При цензурировании сверху функция максимального правдоподобия имеет вид (16.8), за исключением того, что $d=1$, если $y<U$ и $d=0$, в ином случае, а вместо $F^{*}(L|x,\theta)$ используется $1-F^{*}(U|x,\theta)$. Хорошим примером являются справа усеченные данные по длительности. 


\subsubsection*{Усеченная ММП-оценка}


При усечении снизу в точке $L$ условная плотность распределения $y$ равна (зависимость от $x$ опущена для простоты)


\[
f(y)= f^{*}(y|y>L)=f^{*}(y)/\Pr[ y|y>L] =f^{*}/[1-F^{*}(L)].
\]

Тогда, усеченная ММП-оценка максимизирует значение функции


\begin{equation}
\ln L_N(\theta) =\sum_{i=1}^N \lbrace \ln{ f^{*}(y_i,x_i,\theta)-\ln[1-F^{*}(L_i|x_i,\theta)]}\rbrace.
\end{equation}

При усечении сверху, функция максимального правдоподобия примет вид (16.9), где $F^{*}(L|x,\theta)$ заменяется на $F^{*}(U|x,\theta)$.


Если не учитывать усечение или цензурирование оценки параметров будут несостоятельны. Например, если не учитывать усечения, тогда будет максимизироваться значение функции $\sum_{i} \ln{  f^{*}(y_i,x_i,\theta)}$, что является неверным, поскольку отбрасывается второе слагаемое (16.9). Состоятельность цензурированных и усеченных ММП-оценок обеспечивается правильной спецификацией $f(\cdot )$, что, в свою очередь, требует правильной спецификации плотности скрытой переменной $f^{*}(\cdot )$. Даже если функция плотности $f^{*}(\cdot )$ принадлежит экспоненциальному семействе (см. раздел 5.7.3), не только среднее, но и сама плотность должны быть правильно специфицированы.


\subsubsection*{Оценка ММП для интервальных данных}


Пусть значения скрытой переменной $y^* $ принадлежат одному из $(J+1)$ взаимоисключающих интервалов $(-\infty,a_1], (a_1,a_2],\ldots ,(a_J,\infty)$, где значения $a_1,a_2,\ldots ,a_J$ известны. Поскольку

\[
\Pr[ a_j<y^* <a_{j+1}]=\Pr[ y^*  \leq a_{j+1}]-\Pr[ y^*  \leq a_j] =
F^{*}(a_{j+1})-F^{*}(a_j),
\]

ММП-оценка интервальных данных максимизирует значение функции 

\begin{equation}
\ln{ L_N(\theta)}=\sum_{i=1}^N\sum_{j=0}^J d_{ij} \ln[F^{*}(a_{j+1}|x_i,\theta)-F^{*}(a_j|x_i,\theta)].
\end{equation}


где $d_{ij}$, $j=0,\ldots ,J$, индикатор равный единице, если $y_{ij}{\epsilon}(a_j,a_{j+1}]$ и нулю иначе. Модель аналогична логит- и пробит-регрессии (см. раздел 15.9.1)., за исключением того, что границы интервалов $a_1,\ldots ,a_J$ известны


\subsection{Пример пуассоновской усеченной и цензурированной регрессии}


Предположим, что $y^* $ распределено по Пуассоновскому закону, т.е. $f^{*}(y)=e^{-\mu}\mu^{y}/y!$ и $\ln{ f^{*}(y)}=-\mu+y\ln{ \mu}-\ln{ y!}$, где среднее значение $\mu=\exp (x'\beta)$.

Предположим, что задача состоит в моделировании количества визитов в больницу, но данные известны только для тех, кто посещал врача. Следовательно, данные усечены снизу и $y=y^* $ только для $y^* >0$. Тогда $F^{*}(0)=\Pr[ y^*  \leq 0]=\Pr[ y^* =0]=e^{-\mu}$ и оценка $\beta$ максимизирует значение выражения

\[
\ln{ L_N(\beta)}=\sum_{i=1}^N{\lbrace-\exp (x_i'\beta)+y_ix_i'\beta-\ln{ y!}-\ln[1-\exp (-\exp (x_i'\beta))]\rbrace}
\]


Теперь предположим, что данные цензурированы выше границы равной 10, тогда $y=y^* $ если $y^* <0$ и $y=10$, если $y^* {\geq}10$. Тогда $\Pr[ y^* {\geq}10]=1-\Pr[ y^* <10]=1-\sum_{k=0}^9 f^{*}(k)$. Из формулы (16.8) следует, что цензурированная ММП оценка $\beta$ максимизирует значение функции


\[
\ln{L_N(\beta)}=\sum_{i=1}^N \left\lbrace d_i[-\exp (x_i'\beta)+y_ix_i'\beta-\ln{ y_i}!]+(1-d_i)\ln{ \left[\sum_{k=0}^9 e^{-\exp (x_i'\beta)}(\exp (x_i'\beta))^{k}/k!\right]}\right\rbrace.
\]


В обоих случаях условия первого порядка значительно усложняются из-за усечения и цензурирования. Вместе с тем, максимизация исходной функции плотности, т.е. при отсутствии усечения и цензурирования, даст несостоятельные оценки параметров.

\subsection{Условное среднее в цензурированных и усеченных регрессиях}

Цензурирование или усечение данных приводит к изменению условного среднего.

Например, рассмотрим усеченное в нуле пуассоновское распределение. Усеченная вероятность равна $f^{*}(y)/[1-F^{*}(0)], y=1,2,\ldots $, тогда усеченное среднее $\sum^{\infty}_{k=1}kf^{*}(k)/[1-F^{*}(0)]=\sum^{\infty}_{k=0}=\sum^{\infty}_{k=0}kf^{*}(k)/[1-F^{*}(0)]=\mu/[1-e^{-\mu}]$. Следовательно, условное математическое ожидание примет значение:


\[
\E[y|x]=\exp (x'\beta)/[1-\exp (-\exp (x'\beta))],
\]

а не $\exp (x'\beta)$, что справедливо для полных данных.

Это значение $\E[y|x]$ может использоваться для получения оценок с помощью нелинейного МНК. Тем не менее, преимущество нелинейного МНК по сравнению с ММП невелико, т.к. нелинейный МНК использует сильные предпосылки о распределении, которые в целом настолько же сильные, как и предпосылки, выполнение которых необходимо для получения более эффективных ММП-оценок.

\section{Тобит-модель}

Усеченные и цензурированные данные чаще всего встречаются в эконометрическом анализе линейных регрессионных моделей с нормально-распределенными ошибками, когда наблюдаются только положительные значения зависимой переменной. Эти модели получили название тобит-модели, в честь Тобина (1958), исследователя, который применил модель с выше обозначенными характеристиками для оценки затрат индивидов на товары длительного пользования. На практике модель слишком ограничена. Тем не менее, существует необходимость в детальном изучении модели Тобина, поскольку тобит-модель является основой для построения моделей более общего класса, которые будут рассмотрены в последующих разделах этой главы.

\subsection{Тобит-модель}

Цензурированная регрессионная модель с нормально распределенными данными, или тобит-модель, принадлежит классу моделей с цензурированием снизу в нуле, где латентная переменная линейно зависит от параметров, ошибки аддитивны, нормально распределены и гомоскедастичны. Таким образом, 

\begin{equation}
 y^* =x'\beta+\varepsilon,
\end{equation}
где ошибки распределены нормально, с параметрами $0$ и $\sigma^2$:

\begin{equation}
\varepsilon{\sim}N[0,\sigma^2],
\end{equation}

и $\sigma^2$ константа. Следовательно, распределение скрытой переменной имеет вид: $y^*  \sim N[x'\beta,\sigma^2]$. Наблюдаемые значения $y$ определены в выражении (16.2) с $L=0$, поэтому 

\begin{equation}
y=
\begin{cases}
y^* , &\text{если } y^* >0, \\
-, & \text{если } y^*  \leq 0,
\end{cases}
\end{equation}

где $-$ означает, что $y$ пропущено. При $y^*\leq 0$ может не наблюдаться конкретное значение $y$, однако, в некоторых случаях, например для затрат на товары длительного пользования, мы наблюдаем $y=0$. 

Уравнения (16.11)-(16.13) задают базовую тобит-модель, рассмотренную Тобином (1958). В более общем виде, тобит-модель для латентных переменных начинается с (16.11) и (16.12) и может включать различные способы цензурирования: сверху, одновременно сверху и снизу (тобит-модель с двумя порогами) и интервальное цензурирование, -- общий вид модели может меняться. В этом разделе все цензурирование удовлетворяет условию (16.13). Модели последующих разделов иногда называют обобщенными тобит-моделями.

Приравнивание $L$ к нулю является не только естественным, но и необходимым условием для линейной модели с константой и постоянным пороговым значением $L$. Тогда, переменная $y$ наблюдаема, при $y^* >L$, что эквивалентно выражению $\beta_1+x'_2\beta_2+\varepsilon$ или $(\beta_1-L)+x'\beta_2+\varepsilon>0$. То есть идентифицируемо только значение $\beta_1 - L$. В общем случае, модель с латентными переменными $y^* =x'\beta$ с переменным порогом цензурирования $L=x'\gamma$, эквивалентна модели $y^* =x'(\beta-\gamma)+\varepsilon$ с фиксированным значением порога $L=0$. Эти выводы являются следствием цензурирования в линейной модели с аддитивной ошибкой и не применимы к нелинейным моделям, например, для пуассоновского распределения.

Используя выражение (16.7) для плотности цензурированного распределения, где $f^{*}(y) \sim N[x'\beta, \sigma^2]$ получаем:

\[
F^{*}(0)=\Pr[ y^* \leq0] =\Pr[ x'\beta+\varepsilon \leq 0] =\Phi(-x'\beta/\sigma) 
=1-\Phi(x'\beta\sigma)
\]

где $\Phi(\cdot )$ функция стандартного нормального распределения и для получения последнего равенства используется свойство симметрии стандартного нормального распределения. Тогда, плотность цензурированного распределения равна:

\begin{equation}
f(y)=
\left[\dfrac{1}{\sqrt{2\pi\sigma^{2}}}\exp \lbrace-\dfrac{1}{2\sigma^{2}}(y-x'\beta)^{2}\rbrace\right]^d
\left[1-\Phi\left(\dfrac{x'\beta}{\sigma}\right)\right]^{1-d},
\end{equation}

где значение бинарного индикатора $d$ определено в (16.6) при $L=0$. Оценка параметра тобит-модели методом максимального правдоподобия $\hat{\theta}=(\hat{\beta}',\hat{\sigma}^2)'$ является максимумом цензурированной функции (16.8). Используя (16.4), получим смешение плотностей дискретного и непрерывного распределения: 

\begin{multline}
\ln L_N(\beta,\sigma^2)=\sum_{i=1}^N
\left\lbrace d_i
\left(-\frac{1}{2}\ln 2\pi-\frac{1}{2}\ln{ \sigma^2-\frac{1}{2\sigma^2}(y_i-x_i')^2}\right)
+\right.\\
\left.
(1-d_i)\ln \left(1-\Phi
\left( \frac{x_i'\beta}{\sigma}\right)
\right)
\right\rbrace,
\end{multline}

и условие первого порядка:

\begin{equation}
\begin{array}{l}
\dfrac{\partial \ln L_N}{\partial\beta}=\sum_{i=1}^N\dfrac{1}{\sigma^2}\left(d_i(y_i-x_i'\beta)-(1-d_i)\dfrac{\sigma\phi_i}{(1-\Phi_i)}\right)x_i=0 \\
\dfrac{\partial \ln L_N}{\partial\sigma^2}=\sum_{i=1}^N\sum\left\lbrace d_i \left(-\dfrac{1}{2\sigma^2}+\dfrac{(y_i-x_i')^2}{2\sigma^4}\right)+(1-d_i)\dfrac{\phi_i x_i'\beta}{(1-\Phi_i)2\sigma^3}\right\rbrace=0
\end{array}
\end{equation}


мы используем тот факт, что $\partial\Phi(z)/\partial{z}$, где $\phi(\cdot )$ функция плотности нормальной стандартной случайной величины и определяем $\phi_i=\phi(x_i'\beta/\sigma)$ и $\Phi_i=\Phi(x_i'\beta/\sigma)$.


Как обычно, $\hat{\theta}$ состоятельна, если плотность условного распределения корректно определена, т.е. модель задана уравнением (16.11) и (16.12) и принцип цензурирования отражен в (16.13). Оценка максимального правдоподобия имеет нормальное распределение, ковариационную матрицу для нормального распределения можно найти в работах Маддала (1983, стр. 155) и Амэмия (1985, стр.373).

Тобин (1958) предложил оценивать тобит-модель методом максимального правдоподобия и обосновал возможность применения ММП оценки. В работе Амэмия (1973) представлено формальное доказательство применимости стандартной теории для плотности смешенного, дискретно-непрерывного, цензурированного распределения. В приложении к классической работе Амэмия подробно рассмотрена асимптотическая теория для экстремумов оценок, представленных в разделе 5.3. 

Если данные усечены снизу от нуля, а не цензурированы, тогда ММП-оценка тобит-регрессии $\hat{\theta}=(\hat{\beta}',\hat{\sigma}^2)$ является максимумом для усеченной функции правдоподобия


\begin{equation}
\ln{ L_N(\beta,\sigma^2)}=\sum_{i=1}^N{\sum\left\lbrace -\dfrac{1}{2}\ln{ \sigma^2}-\dfrac{1}{2}{\ln{ 2\pi}}-\dfrac{1}{2\sigma^2}(y_i-x_i'\beta)^2-\ln{ \Phi(x_i'\beta/\sigma)}\right\rbrace }
\end{equation}

где для $y^* $ используется выражение (16.9) и параметры распределения $y^* $ определены в (16.11) и (16.12).

\subsection{Несостоятельность тобит оценок метода максимального правдоподобия}

Главным недостатком тобит-моделей является жесткость предпосылок о распределении ошибок. При нарушении предпосылки о гомоскедастичности ошибок или о нормальном распределении оценки максимального правдоподобия будут несостоятельны.

Это утверждение следует из условий первого порядка (16.16) максимального правдоподобия, заданные сложной функцией, зависящей от переменных $d_i, y_i, \phi_i$ и $\Phi_i$. Первое уравнение в (16.16) удовлетворяет условию $\E[\partial{lnL_N}/\partial\beta]=0$, которое являются необходимыми условием состоятельности оценок (см. Раздел 5.3.7), если

\[
\begin{array}{l}
\E[d_i]=\Phi_i, \\
\E[d_{i}y_{i}]=\Phi_{i}x'_{i}\beta+\sigma\phi_i.
\end{array}
\]

Можно показать, что эти моментные условия выполняются, если спецификация модели задана выражениями (16.11) и (16.12) и принцип цензурирования соответствует (16.13). Тем не менее, эти условия скорее всего не будут выполняться, если спецификация модели определена иначе, чем в (16.12) и (16.13), поскольку не будет выполняться условие гомоскедастичности или предпосылка о нормальности распределения. Например, если ошибки гетероскедастичны оценка будет несостоятельна, поскольку $\E[d_i]=\Phi(x'_{i}\beta/\sigma_{i}){\neq}\Phi_{i}$, кроме случая $\sigma^2_i=\sigma^{2}$.

Тем не менее, при гетероскедастичности нормально распределенных ошибок возможно получение состоятельных оценок, если задать модель для дисперсии $\sigma^2_i=\exp (z'_i\gamma)$. При цензурировании данных ниже нуля в функции максимального правдоподобия $lnL_{N}(\beta,\gamma)$, определенной в (16.15), значение дисперсии $\sigma^2$ заменяется на $\exp (z'_i\gamma)$. Обязательными условиями состоятельности оценки являются нормально распределенные ошибки и верно заданная функциональная форма для гетероскедастичности.


Ясно, что выполнение предпосылок о распределении более важно в цензурированных и усеченных моделях, даже если это распределение устойчиво к неправильной спецификации когда цензурирование или усечение отсутствуют. Тесты на спецификацию тобит-модели представлены в разделе 16.3.7. Для многих цензурированных данных тобит регрессии не применимы. Модели более общей формы рассмотрены далее.


\subsection{Цензурированное и усеченное среднее в линейной регрессии}


Для цензурированных и усеченных линейных регрессий (16.11) условное математическое ожидание наблюдаемой переменной $y$ отлично от $x'\beta$, а условная дисперсия отлична от $\sigma^2$ даже если ошибки $\varepsilon$ гомоскедастичны. Распределение $y$ также не является нормальным, несмотря на нормальное распределение ошибок. В этом разделе приведены общие результаты для линейной регрессии, в последующих разделах (16.3.4-16.3.7) рассмотрен частный случай, когда ошибки нормально распределены. Результаты учитывают цензурирование и усечение и являются основой для методов, не входящих в группу методов максимального правдоподобия.

Вначале рассмотрим усеченное среднее. Влияние усечения данных можно предположить интуитивно. Усечение данных слева исключает малые значения, следовательно, среднее значение должно увеличиться, а усечение справа, наоборот, предполагает уменьшение среднего. Поскольку усечение предполагает сокращение диапазона значений, дисперсия также должна сократиться.

При усечении слева, значения $y$ наблюдаемы, если $y^* >0$. Опуская обозначение $x$ в зависимости математического ожидания, получаем

\begin{multline}
\E[y]=\E[y^* |y^* >0]=\\
=\E[x'\beta+\varepsilon|x'\beta+\varepsilon>0]\\
=\E[x'\beta|x'\beta+\varepsilon>0]+\E[\varepsilon|x'\beta+\varepsilon>0]\\
=x'\beta+\E[\varepsilon|\varepsilon>-x'\beta],
\end{multline}

где второе равенство использует (16.11), а в последнем равенстве используется свойство независимости $\varepsilon$ от $x$. Как и ожидалось, усеченное среднее превышает значение $x'\beta$, поскольку $\E[\varepsilon|\varepsilon>c]$ превосходит $\E[\varepsilon]$ для любого значения константы $c$.

Для данных, цензурированных слева от нуля, положим, что вместо $y^*  \leq 0$ мы наблюдаем $y=0$. Цензурированное среднее рассчитывается путем взятия сначала условного среднего $y$ при фиксированном индикаторе $d$, определенном в (16.6) с $L=0$, а потом взятием безусловного среднего. Снова опуская зависимость от $x$, получаем, что среднее значение слева цензурированных данных равно:


\begin{multline}
\E[y]=E_{d}[E_{y|d}[y|d]] \\
=\Pr[ d=0]{\times}\E[y|d=0]+\Pr[ d=1]{\times}\E[y|d=1]\\
=0{\times}\Pr[ y^*  \leq 0]+\Pr[ y^* >0]{\times}\E[\varepsilon|\varepsilon>-x'\beta],\\
=\Pr[ y^* >0]{\times}\E[y^* |y^* >0],
\end{multline}

где $\Pr[ y^* >0]=1-\Pr[ y^*  \leq 0]=\Pr[ \varepsilon>-x'\beta]$ равно единица минус вероятность цензурирования и $\E[y^* |y^* >0]$ усеченное среднее, ранее выведенное в (16.18).

Таким образом, при цензурировании или усечении данных с пороговым значением ноль условное среднее будет равно:

для скрытой переменной:

\[\E[y|x^{*}]=x'\beta\]

для модели, усеченной слева (в нуле):

\begin{equation}
\E[y|x,y>0]=x'\beta+\E[\varepsilon|\varepsilon>-x'\beta],
\end{equation}

для модели, цензурированной слева (в нуле): 

\[
\E[y|x]=\Pr[ {\varepsilon}>-x'\beta]\lbrace{x'\beta+\E[\varepsilon|\varepsilon>{-x}'\beta]}\rbrace.
\]

Очевидно, что даже если условное среднее начальной переменной линейно, то при цензурировании или усечении данных линейность исчезает, что приводит к несостоятельности МНК-оценок.

Для решения проблемы несостоятельности можно использовать параметрический метод, накладывая ограничения на распределение $\varepsilon$. Затем можно получить выражения для $\E[\varepsilon|\varepsilon>x'\beta]$ и $\Pr[ \varepsilon>x'\beta]$ и, следовательно, можно рассчитать цензурированное или усеченное среднее. Параметрический метод оценки с нормально распределенными ошибками представлен в следующем разделе.

Рисунок 16.2.

Inverse Миллс Ratio as Cutoff Varies --- Обратное отношение Миллса и пороговое значение

inverse Миллс ratio --- обратное отношение Миллса
N[0,1] cdf --- N[0,1] функция распределения
N[0,1] pdf --- N[0,1] функция плотности

Inverse Миллс, pdf and cdf --- Обратное отношение Миллса, функции плотности и распределения
Cutoff point c --- Пороговое значение c

Рисунок 16.2. Изменение обратного отношения Миллса в зависимости от значения порога цензурирования или усечения c. На рисунке также изображены функции распределения и плотности для стандартной нормальное величины.



Цель второго подхода заключается в поиске решений, для которых отсутствует необходимость принятия параметрических предпосылок. Этот подход будет рассмотрен в последующих разделах, но следует отметить, что выражение для усеченного среднего это одноиндексная модель с подправочным членом, убывающим по $x'\beta$, поскольку $\E[\varepsilon|\varepsilon>-x'\beta]$ монотонно убывающая функция от $x'\beta$.

\subsection{Цензурированное и усеченное среднее в тобит-модели}

Для тобит-модели остатки регрессии $\varepsilon$ нормально распределены. При дальнейшем рассмотрении будем использовать вывод из раздеда 16.10.1:

Утверждение 16.1 (Усеченные моменты для стандартного нормального распределения): Предположим, что $z \sim N[0,1]$. Тогда моменты усеченных слева значений $z$ равны:


(i) $\E[z|z>c]=\phi(c)/[1-\Phi(c)]$, и $\E[z|z>-c]=\phi(c)/\Phi(c)$,\\
(ii) $\E[z^{2}|z>c]=1+c\phi(c)/[1-\Phi(c)]$, и \\
(iii) $\V[z|z>c]=1+c\phi(c)/[1-\Phi(c)]-\phi(c)^{2}/[1-\Phi(c)]^{2}$ \\

Результат (i) Утверждения 16.1 изображен на рисунке (16.2). Рассмотрим усечение нормально распределенной величины $z \sim N[0,1]$ ниже некоторого значения $c$, где $c$ принадлежит промежутку от $-2$ до $2$. Нижняя кривая --- это функция плотности $\phi(c)$ для нормальной стандартной случайной величины. Средняя линия изображает график функции распределения $\Phi(c)$ в точке $c$ и задает вероятность усечения при усечении в точке $c$. Значение этой вероятности приближенно равно $0.023$ в точке $c=-2$ и $0.977$ при $c=2$. Верхняя линия изображает график усеченного среднего $\E[z|z>c]=\phi(c)/[1-\Phi(c)]$. Как и предполагалось, оно близко к  $\E[z]=0$  при $c=-2$, т.к. усечение невелико и $\E[z|z>c]>c$. Неожиданным оказывается, что $\phi(c)/[1-\Phi(c)]$ приближенно линейна. Значение моментов при усечении сверху можно рассчитать по формуле $\E[z|z<c]=-\E[-z|-z>-c]=-\phi(c)/\Phi(c)$. 

Используя (16.18), получаем, что усеченное среднее случайной ошибки равно:

\begin{multline}
\E[\varepsilon|\varepsilon>{-x}'\beta]={\sigma}\E\left[\dfrac{\varepsilon}{\sigma}|\dfrac{\varepsilon}{\sigma}>\dfrac{{-x}'\beta}{\sigma}\right] \\
=\sigma\phi(-\dfrac{x'\beta}{\sigma})/[1-\Phi(-\dfrac{x'\beta}{\sigma})] \\
=\sigma\phi(\dfrac{x'\beta}{\sigma})/[\Phi(\dfrac{x'\beta}{\sigma})] \\
=\sigma\lambda(\dfrac{x'\beta}{\sigma}),
\end{multline}

где второе равенство следует из утверждения 16.1, для получения третьего равенства использовалось свойство симметрии функции $\phi(z)$ относительно нуля и мы определяем 

\begin{equation}
\lambda(z)=\dfrac{\phi(z)}{\Phi(z)}.
\end{equation}

При определении $\lambda$ мы руководствовались определениями и терминологией, используемых в работе Амэмия (1985), параметр $\lambda(\cdot )$ получил название обратное отношение Миллса. В своей работе Джонсон и Котц (1970, p.278) отметили, что фактически Миллс составил таблицу отношений $(1-\Phi(z))/\phi(z)$ и обратное значение этого отношения $\phi(z)/[1-\Phi(z)]=\phi(z)/\Phi(-z)$, которое задает функцию риска нормального распределения. Ряд авторов записывают (16.21) в другом виде, а именно: $\E[\varepsilon|\varepsilon>-x'\beta]=\sigma\lambda^{*}(-x'\beta/\sigma)$, где $\lambda^{*}(z)=\phi(z)/\Phi(-z)$ обратное отношение Миллса.

Вместе с тем, $\Pr[ \varepsilon>-x'\beta]=\Pr[ -\varepsilon<x'\beta]=\Pr[ -\varepsilon/\sigma<x'\beta/\sigma]=\Phi(x'\beta/\sigma)$. Тогда условное среднее в (16.20) равно:

для скрытой переменной: 

\begin{equation}
\E[y^* |x]=x'\beta
\end{equation},

для моделей c усеченными слева данными (в нуле): 
\[
\E[y|x,y>0]=x'\beta+\sigma\lambda(x'\beta/\sigma)
\]

для моделей с цензурированными слева данными (в нуле): 
\[
\E[y|x]=\Phi(x'\beta/\sigma)x'\beta+\sigma\phi(x'\beta/\sigma)
\]


Дисперсия рассчитывается по аналогичным формулам (см. упражнение 16.1). Пусть $w=x'\beta/\sigma$, тогда 

для скрытой переменной:
\begin{equation}
\V[y^* |x]=\sigma^2
\end{equation}

для моделей с усеченными слева данными (в нуле): 
\[
\V[y|x,y>0]=\sigma^2[1-w\lambda{w}-\lambda(w)^2]
\]

для моделей с цензурированными слева данными (в нуле):
\[
\V[y|x,y>0]={\sigma}^2\Phi(w)\left\lbrace w^2+w\lambda(w)+1-\Phi(w)[w+\lambda(w)]\right\rbrace^2
\]

Очевидно, что усечение и цензурирование приводят к гетероскедастичности ошибок и при усечении $\V[y|x<\sigma^2]$, следовательно, усечение уменьшает изменчивость значений.

Необходимым условием достижения выше обозначенных результатов является выполнение предпосылки о нормальном распределении ошибок. В работе Маддалы (1983, стр. 369) представлены результаты аналогичные утверждению 16.1 для лог-нормального, логистического, экспоненциального, гамма распределений и распределения Лапласа.

\subsection{Предельные эффекты в тобит-модели}

Под предельным эффектом понимают влияние изменений значений регрессоров на условное среднее зависимой переменной. Значение предельного эффекта может меняться в зависимости от интересующего объекта, это может быть среднее значение скрытой переменной, $x'\beta$,  или среднее цензурированных или усеченных данных, определенное в (16.23).

Дифференцируя каждое выражение по $x$, получим следующие результаты:


Скрытая переменная: 
\begin{equation}
\partial \E[y^* |x]/{\partial{x}}=\beta
\end{equation}

модель с данными усеченными слева (от нуля): 
\[
\partial{E}[y,y>0|x]/{\partial}x={1-w\lbrace\lambda}(w)-\lambda(w)^2\rbrace\beta
\]


модель с данными цензурированными слева (от нуля): 
\[
\partial \E[y|x]/\partial x=\Phi(w)\beta
\]



где $w=x'\beta/\sigma$, $\partial\Phi(z)/\partial{z}=\phi(z)$ и $\partial\phi(z)/\partial{z}=-z\phi{(z)}$. Простое выражение для среднего цензурированных данных получается после некоторых преобразований. Можно разложить эффект на два: для $y=0$ и для $y>0$ (см. МакДоналд и Моффитт, 1980).

В некоторых случаях, цензурирование или усечение вызваны особенностями сбора данных, и значение усеченного и цензурированного среднего не представляют интереса сами по себе, а интересно лишь значение $\partial \E[y^* |x ]\partial x=\beta$. Например, при цензурированных сверху данных о зарплате мы хотим рассчитать эффект образования именно на нецензурированное среднее.


В других случаях, усечение и цензурирование имеют поведенческую интерпретацию. Например, при моделировании рабочего времени рассчитываются три предельных эффекта (16.25), а именно: влияние изменения регрессоров на (1) желаемое количество часов работы, (2) фактическое количество отработанных часов для работающих и (3) фактическое количество отработанных часов работающими и неработающими. Для расчета (1) требуется значение оценки $\beta$, для расчета (2) и (3) коэффициенты наклона МНК-регрессии, хотя и являются несостоятельными для $\beta$, могут дать разумную примерную  оценку предельного эффекта, поскольку среднее цензурированных и усеченных данных примерно линейно по $x$.

\subsection{Альтернативные способы оценки тобит-модели}

Следует отметить, что нелинейный метод наименьших квадратов (НМНК) так же, как и метод максимального правдоподобия может дать состоятельную оценку параметров при корректном определении среднего усеченных и цензурированных данных. Далее рассмотрим МНК-оценки и нелинейные МНК-оценки.

\subsubsection*{Нелинейный МНК}

На основе полученных результатов (16.23) с помощью нелинейного МНК могут быть рассчитаны состоятельные оценки параметров тобит-модели. Например, для усеченных данных минимизируется значение функции

\[
S_{N}(\beta,\sigma^2)=\sum^N_{i=1}\left(y_{i}-{x'}_{i}{\beta}-{\sigma}{\lambda}({x'}_{i}{\beta}/{\sigma})\right)^2
\]

по параметрам $\beta$ и ${\sigma}^2$, и далее статистические выводы производятся с учетом  гетероскедастичности (16.24). Аналогичные вычисление могут быть сделаны для цензурированных данных. Аналогичный способ может применяться к цензурированным данным.

На практике нелинейный МНК не используется. Неотъемлемым условием состоятельности оценок является корректная спецификация усеченного среднего, при этом, согласно (16.21), корректная спецификация возможна при выполнении предпосылки о нормальном распределении и гомоскедастичности случайных ошибок. Поэтому оценка $S_N$ может быть рассчитана методом максимального правдоподобия, основанном на строгих предпосылках о распределении, что позволяет получить эффективные оценки. Кроме того, на практике оценка НМНК может быть неточной. Из графика 16.2 можно сделать вывод, что $\lambda(x'\beta/\sigma)$ приближенно линейна по $x'\beta/\sigma$, что приводить к мультиколлинеарности, поскольку $x$ также является регрессором. В разделе 16.5 будут представлены модели, в которых используется поправочный коэффициент похожий на $\sigma\lambda(x'\beta/\sigma)$ в (16.23), но не зависящий от $x$.


\subsubsection*{Двухшаговая процедура Хексмана}

Из (16.23) следует, что среднее усеченных (в нуле) данных  равно

\begin{equation}
\E[y|x]=x'\beta+\sigma\lambda(x'\beta/\sigma).
\end{equation}

Вместо НМНК, если данные цензурированы, значение можно оценить по следующей двухшаговой процедуре. На первом шаге, оцениваем пробит-регрессию $d$ по $x$, где бинарная переменная $d$ равна 1, если $y>0$, в результате получим состоятельные оценки параметра $\hat{\alpha}$, где $\alpha=\beta/\sigma$. На втором шаге, для расчета состоятельных оценок $\beta$ и $\sigma$, оцениваем регрессию по усеченными данными $y$ на $x$ и $\lambda(x'\hat{\alpha})$ с помощью МНК.

Описание и применение процедуры Хекмана (1976, 1979) для моделей с самоотбором выборки приведены  в разделе 16.5.4. 
Вывод формулы для оценки стандартной ошибки $\hat{\beta}$, которая учитывает наличие регрессора $\lambda(x'\hat{\alpha})$ и  гетероскедастичности, вызванной усечением, содержится в разделе 16.10.2.

\subsubsection*{МНК оценка тобит-модели}

МНК оценки коэффициента $\beta$ при усечении и цензурировании несостоятельны. Это вызвано тем, что цензурированное и усеченное среднее в формуле (16.23) не равняются $x'\beta$, что нарушает существенное условие состоятельности МНК оценок.

Для цензурированных данных МНК дает линейную аппроксимацию к нелинейной цензурированной зависимости среднего. Из рисунка 16.1 и (16.25) ясно, что эта кривая более пологая, чем линия регрессии для нецензурированных данных, наклон которой равен истинному угловому коэффициенту. Голдбергер (1981) аналитически доказал, что если $y$ и $x$ имеют совместное нормальное распределение, а цензурирование производится в нуле, то МНК оценки наклона сходятся к истинному угловому коэффициенту, домноженному на $p$, где $p$ --- доля наблюдений с положительными значениями $y$. Жесткие предпосылки данного доказательства были ослаблены в работе Рууд (1986). На практике данное соотношение даёт хорошую меру несостоятельности МНК оценок в тех случаях, когда следует использовать тобит-модель.

Усеченная кривая зависимости среднего также более полога, чем неусеченная. Голдбергер (1981) получил аналогичный случаю цензурированных данных результат. Если $y$ и $x$ имеют совместное нормальное распределение, а усечение производится в нуле, то МНК оценки сходятся к истинному угловому коэффициенту, домноженному на некоторую подправочную константу. Эта подправка имеет громоздкое выражение, лежит от нуля до единицы и одинакова для всех коэффициентов модели. Таким образом, МНК занижает абсолютную величину истинных коэффициентов наклона. 


\subsection{Тесты на спецификацию для тобит-моделей}


Учитывая жесткие предпосылки тобит-модели, хорошей практикой является тестирование предпосылок о распределении. Существует 4 стратегии.

Согласно первому подходу, модель дополняют параметрами и проводят тест Вальда, LR или LM тест. Наиболее простым вариантом является использование LM теста, поскольку требуется оценка только тобит-модели. LM тест против альтернативной гипотезы о гетероскедастичности вида $\sigma_i^2=\exp (x_i'\alpha)$ очень прост. Используя форму внешнего произведения градиентов (см. раздел 7.3.5) находится домноженный на $N$ коэффициент $R^2$ во вспомогательной регрессии $1$ на $s_{1i}$ и $s_2i$. Здесь $f_i=f(y_i|x_i,\beta,\alpha)$ --- плотность, определенная в (16.14), где $\sigma$ заменяется на $\exp (x'\alpha)$, а $s_{1i}=\partial \ln f_i/\partial\beta$ и $s_{2i}=\partial \ln f_{i}/\partial \alpha$.  Тильда обозначает подстановку в  цензурированну тобит-модель зануленных компонент $\alpha$, кроме свободного члена. Аналогичные тест на нормальность гораздо труднее, т.к. не существует общепринятого распределения, обобщающего нормальное.


Во втором подходе проводятся тесты условных моментов (см. раздел 8.2), которые не требуют определения альтернативной модели. В частности, согласно условиям первого порядка (16.16) для цензурированной тобит-регрессии, имеет смысл строить тесты на условные моменты основываясь на обобщенных остатках: 

\[
e_i=d_i\dfrac{y_i-x'_i\beta}{\sigma^2}-(1-d_i)\dfrac{\phi_i}{\sigma(1-\Phi_i)}.
\]

При правильной спецификации тобит-модели $\E[e_i|x_i]=0$, поскольку условия регулярности приводят к тому, что  $\E[\partial{\ln{ f(y_i)}}/\partial\beta]=0$. Тогда можно проверить с помощью М-теста нулевую гипотезу $H_0:\E[ez]=0$ против альтернативной $H_a:\E[ez]{\neq}0$ с помощью $N^{-1}\sum_{i=1}^N{\hat{e}_iz_i}$, где $\hat{e}_i=e_i$ посчитанному в точке  ММП-оценки тобит-модели  $(\hat{\beta};\hat{\sigma}^2)$. 
Из раздела 8.2.2 следует, что тестовую статистику можно посчитать помножив $N$ на $R^2$ вспомогательной регрессии -- единицы на $\hat{e}_iz_i$, $\hat{s}_{1i}$ и $s_{2i}$, где  $f_{i}=f(y_{i}|x_{i},\beta,\sigma^2)$ --- плотность, определенная в (16.14), а $s_{1i}=\partial{ln}f_{i}/\partial\beta$ и $s_{2i}=\partial{ln}f_{i}/\partial\sigma^2$ из (16.16) рассчитаны в точке $(\hat{\beta},\hat{\sigma}^2)$. Переменные $z_i$ и $x_i$ могут не совпадать, тогда тест можно интерпретировать как тест на пропущенные переменные. Также существуют моментные тесты, использующие моменты более высокого порядка. Более подробный анализ представлен у Чешера и Айриш (1987) и Пагана и Велла (1989). 

Третий подход состоит в адаптации диагностических методов и методов тестирования, разработанных для данных по продолжительности жизни цензурированных справа, к нормально распределенным данным цензурированным слева.


Согласно четвертому подходу, вместо ММП-оценивания $\beta$ тобит-модели используется полупараметрические методы, представленные в разделе 16.9, которые являются состоятельными при  более слабых предпосылках о распределении.

Для более глубокого изучения спецификации тестов тобит-моделей можно обратиться к работе Пагана и Велла, в которой рассмотрена теория и практические примеры, а также к работе Меленберга и Ван Суета (1996), где автор более подробно остановился на рассмотрении практических вопросов. Обе работы посвящены вопросу  тестов на спецификацию как для тобит-модели, так и для более общей модели с самоотбором выборки (см. раздел 16.5).


\section{Двухчастная модель}

В ранее рассмотренных моделях с цензурированными данными механизм цензурирования был неразрывно связан с процессом, порождающим значения зависимой переменной. В более общем случае, способ цензурирования и процесс, порождающий зависимую переменную могут быть разделены. Например, для расчета ежегодных затрат индивида на госпитализацию одна модель может быть использована для ответа на вопрос, будет ли госпитализация, вторая --- для моделирования последующих затрат на госпитализацию, если она состоялась. Необходимость использования двух моделей может быть вызвана тем, что некоторые значения получаются слишком часто или, наоборот, слишком редко, чем это характерно для простейшей модели. Например, частота появления нуля может быть выше, чем это характерно для пуассоновского распределения. Возможность генерации нулевых и ненулевых значений при помощи разных законов распределения повышает гибкость. Следует отметить, что двухчастная модель --- это частный случай модели смеси.

Существует два подхода к обобщению, а именно: рассмотренная в этом разделе двухчастная модель описывает механизм цензурирования и процесс, порождающий значения зависимой переменной, для наблюдаемых исходов. Модель с самоотбором выборки, рассмотренная в следующем разделе, задает совместное распределение для индикатора цензурирования и зависимой переменной, из чего можно получить условный закон распределения зависимой переменной для наблюдаемых исходов.  Сравнение этих подходов дано в разделе 16.5.7.

\subsection{Двухчастная модель}

Пусть индивид, данные о котором наблюдаемы называется \textbf{участником} изучаемого процесса. Бинарная переменная $d$ равна 1 для участников и 0 для неучастников. Предположим, что для участников наблюдаемое значение $y>0$, а для тех, кто не обучается $y=0$. Для неучастников, наблюдается только значение $\Pr[ d=0]$. Условная плотность $y$ для участников при $y>0$ задана выражением $f(y|d=1)$, для некоторой плотности $f(\cdot )$. Двухчастная модель для $y$ определена следующей системой:

\begin{equation}
f(y|x)=
\begin{cases}
\Pr[d=0|x], \, \text{ если } y=0, \\
\Pr[d=1|x]f(y|d=1,x) \, \text{ если } y>0.
\end{cases}  
\end{equation}

Эта модель была предложена в работе Крэгга (1971) как обобщение тобит-модели. Действительно, тобит-модель можно определить как частный случай (16.27). Смоделировать индикатор участия $d$ можно с помощью логит или пробит-модели. Скрытая переменная $d$ равна 1, если $I=x'\beta+\varepsilon$ больше нуля и тогда модель может быть рассмотрена как модель преодоления порогов, поскольку пересечение порогового значения будет означать участие. Для того, чтобы обеспечить положительные значения $y$ для участников, плотность $f(y|d=1,x)$ должна быть плотностью положительной случайной величины, например, подойдет лог-нормальное распределение, также можно взять плотность нормально распределенных значений, усеченную левее нуля.

Для простоты, одинаковые регрессоры фигурируют в формуле как на первом, так и на втором шаге, однако это  не обязательно, а иногда даже желательно. Метод максимального правдоподобия легко реализовать, поскольку он позволяет отдельно оценивать модель для учатсия, где используется полный массив данных и оценку параметров плотности $f(y|d=1,x)$ для положительных значений, $y>0$.


\subsection{Пример двухчастной модели}


Дуан и др. (1983) приводят хороший пример применения двухчастной модели для прогнозирования расходов на лечение, используя базу данных Эксперимента по страхованию здоровья корпорации RAND (Rand Health Insurance Experiment). Авторы скомпоновали пробит-модель, которая предсказывает, тратил ли индивид средства на лечение в течении года, где $\Pr[ d=1|x]=\Phi(x'_1+\beta_{1})$, и модель затрат на лечение с лог-нормальным распределением ненулевых затрат, заданную уравнением $\ln{ y|d}=1, x \sim N[x'_2\beta_{2},\sigma^{2}_{2}]$. Тогда ожидаемое значение затрат на лечение по всей выборки равно

\begin{equation}
\E[y|x]=\Phi(x'_1\beta_1)\exp [\sigma^{2}_2/2+x'_2\beta_2],
\end{equation}

где второй член выражения получается в силу того, что из $\ln{ y} \sim N[\mu,\sigma^2]$ следует, что $\E[y]=\exp (\mu+\sigma^{2}/2)$. Более подробно вопрос преобразований в этой модели рассмотривает в своей работе Маллахай (1998).

Двухчастная модель часто используется для моделирования счетных данных. Например, при моделировании количества визитов к доктору одна модель позволяет определить посещал ли пациент врача, а вторая модель рассчитывает количество повторных визитов только для индивидов, которые имеют хотя бы по одному визиту к врачу. Тогда $\Pr[ d=1]$ задается как  вероятность того, что переменная с пуассоновским или отрицательным биномиальным распределением окажется больше нуля, в то время как вероятность $f(y|d=1)$ определяется как вероятность пуассоновского или отрицательного биномиального  распределения, усеченная снизу. В литературе по счетным данным, эта модель, предложенная в работе Маллахай (1986), получила название модель преодоления порогов. Она подробно рассматривается в разделе 20.4.5. 

Двухчастная модель с непрерывными данными используется, например, для оценки модели затрат с большим количеством нулевых данных (идея предложена Крэггом). В следующем разделе рассмотрен альтернативный метод --- модели с самоотбором выборки.


\section{Модели с самоотбором выборки}


Самоотбор наблюдений может возникнуть в разных ситуациях. В настоящее время разработано много моделей с самоотбором выборки. Прежде чем приводить примеры практического применения, рассмотрим теоретическую сторону вопроса. К самым известным моделям с самоотбором выборки относят бинарную модель выбора Хекмана (1979) и модель Роя (см. раздел 16.7).


\subsection{Модели с самоотбором выборки}


Исследования по наблюдаемым данным  редко основаны на чисто случайных выборках. Как правило используется экзогенно заданная выборка (см. раздел 3.2.4) и, следовательно, могут быть применены стандартные методы. В случае если выборка, намеренно или случайно, сформирована на базе  значений зависимой переменной, оценки параметров будут несостоятельными, только если не будут применены коррекционные меры. Такие выборки называются выборками с самоотбором наблюдений.


Существует много моделей для  выборок с самоотбором, поскольку существует много механизмов самоотбора. Вместе с тем, можно не знать, имел ли самоотбор наблюдений место. Например, рассмотрим интерпретацию среднего балла за тестирование по успеваемости, такого, как Экзамен на определение академически способностей (Scholastic Aptitude Test), когда участие добровольно. Понижение среднего балла может быть следствием ухудшения знаний студентов. Понижение также может быть обусловлено тем, что большее количество студентов стали участвовать в тесте и появились участники с более низким баллом.

Самоотбор выборки может быть результатом решения конкретных индивидов принимать или не принимать участие в изучаемой деятельности. Самоотбор выборки также может быть результатом смещения выборки, например, в выборку может попасть существенно больше участвующих в изучаемой деятельности. В экстремальном случае в выборку попадают только участники деятельности. В любом случаем проблемы получаются сходные, и данное множество моделей называют моделями с самоотбором выборки.


В этой главе рассматриваются только три модели отбора наблюдений. Самая простая это тобит-модель, см. раздел 16.3. Базовая широко используемая модель, получившая название двумерной модели самоотбора (bivariate sample selection model), рассмотрена в этом разделе. Модель двумерного самоотбора обобщает тобит-модель путем введения скрытой переменной цензурирования, не совпадающей со скрытой переменной, определяющей наблюдаемые значения зависимой переменной. В разделе 16.7 рассмотрена другая популярная модель --- модель Роя. В модели Роя выбор одного из двух возможных значений зависимой переменной определяется цензурирующей переменной. Перечисленные модели по терминологии Амэмия (1985, стр. 384), называются тобит-моделями 1,2 и 5.


Состоятельность оценки требует сильных предпосылок о законах распределения даже в случае полупараметрических методов. Исследования на экспериментальных данных являются альтернативой моделям самоотбора наблюдений, поскольку случайный характер выборки может решить проблему самоотбора. Однако, использование экспериментальных данных может быть затруднено высокими издержками на его проведение и рядом других причин. В главе 25 подробно рассмотрено оценивание эффектов воздействий, при этом к наблюдаемым данным применяется экспериментальный подход.


\subsection{Модель двумерного самоотбора выборки, тобит-2}


Пусть $y^*_{2}$ обозначает интересующий нас результат. В стандартной тобит-модели с усеченными данными результат известен только при $y^* _{2}>0$. В моделях общего класса вводится дополнительная латентная переменная $y^* _1$ и результат $y^* _2$ известен, только при $y^* _{1}>0$. Например, $y^* _1$ --- это решение, работать или не работать, а $y^* _2$ определяет количество рабочих часов, при этом $y^* _1{\ neq}y^* _2$, поскольку существуют фиксированные издержки, например, ежедневные затраты на проезд, которые более значимы для решения работать или нет, чем количество желаемых рабочих часов.

Модель двумерного  выбора включает уравнение участия,

\begin{equation}
y_1=
\begin{cases}
1, \text{ если } y_1^{*}>0, \\
0, \text{ если } y_1^{*} \leq 0
\end{cases}
\end{equation}

и уравнение наблюдаемой переменной:

\begin{equation}
y_2=
\begin{cases}
y_2^{*}, \text{ если } y_1^{*}>0, \\
-, \text{ если }y_1^{*} \leq 0
\end{cases}
\end{equation}

Согласно этой модели $y_2$ наблюдается, только при $y^* >0$. При $y^* _{1} \leq 0$ величина $y_2$ не обязана иметь смысл. В стандартной модели определяется линейная зависимость для латентной переменной и аддитивными ошибками:

\begin{equation}
\begin{array}{l}
y_1^{*}=x'_{1}\beta_1+\varepsilon_1, \\
y_2^{*}=x'_{2}\beta_2+\varepsilon_{2},
\end{array}
\end{equation}

при наличии корреляции между $\varepsilon_1$ и $\varepsilon_2$ появляются проблемы при оценке $\beta_2$. Тобит-модель является частным случаем этой модели, когда $y^* _1=y^* _2$.

Для этой модели отсутствует общепринятое название. Хекман (1979) использовал эту модель для демонстрации результатов оценки регрессии при самоотборе наблюдений. 
По своей спецификации модель эквивалентна тобит-модели со стохастическим пороговым значением (Нельсон, 1977). 
Предположим, что значение $y^* _2$ наблюдаемо, если $y^* _{2}>L^{*}$, где $y^* _2$ определено в (16.31) и пороговое значение задается выражением $L^{*}=z'\gamma+u$ вместо $L^{*}=0$, как в разделе 16.3. Тогда, равнозначно, что $y^* _2$ наблюдаема, если $y^* _1>0$ и $y^* _{1}=y^* _{2}-L^{*}=(x'_2\beta_{2}-z'\gamma)+(\varepsilon_2-\nu)=x'_{1}\beta_{1}+\varepsilon_1$ и где $x_1$ объединяет $x_2$ и $z$, и $\beta_1$ и $\varepsilon_1$ определяются естественным образом. Амэмия (1985, стр. 384) назвал эту модель тобит-модель второго типа. Вулдридж (2002, стр. 506) предложил название модель с пробит-уравнением выбора. В других работах можно встретить названия обобщенная тобит-модель или модель самоотбора выборки.

Естественным способом оценки является ММП метод, при условии, что коррелированные ошибки имеют совместное нормальное распределение и гомоскедастичны:

\begin{equation}
\begin{bmatrix}
\varepsilon_1\\ \varepsilon_2
\end{bmatrix}
\sim
N
\left[
\begin{bmatrix}
0 \\ 0
\end{bmatrix},
\begin{bmatrix}
1 &\sigma_{12} \\ \sigma_{12}&\sigma_{22}^2
\end{bmatrix}
\right]
\end{equation}

Как и для пробит-модели в разделе 14.4.1, нормируем дисперсию $\sigma^{2}_1=1$, поскольку известен только знак $y^* _1$. 

Из (16.29) и (16.30) следует, что для  $y^* >0$ вероятность значения $y^* _2$ равна вероятности того, что $y^* >0$ умноженной на вероятность $y^* _2$ условную по $y^* >0$. Следовательно, для положительных значений $y_2$ значение плотности равно $f^{*}(y^* _2|y^* _1>0){\times}\Pr[ y^* _1>0]$. Для события $y^* _{1} \leq 0$ мы знаем только лишь, что оно произошло и вероятность равна $\Pr[ y_{1}^{*} \leq 0]$. Следовательно, функция правдоподобия для модели двумерного самоотбора имеет вид:

\begin{equation}
L=\Pi_{i=1}^n{\lbrace \Pr[ y_{1i}^{*} \leq 0]\rbrace}^{1-y_i}{\lbrace f(y_{2i}|y_{1i}^{*}>0){\times}\Pr[ y_{1i}^{*}>0]\rbrace}^{y_{1i}}
\end{equation}

где первый множитель --- это дискретная составляющая при $y^* _{1i} \leq 0$, поскольку тогда $y_{1i}=0$, и второй множитель --- это непрерывная составляющая, для $y^* _{1i}>0$. Эта функция правдоподобия может использоваться не только для линейных регрессий с нормально распределенными ошибками, но и при оценке более широкого класса моделей.

Для линейной регрессии с совместно  нормально распределенными ошибками двумерная плотность  $f^{*}(y^* _1,y^(*)_2)$ является нормальной, поэтому условная функция плотности второго аргумента является одномерной нормальной и с ней легко работать. Подробный анализ можно найти у Амэмия (1985, стр. 385-387), в том числе точную форму функции правдоподобия. 

Изначально эта модель использовалась для анализа предложения труда, где $y^* _1$ ненаблюдаемое желание работать или способность к труду, а $y_2$ фактически отработанное количество часов. Модель больше подходит к оценке предложения труда, поскольку в отличие от тобит-модели не требует искусственной переменной <<желаемых>> часов работы. При анализе предложения труда возникала трудность связанная с неизвестной зарплатой для неработающих индивидов.
Эту трудность можно преодолеть путем добавления уравнения для предлагаемой заработной платы и, далее, подставить данное значение в исходное уравнение, хотя модель при этом перестает быть моделью двумерного самоотбора. Хороший пример оценки предложения труда рассмотрен в работе Мроз (1987).

\subsection{Условное среднее в модели двумерного самоотбора}

В этом разделе рассмотрим условное среднее усеченных данных модели двумерного самоотбора. Значение среднего не равно $x'_{2}\beta_2$, а МНК регрессия $y_2$ на $x_2$ дает несостоятельные оценки параметров. Тем не менее, выражение для условного среднего может быть использовано в качестве мотивации для использования альтернативного метода оценивания, рассматриваемого в последующем разделе и использующего более слабые предпосылки о распределении чем для ММП. 

Рассмотрим усеченное среднее в модели двумерного самоотбора выборки, где используются только положительные значения $y_2$. Общий вид модели:

\begin{equation}
\E[y_2|x,y_1^{\ast}>0]=\E[x_2^{\prime}\beta_2+\varepsilon_2|x'\beta_1+\varepsilon_1>0] = x_2'\beta_2+\E[\varepsilon_2 | \varepsilon_1>x_1'\beta_1]],
\end{equation}

где $x$ объединяет $x_1$ и $x_2$. Если случайные ошибки не зависят друг от друга, тогда последнее слагаемое упрощается и принимает вид: $\E[\varepsilon_2]=0$ и МНК-регрессия $y_2$ на $x_2$ даст состоятельные оценки $\beta_2$. Однако, если существует зависимость между  ошибками, тогда усеченное среднее не равно $x'\beta_2$, и необходимо учитывать механизм отбора. 

Для расчета $\E[\varepsilon_2|\varepsilon_1>-{x'}_1\beta_1]$, где $\varepsilon_1$ и $\varepsilon_2$ коррелированы Хекман (1979) отметил, что если ошибки $(\varepsilon)1,\varepsilon_2)$ в (16.31) имеют совместное нормальное распределение как в (16.32), тогда для уравнения (16.36) справедливо, что 

\begin{equation}
\varepsilon_2=\sigma_{12}\varepsilon_1+\xi,
\end{equation} 

где случайная переменная $\xi$ независима от $\varepsilon_1$. Для получения этого результата заметим, что из совместного нормального распределения

\[
\begin{bmatrix}
z_1\\z_2
\end{bmatrix}
\sim
N
\left[
\begin{bmatrix}
{\mu}_1\\{\mu}_2
\end{bmatrix},
\begin{bmatrix}
\Sigma_{11}&\Sigma_{12} \\ \Sigma_{21}&\Sigma_{22}
\end{bmatrix}
\right],
\]

следует нормальность условного распределения

\[
z_2|z_1 \sim N[\mu_2+\Sigma_{21}{\Sigma^{-1}}_{11}(z_{1}-\mu_1),\Sigma_{22}-\Sigma_{21}{\Sigma^{-1}}_{11}\Sigma_{12}],
\]

следовательно

\begin{equation}
z_2=\mu_2+\Sigma_{21}{\Sigma^{-1}}_{11}(z_1-\mu_1)+\xi,
\end{equation}

где $\xi \sim N[0,\Sigma_{22}-\Sigma_{21}{\Sigma^{-1}}\Sigma_{12}]$ независима от $z_1$. Для  совместной плотности распределения (16.32) известно, что $\mu_1=\mu_2=0$ и ${\sigma^{2}}_1=1$, следовательно, выражение (16.36) приводит к  (16.35).

Подставляя (16.35) в формулу для усеченного среднего, (16.34), получим:

\[
\E[y_2|x,y^* _1>0]=x'_2\beta_2+\E[(\sigma_{12}\varepsilon_1+\psi)|\varepsilon_1>-x^{-1}\beta_1]=x'_{2}\beta_2+\sigma_{12}\E[\varepsilon_1|\varepsilon_1>-x'_1\beta_1],
\]
 
при этом используется свойство независимости $\psi$ и $\varepsilon_1$. Составляющая самоотбора  аналогична простой тобит-модели и используя выражение для $\E[z|z>{-c}]$ из Утверждения 16.1 получаем:


\begin{equation}
\E[y_2|x,{y^* }_1>0]={x'}_2\beta_2+\sigma_{12}\lambda({x'}_1\beta_1),
\end{equation}

где $\lambda(z)=\phi(z)/\Phi(z)$ и мы воспользовались тем, что ${\sigma}_1=1$. Вместе с тем, результат 16.1 (iii) позволяет выразить дисперсию усеченных данных

\begin{equation}
\V[y_2|x,{y^* }_1]={\sigma^2}_2-{\sigma^2}_{12}\lambda({x'}_1\beta_1)({x'}_1\beta_1+\lambda({x'}_1\beta_1)).
\end{equation}

В предыдущем анализе не указывалось значение $y_2$ при ${y^* } \leq 0$. В некоторых случаях, при ${y^* }<0$ значение $y_2$ может быть равно нулю. В таком случае осмысленно рассматривать среднее значение усеченных данных. 


Рассчитаем условное среднее $y_2$ при фиксированных $y^* _1$ и $y^* _2$, а затем возьмем безусловное:

\begin{multline}
\E[y_2|x]=E_{{y^* }_1}[\E[y_2|x,{y^* }_1]] \\
=\Pr[ {y^* }_1 \leq 0|x]{\times}0+\Pr[ {y^* }_1>0|x]{\times}\E[{y^* }_2|x,{y^* }_2>0] \\
=0+\Phi({x'}_1\beta_1)\left\lbrace{x'}_2\beta_2+\sigma_{12}\lambda\left({x'}_1\beta_1\right)\right\rbrace \\
=\Phi({x'}_1\beta_1){x'}_2\beta_2+\sigma_{12}\phi({x'}_1\beta_1),
\end{multline}

где третья строка получается на основе соотношения (16.37) и последняя строка на основе $\lambda(z)=\phi(z)/{\Phi}(z)$. Можно показать, что дисперсия цензурированных данных гетероскедастична.


\subsection{Двухшаговая оценка Хекмана}

МНК-оценки регрессии $y_2$ на $x_2$ при использовании только наблюдаемых и положительных значений $y_2$ дает несостоятельные оценки $\beta$, за исключением случая некоррелированных ошибок, т.е. $\sigma_{12}=0$. Это следует из формулы среднего усеченных данных (16.37), где  присутствует дополнительный <<регрессор>> $\lambda(x'_1\beta_1)$. 

Иногда двухшаговую оценку Хекмана называют хекит-оценкой (Heckit estimator), где в МНК регрессию добавляется пропущенная переменная $\lambda(x'_1\beta_1)$. Таким образом, используя положительные значения $y_2$ с помощью МНК оценивается  регрессия:

\begin{equation}
y_{2i}=x'_{2i}\beta_2+\sigma_{12}\lambda(x'_{1i}\hat{\beta}_1)+\nu_i,
\end{equation}

где $\nu$ случайная ошибка, $\hat{\beta}_1$ оценка, полученная на первом шаге пробит- регрессии $y_1$ на $x_1$, поскольку $\Pr[ y^* _1>0]=\Phi({x'}_1\beta_1)$ и $\lambda(x'_1\hat{\beta}_1)=\phi(x'_1\hat{\beta}_1)/{\Phi}(x'_1\hat{\beta}_1)$ оценка обратного отношения Миллса. Данная модель не дает прямой оценки дисперсии $\sigma^2_2$, но из формулы усеченной дисперсии (16.38) можно получить оценку $\hat{\sigma}_2^2=N^{-1}\sum_i[\hat{v}_i^2+\hat{\sigma}_{12}^2\hat{\lambda}_i(x_1'\hat{\beta}_1+\hat{\lambda}_i)]$, где $\hat{v}_i$ МНК остатки  регрессии (16.40) и $\hat{\lambda}_i=\lambda(x_{1i}'\hat{\beta}_1)$. Корреляция между ошибками в (16.32) может быть оценена по формуле $\hat{\rho}=\hat{\sigma}_{12}/\hat{\sigma}_2.$

Тестирование гипотезы $\sigma_{12}=0$ или $\rho=0$ есть проверка наличия корреляции ошибок и необходимости подправки вызванной самоотбором наблюдений. Одним из таких тестов является тест Вальда, основу которого составляет обратное отношение Миллса, $\hat{\sigma}_{12}$. 

Следует отметить, что как классические МНК стандартные ошибки коэффициентов, так и робастные к гетероскедастичности стандартные ошибки коэффициентов для регрессии (16.40) некорректны. Для правильного расчета стандартных ошибок на втором этапе оценивания нужно учесть два фактора. Во-первых, даже если $\beta_1$ известно ошибки в (16.40) гетероскедастичны, что следует из (16.38). Во-вторых, вместо $\beta_1$ используется оценка параметра, эта трудность рассмотрена в разделе 6.6, а применение к обычной тобит-модели --- в разделе 16.10.2. Корректные формулы для расчёте стандартных ошибок даны в работе Хекмана (1979); см. также Грин (1981). Вывод формулы для случая простой тобит-регрессии рассмотрен в разделе 16.10.2. Реализация формулы может вызвать трудности, поэтому лучше воспользоваться готовыми возможностями статистического пакета или использовать бутстрэп.

Полученная оценка $\beta_2$ состоятельна . Несмотря на потерю эффективности по сравнению с ММП-оценкой при предпосылки о совместном нормальном распределении ошибок, метод популярен по ряду причин: 1) прост в применении; 2) применим к моделям самоотбора выборки, в частности, к моделям из раздела 16.7; 3); 3) используются более слабые предпосылки о распределении, чем  совместное нормальное распределения случайных ошибок $\varepsilon_1$ и $\varepsilon_2$; и 4) предпосылки о распределении можно ослабить еще больше и воспользоваться полупараметрическими методами, см. раздел 16.9.

Предпосылка, которая всегда должна выполняться --- это (16.35), в общем виде может быть записана:

\begin{equation}
\varepsilon_2=\delta\varepsilon_1+\xi,
\end{equation}

где $\xi$ независима от $\varepsilon_1$. Эта модель довольно осмысленна. Согласно этой модели, например, случайная ошибка в уравнении расходов на товары длительного пользования кратна ошибке в уравнении принятия решения о покупке с учетом некоторого шума. Фактически уравнение (16.41) --- это линейная регрессия для случайных ошибок.
Учитывая предположение (16.41) условное среднее (16.34) можно записать как:

\begin{equation}
\E[y_2|y_1^{*}>0]=x_2'\beta_2+\delta \E[\varepsilon_1|\varepsilon_1>-x_1^{*}\beta_1].
\end{equation}

Если $\varepsilon_1$ имеет стандартное нормальное распределение, тогда мы получаем (16.37), основание для МНК регрессии (16.40).

В общем случае, уравнение (16.42) можно оценить методом Хекмана, если ошибки $\varepsilon_1$ имеют распределение, отличное от нормального; см. например Олсен (1980). Вместе с тем, можно полупараметрические методы, при этом  не накладываются ограничения на функциональную форму $\E[\varepsilon_1|\varepsilon_1>-x_1'\beta_1]$ (см. раздел 16.9). 

\subsection{Идентификация}

Теоретически, модель двумерного самоотбора с нормально распределенными ошибками может быть идентифицирована без дополнительных ограничений на регрессоры. В частности, одни и те же регрессоры могут использоваться для обеих переменных $y_1^{*}$ и $y_2^{*}$. 

Вместе с тем, если в модели с нормально распределенными ошибками в обоих уравнениях используются ровно одни и те же регрессоры, модель близка к  неидентифицируемой. Если $x_1=x_2$, тогда, используя (16.37) и вывод из раздела 16.3.2 о том, что обратное отношение Миллса, $\lambda(\cdot )$, можно приближенно описать как линейное по параметрам, $\E[y_2|y_1^{*}]{\simeq}x_2'\beta_2+a+bx_2'\beta_1$. Идентичность регрессоров приводит к мультиколлинеарности. Анализу мультиколлинеарности посвящено много работ среди которых Навата (1993), Навата и Нагасэ (1996) и Льюнг и Ю (1996). Мультиколлинеарность можно обнаружить используя параметр обусловленности, приведенный в разделе 10.4.2, где как следует из (16.40) регрессорами являются $x_2$ и $\lambda(x'_1\hat{\beta}_1)$. Проблема оказывается менее ярко выраженной, если велика изменчивость $x'_1\hat{\beta}_1$, т.е. если пробит-модель может хорошо различать участников и неучастников.

Полупараметрическая версия двухшаговой процедуры Хекмана (см. раздел 16.9.3) требует дополнительного исключающего ограничения. То есть идентификация в модели двумерного самоотбора достигается за счет предположений о функциональной форме.

Таким образом на практике может потребоваться чтобы в модели двумерного самоотбора один из регрессоров уравнения участия ($y_1^*$) не включен в уравнения наблюдаемой зависимой переменной ($y_2^*$). Например, фиксированные издержки, не зависящие от количества часов работы будут влиять на решение работать или нет, но не будут влиять на количество рабочих часов. Это исключение регрессоров может быть довольно сильным ограничением в некоторых приложениях, см. раздел 16.6, т.к. зачастую трудно сформулировать осмысленные ограничения.

\subsection{Предельные эффекты}

Предельные эффект в модели двумерного самоотбора зависят от того, рассматриваем ли мы среднее скрытой переменной, усеченное среднее из (16.37) или цензурированное среднее (там где оно имеет смысл).

Удобно обозначить за $x$ объединение регрессоров $x_1$ и $x_2$, и переобозначить $x_1'\beta_1$ как $x_1'\gamma_1$ и $x_2'\beta_2$ как $x_2'\gamma_2$. Например, усеченное среднее записывается как $\E[y_2|x]=x'\gamma_2+\sigma_{12}\lambda(x'\gamma_1)$. Заметим что в $\gamma_1$ и $\gamma_2$ будут нулевые элементы если $x_1\neq x_2$. Дифференцируя по $x$  получаем предельные эффекты:

Для нецензурированного среднего:
\begin{equation}
\partial \E[y^*_2|x] /\partial x=\gamma_2
\end{equation}

Для усеченного среднего (в нуле):
\[
\partial \E[y_2|x,y_1=1] /\partial x=\gamma_2-\sigma_{12}\lambda(x'\gamma_1)(x'\gamma_1+\lambda(x'\gamma_1))
\]

Для цензурированного среднего (в нуле):
\[
\partial \E[y_2|x] /\partial x=\gamma_1 \phi(x'\gamma_1)x'\gamma_2+
\Phi(x'\gamma_1)x'\gamma_2-\sigma_{12}x'\gamma_1 \phi(x'\gamma_1)\gamma_1
\]

где $\lambda(z)=\phi(z)/\Phi(z)$ и мы пользуемся тем, что $\partial \phi(z)/\partial z=-z\phi(z)$ и $\partial \lambda(z)/\partial z=-z\phi(z)/\Phi(z)-\phi(z)^2/\Phi(z)^2=-\lambda(z)(z+\lambda(z))$. Интерпретация этих производных аналогична обсуждавшейся ранее в Разделе 16.5.3. Как было отмечено, анализ цензурированного среднего корректен если $y_2$ принимает нулевое значение при $y_1=0$. В приложениях рассматриваемых далее, например для случая логнормальных расходов на здоровье, цензурированное среднее не используется.

\subsection{Самоотбор по наблюдаемым и ненаблюдаемым переменным}

Существует много ситуаций, которые могут быть проанализированы с помощью двухшагового процесса принятия решений, сначала принимается решение об участии, а затем принимается решение об уровне участия. Эти решения тесно связаны между собой и зависят от общих факторов. Естественной моделью для таких ситуация является модель двумерного самоотбора, (16.29)-(16.31).

После включения регрессоров ошибки ($\varepsilon_1$ и $\varepsilon_2$) в двух уравнениях могут оказаться некоррелированы. Например, в модели госпитализации, после учета индивидуальных характеристик может оказаться, что отсутствует корреляция между ошибкой в уравнении отвечающем за госпитализацию и ошибкой в уравнении, определяющем длину госпитализации. В этом случае анализ упрощается, т.к. самоотбор происходит только на основе наблюдаемых данных, например, (16.37) упрощается до $\sigma_{12}=0$. Два явления могут моделироваться независимо друг от друга и может быть использована более простая двухчастная модель из раздела 16.4.

В других случая ошибки могут быть коррелированы даже после введения регрессоров. Например, в модели предложения труда, ненаблюдаемые факторы способствующие тому, что индивид принимает решение работать, могут также способствовать тому, что он принимает решение работать дольше, чем предсказывается наблюдаемыми регрессорами. Можно протестировать наличие данной корреляции между ошибками. Если корреляция есть, то на сцену выходят методы данной главы. Требуются довольно сильные предположения о распределении даже для двухшаговой процедуры Хекмана.

Исследование Дуана и др. (1983), изложенное в 16.4.2, критиковали за использование двухчастной модели, более ограничивающей, чем модель с самоотбором выборки. Работа вызвала широкую дискуссию, список релевантных статей можно найти в работе Льюнга и Ю (1986), которые подчеркивают важную роль корреляции обратного отношения Миллса и регрессоров.

Модели с самоотбором выборки, такие как модель с двумерным самоотбором, могут рассматриваться как модели, в которых самоотбор возможен как по наблюдаемым, так и по ненаблюдаемым регрессорам. Часто говорят про модели с самоотбором по ненаблюдаемым переменным, опуская неявно самоотбор по наблюдаемым переменным. Эта глава как раз адресована анализу этих моделей.

Если же самоотбор происходит только по наблюдаемым переменным, анализ существенно упрощается. В качестве примера можно рассмотреть двухчастную модель из этой главы. Глава 25 об эффектах воздействия сосредоточена на самоотборе по наблюдаемым переменным (см. обсуждение в разделе 25.3.3). 

\section{Модель самоотбора выборки: оценка затрат на здоровье}

Для иллюстрации метода используем данные Эксперимента по страхованию здоровья корпорации RAND (Rand Health Insurance Experiment). Данные взяты из статьи Деба и Триведи (2002), авторы использовали модель счетных данных для анализа посещений амбулаторных больных к лечащему врачу и другим специалистам. В разделе 20.3 обобщаются данные, а в разделе 20.7 представлены выводы по некоторым стандартным счетным моделям.


В этом разделе модель строится для ежегодных затрат на здоровье. Регрессоры определены в Таблице 20.4 и могут быть разбиты на несколько групп: медицинское страхование (LC, IDP, LPI и FMDE), социально-экономические характеристики (LINC, LFAM, AGE, FEMALE, CHILD, FEMCHILD, BLACK и EDUCDEC) и индикаторы состояния здоровья (PHYSLIM, NDISEASE, HLTHG, HLTHF и HLTHP). В главе 20 анализ строится на данных за 4 года, в этом примере данные взяты только для второго года, что дает 5574 наблюдения и описательные статистики похожи на результаты в Таблице 20.4, но в точности не совпадают.

Зависимая переменная $y$ обозначает ежегодные затраты индивида на здоровье. При построении эконометрической модели необходимо учитывать два фактора: 1) затраты на здоровье равны нулю для $23.2\%$ наблюдений и 2) распределение $y$ скошенно вправо, среднее значение равно $221 $долл. при медиане $53 $долл. Логарифмирование затрат позволяет понизить скошенность, для логарифма  среднее равно $4.07$, а медиана $3.96$; статистика скошенности сокращается от 24.0 до 0.3. Куртозис равен $3.29$, что примерно равно нормальному значению, $3$. 

Далее основное внимание будет сосредоточено на моделировании $\ln{ y}$, где $y$ положительная величина. Затраты на здоровье можно также оценивать используя двухчастная модель, см. раздел 16.4.2. и модель двумерного самоотбора, см. раздел 16.5.2., где $y_1$ в (16.29) индикатор положительного значения затрат, а $y_2$ в (16.30) равен $\ln{ y}$. Следует отметить, что при $y_1=0$ $y_2$ не имеет смысла, поскольку значение $\ln{ 0}$ неопределено. Двухчастная модель  --- это частный случай модели двумерного самоотбора, когда $\sigma_{12}=0$, (16.32).



\begin{table}[h!]
\caption{\label{tab:16.1} Затраты на здоровье: результаты оценок двухчастной модели и модели самоотбора выборки}
\begin{center}
\begin{tabular}{lcccccc}
\hline
\hline
& \multicolumn{2}{c}{Двухчастная} & \multicolumn{2}{c}{Двухшаговая, самоотбор} & \multicolumn{2}{c}{ММП, самоотбор} \\
Модель & DMED & LNMED & DMED & LNMED & DMED & LNMED \\
\hline
LC & -0.119 (-4.41) & -0.016 (-0.52) & -0.119 (-4.41) &   -0.028 (-0.70)  & -0.107 (-4.03) & -0.076 (2.25) \\
IDP & -0.128 (-2.45) & -0.079 (-1.28) & -0.128 (-2.45) & -0.028 (-0.70) &  -0.109 (-2.13) & -0.150 (-2.26) \\
LPI & 0.028 (3.19)  &  0.003 (0.28)  &  0.028 (3.19)  & 0.005 (0.47)  & 0.029 (3.42)  & 0.015 (1.42) \\
FMDE & 0.008 (0.47)  &  -0.031 (-1.69) &   0.008 (0.47)  & -0.030 (-1.62) &  0.001 (0.05) & -0.024 (1.21) \\
PHYSLIM & 0.273 (3.67)  &  0.262 (3.81)  &  0.273 (3.67)  &  0.281 (3.50) & 0.285  (3.94) & 0.355 (4.70) \\
NDISEASE  & 0.022 (6.25)  &  0.020 (5.78)  &  0.022 (6.25)  & 0.022 (4.29) & 0.021  (6.03) & 0.029 (7.54) \\
HLTHG &  0.039 (0.88)  &  0.144 (2.97)  &  0.039 (0.88)  &  0.147 (3.01) & 0.058  (1.35) & 0.156 (2.99) \\
HLTHF & 0.192 (2.29)  &  0.364 (4.13)  &  0.192 (2.29)   &   0.382 (3.98) & 0.224  (2.75) & 0.445 (4.66) \\
HLTHP & 0.640 (3.01)  &  0.787 (4.63)  &  0.640 (3.01)   &   0.833 (4.22) & 0.798  (3.90) & 0.999 (5.32) \\
$\rho$ & & 0.000 & & 0.168 & & 0.736 \\
$\sigma_2$ & &  & & 1.401 & & 1.570 \\
$\sigma_{12}=\rho \sigma_2$ & & 0.000 & & 0.236 (0.47) & & 1.155 (16.43) \\
$-\ln L$ & \multicolumn{2}{c}{10184.1} & & & \multicolumn{2}{c}{10170.1} \\
\hline
\hline
\end{tabular}
\end{center}
\end{table}

В скобках приведены $t$-статистики. Регрессоры также включают социо-экономические характеристики. DMED --- это индикатор положительности расходов на здоровье. LNMED --- это логарифм расходов, если они положительны. В случае двухшаговой модели $t$-статистики второго шага основаны на ошибках, учитывающих получение оценок обратного отношения Миллса на первом шаге.


В таблице 16.1 показаны оценки коэффициентов для группы факторов медицинского страхования и индикаторов состояния здоровья. Оценки коэффициентов группы социально-экономических характеристик сокращены для краткости изложения.

Для начала сравним результаты двухчастной модели и модели двумерного самоотбора с двухшаговой процедурой оценки. Оценки DMED идентичны, поскольку используется пробит-модель с одними и теми же регрессорами. Вместе с тем, оценки LMED отличаются, поскольку в модели двумерного самоотбора на втором шаге в МНК регрессию добавляется предсказанное значение обратного отношения Миллса. Этот дополнительный регрессор статистически незначим $(t=0.47)$ и его абсолютное значение мало, а $\hat{\rho}$ близко к нулю и равно $0.168$. В результате две модели дают аналогичные оценки коэффициентов для LNMED.


Как было отмечено в разделе 16.4.4 двухшаговый метод может не дать хороших результатов, если обратное отношение Миллса сильно коррелировано с другими регрессорами. В рассматриваемом примере результаты корректны, поскольку прогнозное значение вероятности может меняться в диапазоне от $0.15$ до $0.99$, а индекс обусловленности (см. раздел 10.4.4) для регрессоров на втором шаге	увеличивается лишь в два раза после добавления обратного отношения Миллса. Несмотря на то, что для улучшения качества оценок, как правило, накладываются ограничения исключения на параметры, для рассматриваемого примера трудно установить какие именно факторы  из DMED уравнения следует исключить из уравнения LNMED.

ММП-оценка модели двумерного самоотбора выборки, значительно отличается от двух предыдущих, как для DMED, так и для LNMED уравнения. Случайные ошибки скрытых переменных DMED и LNMED высоко коррелированы, оценка корреляции  равна  $\hat{\rho}=0.736$ и коэффициент корреляции статистически значим $(t=16.43)$. 
Большая разница между оценками $\sigma_{12}$ (или $\rho$), полученными двухшаговой процедурой и ММП-спасобом, может означать неадекватность модели двумерного самоотбора. Отвергание нулевой гипотезы теста Хаусмана (см. раздел 8.4) согласно которой оценки параметров имеют одинаковый предел по вероятности, может проинтерпретировать как отвержение гипотезы о совместном нормальном распределении ошибок, необходимой при переходе от двухшаговой процедуры к оцениванию ММП. 
Также возникает более фундаментальный вопрос об адекватности предпосылки (16.41) и предпосылки о независимых и нормально распределенных ошибках $\varepsilon_1$. Неустойчивость довольно часто присуща моделям двумерного самоотбора, особенно, если одни и те же параметры используются в обеих частях модели. Кроме того, в модели затрат на здоровье предпосылка о нормальном распределении остатков может не выполняться из-за вероятности больших выбросов. Несмотря на то, что скошенность LNMED близка к нулю, а куртозис близок к $3$, стандартные тесты на гетероскедатичность, скошенность и куртозис отвергают нулевую гипотезу о нормальном распределении LNMED, с P-значением равным 0.000.


\section{Модель Роя}


В модели двумерного самоотбора отдельное значение зависимой переменной может быть не наблюдаемо. Таким образом,  $y_2$ наблюдаема, если $y_1=1$ и возможно полностью не наблюдаемо при $y_1=0$. В этом разделе рассмотрим модель, в которой $y_2$ наблюдаема для всех индивидов, но только в одном из двух возможных состояний. Эта важная модель подчеркивает роль фактов, опровергающих гипотезу, а также связана с оценкой эффективности программ, подробно рассмотренной в Главе 25. 

\subsection{Модель Роя}

В одной из самых популярных работ Роя, 1951 анализируется распределение зарплат в зависимости от вида деятельности (внимание уделено и среднему и дисперсии), индивиды отличаются навыками, а также индивиды выбирают род деятельности самостоятельно. Изложение являлось довольно неформальным, и не использовался математический аппарат. Вместе с тем, предполагалось, что производительность индивида, принадлежащего к определенной профессии, имеет лог-нормальное распределение при отсутствии возможности выбора. Вместе с тем, формальная модель не оценивалась. В 1970-е г. ряд авторов, независимо друг от друга, разработали аналогичные модели, оценка которых строилась на пространственных данных, а самоотбор происходи и по наблюдаемым и по ненаблюдаемым переменным. Эти модели получили название модели Роя. 

Определим типовую форму модели Роя. Значение скрытой переменной $y^* _1$ определяет значение наблюдаемой переменной, $y^*_2$ или $y^*_3$. В частности, мы наблюдаем знак $y_1^{*}$

\begin{equation}
y_1=
\begin{cases}
	1, & \text{ если } y_1^{*}>0, \\
	0, & \text{ если } y_1^{*}\leq0,
\end{cases}
\end{equation}

и наблюдаем ровно одно из значений $y_2^{2}$ и $y_2^{3}$:

\begin{equation}
y=
\begin{cases}
	y_2^{2}, & \text{если $y_1^{*}>0$,} \\
	y_2^{3}, & \text{если $y_1^{*}\leq0$,}
\end{cases}
\end{equation}

Как правило определяют линейную модель с аддитивной ошибкой для скрытых переменных,

\begin{equation}
y_1^{*}=x_1^{*}\beta_1=\varepsilon_1,
\end{equation}

\[
y_2^{*}=x_2^{*}\beta_2=\varepsilon_2,
\]

\[
y_3^{*}=x_2^{*}\beta_3=\varepsilon_3.
\]

Модель с аддитивным эффектом имеет вид $x_3'\beta_3=x_2'\beta_2+\alpha$. Самая простая параметрическая модель с коррелированными случайными ошибками получается, если ошибки имеют совместное нормальное распределение с параметрами:

\begin{equation}
\begin{bmatrix}
\varepsilon_1 \\ \varepsilon_2 \\ \varepsilon_3
\end{bmatrix}
 \sim N
\left[
	\begin{bmatrix}
	0 \\ 0 \\ 0
	\end{bmatrix},
	\begin{bmatrix}
	1 & \sigma_{12} & \sigma_{13} \\
	\sigma{12} & \sigma_2^{2} & \sigma_{23} \\
	\sigma_{13} & \sigma_{23} & \sigma_3^2
	\end{bmatrix}
\right]
\end{equation}

где значение дисперсии $\sigma^2_1$ нормируется к единице, поскольку известен только знак $y_1^{*}$.

Вид функции максимального правдоподобия похож на модель двумерного самоотбора (раздел 16.5), за исключением того, что $y_3^{*}$ наблюдается при $y_1^{*} \leq 0$, следовательно в (16.33) выражение  $\Pr[ y_{1i}^{*} \leq 0]$ заменяется на $f(y_{3i}|y_{1i}^* \leq 0){\times}\Pr[ y_{1i}^{*} \leq 0].$

Как правило, модель с аддитивными ошибками оценивают используют двухшаговый метод Хекмана, который применяется для оценки математического ожидания усеченных данных:

\begin{multline}
\E[y|x,y_1^* > 0]=x_2'\beta_2+\sigma_{12}\lambda(x_1'\beta_1), \\
\E[y|x,y_1^*\leq 0]=x_3'\beta_3-\sigma_{13}\lambda(-x_1'\beta_1)
\end{multline}

где $\lambda(z)=\phi(z)/\Phi(z)$ и $\sigma_1^{2}=1$. На первом шаге, при оценивании пробит модели для $y_1^{*}>0$, рассчитываются оценки $\beta_1$ и $\lambda(x_1'\hat{\beta}_1)$. Далее рассчитываются $\beta_2, \sigma_{12}$ и $(\beta_3,\sigma_{13})$ с помощью двух отдельных МНК регрессий. Оценки дисперсий $\sigma_2^2$ и $\sigma_3^2$ можно вычислить используя значения квадратов остатков регрессий, по аналогии с техникой в модели двумерного самоотбора, применяемой после (16.40). Маддалла (1983, стр. 225) детально изучил особенности этой модели, назвав её моделью переключающейся регрессии с эндогенным переключением. В работе Амэмия (1985, стр.399) модель обозначена как тобит-модель пятого типа.

\subsection{Вариации модели Роя}

Многие модели принадлежат классу моделей Роя. В работе Маддалла (1983, глава 9) дается много ссылок на модели самоотбора. Также см. Амэмия (1985, глава 10). Далее мы рассмотрим несколько показательных примеров.

К частному случаю модели Роя можно отнести модель двумерного самоотбора, когда обнуляется $y_3$ и модель строится только для среднего усеченных данных $\E[y_2^{*}|y_1^{*}>0]$. Более показательным примером может быть модель двумерного самоотбора, где $y=0$ при $y_1^{*} \leq 0$, например, модель предложения труда, когда $y=y_2^{*}$ или $y=0$, следовательно, $y_3^{*}=0$.

В исследовании Л.-Ф. Ли (1978), $y_2^{*}$ и $y_3^{*}$ обозначают заработную плату по профсоюзным ставкам и не по профсоюзным ставкам, соответственно, и $y_1^{*}$ стремление вступить в профсоюз. Следовательно, появляется дополнительное уравнение

\[
y_1^{*}=y_2^{*}-y_3^{*}+z'\gamma+\zeta,
\]

где $z'\gamma+\zeta$ отражает затраты на членство в профсоюзе и эта составляющая очень по духу близка к работе  Роя (1951). Заменяя $y_2^{*}$ и $y_3^{*}$, получим приведенную форму для $y_1^{*}$:

\[
y_1^{*}=(x_2'\beta_2-x_3'\beta_3+z'\gamma)+(\varepsilon_2-\varepsilon_3+\zeta).
\]

Эта модель идентична  ранее полученной модели,  корректирующим членом $\lambda(x_1'\hat{\beta}_1)$ получаемым путем построения пробит-модели $y_1$ на $x_1$  на первом шаге, где $x_1$ обозначает единственные регрессоры в $x_2$, $x_3$ и $z$.

Если константа может принимать два значения, отстоящих например, на $\alpha$, тогда  модель Роя можно записать с помощью двух латентных переменных: 

\[
y_1^{*}=x_1'\beta_1+\varepsilon_1,
\]

\[
y^* =x'\beta+{\alpha}y_1+\varepsilon,
\]

где $y=y^* $ всегда наблюдается, а бинарная переменная $y_1$ равна единице, если $y_1^{*}{\geq}0$ и нулю, в ином случае. Эта модель может быть отнесена к моделям с эндогенной дамми переменной $(y_1)$. Модель может быть оценена с применением двухшаговой процедуры Хекмана к $\E[y^*|x]$. Кроме того, для оценки может использоваться метод инструментальных переменных, при наличии инструментов для $y_1$. При этом требуется наличие регрессора, которые не определяет уровень результата, но определяет какой из результатов был выбран.

Рассмотренные модели Роя похожи на модели, рассмотренные в литературе по оценке эффективности воздействия. Существует два варианта исхода, $y_2^{*}$ и $y_3^{*}$, но мы можем наблюдать только один из них. Подход, обозначенный в этой главе, основан на более сильных предпосылках о характере распределения ненаблюдаемых переменных. В главе 25 представлены альтернативные методы, особое внимание стоит обратить на раздел 25.3, в котором проводится сравнение подходов.

\section{Структурные модели}

Особенность моделей  самоотбора выборки заключается в том, что исход частично зависит от решения об участии, которое в свою очередь зависит от ожидаемого исхода. При этом, решение об участии и желаемый исход определяются одновременно. В предыдущих главах для описания этой взаимозависимости использовалась приведенная форма уравнения участия. В частности см. изложение работы Ли (1978) в разделе 16.7.2. Использование приведенной формы --- часто используемый прием, хотя и менее эффективен, чем использование полностью структурной версии модели. 

В этом разделе подробно анализируются структурные экономические модели, в основе которых лежит максимизация полезности, и структурные статистические модели, позволяющие обобщить системы одновременных уравнений чтобы учесть влияние цензурирования и усечения, включая модели для бинарных исходов.

\subsection{Структурные модели, построенные на принципе максимизации полезности}

Изначально, структурные модели использовали для анализа предложения женского труда. Модель из учебника по экономике включает описывает индивидов, максимизирующих полезность от потребления благ и времени досуга, при условии бюджетного ограничения и временного ограничения, состоящего в том, что всё доступное время делится между досугом и работой.
Для решения во внутренней области предельная норма замещения (MRS) работы досугом равна ставке заработной платы. Однако, угловое решение, когда женщины решают не работать, может возникнуть, если MRS превышает ставку заработной платы. Гронау (1973) и Хекман (1974) представили эконометрические модели, согласующиеся с моделями максимизации полезности. Модели Гронау и Хекман похожи на тобит-модели; они учитывают тот факт, что для неработающих женщин размер предлагаемой зарплаты неизвестен. В последующих вариантах добавляются фиксированные издержки работы, что приводит к  моделям самоотбора выборки, а также используются панельные данные, что приводит к панельной тобит-модели. Исследование этих моделей проводили Киллингсворс и Хекман (1986), а также Бланделл и Маккарди (2001), а практическое применение продемонстрировал Мроз (1987).


Для иллюстрации структурного подхода приведем следующий пример. Дубин и МакФадден (1984) построили модель потребления электроэнергии домашними хозяйствами (ДХ) (электричества или природного газа) и выбора электрического прибора (например, электрической или газовой печи) как взаимосвязанных решений, принимаемых исходя из функции полезности. Так, для $j$-го из $m$ бытовых приборов косвенная функция полезности отдельного домашнего хозяйства имеет вид:

\begin{equation}
V_j=\lbrace\alpha_1/\beta+\alpha_1{p_1}+\alpha_2{p_2}+w'\gamma+\beta(y-r_j)+\eta\rbrace{e}^{-{\beta}p_i}+\varepsilon_j,
\end{equation}

где $p_1$ и $p_2$ цены на электроэнергию и газ, $y$ доход ДХ и $r_j$ усредненные ежегодные затраты на энергию $j$-го прибора:

\[
r_j=p_1q_{1j}+p_2q_{2j}+\rho{c}_j,
\]

где $q_{1j}$ и $q_{2j}$ количество потребляемого газа и электроэнергии прибором $j$, $c_j$ затраты на прибор и $\rho$ ставка дисконтирования. Отличия в предпочтениях ДХ определяется наблюдаемой переменной $w$, ненаблюдаемой переменной $\eta$ и случайной ошибкой $\varepsilon_j$ для каждого прибора; $\varepsilon_j$ независимы между собой, но коррелированы с $\eta$. Вместе с тем, существует <<фактор вкуса>> $\alpha_{0j}$.


Спрос на энергию для $j$-го прибора равен $-(\partial{V_j}/\partial{p_1})(\partial{V_j}/\partial{y})$, используя тождество Роя, мы получаем


\[
x_1-q_{1j}=\alpha_{0j}+\alpha_{1}p_1+\alpha_2{p_2}+w'\gamma+\beta(y-r_j)+\eta.
\]

Чтобы подчеркнуть, что выбор прибора $j$ задан эндогенно, рассмотрим $m$ индикаторных переменных $\delta_{jk},k=1,\ldots ,m$, где

\[
\delta_{jk}=
	\begin{cases}
	1, & \text{если }k=j\\
	0, & \text{если }k{\neq}j
	\end{cases}
\]

Тогда спрос на электроэнергию $x_1$ при заданном приборе $j$  определяется следующим выражением:

\begin{equation}
x_1-q_{1j}=\sum_{k=1}^m 
\sum\alpha_{0k}\delta_{jk}+\alpha_{1}p_1+\alpha_{2}p_2+w'\gamma+\beta\left(y-\sum_{k=1}^{m}{r_j\delta_{jk}}\right)+\eta.
\end{equation}

Даже если модель (16.50) линейна по параметрам, МНК-оценки будут несостоятельны из-за эндогенного характера $\delta_{jk}$. Дубин и МакФадден (1984) рассмотрели два варианта алгоритма оценки.

Подход инструментальных переменных позволяет оценить (16.50), используя в качестве инструментов для $\delta_{jk}$ и $r_j\delta_{jk},k=1,\ldots ,m$ переменные $\hat{p}_k$ и $r_{j}\hat{p}_k$, соответственно, где $\hat{p}_k$ предсказанные значения вероятности выбора заданного прибора. $V_j$ обозначает косвенную функцию полезности, которая состоит из детерминистической функции $U_i$ и стохастического компонента. Такая запись функции соответствует представлению $U_i$ в разделе 15.5.1 с помощью модели полезности с аддитивными случайными ошибками (additive random utility model, ARUM). Согласно аналогичному подходу, 

\[
p_k=\Pr[ V_k>V_l,l{\neq}k,l=1,\ldots ,m]
\]

\[
=\Pr[ \varepsilon_l-\varepsilon_k<\lbrace(\alpha_{0k}-\alpha_{0l})-\beta(r_k-r_l)\rbrace{e^{-\beta{p_1}}},\text{ для всех } l{\neq}k]
\]

\[
=\dfrac{\exp [(\alpha_{0k}-\beta{r_k})e^{-\beta{p_1}}\pi/\lambda\sqrt{3}]}{\sum_{l=1}^{m} \exp [(\alpha_{0l}-\beta{r_l})e^{-\beta{p_1}}\pi/\lambda\sqrt{3}]},
\]


при предпосылке, что $\varepsilon_j, j=1,\ldots ,m$ независимо и имеют одинаковое распределение экстремальных значений второго типа с функцией распределения $F(\varepsilon)=\exp (-\exp (-\gamma-\varepsilon\pi/\lambda\sqrt{3}))$, где $\gamma{\approx}0.5772$ константа Эйлера. В данном примере среднее значение $\varepsilon_j$ равно нулю, а дисперсия равна $\lambda^2/2$, что отличается от способа описания распределения экстремальных значений второго рода, используемого в Главе 14 и 15. Оценка нелинейной мультомиальной логит-модели дает предсказанные значения вероятности $\hat{p}_k$.

В альтернативном методе самоотбора выборки $\E[\eta|j-\text{ый прибор выбран}]{\neq}0$ и для расчета математического ожидания используются предпосылки о распределении $\eta$ и $\varepsilon_1,\ldots ,\varepsilon_m$. В частности, предположим, что $\eta|\varepsilon_1,\ldots ,\varepsilon_m$ независимы и одинаково распределены со средним $\sqrt{2}\sigma/\lambda\sum_{k=1}^{m}R_{k}\varepsilon_k$ и дисперсией $\sigma^{2}(1-\sum_{k=1}^m R_k^2)$, где $\sum_{k=1}^m R_k=0$ и $\sum_{k=1}^m R_k^2<1$ и распределение $\varepsilon_k$ было определено ранее. После математических преобразований, см. в работу Дубина и МакФаддена, получим значение 

\[
\E[\eta|j-\text{ый прибор выбран}]{\neq}0=\sum_{k \neq j}^m
(\sigma\sqrt{6}R_k/\pi)[\dfrac{p_k\ln{ p_k}}{1-p_k}+\ln{ p}].
\]

Тогда приходим к оцениванию с помощью МНК: 

\[
x_1-q_{1j}=\sum_{k=1}^m a_{0k}\delta_{jk}+\alpha_1p_1+\alpha_2 p_2 +w'\gamma +
\beta \left(y-\sum_{k=1}^m r_j\delta_{jk}\right)+
\sum_{k\neq j}^m \gamma_k 
\left[ \frac{\hat{p}_k \ln \hat{p}_k}{1-\hat{p}_k} +\ln \hat{p}_k \right] +\xi
\]

где $\hat{p}_k$ предсказанное значение вероятности $p_k$ и $\xi$ ошибка с асимптотическим средним равным нулю.


Дубин и МакФадден оценили эти модели, используя наблюдения по $3,249$ ДХ для двух вариантов благ: электрический нагрев воды и обогрев помещения и газовый. 

Ханеманн (1984) аналогичным образом моделировал уровень потребления брендовых товаров,  индивиды покупают товар только одной марки из множества, а Кэмерон (1988) моделировал спрос на медицинское обслуживание в зависимости от выбора конкретного медицинского полиса из нескольких возможных.

Некоторая изворотливость может потребоваться, чтобы сконструировать модель, приводящую к аналитическому выражению для вероятности выбора и для величины спроса при условии выбора услуги, как в примере Дубина и МакФаддена. Вычислительные методы детально рассмотрены в 12-ой и 13-ой главах. Эти методы позволяют определить оценки модели, даже когда отсутствует аналитическое решение. Тем не менее, результат всегда зависит от функциональной формы полезности и распределения ненаблюдаемых переменных.

\subsection{Системы одновременных уравнений в тобит- и логит-моделях}

Для иллюстрации вопросов, возникающих при обобщении систем одновременных уравнений, рассмотренных в разделе 2.4, возьмем модель самоотбора выборки с двумя латентными переменными и зададим уравнение для латентной переменной. Общий вид такой модели:

\begin{equation}
\begin{array}{l}
y_1^{*}=\alpha_1y_2^{*}+\gamma_1y_1+\delta_1y_2+x'_1\beta_1+\varepsilon_1,\\
y_2^{*}=\alpha_2y_1^{*}+\gamma_2y_1+\delta_2y_2+x_2'\beta_2+\varepsilon_2,
\end{array}
\end{equation}

где $y_1^{*}$ и $y_2^{*}$ частично наблюдаемы, но они определяют наблюдаемые значения $y_1$ и $y_2$, и ошибки имеют совместное нормальное распределение. Например, может быть известно значение индикатора, $y_1=1$, если $y_1^{*}>0$ и значение переменной $y_2=y_2^{*}$, если $y_1^{*}>0$. Теоретически и латентная переменная, или  наблюдаемая зависимая или обе  могут выступать в качестве регрессоров, хотя для идентификации модели необходимы ограничения, описанные ниже.

\subsubsection*{Эндогенные скрытые переменные}


Самое простое ограничение: только латентные переменные могут быть регрессорами в (16.51). Тогда,

\begin{equation}
\begin{array}{l}
y_1^{*}=\alpha_{1}y_2^{*}+x_1'\beta_1+\varepsilon_1,\\
y_2^{*}=\alpha_{2}y_1^{*}+x_2'\beta_2+\varepsilon_2.
\end{array}
\end{equation}

Примером является модель двумерного самоотбора с эндогенными скрытыми переменными (16.31), она получается при добавлении  дополнительного условия $\alpha_2=0$ и записана в приведенной а не в структурной форме по $y_1^*$. Модель (16.52) можно легко оценить, поскольку приведенная форма $y_1^{*}$ и $y_2^{*}$ легко получаемся способом, который используется для обычной системы одновременных уравнений. Параметры приведенной форме можно легко оценить, используя пробит или тобит-модель, в зависимости от способа расчета $y_1$ и $y_2$ при заданных значениях $y_2^{*}$ и $y_1^{*}$. Оценки параметров структурной формы (16.52) могут быть рассчитаны через замену регрессоров $y_2^{*}$ и $y_1^{*}$ на прогнозы из приведенной формы $\hat{y_2^{*}}$ и $\hat{y_1^{*}}$. 

Модели типа (16.52) получили название систем одновременных тобит-моделей. Если зависимые переменные $y_1$ и $y_2$ бинарны, тогда (16.52) --- двумерная пробит-модель. Способы оценки рассмотрены в работах Нельсон и Olson (1978), Амэмия (1979), а также в статье Ли, Маддала и Троста (1980) и общий подход представил Л.-Ф. Ли (1981). Стандартные ошибки рассчитываются с использованием двухшаговой процедуры для М-оценок. Однако, намного проще рассчитать стандартные ошибки методом бутстрэпа, см. раздел 11.2. Идентификация требует наложения ограничений на систему уравнений (16.51) по аналогии с линейной системой уравнения.

\subsubsection*{Эндогенные регрессоры}

Популярным вариантом модели (16.52) является тобит-модель с полностью наблюдаемыми эндогенными переменными. Тогда  $y_2^{*}$ полностью наблюдаема, т.е. $y_2=y_2^{*}$, в то время как $y_1=y_1^{*}$, если $y_1^{*}>0$ и $y_1=0$, в иных случаях. Модель можно записать:

\begin{equation}
\begin{array}{l}
y_1^{*}=\alpha_1{y_2}+x_1'\beta_1+\varepsilon_1 \\
y_2=x'\pi+\nu 
\end{array}
\end{equation}

где первое уравнение ---  структурное, а второе --- приведенная форма для эндогенного регрессора $y_2$. Вновь отметим, что в этом примере $y_2$ непрерывна. Для совместно нормально распределенных ошибок $\varepsilon_1=\gamma\nu+\xi$, где $\xi$ независимые нормально распределенные ошибки (см. раздел 5.1),  $y_1^{*}=\alpha_1{y_2}+x_1'\beta_1+\gamma\hat{\nu}+e_1$.

В двухшаговой процедуре оценки сначала рассчитываются остатки  $\hat{v}=y_2-x'\hat{\pi}$ 
в регрессии МНК $y_2$ на $x$, а затем оценивается тобит-модель

\[
y_1^*=\alpha_1 y_2 + x'_1\beta_1+\gamma \hat{v}+e_1,
\]



где $e_1$ нормально распределены. Для тестирование на эндогенность $y_2$ можно использовать тест Вальда с нулевой гипотезой $\gamma=0$, а стандарнтные ошибки взять из тобит-модели. Данный тест является расширением вспомогательной регрессии, реализуемой в тесте эндогенности Хаусмана в линейной модели (см. Раздел 8.4.3). Если нулевая гипотеза отвергается, то на упомянутом втором шаге тобит-регрессии получаются состоятельные оценки $\alpha_1$ и $\alpha_2$, но стандартные ошибки нужно скорректировать из-за дополнительного регрессора $\hat{v}$ на первом шаге. См. работу Смита и Бланделла (1986) для подробностей по тобит-модели и Риверса и Вуонга (1988), где изложена аналогичная процедура с оценкой пробит-модели на втором шаге.


\subsubsection*{Эндогенные цензурированные или бинарные переменные}

Анализ усложняется, если цензурированные или бинарные эндогенные переменные $y_1$ или $y_2$ выступают в роли регрессоров в (16.51). Хекман (1978) проанализировал следующую модель:

\begin{equation}
\begin{array}{l}
y_1^{*}=\gamma_1{y_1}+\delta_1{y_2^{*}}+x_1'\beta_1+\varepsilon_1, \\
y_2^{*}=\alpha_2{y_1^{*}}+\gamma_2{y_1}+x_2'\beta_2+\varepsilon_2,
\end{array}
\end{equation}

где наблюдается $y_1=1$, если $y_1^{*}>0$, и $y_1=0$, если $y_1^{*}{\geq}0$ и $y_2=y_2^{*}$ всегда наблюдаема. Модель усложняется тем, что $y_1$ выступает в роли регрессора. В приведенной форме для $y_1^{*}$ регрессорами могут быть только $x_1$ или $x_2$. Следовательно, должно выполняться условие согласованности, $\delta_1\gamma_2+\gamma_1=0$. Если это условие выполняется, тогда приведенная форма примет вид:

\[
\begin{array}{l}
y_1^{*}=x'\pi_1+\nu_1,\\
y_2=\gamma_2{y_1}+x'\pi_2+\nu_2.
\end{array}
\]

Это частный случай модели Роя, где участие ($y=1$) приводит только к сдвигу зависимой переменной (через $\gamma_2$). В целом, модели с цензурированными или усеченными эндогенными переменными трудно оценивать. Например, см. Бланделл и Смит (1989).

\subsubsection*{Пример}

Брукс, Кэмерон и Картер (1998) применили систему одновременных тобит-моделей для объяснения результатов голосования представителей конгресса в поддержку производителей сахара. Голос конгрессмэна (за или против), денежную поддержку от производителей сахара и денежную поддержку от производителей сахарозаменителей обозначим  $y_1, y_2$ и $y_3$, соответственно; $y_1$ -- бинарная переменная, а $y_2$ и $y_3$ цензурированы в нуле. Авторы составили систему одновременных уравнений для $y_1^{*}$, $y_2^{*}$ и $y_3^{*}$, таким образом, структурная модель имеет простейший вид (16.52).

Насколько приемлема данная спецификация? В рассматриваемом примере $y_2^{*}$ и $y_3^{*}$ должны зависеть от скрытой переменной $y_1^{*}$, поскольку фактический результат голосования  будет известен позже финансирования. С $y_1^{*}$ сложнее, т.к. данная латентная переменная зависит от фактических значений вкладов $y_2$ и $y_3$, а не от латентных денежных вкладов. Однако, если данная ситуация рассматривается как повторяемая игра, тогда можно использовать скрытые переменные $y_2^{*}$ и $y_3^{*}$. Очевидно, что обоснованность данного предположения будет зависеть от конкретного случая. Идентификация параметров модели обеспечивается исключением некоторых экзогенных регрессоров. Оценки параметров будут состоятельны, если ошибки действительно имеют совместное нормальное распределение.

\subsection{Полупараметрическое оценивание}

Предыдущие методы решали проблему частично пропущенных данных с помощью предположения об их законе распределения. При этом получалась либо функция правдоподобия, либо функция цензурированного, усеченного среднего или среднего с учетом самоотбора.


Оценки уязвимы даже к незначительным погрешностям в предпосылках о распределении случайных ошибок. Например, оценки, полученные методом максимального правдоподобия или с помощью двухшаговой процедуры Хекмана в стандартной тобит-модели несостоятельны, если нарушается либо предпосылка о нормальности, либо предпосылка о гомоскедастичности. См. работу Паарш (1982) и ссылки в ней.

Много исследований было посвящено развитию полупараметрических методов, состоятельных при менее строгих предпосылках. Перед изложением полупараметрических методов отметим, что в качестве альтернативы можно продолжать использовать полностью параметрические методы, в основе которых лежат менее строгие предпосылки о распределении.

\subsection{Гибкие параметрические модели}

Для начала возьмем классическую тобит-модель $y_i^{*}=x_i'\beta+\varepsilon_i$. Возможны два варианта ослабления предпосылки $\varepsilon_i \sim N[0,\sigma_i^2]$. Во-первых, можно допускать гетероскедастичность, $\sigma_i^2=\exp (z_i'\gamma)$, тогда нужно оценить оба параметра, $\beta$ и $\gamma$. Во-вторых, можно использовать менее строгое распределение. Например, полиномиальное разложение нормального распределения (см. раздел 9.7.7).

Для модели двумерного самоотбора выборки может использоваться аналогичный подход, т.е. можно предположить более широкий класс для совместного  распределении случайных ошибок $(\varepsilon_1,\varepsilon_2)$. Ли (1983) предложил использовать некие преобразованные $(\varepsilon_1^{*},\varepsilon_2^{*})$ вместо $(\varepsilon_1,\varepsilon_2)$ для которых предпосылка о двумерном нормальном распределении может быть более реалистична. 

Можно также использовать байесовские методы. Чиб (1992) рассматривал цензурированную тобит-модель. Автор рассматривает скрытую переменную $y^* $ как вспомогательную и использовал метод пополнения данных (см. раздел 13.7). При сэмплировании по Гиббсу последовательно генерируются  (1) условное апостериорное для $\beta|y,y^* ,\sigma^2$, (2) условное апостериорное для $\sigma^2|y,y^* ,\beta$ и (3) апостериорное для $y^* |y,\beta,\sigma^2$.

Гибкий параметрический подход хорошо подходит для анализа нелинейных цензурированных, усеченных регрессий и нелинейных регрессий самоотбора для счетных данных, данных по длительности и смешенных данных, поскольку в этом случае трудно найти полупараметрический метод.


\subsection{Полупараметрические методы оценки цензурированных регрессий}


Рассмотрим линейную модель для скрытой переменной $y_i^{*}=x_i'\beta+\varepsilon_i$, цензурированной слева  в нуле, т.е. $y_i=y_i^{*}$, если $y_i^{*}>0$ и $y_i=0$, если $y_i^{*}{\geq}0$. Как правило полупараметрическая модель имеет вид:

\begin{equation}
y_i=\max(x_i'\beta+\varepsilon_i,0).
\end{equation}

Это тобит-модель (16.11)-(16.13), за исключением того, что не указан закон распределения $\varepsilon$. С небольшими изменениями эту модель можно использовать для данных цензурированных слева но не в нуле, или для данных цензурированных справа. Например, если $y=\min(x'\beta+\varepsilon,U)$, тогда $U-y=\max(U-x'\beta_\varepsilon,0)$. Цель состоит в получении состоятельных оценок без предположения о законе распределения $\varepsilon_i$. Методы оценки называются полупараметрическими, поскольку неусеченное среднее $x_\beta'$ параметризировано, а распределение случайных ошибок --- нет. Представленные ниже методы отличаются по предпосылкам о распределении $\varepsilon$. 

Согласно (16.8) ММП оценивание возможно, если известна функция  распределения $y^* $ и, следовательно, $\varepsilon$. Функция распределения $\varepsilon$ может быть состоятельно оценена с использованием оценки Каплана-Майера, приведенной в главе 17  для цензурированных справа данных по длительности. Также, непараметрическая оценка распределения $\varepsilon$ может быть получена с помощью  разложения в ряд Галланта и Нычки (1987); см. раздел 9.7.7. Эти полупараметрические методы максимального правдоподобия редко используются.

В литературе наиболее часто встречается расчёт оценки через условные моменты. Из (16.20) следует, что формула условного цензурированного среднего $\E[y|x]$ соответствует одноиндексной модели с $\E[y|x]=g(x'\beta)$, где функция $g(\cdot )$ неизвестна, если распределение $\varepsilon$ не указано. Модели для одноиндексных моделей  (см. раздел 9.7.4) можно применять, однако, как сказано в разделе, $\beta$ может быть оценена только с точностью до сдвига и масштаба. 

В более популярном подходе рассматриваются условные цензурированные моменты, которые менее изменчивы при цензурировании. Пауэлл (1984) предложил использовать условную медиану. Главная предпосылка состоит в том, что медиана $\varepsilon|x$ равна нулю, следовательно, условная медиана $y|x$ равна условному среднему $x'\beta$. Интуицию метода Пауэлла легче всего понять, предположив, что $y$ независимы и одинаково распределены. Если меньше половины значений выборки цензурировано, то меньше половины выборки принимают нулевые значения, значит больше половины выборки положительны, тогда выборочная медиана цензурированных значений является состоятельной оценкой медианы по совокупности. 
Пауэлл (1984) применил эту идею к регрессионному анализу, действуя по такой же логике для тех наблюдений, у которых меньше половины значений $\varepsilon|x$ цензурированы, где $\varepsilon=y-x'\beta$ и для расчета используются оцененные значения $\beta$.  Тогда регрессионный анализ производится по аналогии с оценкой медианы по методу наименьших абсолютных отклонений (см. раздел 4.6). В результате получается цензурированная оценка метода наименьших абсолютных отклонений (censored least absolute deviation, CLAD) минимизирующая:

\begin{equation} 
Q_N(\beta)=N^{-1}\sum_{i=1}^N|y_i-\max(x_i'\beta,0)|.
\end{equation}

Необходимым условием состоятельности оценок является равенство медианы $\varepsilon|x$ нулю. При выполнении этого условия оценки будут состоятельны, даже если  ошибки условно гетероскедастичны. Оценка $\beta$ является $\sqrt{N}$ состоятельной и асимптотически нормальной. Эффективность оценки может возрасти, если взвешивать слагаемые в сумме с помощью $f(0|x_i)$, значений условной плотности $\varepsilon_i|x_i$ в нуле. Цензурированный метод наименьших абсолютных отклонений может также использоваться для расчета условных квантилей.

Вместо медианы можно использовать симметрично усеченное среднее, значение которого также не меняется при цензурировании. Пусть $\varepsilon|x$ симметрично распределено. Тогда, для наблюдений с положительным средним (т.е. $x'\beta>0$) $y|x$ имеет симметричное распределение на интервале $(0,2X'\beta)$. Следовательно, с равной вероятностью $x'\beta+\varepsilon+<0$ и $y=0$ или $x'\beta+\varepsilon>2x'\beta$ и все данные искусственно приравниваются к $2x'\beta$ для сохранения симметрии относительно $x'\beta$. В результате:

\begin{equation}
\E[1(x'\beta>0)[\min(y,2x'\beta)-x'\beta]x]=0,
\end{equation}

где $1(x'\beta)>0$ ограничение, согласно которому используются только наблюдения с положительным средним и зависимая переменная или $y=0$, или $0<y<2x'\beta$, или $2x'\beta$ если $y>2x'\beta$. Моментная оценка  по формуле (16.57) не дает единственного решения для $\beta$. Пауэлл (1986b) предложил симметричный цензурированный метод наименьших квадратов (symmetric censored least squares, SCLS):

\begin{equation}
Q_N(\beta)=N^{-1}\lbrace[y_i-\max(y_i/2,x'\beta)]^{2}+1(y_i>2x_i'\beta)[y_i^2/4-\max(0,x_i'\beta)]^2\rbrace.
\end{equation}

Можно показать что выражение (16.58) является выборочным аналогом условий первого порядка для моментов (16.57). Чей и Оноре (1998) описывает цензурированный МНК графически. Также они приводят графическое описание оценивания с помощью попарных разностей  Оноре и Пауэлла (1994). 

Меленберг и Ван Сует (1996), Чей и Оноре (1998) и Чей и Пауэлл (2001) продемонстрировали практические примеры для некоторых из этих методов. Паган и Улла (1999) приводят дополнительныме методы и излагают теориюы.

Рассмотрим пример использования цензурированного метода наименьших абсолютных отклонений для оценки данных тобит-модели с нормальными ошибками. Оценка параметра наклона, при истинном  значении 1000, получилась равной 956 (стандартная ошибка равна 117) при использовании ММП и равной 838 (стандартная ошибка 165) по методу цензурированных наименьших абсолютных отклонений. Как и предполагалось, устойчивость цензурированного метода наименьших абсолютных отклонений к ненормальности ошибок достигается за счет снижения  эффективности оценок.

\subsection{Полупараметрическая оценка для моделей самоотбора}

Полупараметрическая оценка моделей самоотбора выборки вызывает больше сложностей. В качестве примера, рассмотрим часто встречающуюся модель --- двумерную модель самоотбора выборки ---, определение которой было дано в разделе 16.5.2, однако здесь опускается предположение о совместном нормальном распределении ошибок $\epsilon_1,\epsilon_2$.


Возможно найти полупараметрическую ММП-оценку. В частности, Галлант и Нычка (1987) в явном виде рассматривали  двумерную  модель самоотбора выборки как подходящую для использования их метода с использованием разложения в ряд, см. раздел 9.7.7.

Вместе с тем, в литературе в качестве отправной точки, как правило, используют выражение для усеченного условного среднего, которое, согласно (16.34), равно: 

\begin{equation}
\E[y_{2i}|x_i,y_{1i}^{*}>0]=x_{2i}^{*}\beta_2+\E[\varepsilon_2|\varepsilon_1>-x_{1i}'\beta_1]\\
=x_{2i}'\beta_2+g(x_{1i}'\beta_1),
\end{equation}

по аналогии с (16.41), предполагается, что распределение $\varepsilon_{2i}|x_i,\varepsilon_{1i}$ зависит только от $x_{1i}$. Поскольку распределение $(\varepsilon_1,\varepsilon_2)$ не указано,  функция $g(\cdot )$ неизвестна, следовательно требуется применение полупараметрических методов. Поскольку возможно, что $g(x_1'\beta_1)=x_1'\beta_1$ идентификация модели с неспецифицированной функцией $g()$ требует исключающего ограничения, а именно, хотя бы один из регрессоров в $x_1$ не должен входить в $x_2$. Чем менее коррелированы $x_1'\beta_1$ и $x_2$ тем лучше разделяются $\beta_2$ и $g()$. Модель (16.59) является частично линейной и может быть оценена с помощью методов из раздела 9.7.3. Популярными методами являются метод Робинсона (1998а), оценка с помощью взятия разностей, или разложение функции $g(x_1'\beta_1)$ в ряд. Поскольку $\beta_1$ неизвестно, используется регрессия $y_{2i}$ на $x_{2i}'\beta_2+g(x_1'\beta_1)$, где $\beta_1$ может быть получено путем построения регрессии $y_{1i}$ на $x_{1i}$ с использованием одной из полупараметрических оценок раздела 14.7. Эти методы дают состоятельную оценку для $\beta_2$. Чтобы в дополнение к коэффициенту наклона в зависимости $y_2$ состоятельно оценить константу можно обратить внимание на работу Эндрюса и Шафгенса (1998). 

Ньюи, Пауэлл и Уолкер (1990) применили данный подход к исследованию предложения труда женщин. Модель для индикатора участия была оценена разными методами, а уравнения для $y_2$ было оценено методом Робинсона (1988а). Меленберг и Ван Суент моделировали расходы на путешествия используя разнообразные полупараметрические методы и для двумерных моделей самоотбора и для цензурированных регрессий. Широкий класс моделей рассматривают Дас, Ньюи и Велла (2003).

Мански (1989) изучал идентификацию в двумерной модели самоотбора при довольно минимальных предположениях и приводит границы для среднего и предельных эффектов при заданных значениях регрессоров и условии участия.



\section{Вывод тобит-модели}

\subsection{Моменты стандартного нормального распределения для усеченных данных}

Пусть $z \sim N[0,1]$, плотность распределения равна $\phi(z)=(1/\sqrt{2\pi})\exp (-z^2/2)$ и функция распределения обозначена $\Phi(z)$. Поскольку $\Pr[ z>c]=1-\Phi(c)$, условная плотность $z|z>c$ равна $\phi(z)/(1-\Phi(c))$. Следовательно,


\begin{multline}
\E[z|z>c]=\int_c^{\infty}z(\phi(z)/[1-\Phi(c)])\,dz\\
=\int_c^{\infty} z (1/\sqrt{2\pi})\exp (-z^2/2) \,dz /[1-\Phi(c)]\\
=\int_c^{\infty}
\dfrac{\partial}{\partial{z}}
\left(-(1/\sqrt{2\pi})\exp (-z^2/2)\right)
dz/
[1-\Phi(c)]\,dz\\
=\left[ -(1/\sqrt{2\pi})\exp (-z^2/2) \right]_c^{\infty}/ [1-\Phi(c)]\\
=\phi(c)/[1-\Phi(c)].
\end{multline}

Аналогичным образом, 

\begin{multline}
\E[z^2|z>c]=\int_c^{\infty}z^2(\phi(z)/[1-\Phi(c)])\,dz \\
=\int_c^{\infty} z{\times}z \times (1/\sqrt{2\pi}\exp (-z^2/2)/[1-\Phi(c)]\,dz \\
=\int_c^{\infty}z{\times}\dfrac{\partial}{\partial{z}}\left(-(1/\sqrt{2\pi})\exp (-z^2/2)\right)dz/[1-\Phi(c)]\,dz \\
=\left[z{\times}-(1/\sqrt{2\pi})\exp (-z^2/2)\right]_c^{\infty}/[1-\Phi(c)] \\
-\int_c^{\infty}z{\times}\dfrac{\partial}{\partial{z}}(z)\left(-(1/\sqrt{2\pi})\exp (-z^2/2)\right)dz/[1-\Phi(c)]\,dz\\
=c\phi(c)/[1-\Phi(c)]+(1-\Phi(c))/[1-\Phi(c)] \\
=c\phi(c)/[1-\Phi(c)]+1.
\end{multline}

После некоторых математических преобразований, 


\begin{align}
V[z|z>c]=\E[z^2|z>c]-(\E[z|z>c])^2 \\
=1+c\phi(c)/[1-\Phi(c)]-\phi(c)^2/[1-\Phi(c)]^{2}.
\end{align}



\subsection{Асимптотика двухшаговой процедуры Хекмана к тобит-модели}

Получение  асимптотической ковариационной матрицы для метода Хекмана осложнено зависимостью  от оценок параметров, полученных на первом шаге. Существует несколько способов расчета асимптотической ковариационной матрицы, ряд из которых рассмотрен в работе Амэмия (1985, стр.369-370). В этом разделе в основное расчета лежат общие методы для ковариационной матрицы при использовании двухшаговой М-оценки, ранее рассмотренной в разделе 6.6. В качестве примера приведем самый простой случай для тобит-модели (см. раздел 16.3.6). Метод Хекмана можно применить к двухшаговой оценке моделей двумерного самоотбора (см. раздел 16.5.4) и системе одновременных уравнений тобит-модели. Наиболее простым способ является парный бутстрэп (см. раздел 11.2).

Оценим параметр $\gamma=[\beta'\sigma]$ в (16.26) из модели для положительных значений $y_i$:

\[
y_i=x'_{i}\beta+\sigma\lambda(x'_{i}\alpha)+\eta_i
\]

\[
=w_i(\alpha)'\gamma+\eta_i,
\]

где $w_{i}(\alpha)={[{x'}_i \,\,\, \lambda_i({x'}_i\alpha)]}'$ и $\eta_i=y_i-{x'}_i\beta-\sigma\lambda({x'}_i\alpha)$ гетероскедастичны с дисперсией $\sigma^2_{\eta i}$, которая определена в (16.24). На первом шаге находим оценку $\hat{\alpha}$ неизвестного параметра $\alpha$ ММП для пробит-модели. Двухшаговая процедура Хекмана может быть записана двумя уравнениями:

\begin{equation}
\sum_{i=1}^N(y_i-\Phi({x'}_i\alpha))
\frac{\phi^2({x'}_i\alpha)}{\Phi({x'}_i\alpha)(1-\Phi({x'}_i\alpha))}x_{i}=0
\end{equation}


\[
-\sum_{i=1}^N d_iw_i(\alpha)(y_i-w_i(\alpha)'\gamma) =0,
\]

где первое уравнение есть условие первого порядка в пробит-модели для оценки $\alpha$ и второе уравнение есть условие первого порядка МНК для оценки  $\gamma$ при положительных значениях $y_{i}(d=1)$.

В общем виде эти уравнения можно записать: $\sum_{i=1}^Nh(x_i,\theta)=0$, где $\theta=(\alpha',\gamma')'$. При обычном разложении в ряд Тейлора до первой степени $\hat{\gamma}-\gamma\overset{d}{\to} N[0,G^{-1}_0S_0(G^{-1}_0)']$, где $G_0=\lim N^{-1}\E[\sum_{i=1}^N\partial h(x_i,\theta)/\partial\theta]$ и $S_0=\lim N^{-1}\E[\sum_{i=1}^N  h(x_i,\theta)h(x_i,\theta)]'$. Нас интересуют компоненты относящиеся к $\gamma$. Выражение можно упростить, поскольку матрица $\partial h(x_i,\theta)/\partial\theta$ оказывается блочно-треугольной, поскольку $\gamma$ отсутствует в первом блоке уравнений. Разделение дает следующий результат:

\[
V[\hat{\theta}_2]=
G_22^{-1}\left\lbrace S_22+G_21[G_11^{-1}S_11G_11^{-1}]G_21'-G_{21}G_{11}'S_{12}-S_{21}G_{11}^{-1}G_{21}'\right\rbrace{G_{22}^{-1}},
\]

где матрицы определены в разделе 6.6.

Применительно к нашей задаче, сначала рассмотрим составляющие в  $G_0$. Тогда
\[
G_{11}=\lim\dfrac{1}{N}\sum_{i=1}^N \dfrac{\phi^2(x_i'\alpha)}{\Phi(x_i'\alpha)(1-\Phi(x_i'\alpha))}x_ix_i'
\]

\[
G_{21}=\lim\dfrac{1}{N}\sum_{i=1}^N d_iw_i\dfrac{\partial\lambda(x'_{i}\alpha)}{\partial\alpha},
\]

\[
G_{22}=\lim\dfrac{1}{N}\sum_{i=1}^N \E[d_{i}w_iw'_i].
\]

Для выражения $G_11$ используется факт того, что $G^{-1}_{11}$ равно дисперсии ММП оценки пробит-модели. Выражение $G_{21}$ получается при использовании 

\[
\E\left[\dfrac{{\partial}h_{2i}}{\partial\theta'_{i}}\right]=\E \left[-\dfrac{\partial d_{i}w_{i}(\alpha)(y_{i}-w_{i}(\alpha)')\gamma}{\partial{\alpha}}\right]
\]

\[
=\E\left[w_{i}\dfrac{{\partial}d_{i}w_{i}(\alpha)}{\partial\alpha'}\right]
\]

\[
=\E\left[d_{i}w_{i}\dfrac{\partial\lambda(x'_{i}\alpha)}{\partial\alpha}\right].
\]

Выражение для $G_{22}$ получено с использованием

\[
\dfrac{\partial h_{2i}}{\partial\theta'_{2}}=\dfrac{{\partial}d_{i}w_{i}(\alpha)(y_i - w_{i}(\alpha)'\gamma)}{\partial\gamma}=d_{i}w_{i}w'_i
\]

Переходя к $S_0$, получим:

\[
S_{11}=G^{-1}_{11},
\]

\[
S_{21}=0
\]

\[
S_{22}=\lim\dfrac{1}{N}\sum^{N}_{i=1}\E[d_{i}(y_i - w_{i}(\alpha)'\gamma)^2]
\]

Для получения выражения $S_{11}$ применяем равенство информационной матрицы. Взяв математическое ожидание и сделав несколько манипуляций, получим, что $S_{21}=0$ и $S_{22}$ равно $V[\eta_i]$.

Объединив эти результаты, можно рассчитать двухшаговую оценку Хекмана $\hat{\gamma} \overset{a}{\sim} N(\gamma,V_{\gamma})$, где

\begin{equation}
\hat{V}_{\gamma}=(\hat{W}'\hat{W})^{-1}(\hat{W}'\Sigma_{\hat{\eta}}\hat{W}+\hat{W}'\hat{D}\hat{V}_{\alpha}\hat{D}\hat{W})(\hat{W}'\hat{W})^{-1},
\end{equation}

и, где, $\hat{W}'\hat{W}=\sum^{N}_{i=1}d_{i}\hat{w}_{i}\hat{w}'_{i},\hat{D}=\mathrm{Diag}[\partial\lambda(x'_{i}\alpha)/\partial\alpha|_{\hat{\alpha}}]$, $\hat{V}_{\alpha}$ ковариационная матрица для ММП оценки пробит-модели на первом шаге, а $\Sigma_{\hat{\eta}}$ --- диагональная матрица, в которой значения на диагонали равны $\hat{\sigma}^{2}_{\eta_i}$. Значение $\hat{\sigma}^{2}_{\eta_i}$. Можно легко рассчитать, если программа предусматривает расчет матриц. Сложнее всего рассчитать $\sigma^{2}_{\eta_i}=V[\eta_i]$, данное в (16.24). При возникновении трудностей в расчете $\sigma^{2}_{\eta_i}=\V[\eta_i]$, возможно использовать подход Уайта (1980) и использовать $\hat{\sigma}^{2}_{i}=(y_{i}-x'_{i}\hat{\beta}+\hat{\sigma}\lambda_{i}(x'_{i}\hat{\alpha}))^2$

\section{Практические соображения}

В большинстве статистических пакетов предусмотрена ММП оценка тобит-модели при выполнении предпосылки о нормальности распределения. Двухчастную модель легко оценить, поскольку можно по-отдельности оценить каждую часть. Двумерная модель самоотбора может быть оценена по двухшаговой процедуре Хекмана, используя пробит и МНК. Тем не менее, стандартные ошибки трудно рассчитать через пробит и МНК из-за двухшаговой процедуры и намного проще использовать пакет, где двухшаговая процедура Хекмана уже реализована. Как правило, использование полупараметрических методов требует написания специальной программы в языке программирования, например, можно использовать GAUSS. В некоторых статистических пакетах предусмотрена возможность ММП оценки цензурированных или усеченных данных, например модели Пуассона или отрицательного биномиального распределения для счетных данных.

Цензурирование и усечение не составляют проблемы, если закон распределения переменных верно специфицирован. Например, цензурированные сверху данные могут быть легко использованы для оценки параметров, если логнормальное распределение хорошо согласовывается с данными. Также может использоваться цензурированный метод наименьших абсолютных отклонений, предпосылки о распределении в котором менее строгие. 

Больше проблем возникает с оценкой моделей самоотбора выборки. Большинство моделей этого типа полагаются на строгие предпосылки о распределении. Полупараметрические методы сталкиваются с требованием к идентифицируемости модели, состоящем в том, что переменная, которая определяет участие, не должна определять значение зависимой переменной. Более перспективный метод, часто используемый в литературе по эффектам воздействия, заключается в рассмотрении только случаев, когда самоотбор происходит только  на основе  наблюдаемой переменной.

\section{Библиографические заметки}

Существует много литературы по моделям самоотбора выборки. Подробный список источников дают  в своих работах Маддала (1983) и Гурьеру (2000), более краткий список приводят Амэмия (1984,1985) и Грин (2003). 

16.3 Тобин (1958) предложил и применил тобит-модель к анализу расходов. Амэмия (1973) формально обосновал состоятельность оценок и их нормальное асимптотическое распределение. Хекман (1974) применил тобит-модель к анализу предложения женского труда с последующим детальным анализом результатов.

16.4. Многие исследования Эксперимента по страхованию здоровья корпорации RAND (Rand Health Insurance Experiment), например, работа Дуана и др. (1983) могут послужить примером двухчастной модели.

16.5 Хекман (1976, 1979) рассмотрел двухшаговый метод оценки двумерной модели самоотбора, который лежит в основе многих полупараметрических способов оценки. Мроз (1987) применяет процедуру к анализу предложения труда женщин и показал роль принятия предположения об экзогенном характере заработной платы.

16.7. Существует много интерпретаций идей Роя (1951), также как много вариантов тобит-модели. Л.-Ф. Ли (1978) оценил разницу между оплатой труда работников, состоящих и не состоящих в профсоюзе.

16.8. Работа Дубина и МакФаддена (1984) является одним из основных примеров структурного микроэконометрического анализа, построенного на полной спецификации функции полезности и распределения ненаблюдаемых значений.

16.9 Полупараметрический метод оценки моделей двумерного самоотбора детально рассмотрен в книгах М.-Дж. Ли (1996), Пагана и Улла (1999), обзорах  Велла (1998) и Л.-Ф. Ли (2001). Вместе с тем, Чей и Оноре (1998), а также Чей и Пауэлл (2001) приводят примеры использования моделей цензурированных данных, а Меленберг и Ван Сует (1996) примеры использования двумерной  модели самоотбора. 


\section{Упражнения}

\begin{enumerate}
\item [$16 - 1$] В упражнении рассматривается как влияет степень усечения на тобит-модель.
\begin{enumerate}
\item Сгенерируйте 2000 значений скрытой переменную $y^* =k+3x+u$, где $u \sim N[0,3]$ и $x \sim U[0,1]$. Выберите такое $k$, чтобы примерно $30\%$ значений $y^* $ были бы отрицательны.

\item Сгенерируйте усеченную или цензурированную подвыборку, отбрасывая наблюдения для которых $y^* <0$

\item Оцените модель методом наименьших квадратов для 2000 наблюдений, предполагая, что скрытая переменная наблюдаема. Сравните полученные результаты с теорией, учитывая, что была сделана одна репликация.

\item  Оцените модели МНК используя усеченную снизу подвыборку, т.е. $y>0$.

\item Найдите оценки параметров, используя метод максимального правдоподобия для усеченных данных. Прокомментируйте результаты, учитывая теоретические свойства ММП. Сравните полученные результаты с выводами в (c) и (d)

\item  Повторите шаги (a)-(f) для других $k$ чтобы цензурировалось $20\%, 40\%$ и $50\%$ наблюдений. Сравните результаты с выводами при $30\%$ цензурируемых наблюдений. На основе полученных результатов, сделайте вывод о влиянии цензурирования высокого порядка на значения оценок параметров. Подкрепите ваши аргументы теорией.
\end{enumerate}

\item [$16 - 2$] Рассмотрите модель скрытой переменной $y_i^{*}=x_i'\beta+\varepsilon_i$, $\varepsilon_i \sim N[0,\sigma^2]$. Предположим, что $y_i^{*}$ цензурирована сверху, т.е. мы наблюдаем $y_i=y_i^{*}$, если $y_i^{*}<U_i$ и $y_i=U_i$, если $y_i^{*}\geq U_i$, верхний предел $U_i$ известная константа для каждого индивида и может меняться от индивида к индивиду.
\begin{enumerate}
\item  Запишите логарифмическую функцию максимального правдоподобия. (Подсказка: Функция имеет вид отличный от стандартной функции максимального правдоподобия, поскольку добавляется  $U_i$ и равенства соответствуют случаю $y_i=y_i^{*}$, если $y_i^{*}<U_i$)

\item  Запишите выражения для усеченного среднего $\E[y_i | x_i,y_i<U_i]$. (Подсказка: Для $z \sim N[0,1]$ мы имеем $\E[z|z>c]=\phi(c)/[1-\Phi(c)]$; $\E[z|z<c]=-\E[-z|-z>-c]$ и $-z \sim N[0,1].$)

\item Запишите двухшаговую процедуру Хексмана для оценки модели из пункта (a).

\item  Запишите выражение для усеченного среднего $\E[y_i|x_i]$. (Подсказка: Ответ к пункту (b) существенно поможет)
\end{enumerate}
\item [$16 - 3$] Это упражнение посвящено последствиям неправильной спецификации тобит-модели. Для анализа возьмите модель из упражнения 16.1.
\begin{enumerate}
\item  Сгенерируйте $y^* $ при гетероскедастичных случайных ошибках, т.е. $u \sim N[0,\sigma^2z]$, где $z>0$ и корреляция между $z$ и $x$ ненулевая, но и не равна единице. Подберите $k$ чтобы $30\%$ всех данных оказались цензурированы. Оцените модель цензурированных данных с нормально распределенными ошибками с помощью ММП и сравните полученные результаты с оценками, когда случайные ошибки гомоскедастичны.

\item  Пусть предпосылка о нормальности не выполняется. Используя процедуру Монте-Карло, оцените модель, если количество наблюдений равно 1000 и процедура повторяется 500 раз. Для каждого повтора сгенерируйте цензурированную выборку так, чтобы распределение случайной ошибки было смесью двух распределений  $N[1,9]$ или $N[0.4,1]$ c вероятностями $0.4$ и $0.6$, соответственно. Оцените модель и сравните полученные результаты с результатами, когда предпосылка о нормальности выполняется.
\end{enumerate}
\item [$16 - 4$] Рассмотрим модель пуассоновской регрессии, где вероятность $y^* $ равна $f^{*}(y^* )=e^{-\mu}\mu^{y}/y^* , y_i^{*}=0,1,2,\ldots $ и наблюдения не зависимы по $i$. Из-за ошибки, связанной с записью данных, значение $y^*$ полностью наблюдаемо при $y^* {\geq}2$. Если $y^* =0$ или $1$ известно только, что $y^*  \leq 1$, в данных такие наблюдения записаны как $y^* =1$. Определим наблюдаемые данные, $y=y^* $ для $y_i^{*}{\geq}2$ и $y=1$ для $y_i^{*}=0$ или $1$.
\begin{enumerate}
\item  Найдите вероятность $f(y)$ для наблюдаемых значений $y$.
\item  Найдите $\E[y]$. Здесь требуются некоторые математические преобразования.


Теперь введем регрессоры с $\E[y^*|x]=\exp(x'\beta)$ и определим индикатор $d=1$ для $y^*\geq 0 $ и $d=0$ для $y^*$ равного $0$ или $1$.

\item  Запишите точную формулу для целевой функции правдоподобия, которая позволяла бы рассчитать состоятельные оценки $\beta$, используя данные $y_i$, $d_i$ и $x_i$.


\item  Запишите точную формулу для целевой функции правдоподобия, которая позволяла бы рассчитать состоятельные оценки $\beta$, используя только данные $d_i$ и $x_i$.

\item  Возможно ли получить состоятельные оценки $\beta$, используя только $d_i$ и $x_i$. Поясните ваш ответ.
\end{enumerate}
\item [$16 - 5$]
Используя 50\% данных RAND по затратам на здоровье за 12 месяцев мы хотим дать ответ на широкий вопрос: какая из моделей наилучшим образом подходит для оценки затрат? 
\begin{enumerate}
\item 	Используя описательные статистики по  затратам, проанализируйте к чему приводит большая пропорция нулевых наблюдений. Приводит ли это к нарушению предпосылки о нормальности? Существует ли преобразование, которое бы позволяло сделать предпосылку о нормальности более применимой?
\item 	Проанализируйте три модели, с одинаковыми независимыми переменными. Регрессоры точно такие же как в  упражнении 20.6 по счетным данных. Среди моделей: (i) тобит-модель; (ii) двухчастная модель (модель преодоления порогов) и (iii) модель самоотбора выборки. Опишите соотношения и взаимосвязи между моделями, предложите способ сравнить и выбрать одну из моделей. Если вы предполагаете, что можно столкнуться с проблемами, касающихся спецификации или оценки параметров, обозначьте их и предложите возможные пути решения. Особое внимание обратите на ограничения.
\item 	Оцените все три модели. В двухчастной модели второе уравнение составлено только для индивидов с ненулевыми затратами. Для оценки модели самоотбора выборки используйте ММП-оценка и двухшаговая процедура Хекмана. Изложите ваши соображения по исключающему ограничению, необходимому для идентификации модели. Является ли самоотбор проблемой в данном случае?
\item Каким образом можно сравнить качество подгонки по трем моделям? Какая модель наилучшим образом согласуется с данными? Какие критерии вы используете для сравнения?
\item 	Оцените влияние двух переменных, log income и log (1 + уровень совместного страхования) на уровень затрат на здоровье. Сравните предельные эффекты изменения этих переменных для тобит-модели и двухчастной модели. Подумайте, как наилучшим образом проинтерпретировать полученные результаты для гетерогенной выборки.
\item 	Дайте краткое пояснение, при выполнении каких предпосылок квантильная регрессия (см. раздел 4.6) может быть альтернативным методом оценки. Какие преимущества и недостатки такой альтернативы?
\end{enumerate}
\end{enumerate}


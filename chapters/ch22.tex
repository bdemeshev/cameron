
\chapter{Линейные модели анализа панельных данных: дополнения}

\subsection{Введение}
В предыдущей главе рассмотрены варианты линейных моделей панельных данных с фиксированными или случайными эффектами и строго экзогенными регрессорами. Сейчас мы перейдем к рассмотрению различных видов линейных моделей, делая акцент на ослабление предпосылок экзогенности, с целью состоятельного оценивания моделей с эндогенными переменными и/или лаговыми зависимыми переменными.

Стандартный метод, применяемый при эндогенных регрессорах, --- использование инструментальных переменных. Получить инструментальные переменные для панельных данных гораздо проще, чем в случае с данными пространственного типа, так как в качестве инструментов для текущего периода могут быть использованы экзогенные регрессоры других периодов. Единственная трудность заключается в том, чтобы сначала учесть фиксированные или случайные эффекты.

При использовании панельных данных в состав регрессоров могут быт дополнительно включены лаговые зависимые переменные, что невозможно в случае наличия одного периода. Благодаря этому мы можем оценивать динамические модели, и можем различать причины постоянства доходов, например, отличить изменчивость вокруг ненаблюдаемых индивидуальных эффектов (как в главе 21), от влияния предыдущих значений на значения текущего периода. Оценки главы 21, которые учитывают индивидуальные  эффекты становятся несостоятельными, когда одним из регрессоров становится лаговая переменная. Метод инструментальных переменных с использование более длинных лагов в качестве инструментальных переменных дает состоятельные оценки.

Благодаря обилию инструментов панельные данные обеспечивают излишек доступных для оценивания моментных условия. Ошибки моделей панельных данных обычно не являются независимыми и одинаково распределенными. Естественный подход к оцениванию --- ОММ, описанный подробно в разделе 22.2. Подход проиллюстрирован на примере оценивания эластичности предложения труда в разделе 22.3. Дальнейшие подробности оценивания с индивидуальными эффектами и эндогенными регрессорами или лаговыми зависимыми переменными представлены в разделе 22.4 и 22.5. Обсуждение достаточно объемно, так как в нем затронуто множество различных усложнений моделей. Это включение индивидуальных фиксированных или случайных эффектов, различные предположения об экзогенности, идентифицируемые и сверхидентифицируемые модели.

Остаток главы посвящен другим отдельным темам, которые в принципе не требуют знания разделов 22.2---22.5. Модели,  близко связанные с моделями анализа панельных данных, рассмотрены в разделах 22.6---22.8, а именно модели повторяющихся пространственных данных, <<разность разностей>> и иерархические модели.

\section{ОММ оценивание линейных моделей панельных данных}

Регрессионные модели панельных данных в главе 21 накладывали ограничение на скалярную зависимую переменную $y_{it}$. Предполагалось, что она зависит только от одновременных значений регрессоров $\x_{it}$, хотя вполне возможно, что все значения $\x_{i1}, \dots, \x_{iT}$ влияют на $y_{it}$ в предположении главы 21 о строгой экзогенности. Таким образом, нам предоставлена возможность более эффективно оценить параметры модели благодаря использованию невключенных регрессоров других периодов в качестве инструментов для значений регрессоров текущего периода.

Более того, регрессоры других периодов могут быть валидными инструментами для эндогенных регрессоров текущего периода, либо для лагов зависимых переменных. Так инструменты обеспечивают состоятельные оценки в ситуациях, когда из-за нарушения предположения о строгой экзогенности оценки главы 21 оказываются несостоятельными.

В этом разделе рассматривается оценивание ОММ для панельных данных. Это очень полезная база для \textbf{оценивания панельных данных с помощью инструментальных переменных}, к которому мы будем прибегать на протяжении разделов 22.2---22.5. Затем описывается использование экзогенных переменных (регрессоров или инструментов)  других периодов в качестве инструментов. Когда эти основные аспекты будут рассмотрены, останется только добавить в модели фиксированные или случайные эффекты. Внедрение в модели фиксированных и случайных эффектов отложено на последующие разделы. 

\subsection{ОММ для панельных данных}

Рассмотрим \textbf{линейную модель панельных данных}
\begin{align}
y_{it}=\x'_{it}\bm\beta+u_{it},
\label{Eq:22.1}
\end{align}
где $\x_{it}$ может содержать как регрессоры, меняющиеся во времени, так и регрессоры, не меняющиеся во времени, а также могут включать свободный член. Здесь пока нет \textbf{индивидуальных эффектов} $\alpha_i$ (предположение перестает действовать в разделе 22.3), $\x_{it}$  включает переменные только текущего периода (предположение перестает действовать в разделе 22.5). Предполагается, что наблюдения независимы по $i$, используется \textbf{короткая панель}, т.е. $T$ фиксировано, и $N \rightarrow \infty$.

Начнем с записи $T$ наблюдений для $i$-го индивидуума в столбик
\begin{align}
\mathbf y_{i}=\mathbf X'_{i}\bm\beta+\mathbf u_{i},
\label{Eq:22.2}
\end{align}
где $\mathbf y_i$ и $\mathbf u_i$ --- векторы размерности $T \times 1$ и $\mathbf X_i$ --- матрица размерности $T \times K$ c $t$-ой строкой $\x'_{it}$, т.е.
\begin{align}
\mathbf y_{i}=
\begin{bmatrix}
y_{i1} \\
\vdots \\
y_{iT}
\end{bmatrix};
\mathbf X'_{i}=
\begin{bmatrix}
\x'_{i1} \\
\vdots \\
\x'_{iT}
\end{bmatrix};
\mathbf u_i=
\begin{bmatrix}
u_{i1} \\
\vdots \\
u_{iT}
\end{bmatrix}.
\nonumber
\end{align}
Модель \ref{Eq:22.2} определяет линейную систему уравнений. Можно напрямую применять результаты раздела 6.9.5 для оценивания методом инструментальных переменных на данных, независимых по $i$.

Предположим, что существует \textbf{матрица инструментов $Z_i$} размерности $T \times r$, где $r \geq K$ --- количество \textbf{инструментов}, которые удовлетворяют $r$ моментным условиям
\begin{align}
\E [\mathbf Z'_i \mathbf u'_i]=\mathbf 0.
\label{Eq:22.3}
\end{align}
ОMM оценка, основанная на этих моментных условиях, минимизирует квадратичную форму
\begin{align}
Q_N(\be)=\left[ \sum^N_{i=1} \mathbf Z'_i \mathbf u_i \right]'
\mathbf W_N
\left[ \sum^N_{i=1} \mathbf Z'_i \mathbf u_i \right],
\nonumber
\end{align}
где $\mathbf W_N$ обозначает взвешивающую матрицу $r \times r$. Используя $\mathbf u_i=\mathbf y_i - \mathbf X_i\be$, с помощью  алгебраических преобразований можно получить \textbf{ОMM оценку для панельных данных} (Panel Generalized Method of Moments, PGMM) 
\begin{align}
\hat{\be}_{PGMM}=
\left[ \left( \sum^N_{i=1} \mathbf X'_i \mathbf Z_i \right) \mathbf W_N
\left( \sum^N_{i=1} \mathbf Z'_i \mathbf X_i \right) \right]^{-1}
\left[ \left( \sum^N_{i=1} \mathbf X'_i \mathbf Z_i \right) \mathbf W_N
\left( \sum^N_{i=1} \mathbf Z'_i \mathbf y_i \right) \right].
\nonumber
\end{align}
Существенное условие состоятельности этой оценки --- предположение \ref{Eq:22.3}.

Во многих приложениях $\mathbf Z_i$ состоит из текущих и лаговых значений экзогенных регрессоров. Например, предположим, что все регрессоры одновременно экзогенны. Тогда $\E[\x_{it} u_{it}]= \mathbf 0$ означает \ref{Eq:22.3} с $\mathbf Z_i=[\x'_{it} \dots \x'_{iT}]$. В этом случае модель \textbf{идентифицируема} и, так как $\mathbf Z_i=\mathbf X_i$, $\hat{\be}_{PGMM}$ упрощается до МНК оценки сквозной регрессии главы 21. Если дополнительно предполагается, что $\E[\x_{it-1}u_{it}]=\mathbf 0$, тогда $\x_{it-1}$ доступна как дополнительный инструмент для $i$-го наблюдения, модель \textbf{сверх-идентифицируема}, и более эффективное оценивание возможно при использовании ОММ оценки для панельных данных (PGMM).

Использование \textbf{различных предположений об экзогенности} для формирования матрицы инструментов $\mathbf Z_i$  подробно рассмотрено в разделе 22.2.4. Анализ требует внедрения в модели панельных данных индивидуальных эффектов $\alpha_i$. Это проиллюстрировано в эмпирическом примере в разделе 22.3 и детально обсуждается в разделах 22.4 и 22.5.

\subsection{Робастные статистические выводы для панельных данных }

Чтобы выразить формулу распределения ОMM оценки для панельных данных, удобно использовать более компактную запись. Перепишем
\begin{align}
\hat{\be}_{PGMM}=[\mathbf X' \mathbf Z \mathbf W_N \mathbf Z' \mathbf X]^{-1} \mathbf X' \mathbf Z \mathbf W_N \mathbf Z' \mathbf y,
\label{Eq:22.4}
\end{align}
где  $\mathbf X'=[\mathbf X'_1 \dots \mathbf X'_N], \mathbf Z'=[\mathbf Z'_1 \dots \mathbf Z'_N]$, $\mathbf y'=[\mathbf y'_1 \dots \mathbf y'_N]$. Тогда $\hat{\be}_{PGMM}$ \textbf{асимптотически нормальна}  с оцененной асимптотической ковариационной матрицей
\begin{align}
\hat{\V} [\hat{\be}_{PGMM}]=[\mathbf X' \mathbf Z \mathbf W_N \mathbf Z' \mathbf X]^{-1} \mathbf X' \mathbf Z \mathbf W_N \mathbf Z' 
(N \hat{\mathbf S}) \mathbf W'_N \mathbf Z' \mathbf X
[\mathbf X' \mathbf Z \mathbf W_N \mathbf Z' \mathbf X]^{-1},
\label{Eq:22.5}
\end{align}
см. уравнение \ref{Eq:6.97}, где $\hat{\mathbf S}$ --- состоятельная оценка матрицы размера $r \times r$ 
\begin{align}
\mathbf S =\mathrm{plim} \frac{1}{N} \sum^N_{i=1} \mathbf Z'_i \mathbf u_i \mathbf u'_i \mathbf Z_i,
\label{Eq:22.6}
\end{align}
 также предполагается независимость по $i$. Существенным предположением для этого результата является $N^{-1/2} \mathbf Z' \mathbf u = N^{-1/2} \sum_i \mathbf Z'_i \mathbf u_i \overset{d}\rightarrow \mathcal N[\mathbf 0, \mathbf S]$. Робастная ошибка $\mathbf S$ в форме Уайта  
\begin{align}
\hat{\mathbf S} = \frac{1}{N} \sum^N_{i=1} \mathbf Z'_i \hat{\mathbf u}_i \hat{\mathbf u}'_i \mathbf Z_i,
\label{Eq:22.7}
\end{align}
где  $\hat{\mathbf u}_i=\mathbf y_i- \mathbf X_i \hat{\be}$ --- вектор оцененных остатков размерности $T \times 1$. 

Оценка \ref{Eq:22.5} дает \textbf{робастные} к гетероскедастичности и корреляции во времени \textbf{стандартные ошибки для панельных данных}. В качестве альтернативы можно использовать \textbf{панельный бутстреп}. Подробнее см. обсуждение в разделе 21.2.3, где  применяется данный метод.

\subsection{Одношаговый и двухшаговый ОMM для панельных данных}

Различные взвешивающие матрицы полного ранга $\mathbf W_N$ в \ref{Eq:22.4} приводят к различным системам ОММ оценок, за исключением случая идентифицируемости, когда $r=K$, а оценка ОММ для панельных данных упрощается до оценки инструментальных переменных $[\mathbf Z' \mathbf X]^{-1}\mathbf Z'\mathbf y$ для любого $\mathbf W_N$. Обсуждение этого результат приводится в разделе 6.4.2. Два самых используемых вида матрицы $\mathbf W_N$ даны в данном разделе.

{\centering Одношаговый ОMM}

В \textbf{одношаговом ОMM} или \textbf{двухшаговой МНК оценке} используется взвешивающая матрица $\mathbf W_N=[\sum_i \mathbf Z'_i \mathbf Z_i]^{-1}=[\mathbf Z' \mathbf Z]^{-1}$.  Тогда оценка имеет вид
\begin{align}
\hat{\be}_{2SLS} = [\mathbf X' \mathbf Z ( \mathbf Z' \mathbf Z)^{-1} \mathbf Z' \mathbf X]^{-1} \mathbf X' \mathbf Z (\mathbf Z' \mathbf Z)^{-1} \mathbf Z' \mathbf y.
\label{Eq:22.8}
\end{align}
На основании \ref{Eq:22.3} может быть показано, что это оптимальная оценка ОММ для панельных данных, если $\mathbf u_i | \mathbf Z_i$ независимы и одинаково распределены  с параметрами $[\mathbf 0, \sigma^2 \mathbf I_T]$.

Эта оценка называется одношаговой ОMM, потому что она может быть напрямую вычислена по формуле \ref{Eq:22.8}. Она также называется двухшаговой МНК оценкой (2SLS), так как может быть получена в два шага: (1) МНК $\mathbf X_i$ на $\mathbf Z_i$, откуда получаем предсказанные значения $\hat{\mathbf X}_i$, и (2) МНК $\mathbf y_i$ на $\hat{\mathbf X}_i$. В качестве оценки ковариационной матрицы $\hat{\be}_{\mathrm{2SLS}}$, которая является робастной в случае гетероскедастичности и панельных данных, используется \ref{Eq:22.5} с $\mathbf W_N=[\mathbf Z' \mathbf Z]^{-1}$.

{\centering Двухшаговый ОМM}

Самая эффективная ОMM оценка, основанная на безусловном моментном условии \ref{Eq:22.3}, использует взвешивающую матрицу $\mathbf W_N=\hat{\mathbf S}^{-1}$, где $\hat{\mathbf S}$  является состоятельной оценкой для $\mathbf S$, определенной в \ref{Eq:22.6}; для более общего результата см. раздел 6.4.2. Используя $\hat{\mathbf S}$ в \ref{Eq:22.7}, получаем \textbf{двухшаговую ОMM оценку}
\begin{align}
\hat{\be}_{2SGMM} = [\mathbf X' \mathbf Z \hat{\mathbf S}^{-1} \mathbf Z' \mathbf X]^{-1} \mathbf X' \mathbf Z \hat{\mathbf S}^{-1} \mathbf Z' \mathbf y.
\label{Eq:22.8}
\end{align}
Тогда выражение \ref{Eq:22.5} значительно упрощается и $\hat{\mathrm V}[\hat{\be}_{2SGMm}]=[\mathbf X' \mathbf Z (N \hat{\mathbf S})^{-1} \mathbf Z' \mathbf X]^{-1}$.

Оценка называется двухшаговой ОMM оценкой, так как состоятельная оценка $\be$ на первом шаге, например $\hat{\be}_{\mathrm 2SLS}$, нужна для вычисления остатков $\hat{\mathbf u}_i$, которые в свою очередь используются для вычисления $\hat{\mathbf S}$.

{\centering Увеличение эффективности}

В этой главе мы сосредоточим внимание на ситуациях, в которых $\mathbf Z$ не содержит $\mathbf X$ вследствие того, что некоторые компоненты $ \mathbf X$ эндогенны. ОMM для панельных данных дает состоятельные оценки в отличие от МНК. Двухшаговый ОMM дает наиболее эффективную оценку на основании моментного условия $\E[ \mathbf Z'_i \mathbf u_i]=\mathbf 0$.

Даже если регрессоры строго экзогенны, двухшаговый ОMM будет \textbf{более эффективным}, чем МНК сквозной регрессии. Чтобы это продемонстрировать, предположим, что $\mathbf X$ строго экзогенны. Когда $\mathbf Z = \mathbf X$, двухшаговая ОMM оценка упрощается до  $[\mathbf X' \mathbf X]^{-1}\mathbf X' \mathbf y$, и использование ОMM для панельных данных не дает никакого преимущества. Однако если вместо этого в $\mathbf Z$  входит $\mathbf X$ и некоторые дополнительные переменные, такие как степени регрессоров или значения регрессоров в других периодах, тогда двухшаговый ОMM по меньшей мере такой же эффективный, как МНК при выполнении равенства, если ошибки $u_{it}$ независимы и одинаково распределены.

Даже более эффективные, чем $\hat{\be}_{\mathrm{2SGMM}}$, оценки возможны благодаря расширению определения $\mathbf Z_i$ и использованию оптимального моментного условия $\E[ \mathbf u_i| \mathbf Z_i]= \mathbf 0$, а не $\E[ \mathbf Z'_i \mathbf u_i] = \mathbf 0$ (см. раздел 22.4.3), а также использованию дополнительных моментных ограничений. Мы не решаемся назвать двухшаговую ОММ \textbf{оптимальной ОММ} оценкой,  как в разделе 6.3, так как она оптимальна только при выполнении условия \ref{Eq:22.3}.

{\centering Тест на сверх-идентифицирующие ограничения}

Если есть $r$  инструментов и только $K$  параметров для оценивания, то в ОММ для панельных данных остается $(r-K)$ сверх-идентифицирующих ограничений. Из раздела 6.3.8 это позволяет сделать \textbf{тест на сверх-идентифицирующие ограничения}
\begin{align}
OIR= \left[ \sum^N_{i=1} \hat{\mathbf u}'_i \mathbf Z_i \right] 
(N \hat{\mathbf S})^{-1}
 \left[ \sum^N_{i=1} \mathbf Z'_i \hat{\mathbf u}_i \right] 
\label{Eq:22.10}
\end{align}
где $\hat{ \mathbf u}_i=\mathbf y_i- \mathbf Z'_i \hat{\be}_{\mathrm{2SGMM}}, \hat{\mathbf S}$ дано в \ref{Eq:22.7}. Предполагается независимость по $i$, но допускаются гетероскедастичность и корреляция по $t$ для данного $i$. Заметим, что должна использоваться $\hat{\be}_{\mathrm{2SGMM}}$, а не $\hat{\be}_{\mathrm{2SLS}}$.

Эта тестовая статистика  распределена как $\chi^2(r-K)$ при выполнении нулевой гипотезы, что сверх-идентифицирующие ограничения верны. Если OIR  велика, то сверх-идентифицирующие моментные условия отвергаются, и мы делаем вывод, что некоторые из инструментов в $\mathbf Z_i$ коррелированы  с ошибкой, а следовательно эндогенны.

\subsection{Выбор инструментальных переменных}

В обсуждении предполагалось существование матрицы инструментов $\mathbf Z_i$ размерности $T \times r$, удовлетворяющей условию \ref{Eq:22.3}. Перейдем к подробному обсуждению, как получить инструменты в случае использования панельных данных.

В моделях пространственных данных, для эндогенных переменных применяются переменные, не являющиеся регрессорами в оцениваемом уравнении. Такие переменные также могут быть использованы и в случае с панельными данными. В случае с моделями панельных данных, однако, дополнительные временные периоды обеспечивают дополнительные моментные условия и дополнительные инструменты, которые легко могут привести к идентифицируемости или сверх-индентифицируемости $\be$.

Число доступных моментных условий и инструментов увеличивается по мере того, как делаются более строгие предположения о корреляции между $u_{it}$ и $\mathbf z_{is}$, $s,t=1, \dots, T$. Мы рассмотрим эффект более сильного \textbf{предположения об экзогенности} согласно М.Дж. Ли (2002), см. раздел 2.3. Акцент сделан на использовании экзогенных компонент регрессоров в качестве инструментов более чем один раз. Однако в методе также применяются более традиционные инструменты, т.е. переменные, не включенные в регрессию \ref{Eq:22.1}.

{\centering  Предположение о сумме}

Определим $\mathbf Z_i$ аналогично определению $\mathbf X_i$. Тогда
\begin{align}
\mathbf Z_i=
\begin{bmatrix}
\mathbf z'_{i1} \\
\mathbf z'_{i2} \\
\vdots \\
\mathbf z'_{iT}  
\end{bmatrix},
\mathbf u_i=
\begin{bmatrix}
u_{i1} \\
u_{i2} \\
\vdots \\
u_{iT}
\end{bmatrix},
\label{Eq:22.11}
\end{align}
где $\mathbf z_{it}$ имеет размерность $r \times 1$ и $\E[ \mathbf Z'_i \mathbf u_i]= \mathbf 0$, если \textbf{предположение о сумме}
\begin{align}
\E \left[ \sum^T_{i=1} \mathbf z_{it} u_{it} \right] =\mathbf 0
\label{Eq:22.12}
\end{align}
выполнено.

Это предположение соответствует предположению, использованному в сквозной МНК регрессии $y_{it}$ на $\mathbf x_{it}$, так как если $\mathbf z_{it}=\mathbf x_{it}$ в \ref{Eq:22.12}, то панельная ОММ оценка, определенная в \ref{Eq:22.4}, упрощается до $(\sum_i \mathbf Z'_i \mathbf X_i)^{-1} \sum_i \mathbf Z'_i \mathbf y_i$.

Чтобы эта оценка была доступна, как минимум необходимо, чтобы выполнялось порядковое условие, т.е. $r \geq K$. При выполнении предположения о сумме, найти инструменты в случае с панельными данными так же сложно, как и в случае с пространственными данными.

{\centering Предположение об одновременной экзогенности}

Более сильное и естественное предположение --- \textbf{это предположение об одновременной экзогенности}:
\begin{align}
&\E [\mathbf z_{it} u_{it}] =\mathbf 0
& t=1, \dots, T,
\label{Eq:22.13}
\end{align}
т.е. предполагается, что инструменты одновременно не коррелированы с ошибкой.

Этому предположению соответствует намного больше моментных условий. Получается $Tr$ моментных условий, где $r=\mathrm{dim}[\mathbf z_{it}]$. Чтобы их использовать, определим
\begin{align}
\mathbf Z_i=
\begin{bmatrix}
\mathbf z'_{i1}  &\mathbf 0 & \dots & \mathbf 0\\
\mathbf 0 & \mathbf z'_{i2} & & \vdots \\
\vdots & & \ddots & \mathbf 0 \\
\mathbf 0 & \hdots & \mathbf 0 & \mathbf z'_{iT}  
\end{bmatrix},
\mathbf u_i=
\begin{bmatrix}
u_{i1} \\
u_{i2} \\
\vdots \\
u_{iT}
\end{bmatrix},
\label{Eq:22.14}
\end{align}
где $\mathbf Z_i$ имеет размерность $Tr \times T$. Моментное условие \ref{Eq:22.3} выполнено, так как $\E[ \mathbf Z'_i \mathbf u_i]= \mathbf 0$ из \ref{Eq:22.13}, но \ref{Eq:22.3} сейчас определяет $Tr$ моментных условий, которые могут быть использованы для оценки $K$ компонент $\be$.

Этот примечательный результат очевидного избытка моментных ограничений исходит из неявного предположения, что $\be$ не меняется во времени, так что каждый дополнительный временной период добавляет дополнительные моментные ограничения.

Количество дополнительных моментных ограничений уменьшается в той мере, в которой $\be$ меняется во времени. В частности, свободный член зачастую меняется во времени при включении в $\x_{it}$ $(T-1)$ временных фиктивных переменных $d_{s,it}=1$, если $t=s$ и 0 иначе, для $s=2, \dots, T$. Тогда условие $\E[d_{s,it} u_{it}]=0$ не может быть использовано, так как оно повторяет условие $\E[1 \times u_{it}]=0$ включением свободного члена в $\x_{it}$. В предыдущем примере, если $\x_{1it}$ включает временные фиктивные переменные, то доступно только $TK-(T-1)$ моментных условий. Любые меняющиеся во времени регрессоры могут быть использованы как инструменты.

{\centering  Слабое предположение об экзогенности}


В моментном условии \ref{Eq:22.13} рассматривается только одновременная корреляция между инструментами и регрессорами. Более строгое предположение  --- \textbf{слабое предположение об экзогенности} или \textbf{предположение о предопределенных  инструментах}. Оно состоит в том, что лаговые значения инструментов не коррелированы с ошибкой в текущем периоде, т.е.
\begin{align}
&\E [\mathbf z_{it} u_{it}] =\mathbf 0
& s \leq t &
& t=1, \dots, T.
\label{Eq:22.15}
\end{align}
В условии \ref{Eq:22.15} $\mathbf z_{i1}, \dots, \mathbf z_{it}$ могут быть инструментами для $u_{it}$, хотя будущие значения $\mathbf z_{is}$ не могут быть использованы. Инструмент $ \mathbf Z_i$ имеет структуру, аналогичную \ref{Eq:22.14}, за тем исключением, что $\mathbf z'_{it}$  заменяется на вектор инструментов $[\mathbf z'_{i1}, \dots, \mathbf z'_{it}]$, увеличивающийся по мере увеличения $t$.

Условия такого типа используются в моделях рациональных ожиданий и в моделях межвременного принятия решений в условиях неопределенности, что приводит к \textbf{уравнениям Эйлера} вида $\E[u_{it} | \mathcal I_{it}]=0$, где $\mathcal I_{it}$ --- это информация, доступная в момент $t$, а пример $u_{it}$ дан в разделе 6.2.7. Если информация включает текущие и прошлые значения $\mathbf z_{it}$, то $\E[u_{it} | \mathbf z_{is}]=0, s \leq t$, что приводит к \ref{Eq:22.15}.

Эти условия становятся существенными в динамических моделях с лаговыми зависимыми переменными (см. раздел 22.5). В некоторых примерах одновременная корреляция не исключена, так что неравенство $s \leq t$ в \ref{Eq:22.15} заменено на $s < t$.

Заметим, что инструменты, не меняющиеся во времени, могут быть использованы лишь однажды. Т.е. если $\mathbf z_{it} = [ \mathbf z_{1i} \; \mathbf z_{2it}]$, то $\mathbf z_{1i}$ и $\mathbf z_{2i1}, \dots, \mathbf z_{2it}$ доступны в качестве инструментов.

{\centering  Сильное предположение об экзогенности}

Более сильное предположение  о слабой экзогенности  --- \textbf{предположение о сильной экзогенности}. Оно состоит в том, что будущие значения инструментов не коррелированы с ошибкой текущего периода, так что
\begin{align}
&\E [\mathbf z_{is} u_{it}] =\mathbf 0
& s, t=1, \dots, T.
\label{Eq:22.16}
\end{align}
Тогда текущие, прошлые и будущие значения $\mathbf z_{is}$ --- годные инструменты для $u_{it}$.

Это предположение было выполнено для регрессоров $\mathbf x_{it}$  в главе 21, так как из $\E[u_{it}| \mathbf x_{i1}, \dots, \mathbf x_{iT}]=0$ следует $\E[u_{it} | \x_{is}]=0, 1 \leq s \leq T$, а следовательно $\E[\x_{is} u_{it}]= \mathbf 0$. Прошлые предпосылки подходят для статических моделей, но в случае динамических моделей самое большое, что может предполагаться, --- это слабая экзогенность инструментов.

Условие \ref{Eq:22.16} показывает, что $\mathbf z_{i1}, \dots, \mathbf z_{iT}$ могут быть инструментами для $u_{it}$. Структура инструментов $\mathbf Z_i$ имеет вид \ref{Eq:22.14}, за исключением того, что $\mathbf z'_{it}$ в \ref{Eq:22.14} заменяется на расширенный вектор инструментов $[\mathbf z'_{i1}, \dots, \mathbf z'_{iT}]$.

В случае слабой экзогенности не меняющиеся во времени инструменты могут быть использованы только один раз. Если $\mathbf z_{it}=[\mathbf z_{1i} \; \mathbf z_{2it}]$, тогда доступно $T(r_{TI}+Tr_{TV})$ моментных условий, где $r_{TI}$ и $r_{TV}$  обозначают номера не меняющихся (Time Invariant, TI) и меняющихся (Time Varying, TV) во времени инструментов.

 Огромное количество моментных условий, $rT^2$, объясняется исключающими ограничениями, неявно сделанными в модели панельных данных \ref{Eq:22.1}. Для простоты предположим, что все компоненты $\x_{it}$ строго экзогенны и мы используем их в качестве инструментов всегда, когда это возможно. В общем $y_{it}$ может зависеть от регрессоров всех временных периодов, $\x_{i1}, \dots, \x_{iT}$. В модели панельных данных $y_{it}=\x'_{it}\be+u_{it}$ c $\E[\x_{it} u_{it}]=\mathbf 0$, напротив, $y_{it}$ зависит только от $\x_{it}$. Предположение о сильной экзогенности, что $\E[\x_{is} u_{it}]=\mathbf 0$, делает возможным использование регрессоров других периодов $\x_{is}, s \neq t$ (не только $\x_{it}$)  в качестве инструментов. 

{\centering  Излишние инструменты}
 
Если $\mathbf z_{it}$ меняется по $i$ и $t$, то лаги могут быть использованы как инструменты, в зависимости от сделанных предположений об экзогенности. Для $i$-го наблюдения в случае одновременной экзогенности доступны инструменты $\mathbf z_{it}$, в случае слабой экзогенности  --- $\mathbf z_{i1}, \dots, \mathbf z_{it}$, а в случае сильной экзогенности --- $\mathbf z_{i1}, \dots, \mathbf z_{iT}$. Благодаря этому идентифицирование возможно при использовании только экзогенных регрессоров в качестве инструментов. Трудности нахождения годных инструментов, сравнимые с трудностями в случае использования пространственных данных, возникают только при выполнении предположения о сумме.

На практике, однако, нет такого количества доступных инструментов, как обещано выше. \textbf{Не меняющиеся во времени инструменты} $\mathbf z_{it}= \mathbf z_{i}$ могут быть использованы только один раз, так как в этом случае $\mathbf z_{it}= \mathbf z_{is}$ для любых $s$ и $t$. Например, это относится к свободному члену или идентификаторам расы и пола. Если в качестве инструмента в модели используется сам регрессор или его лаговые значения, то количество доступных инструментов снижается. Инструменты, изменяющиеся во времени систематическим образом, могут быть доступны не во всех периодах. Поэтому инструменты,  построенные как произведение временных дамми и регрессоров, не меняющихся во времени, должны быть включены в регрессию только один раз в случае использования полного набора временных дамми. В примерах присутствуют временные дамми и произведения временных дамми с индикаторами расы или пола. Инструменты, имеющие линейную функцию по времени, следует использовать только один раз. Например,  если год является инструментальной переменной, то не следует использовать лаговые значения временных дамми. Этот комментарий не относится к возрасту, который увеличивается линейно для каждого индивидуума, но изменяется по $i$.

Довольно легко непреднамеренно использовать \textbf{излишние инструменты}. ОММ оценки для панельных данных по прежнему доступны и обычные результаты верны, если  есть достаточное количество годных инструментов. Например, если используется $r$ инструментов, и два из них излишние, то модель все еще может быть оценена при $r \geq K +2$, так как $\mathbf Z' \mathbf X$ имеет полный ранг $K$. Если используется слишком много излишних инструментов, что приводит к недоидентифицируемости модели, могут возникнуть проблемы сингулярности в оценивании ОММ. Даже если модель сверх-идентифицируема, количество степеней свободы в тесте на сверх-идентифицирующие ограничения будет уменьшено, если некоторые инструменты излишни.

{\centering  Слабые инструменты}
 

Слабые инструменты не стоит путать со слабой экзогенностью, которая описана в разделе 4.9. Не существует устоявшегося формального теста на \textbf{слабые инструменты}. Стандартные $R^2$ и $F$-статистика представлены в разделе 4.9. Важно именно увеличение объясняющею силы при использовании инструментов. Следует использовать частичный $R^2$, который учитывает экзогенные регрессоры, входящие в набор инструментов. Более того, в то время как строится регрессия эндогенных регрессоров на  все инструменты, $F$-статистика должна показывать значимость части инструментальных переменных, которые не являются экзогеными регрессорами.

Так как ошибки здесь  не являются независимыми и одинаково распределенными, $F$-статистика должна быть вычислена на основании робастных стандартных ошибок для  панельных данных. Она может быть вычислена как $W/r^*$, где $W$  --- это тестовая статистика Вальда, распределенная как $\chi^2$, для ограничений раздела 7.2.7, и $r^*$ --- количество инструментов, не являющихся регрессорами первоначальной модели.

\subsection{Вычисление оценок ОММ для панельных данных}

Моментные условия, которые обсуждались в предыдущем разделе, приводят к матрице инструментов $\mathbf Z_i$. Тогда, зная $\mathbf Z_i$, можно оценить $\be$ с помощью $\hat{\be}_{\mathrm{2SLS}}$, определенной в \ref{Eq:22.8}, или $\hat{\be}_{\mathrm{2SGMM}}$, определенной в \ref{Eq:22.9}.

Применение двухшаговой МНК оценки проще, чем двухшаговой ОММ оценки. Рассмотрим оценивание в предположении о сумме; $\mathbf Z_i$ определено в \ref{Eq:22.11}. Тогда $\hat{\be}_{\mathrm{2SLS}}$ дана в \ref{Eq:22.8}, где $\mathbf Z' \mathbf X =\sum_i \mathbf Z_i' \mathbf X_i=\sum_i \sum_t \mathbf z_{it} \x'_{it}$. Похожие алгебраические преобразования применяются для других произведений. В результате этих преобразований получается стандартная формула из учебника для двухшагового МНК, за исключением того, что суммирование производится по $i$ и по $t$. Таким образом, $\hat{\be}_{\mathrm{2SLS}}$ может быть получена с помощью двухшаговой МНК регрессии $y_{it}$ на $\x_{it}$ в пакете анализа данных пространственного. \textbf{Робастные для панельных данных} стандартные ошибки могут быть получены благодаря использованию кластерно-робастной опции с кластеризацией по $i$, или \textbf{панельного бутстрепа} с ресемплингом только по $i$, а не по  $i$ и $t$. Подходы похожи на те, что применялись для МНК сквозной регрессии, подробно описанной в разделе 21.2.3. 

Для предположений отличных от предположения о сумме можно использовать статистический пакет для реализации двухшагового МНК для данных пространственного типа, подходящим образом определяя матрицу инструментов $\mathbf Z_i$, которая имеет более сложную форму. Для  предположения об  одновременной экзогенности, $\mathbf Z_i$ определена в \ref{Eq:22.14}. Это будет та же самая форма, как и в \ref{Eq:22.11}, если $t$-ую строку в \ref{Eq:22.11}, $\mathbf z'_{it}$, заменить на 
\begin{align}
[\mathbf 0'_{r_1} \dots \mathbf 0'_{r_{t-1}} \mathbf z'_{it} \mathbf 0'_{r_{t+1}} \dots \mathbf 0'_{r_{T}}],
\label{Eq:22.17}
\end{align}
где $r_s=\mathrm{dim}[\mathbf z_{is}]$ и $\mathbf 0_{r_s}$ обозначает нулевой вектор размерности $r_s \times 1$. Подобным образом, для предположения о слабой экзогенности, $\mathbf Z_i$  --- это \ref{Eq:22.11} с $t$-ой строкой $\mathbf z'_{it}$, замененной на
\begin{align}
[\mathbf 0'_{r_1} \dots \mathbf 0'_{r_{t-1}} (\mathbf z^t_{it})' \mathbf 0'_{r_{t+1}} \dots \mathbf 0'_{r_{T}}],
\label{Eq:22.18}
\end{align}
где $(\mathbf z^t_{it})'=[\mathbf z'_{i1} \dots \mathbf z'_{it}]$ и $r_s=\mathrm{dim}[\mathbf z^s_{is}]$, и для предположения о строгой экзогенности, $\mathbf Z_i$ --- это \ref{Eq:22.11} c $t$-ой строкой $\mathbf z'_{it}$, замененной на
\begin{align}
[\mathbf 0'_{r_1} \dots \mathbf 0'_{r_{t-1}} (\mathbf z^T_{it})' \mathbf 0'_{r_{t+1}} \dots \mathbf 0'_{r_{T}}],
\label{Eq:22.19}
\end{align}
где $(\mathbf z^T_{it})'=[\mathbf z'_{it} \dots \mathbf z'_{iT}]$ и $r_s=\mathrm{dim}[\mathbf z^T_{is}]$. Практический пример генерирования инструментов представлен в разделе 22.3.

На практике может быть очень много моментных условий. Например, для данных с 10 временными периодами и 5 изменяющимися во времени регрессорами предположение о сильной экзогенности дает $5 \times 10^2=500$ моментных условий (и рассмотренный вектор-строка имеет 500 элементов) с всего лишь 5 параметрами для оценки. Предельная значимость инструментов может быть очень маленькой в связи с увеличивающейся мультиколлинеарностью инструментов, что приводит к ситуации слабых инструментов. Один из полезных приёмов приемов --- те инструменты, которые незначительно меняются во времени, использовать как не меняющиеся во времени инструменты. Например, в качестве инструментов использовать только значения первого периода. Даже у регрессоров, которые значительно меняются во времени, могут быть использованы значения не во все возможные временные периоды, а только за несколько периодов.

Вычисление более эффективных оценок $\hat{\be}_{\mathrm{2SGMM}}$ невозможно при использовании только статистического пакета для двухшагового МНК. Вместо этого необходимо либо использовать более специализированные пакеты, либо запрограммировать оценку, используя матричные операции.

В таблице 22.1 кратко представлены четыре предположения об экзогенности и соответствующие годные инструменты.

\subsection{Другие типы оценивания ОММ}

Хотя $\hat{\theta}_{\mathrm{2SLS}}$ более эффективна, чем $\hat{\theta}_{\mathrm{2SLS}}$, некоторые исследования показывают, что она имеет большее смещение в связи с ограниченной выборкой,  чем $\hat{\theta}_{\mathrm{2SLS}}$, особенно когда $r$ намного больше, чем $K$. Для объяснения см. обсуждение смещения оптимального ОММ в связи с ограниченной выборкой в разделе 6.3.5.

Один из подходов состоит в том, чтобы быть умеренным в использовании инструментов, хотя в таком случае возможные выгоды в эффективности от использования дополнительных инструментов будут потеряны.

Несколько авторов предложили альтернативные ОММ оценки, которые с меньшей вероятностью страдают от смещения в ограниченных выборках. Многие из них представлены в разделе 6.4.4 и используются в исследовании панельных данных Зилиак (1997).

\begin{table}[ht]
\caption{{\it Предположения об экзогенности для панельных данных и инструменты}} 
\centering
\begin{tabular}{ccc}
\hline \hline
	Предположение об экзогенности & Моментное условие & Вектор инструментов  $^a$ \\
\hline
О сумме & $\E[\sum_t \mathbf z_{it} u_{it}=\mathbf 0$ & $[\mathbf z_{it}]$\\
Одновременная  & $\E[\mathbf z_{it} u_{it}=\mathbf 0$ для всех $t$ & $[\mathbf 0'_{r_1} \dots \mathbf 0'_{r_{t-1}} \mathbf z^t_{it} \mathbf 0'_{r_{t+1}} \dots \mathbf 0'_{r_{T}}]$ \\
Слабая	 & $\E[\mathbf z_{it} u_{it}=\mathbf 0$ $s\leq t$ для всех $t$ & $[\mathbf 0'_{r_1} \dots \mathbf 0'_{r_{t-1}} (\mathbf z^t_{it})' \mathbf 0'_{r_{t+1}} \dots \mathbf 0'_{r_{T}}]$\\
Сильная & $\E[\mathbf z_{it} u_{it}=\mathbf 0$ для всех $t$ и $s$ & $[\mathbf 0'_{r_1} \dots \mathbf 0'_{r_{t-1}} (\mathbf z^T_{it})' \mathbf 0'_{r_{t+1}} \dots \mathbf 0'_{r_{T}}]$ \\
\hline \hline
\multicolumn{3}{p{14cm}}{$^a$ Вектор инструментов --- это $t$-ая строка $\mathbf Z_i$ в \ref{Eq:22.11}; $(\mathbf z^t_{it})'=[\mathbf z'_{i1} \dots \mathbf z'_{it}]$, $(\mathbf z^T_{it})'=[\mathbf z'_{i1} \dots \mathbf z'_{iT}]$;  и $r_s=\mathrm{dim}[\mathbf z'_{is}]$ или $\mathrm{dim}[\mathbf z^s_{is}]$ или $\mathrm{dim}[\mathbf z^T_{is}]$.}\\
\end{tabular}
\label{Tab:22.1}
\end{table}

\subsection{Оценка Чемберлина}

Рассмотрим модель с индивидуальными эффектами
\begin{align}
y_{it}=\alpha_i +\x'_{it}\be +u_{it},
\label{Eq:22.20}
\end{align}
где регрессоры строго экзогенны, как в главе 21. В разделах 21.2.3 и 21.6.1 представлены методы получения робастных стандартных ошибок для оценки within для панельных данных.

Здесь требуется использование робастных статистических выводов для панельных данных в виду того, что $\e_{it}$ не являются независимыми и одинаково распределенными. Оценки, описанные в главе 21, не будут эффективными. Более эффективные оценки возможны благодаря использованию оптимального ОММ, примененного к сверх-идентифицируемой модели. Здесь $\x_{is}$, $s \neq t$, доступны как дополнительные инструменты и ОММ можно применить к преобразованной модели, если $\alpha_i$ необходимо элиминировать (см. раздел 22.4.2). Улучшение эффективности такое же, что и при использовании данных пространственного типа с гетероскедастичностью (см. раздел 6.3.5).

Чемберлин (1982, 1984) предложил следующую более эффективную оценку. Модель  \ref{Eq:22.20} можно представить в виде 
\begin{align}
\mathbf y_{i}=\mathbf e \alpha_i + (\mathbf I_T \otimes \be')\x_i+\mathbf u_i,
\label{Eq:22.21}
\end{align}
где $\mathbf e=(1, 1, \dots, 1)'$ --- единичный вектор размера $T \times 1$, $\x_i=[\x'_{i1} \dots \x'_{iT}]$ вектор размера $TK \times 1$, и $\mathbf y_i$ и $\mathbf u_i$ вектора размера $T \times 1$. Из уравнения \ref{Eq:22.21} понятны ограничения, которые сделаны неявно в статических моделях, в которых $y_{it}$ зависит только от $\x_{it}$ одновременных периодов. Чемберлин использовал аргументы линейной проекции, которые опираются на более слабые предположения, чем при условном ожидании. Пусть
\begin{align}
\E^*[\alpha_i| \x_i]=\mu+\sum_t \bm\lambda'_t \x_{it}=\bm\mu + \bm\lambda' \x_i,
\nonumber
\end{align}
где $\E^*$ обозначает линейную проекцию. Зная, что $\E[ \mathbf u_i | \alpha_i, \x_i]=\mathbf 0$, из \ref{Eq:22.21} получаем
\begin{align}
\E^*[\mathbf y_i| \x_i]=\mathbf e \mu+( \mathbf I_T \otimes  \be' +\mathbf e \bm\lambda') \x_i.
\nonumber
\end{align}
Это накладывает ограничения на неограниченную линейную проекцию
$\E^*[\mathbf y_i| \mathbf \x_i]=\pi_o + \pi'\x_i$, а именно $\pi-\mathbf I_T \otimes \be' + \mathbf e \bm\lambda'=\mathbf 0$.

Вместо использования ОММ, Чемберлин предложил следующую двухшаговую процедуру. Во-первых, с помощью многомерной МНК регрессии $\mathbf y_i$ на $\x_i$ и свободный член получить $\hat{\pi}$. Во-вторых, получить \textbf{оптимальную  оценку минимального расстояния} (см. раздел 6.7), при которой минимально
\begin{align}
Q_N(\be, \lambda)=(\mathrm{Vec}[\hat{\pi}-\mathbf I_T \otimes \be' - \mathbf e \lambda'])'\mathbf W_N(\mathrm{Vec}[\hat{pi}-\mathbf I_T \times \be' - \mathbf e \la']),
\nonumber
\end{align}
где оптимальная взвешивающая матрица $\mathbf W_N=(\hat{\mathrm{V}}[\mathrm{Vec}[\pi]])^{-1}$. Полученная оценка $\hat{\be}$ более эффективная, чем МНК оценка \ref{Eq:22.20}, если $u_{it}$ гетероскедастична.

Оценка минимального расстояния была вытеснена ОММ; см. Ареллано (2003, c. 22---23) и Крепон и Мересс (1995) для сравнения оценки минимального расстояния с ОММ оценкой. Однако подход Чемберлина получения моментных ограничений через предположения об экзогенности и предположения об индивидуальных эффектах оказали большое влияние на литературу, посвященную анализу панельных данных. Его оценка минимального расстояния также использовалась для оценивания ковариационных структур (см. раздел 22.5.4). 


\section{Пример оценивания ОММ для панельных данных: Часы и заработная плата}

Вернемся к примеру раздела 21.3 о заработной плате и количестве часов работы. В отличие от главы 21 регрессоры теперь могут быть эндогенными. Включим также фиксированные индивидуальные эффекты. Оценивание производится с помощью методов раздела 22.2 после взятия первых разностей для уничтожения фиксированных эффектов.

Регрессионная модель
\begin{align}
lnhrs_{it}=\alpha_i+\beta_1 lnwg_{it} + \beta_2 kids_{it} +\beta_3 age_{it} + \beta_4 agesq_{it} + \beta_5 disab_{it} + u_{it},
\nonumber
\end{align}


где нас интересует межвременная эластичность замещения предложения труда по заработной плате, $\be_1$, коэффициент lnwg. Дополнительные регрессоры - количество детей, возраст, возраст в квадрате, и индикатор нетрудоспособности.

МаКарди (1981) вывел эту взаимосвязь, используя модель жизненного цикла предложения труда в условиях неопределенности. Эта модель является <<$\la$-постоянной>>, где $\alpha_i$  равно $\la_i$ и пропорционально предельной полезности от первоначальное богатство, неизменно во времени, но изменяется по $i$. Так как $\la_i$ зависит от переменных и ограничений, его необходимо воспринимать как фиксированный, а не случайный эффект. В литературе, посвященной анализу предложения труда, представлено несколько методов для учета фиксированных эффектов.

Один метод, который будет обсуждаться далее в разделе 22.4.2, заключается во взятии разностей в регрессионной модели:
\begin{align}
\Delta lnhrs_{it}=\beta_1 \Delta lnwg_{it} + \beta_2\Delta kids_{it} +\beta_3 \Delta age_{it} + \beta_4 \Delta agesq_{it} + \beta_5 \Delta disab_{it} + \Delta u_{it}.
\label{Eq:22.22}
\end{align}
Если все регрессоры экзогенны, то МНК оценивание даст состоятельные оценки для $\be$. Заметим, что взятие разностей приводит к тому, что ошибки становятся коррелированными во времени, даже если $u_{it}$ независимы и одинаково распределены. Вследствие этого нужно использовать робастные стандартные ошибки для панельных данных.

Зилиак (1997) же предполагает, что $lnwg_{it}$ одновременно коррелирует с $u_{it}$ из-за ошибки измерения в заработной плате или из-за изломов кривой бюджетного ограничения. Тогда МНК оценка \ref{Eq:22.22} несостоятельна.

Зилиак предложил оценивание методом инструментальных переменных c использованием в качестве инструментов соответствующих лагов регрессоров. Предположим, что заработные платы прошлых периодов не коррелированы с ошибкой, вследствие чего lnwg будет слабо экзогенным помимо того, что он одновременно коррелирован с ошибкой. Тогда из $\E[lnwg_{is}u_{it}]=0$ при $s \leq t-1$ следует, что для ошибки модели в разностях $\E[lnwg_{is}\Delta u_{it}]=0$ при $s \leq t-2$. Значит, в качестве инструментов в модели в первых разностях могут быть использованы лаги второго и более высокого порядка. Заметим, что по крайней мере три периода данных нужны для идентифицируемости $\be$.

Центральным предметом исследования Зилиака являются характеристики ОММ оценок с эндогенными регрессорами. Он считал, что все регрессоры эндогенны и использовал в качестве инструментов лаги первого и более высокого порядков в уровнях других  четырех регрессоров. Для простоты не были включены свободный член и временные дамми, а также инструменты, неизменные для индивидуумов и используемые только один раз. В связи с включением свободного члена в состав зависимых переменных в модель в разностях результаты немного отличаются. Так как $lnwg_{i,t-2}$ всегда использовалось в качестве инструмента, первые два года были исключены и для оценки \ref{Eq:22.22} использовалось данные только за восемь лет 1981---1988.


\begin{table}[ht]
\caption{{\it  Часы и заработная плата: GMM-IV оценки линейных моделей панельных данных$^a$}} 
\centering
\begin{tabular}{cccccc}
\hline \hline
 & &	\multicolumn{2}{c}{Базовый случай} & \multicolumn{2}{c}{Составной}  \\
 & \textbf{МНК} &	\textbf{2SLS} & \textbf{2SGMM} & \textbf{2SLS} & \textbf{2SGMM}   \\
\hline
 $\beta_1$ & 0.112 &	0.209 & 0.547 & 0.543 & 0.330   \\
Панельные ст.ош. & (.096) &	 (.374)  & (.327)  & (.209) & (.110)   \\
Гетеро ст.ош. & [.079] & [.423] & [-] & [.226] & [-]   \\
Cт.ош. по умолчанию & \{.023\} &	 \{.389\}  & \{-\}  & \{.169\} & \{-\}   \\
RMSE & .283 &	 .296  & .307  & .307 & .298   \\
Инструменты & 5 &	 9  & 9  & 72 & 72   \\
OIR Tест & - &	 -  & 5.45  & - & 69.51   \\
ст. своб. & - &	 -  & 4  & - & 67   \\
P-значение & - &	 -  & .244  & - & .393   \\
N & 4256 & 4256  & 4256  & 4256 & 4256   \\
\hline \hline
\multicolumn{6}{p{14cm}}{$^a$ Для регрессии в разностях используются ежегодные данные за период с 1981 по 1988 гг. для 532 мужчин. В таблице представлены значения $\be_1$, коэффициент $\Delta lnwg$, и три типа оцененных стандартных ошибок: робастные для панельных данных в круглых скобочках, робастные к гетероскедастичности в квадратных скобках, и обычные оценки по умолчанию в фигурных скобках, которые предполагают независимые и одинаково распределенные ошибки. Все регрессии дополнительно включают $\Delta kids$, $\Delta age$, $\Delta agesq$ и $\Delta disab$ в качестве регрессоров, но оценки их коэффициентов не представлены. В качестве инструментов используются лаг lnwg второго порядка и лаги  kids, age, agesq и disab первого и второго порядка. Для базового случая используется 9 инструментов и $9 \times 8=72$ составных инструмента. RMSE -  средняя квадратичная ошибка остатков. OIR --- статистика теста на сверх-идентифицируемость ограничений, ст. своб. --- степени свободы и Р-значение --- это точное Р-значение для этого теста.}\\
\end{tabular}
\label{Tab:22.2}
\end{table}

Таблица \ref{Eq:22.2} представляет небольшую подвыбрку результатов, данных в таблицах 1 и 2 Зилиак (1997). Для полноты даны оценки различных стандартных ошибок, но использоваться должны робастные стандартные ошибки для панельных данных.

\textbf{МНК}: В колонке МНК представлены результаты МНК оценивания модели \ref{Eq:22.22}. Эластичность предложения труда 0.112 немного отличается от оценки 0.109 в колонке первых разностей таблицы \ref{Tab:21.2}, так как здесь четыре демографические переменные включены как регрессоры и данные дополнительного года были исключены из анализа. Так как моделируются первые разности, качество подгонки модели оставляет желать лучшего, и $R^2$ с включением свободного члена равен  0.006.

\textbf{2SLS  с инструментами базового случая}: Инструменты базового случая используют матрицу $\mathbf Z_i$, определенную в \ref{Eq:22.11}, где $\mathbf z_{it}$ состоит из девяти компонент: $\mathrm{lnwg}_{i,t-2}$, $\mathrm{kids}_{i,t-1}$, $\mathrm{age}_{i,t-1}$, $\mathrm{agesq}_{i,t-1}$, $\mathrm{disab}_{i,t-1}$, $\mathrm{kids}_{i,t-2}$, $\mathrm{age}_{i,t-2}$, $\mathrm{agesq}_{i,t-2}$ и $\mathrm{disab}_{i,t-2}$. Модель сверх-идентифицируема и включает девять инструментов  и пять параметров для оценки. Двухшаговая МНК оценка (2SLS) $\be_1$ менее точна, чем МНК оценка. Стандартные ошибки этой оценки превышают МНК оценки в четыре раза: 0.096 против 0.374. Для других регрессоров, которые не представлены в таблице, потеря эффективности намного меньше.

\textbf{2SLS  с составными инструментами}: Базовый случай --- это ОММ, основанный на девяти моментных условиях $\E[\sum^{10}_{t=3} \mathbf z_{it} u_{it}]=\mathbf 0$. В случае же составных инструментов используется 72 ($8 \times 9$) моментных условия $\E[\mathbf z_{it} u_{it}]=\mathbf 0, t=3,\dots,10$, где $\mathbf z_{it}$ то же, что и в базовом случае. Здесь используется матрица $\mathbf Z_i$, определенная в \ref{Eq:22.14}, где $\mathbf Z_i$ включает в себя 8 лет по 72 инструмента. $t$-ая строка $\mathbf Z_i$ представлена в \ref{Eq:22.17}, где $\mathbf z_{it}$ --- это вектор инструментов размерности $9 \times 1$ для базового случая. Для конструирования инструментов в первую очередь генерируется 72 переменных $ztj$, равные нулю для всех $i$ и $t$, где $t$ обозначает год и $j$ обозначает $j$й инструмент. Затем $ z sj_{it}$ заменяется на $ z_{it,j}$, если $t=s$, и остается равным нулю, если $t \neq s$. Например, если $t=3$ (третий год) $ z 35$ заменяется на $\mathrm{disabl}_{i,2}$, если пятый инструмент $\mathrm{disabl}_{i,t-1}$ и $ z t5$ равно нулю при $t \neq 3$. 2SLS оценки могут получены с помощью стандартной двухшаговой МНК регрессии (2SLS) $\Delta \mathrm{lnhrs}_{it}$ на пять регрессоров в \ref{Eq:22.22} c этими 72 переменными, сконструированными в качестве инструментов. При использовании расширенных инструментов стандартные ошибки упали с 0.374 до 0.209, и теперь они всего лишь в два раза больше стандартных ошибок МНК оценки.

\textbf{Двухшаговый ОММ}:  Двухшаговые ОММ оценки в таблице \ref{Eq:22.2} отличаются от оценок таблицы 1 Зилиака (1997), так как робастная оценка $\hat{\mathbf S}$, определенная в   \ref{Eq:22.7}, здесь используется для формирования взвешивающей матрицы, в то время как Зилиак использовал робастную к гетероскедастичности оценку $\hat{\mathbf S}=N^{-1} \sum_i\hat{u}^2_{it} \mathbf z_{it} \mathbf z'_{it}$. Как и ожидалось, двухшаговая ОММ оценка более эффективна, чем оценка двухшагового МНК (2SLS). Стандартные ошибки уменьшились с 0.374 до 0.327 в базовом случае, и с 0.209 до 0.110 --- в случае с составными инструментами. Эта последняя стандартная ошибка ненамного больше, чем стандартная ошибка для МНК оценки.

\textbf{Тест на сверх-идентифицирующие ограничения}: Тестовая статистика на сверх-идентифицирующие ограничения, когда $q >k$, дана в \ref{Eq:22.10}. Из таблицы \ref{Tab:22.2} для обоих базовых случаев и случаем в составными инструментами P-значения тестовых статистик намного выше 0.05, поэтому ограничения не отвергаются и мы делаем заключение, что перед нами годные инструменты. 

\textbf{Тест на слабые инструменты}: Способы диагностирования слабых инструментов были представлены  в разделе 22.2.4 и разделе 5.9. Так как ни один из регрессоров не используется как инструмент, используется общая F-статистика из регрессии первого шага, а не F-статистика подмножества регрессоров. Для инструментов базового случая, в случае регрессии $\Delta \mathrm{lnwg}$ на девять инструментов и константу получается робастная для панельных данных $F=2.80$, и в случае схожей регрессии для 72 составных инструментов  $F=1.90$. Это говорит о том, что, скорее всего, смещение в связи с конечностью выборки имеет место. В случае схожих регрессий для $\Delta \mathrm{kids}$, $\Delta \mathrm{age}$, $\Delta \mathrm{agesq}$, $\Delta \mathrm{disab}$, регрессоров модели \ref{Eq:22.22}, которые тоже считались эндогенными, $F > 8.5$  во всех случаях. Частный коэффициент детерминации $R^2$ (см. раздел 4.9.1) Шеа равен 0.0036 для $\Delta \mathrm{lnwg}$ и превышает 0.075 для  четырех других эндогенных регрессоров. Проблема слабых инструментов обуславливается трудностями в нахождении хороших инструментов для $\Delta \mathrm{lnwg}$.

\textbf{Увеличение эффективности}: В этом примере оценки ОММ были использованы для того, чтобы решить проблему эндогенности. Однако даже если все регрессоры строго экзогенны, ОММ оценки остаются привлекательными, так как они более эффективны, чем МНК, при независимых и одинаково распределенных ошибках $u_{it}$; см. обсуждение после \ref{Eq:22.20}. Например, двухшаговое ОММ оценивание, где в качестве инструментов используются базовые инструменты и пять первоначальных регрессоров в \ref{Eq:22.22}, дает $\hat{\be}_1=0.016$ со стандартными ошибками 0.076 (ниже, чем стандартные ошибки при МНК, равные 0.096).

\section{ОММ для панельных данных со случайными и фиксированными эффектами}

Теперь расширим модель панельных данных \ref{Eq:22.1}, включив в нее аддитивные \textbf{индивидуальные эффекты} $\alpha_i$, не меняющиеся во времени:
\begin{align}
y_{it}=\alpha_i+\x'_{it}\be + \e_{it}.
\label{Eq:22.23}
\end{align}
Тогда ошибка в \ref{Eq:22.1} имеет вид $u_{it}=\alpha_i +\e_{it}$. Для простоты одни и те же  обозначения используются и в моделях с фиксированными, и со случайными эффектами. В случае модели со случайными эффектами общий свободный член $\mu$ в разделе 21.7 включен в $\x'_{it} \be$.

Предполагается, что некоторые компоненты $\x_{it}$ \textbf{эндогенны}, $\E[\x_{it}(\alpha_i+\e_{it}] \neq \mathbf 0$, вследствие чего МНК оценка $\be$ несостоятельна. В этом разделе мы предлагаем оценивание методом инструментальных переменных, которое дает состоятельные оценки $\be$ в различных постановках моделей, включая модели с фиксированными и случайными эффектами, гибрид этих двух моделей и системы уравнений.

\subsection{Фиксированные или случайные эффекты?}

Вспомним из 21 главы, что индивидуальные эффекты $\alpha_i$ могут рассматриваться как случайные эффекты в моделях как с фиксированными, так и со случайными эффектами. Эта случайная переменная $\alpha_i$ не зависела от $\x_{it}$ в модели со случайными эффектами, но была коррелирована  с $\x_{it}$ в модели с фиксированными эффектами. В модели со случайными эффектами могут быть оценены коэффициенты любых регрессоров, в то время как в модели с фиксированными эффектами невозможно оценить коэффициенты регрессоров, не  меняющихся во времени. Это объясняется тем, что для состоятельного оценивания при взятии разности  уничтожаются $\alpha_i$ и регрессоры, не меняющиеся во времени.

В текущей главе, имея дело в том числе с эндогенными регрессорами, мы рассматриваем модель со \textbf{случайными эффектами}, если существуют инструменты $\mathbf Z_i$, удовлетворяющие $\E[\mathbf Z'_i(\alpha_i+\e_{it})]=\mathbf 0$. Тогда используя методы раздела 22.2, можно получить состоятельные оценки параметров всех регрессоров. Если возможно найти только те инструменты, для которых $\E[\mathbf Z'_i \e_{it}]=\mathbf 0$ и $\E[\mathbf Z'_i \alpha_i] \neq 0$, то модель рассматривается как модель с \textbf{фиксированными эффектами}. Тогда $\alpha_i$ 
должны быть уничтожены при взятии разности. И в этом случае идентифицируемы будут только коэффициенты изменяющихся во времени регрессоров.

\subsection{Инструментальные переменные для моделей с фиксированными эффектами}

Применяя к \ref{Eq:22.23} различные комбинации взятия разностей, данные в разделе 21.2, получаем \textbf{преобразованную модель} вида
\begin{align}
\tilde{y}_{it}= \tilde{\x}'_{it}\be+\tilde{\e}_{it},
\nonumber
\end{align}
где тильда $\tilde{}$  обозначает преобразование, при которой уничтожается $\alpha_i$. Основные примеры преобразований представлены ниже. Будем использовать запись
\begin{align}
\tilde{\mathbf y}_{i}= \tilde{\mathbf X}'_{i}\be+\tilde{\bm\e}_{i}.
\label{Eq:22.24}
\end{align}
Если $\E[\x_{it} \e_{it}] \neq 0$, тогда $\E[\tilde{\x_{it}} \tilde{\e}_{it}] \neq 0$ и МНК оценивание \ref{Eq:22.24} дает несостоятельные оценки.

Сейчас рассмотрим оценивание методом инструментальных переменных, предполагая существование инструментов $\mathbf Z_i$, которые удовлетворяют  условию $\E [\mathbf Z'_i \tilde{\bm\e}_i]=\mathbf 0$. Тогда ОММ оценивание (инструментальные переменные, IV, двухшаговый МНК, 2SLS, или двухшаговый ОММ, 2SGMM), модели \ref{Eq:22.24} с инструментами $\mathbf Z_i$ дает состоятельные оценки коэффициентов изменяющихся во времени регрессоров. Робастные стандартные ошибки для панельных данных могут быть вычислены так, как это обсуждалось в разделе 22.2.2.

Один из способов: инструменты могут быть получены аналогично случаю пространственных данных. Годный инструмент --- это переменная, которая коррелирует с регрессором, но не корелирована с ошибкой,  а также та, которая может быть исключена из правой части уравнения \ref{Eq:22.23}. Другой способ получить инструменты — через экзогенные регрессоры не текущих периодов с использованием предположения об экзогенности, которое подробно обсуждается в разделе 22.2.4.

Обычные предположения для пригодности использования инструментов основаны на корреляции между $\mathbf z_{is}$ и $\e_{it}$. Однако в текущем случае для нас важна корреляция между $\mathbf z_{is}$ и разности ошибки $\tilde \e_{it}$. В общем случае взятие разностей, необходимое для уничтожения фиксированных эффектов, уменьшает количество доступных инструментов. Некоторые операции взятия разностей  приводят к большим потерям, чем другие, и могут даже приводить к несостоятельному оцениванию методом инструментальных переменных. Мы рассмотрим три операции взятия разностей, уделяя особое внимание \textbf{слабо экзогенным инструментам}. На практике это может быть более реалистичным предположением, особенно применительно к динамическим моделям.

{\centering  Инструментальные переменные для модели в первых разностях}

\textbf{Оценка инструментальных переменных, IV, в первых разностях} --- это оценка инструментальных переменных, IV, или оценка двухшагового МНК, 2SLS, или ОММ оценка \textbf{модели в первых разностях}
\begin{align}
& y_{it}-y_{i,t-1}=(\x_{it}-\x_{i,t-1})'\be + (\e_{it}- \e_{i,t-1}),
& t=2, \dots, T.
\label{Eq:22.25}
\end{align}
Из предположения о слабой экзогенности $\E[ \mathbf z_{is} \e_{it}]=\mathbf 0$ для $s \leq t$ следует
$\E[\mathbf z_{is} (\e_{it}-\e_{i,t-1})]=\mathbf 0$ для $s \leq  t-1$. Взятие первых разностей уменьшает временной ряд доступных инструментов на один период, поэтому только $\mathbf z_{i,t-1}, \mathbf z_{i,t-2}, \dots$ доступны в качестве инструментов. Предполагая слабую экзогенность, получаем состоятельную IV оценку для $\be$.

Использование лагов регрессоров в качестве инструментов было впервые предложено Андерсоном и Хсяо (1981) в контексте динамических моделей панельных данных и было расширено Хольтц-Экином, Ньюи и Розеном (1988) и Ареллано Бондом (1991) (см. раздел 22.5.3). В разделе 22.3  дан подробный эмпирический пример этого подхода.

Заметим, что вместо этого можно использовать преобразованные инструменты $\tilde{\mathbf z}_{is}=\Delta \mathbf z_{is}=\mathbf z_{is}-\mathbf z_{i,s-1}$, $s \leq t-1$. Однако это не дает преимуществ, так как использование $\Delta \mathbf z_{i,t-1}, \dots, \Delta \mathbf z_{i2}, \mathbf z_{i1}$ эквивалентно использованию $\mathbf z_{i,t-1}, \dots, \mathbf z_{i2}, \mathbf z_{i1}$ в качестве инструментов, и, если данные начинаются в момент $t=1$, то в нашем распоряжении только $\mathbf z_{i1}$, а не $\Delta \mathbf z_{i1}$.

{\centering  IV для модели within или модель отклонения от среднего }

\textbf{IV оценка within} --- это IV или 2SLS или ОММ оценка \textbf{модели within} или \textbf{модели отклонения от среднего}
\begin{align}
& y_{it}-\bar{y}_{i}=(\x_{it}-\bar{\x}_{i})'\be + (\e_{it}- \bar{\e}_{i}).
\label{Eq:22.26}
\end{align}
Тогда $\E[\mathbf z_{is} \e_{it}]=\mathbf 0$ для $s \leq t$ больше не подразумевает, что $\E[\mathbf z_{is} (\e_{it}-\bar{\e}_i)]=\mathbf 0$ даже для $s$ значительного меньше, чем $t$. Продемонстрируем это. Предположим, что $\E[\mathbf z_{is} \e_{it}] \neq \mathbf 0$ для $s > t$. Тогда $\E[\mathbf z_{is} \bar{\e}_i] \neq \mathbf 0$ для всех $s$, так как в $\bar{\e}_i=T^{-1}\sum \e_{it}$ содержатся значения $\e_{it}$ прошлых периодов, которые коррелируют с $\mathbf z_{is}$.

Поэтому IV оценивание модели within дает несостоятельную оценку $\be$, если инструменты слабо экзогенны или удовлетворяют более слабому предположению об одновременной экзогенности или условию в виде суммы. Если инструменты строго экзогенны, можно использовать только преобразование within.

{\centering  IV для модели форвардных ортогональных отклонений }

Ареллано и Бовер (1995) предложили альтернативный взятию первых разностей метод, для которого требуется только слабая, а не строгая экзогенность инструментов. Мы также продемонстрируем этот метод, хотя метод взятия первых разностей используется гораздо чаще.

Для модели \ref{Eq:22.2}  для $i$-го наблюдения, после преобразования в первые разности получается модель $\mathbf D \mathbf y_i=\mathbf D \mathbf X_i \be + \mathbf D \bm\e_i$, где $\mathbf D$ --- это матрица размерности $(T-1) \times T$ с элементами $\mathbf D_{st}$, $t=1, \dots, T-1, s=1, \dots, T$, равными минус единице, если  $s=t$, единице, если $s=t+1$, и нулю в противном случае. Если $\e_{it}$ независимы и одинаково распределены, преобразованная ошибка описывается процессом MA(1)  и $\mathrm{V}[\mathbf D \mathbf u_i]=\sigma^2 \mathbf D \mathbf D'$. ОМНК оценка домножает $\mathbf D \mathbf \e_i$ на $(\mathbf D \mathbf D')^{-1/2}$, или домножает $\bm\e_i$ на $(\mathbf D \mathbf D')^{-1/2} \mathbf D$. Получается преобразованная модель  вида \ref{Eq:22.24}, где тильда обозначает предварительное умножение на $(\mathbf D \mathbf D')^{-1/2} \mathbf D$.

Если для получения $(\mathbf D \mathbf D')^{-1/2}$ была использована факторизация верхне-треугольной матрицы Холецкого  \textbf{модель форвардных ортогональных отклонений}
\begin{align}
c_t(y_{it}-\bar{y}^F_{it})=c_t(\x_{it}-\bar{\x}_{it}^F)'\be + c_t (\e_{it} - \bar{\e}^F_{it})
\label{Eq:22.27}
\end{align}
(см. Ареллано, 2003, p.17), где $c^2_t=(T-t)/(T-t+1)$, и верхний индекс $F$ обозначает, что для формирования среднего используются только будущие значения. Например, $\bar{y}^F_{it}=(T-t)^{-1} \sum^T_{s=t+1} y_{is}$.

Такое преобразование называется \textbf{ортогональными отклонениями (orthogonal deviations)}, так как преобразованные ошибки $c_t(\e_{it}-\bar{\e}_i^F)$ не коррелированы между собой и имеют единичную дисперсию. Прилагательное \textbf{форвардные} добавлено, так как преобразованные ошибки зависят только от текущих и будущих значений первоначальных ошибок. МНК оценивание \ref{Eq:22.27} дает оценку within, представленную в главе 21. Поэтому преобразование ортогональных отклонений оптимально, если $\e_{it}$ независимы и одинаково распределены.

\textbf{IV оценка форвардных ортогональных отклонений} --- это IV или 2SLS или ОММ оценка модели \ref{Eq:22.27}. Для слабо экзогенных инструментов из $\E[\mathbf z_{is} \e_{it}]=\mathbf 0$ для $s \leq t$ следует, что $\E[\mathbf z_{is} (\e_{it}-\bar{\e}^F_i)]=\mathbf 0$ для $s \leq t$. Следовательно, форвардные ортогональные отклонения не приводят к потере  количества доступных инструментов. Обычно к инструментам преобразование не применяется, так как в $(\mathbf z_{it} - \bar{\mathbf z}^F_i)$ включены будущие значения $\mathbf z_{it}$, которые зачастую коррелированы с $\e_{it}$.

\subsection{IV для моделей со случайными эффектами}

Модель для $i$-го наблюдения 
\begin{align}
\mathbf y_{i}=\mathbf X_i \be + \mathbf e \alpha_i + \bm\e_i,
\nonumber
\end{align}
где $ \mathbf e$ --- это единичный вектор размерности $T \times 1$. Состоятельные, но неэффективные оценки могут быть получены с помощью прямого применения ОММ для панельных данных раздела 22.2. При этом используются инструменты $ \mathbf Z_i$, полученные с помощью использования ограничений исключения или подходящих условий об экзогенности, т.е. $\E[ \mathbf Z'_i( \mathbf e \alpha_i + \e_i)]= \mathbf 0$. Здесь мы заглянем несколько глубже и рассмотрим более эффективное оценивание, в котором, как и в главе 21, учитывается корреляция ошибок во времени, когда ошибки модели имеют вид $u_{it}=\alpha_i +\e_{it}$. 

{\centering  IV оценивание преобразованной модели }

Предположим, что инструменты $\mathbf Z_i$  удовлетворяют условию $\E[\mathbf u_i| \mathbf Z_i]=\mathbf 0$ и $\mathrm V[\mathbf u_i | \mathbf Z_i]=\bm\Omega_i$, где $\bm\Omega_i$ имеет ту же форму, что и в случае стандартной модели со случайными эффектами, где на диагонали стоят $\sigma^2_{\alpha}+\sigma^2_{\e}$ и вне диагонали - $\sigma^2_{\alpha}$. Заметим, что это более строгое предположение, чем $\E[\mathbf Z'_i \mathbf u_i]=\mathbf 0$. Поэтому наложим ограничения на доступные инструменты.

При условном моментном ограничении $\E[\mathbf u_i| \mathbf Z_i]=\mathbf 0$, из раздела 6.3.7 оптимальное безусловное моментное ограничение имеет вид
\begin{align}
\E[\mathbf Z'_i \bm{\Omega}^{-1}_i \mathbf u_i]= \E[(\bm{\Omega}^{-1/2}_i \mathbf Z_i)'(\bm{\Omega}^{-1/2}_i \mathbf u_i)]=\mathbf 0.
\nonumber
\end{align}
Это приводит к ОММ оцениванию преобразованной системы $\mathbf y^*_i=\mathbf X^*_i\be+\mathbf u^*_i$ с преобразованными инструментами $\mathbf Z^*_i$. Звездочка обозначает умножение на матрицу $\bm\Omega^{-1/2}_i$ размерности $T \times T$ или ее состоятельную оценку $\hat{\bm\Omega}^{-1/2}_i$.

Из раздела 21.7.1 умножение на  $\hat{\bm\Omega}^{-1/2}_i$ приведет к модели
\begin{align}
y_{it}-\hat{\la}\bar{y}_i=(\x_{it}-\hat{\la}\bar{\x}_i)'\be+\{(1-\hat{\la})\alpha_i+(\e_{it}-\hat{\la}\bar{\e}_i)\},
\label{Eq:22.28}
\end{align}
где $\hat{\la}$ --- это состоятельная оценка $\la=1-\sigma_{\e}/\sqrt{\sigma^2_{\e}+T\sigma^2_{\alpha}}$. \textbf{IV оценка со случайным эффектом} --- это IV или 2SLS оценка этой модели с преобразованными инструментами $\tilde{\mathbf z}_{it}=(\mathbf z_{it} - \hat{\la} \bar{\mathbf z}_i$), или, что эквивалентно, с инструментами $\mathbf z_{it} - \bar{\mathbf z}_i$ и $\bar{\mathbf z}_i$.

Этот метод требует состоятельной оценки $\la$. Для $\sigma^2_{\e}$ мы используем $\hat{\sigma}^2_{\e}=\sum_i \tilde{\e}^2_{it}/N(T-1)$, где $\tilde{\e}_{it}$ --- остатки IV регрессии within $y_{it}-\bar{y}_i$ на $(\x_{it}-\bar{\x}_i)$ c инструментами $(\mathbf z_{it} - \bar{\mathbf z}_i)$ (см. \ref{Eq:22.26}). Также $(\sigma^2_{\e}+T\sigma^2_{\alpha})$ может быть оценено с помощью $\sum_i \bar{u}^2_i/N$, где $\bar{u}_i$ --- это остатки IV регрессии between $\bar{y}_i$ на $\bar{\x}_i$ с инструментами $\bar{\mathbf z}_i$. Результирующая IV оценка $\be$ называется \textbf{2SLS оценка с составной ошибкой} (Error Components 2SLS) (см. Бальтаджи, 1981).

Эти результаты зависят от спецификации определенной функциональной формы $\bm\Omega_i$. Результат в разделе 22.2.2 позволяет делать статистические выводы, которые робастны к неверной спецификации $\bm\Omega_i$, с использованием \ref{Eq:22.5}, где $\mathbf y, \mathbf X, \mathbf Z$ и $\mathbf W_N=[\mathbf Z' \mathbf Z]^{-1}$ заменяются на преобразованные переменные в \ref{Eq:22.28}.

Более важное ограничение состоит в том, что этот метод может быть использован, только если первоначальные инструменты строго экзогенны. Здесь для состоятельности необходимо выполнение предположения $\E[\mathbf Z'_i \bm\Omega^{-1}_i \mathbf u_i]=\mathbf 0$, более сильного предположения, чем $\E[\mathbf Z'_i \mathbf u_i]=\mathbf 0$, которое неизбежно требует $\E[\mathbf u_i| \mathbf Z_i]= \mathbf 0$. Например, предположим $\E[\mathbf z_{it} \alpha_i]=\mathbf 0$ для всех $t$, в то время как $\E[\mathbf z_{is} \e_{it}]=0$ для $ s \leq t$, а $\E[ \mathbf z_{it} \e_{it}] \neq \mathbf 0$ для $s > t$. Тогда $\E[\mathbf z_{it} \e_{it}] \neq \mathbf 0$, что приведет к корреляции инструментов с ошибкой в \ref{Eq:22.28}.

\subsection{IV для гибридной модели Хаусмана-Тейлора}

Один из распространенных примеров эндогенности --- включение регрессоров, коррелированых с индифидуальными эффектами $\alpha$. Это приводит к несостоятельности оценки со случайным эффектом главы 21. Очевидное решение  --- вместо этого использовать состоятельную оценку within (или оценку с фиксированным эффектом). Однако в таком случае коэффициенты индивидуальных регрессоров, не меняющихся во времени, не могут быть идентифицированы. Это лишает многие исследования на панельных данных их основной цели --- оценивание эффектов не меняющихся во времени регрессоров, таких как эффект уровня образования  в регрессиях, описывающих уровень заработка.

Хаусман и Тейлор (1981) рассмотрели следующий вариант \ref{Eq:22.23}:
\begin{align}
y_{it}=\x'_{1it}\be_1+\x'_{2it}\be_2 + \mathbf w'_{1i}\gamma_1+ \mathbf w'_{2i} \gamma_2 + \alpha_i + \e_{it},
\label{Eq:22.29}
\end{align}
где некоторые регрессоры по предположению коррелируют с $\alpha_i$, а другие нет. $\mathbf w$ обозначает регрессоры, не меняющиеся во времени. А именно $\x_{1it}$ и $\mathbf w_{1i}$ не коррелированы c $\alpha_i$, и $\x_{2it}$ и $\mathbf w_{2i}$ коррелированы c $\alpha_i$. Предполагается, что все регрессоры некоррелированы с $\e_{it}$. В этой модели $\alpha_i$ можно рассматривать как своеобразный \textbf{гибрид} случайных и фиксированных эффектов.

Хаусман и Тейлор (1981) предложили два способа использования меняющихся во времени экзогенных регрессоров $\x_{1it}$: для оценивания $\be_1$  и в качестве инструментов для  $\mathbf w_{2i}$, позволяющих оценивание $\gamma$. Тогда $\gamma$ идентифицируемо, если количество меняющихся во времени экзогенных регрессоров равно или превышает количество неизменных во времени эндогенных регрессоров. Амэмия и МаКарди (1986) предложили более эффективную оценку, которая использует $\x_{1it}$ $(T+1)$ способами: для оценивания $\be_1$ и в качестве $T$ инструментов для $\mathbf w_{2i}$, позволяющих идентификацию, если $\mathrm{dim}[\mathbf w_{2i}] \geq T \mathrm{dim}[\x_{1it}]$. Такой подход получения инструментов из значений экзогенных регрессоров не текущего периода подробно обсуждался в разделе 22.2.4.

Различные \textbf{проекции}, некоторые эквиваленты могут быть использованы для формирования подходящих инструментов. Бройш, Мизон и  Шмидт (1989) предложили более простой вид и проекцию, которая позволяет оценивание при использовании только пакета для двухшагового МНК (2SLS).

Для начала рассмотрим состоятельную, но неэффективную оценку, которая игнорирует корреляционную структуру $(\alpha_i+\e_{it})$. После преобразования within уничтожается корреляция с $\alpha_i$, благодаря чему $\ddot{\x}_{2it}=\x_{2it}-\bar{\x}_{2i}$ может быть использован как инструмент для эндогенного $\x_{2it}$. Инструмент для $\x_{1it}$ --- $\ddot{\x}_{1it}$, а не $\x_{1it}$. Тогда $\bar{\x}_{1i}$ используется как инструмент для эндогенного $\mathbf w_{2i}$, а экзогенный $\mathbf w_{1i}$ используется как инструмент для самого себя.

Сейчас рассмотрим эффективное оценивание в предположении случайных эффектов о том, что компоненты $\alpha_i$ и $\e_{it}$ гомоскедастичны. Тогда из \ref{Eq:22.27} после применения \textbf{преобразования взятия разностей со случайным эффектом} (см. \ref{Eq:22.28}) получается
\begin{align}
\tilde{y}_{it}=\tilde{\x}'_{1it}\be_1+\tilde{\x}'_{2it}\be_2 + \tilde{\mathbf w}'_{1i}\gamma_1+ \tilde{\mathbf w}'_{2i} \gamma_2  + v_{it},
\label{Eq:22.30}
\end{align}
где, например, $\tilde{\x}_{1it}=\tilde{\x}_{1it}-\hat{\la}\bar{\x}_{1i}$, где оценка для скалярной величины $\hat{\la}$  была представлена в конце предыдущего раздела. Оценка Хаусмана-Тейлора эквивалентна IV оцениванию \ref{Eq:22.30}  с использованием инструментов $\ddot{\x}_{1it}, \ddot{\x}_{2it}, \mathbf w_{1i}$ и $\bar{\x}_{1i}$. Экзогенные меняющиеся во времени регрессоры $\x_{1it}=\ddot{\x}_{1it}+\bar{\x}_{1i}$ дважды используются как инструменты, в которых преобразование within $\ddot{\x}_{1it}$ используется как инструмент для $\x_{1it}$ и среднее по времени $\bar{\x}_{1i}$ используется как инструмент для $\mathbf w_{2i}$. Оценка Амэмия и МаКарди (1986) в качестве инструментов использует $\ddot \x_{1it}, \ddot \x_{2it}, \mathbf w_{1i}$ и $\x_{1it}, \dots, \x_{1iT}$, так что в качестве инструментов используются значения во все значения временные периоды, а не только среднее по $t$. Для этого необходимо, чтобы  $\E[\x_{1it} \alpha_i] = \mathbf 0$ для $t=1, \dots, T$. Это более сильное предположение, чем  $\E[\bar{\x}_{1i} \alpha_i]=\mathbf 0$ (см. раздел 22.2.4). Бройш и др. (1989) предложили даже более эффективную оценку с использованием $\ddot{\x}_{2is}, s \neq t$ в качестве дополнительных инструментов.

Основное ограничение этого подхода состоит в том, что он требует разграничения регрессоров на коррелированные и некоррелированные с $\alpha_i$. В регрессии отдачи от образования Хаусман и Тейлор начинают с предположения, что все три меняющихся во времени регрессора (опыт работы, плохое здоровье и прошлогодний уровень безработицы) экзогенны, два не меняющихся регрессора (раса и семейное положение) экзогенны, и не меняющийся во времени регрессор (уровень образования) эндогенный. В такой спецификации есть два сверх-идентифицирующих ограничения. Тестирование спецификации модели возможно с помощью теста Хаусмана, основанном на разнице между $\hat{\be}_{HT}$ и $\hat{\be}_W$, так как оценка within для $\be$, несмотря на то, какие компонент $\x_{it}$ и $\mathbf w_i$ коррелированы с $\alpha_i$. В своем эмпирическом исследовании Корнуолл и Руперт (1988) сравнивают различные оценки.

\subsection{Внешне не связанные уравнения и оценка одновременных уравнений}

В предшествующем анализе панельных данных внимание было сосредоточено исключительно на оценивании одного уравнения в отдельности. В некоторых случаях желательно оценивать систему уравнений, такую как систему уравнений спроса, где зависимые переменные и регрессоры наблюдаются для многих индивидуумов в разные периоды времени. Если нет ограничений на параметры, то оценивание одного уравнения дает состоятельные оценки.  Более эффективная оценка возможна благодаря совместному оцениванию уравнений, которая учитывает корреляцию между ошибками разных уравнений.

В рамках подхода со строго экзогенными регрессорами главы 21, более эффективная оценка --- это обобщение оценки внешне не связанных уравнений (seemingly unrelated regressions, SUR) от случая пространственных данных до панельных данных. В \textbf{модели внешне не связанных уравнений с составной ошибкой, error components SUR}  $g$-е из $G$ уравнений задается в виде
\begin{align}
& y_{git}=\x'_{git}\be+\alpha_{gi}+\e_{git},
&g=1, \dots, G,
\label{Eq:22.31}
\end{align}
где, как и в случае панельных данных, для $\alpha_{gi}$ выполняется независимость по $i$, $\e_{dit}$ --- независимость по $i$ и $t$, $\alpha_{gi}$ и $\e_{git}$ не зависят друг от друга. Однако компоненты ошибок могут коррелировать между собой, т.е. $\mathrm{Cov}[\alpha_{gi},\alpha_{hi}] \neq 0$ и $\mathrm{Cov}[\alpha_{git},\alpha_{hit}] \neq 0$ для $g \neq h$. Тогда методы главы 21 дают состоятельные оценки. Очевидная оценка --- это оценка со случайным эффектом, т.е. доступный ОМНК, учитывающий корреляцию внутри каждого уравнения. Более эффективные ОМНК оценки, которые дополнительно учитывают корреляцию  ошибок между уравнениями, подробно описаны у Авери (1997) и Бальтаджи (1980).

Подобное увеличение эффективности возможно в случае, когда одно из \textbf{одновременных уравнений}, где регрессор $\x_{git}$ в \ref{Eq:22.31} может включать один или более эндогенных регрессоров $y_{hit}$  из других уравнений. Тогда IV или ОММ каждого уравнения в отдельности дает состоятельные оценки. Очевидной оценкой при составной структуре ошибок будет IV или EC2SLS оценка раздела 22.4.3. Более эффективные оценки получаются при использовании \textbf{оценки трехшагового МНК с составной ошибкой} (Error components 3SLS, EC3SLS), предложенной Бальтаджи (1981).

Оценки систем уравнений получить сложнее, поэтому более адекватным может быть отдельное оценивание каждого уравнения. Но даже при таком подходе к оцениванию, выгодно специфицировать систему одновременных уравнений, так как это позволяет идентификацию коэффициентов эндогенных регрессоров с использованием экзогенных регрессоров, исключенных из основного уравнения, в качестве инструментов.  Это более традиционный подход получения инструментов, чем использование экзогенных регрессоров остальных периодов в качестве инструментов.


\section{Динамические модели}

В этом разделе мы будем рассматривать обычные модели панельных данных с индивидуальными эффектами с тем усложнением, что регрессоры будут включать первые лаги зависимых переменных. Тогда мы получаем \textbf{динамическую модель}
\begin{align}
& y_{it}=\gamma y_{i,t-1} +\x'_{it} \be +\alpha_i+\e_{it},
& i=1, \dots, N &
&t=1, \dots, T.
\label{Eq:22.32}
\end{align}
Как обычно панель является короткой, а данные  независимы по $i$. Предполагается, что $|\gamma| < 1$. Это предположение ослабляется в разделе 22.5.4.

Важным результатом является то, что даже если $\alpha_i$  --- это случайный эффект, МНК оценивание \ref{Eq:22.32} приводит к несостоятельной оценке $\gamma$  и $\be$. Это объясняется тем, что регрессор $y_{i,t-1}$ коррелирован с $\alpha_i$, а поэтому и с составной ошибкой $(\alpha_i+\e_{it})$. Даже в случае со случайными эффектами необходимы альтернативные оценки.

Мы будем рассматривать оценивание, когда $\alpha_i$ --- это фиксированный эффект, $|\gamma| <1$, ошибка $\e_{it}$ не коррелирована во времени, а панель является короткой (см. раздел 22.5.3). Хотя это базовый случай для микроэконометрических приложений, существует обширный объем литературы, в которой меняется одно или более из этих предположений. Индивидуальные эффекты могут быть чисто случайными, ошибки могут быть коррелированы во времени, данные могут быть нестационарными. Кроме того, панель может быть длинной, но мы всего лишь частично затронем эту литературу. 

\subsection{Зависимость от состояния и ненаблюдаемая гетерогенность}

Прежде чем рассматривать оценивание, заметим, что корреляция $y_{it}$ во времени теперь напрямую вызывается $y_{i,t-1}$ в дополнении к косвенным эффектам через $\alpha_i$, которые уже рассматривалось в главе 21. Эти две причины приводят к довольно  разным интерпретациям \textbf{корреляции во времени}, например, в индивидуальной заработной плате или восприятии богатства.

Для простоты пусть $\be=\mathbf 0$, так что $y_{it}=\gamma y_{i,t-1}+\alpha_i+\e_{it}$. Тогда $\E[y_{it}|y_{i,t-1}, \alpha_i]=\gamma y_{i,t-1} + \alpha_i$  и $Cor[y_{it}, y_{i,t-1}|\alpha_i]=\gamma$. При фиксированных $\alpha_i$ применимы стандартные результаты для модели AR(1) с зависимостью по времени в $y_{it}$, определяемой только авторегрессионным параметром $\gamma$. Однако $\alpha_i$  неизвестно и мы на самом деле наблюдаем $\E[y_{it}|y_{i,t-1}]=\gamma y_{i,t-1} + \E[\alpha_i|y_{i,t-1}]$  и $Cor[y_{it},y_{i,t-1}]= \neq \gamma$. Из \ref{Eq:22.32} с $\be=\mathbf 0$
\begin{align}
\mathrm{Cor}[y_{it}, y_{i,t-1}]
&=\mathrm{Cor}[\gamma y_{i,t-1}+\alpha_i+\e_{it},y_{i,t-1}] \nonumber \\
&=\gamma+ \mathrm{Cor}[\alpha_i,y_{i,t-1}] \nonumber \\
&=\gamma+ \frac{(1-\gamma)}{1+(1-\gamma)\sigma^2_{\e}/(1+\gamma)\sigma^2_{\alpha}}, \nonumber \\
\label{Eq:22.33}
\end{align},
где второе равенство предполагает $\mathrm{Cor}[\e_{it},y_{i,t-1}]=\mathrm 0$, и третье равенство получается после некоторых алгебраических преобразований для частного  случая случайных эффектов при независимых и одинаково распределенных $\e_{it}$ с параметрами $[0, \sigma^2_\e]$ и независимых и одинаково распределенных $\alpha_i$ с параметрами $[0, \sigma^2_{\alpha}]$.

Из результата \ref{Eq:22.33} становится понятно, что есть две возможные причины корреляции $y_{it}$ и $y_{it-1}$.

\textbf{Зависимость от состояния, True state dependence} имеет место тогда, когда корреляция во времени объясняется обычным механизмом: $y_{i,t-1}$ последнего периода определяет $y_{it}$ текущего периода. Эта зависимость относительно велика, если индивидуальный эффект $\alpha_i \simeq 0$, так как в таком случае $Cor[y_{it}, y_{i,t-1}] \simeq \gamma$. Как правило, эту случается когда $\sigma^2_{\alpha}$ очень мала по сравнению с $\sigma^2_{\e}$. 

Корреляция, обусловленная \textbf{ненаблюдаемой гетерогенностью}, имеет место быть даже в отсутствие вышеописанного механизма, т.е. если  $\gamma=0$. Но тем не менее корреляция все же присутствует, так как $\mathrm Cor[y_{it},y_{i,t-1}]$ упрощается до $\sigma^2_{\alpha}/(\sigma^2_{\alpha}+\sigma^2_{\e})$,  если $\gamma=0$, как в главе 21.

Обе крайности допускают то, что корреляция будет близка к единице, так как либо $\gamma \rightarrow 1$, либо $\sigma^2_{\e}/\sigma^2_{\alpha}$. Однако этим двум случаям соответствуют два совершенно разных объяснения с различными выводами для принятия решений. Объяснение зависимости от сложившегося состояния, что доходы $y_{it}$ постоянно высоки даже при учете регрессоров $x_{it}$, состоит в том, что будущие доходы определяются прошлыми и $\gamma$ большое. Ненаблюдаемая гетерогенность объясняется тем, что в действительности $\gamma$ маленькое, но некоторые существенно важные переменные были пропущены, что привело к высокому значению $\alpha_i$ в каждом периоде. Для данных длительности состояний различие между зависимости от состояния и ненаблюдаемой гетерогенностью было рассмотрено в главе 18. В статических линейных моделях панельных данных главы 21 принималась во внимание только ненаблюдаемая  гетерогенность.

\subsection{Несостоятельность стандартных оценок панельных данных}

Оценки из предыдущей главы \textbf{состоятельны}, если в регрессоры включены лаги зависимых переменных, даже в случае модели со случайными эффектами. Рассмотрим оценивание модели \ref{Eq:22.32}, где обычно предполагается, что $\e_{it}$ некоррелированы во времени.

Во-первых, рассмотрим \textbf{МНК оценивание} $y_{it}$  на $y_{i,t-1}$ и $\x_{it}$. Ошибка имеет вид $(\alpha_i+\e_{it})$. Она коррелирована с регрессором $y_{i,t-1}$, так как  $y_{i,t-1}=\gamma y_{i,t-2}+\x'_{i,t-1}\be+\alpha_i+\e_{i,t-1}$, вследствие чего $y_{i,t-1}$ коррелирована с $\alpha_i$. Заметим, что это отличается от предыдущего результата для МНК оценивания модели со случайными эффектами, не включающих лаги зависимых переменных. В том случае МНК регрессия $y_{it}$ на $\x_{it}$ дает состоятельные, хотя и неэффективные, оценки. Это также отличается от обычного результата, что МНК регрессия $y_{it}$  на $y_{i,t-1}$ дает состоятельные оценки (хотя и слегка смещенные на малых выборках), если ошибка не коррелирована во времени.

Во-вторых, рассмотрим \textbf{оценку within}, которая получается из оценивания регрессии $(y_{it}-\bar{y}_i)$ на $(y_{i,t-1}-\bar{y}_{i,t-1})$ и $(\x_{it}-\bar{\x}_i)$. В этой регрессии ошибка имеет вид $(\e_{it}-\bar{\e}_i)$. Сейчас по \ref{Eq:22.32} $y_{it}$ коррелирован с $\e_{it}$,   $y_{i,t-1}$ коррелирован с $\e_{i,t-1}$, а следовательно и с $\bar{\e}_i$. Однако из этого следует, что регрессор $(y_{i,t-1}-\bar{y}_i)$  коррелирован с ошибкой $(\e_{it}-\bar{\e}_i)$. Тогда МНК оценивание модели within приводит к несостоятельным оценкам параметров, так как регрессор коррелирован с ошибкой. Для состоятельности нужно, чтобы $\bar{\e}_i$ была очень мала по сравнению с $\e_{it}$. Для этого необходимо, чтобы $T \rightarrow \infty$, т.е. чтобы панель была длинная, а не короткая. Это рассматривает Никелл (1981).

 \textbf{Оценка со случайным эффектом}, данная в главе 21, тоже может быть несостоятельна, так как это линейная комбинация оценок within и between. Для моделей со случайными эффектами Андерсен и Хсяо (1981) рассматривали ММП оценивание, когда $\e_{it} \thicksim \mathcal N[0. \sigma^2]$; см. также Бхаргава и Сарган (1981). В коротких панелях распределение оценки ММП зависит от предположений относительно $y_{i0}$, первоначального значения зависимой переменной. Андерсен и Хсяо (1981) различали несколько предположений относительно \textbf{начальных условий}: (1) фиксированные начальные наблюдения, (2) случайные начальные наблюдения с общим средним, (3) случайные начальные наблюдения с разными средними, и (4) случайные начальные наблюдения со стационарными распределениями.

\textbf{МНК оценка в первых разностях} также несостоятельна. Состоятельные оценки могут быть получены с помощью IV подхода. Сейчас приступим к описанию этой оценки.

\subsection{Оценка Ареллано-Бонда}
Модель \ref{Eq:22.32} приводит к модели в первых разностях
\begin{align}
& y_{it}-y_{i,t-1}=\gamma (y_{i,t-1} - y_{i,t-2})+(\x'_{it} -\x_{i,t-1})'\be + (\e_{it}-\e_{i,t-1}),
&t=2, \dots, T.
\label{Eq:22.34}
\end{align}
МНК оценка несостоятельна, так как $y_{i,t-1}$ коррелирован с $\e_{i,t-1}$ из \ref{Eq:22.32}, поэтому регрессор $(y_{i,t-1}-y_{i,t-2})$ коррелирован с ошибкой $(\e_{it}-\e_{i,t-1})$  в \ref{Eq:22.34}.

Андерсен и Хсяо (1981) предложили оценивать \ref{Eq:22.34}, используя \textbf{IV оценку}  с инструментом $y_{i,t-2}$ для $(y_{i,t-1}-y_{i,t-2})$. Этот инструмент является годным, так как $y_{i,t-2}$ не коррелирован с $(\e_{it}-\e_{i,t-1})$ в предположении некоррелированности ошибок $\e_{it}$ во времени. Более того, $y_{i,t-2}$ --- хороший инструмент, так как он коррелирован с $(y_{i,t-1}-y_{i,t-2})$. Метод требует наличия хотя бы трех временных периодов данных для каждого индивидуума. Можно использовать $\Delta y_{i,t-2}$ как инструмент для $\Delta y_{i,t-1}$, для чего требуется как минимум четыре периода для одного наблюдения. Андерсон и Хсяо (1981) сделали вывод, что в обычном случае ($\gamma > 0$) IV оценка более эффективна, если в качестве инструмента используется $\Delta y_{i,t-2}$, а не $y_{i,t-2}$. В другом случае $(\x_{it}-\x_{i,t-1})$ используется как инструмент для самого себя.

\textbf{Более эффективное оценивание} возможно благодаря использованию дополнительных лагов зависимых переменных в качестве инструментов. Например, и $y_{i,t-2}$, и $y_{i,t-3}$ можно использовать как инструменты. Тогда модель сверх-идентифицируема, и оценивание должно проводиться с помощью двухшагового МНК или ОММ для панельных данных. Более того, количество доступных инструментов для зависимой переменной будет наибольшим в момент $t$, который наиболее близок к $T$. В момент времени 3 в качестве инструмента доступно только $y_{i1}$, в момент времени 4 доступны  $y_{i1}$ и  $y_{i2}$, в 5 ---  $y_{i1}$,  $y_{i2}$
 и  $y_{i3}$ и т.д. Хольтц-Экин и др. (1988), а также  Ареллано и Бонд (1991) предложили ОММ оценки с использованием более широких несбалансированных наборов инструментов.

В литературе по микроэконометрике результирующая ОММ оценка для панельных данных называется \textbf{оценкой Ареллано-Бонда}. Общая процедура уже была представлена в разделе 22.4.2, где динамический аспект не был описан явно. Оценка
\begin{align}
\hat{\be}_{AB}=\left[ \left( \sum^N_{i=1} \tilde{\mathbf X}'_i \mathbf Z_i \right) \mathbf W_N
\left( \sum^N_{i=1} \mathbf Z_i  \tilde{\mathbf X}'_i \right) \right]^{-1}
\left( \sum^N_{i=1} \tilde{\mathbf X}'_i \mathbf Z_i \right) \mathbf W_N
\left( \sum^N_{i=1} \mathbf Z_i  \tilde{\mathbf y}'_i \right)
\label{Eq:22.35}
\end{align}
где $\tilde{\mathbf X}_i$ --- это матрица размерности $(T-2) \times (K +1)$ с $t$-ой строкой $(\Delta y_{i,t-1}, \Delta \x'_{it}$, $t=3, \dots, T, \tilde{\mathbf y}_i$ --- это вектор размерности $(T-2) \times 1$ с $t$-ой строкой $\Delta y_{it}$, и $\mathbf Z_i$  --- матрица инструментов размерности $(T-2) \times r$
\begin{align}
\mathbf Z_i=
\begin{bmatrix}
\mathbf z'_{i3} & \mathbf 0 & \dots & \mathbf 0 \\
\mathbf 0 & \mathbf z'_{i4} & & \vdots \\
\vdots & & \ddots & \mathbf 0 \\
\mathbf 0 & \dots & \mathbf 0 & \mathbf z'_{iT} 
\end{bmatrix},
\label{Eq:22.36}
\end{align}
где $\mathbf z'_{it} = [y_{i,t-2}, y_{i,t-3}, \dots, y_{1i}, \Delta \x'_{it}]$. Лаги $\x_{it}$ или $\Delta \x_{it})$ могут быть дополнительно использованы как инструменты, и для средних или больших $T$ максимальный лаг $y_{it}$, используемый в качестве инструмента, может не превышать $y_{i,t-4}$. Двухшаговый МНК и двухшаговый ОММ соответствуют разным взвешивающим матрицам $\mathbf W_N$ (см. раздел 22.2.3).

Метод достаточно просто может быть применен к модели AR(p), с $\gamma y_{i,t-1}$ в \ref{Eq:22.32}, замененным на $\gamma_1 y_{i,t-1}+\gamma_2 y_{i,t-2}+ \dots + \gamma_p y_{i,t-p}$, хотя для состоятельной оценки нужно более, чем три периода данных.

Пример в разделе 22.3 --- это главным образом  пример  оценивания Ареллано-Бонда, так как  к модели в первых разностях применяется IV оценивание с лаговыми значениями регрессоров в качестве инструментов.

Ан и Шмидт (1995) заметили, что возможно еще более эффективное оценивание с использованием дополнительных моментных условий. Рассмотрим версию \ref{Eq:22.32}, где $\be= \mathbf 0$ и сделаем стандартное предположение, что $\e_{it}$ некоррелировано с $\alpha_i$, $\e_{is}$ для $s \neq t$ и наблюдения $y_{i1}$. Оценка Ареллано-Бонда  использует моментное условие $\E[y_{is} \Delta u_{it}]=0$ для $s \leq t-2$, где  $u_{it}=\e_{it}+\alpha_i$. Ан и Шмидт (1995) получили более эффективную оценку, используя дополнительно моментные условия $\E[u_{iT}\Delta u_{it}]=0$. Они показали, что эта оценка эффективно использует предположения о моментах второго порядка и асимптотически эквивалентна оптимальной оценке  минимального расстояния Чемберлина (1982, 1984).

Дополнительные предположения позволяют использовать дополнительные моментные условия, и поэтому получать более эффективные оценки. Если $\mathrm V[\e_{it}]=\mathrm V[\e_{is}]$, то в предположении о гомоскедастичности $\e_{it}$ $\E[\bar{u}_i \Delta u_{it}]=0$ (см. Ан и Шмидт, 1995). Ареллано и Бовер (1995) предложили использовать условие $\E[u_{it} \Delta y_{is}]=0$  для $s \leq t-1$. Бланделл и Бонд (1998) рассматривают эти и другие предположения и показывают, что выигрыш может быть значительным, особенно когда $\gamma$ принимает высокое значение, а $T$ --- маленькое. Ареллано и Оноре (2001) представляют множество возможных предположений и соответствующие им моментные условия, которые могут быть использованы в оценивании.

Хсяо, Песаран, и Тамишоглу (2002) предлагают \textbf{преобразованную ММП оценку}. Предположим, что $\e_{it}$ независимы и имеют нормальное распределение $\mathcal N [0, \sigma^2]$. Это предположение может быть ослаблено. Вместо того, чтобы формировать функцию максимального правдоподобия на основе $\e_{i1}, \dots, \e_{iT}$, они основывают функцию правдоподобия на разностях ошибок $\Delta \e_{i1}, \dots, \Delta \e_{iT}$. Для модели временного ряда AR(1) $\Delta \e_{it}=\Delta y_{it}  - \gamma \Delta y_{i,t-1}$ для $t >1$. Плотность $\Delta \e_{i1}$ зависит от предположений относительно первоначальных условий: или $\Delta \e_{i1} = \Delta y_{i1}$, или $\Delta \e_{i1} = \Delta y_{i1}-b$, где $b=\E[\Delta y_{i1}]$ --- дополнительный параметр для оценки. Результирующая оценка --- квази-ММП оценка, которая обеспечивает состоятельность, даже когда $\e_{it}$ не имеет нормального распределения. Если  $\e_{it}$ независимы и одинаково распределены с параметрами $[0,\sigma^2]$, то преобразованная оценка ММП более эффективна, чем предшествующие ОММ оценки.


\subsection{Оценивание ковариационных структур}

\textbf{Ковариационные структуры} --- это модели, которые специфицируют структуру ковариационной матрицы ошибки регрессии. Приложения включают структуры для динамики ошибок и для ошибки измерения. Цель --- оценить параметры структуры.

Например, предположим, что процесс, генерирующий $y_{it}$, является моделью со случайными эффектами с ошибками вида MA(1), т.е.
\begin{align}
y_{it}=\alpha_i+\e_{it}+\phi \e_{i,t-1},
\nonumber
\end{align}
где $\alpha_i \thicksim [0, \sigma^2_{\alpha}]$ и $\e_{it} \thicksim [0, \sigma^2_\e]$ и $|\phi| <1$. Тогда автокорреляции $\gamma_j=\mathrm{Cov}[y_{it}, y_{i,t-j}]$ удовлетворяют $\gamma_0=\sigma^2_{\alpha}+(1+\phi^2)\sigma^2_\e, \gamma_1=\sigma^2_{\alpha}+\phi \sigma^2_{\e}$ и $\gamma_j=\sigma^2_{\alpha}$ для $j \geq 2$. Если $T=3$, то из этих уравнений получаются оценки $\hat{\sigma}^2_{\alpha}$,  $\hat{\sigma}^2_{\e}$, и $\hat{\phi}$ при автоковариациях $\hat{\gamma}_0$, $\hat{\gamma}_1$ и $\hat{\gamma}_2$. Если $T>3$, то модель сверх-идентифицируема и остается только три параметра для оценки, но более трех оценок ковариаций. Самым очевидным будет использование оценки минимального расстояния.

Пусть $\bm\theta$ обозначает $q$  структурных параметров, и предположим, что $\mathbf{g}(\bm\theta)=\bm\gamma$, где $\gamma=[\gamma_0, \dots, \gamma_{T-1}]'$ --- это вектор $T \geq q$ автоковариаций. Тогда \textbf{оценка минимального расстояния} $\hat{\bm\theta}_{MD}$ минимизирует
\begin{align}
Q_N(\bm\theta)=(\hat{\bm\gamma}-\mathbf{g}(\bm\theta))'\mathbf{W}_N (\hat{\bm\gamma}-\mathbf{g}(\bm\theta)),
\label{Eq:22.37}
\end{align}
где $\hat{\gamma}=[\hat{\gamma}_1, \dots, \hat{\gamma}_{T-1}]'$,
\begin{align}
\hat{\gamma}_j=[N(T-j)]^{-1} \sum^T_{t=j+1} \sum^N_{i=1} (y_{it}-\bar{y}_t)(y_{i,t-j}-\bar{y}_{t-j}),
\label{Eq:22.38}
\end{align}
и $\bar{y}_{t-j}=N^{-1} \sum_i y_{i,t-j}$. Взвешивающая матрица $\mathbf{W}_N$ и другие подробности  оценивания методом минимального расстояния содержатся в разделе 6.7. Ограничения модели могут быть проверены с помощью тестовой статистики $\chi^2$, данной в разделе 6.7. Таким образом мы наложили  ограничение слабой стационарности. В более общем виде
$\gamma_{tj} \neq \gamma_{sj}$ для $t \neq s$, где  $\gamma_{tj}=\mathrm{Cov}[y_{it},y_{i,t-1}]$. Тогда $\bm\gamma$ содержит $T(T+1)/2$ компонент $\gamma_{tj}, t= j+1, \dots, T$ и $j=0, \dots, T-1$. Предположение о стационарности  можно протестировать. Более того, регрессоры могут быть включены заменой $y_{it}$ на остатки $y_{it}-\x'_{it} \be$.

Абоуд и Кард (1989) одними из первых применили этот подход к совместному моделированию заработной платы и количества часов работы. Альтоньи и Сегал (1996) продемонстрировали, что оптимальная оценка минимального расстояния может быть достаточно смещенной в ограниченных выборках (см. раздел 6.3.5). С помощью этой оценки в основном моделируются заработные платы; см. недавний пример Бейкер и Солон (2003).

Подход минимального расстояния, MD,  хорошо подходит для оценивания ковариационных структур. Набор панельных данных может быть большим, но после оценивания ковариаций всё оценивание сводится к  минимизации \ref{Eq:22.37}. Другие подходы к оцениванию возможны. В частности см. МаКарди (1982b), где используются модели типа Бокса-Дженкинса для панельных данных. 

\subsection{Нестационарные панели}

В литературе, посвященной анализу единичных корней и нестационарности панельных данных, делается акцент на панели, в которых $N$ и $T$ велики. Одну из первых ключевых работ, посвященных \textbf{тестам на единичные корни}, написали Левин и Лин (1992) и позже опубликовали Левин, Лин и Чу (2002); Песаран и Смит (1999) одними из первых написали работу о \textbf{коинтеграции}. Филлипс и Мун (1999)  и Педрони (2004) приводят общую теорию статистических выводов в случае нестационарных панельных данных. Этот анализ довольно прост, в нем используется \textbf{последовательное взятие пределов}, где сначала фиксируется  $N$, а $T \rightarrow \infty$. После этого $N \rightarrow \infty$. Более робастный подход использует \textbf{совместные пределы}, когда  $T \rightarrow \infty$ и $ N \rightarrow \infty$ одновременно. Среди недавних обзоров литературы следует отметить работы Филлипса и Муна (2000) и Бальтаджи (2001, глава 12). 

В меньшей степени были рассмотрены нестационарные данные в \textbf{коротких панелях}. Харрис и Тцавалис (1999) применяют тест на единичные корни Левина и Лина (1992) в коротких панелях. Пусть $\hat{\gamma}$ обозначает оценку within $\gamma$ в модели с фиксированными эффектами вида AR(1) $y_{it}=\alpha_i+\gamma y_{i,t-1} + \e_{it}$, где $\e_{it} \thicksim \mathcal N [0, \sigma^2]$. Мы проверяем нулевую гипотезу о единичном корне, т.е. $\gamma=1$ и отсутствие свободного члена $\alpha_i=0$, которая соответствует случаю 2 во временных рядах у Гамильтона (1994, c. 490). При нулевой гипотезе статистика тестирования единичного корня

\begin{align}
\frac{\sqrt{N}(\hat{\gamma}-1+3/(T+1))}
{[3(17T^2-20T+17)]/[5(T-1)(T+1)^3]} \rightarrow \mathcal N [0,1],
\nonumber
\end{align}
если $N \rightarrow \infty$ для фиксированного $T$. При больших отрицательных значениях статистики гипотеза о наличии единичного корня отвергается. Левин и Лин (1992) предложили дополнительные тесты, как, например, для моделей с индивидуальными временными трендами.

Байндер, Хсяо и Песаран (2003) изучают оценивание динамических моделей панельных данных с фиксированными эффектами с наличием единичных корней и коинтеграцией. При наличии единичных корней оценка Ареллано-Бонда несостоятельна, хотя видоизмененные оценки Ана Шмидта (1995) и др., которые обсуждаются в разделе 22.5.3, дают состоятельные оценки. Байндер и др. (2003) предлагают оценки квази-ММП, которые более предпочтительны для коротких панелей при наличии единичных корней.

\section{Оценка разность разностей}

Литература, представленная в главе 25, фокусируется на измерении \textbf{эффекта воздействия}. В самом простом случае влияние или предельный эффект одного бинарного регрессора равен единице, если происходит какое-то изменение внешних условий, и равен нулю иначе.
Например, мы можем быть заинтересованы в измерении эффекта изменения политики (воздействие бинарного типа) на заработную плату. Изменение политики состоит в изменении налоговой ставки или благосостояния или доступа отдельных индивидуумов к обучению.

В этом разделе мы связываем один из методов главы 25 с методами анализа панельных данных. В частности, эффект воздействия может измеряться с помощью стандартных методов анализа панельных данных, если данные доступны до и после данного изменения, и если не все индивидуумы попадают под данное влияние. Тогда оценка в первых разностях для модели с фиксированными эффектами сводится к простой оценке, называемой оценкой <<разность разностей>>, представленной в разделе 3.4.2, и проанализированной в разделе 25.5. 
Преимущество последней оценки состоит в том, что она может применяться даже тогда, когда доступны повторные данные пространственного типа, а не панельные данные. Однако эта оценка зависит от предположений модели, которые зачастую явно не определены. Изложение этого раздела следует подходу Бланделла и МаКарди (2000).

\subsection{Фиксированные эффекты и бинарное воздействие}

Пусть бинарный регрессор, интересующий нас, будет иметь вид 
\begin{align}
D_{it}=
\begin{cases}
 1, \text{если индивид $i$ подвержен внешнему влиянию в момент $t$}, \\
0, \text{в противном случае}.
\end{cases}
\label{Eq:22.39}
\end{align}
Предположим, что модель содержит фиксированные эффекты для $y_{it}$ с 
\begin{align}
y_{it}=\phi D_{it} + \delta_t + \alpha_i + \e_{it},
\label{Eq:22.40}
\end{align}
где $\delta_t$ --- это временной фиксированный эффект, а $\alpha_i$ --- индивидуальный фиксированный эффект. Как замечено в разделе 21.2.1, это эквивалентно регрессии $y_{it}$ на $D_{it}$ и набор временных фиктивных переменных, только с индивидуальными фиксированными эффектами. Для простоты пусть в регрессии не содержится других регрессоров.

Индивидуальные эффекты $\alpha_i$  можно уничтожить взятием первых разностей. Тогда
\begin{align}
\Delta y_{it}=\phi \Delta D_{it} + (\delta_t - \delta_{t-1}) + \Delta \e_{it}.
\label{Eq:22.41}
\end{align}
$\phi$, эффект воздействия, может быть состоятельно оценен с помощью МНК сквозной регрессии $\Delta y_{it}$ на $\Delta D_{it}$ и временные дамми.

\subsection{Разность разностей}

Сейчас рассмотрим случай только двух периодов. Более того, предположим, что воздействие происходит в период 2, т.е. в период 1 $D_{i1}=0$ для всех индивидуумов и в периоде 2 $D_{i2}=1$ для всех, на кого было оказано воздействие (опытная группа), и $D_{i2}=0$ для тех, кто не был подвержен влиянию (контрольная группа). Тогда нижний индекс $t$ в \ref{Eq:21.41} можно отбросить  и 
 \begin{align}
\Delta y_{i}=\phi  D_{i} + \delta+ v_{i}.
\label{Eq:22.42}
\end{align}
где $D_i$ --- это бинарная переменная, обозначающая, был ли индивидуум подвержен воздействию.

Эффект воздействия может быть оценен МНК регрессией $\Delta y$ на свободный член и бинарный регрессор $D$. Пусть $\Delta\bar{y}^{tr}$ обозначает выборочное среднее $\Delta y_i$ для тех, кто был подвержен влиянию $(D_i=1)$, и  $\Delta \bar{y}^{nt}$ обозначает выборочное среднее $\Delta y_i$ для тех, кто не был подвержен влиянию $(D_i=0)$. Тогда МНК оценка будет иметь вид
 \begin{align}
\hat{\phi}=\Delta \bar{y}^{tr} - \Delta \bar{y}^{nt}.
\label{Eq:22.43}
\end{align}
Эта оценка называется \textbf{оценкой разность разностей, differences-in-differences, DID}, так как оцениваются разности для подверженных и не подверженных воздействию, а затем берется разность этих разностей.

Эта оценка привлекательна своей простотой. К тому же, она может быть применена не только к панельным данным, но и для данных пространственного типа, значения которых доступны для двух периодов во времени. Во втором периоде считаются средние $\bar{y}^{tr}_2$ и $\bar{y}^{nt}_2$ для двух групп (подверженных и не подверженных воздействию). Аналогично вычисляются средние для первого периода $\bar{y}^{tr}_1$ и $\bar{y}^{nt}_1$. Предполагается, что в первом периоде можно идентифицировать, подвержено ли индивидуальное наблюдение воздействию.  Это легко сделать, когда, к примеру, влияние оказывается только на женщин, и доступны данные о поле. Тогда вычисляется
 \begin{align}
\hat{\phi}=(\bar{y}^{tr}_2- \bar{y}^{tr}_1) - ( \bar{y}^{nt}_2 - \bar{y}^{nt}_1).
\label{Eq:22.44}
\end{align}

Например, если средняя ежегодная заработная плата для группы, которая будет подвержена воздействию, равна 10,000 до изменений и 13,000 после, тогда $\bar{y}^{tr}_2- \bar{y}^{tr}_1=3,000$. Аналогично, если средняя ежегодная заработная плата для группы, которая не будет подвержена воздействию, равна 15,000 до изменений и 17,000 после, то  $\bar{y}^{nt}_2 - \bar{y}^{nt}_1=2,000$. DID оценка эффекта $\hat{\phi}$ будет равна $3,000-2,000=1,000$.

\subsection{Предположения для DID оценки}

Предыдущая формулировка DID оценки выявляет необходимые предположения для состоятельного оценивания $\phi$.

Во-первых, предполагается, что временные эффекты $\delta_t$ характерны как для индивидуумов, подверженных воздействию, так и для тех, на кого изменения не будут оказывать влияния. Например, временной тренд может различаться в зависимости от пола. В этом случае идентификация $\phi$  будет проблематичной, если влияние зависит от пола. Предположение об общем тренде необходимо как в случае панельных, так и в случае пространственных данных.

Во-вторых, если используются данные пространственного типа, то структуры обеих групп постоянны до и после воздействия. В случае с панельными данными после взятия разностей уничтожаются фиксированные эффекты $\alpha_i$. С повторяющимися данными пространственного типа при первоначальной модели \ref{Eq:22.40} $\bar{y}^{tr}_t=\phi+\delta_t + \bar{\alpha}^{tr}_t+\bar{\e}^{tr}_t$ и $\bar{y}^{nt}_t=\delta_t+\bar{\alpha}^{nt}_t+\bar{\e}^{nt}_t$. Если изменение происходит только во втором периоде, то 
\begin{align}
\phi=(\bar{y}^{tr}_2- \bar{y}^{tr}_1) - ( \bar{y}^{nt}_2 - \bar{y}^{nt}_1)+ (\bar{\alpha}^{tr}_2- \bar{\alpha}^{tr}_1) - ( \bar{\alpha}^{nt}_2 - \bar{\alpha}^{nt}_1) + v,
\nonumber
\end{align}
где $v=(\e^{nt}_2 - \bar{\e}^{nt}_1)-(\bar{\e}^{nt}_2-\bar{\e}^{nt}_1)$. Оценка $\hat{\phi}$ будет состоятельной в \ref{Eq:22.44}, если $\mathrm{plim} (\bar{\alpha}^{tr}_2-\bar{\alpha}^{tr}_1)=0$ и $\mathrm{plim} (\bar{\alpha}^{nt}_2-\bar{\alpha}^{nt}_1)=0$. Это будет выполняться, если индивиды, подверженные воздействию, выбраны случайно. Однако это зачастую не выполняется на практике.


\subsection{Другие более сложные модели}

На практике используются более сложные модели. Очевидным является включение в модели других регрессоров помимо временных дамми и индикатора влияния. Индивидуальные эффекты могут по меньшей мере различаться в среднем по группам. Общий алгоритм оценивания состоит в оценке уравнения
\begin{align}
y_{igt}=\phi D_{igt} + \delta_t + \alpha_i + \e_{it},
\nonumber
\end{align}
где $g$ обозначает  $g$-ю группу.

В классическом примере DID оценивания, Кард (1990) изучал эффект внезапного приток иммигрантов из Кубы на безработицу среди рабочих с низкой заработной платой в Маями. Этот пример также рассматривают Ангрист и Крюгер (1999). Этей и Имбенс (2002) приводят обобщения для нелинейных моделей.

\section{Повторяющиеся пространственные данные и псевдо-панели}

Ключевое преимущество панельных данных возникает благодаря возможности наблюдения субъектов в разные периоды времени. Благодаря этому можно  учитывать ненаблюдаемую индивидуальную гетерогенность, разницу в первоначальных условиях, динамическую зависимость исходов. Во многих случаях, однако, подлинные панельные данные недоступны.

\subsection{Повторяющиеся пространственные данные}

Мы рассмотрим анализ, когда имеются данные для нескольких \textbf{повторяющихся пространственных выборок}. Данные взяты из ответов на серию независимых исследований, где независимость означает, что каждый субъект появляется только в одном исследовании. Пример такого исследования --- Исследование расходов английских домохозяйств, U.K. Family Expenditure Survey, в котором ежегодно собирается информация о расходах домохозяйств, но каждый год в исследовании принимают участие разные семьи. Также, если доступна только очень короткая панель (например, $T=2$), то  предпочтительными являются данные повторяющихся пространственных выборок, если они генерируют более большую и богатую выборку.

Для модели \textbf{со случайными эффектами} данные повторяющихся пространственных выборок не составляют особых трудностей. Необходимо просто оценить сквозную регрессию $y_{it}$  на $\x_{it}$ (см. раздел 21.5). Получение статистических выводов довольно простое, так как нужна только коррекция на гетероскедастичность вследствие того, что ошибки независимы по $i$ и $t$.

С фиксированными эффектами, однако, сквозная регрессия приводит к несостоятельным оценкам параметров. Более того, альтернативные методы, такие как оценивание within или в первых разностях недоступны, если индивидуальные наблюдения наблюдаются лишь однажды. В данном разделе будут использоваться данные пространственного типа для конструирования \textbf{псевдо-панелей} или \textbf{синтетических панельных данных}, которые имеют некоторые преимущества настоящих панельных данных (самое существенное из которых --- возможность учета фиксированных эффектов). Специальный случай --- это DID оценивание, представленное в разделе 22.6.

\subsection{Псевдо-панели}

Браунинг, Дитон, и Айриш (1985), а также Дитон (1985) в своих эмпирических исследованиях, основанных на U.K. Family Expenditure Survey, рассматривали методы для анализа повторяющихся пространственных данных. Их предложением было конвертировать индивидуальные данные в \textbf{когорты}. Хотя индивидуальные расходы домохозяйств не могут регулярно записываться на протяжении некоторого времени, то для когорт это возможно.

\textbf{Когорта} --- это <<группа постоянных членов, индивидуумы которой могут идентифицироваться по мере того, как они <<показываются>> в исследованиях>> (Дитон, 1985, c.109). Примером может служит возрастная когорта, такая как мужчины с годом рождения от 1965 до 1970. Для больших выборок, последующие исследования будут генерировать случайные выборки членов каждой когорты.

Временные ряды выборочных средних когорт могут формировать основу регрессионных моделей. Ключевой вопрос: могут ли служить синтетические панели, основанные на когортных данных, заменой настоящим панелям. Для таких моделей процедуры получения статистических выводов рассматриваются в рамках темы о повторяющихся пространственных данных. Здесь мы фокусируемся на статических моделях псевдо панелей. Колладо (1997) и Гирма (2000) тоже рассматривают динамический случай;

Отправной точкой является статическая линейная регрессия с индивидуальными эффектами $\alpha_i$, основанная на $T$ последующих пространственных выборках,
 \begin{align}
& y_{it}=\alpha_i+\x'_{it} \be + u_{it},
& t=1, \dots, T.
\label{Eq:22.45}
\end{align}
Предполагается, что объясняющие переменные строго экзогенны по отношению к исследуемым параметрам, $\be$, $\E[\x'_{it} u_{is}]=\mathbf 0$ для любых $t$ и $s$. Для простоты мы предполагаем, что в каждой кросс секции доступно $N$ наблюдений. Каждое индивидуальное наблюдение присутствует только в одном периоде, поэтому индивидуальные эффекты $\alpha_i$ не могут быть уничтожены с помощью взятия разностей по индивидуальным данным.

Пусть $g$  --- это случайная переменная, которая определяет членство в когорте для каждого $i$.  $i$ принадлежит кластеру $c$ тогда и только тогда, когда $g_i$ относится к набору $I_c$. Предположим, что имеется  $C$ когорт, и $c$ --- это индекс когорты, $c=1, \dots, C$. Взяв условное математическое ожидание, получаем
 \begin{align}
\E[y_{it}|g_i I_c]=\E[\alpha_i|g_i I_c]+\E[\x'_{it}|g_i I_c]\be+\E[u_{it}|g_i I_c].
\label{Eq:22.46}
\end{align}
Мы получаем когортную версия модели \ref{Eq:22.45}
 \begin{align}
y*_{ct}=\alpha*_c+\x*_{ct}'\be +u*_{ct},
\label{Eq:22.47}
\end{align}
где звездочка обозначает ненаблюдаемые средние по индивидам когортам. Например, $y_{ct}^*=\E[y_{it}|g_i \in I_c]$.

Параметр $\alpha^*_c=\E[\alpha_i|g_i \in I_c]$ является \textbf{фиксированным эффектом когорты}. Важное предположение, которое делается в случае фиксированных эффектов, состоит в том, что генеральная совокупность стационарна. Поэтому можно предполагать, что $\alpha^*_c$ постоянна во времени. Это предположение аналогично тому, что необходимо для состоятельности DID оценки, описанному в конце раздела 22.6.3. В условиях обычного предположения о слабой экзогенности $\E[u^*_{ct}|\x^*_{ct}]=0$. Однако ненаблюдаемый фиксированный эффект $\alpha^*_c$ будет коррелирован с $\x^*_{ct}$, если $\alpha_i$ коррелировано с $\x_{it}$ в первоначальной версии модели \ref{Eq:22.45}. Для оценивания необходимо учитывать фиксированные эффекты.

На практике средние по индивидам когорты ненаблюдаемы, вместо этого мы работаем со \textbf{средними когорт по времени} $\bar{y}_{ct}$ и $\bar{\x}_{c}$. Тогда регрессия имеет вид
 \begin{align}
& \bar{y}_{ct}=\bar{\alpha}_c+\bar{\x}'_{ct}\be+\bar{u}_{ct}
& c=1, \dots, C, &
& t=1, \dots, T.
\label{Eq:22.48}
\end{align}

Это шаг представляет нам дополнительный источник ошибки, так как оценки $\bar{y}_{ct}$ и $\bar{\x}_{ct}$ для средних по индивидам когорты загрязнены  ошибками, т.е.
 \begin{align}
& \bar{y}_{ct}=y^*_{ct}+\theta_{ct},
\label{Eq:22.49} \\
& \bar{\x}_{ct}=\x^*_{ct}+v_{ct}. \nonumber
\end{align}

Если \textbf{ошибка измерения} очень мала благодаря тому, что количество наблюдений в когорте для одного временного периода ($N_{ct}$) достаточно велико, то $\bar{y}_{ct} \simeq y^*_{ct}$ и $\bar{\x}_{ct}=\x^*_{ct}$  и ошибка измерения может быть проигнорирована. Состоятельная оценка $\be$ может быть получена с помощью оценивания within уравнения \ref{Eq:22.48}, т.е. МНК регрессии $(\bar{y}_{ct}-\bar{y}_c)$ на $(\bar{\x}_{ct}-\bar{\x}_c)$, где $\bar{y}_c=T^{-1} \sum_t \bar{y}_{ct}$ и $\bar{\x}_c=T^{-1}\sum_t \bar{\x}_{ct}$.

К сожалению, ошибка измерения часто слишком велика, чтобы ее игнорировать. Тогда within оценивание уравнения \ref{Eq:22.48}, или даже МНК оценивание \ref{Eq:22.48}, когда $\bar{\alpha}_c$  --- это случайный эффект, приводит к несостоятельной оценке $\be$. Вместо этого необходимо использовать оценки для случая ошибок измерения в переменных. Такие оценки могут здесь применяться, так как индивидуальные данные дают необходимые оценки моментов ошибок измерения, см. раздел 26.3.3.


\subsection{Оценки ошибок измерения для псевдо-панелей}

Классический способ решения проблемы ошибок измерения --- это использование повторных наблюдений для оценки ковариационной матрицы ошибок измерения, а затем использование этих оценок для <<корректирования>>  выборочных моментов переменных, содержащих ошибки, до применения процедуры МНК (см. раздел 26.3.4). Дитон (1985) предложил использование этого метода в текущей задаче.

Предположим, что индивидуальные наблюдения удовлетворяют уравнениям
 \begin{align}
y_{it}=y^*_{ct}+\e_{it} \nonumber \\
\x_{it}=\x^*_{ct}+\bm\eta_{it} \nonumber.
\end{align}
Cистема похожа на ту, что в разделе 26.2.1, за исключением того, что здесь также присутствует ошибка измерения в зависимой переменной. Предположим, что для любого индивидуума в данной когорте $c$ 
 \begin{align}
\begin{bmatrix}
\e_{it} \\
\bm\eta_{it}
\end{bmatrix}
\thicksim
\mathrm{iid}
\left[
\begin{bmatrix}
0, \\
\mathbf 0
\end{bmatrix}, 
\begin{bmatrix}
\sigma^2_0 & \sigma'_{01} \\
\bm\sigma_{01} & \bm\sum
\end{bmatrix}
\right].
\nonumber 
\end{align}
Выборочные оценки $(\sum, \bm\sigma_{01})$, обозначенные как $(\hat{\sum}, \bm\sigma_{01})$, могут быть получены при данных $(\bar{y}_{ct}, \bar{\x}_{ct})$ из индивидуальных данных. Определим $\mathbf d_c$ как столбец вектора фиктивных переменных размерности $C \times 1$, соответствующий фиксированным эффектам $\alpha^*_c$  (см. раздел 21.2.1). Этот вектор-регрессор, безусловно, не содержит ошибку измерения. Тогда, если $T$ достаточно большое, и существуют необходимые обратные матрицы, то регрессия
 \begin{align}
\begin{bmatrix}
\hat{\bar{\alpha}}_{ct} \\
\hat{\bar{\be}}_{ct}
\end{bmatrix}
=\left(
\sum^C_{c=1}
\sum^T_{t=1}
\begin{bmatrix}
\mathbf d'_c \mathbf d_c & \mathbf d'_c \bar{\x}_{ct} \\
\bar{\x}'_{ct} \mathbf d_c & \bar{\x}'_{ct} \bar{\x}_{ct}-\hat{\sum}
\end{bmatrix}
\right)^{-1}
\left[
\sum^C_{c=1}
\sum^T_{t=1}
\begin{pmatrix}
\mathbf d'_c \bar{y}_{ct} \\
\bar{\x}'_{ct} \mathbf d_c - \hat{\bm\sigma}_{01}
\end{pmatrix}
\right]
\label{Eq:22.50}
\end{align}
будет давать состоятельные оценки при $CT \rightarrow \infty$. Эта оценка та же, что дана в разделе 26.3.4, только с поправкой, так как $\bar{y}_{ct}$ тоже измеряется с ошибкой, и с упрощением, так как только подмножество регрессоров, $\hat{\bar{\x}}_{ct}$, измеряется с ошибкой. Вербик и Нейман (1992) приводят более подробное обсуждение выборочных свойств, и Дитон (1985) обсуждает оценивание дисперсии. См. также Вербик (1995).

Предыдущая оценка учитывает фиксированные эффекты путём оценивания модели с фиктивными переменными, делая поправку на ошибку измерения посредством использования повторных данных и оценки раздела 26.3.4.

Колладо (1997) рассматривал альтернативный подход для уничтожения эффектов когорт взятием первых разностей, а затем учетом ошибки измерения с помощью IV оценивания. Эта альтернативная стратегия идентификации ошибки измерения обсуждается в разделе 26.3.2.

Заменяя \ref{Eq:22.49} на \ref{Eq:22.47}, получаем 
 \begin{align}
\bar{y}_{ct}-\theta_{ct}=\alpha^*_c+(\bar{\x}'_{ct}-\mathbf v'-{ct}) \be+u*_{ct}, \nonumber \\
\bar{y}_{ct}=\alpha^*_c+\bar{\x}'_{ct}\be+w_{ct}
\nonumber,
\end{align}
где ошибка $w_{ct}=u^*_{ct}-\mathbf v'_{ct}\be+\theta_{ct}$. При взятии первых разностей уничтожаются $\alpha^*_c$, что приводит к 
 \begin{align}
& \Delta \bar{y}_{ct}=\Delta \bar{\x}'_{ct}\be+\Delta w_{ct},
&t=2, \dots, T.
\label{Eq:22.51}
\end{align}
Сейчас из-за ошибки измерения объясняющие переменные $\Delta \bar{\x}'_{ct}$ будут коррелированы с $\Delta w_{ct}$, и поэтому МНК будет давать несостоятельные оценки. Состоятельные оценки могут быть получены с помощью IV оценивания, основанного на лагах экзогенных переменных, т.е. $\bar{\x}'_{c,t-1}$. Этот подход легко обобщить на модели с  лаговыми зависимыми переменными. Подробности см. в Колладо (1997).

\section{Смешанные линейные модели}

В модели, называемой эконометристами моделью со случайными эффектами, только свободный член является случайным. В более общих моделях со случайными эффектами, которые широко используются в прикладной статистике, случайными могут быть и коэффициенты наклона. В этом разделе мы представляем смешанные линейные модели. Они также называются моделями смешанных эффектов, иерархическими или многоуровневыми линейными моделями (см. главу 24), моделями со случайными коэффициентами, и моделями с составной дисперсией.

Эти модели применяются в таких случаях, когда МНК оценка сквозной регрессии все еще состоятельна. В частности, нет фиксированных эффектов.  Предпосылки смешанных линейных моделей позволяют использовать доступный ОМНК для получения более эффективных оценок.

 \subsection{Смешанные линейные модели}

\textbf{Смешанная линейная модель} специфицирована следующим образом
 \begin{align}
y_{it}=\mathbf z'_{it} \be + \mathbf w'_{it} \bm\alpha_i + \e_{it},
\label{Eq:22.52}
\end{align}
где регрессоры $\mathbf z_{it}$ включают свободный член, $\mathbf w_{it}$  --- это вектор наблюдаемых характеристик, $\bm\alpha_i$  --- это случайный вектор с нулевым математическим ожиданием, и $\e_{it}$ обозначает ошибку. Эта модель называется \textbf{смешанной моделью}, так как в ней есть \textbf{фиксированные параметры} $\be$ и  \textbf{случайные параметры} с нулевым средним  или \textbf{случайные эффекты} $\alpha_i$.

Модель со случайным свободным членом $y_{it}=\mathbf z'_{it}\be+\alpha_i+\e_{it}$ --- это частный случай модели \ref{Eq:22.52}  с $\mathbf w'_{it} \bm\alpha_i=\alpha_i$.

Другой частный случай \ref{Eq:22.52} --- это \textbf{модель со случайными коэффициентами} или \textbf{модель со случайными параметрами}. В терминах регрессионной модели мы предполагаем, что
 \begin{align}
y_{it}=\mathbf z'_{it} \be + \e_{it}.
\nonumber
\end{align}
Это обычная линейная регрессия за исключением того, что компоненты вектора параметров регрессии сейчас различаются 
 \begin{align}
\be_i=\be+\bm\alpha_i
\nonumber
\end{align}
где $\bm\alpha_i$ --- случайный вектор с нулевым ожиданием. Подставляя это выражение в предыдущее, получаем $y_{it}=\mathbf z'_{it} \be + \mathbf z'_{it} \bm\alpha_i+\e_{it}$, что дает нам \ref{Eq:22.52} с $\mathbf w_{it}=\mathbf z_{it}$.

Множество моделей находятся где-то между моделью со случайным свободным членом и моделью со случайным коэффициентом, где $\mathbf w_{it}$ является подмножеством $\mathbf z_{it}$. В частности,  стандартные смешанные и случайные \textbf{ANOVA} модели также являются частным случаем, где $k$-й элемент вектора $\mathbf w_{it}$ равен нулю или единице, в соответствии с различными возможными моделями кластеризации данных. Например, один из элементов в $\mathbf z_{it}$ может быть переменная-индикатор расы или пола. Тогда условное среднее $y_{it}$ изменяется в зависимости от расы и пола. Может случиться, что условная дисперсия $y_{it}$ тоже меняется в зависимости от расы и пола и может быть учтена  в $\mathbf w_{it}$. Смешанная модель выросла из моделей ANOVA. \textbf{Иерархичечские линейные модели} или \textbf{многоуровневые линейные модели} (см. раздел 24.6.2) тоже могут быть представлены как частный случай модели \ref{Eq:22.52}. 

 \subsection{Оценивание}

Цель --- оценить фиксированные параметры регрессии $\be$, дисперсию и ковариации распределений для $\bm\alpha_i$ и $\e_{it}$. Одна из первых попыток оценивания этой модели была осуществлена в байесовском контексте  (Линдлей и Смит, 1972). Простым примером их подхода является модель со случайными коэффициентами с $y_{it} \thicksim \mathcal N [\mathbf z'_{it}\be_i, \sigma^2]$, где $\be_i \thicksim \mathcal N [\bm\gamma, \bm\Gamma]$. Подробнее \textbf{байесовский анализ} линейной модели панельных данных см., например,  у Купа (2003).

Здесь мы будем следовать \textbf{классическому подходу}, основанному на работе Харвилла (1977), который ссылается на более раннюю литературу. Смешанная модель \ref{Eq:22.52}  может быть разделена на детерминистическую компоненту $\x'_{it} \be$ и случайную компоненту $\mathbf w'_{it} \bm\alpha_i+\e_{it}$. Стохастические предположения включают в себя также предположение, что регрессоры $\x'_{it}$ независимы от случайных компонент $\bm\alpha_i$ и $\e_{it}$  с нулевым ожиданием. Поэтому сквозная МНК регрессия $y_{it}$ на $\x_{it}$  дает состоятельные оценки $\be$. Мы будем придерживаться рамок раздела 21.5, где для оценки можно  использовать доступный ОМНК, так как на ковариационную матрицу ошибки $\mathbf w'_{it} \bm\alpha_i+\e_{it}$ была наложена структура. В этом разделе мы представляем доступную ОМНК оценку наряду с двумя различными методами для оценки дисперсий и ковариаций $\bm\alpha$ и $\e_{it}$. Мы также рассмотрим предсказание случайной компоненты $\bm\alpha_i$.

Совместим наблюдения по времени для данного индивидуума обычным способом, так что \ref{Eq:22.52} превратится в
\begin{align}
\mathbf y_i=\mathbf Z_i \be+(\mathbf W_i \bm\alpha_i+\bm\e_i).
\label{Eq:22.53}
\end{align}
Обычно предполагается, что $\bm\alpha_i$ и $\bm\e_i$  независимы по $i$  и между друг другом с $\bm\alpha_i \thicksim [\mathbf 0, \sum_{\alpha}]$ и $\bm\e_i \thicksim [ \mathbf 0, \sum_{\bm\e}]$, так что ошибка 
 \begin{align}
\mathbf W_i \bm\alpha_i+\bm\e_i \thicksim [\mathbf 0, \bm\Omega_i = \mathbf W_i \bm\Sigma_{\alpha} \mathbf W'_i + \bm\Sigma_{\bm\e}].
\nonumber
\end{align}
Тогда \textbf{доступная ОМНК оценка} будет иметь вид
\begin{align}
 \hat{\be}_{FGLS}=\left[ \sum^N_{i=1} \mathbf Z'_i \hat{\bm\Omega}^{-1}_i \mathbf Z_i 
\right] ^{-1}
\sum^N_{i=1} \mathbf Z'_i \hat{\bm\Omega}^{-1}_i \mathbf y_i,
\label{Eq:22.54}
\end{align}
где $\hat{\bm\Omega}_i$ является состоятельной оценкой для $\bm\Omega_i$.

Осуществление этого подхода требует состоятельного оценивания $\bm\Omega_i$. Это уже обсуждалось в разделе 21.7 для более простого случая случайного свободного члена. В распоряжении имелось несколько различных способов состоятельно оценить компоненты дисперсии $\sigma^2_{\alpha}$ и $\sigma^2_{\e}$ с такими возможными осложнениями, как смещение или возможность получения отрицательных оценок. Схожие проблемы появляются здесь при оценивании $\bm\Sigma_{\bm\alpha}$ и $\bm\Sigma_{\bm\epsilon}$.

Мы представляем две оценки, основанные на дополнительном предположении о нормальном распределении случайных компонент. Более общая модель имеет вид 
\begin{align}
\mathbf y= \mathbf Z \be + (\mathbf W \bm\alpha + \e),
\label{Eq:22.55}
\end{align}
который может быть получен, например, соответствующим совмещением наблюдений модели \ref{Eq:22.53}. Предполагается, что $\bm\alpha \thicksim \mathcal N [\mathbf 0, \mathbf G]$ и  $\e \thicksim \mathcal N[\mathbf 0, \mathbf R]$, где в текущем случае $\mathbf G$ и $\mathbf R$  --- функции $\bm\Sigma_{\bm\alpha}$ и $\bm\Sigma_{\bm\e}$. \textbf{Доступная ОМНК оценка} для смешанной модели
\begin{align}
\hat{\be}_{FGLS}=[\mathbf Z' \hat{\mathbf V}^{-1} \mathbf Z]^{-1}
\mathbf Z' \hat{\mathbf V}^{-1} \mathbf y,
\nonumber
\end{align}
где $\hat{\mathbf V}$ --- это состоятельная оценка для $\mathbf V=\mathrm V[\mathbf W \bm\alpha +\bm\e]=\mathbf W \mathbf G \mathbf W' + \mathbf R$. См. Свами (1970).

Очевидный метод для получения $\hat{\mathbf V}$ --- метод максимального правдоподобия. Функция логарифма максимального правдоподобия основана на многомерном нормальном, после замены вектора $\be$ на его ОМНК оценку $[\mathbf Z' \mathbf V^{-1} \mathbf Z]^{-1} \mathbf Z' \mathbf V^{-1} \mathbf y$,
\begin{align}
\mathrm{ln} L(\mathbf G, \mathbf R)=-\frac{1}{2} \mathrm{ln}|\mathbf V| - \frac{NT}{2} \mathrm{ln} \mathbf r' \mathbf V^{-1} \mathbf r - \frac{NT}{2} \left[ 1+ \mathrm{ln}(\frac{2\pi}{NT})\right],
\nonumber
\end{align}
где $\mathbf r= \mathbf y - \mathbf Z[\mathbf Z'\mathbf V^{-1} \mathbf Z]^{-1} \mathbf Z' \mathbf V^{-1} \mathbf y$ и $|\mathbf V|$ обозначает определитель $\mathbf V$. Максимизация по параметрам в $\mathbf G$ и $\mathbf R$ дает $\hat{\mathbf V}=\mathbf W \hat{\mathbf G} \mathbf W' + \hat{\mathbf R}$.

Недостаток оценок ММП компонент дисперсий заключается в том, что они смещены в малых выборках. Например, для линейных пространственных регрессий с гомоскедастичными ошибками оценка ММП $\hat{\sigma}^2=N^{-1} \sum_i \hat{u}^2_i$ смещена, и поэтому делить лучше на $(N-K)$. Для модели \ref{Eq:22.53} коррекции на степени свободы обеспечиваются оценкой \textbf{ограниченной функцией правдоподобия}, которая максимизирует
\begin{align}
\mathrm{ln} L_R(\mathbf G, \mathbf R)=-\frac{1}{2} \mathrm{ln}|\mathbf V| - \frac{NT-p}{2} \mathrm{ln} \mathbf r' \mathbf V^{-1} \mathbf r - \frac{NT-p}{2} \left[ 1+ \mathrm{ln}(\frac{2\pi}{NT-p})\right]
-\frac{1}{2}\mathrm{ln}| \mathbf Z' \mathbf V^{-1} \mathbf Z|,
\nonumber
\end{align}
где  $p$  --- ранг $\mathbf Z$. Преимущества использования  $\mathrm{ln} L_R(\mathbf G, \mathbf R)$ см. у Харвилла (1977).

В качестве примера смешанной линейной модели рассмотрим регрессию ln(hours)-ln(wage) раздела 21.3, в которой и свободный член и параметры наклона могут быть случайными. Тогда модель со случайными коэффициентами будет выглядеть как lnhrs=7.734-0.02lnwg, стандартная ошибка коэффициента наклона равна 0.046 (по умолчанию) или 0.020 (панельный бутстреп). Коэффициент наклона отличается от оценок, данных в таблице 21.2.

 \subsection{Прогнозирование}

Мы можем желать \textbf{спрогнозировать} случайные параметры $\bm\alpha$  в дополнение к фиксированным параметрам $\be$ и ковариациям.

Совместные  уравнения для $\hat{\be}$ и $\hat{\bm\alpha}$ при условии, что   оценки $\hat{\be}$ и $\hat{\bm\alpha}$ состоятельны, могут быть записаны как
\begin{align}
\begin{bmatrix}
\mathbf Z' \hat{\mathbf R}^{-1} \mathbf Z & \mathbf Z' \hat{\mathbf R}^{-1} \mathbf W \\
\mathbf W' \hat{\mathbf R}^{-1} & \mathbf W' \hat{\mathbf R}^{-1} \mathbf W + \hat{\mathbf G}^{-1}
\end{bmatrix}
\begin{bmatrix}
\hat{\be} \\
\hat{\bm\alpha}
\end{bmatrix}
=
\begin{bmatrix}
\mathbf Z' \hat{\mathbf R}^{-1} \mathbf y \\
\mathbf W' \hat{\mathbf R}^{-1} \mathbf y
\end{bmatrix}.
\nonumber
\end{align}
Решая уравнение относительно $\hat{\be}$, получаем прежнюю оценку $\hat{\be}_{FGLS}$, в то время как
\begin{align}
\hat{\bm\alpha}=\hat{\mathbf G} \mathbf W' \hat{\mathbf V}^{-1} (\mathbf y - \mathbf Z' \hat{be}).
\nonumber
\end{align}
В случае независимости по $i$ $\hat{\bm\alpha}_i=\hat{\sum_{\bm\alpha}}\mathbf W'_i \hat{\mathbf V}_i^{-1} (\mathbf y_i - \mathbf Z'_i \hat{\be})$. Это \textbf{лучший линейный несмещенный прогноз} при известных ковариационных матрицах.


\section{Практические замечания}

Оценки двухшагового МНК для панельных данных могут быть получены с помощью обычного двухшагового алгоритма для пространственных данных (см. раздел 22.2.5), хотя вычисленные стандартные ошибки должны быть робастными. Оптимальные ОММ оценки могут быть получены с помощью команд для работы с матрицами в статистических пакетах или в языках программирования, таких как GAUSS или R. Статистические пакеты все чаще предлагают команды для работы с панельными данными, которые автоматически вычисляют оценки этой главы, в том числе оценку Ареллано-Бонда.

\section{Литература}

В этой главе активно описывается развивающаяся область исследования, которая освещается в нескольких недавних работах, посвященных панельным данным, а именно Бальтаджи (1995, 2001), Хсяо (1986, 2003), М.–Дж. Ли (2002), и Ареллано (2003). Более сложные методы представлены в работах Матиас и Севестр (1995) и Ареллано and Оноре (2001).

\textbf{22.2} Чемберлин (1982, 1984) делал акцент на использовании предположений об экзогенности. Он использовал оценку минимального расстояния. Впоследствии чаще использовался ОММ. М.–Дж. Ли (2002) и Ареллано (2003) придавали особое значение ОММ оцениванию. См. также работу Ана и Шмидта (1999).

\textbf{22.4} Привлекательна модель Хаусман and Тейлор (1981). Предполагая некоррелированность некоторых регрессоров с индивидуальными эффектами, можно  идентифицировать коэффициенты регрессоров, не меняющихся во времени.

\textbf{22.5} Линейные динамические модели представлены в литературе достаточно кратко по сравнению с объемом литературы, которая началась с Балестра and Нерлова (1966). Более сложные обсуждения даны в Бальтаджи (2001, глава 8), Хсяо (2003, глава 4), и Ареллано (2003, глава 5–8). Особенно популярна оценка Ареллано–Бонда (1991), так как она подходит для динамических моделей с фиксированными эффектами.

\textbf{22.6} Из-за своей простоты очень популярен метод <<разность разностей>>. Хотя он может быть использован для повторяющихся пространственных данных, интерпретация для панельных данных помогает сделать явными первоначальные предположения. Бертран и др. (2004) демонстрируют важность корректирования корреляции во времени на индивидуальном уровне с использованием методов раздела 22.2.3.

\textbf{22.8} Смешанные линейные модели особенно популярны в статистической литературе. Они реже используются в эконометрической литературе из-за того, что они не накладывают структуру на индивидуальные фиксированные эффекты, не меняющиеся во времени.

 {\centering
{\bf Упражнения}\\}

\textbf{22-1} Рассмотрим ОММ оценку для панельных данных, представленную в разделе 22.2.1.
\begin{itemize}
\item[{\bf (a)}] Покажите, что минимизируя по $\be$  квадратичную функцию $Q_N(\be)$, выписанную после уравнения \ref{Eq:22.3}, мы получим ОММ оценку для панельных данных, выражение для которой дано после $Q_N(\be)$ с использованием обозначения суммы.
\item[{\bf (b)}] Покажите, что эта оценка эквивалентна оценке, определенной в \ref{Eq:22.4}.
\item[{\bf (с)}] Предположим для простоты, что матрицы $\mathbf Z$ и $\mathbf X$ в \ref{Eq:22.4} не являются стохастическими, и что $\mathbf y = \mathbf X \bm\be + \mathbf u$, где среднее и дисперсия $\mathbf u$ равны $0$ и $\Omega$ соответственно. Получите ковариационную матрицу оценки в \ref{Eq:22.4} для ограниченной выборки и сравните ее с результатами для асимптотики в \ref{Eq:22.5}.
\item[{\bf (d)}] Упростите ОММ оценку для панельных данных для случая, когда $r=K$.
\end{itemize}

\textbf{22-2} Рассмотрим модель анализа панельных данных $y_{it}=\alpha + \beta \x_{it}+\gamma w_{it}+u_{it}, i=1, \dots, N, t=1, \dots, T$, где для простоты отсутствуют индивидуальные эффекты. Предположим, что регрессор-скаляр $x_{it}$  коррелирован с $u_{it}$ для всех $t$ и $s$. Для каждого варианта определите, возможно ли состоятельное IV оценивание параметров $\be$ и $\gamma$, и если это так запишите все подходящие инструменты, основываясь на обсуждении в главе 22.2. Предполагайте, что доступно три периода данных, $T=3$. Заметьте, что переменная может не быть доступна как инструмент во всех периодах, и что в разные годы могут быть доступны разные инструменты.
\begin{itemize}
\item[{\bf (a)}] Регрессор $w_{it}$ удовлетворяет предположению о сумме $\E[\sum_t w_{it} u_{it}]=0$.
\item[{\bf (b)}] Регрессор $w_{it}$ удовлетворяет предположению об одновременной экзогенности $\E[w_{it} u_{it}]=0, t=1, \dots, 3$.
\item[{\bf (с)}] Регрессор $w_{it}$ удовлетворяет предположению о слабой экзогенности $\E[w_{is} u_{it}]=0, s \leq t, t=1, \dots, 3$.
\item[{\bf (d)}] Регрессор $w_{it}$ удовлетворяет предположению о строгой экзогенности $\E[w_{it} u_{it}]=0, s, t=1, \dots, 3$.
\end{itemize}

\textbf{22-3} Повторите задание 2, тоже с тремя периодами данных, но сейчас возьмите модель $y_{it}=\alpha + \beta \x_{it}+\gamma w_{it}+u_{it}$, где $\alpha_i$  --- это фиксированный эффект, и используйте IV оценивание, основанное на модели в первых разностях, $y_{it}-y_{i,t-1}=\beta (x_{it} - x_{i,t-1})+\gamma (w_{it}-w_{i,t-1})+(u_{it}-u_{i,t-1})$.

\textbf{22-4} Рассмотрим оценку <<разность разностей>> (DID), представленную в разделе  22.6. Предположим, временной тренд $(\delta_t-\delta_{t-1})$ различается для опытной и контрольной групп.
\begin{itemize}
\item[{\bf (a)}] Будет ли DID оценка $\phi$, основанная на повторяющихся пространственных данных состоятельна? Объясните свой ответ.
\item[{\bf (b)}] Возможно ли получить состоятельную оценку $\phi$, если используются панельные данные. Объясните Ваш ответ.
\end{itemize}
 
\textbf{22-5} Используя данные о часах и заработной плате из работы Зилиака (1997), воспроизведите таблицу 22.2, насколько это возможно, с хорошим объяснением, когда набор инструментов расширяется и включает третий лаг lnwg, kids, age, agesq, и disab, и для оценки \ref{Eq:22.22} используется семь лет 1982–88.

 

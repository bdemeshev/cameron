

\part{Предварительные сведения}

В части 1 обсуждаются основные компоненты микроэконометрического анализа --- экономическая спецификация, статистическая модель и набор данных. 

В главе 1 рассматриваются отличительные особенности микроэконометрики и описывается структура книги.  В главе подчеркивается, что дискретность данных, нелинейность и неоднородность поведенческих отношений являются ключевыми аспектами микроэконометрических моделей на индивидуальном уровне. В конце главы обсуждаются обозначения и соглашения, используемые далее в книге.
%%%
Главы 2 и 3 нацелены на то, чтобы познакомить читателя с ключевыми моделями и типами данными, которые анализируются в последующих главах.


Ключевое различие в эконометрике находится между  описательными по существу статистическими моделями или обобщением данных разных уровней сложности и моделями выходящими за рамки описания взаимосвязей и пытающимися оценить параметры причинно-следственных связей. 
Классическое определение причинности в эконометрике берет свое начало из систем одновременных уравнений Комиссии Коулса, которые проводят четкое различие между экзогенными и эндогенными переменными, а также между параметрами структурной и приведенной форм модели. 
Хотя приведенные формы моделей являются очень полезными для некоторых целей, знание структурных параметров имеет важное значение для анализа последствий проводимой политики. 
Идентификация структурных параметров в рамках одновременных уравнений связана с множеством теоретических и практических сложностей. Широко используемый альтернативный подход на основе модели потенциального результата также пытается выявить параметры причинно-следственных связей, но он делает это ставя более узкие вопросы в более удобных рамках. В главе 2 приведен обзор фундаментальных вопросов, которые возникают в рамках этих и других альтернативных концепций. Читатели, у которых сейчас сложилось впечатление, что материал данной главы сложный, могут вернуться к ней после знакомства с конкретными моделями рассмотренными далее в этой книге.
	
	
Способность исследователей правильно выявить причинно-следственную связь зависит не только от статистических инструментов и моделей, но и от типа доступных данных. Экспериментальные данные являются стандартом для установления причинно-следственных связей. Тем не менее, не экспериментальные, а описательные данные лежат в основе большинства эконометрических исследований. В Главе 3 рассмотрены  плюсы и минусы трёх основных типов данных: описательных данных, данных из социальных экспериментов и данных из естественных экспериментов; рассматриваются сильные и слабые стороны проведения причинного-следственного вывода в каждом случае.

\chapter{Обзор}

\section{Введение}

Эта книга содержит подробное рассмотрение микроэконометрического анализа, т.е. анализа  данных на индивидуальном уровне, например, экономического поведения отдельных лиц или фирм. Более широкое определение будет также включать сгруппированные данные. Обычно регрессионный метод применяются пространственным (cross-section) или панельным данным.
	
	
Анализ индивидуальных данных имеет долгую историю. Эрнст Энгель (1857) был одним из первых исследователей  бюджетов домашних хозяйств. Аллен и Боули (1935), Хаутаккер (1957), и Прайс и Хаутаккер (1955) внесли важный вклад  в том же направлении, используя те же подходы. Другие исследования также оказали значительное влияние на развитие микроэконометрики, однако они не всегда использовали информацию на индивидуальном уровне, например, Маршак и Эндрюс (1944) занимались  теорией производства и Вольд и Джюрин  (1953), Стоун (1953 ) и Тобин (1958) --- потребительским спросом.
	
	
Материал описанный в этой книге также связан с моделями дискретного выбора и моделями цензурированных и усеченных переменных. Серьёзное эконометрическое исследование этой темы было начато  в работе МакФаддена (1973 , 1984) и Хекмана (1974, 1979). Эти работы отходят от использования линейных моделей, которые использовались в более ранних работах . Впоследствии эти работы привели к значительным методологическим инновациям в эконометрике. Среди ранних учебников, в которых освещён этот материал (и более) являются работы Маддалы (1983) и Амэмии (1985). Как отмечали Хекман (2001), МакФадден (2001 ) и другие, многие из фундаментальных вопросов, которые фигурировали в ранних работах по рыночным данным, остаются важными и в настоящее время. Особенно это справедливо в отношении условий, необходимых для идентифицируемости причинно-следственных экономических отношений. Тем не менее, микроэконометрика имеет свой особый стиль, что оправдывает написание специального учебника.
	
%%%
Современная  микроэконометрика основанна на данных об индивидах, домохозяйствах и фирмах, и во многом  обязана широкой доступности различных данных от пространственных и панельных выборок до данных переписей. В последние два десятилетия с использованием электронного способа хранения данных  и сбора данных на индивидуальном уровне объем доступных данных возрос скачкообразно. 
Так же существенно возросли вычислительные мощности для анализа больших и сложных наборов данных. 
Во многих случаях доступны данные событийного уровня, например, маркетинговые исследования часто имеет дело с данными покупок, собранных при помощи электронных сканеров в супермаркетах, а в литературе по теории отраслевых рынков встречаются эконометрические исследования, основанные на данных авиаперевозок, собранных системами онлайн-бронирования билетов. В настоящее время появляются новые отрасли экономики, такие как социальное экспериментирование и  экспериментальная экономика, которые создают <<экспериментальные>> данные. Появляются новые возможности моделирования, которое отсутствует, в случае доступности только агрегированных рыночных данных. Между тем стремительный рост объемов и типов данных также породил множество методологических вопросов. Обработка и эконометрический анализ таких  микроданных большого объёма, с целью выявления моделей экономического поведения, составляет ядро микроэконометрики. Эконометрический анализ таких данных является главным предметом этой книги.
	
	
Основными предшественниками этой книги являются книги Маддалы (1983) и Амэмии (1985). Подобно им мы охватываем темы, которые представлены лишь кратко, или вообще не представлены, во вводных курсах или курсах первого года обучения. Особенно по сравнению с книгой Амэмии (1985) наша книга больше ориентирована на практическое применение. Уровень изложения тем не менее достаточно продвинутый, особенно для исследователей практиков из областей менее математизированных, чем экономика.
	
	
Относительно продвинутое математическое изложение  необходимо по нескольким причинам. Во-первых, данные часто являются дискретными или цензурированными, и в этом случае используются нелинейные методы, такие как логит, пробит, и тобит модели. Это приводит к статистическому анализу, основанному на более сложной асимптотической теории.
	
	
Во-вторых, предположения о распределении для анализа таких данных становятся крайне важными. Возможным решением является разработка сильно параметризованных моделей, которые являются достаточно гибкими, чтобы охватить всю сложность данных, но эти модели сложно оценить. Более распространенным решением является минимизация параметрических предположениях и предоставление статистических выводов на основе стандартных ошибок, которые являются робастными к таким проблемам, как гетероскедастичность и кластеризации. В таких случаях значительные знания могут быть необходимых для получения достоверных статистических выводов, даже если используется стандартный пакет регрессии.
	
	
В-третьих, экономические исследования часто ставят целью определение причинности, а не просто измерение корреляции, несмотря на доступ к данным наблюдений, а не экспериментальным данным. Это приводит к методам, которые способны выявить причинность, например, таким как инструментальные переменные, системы одновременных уравнений, коррекция ошибки измерения, коррекция смещения отбора, панельные данные с фиксированными эффектами, и разность разностей.
	
	
В-четвертых, микроэкономические данные, как правило, собираются с использованием пространственных или  панельных обследований, переписей или социальных экспериментов. Данные обследований, собранные с помощью этих методов, могут приводить к проблемам, связанным со сложной методологии обследования, отклонением от предположения случайности выборки, и проблеме самоотбора, ошибкам измерения, и неполным, и / или пропущенным данным. Для корректного оценивания таких эконометрических моделей необходимо использование продвинутых методов.
	
	
Наконец, довольно часто два или более осложнения происходят одновременно, например, эндогенность в логит модели с панельными данными. В этом случае очень трудно реализовать стандартные подходы. Вместо этого, необходимо значительное понимание теории, лежащей в основе методов, так как исследователю может потребоваться  чтение современных статей и адаптация под свои нужды существующего программного обеспечения.  


\section{Отличительные аспекты микроэконометрики}

Сейчас мы рассмотрим несколько преимуществ микроэконометрики, которые вытекают из её отличительных особенностей.

%%% --->
\subsection{Дискретность и нелинейность}

Первым и наиболее очевидным достоинством является то, что микроэконометрические данные, как правило, слабо агрегированы. Этот факт влияет на выбор функциональных форм, которые используются для анализа переменных, представляющих интерес. Во многих, если не в большинстве случаев, линейные функциональные формы не подходят. Более существенно то, что дезагрегирование выдвигает на первый план неоднородность индивидов, фирм и организаций, которая должны надлежащим образом учитываться, если необходимо сделать верные выводы о лежащих в основе данных отношениях. Мы обсудим эти вопросы более подробно в следующих разделах.
	
	
Конечно агрегирование не полностью отсутствует в микроданных, например, в данных по домохозяйствам или учреждениям, однако уровень агрегирования, как правило, на порядок ниже, чем это принято в макро анализе. 
В последнем случае процесс агрегирования приводит к сглаживанию, разнонаправленные изменения взаимно уничтожаются. 
Агрегированные переменные часто показывают более гладкое поведение, чем их компоненты, более того отношения между агрегированными переменными также часто показывают большую гладкость, чем отношение между компоненты. Например, связь между двумя переменными на микроуровне может быть кусочно-линейной со многими узлами. После агрегирования, связь будет хорошо аппроксимироваться гладкой функцией. Поэтому, следствием дезагрегирования данных является отсутствие непрерывности и гладкости как самих переменных, так и взаимосвязей между ними.
	
	
Обычно индивидуальные данные и данные фирм сильно изменчивы, как в одномоментной выборке так и во временных рядах. Например, среднее еженедельное потребление, например,  говядины, очень вероятно будет положительным и плавно меняющимся, тогда как потребление отдельного домохозяйства за конкретную неделю часто равно нулю и может становиться положительным значениям время от времени. 
Среднее число часов, отработанных работником женского пола вряд ли будет нулевым, но если рассматривать женщин по отдельности то многие из них имеют ноль часов работы (угловое решение), переходя на положительные значения в другое время в ходе своего трудового стажа. Средние расходы на отпуск, как правило, положительные, но многие домохозяйства могут иметь нулевое значение расходов на отпуск за конкретный год. Среднедушевое потребление табачных изделий, как правило, положительное, но многие индивиды из генеральной совокупности никогда не потребляли сигареты и никогда не будут, независимо от цены и дохода. Как отмечает Падни (1989), микро-данные чаще содержат  <<дыры, изломы и углы>>. Дыры соответствуют неучастию в исследуемой деятельности, изломы соответствуют смене поведения, а углы соответствуют непотреблению или неучастию в определенный момент времен. То есть, дискретность и нелинейность значений --- отличительная черта микроэконометрики.
	
	
Важный класс нелинейных моделей в микроэконометрике имеет дело с ограниченными зависимыми переменными (Маддала, 1983). Этот класс включает в себя множество моделей, которые обеспечивают подходящие рамки для анализа дискретных значений и значений с ограниченным диапазоном изменения. Такие инструменты анализа, конечно, также доступны для анализа макро данных, если это необходимо. Дело в том, что они незаменимы в микроэконометрике и являются её отличительной особенностью.

\subsection{Более высокая реалистичность}


Макроэкономика основана на сильных предположениях; таких, например, как предположение о репрезентативном агенте. Часто с помощью микроэкономических рассуждений обосновывают определенные спецификации моделей   или интерпретируют эмпирические результаты. 
Однако, чаще всего нельзя сказать, как влияет агрегирование по времени и по индивидам. Иногда при агрегировании делаются очень сильные предположения. Например, считается, что агрегирование отражают поведение гипотетического репрезентативного агента. Это предположение  однозначно не вызывают доверия.
	
%%% --->
С точки зрения микроэкономической теории, количественный анализ основанный на микроданных можно рассматривать как более реалистичный, чем  основанный на агрегированных данных. 
Есть три обоснования этого утверждения. Во-первых, измерение переменных, используемых в таких гипотезах, часто более просто (хотя и не обязательно свободно от ошибки измерения) и имеет большее соответствие с теорией, которая тестируется. 
Во-вторых, гипотезы об экономическом поведении, как правило, вытекают из теорий индивидуального поведения. Если эти гипотезы проверяются с использованием агрегированных данных, то необходимо вводить большое количество предположений и упрощений. Упрощающее предположение о репрезентативном агенте вызывает существенную потерю информации и серьезно ограничивает сферу эмпирического исследования. К счастью, таких предположений можно избежать в микроэконометрике, и, как правило,   микроданные обеспечивают более удобную основу для тестирования микроэкономических гипотез. Впрочем, это не означает, что потенциальная выгода от использования микроданных достигается на практике. 
Наконец, реалистичное изображение экономической деятельности должно обеспечивать широкий диапазон результатов, которые являются следствием индивидуальной неоднородности и, которые предсказываются соответствующей теорией. В этом смысле микроэкономические данные могут приводить к более реалистичным моделям.
	
	
Источником микроэконометрических данных часто являются опросы фирм или домохозяйств. В данных отражаются разнообразные формы поведение, результаты многих решений являются дискретными или качественными.  Такие наборы данных имеют много необычных  особенностей, которые требуют использования специальных инструментов при моделировании и анализе. Подобные особенности могут иметь место и в макроэконометрике, но там они встречаются гораздо реже.

%%% --->
\subsection{Более высокая насыщенность информацией}

Потенциальные преимущества микроданных могут быть реализованы, если эти данные являются информативными. Выборочные обследования часто содержат независимые наблюдения по тысячам единиц пространственной выборки, такие данные считаются  более информативными, чем типичные макроэкономические. Макроэкономические временные ряды часто сильно коррелированы и  обычно состоят всего из нескольких сотен наблюдений.
	
	
Как будет отмечено в следующей главе, на практике ситуация не такая идеальная потому, что микроданные могут быть сильно зашумлены. 
На индивидуальном уровне разнообразные уникальные факторы могут играть большую роль в определении поведения. Часто эти факторы ненаблюдаемы, поэтому исследователи трактуют их как случайные компоненты, и эти факторы могут объяснять существенную часть наблюдаемых изменений зависимой переменной. 
В этом смысле случайность играет большую роль в микроданных, чем в макроданных. Конечно же, эта случайность влияет на  показатели адекватности регрессии. Студенты, начинающие изучать эконометрику с анализа агрегированных временных рядов часто хотят увидеть большое значения $R^{2}$. Встретившись с пространственной выборкой в первый раз, они разочаровываются низкой объясняющей силой  регрессии. Тем не менее, есть серьезные основания полагать, что по крайней мере в определенном смысле, большие наборы микроданных несут много информации.
	
%%% --->	
Другая особенность состоит в том, что, когда мы имеем дело с пространственными данными (одномоментная выборка), очень мало  можно сказать о временных свойствах изучаемой зависимости. Временной  аспект поведения можно исследовать с помощью панельных и переходных данных.
	
	
Во многих случаях исследователь заинтересован в определении поведенческой реакции некоторой группы экономических агентов при некоторых заданных экономических условиях. Например, можно исследовать влияние страхования от безработицы на поведение молодых безработных при поиске работы. Другим примером является изменение предложения труда у лиц с низким доходом, которые получают материальную поддержку. Без использования микроданных ответить на эти вопросы напрямую в эмпирическом исследовании невозможно.
	
\subsection{Микроэкономические основания}

	Эконометрические модели различаются по роли, которую они отводят для экономической теории. С одной стороны, есть модели, в которых теоретические основания играют  главную роль в спецификации модели и в выборе процедуры оценивания. С другой стороны, есть эмпирические исследования, которые используют гораздо меньше экономической теории.
	
	%%% -->
	Цель анализа в первом случае заключается в выявлении и оценки фундаментальных параметров модели, которые также называют глубокими параметрами. Эти параметры  характеризуют индивидуальные вкусы и предпочтения и/или технологические отношения. Для краткости обозначения, мы называем это \textbf{структурным подходом}. Его отличительной чертой является сильная зависимость от экономической теории и упор на причинно-следственные выводы. 
		%%% -->
	Такие модели могут требовать большого количества предположений, например, точной спецификации функции затрат, производственной функции или случайной составляющей. Эмпирические выводы в таком случае могут быть неробастными в случае отклонения от сделанных предположений. В разделе 2.4.4 мы будем говорить подробнее об этом подходе. На данном этапе мы просто подчеркнем, что, если структурный подход реализуется по агрегированным данным, то оценки фундаментальных параметров можно получить  только при очень жестких (и вероятно нереалистичных) предпосылках. Микроданные более уместны для структурного подхода, поскольку они обеспечивают большую гибкость при построении модели.
	
		%%% -->
	
	Цель анализа во втором случае состоит в моделировании отношения между интересующей исследователя зависимой переменной и объясняющими переменными, которые рассматриваются как экзогенные. 
	Более формальные определения эндогенности и экзогенности приведены в главе 2. Для краткости, мы называем этот подход подходом приведенной формы (reduced form approach). 
	Важно отметить, что такой подход не всегда учитывает все причинно-следственные взаимозависимости. Регрессионная модель, в которой акцент делается на прогнозировании $y$ при заданных регрессорах $x$, а не на причинной интерпретации параметров регрессии, часто называется регрессией в приведенной форме. 
	Как объясняется в главе 2, параметры приведенной формы часто являются функциями структурных параметров, поэтому они не могут интерпретироваться без некоторой информации о структурных параметрах.



\subsection{Дезагрегирование и неоднородность}

Говорят, что многие проблемы в макроэконометрике возникают из-за автокорреляции макроэкономических временных рядов, а в микроэконометрике проблемы берут своё начало из-за гетероскедастичности на индивидуальном уровне. 
Хотя это утверждение в значительной степени относится к существенной части работ по микроэконометрике, оно нуждается в уточнении.
В ряде  микроэконометрических моделей моделирование динамической зависимости может быть важным вопросом.
	
	
Преимущества дезагрегирования, которые были выделены ранее в этом разделе, имеют свою издержки:  когда данные становятся более детализированными важность учёта неоднородности между индивидами увеличивается. Неоднородность, а точнее ненаблюдаемая неоднородность, играет очень важную роль в микроэконометрики. Очевидно, что многие переменные, которые отражают неоднородность, такие как пол, раса, образование и социально-демографические факторы, являются непосредственно наблюдаемыми и как следствие, могут быть учтены. Напротив, различия в индивидуальной мотивации, способностях, интеллекте и т.д. либо не наблюдаются, либо, в лучшем случае,  наблюдаются частично.
	
	
Самый простой выход из ситуации --- это игнорирование неоднородности, то есть включение её в случайную ошибку регрессии. В конце концов именно так поступают с кучей мелких ненаблюдаемых факторов. Этот шаг, конечно, увеличивает необъясняемую часть вариации зависимой переменной. Более важно, что игнорирование системных индивидуальных различий приводит к \textbf{смешиванию} (confoundign) с другими факторами, которые также являются источниками системных индивидуальных различий. 

Смешивание факторов происходит тогда, когда вклад различных регрессоры  не может быть статистически разделен. Предположим, например, что регрессор $x_{1}$ (образование), может быть источником изменения переменной $y$ (дохода), а другая переменная --- $x_{2}$ (способности), которая является еще одним источником изменения дохода, не включена в модель. Тогда та часть полной изменения зависимой переменной, которая приходится на второй регрессор, может быть неправильно соотнесена с первой переменной. Интуитивно, их относительный вклад в изменение дохода смешан. Основной причиной смешивания регрессоров является ошибочное невключение регрессоров в модель и включения других переменных, которые являются прокси-переменными для пропущенных регрессоров.
	

%%%	<----
Рассмотрим случай, в котором дамми-переменная участия в программе $D$ (0/1) входит в линейную регрессию вместе с вектором регрессоров $x$.
	
\begin{equation}
y=X'\beta+D\alpha+u
\end{equation}


где $u$ --- ошибки модели. Термин <<воздействие>> используется в биологической и экспериментальной литературе для обозначения режима назначенного определенной группе участников эксперимента. 
В эконометрике это обычно относится к участию в какой-либо деятельности, которая может повлиять на исследуемую переменную. 
Участие в этой деятельности может быть назначено  участникам случайным образом или может быть самостоятельно выбрано участником. 
Таким образом, хотя  люди сами выбирают сколько лет им обучаться в школе, считается, что продолжительность обучения можно рассматривать как <<воздействие>>. 
Предположим, что участие в программе является дискретным, тогда коэффициент $\alpha$ измеряет среднее воздействие участия в программе $(D = 1)$ при прочих равных. 
Если не учитывать ненаблюдаемую неоднородность, то интерпретация результатов может быть неоднозначной. 
Если коэффициент $\alpha$ значим, то возникает следующий вопрос: является ли $\alpha$ значимо отличным от нуля, потому что $D$ коррелирует с некоторой ненаблюдаемой переменной, которая влияет на $y$ или потому, что существует причинно-следственная связь между $D$ и $y$? 
Например, если модель рассматривает в качестве воздействия университетское образование, и в регрессоры не включаются способности индивида, причинно-следственная интерпретация становится сомнительной. Так как проблема ненаблюдаемой неоднородности является важной, большое внимание следует уделять тому, как её учитывать. 


В некоторых случаях, когда имеются динамические зависимости, тип доступных данных накладывает  ограничения на выбор способа учета неоднородности. 
Рассмотрим пример с двумя домашними хозяйствами, одинаковыми во всех отношениях за исключением того, что одно из них более склонно потреблять товар A. 
Можно учесть это различие, включив в  индивидуальные функции полезности параметр неоднородности, который отражает различия в предпочтениях. 
Предположим теперь, что существует теория потребительского поведения, которая утверждает, что потребители становятся зависимыми от товара А, в том смысле, что чем больше они потребляют его в первом периоде, тем выше вероятность того, что они будут потреблять его больше в будущем. Эта теория дает другое объяснение индивидуальных различий в потреблении товара А. Учитывая неоднородность предпочтений мы можем проверить, какая причина, неоднородность предпочтений или зависимость от потребления, вызывает разную структуру потребления. 
Этот тип проблемы возникает всякий раз, когда некоторые динамические особенности модели порождают постоянство  наблюдаемых результатов. Несколько примеров такого рода проблем разобраны в различных главах этой книги.
	
	
Различные подходы для моделирования неоднородности существуют в микроэконометрике. Краткое упоминание о некоторых из них следует далее, а подробное изложение отложено до следующих глав.
	
	
Крайним решением будет игнорировать все ненаблюдаемые индивидуальные различия. Если ненаблюдаемая неоднородность не коррелирует с наблюдаемой неоднородностью, и если зависимая переменная не имеет межвременной зависимости, то упомянутые проблемы не возникнут. Конечно, эти  предположения являются очень сильными, однако даже с их учетом  не все эконометрические трудности исчезают.
	
	
Один из способов учета неоднородности индивидов --- трактовать её как наличие фиксированных эффектов в модели, и оценить её с помощью коэффициентов при  индивидуальных дамми-переменных.
Например, в панельных данных, у каждого индивида может быть своя дамми-переменная. Это приводит к резкому увеличению числа параметров, т.к. при добавлении нового индивида добавляется новый параметр. Следовательно, данный подход не будет работать для пространственных данных. 
Бывает доступно несколько наблюдений для отдельного индивида, чаще всего в виде панельных данных, где для каждого из $N$ индивидов имеется временной ряд длины $T$.
Эти повторяющиеся наблюдения позволяют либо оценить, либо устранить фиксированный эффект, например, взятием первых разностей в случае, если модель является линейной с аддитивными  фиксированными эффектами. 
Если, как это часто бывает, модель является нелинейной, то фиксированный эффект обычно неаддитивный, и поэтому должны быть рассмотрены другие подходы.
	
	
Второй подход заключается в использовании моделей со случайными эффектами. Есть несколько способов специфицировать модель со случайными эффектами. Одна из популярных формулировок предполагает, что один или несколько параметров регрессии, чаще всего свободный член, изменяется случайным образом по индивидам. 
В другой формулировке ошибки регрессии состоят из нескольких компонент, включая случайную компоненту для каждого индивида.  Далее модель со случайными эффектами используется для оценивания  параметров распределения случайной составляющей. В некоторых случаях, таких как анализ спроса, случайная компонента может быть интерпретирована как случайное изменение предпочтений. Модели со случайными эффектами можно оценивать, используя как пространственные, так и панельные данные.
	

\subsection{Динамика}

Очень распространенным предположением в анализе пространственных данных является отсутствие межвременной зависимости, то есть, отсутствие динамики. Таким образом, неявно предполагается, что наблюдения соответствуют стохастическому равновесию. А отклонения от равновесия представлены  независимыми случайными ошибками. Даже в микроэконометрике для некоторых данных такое предположение может быть слишком сильным. Например, оно не согласуется с наличием коррелированной ненаблюдаемой неоднородности. Зависимость от лагированной зависимой переменной также нарушает это предположение.

Вышеизложенное иллюстрирует некоторые из потенциальных ограничений анализа пространственной выборки с одним наблюдением для каждого индивида. Некоторые ограничения могут быть преодолены, если имеются повторные пространственные выборки. Тем не менее, если есть динамические зависимости, наиболее простым подходом может оказаться использование панельных данных.

\section{Структура книги}
 
Книга разделена на шесть частей. В части 1 представлены вопросы, связанные с микроэконометрическим моделированием. В частях 2 и 3 мы познакомимся с общей теорией оценивания и статистических выводов для нелинейных регрессионных моделей. Части 4 и 5 посвящены основным моделям, используемым в прикладной микроэконометрике для анализа пространственных и панельных данных. Часть 6 охватывает более широкие темы, которые опираются на материал, представленный в предыдущих главах.

Структура книги представлена в таблице 1.1. Далее в этом разделе будет сказано про каждую часть более подробно.
	
\subsection{Часть 1. Предварительные сведения}

В главах 2 и 3 рассмотрены  особенности микроэконометрического подхода к моделированию и микроэкономические структуры данных в широких рамках  статистического регрессионного анализа. Многие из вопросов, поднятых в этих главах, обсуждаются на протяжении всей книги, по мере того, как развивается необходимый инструментарий.

\begin{table}[h]
\begin{center}
\caption{\label{tab:bookstructure}Структура книги}
\begin{tabular}{p{8cm}p{1cm}p{6cm}}
\hline
\hline
Часть и Глава  & Требуемые главы & Пример \\
\hline
\textbf{1. Предварительные сведения} & & \\
1. Обзор & - & \\
2. Причинно-следственные и статистические модели & - & Системы одновременных уравнений \\
3. Структуры микроэкономических данных & - & Данные наблюдений \\
\textbf{Часть 2. Основные методы} & & \\
4. Линейные модели & - & Метод наименьших квадратов \\
5. Оценивание с помощью метода максимального правдоподобия и
нелинейнего метода наименьших квадратов & - & М-оценки или экстремальные оценки \\
6. Обобщённый метод моментов и системы уравнений & 5 & Инструментальные переменные \\
7. Проверка гипотез & 5 & Тест Вальда, множителей Лагранжа и отношения правдоподобия \\
8. Тесты на спецификацию и выбор моделей & 5,7 & Тест на условный момент \\
9. Полупараметрические методы & - & Ядерная регрессия \\
10. Методы численной оптимизации & 5 & Итерационная процедура Ньютона-Рафсона \\
\textbf{Часть 3. Методы симуляционного моделирования} & & \\
11. Бутстрэп методы & 7 & Метод $t$-перцентилей \\
12. Методы симуляционного моделирования & 5 & Симуляционный метод максимального правдоподобия \\
13. Байесовские методы & - & Метод Монте-Карло по схеме марковской цепи \\
\textbf{Часть 4. Модели пространственных данных} & & \\
14. Модели бинарного выбора & 5 & Логит и пробит для $y=(0,1)$ \\
15. Мультиномиальные модели & 5, 14 & Мультиномиальный логит для $y=(1,2,\ldots, m)$ \\
16. Тобит-модели и модели выбора & 5, 14 & Тобит для $y=\max(y^*,0)$ \\
17. Транзитные данные: Анализ выживаемости & 5 & Модель Кокса пропорциональных рисков \\
18. Модели смеси и ненаблюдаемая гетерогенность & 5, 17 & Ненаблюдаемая гетерогенность \\
19. Модели множественных рисков & 5, 17 & Множественные риски \\
20. Модели счетных данных & 5 & Пуассоновская модель для $y=0, 1, 2, \ldots$ \\
\textbf{Часть 5. Модели анализа панельных данных} & & \\
21. Линейные модели панельных данных: основы & - & Фиксированные и случайные эффекты \\
22. Линейные модели анализа панельных данных: дополнения & 6, 21 & Динамические и эндогенные регрессоры \\
23. Нелинейные модели панельных данных & 5, 6, 21, 22 & Панельные логит, пробит и пуассоновская модель \\
\textbf{Часть 6. Дальнейшие темы} & & \\
24. Стратифицированные и кластеризованные выборки & 5 & Данные $(y_{ij},x_{ij})$ коррелированные по $j$ \\
25. Оценка эффектов воздействия & 5, 21 & Регрессор $d=1$ при участии \\
26. Модели ошибок измерения & 5 & Логит-модель с ошибками измерения\\
27. Пропущенные данные и восстановление данных & 5 & Регрессия с пропущенными наблюдениями 
\end{tabular}
\end{center}
\end{table}
В столбце <<Требуемые главы>> приведены главы необходимые помимо изложения метода наименьших квадратов и взвешенного метода наименьших квадратов в главе 4. 


\subsection{Часть 2. Основные методы}

Главы 4---10 подробно описывают общие методы, используемые при классическом оценивании моделей и статистических выводах. Результаты, приведенные в главе 5, в частности, широко используются на протяжении всей книги.
	
	
В Главе 4 изложены некоторые результаты для линейной регрессионной модели. Внимание  акцентируется  на тех вопросах и методах, которые наиболее актуальны для остальной части книги. Анализ относительно прост, поскольку существуют явные выражения для оценок линейных моделей, например, для метода наименьших квадратов.

В главах 5 и 6 обсуждаются способы оценивания, которые могут быть применены к нелинейным моделям, для которых обычно не существует явного решения. Асимптотическая теория используется для получения распределения оценок, с акцентом на получение робастных стандартных  ошибок коэффициентов, которые опираются на относительно слабые предположения о распределении ошибок. 
Общее изложение методов оценивания, в частности нелинейного метода наименьших квадратов и метода максимального правдоподобия, представлено в главе 5. Более сложные обобщенный метод моментов и метод инструментальных переменных приведены в главе 6.
	
	
Глава 7 описывает классические способы проверки гипотез, когда оценки являются нелинейными или, когда тестируемая гипотеза нелинейна по параметрам. Тесты на спецификацию  в дополнение к проверке гипотез обсуждаются в главе 8.
	
	
В главе 9 представлены полупараметрические методы оценки, такие как ядерная регрессия. Важным примером  является гибкое моделирование условного математического ожидания. В примере с патентами непараметрическая регрессионная модель задана в виде  $\E[y|x] = g(x)$, где функция $g(\cdot)$ не специфицирована и должна быть оценена. Требуется оценить бесконечномерный объект $g(\cdot)$, что приводит к использованию нестандартной асимптотической теории. При наличии дополнительных регрессоров необходимы  дополнительные предположения о структуре модели, такие  методы называются полупараметрическими.
	
	
В Главе 10 изложены способы вычисления оценок параметров, когда оценки определяются неявно, как правило, как решения некоторых условий первого порядка.

\subsection{Часть 3. Методы симуляционного моделирования}
	
В Главах 11---13 изложены методы оценивания и построения статистических выводов, использующие симуляционный подход. Эти методы обычно требуют большого количества вычислений и в настоящее время меньше используются, чем методы, представленные в части 2.
	
	
В главе 11 представлены бутстрэп методы статистического вывода. Бутстрэп позволяет получить  эмпирическое распределение оценки путем генерирования  выборки значений  с помощью симуляций. Например, могут использоваться новые выборки с повторениями из исходной выборки. Бутстрэп может обеспечить простой способ вычисления стандартных ошибок, если асимптотические являются сложными, например, в случае двухшаговых оценок. Кроме того, при корректной реализации, бутстрэп может привести к улучшению статистических свойств оценок в случае малого числа наблюдений.
	
	
В главе 12 рассматриваются методы оценивания, основанные на симуляциях, разработанные для моделей, где используется интеграл по закону распределения, для которого не существует явного аналитического решения. Оценивание возможно путём генерирования нескольких выборок из нужного распределения и последующего усреднения.
	
	
В главе 13 представлен байесовский подход, в котором  распределение наблюдаемых данных комбинируется с  априорным распределения параметров для получения  апостериорного распределения параметров, на базе которого осуществляется оценивание. Современные вычислительные методы  позволяют выполнить расчёты даже в случае при отсутствии аналитического представления для апостериорного распределения. Байесовский подход к оцениванию и статистическим выводам сильно отличается от классического подхода. Тем не менее, во многих случаях только байесовский подход позволяет решить проблемы, которые в противном случае оказываются неразрешимыми.

\subsection{Часть 4. Модели для пространственных данных}

В главах 14-20 представлены основные нелинейные модели для пространственных данных. Эта часть является основным элементом книги и содержит сложные темы, такие как модели для качественных зависимых переменных и самоотбор выборки. Классы моделей определяется исходя из диапазона значений, который принимает зависимая переменная.
	
	
Модели для случая бинарной зависимой переменной, то есть принимающей только два возможных значения, $y = 0$ или $y = 1$, представлены в главе 14. В Главе 15 приведено обобщение до мультиномиальных моделей, в которых зависимая переменная  принимает несколько дискретных значений. В качестве примеров можно привести статус занятости (занятые, безработные, и входящие в рабочую силу) или используемый для поездки на работу транспорт (на автомобиле, автобусе или поезде). Линейные модели могут быть полезными, но не адекватными, так как они могут предсказывать значение вероятности больше 1. Вместо линейных используется логит, пробит, и связанные с ними модели.
	
	
В главе 16 представлены модели с цензурированием, усечением и самоотбором выборки. Примерами могут служить: количество часов работы в год при решении работать, и медицинские расходы при госпитализации. В этих случаях данные не полностью наблюдаются, есть большое количество как $y = 0$, так и $y > 0$. Модель для наблюдаемых данных является нелинейной, даже если  процесс для скрытой переменной является линейным, а линейная регрессия по наблюдаемым данным  может сильно вводить в заблуждение. Возможна простая корректировка для цензурирования, усечения и смещения самоотбора, например, использование тобит моделей, но они сильно зависят от  предположений о законе распределения.
	
	
Модели длительности представлены в главах 17---19. Примером может служить продолжительность безработицы. Стандартные модели регрессии включают экспоненциальную регрессию, модель Вейбулла, и модель пропорциональных рисков Кокса. Кроме того, как и в главе 16, зависимая переменная часто наблюдается лишь частично. Например, может быть известна длительность текущего состояния, а не полная длительность, т.к. состояние не является законченным.
	
	
В Главе 20 представлены  модели для счётных данных. Например, количество визитов к врачу или длительность госпитализации в днях. Опять же модели являются нелинейными так как количества и, следовательно, условные средние неотрицательны. Известными параметрическими моделями являются модель Пуассона и отрицательная биномиальная.

\subsection{Часть 5. Модели анализа панельных данных}

В главах 21-23 рассматриваются способы оценки панельных данных. В этом случае данные наблюдаются в несколько периодов времени для каждого из индивидов в выборке, поэтому зависимая переменная и регрессоры проиндексированы как по времени, так и номеру индивида. Любой анализ должен учитывать возможность положительной корреляции ошибок в разные периоды времени для конкретного индивида. Кроме того, панельных данных обычно достаточно, чтобы учесть ненаблюдаемые постоянные во времени индивидуальные  эффекты, что позволяет делать причинно-следственные выводы при более слабых предположениях, чем  в пространственной выборке.


Основные линейные модели панельных данных представлены в главе 21, с акцентом на модели с фиксированным и случайным эффектом. Обобщения до моделей, допускающих лагированную зависимую переменную в качестве регрессора или эндогенные регрессоры, представлены в главе 22.  Нелинейные методы части 4 для панельных данных изложены в главе 23.


Методы для панельных данных представлены ближе к концу книги для единого и цельного изложения. Главу 21 можно было бы разместить сразу после главы 4. Она написана в доступной манере изложения и не требует дополнительных знаний кроме метода наименьших квадратов.

\subsection{Часть 6. Дальнейшие темы}

В этой части рассматриваются важные темы, которые могут в целом относится ко всем без исключения моделям рассмотренным в частях 4 и 5. Глава 24 посвящена моделированию кластерных данных в рамках различных моделей. 
В главе 25 обсуждается оценивание воздействия. Оценивание воздействия является общим термином, который может охватывать широкий спектр моделей, в которых основное внимание уделяется измерению влияния некоторого воздействия. Воздействие назначается либо экзогенно, либо случайно, при этом интерес состоит в измерении некоторой результирующей переменной. В главе 26 рассматриваются последствия ошибок измерения зависимой переменной или  регрессоров, с акцентом на некоторых распространённых нелинейные модели. В главе 27 рассматриваются некоторые методы работы с пропущенными наблюдениями в линейных и нелинейных моделях регрессии.

\section{Как пользоваться книгой}

Книга предполагает общее представление о модели линейной регрессии в матричной форме. Она написана на математическом уровне первого года Ph.D. по экономике и по уровню сопоставима с учебником Грина (2003).


Хотя часть материала этой книги изучается на первом году обучения по программе Ph.D., большая часть  появляется на втором курсе эконометрики Ph.D. или в курсах микроэкономики ориентированных на реальные данные, таких как экономика труда, государственная экономика, или теория отраслевых рынков. Книга может быть полезна как отдельный учебник эконометрики или как дополнительный учебник к указанным курсам. В целом, книга представляет собой справочник для прикладных исследователей в области экономики, в смежных социальных науках, таких как социология и политология и в эпидемиологии.
	
	
Для читателей, использующих эту книгу в качестве справочника, главы в которых представлены модели по возможности были сделаны максимально независимыми от остального содержания книги. В частях 4 и 5 для ознакомления с конкретными моделями  достаточно прочитать отдельно соответствующую главу. Может лишь потребоваться владение общими результатами теории оценивания из главы 5 и реже из главы 6. Большинство глав начинаются с обсуждения и примера понятного широкой аудитории.
	
\begin{table}[h]
\begin{center}
\caption{\label{tab:plan}План 20-ти лекций 10-ти недельного курса}
\begin{tabular}[t]{llcll|}
\hline
\bf{Лекции} & \bf{Главы} & \bf{Темы} \\
\hline
1-3   & 4, Приложение А & Повторение линейной модели и асимптотической теории \\
4-7   & 5               & Оценивание: М-оценивание, ММП, и НМНК\\
8     & 10              & Оценивание: численная оптимизация \\
9-11  & 14, 15          & Модели: бинарные и мультиномиальные \\
12-14 & 16              & Модели: цензурированные и усечённые \\
15    & 6               & Оценивание: ОММ \\
16    & 7               & Тестирование: проверка гипотез \\
17-19 & 21              & Модели: базовые линейные панели \\
20    & 9               & Оценивание: полупараметрическое \\
\hline
\end{tabular}
\end{center}
\end{table}
	
Преподавателям,  использующим эту книгу в качестве основного учебника курса, стоит ознакомить студентов как можно раньше с нелинейными пространственными моделями и линейными моделями для панельных данных, пропуская множество глав посвященных методам.
Основные нелинейные модели представлены в главах 14---16, они требуют знания метода максимального правдоподобия и метода наименьших квадратов, изложенных в главе 5. В главе 21 рассматриваются линейные модели панельных данных, для её чтения достаточно ознакомиться с главой 4.
	
	
В таблице 1.2 представлен план полусеместрового курса второго года магистратуры в Калифорнийском Университете в Дэвисе, который следует сразу же после курса статистики и эконометрики на первом году обучения. Половины семестра достаточно чтобы охватить основные результаты первой половины глав плана.
При наличии дополнительного времени можно углубиться в детали или покрыть частично материал глав 11-13 по методам требующим большого количества вычислений (симуляционное оценивание, бутстрэп, которые также кратко представлен в главе 7 и байесовские методы); дополнительный материал по пространственным моделям (модели длительности и счётные модели, представленные в главах 17-20); и дополнительный материал по панельным данным (обобщения линейных моделей и нелинейные модели, изложенные в главах 22 и 23).

	
	
В Университете штата Индиана, Блумингтон, 15-ти недельный курс микроэконометрики основан на материале, представленном в частях 4 и 5. Курсы, необходимые для начала  освоения 15-ти недельного курса, покрывают материал части 2.

Упражнения для самоподготовки представлены в конце каждой главы после первых трех вводных главах. Некоторые из них являются чисто методологическими, тогда как другие связаны с анализом фактических данных. Уровень сложности вопроса в основном связан с уровнем сложности темы.
	
\section{Программное обеспечение}

Существует огромное число пакетов программного обеспечения для анализа данных. Популярными пакетами с большими микроэконометрическими возможностями являются LIMDEP, SAS, и STATA, каждый из которых предлагает впечатляющий ассортимент встроенных процедур, а также допускает создание пользовательских процедур на матричном языке. Среди других широко используемых пакетов можно привести Eviews, PCGIVE и TSP,  они ориентированны на временные ряды, но несмотря на это  также могут использоваться для панельных данных и пространственных выборок. Пользователи, которые хотят писать собственные программы для анализа могут воспользоваться такими программами, как GAUSS, MATLAB, OX и SAS/IML. Свежую информацию об этих пакетах можно легко найти в Интернете.

\section{Обозначения и соглашения}

В данной книге весь анализ строится с применением векторов и матриц.
Векторы представлены в виде столбцов и обозначаются строчными буквами. К примеру, для линейной регрессии вектор регрессоров $ x $ --- это столбец размерности $ K \times 1 $, где на $j$-ом месте находитс $ x_{j} $, а вектор параметров $\beta$ --- это столбец размерности $ K \times 1 $, где на $j$-ом месте находится $ \beta_{j} $:

\[
\underset{(K\times1)}{x} = \begin{bmatrix} x_{1} \\ \vdots \\ x_{k} \end{bmatrix} \qquad 
\underset{(K\times1)}{\beta} = \begin{bmatrix} \beta_{1} \\ \vdots \\ \beta_{k} \end{bmatrix}
\]
	
Поэтому линейная регрессионная модель $y=\beta_{1}x_{1}+\beta_{2}x_{2}+\ldots +\beta_{k}x_{k}+u$ представляется в виде $y=x'\beta+u$. Иногда индекс $i$ добавляется для обозначения $i$-го наблюдения. Линейная регресси для $i$-го наблюдения записывается в виде:

\[
y_{i}=x_{i}'\beta+u_{i}
\]

В выборке обычно $N$ наблюдений, ${(y_{i}, x_{i}), i=1,\ldots ,N}$, а наблюдения обычно независимы по $i$.

Матрицы обозначаются заглавными буквами. В матричных обозначения выборка записывается в виде $(y,X)$, где $y$ вектор размера $ N \times 1 $ с $y_{i}$ на $i$-ом месте, а $X$ --- это матрица со строками $x_{i}'$:

\[
\underset{(N\times1)}{y} = \begin{bmatrix} y_{1} \\ \vdots \\ y_{N} \end{bmatrix} \qquad 
\underset{(N\times dim(x))}{X} = \begin{bmatrix} x_{1}' \\ \vdots \\ x_{N}' \end{bmatrix}
\]

В этом случае линейная регрессионная модель для всех наблюдений выглядит так:

\[
y=X\beta+u
\]
,
где $u$ вектор-столбец размера $ N \times 1 $ с $u_{i}$ на $i$-ом месте.

Матричная запись компактна, но иногда удобнее вместо произведения матриц записывать сумму произведений векторов. К примеру,  МНК-оценка может быть записана как:

\[
\hat{\beta}=(X'X)^{-1}X'y=\left(\sum_{i=1}^N x_i x'_i\right)^{-1}\sum_{i=1}^N x_i y_i.
\]
	
\begin{table}[h]
\caption{Часто используемые сокращения}
\label{tab:abbrev}
\begin{tabular}{@{}lll@{}}
\toprule
 \multirow{6}{*}{Линейные} & МНК & Метод наименьших квадратов  \\
 & ОМНК & Обобщенный метод наименьших квадратов \\
 & ДОМНК & Доступный обобщенный метод наименьших квадратов \\
 & - & Метод инструментальных переменных \\
 & 2МНК & Двухшаговый метод наименьших квадратов \\
 & 3МНК & Трехшаговый метод наименьших квадратов \\ \midrule
 \multirow{5}{*}{Нелинейные} & НМНК & Нелинейный метод наименьших квадратов \\
 & ДОНМНК & Доступный обобщенный нелинейный метод наименьших квадратов \\
 & - & Нелинейный метод инструментальных переменных \\
 & Н2МНК & Нелинейный двухшаговый метод наименьших квадратов \\
 & Н3МНК & Нелинейный трехшаговый метод наименьших квадратов \\ \midrule
 \multirow{4}{*}{Общие} & ММП & Метод максимального правдоподобия \\
 & - & Метод квази-максимального правдоподобия \\
 & ОММ & Обобщенный метод моментов \\
 & - & Обобщенный метод оценивающих уравнений \\ \bottomrule
\end{tabular}
\end{table}

Неизвестные параметры как правило обозначаются вектором $\theta$ размера $q\times 1$.  Параметры регрессии представлены в виде   вектора $ \beta $ размера $ K \times 1 $. Эти два вектора могут совпадать, или параметры регрессии могут быть частью вектора $\theta$ в зависимости от контекста.


В книге используются много сокращений и аббревиатур. В таблица 1.3 представлены все сокращения, используемые для некоторых распространенных методов оценки  линейных или нелинейных моделей регрессии. Мы часто используем следующие понятия: процесс порождающий данные (dgp, data generating process), одинаково распределенные и независимые (iid, independently и identically distributed), функция плотности (pdf, probability density function), функция распределения (cdf, cumulative distribution function), функция правдоподобия (L, likelihood), логарифмическая функция правдоподобия ($\ln L$, loglikelihood), фиксированные эффекты (FE, fixed effects), и случайные эффекты (RE, rиom effects)\footnote{В русском переводе количество используемых сокращений уменьшено}.












\part{Предварительные сведения}

	Часть 1 охватывает основные компоненты микроэконометрического анализа --- экономическую спецификацию, статистическую модель и наборы данных. В главе 1 рассматриваются отличительные аспекты микроэконометрики и описывается структура книги.  Глава подчеркивает, что дискретность данных, нелинейность и неоднородность поведенческих отношений являются ключевыми аспектами микроэконометрических моделей для индивидов. В конце главы обсуждаются обозначения и соглашения, используемые далее в книге.
Главы 2 и 3 нацелены на то, чтобы познакомить читателя с ключевыми моделями и данными, которые анализируются в последующих главах.


	Ключевое различие в эконометрике находится между по существу описательными статистическими моделями или обобщением данных разных уровней сложности и моделями выходящими за рамки описания взаимосвязей и пытающимися оценить параметры причинно-следственных связей. Классическое определение причинности в эконометрике берет свое начало из систем одновременных уравнений Комиссии Коулса, которые проводят четкое различие между экзогенными и эндогенными переменными, а также между параметрами структурной и приведенной форм модели. Хотя приведенные формы моделей являются очень полезными для некоторых целей, знание структурных параметров имеет важное значение для анализа. Определение структурных параметров в рамках одновременных уравнений создает многочисленные концептуальные и практические сложности. Широко используемый альтернативный подход на основе модели потенциальных результатов также пытается выявить параметры причинно---следственных связей, но он делает это ставя более ограниченные вопросы в более удобных рамках. В главае 2 приведен обзор фундаментальных вопросов, которые возникают в рамках этих и других альтернативных концепций. Читатели, у которых сейчас сложилось впечатление, что материал данной главы сложный, должны вернуться к ней после знакомства с конкретными моделями рассмотреными далее в этой книге.
	
	
	Способность исследователей правильно выявить причинно---следственную связь зависит не только от статистических инструментов и моделей, но и от типа данных. Экспериментальные данные являются стандартом для установления причинно---следственных связей. Тем не менее, не экспериментальные, а описательные данные лежат в основе большинства эконометрических выводов. В Главе 3 рассмотрены  плюсы и минусы трёх основных типов данных: описательных данных, данных из социальных экспериментов и данных из естественных экспериментов; рассматриваются cильные и слабые стороны проведения причинного---следственного вывода в каждом случае.

\chapter{Обзор}

\section{Введение}

	Эта книга представляет собой подробное рассмотрение микроэконометрического анализа данных на индивидуальном уровне экономического поведения отдельных лиц или фирм. Более широкое определение будет также включать сгруппированные данные. Обычно регрессионный метод применяются пространственным (cross---section) или панельным данным.
	
	
	Анализ индивидуальных данных имеет долгую историю. Ernst Engel (1857) был одним из первых исследователей  бюджетов домашних хозяйств. Allen and Bowley(1935), Houthakker (1957), и Prais и Houthakker (1955) внесли важный вклад следуя в том же направлении, используя те же подходы. Другие исследования также оказали значительное влияние на стимулирование развития микроэконометрики, однако они не всегда использовали информацию на индивидуальном уровне, такие как, Marschak and Andrews (1944) в теории производства и Wold and Jureen  (1953), Stone (1953 ) и Tobin (1958) в потребительском спросе.
	
	
	Материал описанный в этой книге также связан с работой по дискретному анализу выбора и цензуре и усеченных переменных модели, серьёзное эконометрическое приложения которых было в работе McFadden (1973 , 1984) и Heckman (1974, 1979). Эти работы отходят от использования линейных моделей, которые часто встречаются в более ранних работах . Впоследствии они привели к значительным методологическим инновациям в эконометрики. Среди ранних учебников, в которых освещён этот материал (и более) являются работы Maddala (1983) и Amemiya (1985). Как отмечали Heckman (2001), McFadden (2001 ) и др. , многие из фундаментальных вопросов, которые доминировали в ранних работах, основывались на рыночных данных и остаются важными , особенно в отношении условий, необходимых для идентифицируемости причинно---следственных экономических отношений.
	
	
	Современные микроэконометрики основанные на индивидуальных, бытовых, а также созданных уровней данных многим обязаны большой доступности от одномометной выборки и продольных выборочных обследований до данных переписи. В последние два десятилетия, с расширением электронной регистрации и сбора данных на индивидуальном уровне, объем данных растёт экспоненциально. Так же существуют достаточные вычислительные мощности для анализа больших и сложных наборов данных. Во многих случаях данные событийного уровня доступны, например, маркетинг часто имеет дело с данными покупок, собранных при помощи электронных сканеров в супермаркетах, и литература по теории отраслевых рынков содержит эконометрический анализ основанный на данных авиаперевозок, собранных системами онлайн---бронирования билетов. В настоящее время появляются новые отрасли экономики, такие как экспериментальная экономика, которые генерируют «экспериментальные» данные, что позволило создать новые возможности моделирования, которое отсутствует, в случае доступности только данных с рынков. Между тем стремительный рост объемов и типов данных также дал начало многочисленным методологическим вопросам. Обработка и эконометрический анализ таких крупных микроданных, с целью выявления моделей экономического поведения, составляет ядро микроэконометрики. Эконометрический анализ таких данных является главным предметом этой книги.
	
	
	Основные предшественники этой книге, являются книги Maddala (1983) и Amemiya (1985). Подобно им он охватывает темы, которые представлены лишь кратко, или нет вообще, для студентов и магистрантов первого курса эконометрики. Особенно по сравнению с Amemiya (1985) эта книга больше ориентирована на практическое применение.
	
	
	Относительно расширенная презентация тем необходима по нескольким причинам. Во---первых, данные часто являются дискретными или незначительными, и в этом случае используются нелинейные методы, такие как логит, пробит, и Tobit модели. Это приводит к статистическому анализу, основанному на более сложной асимптотической теории.
	
	
	Во---вторых, количество включенных предположений для анализа таких данных становится крайне важным. Возможным решением является разработка высоко параметрических моделей, которые являются достаточно подробными, чтобы охватить всю сложность данных, но эти модели сложно оценить. Более распространенным решением является минимизация параметрических предположениях и предоставление статистических выводов на основе стандартных ошибок, которые являются <<устойчивыми>> к таким проблемам, как гетероскедастичность и кластеризации. В таких случаях значительные знания могут быть необходимых для получения достоверных статистических выводов, даже если используется стандартный пакет регрессии.
	
	
	В---третьих, экономические исследования часто ставят целью определение причинности, а не просто измерение корреляции, несмотря на доступ к фактическим данным, а не экспериментальным. Это приводит к методам, которые способны выявить причинность, такие как инструментальные переменные, система одновременных уравнений, коррекция погрешности измерений, коррекция смещения отбора, панельные данные с фиксированными эффектами, и дифференциальные---различия.
	
	
	В---четвертых, микроэкономические данные, как правило, собираются с использованием одномоментной выборки и панельных обследований, переписей или социальных экспериментах. Обзор данных, собранных с помощью этих методов, могут иметь проблемы комплексной методологии обследования: отклонение от предположения случайности выборки, и проблемам выборки, ошибкам измерения, и неполным, и / или недостающим данным. Для оценивания таких эконометрических моделей необходимо использование передовых методов.
	
	
	Наконец, это не является необычным, что два или более осложнений происходят одновременно, например, эндогенность в логит модели с панельными данными, в этот момент очень трудно реализовать стандартные подходы. Вместо этого, необходимо значительное понимание теории, лежащей в основе методов.


\section{Отличительные аспекты микроэконометрики}

	Сейчас мы рассмотрим несколько преимуществ микроэконометрики, которые вытекают из её отличительных особенностей.

\subsection{Дискретность и нелинейность}

	Первым и наиболее очевидным является то, что микроэконометрические данные, как правило, на низком уровне агрегирования. Это имеет последствия для основных функциональных форм, которые используются для анализа переменных, представляющих интерес. Во многих, если не в большинстве случаев линейные функциональные формы оказываются просто неуместными. Более существенно то, что разукрупнение выдвигает на первый план гетерогенность индивидуальных лиц, фирм и организаций, которая должны надлежащим образом контролироваться, если необходимо сделать значимые выводы о лежащих в основе отношениях. Мы обсудим эти вопросы более подробно в следующих разделах.
	
	
	Конечно агрегация не совсем отсутствуют в микроданных, например, когда собираются данные по домохозяйствам или учреждениям, однако уровень агрегации, как правило, порядком ниже, чем это принято в макро анализов. В последнем случае процесс агрегации приводит к сглаживанию. Агрегированные переменные часто показывают гладкое поведение, чем их компоненты, более того отношения между агрегатами также часто показывают большую гладкость, чем компоненты. Например, отношение между двумя переменными на микроуровне может быть кусочно---линейной со многими узлами. После агрегации, отношение будет хорошо аппроксимироваться гладкой функцией. Поэтому, следствием разукрупнения данных является отсутствие черты преемственности и гладкости как самих переменных, так и взаимосвязей между ними.
	
	
	Обычно индивидуальные данные и данные фирм охватывают огромный диапазон изменений, как в одномоментной выборке так и временных рядах. Например, среднее еженедельное потребление (скажем) говядины, очень вероятно будет положительным и плавно меняющимся, тогда как потребление отдельного домохозяйства в данной недели часто равно нулю и может переходить к положительным значениям время от времени. Среднее число часов, отработанных работником женского пола вряд ли будет нулевым, но если рассматривать женщин по отдельности то многие из них имеют ноль часов работы (угловое решение), переходя на положительные значения в другое время в ходе своего трудового стажа. Средние расходы на отпуск, как правило, положительные, но многие домохозяйства могут иметь нулевое значение расходов на отдых за конкретный год. Среднедушевое потребление табачных изделий, как правило, положительное, но многие индивиды из генеральной совокупности никогда не потребляли сигареты и никогда не будут, независимо от цены и дохода. Из наблюдений Pudney (1989), микро---данные чаще всего имеют пустые значения (<<дыры, перегибы и углы.>>). Дыры соответствуют неучастия в активных действиях, перегибы соответствуют характеру переключения, а углы соответствуют частоте непотребления или неучастия в определенный момент времен. То есть, дискретность и нелинейности значений черта микроэконометрики.
	
	
	Важный класс нелинейных моделей в микроэконометрики имеет дело с ограниченными зависимыми переменными (Maddala, 1983). Этот класс включает в себя множество моделей, которые обеспечивают подходящие рамки для анализа дискретных значение и значений с ограниченным диапазоном изменения. Такие инструменты анализа, конечно, также доступны для анализа макро---данных, если это необходимо. Дело в том, что они незаменимы в микроэконометрики и являются её отличительной особенностью.

\subsection{Большая реалистичность}

	Макроэкономика основана на сильных предположениях; таких как предположение о репрезентативном агенте. Часто микроэкономические рассуждения, оправдывают определенные технические характеристики и интерпретации эмпирических результатов. Однако, чаще всего нельзя сказать, как это влияет на агрегацию по времени и микро единиц. Проблема в том, что считается, что агрегирование, отражают поведение гипотетического репрезентативного агента, что однозначно не вызывают доверия.
	
	
	С точки зрения микроэкономической теории, количественный анализ основанный на микроданных можно рассматривать как более реалистичным, чем то, что основано на агрегированных данных. Есть три обоснования этого утверждения. Во---первых, измерение переменных, используемых в таких гипотез часто более прямое (хотя и не обязательно свободно от ошибки измерения) и имеет большое соответствие с теорией, которое тестируется. Во---вторых, гипотезы об экономическом поведении, как правило, вытекают из теорий индивидуального поведения. Если эти гипотезы проверяются с использованием агрегированных данных, то необходимо вводить большое количество предположений и упрощений. Упрощающее предположение о репрезентативном агенте вызывает большую потерю информации и серьезно ограничивает сферу эмпирического исследования. К счастью, такие предположения можно избежать в микроэконометрике, и, как правило, в  микроданными обеспечивают более реальную основу для тестирования микроэкономических гипотез. Наконец, реалистичное изображение экономической деятельности должно обеспечивать широкий диапазон результатов, и значений, которые являются следствием индивидуальной неоднородности и, которые вытекают из теории. В этом смысле микроэкономические данные могут предоставлять более реалистичные модели.
	
	
	Микроэконометрические данные часто получают от домашних хозяйств или обследования фирм, обычно охватывает широкий диапазон поведения. Такие наборы данных имеют много особенностей, которые требуют специальных инструментов при моделировании и анализе.

\subsection{Большая содержательная информация}

	Потенциальные преимущества микроданных могут быть реализованы, если эти данные являются информативными. Потому что выборочные обследования часто предоставляют независимые наблюдения на тысячи единиц одномоментной выборки, такие данные являются более информативным, чем стандартные, которые часто высокого коррелированы, макроэкономические временные ряды, которые  обычно состоят из не более нескольких сотен наблюдений.
	
	
	Как будет показано в следующей главе, на практике ситуация не такая четкая, потому что микроданные могут иметь  шум. На индивидуальном уровне разнообразные факторы могут играть большую роль в определении влияния. Часто эти факторы нельзя обнаружить, поэтому люди относятся к ним как к случайным компонентам, которые могут быть большой частью наблюдаемых изменений. В этом смысле случайность играет большую роль в микроданных, чем в макроданных, более того, это влияет на измерение показателей адекватности регрессии. Студенты, чьи первоначальные знания эконометрики приходят из анализа агрегированных временных рядов чаще хотят увидеть большое значения $R^{2}$, но когда они встречаются с одномоментной выборкой в первый раз, то разочаровываются низкой объяснительной силой уравнения регрессии. Тем не менее, есть серьезные основания полагать, что по крайней мере в определенных размерах, большие наборы микроданных высоко информативны.
	
	
	Другой уточнение заключается в том, что, когда мы имеем дело с пространственными данными (одномоментная выборка), очень мало что можно сказать о межвременных отношениях в стадии изучения, этот  аспект поведения можно исследовать с помощью панельных и переходных данных.
	
	
	Во многих случаях исследователи заинтересованы в определении поведенческой реакции экономических агентов при некоторых указанных экономических условиях. Одним из примеров является влияние пособия по безработице на поведение молодых безработных при поиске работы. Другим примером является реакция на предложение труда лиц с низким доходом, которые получают материальную поддержку. 
	
\subsection{Микроэкономические основания}

	Эконометрические модели различаются, для разных экономических теорий, различны. С одной стороны, есть модели, в которой теоретизирование может играть главную роль в спецификации модели и в выборе процедуры оценки. На другой стороне, находятся эмпирические исследования, которые используют гораздо меньше экономической теории.
	
	
	Цель анализа в первом случае заключается в выявлении и оценки основных параметров модели, которые также называют глубокими параметрами, в силу того, что они характеризуют индивидуальный вкус и предпочтения и/или технологические отношения. Для краткости обозначения, мы называем это структурным подходом. Его отличительной чертой является сильная зависимость от экономической теории и упор на причинно---следственные выводы. Такие модели могут требовать много предположений, таких, как точные спецификации функции затрат или производственной функции. Применение эмпирических выводов в таком случае может быть не надежным в случае отклонения от предположений. В разделе 2.4.4 мы будем говорить об этом подходе. На данном этапе мы просто подчеркнем, что , если структурный подход реализуется с агрегированным данным, то оценки основных параметров будут получены только при очень жестких ( и даже нереальных) условиях. Однако, микроданные больше уместны для структурного подхода, поскольку они обеспечивают большую гибкость при построении модели.
	
	
	Цель анализа во втором случае является модель отношений между объясняющими переменными, представляющие интерес для исследователя поскольку должны быть экзогенными. Более формальные определения эндогенности и экзогенности приведены в главе 2. Для краткости обозначения, мы называем это reduced form approach. Важно отметить, что такой подход не всегда учитывают все причинно---следственные взаимозависимости. Регрессионная модель, в которой акцент делается на предсказание $y$ при данных регрессорах $x$, а не на причинной интерпретации параметров регрессии, часто называют reduced form regression. Как будет показано в главе 2, в целом параметры reduced form regression являются функциями структурных параметров, поэтому они не могут быть интерпретированы без некоторой информацией о структурных параметров.



\subsection{Разукрупнение и гетерогенность}

	Считается, что многие проблемы в макроэконометрике возникают из---за серийной корреляции макроэкономических временных рядов, а в микроэконометрики проблемы берут своё начало из---за гетероскедастичности на индивидуальном уровне данных. Более того она является полезной характеристикой для моделирования данных в микроэконометрическом анализе, поэтому её усиление даст ряд полезных свойств. В  микроэконометрических моделях, моделирование динамических зависимость может быть важным вопросом.
	
	
	Преимущества разукрупнения, которые были выделены ранее в этом разделе, имеют свою издержки: в тот момент, когда данные становятся более детализированными важность контроля неоднородности между индивидами увеличивается. Гетерогенность, а точнее не наблюдаемая гетерогенность, играет очень важную роль в микроэконометрики. Очевидно, что многие переменные, которые отражают неоднородность, такие как пол, раса, образование и социально---демографические факторы, являются непосредственно наблюдаемым и как следствие, могут быть контролируемы. Напротив, различия в индивидуальной мотивации, способностях, интеллекте и т.д. либо не наблюдается, либо, в лучшем случае, частично наблюдается.
	
	
	Самый простой выход из ситуации, это игнорирование гетерогенности, то есть включение её в ошибки регрессии. Этот шаг, конечно, увеличивает не объясняемую часть вариации. Более детально, игнорирование межличностных различий приводит к смешиванию с другими факторами, которые также являются источниками межличностных различий. Смешивание происходит тогда, когда вклад индивида в различные регрессоры с изменением переменной не может быть статистически разделен. Предположим, например , что фактор $x_{1}$ (образование), может быть источником изменения $y$(доход) , когда другая переменная $x_{2}$ (способность), которая является еще одним источником изменения дохода, не включена в модели. Тогда та часть полной вариации, которая приходится на вторую переменную может быть неправильно отнесена к первой переменной. Интуитивно, их относительный вклад в изменение дохода смешан. Ведущим источником смешанных факторов является неправильное исключение регрессоров из модели и включения других переменных, которые являются прокси для выкинутых переменных.
	
	
	Рассмотрим случай, в котором дамми переменная $D$ (0/1) входит в линейную регрессию с вектором регрессоров $x$.
	
\begin{equation}
y=X'\beta+D\alpha+u
\end{equation}


где $u$ ошибки регрессии. Термин <<лечение>> используется в биологической и экспериментальных науках. В эконометрике это обычно относится к какой---либо деятельности, которое может повлиять на выводы модели. Эта деятельность может быть случайно поставлена перед участниками или может быть самостоятельно выбрана участником. Таким образом, хотя следует признать, что люди выбирают сколько лет им обучаться в школе, некоторые считают, что продолжительность обучения является переменной «лечения». Предположим, что участие в программе принимается дискретное значение, тогда коэффициент $\alpha$ измеряет среднее воздействие участия в программе $(D = 1)$, условная ковариация. Если не контролировать гетерогенность, то потенциальная неоднозначность влияет на интерпретацию результатов. Если считается, что $d$ имеет значительное влияние, то возникает следующие вопросы: является ли $\alpha$ отличным от нуля, потому что $D$ коррелирует с некоторой ненаблюдаемой переменной, которая влияет на $y$ или потому, что существует причинно---следственная связь между $D$ и $y$? Например, если модель рассматривает университетское образование, и в ковариацию не включаются способность индивида, причинно---следственная интерпретация становится сомнительным. Так как проблема не наблюдаемой гетерогенности является важной, большое внимание следует уделять тому, как её контролировать. 


	В некоторых случаях, когда необходимо рассмотреть динамику накладываются ограничения навозможность контроля неоднородности. Рассмотрим пример с двумя домашними хозяйствами, одинаковых во всех отношениях за исключением того, что одна более склонна потреблять товар A. Некоторые могут контролировать гетерогенность, включая в  индивидуальные функции полезности параметр неоднородности, которая отражает различия в предпочтениях . Предположим теперь, что существует теория потребительского поведения, которая утверждает , что потребители становятся зависимыми от товара А, в том смысле, что чем больше они потребляют его в один период, тем больше вероятность того, что они будут потреблять его больше в будущем. Эта теория дает другое объяснение индивидуальных различий в потреблении товара А. Контролируя гетерогенные предпочтения становится возможным проверить, какой источник в потреблении --- неоднородность предпочтений или зависимость --- учитывается в разных моделях потребления. Этот тип проблемы возникает всякий раз, когда некоторые динамические элементы генерирует стойкость в наблюдаемых результатов. Несколько примеров такого рода проблем разобраны в различных главах этой книги.
	
	
	Различные подходы для моделирования гетерогенности сосуществуют в микроэконометрики, краткое упоминание о некоторых из них не следует откладывать на потом.
	
	
	Крайним решением будет игнорировать все меж индивидуальные различия. Если не наблюдаемая неоднородность не коррелирует с наблюдаемой неоднородностью, и если результат не имеет межвременной зависимости, то вышеуказанные проблемы не возникнут. Конечно, эти  предположения являются очень сильными, однако даже с учетом этих предположений не все эконометрические трудности исчезают.
	
	
	Один из способов обработки гетерогенности является использование фиксированных эффектов и оценки коэффициентов дамми переменных. Например, в регрессии одномоментной выборки, каждая микро единица учитывает свою фиктивную переменную. Это приводит к увеличению параметров, потому что, когда добавляется новый индивид сразу же появляется новый свободный член, поэтому, данный подход не будет работать, если наши данные являются одномоментной выборкой. Наличие нескольких наблюдений для отдельного индивида, чаще всего представлена в виде панельных данных с T временными рядами наблюдений для каждого из N индивидом одномоментной выборки, это позволяет либо оценить, либо устранить фиксированный эффект, например,  при первом приближении, если модель является линейной с фиксированными эффектами. Если модель является нелинейной, как это часто бывает, то фиксированный эффект не может быть добавлен, поэтому должны быть рассмотрены другие подходы.
	
	
	Второй подход заключается в использовании моделей со случайными эффектами. Есть несколько способов, в которых модель со случайными эффектами может быть сформулирована. Одна из популярных формулировок предполагает, что один или более параметров регрессии, чаще всего свободный член, изменяется случайным образом по всей одномоментной выборке сечению. В другой формулировке ошибки регрессии являются компонентной структурой, с конкретной индивидуальной  случайной составляющей. Далее модель со случайными эффектами пытается оценить параметры распределения случайной составляющей. В некоторых случаях, таких как анализ спроса, случайный член может быть интерпретирован как случайное изменение предпочтений. Модель со случайными эффектами можно оценить, используя как данные одномоментной выборки, так и панельные данные.
	

\subsection{Динамика}

	Очень распространенным предположением в анализе данных одномоментной выборки является отсутствие межвременной зависимости, то есть, отсутствие динамики. Таким образом, неявно предполагается, что наблюдения соответствуют стохастическому равновесию с отклонением от равновесния, которая может быть представлена в распределении независимых случайных ошибок. Даже в микроэконометрики для некоторых данных такое предположение может быть слишком сильным. Например, она не согласуется с наличием автокорреляции ненаблюдаемой гетерогенности.
	Вышеизложенное иллюстрирует некоторые из потенциальных ограничений анализа одномоментной выборки. Некоторые ограничения могут быть преодолены, если данная выборка может повторятся во времени. Тем не менее, если есть динамические зависимости, наименее проблемным подходом является использование панельных данных.

\section{Структура книги}

	Книга разделена на шесть частей. В части 1 представлены вопросы, связанные с микроэконометрическим моделированием. В части 2 и 3 мы познакомимся с общей теорией оценки и статистические выводов для нелинейных моделей регрессии. Части 4 и 5 специализируются на основных моделях, используемых в прикладной микроэконометрики для анализа одномоментной выборки и панельных данных. Часть 6 охватывает более широкие темы, которые делают более гибким использование материала, представленного в предыдущих главах.
	Структура книги представлена в таблице 1.1. Далее в этом разделе будет сказано про каждую часть более подробно.
	
\subsection{Часть 1. Подготовка}

	В главах 2 и 3 рассмотрены более широкие особенности микроэконометрического подхода к моделированию и микроэкономических структур данных в рамках более общего статистического анализа регрессий. Многие из вопросов, поднятых в этих главах, обсуждаются на протяжении всей книги, что позволяет читателю расширить свой инструментарий.

\begin{table}[h]
\begin{center}
\caption{\label{tab:pred}Структура книги}
\begin{tabular}[t]{llcll|}
\hline
\hline
\end{tabular}
\end{center}
\end{table}

\subsection{Часть 2. Основные подходы}

	Главы 4---10 подробно описывают общие методы, используемые в классической оценки моделей и статистических выводах. Результаты, приведенные в главе 5, в частности, широко используются на протяжении всей книги.
	
	
	Глава 4 предоставляет некоторые результаты для линейной регрессионной модели, акцентируя своё внимание на тех вопросах и методах, которые наиболее актуальны для остальной части книги. Анализ относительно прост, поскольку есть явный подход для оценки линейных моделей, такой как метод наименьших квадратов.
	В главах 5 и 6 обсуждается способы оценивания, которые могут быть применены к нелинейным моделям, для которых обычно нет явного решения. Асимптотическая теория используется для получения распределения оценок, с акцентом на получение стойких стандартных оценок ошибок, которые полагаются на относительно слабых предположений о распределении. Достаточно общие способы, наряду с  методом наименьших квадратов применимых к нелинейным моделям и методом максимального правдоподобия, представлены в главе 5. Более сложные, обобщенный метод моментов и метод инструментальных переменных, приведены в главе 6.
	
	
	Глава 7 описывает классические способы проверки гипотез, когда оценки являются нелинейными или, когда гипотеза тестируется на моделях, являющихся нелинейными по параметрам. Спецификация тестов в дополнение к проверки гипотез обсуждается в главе 8.
	
	
	В главе 9 представлены полупараметрические методы оценки, такие как kernel regression, ярким примером которого является гибкое моделирование условного математического ожидания. К примеру, непараметрической регрессионной моделью является $E[y|x] = g(x)$, где функция $g(\cdot)$ не определена и должна быть оценена, более того данная оценка имеет бесконечномерный компонент $g(\cdot)$, ведущий к нестандартной асимптотической теории. При наличии дополнительных регрессоров некоторые дополнительные структуры необходимы, такие методы называются полупараметрическими или полунепараметрическими.
	
	
	Глава 10 предоставляет способы расчета оценок параметров, когда оценки определяются неявно, как правило, в качестве решения иногда используется условие первого порядка.

\subsection{Часть 3. Моделирование}
	
	Главы 11---13 рассматривают методы оценки и статистические выводы моделей, основанных на моделировании. Эти методы обычно более сложны в расчётах и в настоящее время меньше используются,чем методы, представленные в части 2.
	
	
	Глава 11 представляет bootstrap method статистического вывода. Это дает эмпирическое распределение оценки путем получения новой выборки с помощью моделирования. Bootstrap может обеспечить простой способ получения стандартной ошибки, когда формулы из асимптотической теории являются сложными, в случае двухступенчатых оценок. Кроме того, если это осуществить, то bootstrap может привести к улучшению статистических выводов в случае малого числа наблюдений.
	
	
	В главе 12 рассматриваются методы оценивания на основе моделирования моделей, для которых нет никакого аналитического решения. 
	
	
	Глава 13 представляет байесовский метод, который включает в себя распределение наблюдаемых данных с указаннием априорного распределения параметров для получения их апостериорного распределение, которое позволит правильно найти оценки. Последние достижения науки позволяют сделать вычисления даже в случае того, когда  делают нет аналитического представления для апостериорного распределения. Байесовский анализ обеспечивает подход к оценке и статистическим выводам, который довольно сильно отличается от классического подхода. Тем не менее, во многих случаях только байесовский набор инструментов позволяет решить проблемы, которые в противном случае являются неразрешимыми в классическом подходе.

\subsection{Часть 4. Модели для одномоментной выборки}

	Главы 14---20 показаны основные нелинейные модели для данных одномоментной выборки. Эта часть является важным элементом книги и представляет сложные темы, такие как модели для ограниченных зависимых переменных и выборки. Классы моделей определяется исходя из диапазона значений, которые принимает зависимая переменная.
	
	
	Модели в случае, когда зависимая переменная является бинарной, то есть принимает только два возможных значения, $y = 0$ или $y = 1$, представлены в главе 14. В Главе 15 акцентируется внимание на полиномиальных моделях для зависимой переменной, которая принимает несколько дискретных значений. Примерами являются, статус занятости (занятые, безработные, и входящие в рабочую силу) и используемый для поездки на работу транспорт (на автомобиле, автобусе или поезде). Линейные модели могут быть информативными, но не подходящими, так как они могут предсказать значение вероятности больше 1, вместо этого используется логит, пробит, и связанные с ними модели.
	
	
	В главе 16 представлены модели с censoring, truncation, sample selection. Примерами могут служить: количетсво часов работы в год, условие выбора работы, и расходы на больницу, условие госпитализации. В этих случаях данные не полностью наблюдаются, они сгуппированы $y = 0$ и оставшиеся $y > 0$. Модель для наблюдаемых данных является нелинейной, даже если основной процесс является линейным, всё равно линейная регрессия по данным наблюдениям может вводить в заблуждение. Простая корректировка censoring, truncation, sample selection может быть выполнена при использовании Tobit моделей, но они очень зависят от распределения предположений.
	
	
	Модели для duration data представлены в главах 17---19. Примером могут служить продолжительность безработицы. Стандартные модели регрессии включают экспоненциальный, Вейбулла, и модели пропорциональных рисков Кокса. Кроме того, как и в главе 16, зависимая переменная часто неполно наблюдается. 
	
	
	Глава 20 рассматривает  count data модели. Например,количество визитов к врачу или длительность госпитализации в днях. Опять же модель является нелинейной и, следовательно, условное среднее неотрицательны, в этом случае используются параметрические модели Пуассона и отрицательного биномиального.

\subsection{Часть 5. Модели для панельных данных}

Главы 21---23 рассматривают способы оценки панельных данных. В это случае данные наблюдаются в несколько периодов времени для каждого из индивидов в выборке, так зависимая переменная и регрессоры проиндексированы как по времени, так и по типу индивида. Любой анализ необходимо контролировать на случай положительной корреляции ошибок в разные периоды времени для конкретного индивида. Кроме того, панельных данных обычно достаточно, чтобы контролировать ненаблюдаемые стационарные индивидуальные специфические эффекты , что позволяет выявить причинность при более слабых предположениях, чем это бывает в одномоментной выборке.


	Основные линейные модели панельных данных представлены в главе 21, с акцентом на модели с фиксированным эффектом и случайным эффектом. Модели, которые позволяют предотвратить эндогенность регрессоров представлены в главе 22, а  нелинейные модели в главе 23.

\subsection{Часть 6. Последующие темы}

В этой части рассматриваются важные темы, которые могут в целом относится ко всем без исключения моделям рассмотренным в частях 4 и 5. Глава 24 посвящена моделированию кластерных данных при помощи различных моделей. В главе 25 обсуждается treatment evaluation. Treatment evaluation является общим термином, который может охватывать широкий спектр моделей, в которых основное внимание уделяется оценке влияния некоторых способов борьбы, которые либо экзогенны либо случайны для индивида и влияют на результаты регрессии. Глава 26 рассматривает последствия ошибок измерения результата и / или  регрессоров, с акцентом на некоторых ведущие нелинейные модели. Глава 27 рассматривает некоторые методы обработки недостающих данных в линейных и нелинейных моделях регрессии.

\section{Как пользоваться книгой}

Книга предполагает общее представление о модели линейной регрессии в матричной форме. Она написана на математическом уровне первого курса экономике кандидата в PhD, и по уровню сопоставина с  Greene (2003).


	Хотя некоторые из материала этой книги изучаются на первом году обучения по программе PhD, большинство из них появляются на втором курсе эконометрики PhD или в курсах микроэкономики ориентрированных на данные, такие как экономика труда, государственная экономика, или теория отраслевых рынков. В целом, книга представляет с собой справочник для прикладных исследователей в области экономики, в смежных социальных наук, таких как социология, политология и эпидемиологии.
	
	
	Для читателей, использующих эту книгу в качестве справочника, главы в которых представлены модели являются независимыми от остального содержания книги, насколько это возможно. В частях 4 и 5, где представлены более конкретные модели, достаточно прочитать несколько необходимых глав, в частности главу 5 и 6, чтобы понять всё содержимое и не вдаваться в подробности. Большинство глав структурированы таким образом, чтобы любой тип аудитории готов был начать дискуссию и пример.
	
\begin{table}[h]
\begin{center}
\caption{\label{tab:pred}План 20---ти лекций 10---ти недельного курса}
\begin{tabular}[t]{llcll|}
\hline
\bf{Лекции} & \bf{Главы} & \bf{Темы} \\
\hline
1---3 & 4, Прилож. А & Описание линейной модели и асимптотической теории \\
4---7 & 5 & Оценивание: m-оценивание, ML, и NLS\\
8 & 10 & Оценивание: численная оптимизация \\
9---11 & 14,15 & Модели: бинарные и полиномиальные \\
12---14 & 16 & Модели: censored and truncated \\
15 & 6 & Оценивание: GMM \\
16 & 7 & Тестирование: гипотезы \\
17---19 & 21 & Модели: линейные панели \\
20 & 9 & Оценивание: полупараметрическое \\
\hline
\end{tabular}
\end{center}
\end{table}
	
	Для тех, кто использует эту книгу в качестве основного ресурса проведения курса, стоит ознакомить студентов как можно раньше с нелинейными для одномоментной выборки и линейными моделями для панельных данных, пропуская другие способы оценивания. Первый тип моделей представлен в главах 14---16, они требуют знания метода максимального правдоподобия и метода наименьших квадратов, что представлено в главе 5. Глава 21 рассматривает линейные модели оценивания панельных данных, для подготовки необходимо прочитать главу 4.
	
	
	В таблице 1.2 представлен план одной четверти второго года аспирантуры в Калифорнийском Университете в Дэвисе, сразу же после первого курса статистики и эконометрики. Данный курс покрывает основные результаты, приведенные в первой половине книги. 
	
	
	В университете штата Индиана, Блумингтон, 15---ти недельный курс микроэконометрики основан на материале, представленный в частях 4 и 5, а в части 2 представлен необходимый материал для того, чтобы успешно начать курс.
	Упражнения для самоподготовки представлены в конце каждой главы, после первых трех вводных главах. Некоторые из них являются чисто методологическими, тогда как другие влекут за собой анализ фактических данные, к тому же уровень сложности вопроса, в основном связан с уровнем сложности темы.
	
\section{Программное обеспечение}

Существует огромное число пакетов программного обеспечения для анализа данных. Популярные пакеты с большими микроэконометрическими возможностями являются LIMDEP, SAS, и STATA, каждый из которых предлагает впечатляющий ассортимент встроенных процедур. Другие пакеты, которые также широко используются Eviews, PCGIVE и TSP, но несмотря на то, что они ориентированны на временные ряды, они также могут поддерживать панельные данные и одномоментную выборку. Пользователи, которые хотят писать собственные программы для анализа могут воспользоваться такими программами, как GAUSS, MATLAB, OX и SAS/IML. Всю необходимую информацию об установке и использованию данных пакетов можно найти на официальном сайте разработчиков.

\section{Обозначение и соглашение}

В данной книге весь анализ строится с применением векторов и матриц.
Векторы представлены в виде столбцов. К примеру, для линейной регрессии вектор регрессоров $ x $ -- это столбец размерности $ K \times 1 $ с $ x_{j} $, а вектор параметров $\beta$ это столбец размерности $ K \times 1 $ с $ \beta_{j} $:

\[
\underset{(K\times1)}{x} = \begin{bmatrix} x_{1} \\.\\ x_{k} \end{bmatrix} \qquad 
\underset{(K\times1)}{\beta} = \begin{bmatrix} \beta_{1} \\.\\ \beta_{k} \end{bmatrix}
\]
	
Далее линейная регрессионная модель $y=\beta_{1}x_{1}+\beta_{2}x_{2}+...+\beta_{k}x_{k}+u$ представляется в виде $y=x'\beta+u$, а для $i$-го наблюдения в виде:

\[
y_{i}=x_{i}'\beta+u_{i}
\]

Выборка для одного из $N$ наблюдений, ${(y_{i}, x_{i}), i=1,...,N}$, где наблюдения независимы по $i$.
В матричных обозначения выборка представляется в виде $(y,X)$, где $y$ вектор $ N \times 1 $ с $y_{i}$, а $X$ матрица со строками $x_{i}'$:

\[
\underset{(N\times1)}{y} = \begin{bmatrix} y_{1} \\.\\ y_{N} \end{bmatrix} \qquad 
\underset{(N\times dim(x))}{X} = \begin{bmatrix} x_{1}' \\.\\ x_{N}' \end{bmatrix}
\]

В этом случае линейная регрессионная модель для всех наблюдений выглядит так:

\[
y=X\beta+u
\]
,
где $u$ вектор---столбец $ N \times 1 $ с $u_{i}$

Матричная запись компактна, но иногда удобнее записать произведение матриц, как сумму произведений векторов. К примеру, оценка методом МНК может быть записана как:

\[
\hat{\beta}=(X'X)^{-1}X'y=(\sum_{i=1}^N x_i x'_i)^{-1}\sum_{i=1}^N x_i y_i.
\]
	
\begin{table}[h]
\begin{center}
\caption{\label{tab:pred}Часто используемые аббревиатуры}
\begin{tabular}[t]{llcll|}
\hline

\hline
\end{tabular}
\end{center}
\end{table}	


Параметры регрессии представлены в виде $ K \times 1 $  вектора $ \beta $


В книге используются много сокращений и аббревиатур. Таблица 1.3 демонстрирует все сокращения, используемые для некоторых распространенных методов оценки для линейных или нелинейных моделей регрессии. Мы также используем следующее обозначения: dgp (data generating process), iid (independently and identically distributed), pdf (probability density function), cdf (cumulative distribution function), L (likelihood), ln L (loglikelihood), FE (fixed effects), and RE (random effects).










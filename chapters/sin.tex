accelerated failure time (AFT) model, - модель ускоренной жизни (AFT модель)
coefficient interpretation, - интерпретация коэффициентов
definition, - определение
leading examples, - основные примеры
accept-reject methods, - методы принятия-отбрасывания
ACD. See average completed duration acronyms, - ACD. См. средняя длительность завершенного состояния
AD estimator. See average derivative adaptive estimator, - AD оценка, оценка средней производной. См. адаптивная оценка средней производной
adding-up constraints, - балансовые ограничения
additive model, - аддитивная модель
additive random utility model (ARUM) - аддитивная модель случайной полезности (ARUM)
binary outcome models, - модели бинарного выбора
generalized random utility models, - обобщенные модели случайной полезности
identification, - идентифицируемость
multinomial outcome models, - мультиномиальные модели
nested logit model, - вложенная логит-модель
RPL model, - логит-модель со случайными параметрами
welfare analysis in, - в анализе благосотояния
admissible estimator, - допустимая оценка
AFT. See accelerated failure time - AFT. См. модель ускоренной жизни
aggregated data - агрегированные данные
binary outcomes, - бинарные исходы
cohort-level, - на уровне когорт
nonlinear models, - нелинейные модели
multinomial outcomes, - мультиномиальные исходы
time-aggregated durations, - агрегированные по времени данные о длительности
see also discrete-time duration data - см. также модели времени жизни в дискретно времени
AIC. See Akaike information criterion - AIC. См. Акаике информационный критерий
AID. See average interrupted duration - AID. См. средняя прекращенная длительность
Akaike information criterion (AIC), - Акаике информационный критерий
almost sure convergence, - сходимость почти наверное
analog estimator, - оценка по принципу аналогии
analogy principle, - принцип аналогии
and method of moments estimators, - и оценки метода моментов
analysis of covariance, - анализ ковариации, ковариационный анализ
analysis of variance, - анализ дисперсии, дисперсионный анализ
Anscombe residual, - Анскомба остатки
antithetic sampling, - антитетическое сэмплирование
applications with data - приложения к данных
competing risks models, - модели конкурирующих рисков
duration models, - модели времени жизни
IV estimation, - оценивание с помощью инструментальных переменных
kernel regression, - ядерная регрессия
logit and probit models, - логит и пробит модели
multinomial and nested logit models, - мультиномиальные и вложенные логит модели
Poisson and negative binomial models, - модели Пуассона и отрицательные биномиальные
panel fixed and random effects estimation, - оценки со случайными и фиксированными эффектами для панельных данных
panel GMM linear estimation, - оценка обобщенного метода моментов для панельных данных
panel nonlinear estimation, - нелинейное оценивание в панельных данных
quantile regression, - квантильная регрессия
selection and two-part models, - модели самоотбора и двухуровневые модели
survival function, - функция выживания
treatment evaluation estimation, - оценивание эффекта воздействия
see also data sets used in applications  - см. также наборы данных, используемые в приложениях
Archimedean family, - семейство Архимедовых копул
Arellano-Bond estimator, - Ареллано-Бонда оценка
application, - применения
nonlinear models, - нелинейные корни
unit roots, - единичные корни
ARMA. See autoregressive moving average - ARMA. См. процесс авторегрессии и скользящего среднего
artificial nesting, - искусственное вложение
ARUM. See additive random utility model - ARUM. См. аддитивная модель случайной полезности
asymptotic distribution, - асимптотическое распределение
asymptotic efficiency, - асимптотическая эффективность
asymptotic normal distribution, - асимптотически нормальное распределение
definition, - определение
estimated asymptotic variance, - оценка асимптотической дисперсии
of extremum estimators, - экстремальных оценок
of FGLS estimator, - оценки доступного обобщенного МНК
of FGNLS estimator, - оценки доступного обобщенного нелинейного МНК
of first-differences estimator, - оценки в первых разностях
of fixed effects estimator, - оценки в модели с фиксированными эффектами
of GMM estimator, - GMM оценки, оценки обобщенного метода моментов
of Hausman test statistic, - статистики Хаусмана
of kernel density estimator, - ядерной оценки функции плотности
of kernel regression estimator, - ядерной оценки линии регрессии
of LM test statistic, - LM статистики, статистики множителей Лагранжа
of LR test statistic, - LR статистики, статистики отношения правдоподобия
of m-estimators, - М-оценок
of MD estimator, - MD-оценки, оценки минимального расстояния
of ML estimator, - оценки метода максимального правдоподобия
of MM estimator, - оценки метода моментов
of MSL estimator, - оценки симуляционного максимального правдподобия
of MSM estimator, - оценки симуляционного метода моментов
of m-test statistics, - М-статистики
of NLS estimator, - оценки нелинейного МНК
of NL2SLS estimator, - оценки нелинейного двухшагового МНК
of OIR test statistic, - статистики сверх-идентифицирующих ограничений
of OLS estimator, - МНК оценки
of panel GMM estimator, - панельной оценки обобщенного метода моментов
of quasi-ML estimator, - оценки квази-максимального правдоподобия
of random effects estimator, - оценки модели со случайными эффектами
of Wald test statistic, - статистики Вальда
see also asymptotic theory - см. также асимптотическая теория
asymptotic efficiency, - асимптотическая эффективность
of optimal GMM, - оптимального обобщенного метода моментов
asymptotic refinement, - асимптотическое уточнение
by bootstrap, - с помощью бутстрэпа
definition, - определение
by Edgeworth expansion, - с помощью разложения Эджуорта
by nested bootstrap, - с помощью вложенного бутстрэпа
asymptotic theory definitions, - асимптотические определения
asymptotic distribution, - асимптотическое распределение
asymptotic variance, - асимптотическая дисперсия
central limit theorems, - центральные предельные теоремы
consistency, - состоятельность
convergence in distribution, - сходимость по распределениею
convergence in probability - сходимость по вероятности
laws of large numbers, - законы больших чисел
limit distribution - предельное распределение
limit variance - предельная дисперсия
stochastic order of magnitude, - стохастический порядок малости
summary of definitions and theorems, - список определений и теорем
asymptotic variance, - асимптотическая дисперсия
estimated asymptotic variance,  - оценка асимптотической дисперсии
see also asymptotic distribution - см. также асимптотическое распределение
asymptotically pivotal statistic, - асимптотическая статистика
ATE. See average treatment effect - ATE. См. средний эффект воздействия
ATET. See average treatment effect on the treated - ATET. См. средний эффект воздействия подвергнутой воздействию группы
attenuation bias, - смещение смягчения
attrition bias, - смещение истощения выборки
augmented regression model, - расширенная модель регрессии
autocorrelation - автокорреляция
in panel model errors, - в ошибках панельных моделей
dynamic panel models, - динамические панельные модели
see also panel-robust inference - см. также статистические выводы робастные к панельным данным
autoregressive moving average (ARMA) errors - ARMA ошибки
definition, - определение
NLS estimator, - оценка методом нелинейных наименьших квадратов
panel data, - в панельных данных
auxiliary model, - вспомогательная модель
auxiliary regression - вспомогательная регрессия
bootstrapping, - бутстрэп
example, - пример
Hausman test, - тест Хаусмана
LM test, - LM тест, тест множителей Лагранжа
m-test, - М-тест
available case analysis. See pairwise deletion - анализ доступных случаев. См. попарное удаление
average completed duration (ACD), - средняя длительность завершенного состояния
average derivative (AD) estimator - AD оценка, оценка средней производной
definition, - определение
uses, - использование
average interrupted duration (AID), - средняя прекращенная длительность
average selection bias, - среднее смещение самоотбора
average squared error, - среднеквадратичная ошибка
average treatment effect (ATE), - средний эффект воздействия (ATE)
definition, - определение
difficulties estimating, - трудности при оценивании
local ATE, - локальный ATE, локальный средний эффект воздействия
matching estimators, - оценка сопоставления
potential outcome model, - модель потенциального результата
selection on observables only, - самоотбор по наблюдаемым показателям
selection on unobservables, - самоотбор по ненаблюдаемым показателям
see also ATET; LATE; MTE - см. также ATET, LATE, MTE
average treatment effect on the treated (ATET), - средний эффект воздействия подвергнутой воздействию группы (ATET)
application, - приложения
definition, - определение
difficulties estimating, - трудности при оценивании
matching estimators, - оценки методом сопоставления
selection on observables only, - самоотбор по наблюдаемым показателям
selection on unobservables, - самоотбор по ненаблюдаемым показателям
see also ATE; LATE; MTE - см. также ATET, LATE, MTE
averaged data. See aggregated data - усредненные данные. См. агрегированные данные
backward recurrence time, - обратное время повторения
balanced bootstrap, - сбалансированный бутстрэп
balanced repeated replication, - повторяющиеся сбалансированные репликации
balancing condition, - балансирующее условие
bandwidth, - ширина окна
bandwidth choice for kernel density estimator, - выбор ширины окна для ядерных оценок функции плотности
cross validation, - кросс-валидация
example, - пример
optimal, - оптимальный
Silverman’s plug-in estimate, - оценка Сильвермана
bandwidth choice for kernel regression estimator, - выбор ширины окна для ядерных оценок линии регрессии
cross validation, - кросс-валидация
example, - пример
optimal, - оптимальный
plug-in estimate, - оценка Сильвермана
baseline hazard, - базовый риск
in AFT model, - AFT модель, модель ускоренной жизни
identification in mixture models, - идентификация в моделях смеси
in multiple spells models, - модели многократных событий
in PH model, - в PH модели, в модели пропорциональных рисков
Bayes factors, - Байесовские факторы
Bayes rule. See Bayes theorem - Байеса правило, см. Байеса теорема
Bayes theorem, - Байеса теорема
example, - пример
Bayesian central limit theorem, - Байесовская центральная предельная теорема
Bayesian information criterion (BIC), - Байесовский информационный критерий
see also AIC - см. также AIC
Bayesian methods, - Байесовские методы
Bayes 1764 example, - пример Байеса 1764 года
Bayesian approach, - Байесовский подход
binary outcome models, - модели бинарного выбора
compared to non-Bayesian, - сравнение с не-Байесовским подходом
count models, - счетные модели
data augmentation, - пополнение данных
decision analysis, - анализ при принятии решений
examples, - примеры
hierarchical linear model, - иерархическая линейная модель
importance sampling, - сэмплирование по важности
linear regression, - линейная регрессия
Markov chain Monte Carlo simulation, - метод Монте-Карло по схеме марковской цепи
measurement error model, - модель ошибки измерения
mixed linear model, - смешанная линейная модель
model selection, - выбор модели
multinomial outcome models, - мультиномиальная модель
panel data, - панельные данные
posterior distribution, - апостериорное распределение
prior distribution, - априорное распределение
Tobit model, - тобит модель
BCA method. See bias-corrected and accelerated - См. скорректированный на смещение и ускоренный бутстрэп
before-after comparison - сравнение до и после
application, - приложения
Berkson error model, - Берксона модель ошибки
Berkson’s minimum chi-square estimator, - Берксона метод минимизации хи-квадрат
Berndt, Hall, Hall, and Hausman (BHHH) estimate, - Берндта, Холла, Холла и Хаусмана оценка (BHHH)
Berndt, Hall, Hall, and Hausman (BHHH) iterative method, - итерационный алгоритм Берндта, Берндта, Холла и Хаусмана, BHHH итерационный алгоритм
Bernoulli distribution, - распределение Бернулли
Bernstein-von Mises Theorem, - Бернштейна-фон Мизеса теорема
best linear unbiased predictor, - наилучшая линейная несмещенная оценка
between estimator, - оценка between
application, - приложения
between-group variation, - межгрупповая дисперсия
between model, - модель between
BFGS algorithm. See Boyden, Fletcher, Goldfarb, and Shannon - BFGS алгоритм, см. Бойден, Флетчер, Голдфарб и Шеннон
BHHH estimate. See Berndt, Hall, Hall, and Hausman BHHH method. See Berndt, Hall, Hall, and Hausman
bias-corrected and accelerated (BCA) bootstrap method, - скорректированный на смещение и ускоренный бутстрэп
biased sampling, - выборка со смещением
see also sample selection; endogenous stratification - см. также самоотбор выборки, эндогенная стратификация
BIC. See Bayesian information criterion - BIC. См. Байесовский информационный критерий
binary endogenous variable, - бинарные эндогенные переменные
binary outcome models, - модели бинарного выбора
additive random utility model, - аддитивная модель случайной полезности
aggregated data, - агрегированные данные
alternative-invariant regressors, - регрессоры, постоянные для альтернатив
alternative-varying regressors, - регрессоры изменяющиеся в зависимости от альтернативы
choice-based samples, - самоотбор выборки, отбор при построении выборки
corrected score estimator, - скорректировання скор-оценка
definition, - определение
example, - пример
identification, - идентифицируемость
index function model, - модель индексных функций
marginal effects, - предельные эффекты
measurement error in dependent variable, - ошибка измерения в зависимой переменной
measurement error in regressors, - ошибка измерения в регрессорах
ML estimator, - ML оценка, оценка максимального правдоподобия
model misspecification, - неправильная спецификация модели, мисспецификация
multiple imputation example, - пример множественного восстановления
OLS estimator, - МНК-оценка
panel data, - панельные данные
semiparametric estimation, - полупараметрическое оценивание
see also logit models; probit models - см. также логит-модели: пробит-модели
binding function, - связывающая функция
bivariate counts, - двумерные счетные данные
bivariate negative binomial distribution, - двумерное отрицательное биномиальное распределение
bivariate ordered probit model, - двумерная упорядоченная пробит модель
bivariate Poisson distribution, - двумерное распределение Пуассона
bivariate Poisson-lognormal mixture, - двумерная Пуассона-логнормальная смесь распределений
bivariate probit model, - двумерная пробит модель
bivariate sample selection model, - двумерная модель самоотбора выборки
application, - применения
bounds, - границы
conditional mean, - условное среднее
conditional variance, - условная дисперсия
definition, - определение
Heckman two-step estimator, - двухшаговая оценка Хекмана
identification, - идентифицируемость
marginal effects, - предельные эффекты
ML estimator, - ML оценка, оценка максимального правдоподобия
outcome equation, - уравнение результата
participation equation, - уравнение участие
semiparametric estimator, - полупараметрическая оценка
versus two-part model, - сравнение с двухчастной моделью
Bonferroni test, - Бонферони тест
bootstrap hypothesis tests - проверка гипотез с бутстрэпа
asymptotic refinement, - асимптотическое уточнение
bootstrap critical value, - бутстрэп критическое значение
bootstrap p-value, - бутстрэп P-значение
example, - пример
nonsymmetrical test, - несимметрические тест
power, - мощность
symmetrical test, - симметричный тест
without asymptotic refinement, - без асимптотического уточнения
bootstrap methods, - бутстрэп методы
asymptotic refinement, - асимптотическое уточнение
bias estimate, - оценка смещения
bias-corrected estimator, - оценка скорректированная на смещение
clustered data, - кластеризованные данные
confidence intervals, - доверительные интервалы
consistency, - состоятельность
critical value, - критическое значение
examples, - примеры
for functions of parameters, - для функций от параметров
general algorithm, - общий алгоритм
for GMM, - для обобщенного метода моментов
heteroskedastic data, - гетероскедастичные данные
introduction, - введение
for nonsmooth estimators, - для негладких оценок
number of bootstrap samples, - количество бутстрэп выборок
panel data, - панельные данные
p-value, - P-значение
recentering, - повторное центрирование
rescaling, - изменение масштаба
sampling methods for, - методы сэмплирования
smoothness requirements, - требования гладкости
standard error estimate, - оценка стандартной ошибки
time series data, - временные ряды
variance estimate, - оценка дисперсии
without asymptotic refinement, - без асимптотического уточнения
see also bootstrap hypothesis tests - см. также тестирование гипотез с помощью бутстрэп
bounds identification, - интервальная идентификация
in measurement error models, - в моделях ошибки измерения
bounds in selection model, - границы для модели самоотбора
Boyden, Fletcher, Goldfarb, and Shannon (BFGS) algorithm, - BFGS-алгоритм, алгоритм Бойдена-Флетчера-Голдфарба-Шеннона
CAIC. See consistent Akaike information criterion - CAIC. См. состоятельный Акаике информационный критерий
calibrated bootstrap, - калиброванный бутстрэп
caliper matching, - циркульное сопоставление
canonical link function, - каноническая функция связи
case-control analysis, - исследование случай-контроль
causality, - причинность
examples, - примеры
Granger causality, - причинность по Грейнджеру
identification frameworks and strategies, - стратегии идентификации
in linear regression model, - в линейных регрессионных моделях
in potential outcome models, - в моделях потенциального результата
in simultaneous equations model, - в моделях одновременных уравнений
in single-equation model, - в моделях с одним уравнением
and weighting, - и взвешивание
see also endogeneity - см. также эндогенность
cdf. See cumulative distribution function - функция распределение
censored least absolute deviations (CLAD) estimator, - CLAD, цензурированная оценка наименьших абсолютных отклонений
censored models, - цензурированные модели
conditional mean, - условное среднее
count models, - счетные модели
definitions, - определения
examples, - примеры
ML estimator, - ML оценка, оценка максимального правдоподобия
semiparametric estimation, - полупараметрическое оценивание
see also duration model; selection models; Tobit models; truncated models - см. также модели времени жизни, модели самоотбора, тобит-модели, модели с усечением
censored normal regression model. See Tobit model censoring mechanisms, - цензурированная модель регрессии с нормальными ошибками. См. также механизм цензурирования в тобит модели
censoring from above, - цензурирование сверху
censoring from below, - цензурирование снизу
left censoring, - цензурирование слева
independent censoring, - независимое цензурирование
interval censoring, - интервальное цензурирование
noninformative censoring, - неинформативное цензурирование
random censoring, - случайное цензурирование
right censoring, - цензурирование справа
sample selection, - самоотбор выборки
type 1 censoring, - цензурирование 1-го типа
type 2 censoring, - цензурирование 2-го типа
census coefficient, - коэффициент ценза
central limit theorem (CLT), - центральная предельная теорема
Cramer linear transformation, - линейное преобразование Крамера
Cramer-Wold device, - Крамера-Вольда теорема
definition, - определение
examples of use, - примеры использования
Liapounov CLT, - ЦПТ Ляпунова
Lindeberg-Levy - ЦПТ Линдеберга-Леви
multivariate, - многомерная ЦПТ
sample average, - выборочное среднее
CGF tests. See chi-square goodness-of-fit - хи-квадрат критерий согласия
characteristic function, - характеристическая функция
Chebychev’s inequality, - неравенство Чебышева
chi-square goodness-of-fit (CGF) tests, - хи-квадрат критерий согласия
choice-based samples, - самоотбор выборки, отбор при построении выборки
binary outcome models, - модель бинарного исхода
see also endogenous stratification - см. также эндогенная стратификация
Choleski decomposition, - Холецкого разложение
CL model. See conditional logit - условная логит-модель
CLAD estimator. See censored least absolute deviations - CLAD оценка. См. цензурированный метод наименьших абсолютных отклонений
Clayton copula, - Клейтона копула
CLT. See central limit theorem - ЦПТ. См. центральная предельная теореиа
clustered data, - кластеризованные данные
application, - приложения
cluster bootstrap, - кластеризованный бутстрэп
cluster-robust inference, - статистические выводы робастные для кластеризованных данных
cluster sampling, - кластеризованная выборка
cluster-specific effects, - кластерные эффекты
comparison to panel data, - сравнение с панельными данными
diagnostic tests, - диагностические тесты
dummy variables model, - модель с дамми-переменными
fixed effects estimator, - оценка модели с фиксированными эффектами
hierarchical models, - иерархические модели
large clusters, - большие кластеры
nonlinear models, - нелинейный оценки
OLS estimator, - МНК-оценка
quasi-ML estimator, - оценка квази-максимального правдоподобия
random effects estimator, - оценка модели со случайными эффектами
small clusters, - маленькие кластеры
see also panel data - см. также панельные данные
cluster-robust standard errors - станадртные ошибки устойчивые к кластерам
bootstrap, - бутстрэп
clustered data, - кластеризованные данные
panel data, - панельные данные
see also robust standard errors - см. также робастные стандартные ошибки
cluster-specific fixed effects (CSFE) estimator, - оценка с фиксированными кластерными эффектами
application, - приложения
between estimator, - оценка between
nonlinear models, - нелинейные модели
within estimator, - оценка within
cluster-specific fixed effects (CSFE) model, - модель с фиксированными кластерными эффектами
cluster-specific random effects (CSRE) estimator, - оценка со случайными кластерными эффектами
application, - приложения
cluster-specific random effects (CSRE) model, - модель со случайными кластерными эффектами
cluster variable, - кластерная переменная
CM tests. See conditional moment - тесты на условные моменты. См. условные моменты
coefficient interpretation - интерпретация коэффициентов
in binary outcome models, - в моделях бинарного выбора
in competing risks model, - в модели конкурирующих рисков
in count model, - в счетных моделях
in duration models, - в моделях времени жизни
in misspecified linear model, - в неправильно специфицированных линейных моделях
in multinomial outcome models, - в мультиномиальных моделях
in nonlinear models, - в нелинейных моделях
in Tobit model, - в тобит модели
see also marginal effects - см. также предельные эффекты
coherency condition, - условие согласованности
cohort-level data. See pseudo panels - данные уровня когорт. См. псевдо-панели
cointegration, - коинтеграция
common parameters, - общие параметры
compensating variation, - компенсирующая вариация
competing risks model (CRM), - CRM, модель конкурирующих рисков
application, - приложения
censoring, - цензурирование
coefficient interpretation, - интерпретация коэффициентов
definitions, - определения
dependent risks, - зависимые риски
exit route, - способ перехода
identification, - идентификация
independent risks, - независимые риски
ML estimator, - ML оценка, оценка максимального правдоподобия
proportional hazards, - пропорциональные риски
spell duration, - длительность события
with unobserved heterogeneity,  - с ненаблюдаемой неоднородностью
complementary log-log model, - лог-логистическая модель
complete case analysis. See listwise deletion - анализ полной выборки. См. полное удаление наблюдений с пропусками
complex surveys, - сложные обследования
composition methods, - метод комбинирования
computational difficulties, - вычислительные трудности
concentration parameter, - параметр концентрации
conditional analysis, - условный анализ
conditional expectations, - условное ожидание
conditional independence assumption, - предположение об условной независимости
definition, - определение
for participation, - для участия
given propensity score, - при заданной мере склонности
selection on observables only, - самоотбор по наблюдаемым показателям
unconfoundedness, - несмешиваемость
conditional likelihood, - условная функция правдоподобия
panel models, - модели панельных данных
conditional logit (CL) model, - условная логит-модель
application, - приложения
definition, - определение
fixed effects binary logit, - логит-модель с фиксированными эффектами
marginal effects, - предельные эффекты
ML estimator, - оценка максимального правдоподобия
from ARUM, - для ARUM
see also multinomial outcome models - см. также мультиномиальные моменты
conditional ML estimator, - условная ML оценка, оценка условного максимального правдоподобия
conditional moment (CM) tests, - тест на условные моменты
consistent CM test, - состоятельный тест на условные моменты
in duration models, - в моделях времени жизни
example, - пример
in Tobit model, - в тобит-модели
see also m-tests - см. также М-тесты
conditional mean - условное среднее
squared error loss, - квадратичная функция потерь
conditional mode - условная мода
step loss, 68
condition number, - число обусловленности, индекс обусловленности
conditional quantile - условный квантиль
asymmetric absolute loss, - асимметричная функция абсолютных потерь
confidence intervals, - доверительные интервалы
consistent Akaike information criterion (CAIC), - (CAIC) состоятельный Акаике информационный критерий
consistent test statistic, - состоятельная статистика
consistency definition, - определение состоятельности
of extremum estimators, - экстремальных оценок
of GMM estimator, - GMM оценки, оценки обобщенным методом моментов
of m-estimator, - М-оценки
of ML estimator, - оценки максимального правдоподобия
of NLS estimator, - оценки нелинейного МНК
of OLS estimator, - МНК-оценки
strong consistency, - сильная состоятельность
weak consistency, - слабая состоятельность
see also asymptotic distribution; identification; см. также асимптотическое распределение, идентифицируемость
pseudo-true value - псевдо-истинное значение
constant coefficients model - модель с постоянными коэффициентами
See pooled model contagion, - заражение сквозной модели
contamination bias, - смещение заражения
contemporaneous exogeneity assumption, - предположение об одновременной экзогенности
continuous mapping theorem, - теорема о непрерывном отображении
control function approach, - подход контрольных функций
control function estimator, - оценка контрольных функций
control group, - контрольная группа
conventions, - соглашения
convergence criteria, - критерий сходимости
convergence in distribution, - сходимость по распределению
continuous mapping theorem, - теорема о непрерывном отображении
definition, - определение
limit distribution, - предельное распределение
transformation theorem, - теорема о преобразовании
vector random variables, - векторные случайные величины
see also central limit theorem - см. также центральная предельная теорема
convergence in probability, - сходимость по вероятности
alternative modes of convergence, - другие виды сходимости
consistency, - состоятельность
definition, - определение
probability limit, - предел по вероятности
Slutsky’s theorem, - Слуцкого теорема
uniform convergence, - равномерная сходимость
vector random variables, - векторная случайная величина
see also law of large numbers - см. также закон больших чисел
copulas, - копулы
count example, - счетный пример
definition, - опреление
dependence parameter, - параметр зависимости
leading examples, - основные примеры
ML estimator, - ML оценка, оценка максимального правдоподобия
survival copulas, - копулы выживания
correlated random effects model, - модель с коррелированными случайными эффектами
counterfactual, - контр-фактические значения
see also potential outcome model - см. также модель потенциального результата
count data, - счетные данные
examples, - примеры
heteroskedasticity, - гетероскедастичность
right-skewness, - скошенность вправо
see also count models - см. также счетные модели
count models, - счетные модели, модели счетных данных
censored, - цензурированные
application, - приложения
endogenous regressors, - эндогенные регрессоры
endogenous sampling, - эндогенная выборка
finite mixture models, - модели смеси распределений
hurdle models, модели преодоления порогов
measurement error in dependent variable, - ошибка измерения зависимой переменной
measurement error in regressors, - ошибка измерения регрессоров
mixture models, - модели смеси
multivariate, - многомерные
OLS estimator, - МНК-оценка
negative binomial model, - отрицательная биномиальная модель
NLS estimator, - оценка нелинейного МНК
panel data, - панельные данные
Poisson model, - модель Пуассона
sample selection, - самоотбор выборки
semiparametric regression, - полупараметрическая регрессия
truncated, - усеченная
zero-inflated, - с раздутым нулем
covariance matrix. See variance matrix covariance structures, - ковариационная матрица, см. структуры ковариационной матрицы
covariates. See regressors - Регрессоры
Cox CRM model. See competing risks. См. Конкурирующие риски
Cox PH model. See proportional hazards. См. Пропорциональные риски
Cox-Snell residual, - Кокса-Снелла остатки, обобщенные остатки
CPS. See Current Population Survey - CPS. См. Текущее обследование населения
Cramer linear transformation, - Крамера линейное преобразование
Cramer-Rao lower bound, - граница Крамера-Рао
see also semiparametric efficiency bound Cramer’s theorem. См. Теорема Крамера о полупараметрической границе эффективности
Cramer-Wold device, - теорема Крамера-Вольда
CRM. See competing risks model - CRM. См. модель конкурирующих рисков
cross-equation parameter restrictions, - ограничения на параметры в разных уравнениях
cross-section data, - пространственные данные
cross-validation, - кросс-валидация
CSFE estimator. - оценка с фиксированными кластерными эффектами. See cluster-specific fixed effects CSRE. See cluster-specific random effects. См. фиксированные кластерные эффекты. См. случайные кластерные эффекты
cumulant, - кумулянта
cumulative distribution function (cdf), - функция распределения
cumulative hazard function, - кумулятивная функция риска
definition, - определение
in competing risks model, - в модели конкурирующих рисков
as diagnostic tool, - инструмент для диагностики
in likelihood function, - в функции правдоподобия
Nelson-Aalen estimator, - Нельсона-Аалена оценка
in proportional hazards model, - в модели пропорциональных рисков
Current Population Survey (CPS), - текущее обследование населения
curse of dimensionality, - проклятие размерности
in Bayesian methods, - в Байесовских методах
multivariate kernel density estimator, - многомерная ядерная оценка плотности   
multivariate kernel regression estimator, - многомерная ядерная оценка регрессии
high-dimensional integrals, - многомерные интегралы
data augmentation, - пополнение данных
imputation step, - этап пополнения данных
for missing data, - для пропущенных данных
prediction step, - этап прогнозирования 
regression example, - пример с регрессией
data-generating process (dgp), - процесс, порождающий данные
misspecified, - неправильно специфицированный
data mining, - интеллектуальный анализ данных
data sets. See microdata - наборы данных, см. микроданные
data sets used in applications - наборы данных, используемые в приложениях
Current Population Survey - Текущее Обследование Населения
Displaced Workers Supplement (McCall), - Приложения об уволенных работниках
fishing-mode choice data (Kling and Herriges), - данные выбора способа рыбалки
National Longitudinal Survey (Kling), Национальное Лонгитюдное Обследование
National Supported Work demonstration project (Dehejia and Wahba), проект по национальной поддержке работы
Panel Survey of Income Dynamics cross-section sample, - Панельный опрос динамики доходов на пространственных данных
Panel Survey of Income Dynamics panel sample (Ziliak), - Панельный опрос динамики доходов на панельных данных
patents-R&D panel data (Hausman, Hall, and Griliches), - патенты и НИОКР, панельные данные
Rand Health Insurance Experiment expenditures, - эксперимент по страхованию здоровья корпорации Rand
Rand Health Insurance Experiment medical doctor contacts, - контакты доктора, участвовавшего в эксперименте по страхованию здоровья корпорации Rand
strike duration data (Kennan), - данные о длительности забастовок
Vietnam World Bank Livings Standards Survey, - Вьетнамское исследование стандартов жизни
see also applications with data - см. также приложения к данным
data structures, - структуры данных
data sources, - источники данных
handling microdata, - обработка микроданных
natural experiments, - естественные эксперименты 
observational data, - данные наблюдений
social experiments, - социальные эксперименты
data summary approach to regression, - подход к регрессии, который отталкивается от данных
Davidon, Fletcher, and Powell (DFP) algorithm - DFP алгоритм, алгоритм Дэвидона-Флетчера-Паулла
decomposition of variance, - разложение дисперсии
degenerate distribution, - вырожденное распределение
degrees-of-freedom adjustment, - поправка на количество степеней свободы
delta method, - дельта-метод
bootstrap alternative, - альтернатива бутстрэп
density kernel, - ядерная функция плотности
density-weighted average derivative (DWAD)
estimator, - средневзвешенная оценка производной плотности
dependent variable, - зависимая переменная
descriptive approach to regression, - описательный подход к регрессии
deviance, - отклонение
deviance residual, - остатки отклонений
DFP algorithm. See Davidon, Fletcher, and Powell - DFP алгоритм. См. Дэвидсона-Флетчера-Пауэлла алгоритм
algorithm - DFP алгоритм, см. алгоритм Дэвидона-Флетчера-Пауэлла
dgp. See data-generating process - процесс порождающий данные
diagnostic tests. See specification tests - диагностические тесты, см. тесты на спецификацию
DID estimator. See differences-in-differences - DID оценка. См. оценка разность разностей
differences-in-differences (DID) estimator, - оценка разность разностей
application, - приложения
consistency, - состоятельность
definition, - определение
introduction, - введение
natural experiments, - естественные эксперименты 
with controls, - под контролем
without controls, - без контроля
direct regression, - прямая регрессия
disaggregated data - дезагрегированные данные
contrasted with aggregated data, - сравнение с агрегированными данными 
discrete factor models, - модели дискретных факторов
see also finite mixture models - см. также модели конечной смеси
discrete outcomes. See binary outcomes; counts; - дискретные исходы. См. также бинарные исходы, счетные переменные
multinomial outcomes - мультиномиальные исходы
discrete-time duration data, - данные по времени жизни в дискретном времени
cumulative hazard function, - кумулятивная функция риска
discrete-time proportional hazards, - пропорциональные риски в дискретном времени
gamma heterogeneity, гамма-распределенная гетерогенность
hazard function, - функция риска
logit model, - логит-модель
ML estimator, - оценка максимального правдподобия
nonparametric estimation, - непараметрическая оценка
probit model, - пробит модель
survivor function, - функция выживания
dissimilarity parameter, - параметр расхождения
disturbance term. See error term - случайная составляющая, см. ошибка
double bootstrap, - двойной бутстрэп
dummy endogenous variable model, - модель с эндогенной дамми-переменной
dummy variable estimator, - оценка дамми-переменной
see also LSDV estimator duration data, - МНК-оценка с фиктивными переменными
different types, - различные типы

duration models, - модели времени жизни
accelerated failure time, - модели ускоренной жизни
applications, - приложения
censoring, - цензурирование
competing risks, - конкурирующие риски
cumulative hazard function, - кумулятивная функция риска
discrete time, - в дискретном времени
generalized residual,  - обобщенные остатки
hazard function, - функция риска
key concepts, - ключевые понятия
mixture models, - модели смеси
ML estimator, - оценка максимального правдоподобия
multiple spells, - множественные события 
multivariate, - многомерные
nonparametric estimators, - непараметрические оценки
OLS estimator, - МНК оценка
panel data, - панельные данные
parametric models, - параметрические модели
proportional hazards, - пропорциональные риски
risk set, - множество объектов под риском
semiparametric estimation, - полупараметрическое оценивание
specification tests, - тесты на спецификацию
survivor function, - функция выживания
time-varying regressors, - регрессоры, меняющиеся во времени
see also proportional hazards model
DWAD estimator. See density-weighted average derivative
dynamic panel models, - динамические панельные модели
Arellano-Bond estimator, - Ареллано-Бонда оценка
binary outcome models, - модели бинарного выбора
count models, - счетные модели
covariance structures, - структуры ковариационной матрицы
inconsistency of standard estimators, - несостоятельность обычных оценок
initial conditions, - начальные условия
IV estimators, - оценки методом инструментальных переменных
linear models, - линейные модели
MD estimator, - оценка минимального расстояния
nonlinear models, - нелинейные модели
nonstationary data, - нестационарные данные
transformed ML estimator, - преобразованная ML оценка, преобразованная оценка максимального правдоподобия
true state dependence, - истинная зависимость от состояния
unobserved heterogeneity, - ненаблюдаемая неоднородность
weak exogeneity, - слабая экзогенность
EDF bootstrap. See empirical distribution function bootstrap
Edgeworth expansions, - разложение Эджуорта
efficient score, - эффективное значение
Eicker-White robust standard errors, - Эйкера-Уайта робастные стандартные ошибки
see also heteroskedasticity robust-standard errors - см. также стандартные ошибки, устойчивые к гетероскедастичности
EM algorithm see expectation maximization - EM алгоритм, см. максимизация ожидания
empirical Bayes method, - эмпирический Байесовский подход
empirical distribution function (EDF) bootstrap, - бутстрэп эмпирической функции распределения
see also paired bootstrap - см. также парный бутстрэп
empirical likelihood, - эмпирический метод максимального правдоподобия
empirical likelihood bootstrap, - бутстрэп эмпирического правдоподобия
encompassing principle, - принцип всеобщности
endogeneity - эндогенность
definition, - определение
due to endogenous stratification, - вызванная эндогенной стратификацией
Hausman test for,  - тест Хаусмана на эндогенность
identification frameworks and strategies, - стратегии и подходы к идентификации
see also endogenous regressors; exogeneity - см. также эндогенные регрессоры, экзогенность
endogenous regressors, - эндогенные регрессоры
binary, - бинарные 
in count models, - в счетных моделях
in discrete outcome models, - в моделях дискретного выбора
in duration models, - в моделях времени жизни
dummy, - дамми
inconsistency of OLS, - несостоятельность МНК
in linear panel models, - в линейных панельных моделях
in linear simultaneous equations model, - в линейных системах одновременных уравнений
in nonlinear panel models, - в нелинейных панельных моделях
in potential outcome model, - в моделях потенциального результата
returns-to-schooling example, - пример с доходностью обучения
in selection models, - в моделях самоотбора
in single-equation models, - в моделях с одним уравнением
see also GMM estimator; IV estimator - см. также ОММ оценка (оценка обобщенного метода моментов), оценка инструментальных переменных
endogenous sampling, - эндогенная выборка
consistent estimation, - состоятельное оценивание
leading examples, - основные примеры
see also censored models; endogenous stratification; sample selection models 
endogenous stratification, - эндогенная стратификация
equation-by-equation OLS, - МНК для систем уравнений
equicorrelated errors, - равнокоррелированные ошибки
equidispersion, - равенство математического ожидания и дисперсии
error components model, - модель со составной ошибкой
See RE model error components SEM. См. модель со случайным эффектом системы одновременных уравнений
error components SUR model, - модель внешне не связанных уравнений со составной ошибкой
error components 2SLS estimator, - оценка двухшагового МНК со ставной ошибкой    
error components 3SLS estimator, - оценка трехшагового МНК со ставной ошибкой    
error term, - ошибка, случайная составляющая
additive, - аддитивная
nonadditive, - неаддитивная
errors-in-variables. See measurement error - ошибки в переменных. См. ошибка измерения
estimated asymptotic variance, - оцененная асимптотическая дисперсия
see also asymptotic distribution - см. также асимптотическое распределение
estimated prediction error. See cross-validation 
estimating equations estimator, - оценка оценивающих уравнений
asymptotic distribution, - асимптотическое распределение
clustered data, - кластеризованные данные
computation, - вычисление
definition, - определение
generalized, - обобщенная
variance matrix estimation, - оценка ковариационной матрицы
weighted, - взвешенная
see also MM estimator - см. также оценка метода моментов
Euler conditions, - Эйлера условия
exact identification. See just identification - точная идентифицируемость
exchangeable errors, - взаимозаменяемые ошибки
exhaustive sampling, - исчерпывающий отбор
exogeneity, - экзогенность
conditional independence, - условная независимость
Granger causality, - причинность по Грейнджеру
of instrument, - инструментов
overidentifying restrictions test for, - тест на сверх-идентифицирующие ограничения
panel data assumptions, - предположения для панельных данных
strong exogeneity, - сильная экзогенность
weak exogeneity, - слабая экзогенность
exogenous sampling, - экзогенная выборка
exogenous stratified sampling, - экзогенная стратифицированная выборка
exogenous regressor. See exogeneity - экзогенный регрессор, см. экзогенность
expectation maximization (EM) algorithm, - EM-алгоритм, алгоритм максимизации ожидания
for data imputation, - для восстановления данных
E (Expectation) step, - Е-шаг
for finite mixture model, - для моделей конечной смеси
M (Maximization) step, - М-шаг
compared to NR algorithm, - сравнение с алгоритмом Ньютона-Рафсона
expected elapsed duration, - оценка ожидаемого пройденного времени
experimental data, - экспериментальные данные
control group, - контрольная группа
natural experiments, - естественные эксперименты
social experiments, - социальные эксперименты
treatment group, - экспериментальная группа, группа подвергнутая воздействию
explanatory variables. See regressors - объясняющие переменные. См. регрессоры
exponential conditional mean, - экспоненциальное условное среднее
coefficient interpretation, - интерпретация коэффициентов
exponential distribution, - экспоненциальное распределение
for generalized (Cox-Snell) residual, - для обобщенных остатков (остатков Кокса-Снелла)
exponential family density, - экспоненциальное семейство распределений
conjugate prior for, - сопряженное априорное распределение
see also linear exponential family - см. также экспоненциальное семейство распределений
exponential-gamma regression model, - экспоненциальная-гамма регрессионная модель
exponential-IG regression model, - экспоненциальная-обратная гамма регрессионная модель
exponential regression model - экспоненциальная регрессионная модель
application with censored data, - применение к цензурированным данным
example with uncensored data, - пример с нецензурированными данными
extreme value distribution. See type 1 extreme value - распределение экстремальных значений. См. распределение экстремальных значений 1-го типа
extremum estimator, - экстремальная оценка
asymptotic distribution, - асимптотическое распределение
consistency, - состоятельность
definition, - определение
formal proofs, - формальные доказательства
informal approach, - неформальный подход
statistical inference, - статистические выводы
variance matrix estimation, - оценка ковариационной матрицы
factor analysis, - факторный анализ
factor loadings, - факторная нагрузка
factor model, - факторная модель
Fairlee-Gumble-Morgenstern copula, - копула Фарли-Гамбла-Моргенштерна
fast simulated annealing (FSA) method, - быстрый алгоритм имитации отжига
FD estimator. See first-differences - FD оценка. См. оценка в первых разностях
FE estimator. See fixed effects - FE оценка. См. оценка модели с фиксированными эффектами
feasible generalized least squares (FGLS) estimator, - оценка доступного обобщенного МНК
asymptotic distribution, - асимптотическое распределение
definition, - определение
example, - пример
in fixed effects model, - в модели с фиксированными эффектами
in mixed linear model, - в смешанной линейной модели
nonlinear, - нелинейная
in pooled model, - в сквозной модели
in random effects model, - в модели со случайными эффектами
as sequential two-step m-estimator, - как последовательная двухшаговая М-оценка
systems FGLS, - доступный обобщенный МНК для систем уравнений
feasible generalized nonlinear least squares (FGNLS) estimator, - оценка доступного обобщенного нелинейного МНК
asymptotic distribution, - асимптотическое распределение
definition, - определение
example, - пример
as optimal GMM estimator, - как оценка оптимального обобщенного метода моментов
systems FGNLS, - доступный обобщенный нелинейный МНК для систем уравнений

FGLS estimator. See feasible generalized least squares FGNLS estimator. See feasible generalized nonlinear least squares
FIML estimator. See full information maximum likelihood - Оценка максимального правдподобия с полной информацией
finite mixture models, - модели конечной смеси
counts, - счетные переменные
definition, - определение
EM algorithm, - ЕМ-алгоритм
latent class interpretation, - интерпретация с помощью скрытых (латентных) классов
number of components, - количество компонент (составляющих)
panel data, - панельные данные
see also mixture models. См. модели смеси
finite-sample bias - смещение в малой выборке, смещение в конечной выборке
of GMM estimator, - ОММ оценки, оценки обобщенного метода моментов
of IV estimator, - оценки инструментальных переменных
of tests, - тестовых статистик
finite-sample correction term - корректировка на конечный размер выборки
for sampling without replacement, - для выборки без повторений
first-differences (FD) estimator, - оценка в первых разностях
application, - приложения
asymptotic distribution, - асимптотическое распределение
compared to FE estimator, - сравнение с оценкой модели с фиксированными эффектами
consistency, - состоятельность
definition, - определение
IV estimator, - оценка инструментальных переменных
first-differences (FD) model, - модель в первых разностях
first-differences (FD) transformation, - взятие первой разности
fixed effects (FE) estimator, - FE-оценка, оценка модели с фиксированными эффектами
application, - приложения
asymptotic distribution, - асимптотическое распределение
binary outcome models, - модели бинарного выбора
clustered data, - кластеризованные данные
compared to DID estimator, - сравнение с оценкой разность-разностей
compared to FD estimator, - сравнение с оценкой в первых разностях
as conditional ML estimator, - как оценка условного максимального правдоподобия
consistency, - состоятельность
count models, - счетные модели
definition, - определение
duration models, - модели времени жизни
dynamic models, - динамические модели
as FGLS estimator, - как оценка доступного обобщенного МНК
Hausman test for, - тест Хаусмана
identification, - идентифицируемость
incidental parameters,
inconsistency, - несостоятельность
IV estimators, - оценки методом инструментальных переменных
as LSDV estimator, - как МНК-оценка с фиктивными переменными 
multinomial outcome models, - мультиномиальные модели
selection models, - модели самоотбора выборки
Tobit model, - тобит модель
versus random effects, - сравнение со случайными эффектами
fixed effects (FE) model, - модель с фиксированными эффектами
cohort-level, - с когортами
clustered data, - кластеризованные данные
definition, - определение
dynamic models, - динамические модели
endogenous regressors, - эндогенные регрессоры
identification, - идентифицируемость
incidental parameters, - мешающие параметры
marginal effects, - предельные эффекты
nonlinear models, - нелинейные модели
time-varying regressors, - регрессоры, меняющиеся во времени
versus random effects, - сравнение со случайными эффектами
see also fixed effects estimators - см. также оценку модели с фиксированными эффектами
fixed coefficient, - фиксированный коэффициент (постоянный коэффициент)
fixed design. See fixed in repeated samples - фиксированный дизайн. См. фиксированные в повторяющихся выборках
fixed in repeated samples, - фиксированные в повторяющихся выборках
bootstrap sampling method, - бутстрэп 
in kernel regression, - в ядерной регрессии
Liapounov CLT, - Ляпунова ЦПТ
Markov LLN, - Маркова ЗБЧ
Monte Carlo sampling method, - метод сэмплирования Монте-Карло
fixed regressors, - фиксированные регрессоры. See fixed in repeated samples. См. фиксированные в повторяющихся выборках
flexible parametric models - гибкие параметрические модели
count models, - счетные модели
hazard models, - модели риска
selection models, - модели самоотбора
flow sampling, - выборка типа поток
forward orthogonal deviations IV estimator, - оценка модели форвардных ортогональных отклонений
forward orthogonal deviations model, - модель форвардных ортогональных отклонений

forward recurrence time, - прямое время повторения
Fourier flexible functional form, - гибкая функциональная форма Фурье
frailty, - уязвимость
see also unobserved heterogeneity - см. также ненаблюдаемая неоднородность
Frank copula, - Франка копула
Frechet bounds, - Фреше границы
frequentist approach, - частотный подход
FSA method. See fast simulated annealing - быстрый алгоритм имитации отжига
full conditional distributions, - полные условные распределения
see also Gibbs sampler - см. также сэмплирование по Гиббсу
full information maximum likelihood (FIML) - метод максимального правдоподобия с полной информацией
estimator, - оценка
nested logit model, - вложенная логит модель
nonlinear models, - нелинейные модели 
functional approach - функциональный подход
to measurement error, - к ошибке измерения
functional form misspecification, - неправильная спецификация функциональное формы
diagnostics for, - диагностика
gamma distribution, - гамма-распределение
gamma function, - гамма-функция
Gaussian quadrature, - Гаусса метод интегрирования 
Gauss-Hermite quadrature, - Гаусса-Эрмита интегрирование
Gauss-Legendre quadrature, - Гаусса-Лежандра интегрирование
Gauss-Newton (GN) algorithm, - Гаусса-Ньютона алгоритм
example, - пример
GEE estimator. See generalized estimating equations - GEE оценка. См. обобщенная оценка оценивающих уравнений
general to specific tests, - тесты от общего к частному
generalized additive model, - обобщенная аддитивная модель
generalized cross-validation, - обобщенная кросс-валидация
generalized estimating equations (GEE) estimator, обобщенная оценка оценивающих уравнений (GEE оценка)
generalized extreme value (GEV) distribution, - обобщенное распределение экстремальных значений
see also nested logit model - см. также вложенная логит модель
generalized information matrix equality, - обобщенное равенство информационных матриц
generalized inverse, - обобщенная обратная матрица
generalized IV estimator, - обобщенная оценка инструментальных переменных

generalized least squares (GLS) estimator, - оценка обобщенного МНК
asymptotic distribution, - асимптотическое распределение
definition, - определение
as efficient GMM, - как эффективная оценка обобщенного метода оценок
example, - пример
nonlinear, - нелинейная
generalized linear models (GLMs), - обобщенные линейные модели
count data, - счетные данные
conditional ML estimator, - оценка условного максимального правдоподобия
GEE estimator, - обобщенная оценка оценивающих уравнений (GEE оценка)
quasi-ML estimator, - оценка квази-максимального правдоподобия
see also LEF models - см. также модели экспоненциального семейства распределений

generalized method of moments (GMM) estimator, - GMM-оценка, оценка обобщенного метода моментов
asymptotic distribution, - асимптотическое распределение
based on additional moment restrictions, - основанная на дополнительных моментных ограничениях
based on moment conditions from economic theory, - основанная на моментных ограничениях из экономической теории
based on optimal conditional moment, - основанная на оптимальных условных моментах
bootstrap for, - бутстрэп
computation, - вычисление
definition, - определение
endogenous counts, - эндогенные счетные данные
with endogenous stratification, - с эндогенной стратификацией
with exogenous stratification, - экзогенной стратификацией
examples, - примеры
finite-sample bias, - смещение из-за малого размера выборки
identification, - идентификация
linear IV, - линейный метод инструментальных переменных
linear systems, - линейные системы
nonlinear IV, - нелинейный метод инструментальных переменных
one-step GMM estimator, - одношаговая оценка обобщенного метода моментов
optimal GMM, - оптимальный ОММ (обобщенный метод моментов)
optimal moment condition, - оптимальные моментные условия
optimal weighting matrix, - оптимальная матрица весов
panel data, - панельные данные
practical considerations, - практические замечания
test based on, - тесты основанные на
two-step, - двухшаговая
variance matrix estimation, - оценка ковариационной матрицы
weak instruments, - слабые инструменты
see also panel GMM estimator - см. также оценка обобщенного метода моментов для панельных данных
generalized nonlinear least squares (GNLS) estimator. - оценка обобщенного нелинейного МНК
See feasible generalized nonlinear least squares - см. доступный обобщенный нелинейный МНК
generalized partially linear model, - обобщенная частично линейная модель
generalized random utility models, - обобщенные модели случайной полезности
generalized residual, - обобщенные остатки
in duration models, - в моделях времени жизни
in LM test, - в тесте множителей Лагранжа
plots of, - графики

generalized Tobit model, - обобщенная тобит-модель
generalized Weibull distribution, - обобщенное распределение Вейбулла
genetic algorithms, - генетические алгоритмы
GEV distribution. See generalized extreme value 
Geweke, Hajivassiliou, Keane (GHK) simulator, - GHK-симулятор, Гевеке-Хадживасилу-МакФаддена симулятор
for MNP model, - для мультиномиальной пробит модели
GHK simulator. See Geweke, Hajivassiliou, Keane simulator - GHK-симулятор. См. Гевеке-Хадживасилу-МакФаддена симулятор
Gibbs sampler, - сэмплирование по Гиббсу
data augmentation, - пополнение данных
example, - пример
in latent variable models, - в модели с латентными переменными (со скрытыми переменными)
see also Markov chain Monte Carlo - см. также метод Монте-Карло по схеме Марковской цепи
GLMs. See generalized linear models - GLM. См. обобщенные линейные модели
GLS estimator. See generalized least squares - оценка обобщенного МНК. См. обобщенный МНК
GMM estimator. See generalized method of moments - ОММ оценка. См. оценка обобщенного метода моментов
GN algorithm. See Gauss-Newton - Гаусса-Ньютона алгоритм
GNLS estimator. See feasible generalized nonlinear least squares - оценка обобщенного нелинейного МНК. См. доступный обобщенный нелинейный МНК
Gompertz distribution, - Гомперца распределение
Gompertz regression model, - Гомперца регрессионная модель
gradient methods, - градиентные методы
see also iterative methods - см. также итерационные методы
Granger causality, - причинность по Грейнджеру
grid search methods, - методы поиска на сетке
grouped data. See aggregated data - сгруппированные данные. См. агрегированные данные
Halton sequences, - последовательности Гальтона
Hausman test, - тест Хаусмана
applications, - приложения
asymptotic distribution, - асимптотическое распределение
auxiliary regressions, - вспомогательная регрессия
bootstrap, - бутстрэп
computation, - вычисление
definition, - определение
for endogeneity, - на эндогенность
for fixed effects, - для модели с фиксированными эффектами
for multinomial logit model, - для мультиномиальной логит-модели
power, - мощность
robust versions, - робастные версии
Hausman-Taylor IV estimator, - Хаусмана-Тейлора оценка инструментальных переменных
Hausman-Taylor model, - Хаусмана-Тейлора модель
Hawthorne effect, - Хоторна эффект
hazard function - функция риска
baseline in PH model, - базовый риск в модели пропорциональных рисков
cumulative hazard, - кумулятивный риск
definition, - определение
in mixture models, - в моделях смеси
multivariate, - многомерная
nonparametric estimator, - непараметрическая оценка
parametric examples, - параметрическая оценка
piecewise constant,  - кусочно постоянная
see also duration models - см. также модели времени жизни
Health and Retirement Study (HRS), - Исследование здоровья и выхода на пенсию
Heckit estimator. See Heckman two-step estimator - хекит оценка, см. Хекмана двухшаговая оценка
Heckman two-step estimator - двухшаговая оценка Хекмана
application, - приложения
in Roy model, - в модели Роя
in selection model, - в модели самоотбора
semiparametric estimator, - полупараметрическая оценка
in Tobit model, - в тобит моделях
Hessian matrix - матрица Гессе
estimate, - оценка
Newton-Raphson algorithm, - Ньютона-Рафсона алгоритм
singular, - вырожденная
heterogeneous treatment effects, - неоднородный эффект воздействия
IV estimator, - оценка инструментальных переменных
LATE estimator, - оценка LATE, оценка локального среднего эффекта воздействия
RD design, - разрывный дизайн
heterogeneity - неоднородность
within-cell, - внутри ячейки
see also unobserved heterogeneity - см. также ненаблюдаемая неоднородность
heteroskedastic errors - гетероскедастичные ошибки
adaptive estimation, - адаптивное оценивание
conditional heteroskedasticity, - условная гетероскедастичность
definition, - определение
in GLMs, - в обобщенных линейных моделях
in linear model, - в линейных моделях
multiplicative, - мультипликативная
in nonlinear model, - в нелинейных моделях
residuals, - остатки
tests for, - тесты
Tobit MLE inconsistency, - несостоятельность ML оценок в тобит модели
working matrix for, - рабочая матрица
heteroskedasticity-robust standard errors - стандартные ошибки устойчивые к гетероскедастичности
bootstrap, - бутстрэп
clustered data, - кластеризованные данные
example, - пример
for extremum estimator, - для экстремальных оценок
intuition, - интуиция
for NLS estimator, - для оценки нелинейным МНК
for OLS estimator, - для МНК оценки
panel data, - панельные данные
for WLS estimator, - для оценки взвешенного МНК
see also robust standard errors - см. также робастные стандартные ошибки
hierarchical linear models (HLMs), - иерархические линейные модели
Bayesian analysis, - Байесовский анализ
clustered data, - кластеризованные данные
coefficient types, - типы коэффициентов
individual-specific effects, - индивидуальные эффекты
mixed linear models, - смешанные линейные модели
panel data, - панельные данные
random coefficients model, - модель со случайными коэффициентами
two-level model, - двухуровневая модель
hierarchical models, - иерархические модели
Bayesian analysis, - Байесовский анализ
see also hierarchical linear models - см. также иерархические линейные модели
histogram, - гистограмма
see also kernel density estimator - см. также ядерная оценка функции плотности
HLM. See hierarchical linear model - см. иерархические линейные модели
hot deck imputation, - метод карточной колоды (для восстановления пропущенных наблюдений)
HRS. See Health and Retirement Study - HRS, См. Исследование здоровья и выхода на пенсию
Huber-White robust standard errors, - робастные стандартные ошибки Хубера-Уайта
see also robust standard errors - см. также робастные стандартные ошибки
hurdle model, - модель преодоления порогов
see also two-part model - см. также двухчастная модель
hyperparameters, - гиперпараметры
hypothesis tests, - тестирование гипотез
based on extremum estimator, - основанное на экстремальных оценках
based on ML estimator, - основанное на оценках методом максимального правдоподобия
based on GMM estimator, - основанное на оценках обобщенных методом моментов
based on m-estimator, - основанное на М-оценках
bootstrap, - бутстрэп
for common misspecifications, - на распространенный ошибки спецификации
examples, - примеры
induced test, - индуцированный тест
joint versus separate, - совместные тесты в сравнении с отдельными тестами
power, - мощность
size, - размер теста
see also LM tests; LR test; Wald tests, m-tests - см. также LM-тест (тест множителей Лагранжа), LR-тест (тест отношения правдоподобия), Вальда тест, М-тест

identification - идентифицируемость
in additive random utility models, - в аддитивных моделях со случайной полезносью
in binary outcome models, - в моделях бинарного выбора
bounds identification, - интервальная идентификация
definitions, - определения
in fixed effects model, - в моделях с фиксированными эффектами
of GMM estimator, - GMM-оценки, оценки обобщенного метода моментов
just identification, - точная идентифицируемость
in linear regression model, - в линейных регрессионных моделях
in measurement error models, - в моделях с ошибкой измерения
in mixture models, - в моделях смеси распределений
in multinomial probit model, - в мультиномиальной пробит модели
in natural experiments, - в естественных экспериментах
observational equivalence, - эквивалентность
order condition, - условие порядка
over identification, - сверх идентифицированность
rank condition, - условие ранга
in sample selection model, - в моделях самоотбора выборки
set identification, - идентификация множества
in simultaneous equations model, - в моделях одновременных уравнений
in single-index models, - в одноиндексных моделях
singular Hessian, - вырожденная матрица Гесси
weak identification, - слабая идентифицируемость
see also identification strategies - см. также стратегии идентификации
identification strategies, - стратегии идентификации
control function approach, - подход контрольных функций
exogenization, - присвоение переменной статуса экзогенной
incidental parameter elimination, - оценка мешающих параметров
instrumental variables, - инструментальные переменные
matching, - сопоставление
reweighting, - перевзвешивание выборок 

identified reduced form, - идентифицируемая приведенная форма
IG distribution. See inverse-Gaussian - Обратное Гауссовское распределение
ignorable missingness, - игнорируемые пропуски
estimator consistency if MCAR, - состоятельность оценки при MCAR
estimator inconsistency if MAR only, - несостоятельность оценки при MAR
problems if nonignorable, - проблема неигнорируемости
weak exogeneity, - слабая экзогенность
ignorability assumption, - предположение об игнорируемости
see also conditional independence assumption - см. также предположение об условной независимости
importance sampling, - сэмплирование по важности
accelerated, - ускоренное
GHK simulator, GHK-симулятор, Гевеке-Хадживасилу-МакФаддена симулятор
importance sampling density, - вспомогательная функция плотности при сэмплировании по важности
importance sampling estimator, - оценка сэмплирования по важности
importance weight, - веса сэмплирования по важности
target density, - целевая функция плотности
imputation methods, - методы восстановления данных
data augmentation, - пополнение данных
example, - пример
hot deck imputation, - восстановление пропущенных данных по методу горячей колоды
listwise deletion, - полное удаление наблюдений с пропусками
mean imputation, - заполнение средним
multiple imputation, - множественное восстановление
pairwise deletion, - попарное удаление
regression-based imputation, - восстановление с помощью регрессии
imputation (I) step, - I-шаг, шаг пополнения
IM test. See information matrix test - IM-тест, см. тест информационной матрицы
IMSE. See integrated mean squared error - IMSE. См. интегрированная среднеквадратичная ошибка
incidental parameters, - мешающие параметры
clustered data FE model, - модель с фиксированными эффектами для кластеризованных данных
panel data FE model, - модель с фиксированными эффектами для панельных данных
inclusive value, - включающая величина (лог-сумма)
incomplete gamma function, - неполная гамма-функция
incomplete panels. See unbalanced panels  - неполные панели. См. несбалансированные панели
independence of irrelevant alternatives, - независимость от посторонних альтернатив
independent variables. See regressors - независимые переменных. См. регрессоры
independently-weighted IV estimator, - независимо взвешенная оценка инструментальных переменных
independently-weighted optimal GMM estimator, - независимо взвешенная оценка оптимального метода моментов

index function model - модель индексных функций
binary outcome model, - модель бинарного выбора
bivariate probit model, - двумерная пробит модель  
ordered multinomial model, упорядоченная мультиномиальная модель
Tobit model, - тобит модель
see also single-index model - см. также одноиндексная модель
indicator function, - функция-индикатор, индикатор
indirect inference, - косвенные статистические выводы

individual-specific effects model - модель индивидуальных эффектов
additive, - аддитивная
binary outcome models, - модели бинарного выбора
cluster-specific effects, - кластерные эффекты
count models, - счетные модели
definitions, - определение
duration models, - модели времени жизни
multiplicative, - мультипликативная
one-way, - с односторонними эффектами
parametric, - параметрическиая
selection models, - модели самоотбора
single-index, - одноиндексная
Tobit models, - тобит модели
two-way, - с двусторонними эффектами
see also FE models; RE models - см. также модели с фиксированными эффектами, модели со случайными эффектами
induced test, - индуцированный тест
information criteria, - информационный критерий
Akaike, - Акаике
Bayesian, - Байесовский
consistent Akaike, - состоятельный критерий Акаике
Kullback-Liebler, - Кульбака-Лейблера
Schwarz, - Шварца
information matrix, - информационная матрица
block-diagonal, - блочно-диагональная
information matrix equality, - равенство информационных матриц
generalized, - обобщенное
see also BHHH estimate; OPG version - см. также BHHH оценка, вариант с внешним произведение градиента
information matrix (IM) test, - тест информационной матрицы
bootstrap, - бутстрэп
computation, - вычисления
definition, - определение
example, - пример
power, - мощность
instrumental variables (IV) estimator - оценка метода инструментальных переменных
alternative estimators, - альтернативные оценки
application, - приложения
definition, - определение
example, - пример
finite-sample bias, - смещение из-за малого размера выборки
identification, - идентификация
independently-weighted IV estimator, - независимо взвешенная оценка инструментальных переменных
jackknife IV estimator, - джекнайф оценки инструментальных переменных
LIML estimator, - оценка метода максимального правдоподобия с ограниченной информацией
in linear model, - в линейных моделях
linear IV as GMM estimator, - линейная оценка инструментальных переменных как GMM-оценка (оценка обобщенного метода моментов)
local average treatment effects estimator, - оценка локального среднего эффекта воздействия
in measurement error models, - в моделях с ошибкой измерения
in natural experiments, - в естественных экспериментах
in nonlinear models, - в нелинейных моделях
in panel models, - в моделях панельных данных
quantile regression, - квантильная регрессия
in selection models, - в моделях самоотбора
split-sample estimator, - оценка с делением выборки
systems IV estimator, - оценка инструментальных переменных для систем уравнений
in treatment effects models, - в моделях эффекта воздействия
two-stage IV estimator, - двухшаговая оценка метода инструментальных переменных
two-stage least squares estimator, - двухшаговая оценка МНК
Wald estimator, - оценка Вольда
see also GMM estimator; panel GMM estimator - см. также GMM оценка (оценка обобщенного метода моментов), GMM оценка для панельных данных
instruments - инструменты
definition, - определение
examples, - примеры
by exclusion restriction, - с ограничением исключения
by functional form restriction, - с ограничением функциональной формы
invalid, - негодные
optimal, - оптимальные
for panel data, - для панельных данных
relevance, - релевантность
weak, - слабые
see also instrumental variables estimator - см. также оценка инструментальных переменных
integrated hazard function. See cumulative hazard function - интегрированная функция риск. См. кумулятивная функция риска
integrated mean squared error (IMSE), - IMSE, интегрированная среднеквадратичная ошибка
integrated squared error (ISE), - ISE, интегрированная квадратичная ошибка 
interval data models - модели интервальных данных
definition, - определение
ML estimator, - оценка метода максимального правдоподобия
interruption bias, - смещение прекращенных длительностей
intraclass correlation, - внутриклассовая корреляция
inverse-Gaussian (IG) distribution, - обратное Гауссовское распределение 
inverse law of probability, - метод обратной вероятности
inverse-Mills ratio, - обратное отношение Миллса
inverse transformation method, - метод обратного преобразования
inverse-Wishart distribution, - обратное распределение Уишарта
irrelevant regressors, - лишние регрессоры
ISE. See integrated squared error
iterated bootstrap, - итерационный бутстрэп
iterative methods, - итерационные методы
BFGS, - BFGS
BHHH, - BHHH
convergence criteria, - критерий сходимости
DFP, - DFP алгоритм, алгоритм Дэвидона-Флетчера-Паулла
expectation maximization, - максимизация ожидания
fast simulated annealing, - быстрый алгоритм имитации отжига
Gauss-Newton, - Гаусса-Ньютона
line search, - поиск на прямой
Newton-Raphson, - Ньютона-Рафсона
numerical derivatives, - численные производные
simulated annealing, - алгоритм имитации отжига
starting values, - стартовые значения
step size adjustment, - корректировка величины шага
IV estimator. See instrumental variables - IV оценка, см. оценка метода инструментальных переменных
jackknife, - джекнайф
bias estimate, - оценка смещения
bias-corrected estimator, - оценка, скорректированная на смещение
example, - пример
IV estimator, - IV оценка
standard error estimate, - оценка стандартной ошибки
Jensen’s inequality, - неравенство Йенсена
jittered data, - данные с добавлением искусственного шума
joint duration distributions, - совместное распределение времени жизни
copulas, - копулы
mixtures, - смеси
multivariate hazard function, - многомерная функция риска
multivariate survivor function, - многомерная функция выживания
joint limits, - совместные пределы
joint versus separate tests, - совместные тесты в сравнении с отдельными тестами
just identification, - точная идентифицируемость
Kaplan-Meier (KM) estimator, - КМ-оценка, оценка Каплан-Мейера
application, - приложение
for baseline hazard, - для базового риска
confidence bands for, - доверительные интервалы
definition, - определение
tied data, -  идентичные данные
kernel density estimator, - ядерная оценка функции плотности
alternatives to, - альтернативы
application, - приложения
asymptotic distribution, - асимптотическое распределение
bandwidth choice, - выбор ширины окна
bias, - смещение
confidence interval for, - доверительный интервал
consistency, - состоятельность
convergence rate, - скорость сходимости
definition, - определение
derivative estimator, - оценка производной
examples, - примеры
multivariate, - многомерная
Nadaraya-Watson kernel regression estimator, - Надарайа-Уотсона оценка ядерной регрессии
optimal bandwidth, - оптимальная ширина окна
optimal kernel, - оптимальное ядро
variance, - дисперсия
kernel functions, - ядерные функции
comparison, - сравнение
definition, - определение
higher-order, - высокого порядка
leading examples, - основные примеры
optimal for density estimation, - оптимальные значения для оценивания плотности
properties, - свойства
kernel matching, - ядерное сопоставление
kernel regression estimator, - ядерная оценка регрессии
alternatives to, - альтернативы
asymptotic distribution, - асимптотическое распределение
bandwidth choice, - выбор ширины окна
bias, - смещение
bootstrap confidence interval for, - бутстрэп доверительные интервалы
boundary problems, - граничные проблемы
conditional moment estimator, - оценка условных моментов
confidence interval for, - доверительные интервалы
consistency, - состоятельность
convergence rate, - скорость сходимости
definition, - определение
derivative estimator, - оценка производной
introduction to nonparametric regression, - введение в непараметрическую регрессию
multivariate, - многомерная
optimal bandwidth, - оптимальная ширина окна
optimal kernel, - оптимальное ядро
undersmoothing, - недосглаживание
variance, - диспрерсия
see also nonparametric regression - см. также непараметрическая регрессия
Khinchine’s theorem, - Хинчина теорема
KLIC. See Kullback-Liebler information criterion - KLIC. См. Кульбака-Лейблера информационный критерий
KM estimator. See Kaplan-Meier - КМ-оценка. См. Каплан-Мейера оценка
k-NN estimator. See nearest neighbors estimator
Kolmogorov LLN, - Колмогорова ЗБЧ, Колмогорова закон больших чисел
Kolmogorov test, - Колмогора тест
Kullback-Liebler information criterion (KLIC), - KLIC, Кульбака-Лейблера информационный критерий
LAD estimator. See least absolute deviations - оценка минимума абсолютных отклонений
Lagrange multiplier (LM) test - тест множителей Лагранжа, LM тест
asymptotic distribution, - асимптотическое распределение
based on GMM-estimator, - основанный на оценке обобщенного метода моментов
based on m-estimator, - основанный на М-оценке
bootstrap, - бутстрэп
comparison with LR and Wald tests, - сравнение с LR тестом (тестом отношения правдоподобия) и тестом Вальда
computation, - вычисление
definition, - определение
examples, - примеры
for heteroskedasticity, - на гетероскедастичность
in duration models, - в моделях времени жизни
interpretation, - интерпретация
for omitted variables, - на пропущенные переменные
OPG version, - вариант (теста множителей Лагранжа) с внешним произведение градиента
for random effects, - на случайные эффекты
score test, - скор-тест
in Tobit model, - в тобит модели
for unobserved heterogeneity, - на ненаблюдаемую неоднородность
see also hypothesis tests - см. также тестирование гипотез
Laplace approximation, - Лапласа приближение
Laplace distribution, - Лапласа распределение
Laplace transform, - Лапласа преобразование
LATE estimator. See local average treatment effects - LATE оценка. См. локальный средний эффект воздействия
latent class model, - модель латентных классов, модель скрытых классов
see finite mixture models - см. модели конечной смеси распределений
latent variable, - латентная переменная, скрытая переменная
latent variable models - модели с латентными переменными
additive random utility model, - аддитивная модель случайной полезности
binary outcomes, - бинарные исходы
endogenous, - экзогенные
ordered multinomial model, - упорядоченная мультиномиальная модель
see also censored models; truncated models. См. также цензурированные модели; усеченные модели
law of iterated expectations, - закон повторного ожидания
law of large numbers (LLN), - закон больших чисел (ЗБЧ)
definition, - определение
examples of use, - примеры использования
Khinchine’s theorem, - Хинчина теорема
Kolmogorov LLN, - Колмогорова ЗБЧ
Markov LLN, - Маркова ЗБЧ
sampling schemes, - план выборки
strong law, - сильный закон больших чисел
weak law, - слабый закон больших чисел
least absolute deviations (LAD) estimator - оценка наименьших абсолютных значений
application, - применения
asymptotic distribution, - асимптотическое распределение
binary outcome models, - модели бинарного выбора
bootstrap, - бутстрэп
censored LAD, - цензурированная оценка наименьших абсолютных значений
definition, - определение
two-stage LAD, - двухшаговая оценка наименьших абсолютных значений
see also quantile regression - см. также квантильная регрессия
least-squares dummy variable (LSDV) estimator, - МНК-оценка с фиктивными переменными
least-squares dummy variable (LSDV) model, - модель МНК c фиктивными переменными 
least squares (LS) estimators - МНК оценки, оценки метода наименьших квадратов
clustered data, - кластеризованные данные
feasible generalized LS, - доступный обобщенный МНК
generalized LS, - обобщенный МНК
linear, - линейный МНК
nonlinear LS, - нелинейный МНК
ordinary LS, - МНК
panel data, - панельные данные
systems of equations, - системы одновременных уравнений
see also FGLS; FGNLS; OLS; NLS. См. также допустимый ОМНК; допустимый НМНК; МНК; НМНК.
Leave-one-out estimate, - оценка с отбрасыванием отдельных наблюдений
LEF, - Экспоненциальное семейство распределений. 
See linear exponential family, length-biased sampling. См. линейное экспоненциальное семейство, смещенные выборки
Liapounov CLT, - Ляпунова ЦПТ
likelihood-based hypothesis tests, - тестирование гипотез с помощью функции правдоподобия
comparisons of, - сравнение 
definitions, определения
examples, - примеры
see also LM tests; LR tests; Wald tests - см. LM тест (тест множителей Лагранжа), LR тест (тест отношения правдоподобия), Вальда тест
likelihood function, - функция правдоподобия
conditional likelihood function, - условная функция правдоподобия
definition, - определение
joint, - совместная
leading examples, - основные примеры
marginal, - частная
partial, - частичная
likelihood principle, - принцип максимального правдоподобия
likelihood ratio (LR) test - LR тест (тест отношения правдоподобия)
asymptotic distribution, - асимптотическое распределение
based on GMM-estimator, - основанные на оценке обобщенного метода моментов
based on m-estimator, - основанный на М-оценке
comparison with LM and Wald tests, - сравнение с LM тестом (тестом множителей Лагранжа) и тестом Вальда
definition, - определение
examples, - примеры
nonnested models, - невложенные модели
quasi-LR test statistic, - статистика теста квази-отношения правдоподобия
uniformly most powerful test, - равномерно наиболее мощный тест
see also hypothesis tests - см. также тесты
LIML estimator, - оценка метода максимального правдоподобия с ограниченной информацией.  See limited information maximum likelihood. См. метод максимального правдоподобия с ограниченной информацией 
limit distribution, - предельное распределение
see also asymptotic distribution - см. также асимптотическое распределение
limit variance matrix, - предельная ковариационная матрица
definition, - определение
replacement by consistent estimate, - замена на состоятельную оценку
sandwich form, - сэндвич-форма
limited information maximum likelihood (LIML) estimator, - оценка максимального правдоподобия с ограниченной информацией
Lindeberg-Levy CLT, - Линдеберга-Леви ЦПТ
line search, - поиск на прямой
linear exponential family (LEF) models, - модели линейного экспоненциального семейства распределений
conjugate priors, - сопряженное априорное распределение
conditional ML estimator, - оценка условного максимального правдоподобия
consistency, - состоятельность
leading examples, - основные примеры
pseudo-R2, - псевдо R2
residuals, - остатки
tests based on, - тесты основанные на
see also generalized linear models - см. также обобщенные линейные модели

linear panel estimators, - линейные оценки для панельных данных
application, - приложения
Arellano-Bond estimator, - Ареллано-Бонда оценка
between estimator, - оценка between
covariance estimator, - оценка ковариационной матрицы
conditional ML estimator, - оценка условного максимального правдоподобия
differences-in-differences estimator, - оценка разность разностей
error components 2SLS estimator,  - оценка двухшагового МНК со составной ошибкой
error components 3SLS estimator, - оценка трехшагового МНК со составной ошибкой
first differences estimator, - оценка в первых разностях
first differences IV estimator, - оценка инструментальных переменных в первых разностях
fixed effects estimator, - оценка для модели с фиксированными эффектами
fixed effects IV estimators, - оценка инструментальных переменных для модели с фиксированными эффектами
forward orthogonal deviations IV estimator, - оценка инструментальных переменных для модели форвардных ортогональных отклонений
Hausman-Taylor IV estimator, - Хаусмана-Тейлора оценка инструментальных переменных
LSDV estimator, - МНК-оценка с фиктивными переменными
MD estimator, - оценка минимального расстояния
panel bootstrap, - панельные данные
panel GMM estimators, - панельные оценки обобщенного метода моментов
panel-robust inference, - статистические выводы робастные к панельным данным
pooled OLS estimator, - МНК оценка сквозной регрессии
random effects estimator, - оценка для модели со случайными эффектами
random effects IV estimator, - оценка инструментальных переменных для модели со случайными эффектами
within estimator, - оценка within
within IV estimator, - оценка инструментальных переменных within 
linear panel models, - линейные панельные модели
analysis-of-covariance model, - модель анализа ковариации
application, - приложения
between model, - модель between
dynamic models, - динамические модели
endogenous regressors, - эндогенные регрессоры
first differences model, - модель в первых разностях
fixed effects model, - модель с фиксированными эффектами
fixed versus random effects, - сравнение фиксированных и случайных эффектов
forward orthogonal deviations model, - модель форвардных ортогональных отклонений
Hausman-Taylor model, - Хаусмана-Тейлора модель
incidental parameters problem, - проблема мешающих параметров
individual dummies, - индивидуальные дамми-переменные
individual-specific effects model, - модель с индивидуальными эффектами
LSDV model, - модель МНК с фиктивными переменными
minimum distance estimator, - оценка минимального расстояния
mean-differenced model, - модель отклонения от среднего
measurement error, - ошибка измерения
mixed linear models, - смешанные линейные модели
pooled model, - сквозная модель
random effects differenced model, - модель со случайным эффектом в разностях
random effects model, - модель со случайными эффектами
residual analysis, - анализ остатков
strong exogeneity, - сильная экзогенность
time dummies, - дамми на моменты времени
time-invariant regressors, - регрессоры, постоянные во времени
time-varying regressors, - регрессоры, меняющиеся во времени
two-way effects model, - модель с двусторонними эффектами
unbalanced data, - несбалансированные данные
weak exogeneity, - слабая экзогенность
within model, - модель within
see also linear panel estimators
linear probability model, - линейная вероятностная модель
linear programming methods, - методы линейного программирования
linear regression model - линейная регрессионная модель
definition, - определение
linear systems of equations, - системы линейных уравнений
panel data models as, - как способ представления моделей панельных данных
seemingly unrelated regressions, - внешне несвязанные уравнения
simultaneous equations, - системы одновременных уравнений
systems FGLS estimator, - оценка доступного обобщенного МНК для систем уравнений
systems GLS estimator, - оценка обобщенного МНК для систем уравнений
systems GMM estimator, - оценка обобщенного метода моментов для систем уравнений
systems ML estimator, - оценка  метода максимального правдоподобия для систем уравнений
systems OLS estimator, - оценка МНК для систем уравнений
systems 2SLS estimator, - оценка двухшагового МНК для систем уравнений
linearization method, - линеаризация
link function, - функция связи
listwise deletion, - полное удаление наблюдений с пропусками
consistency under MCAR, - состоятельность при условии MCAR
example, - пример
inconsistency under MAR only, - несостоятельность при MAR
Living Standards Measurement Study (LSMS), - Обследование уровня жизни Всемирного Банка
LLN. See law of large numbers - ЗБЧ. См. закон больших чисел
LM test. See Lagrange multiplier test - LM тест. См. тест множителей Лагранжа
local alternative hypotheses, - локальная альтернативная гипотеза
local average treatment effects (LATE) estimator, - LATE-оценка, оценка локального среднего эффекта воздействия
assumptions, - предпосылки
comparison with IV estimator, - сравнение с оценкой метода инструментальных переменных
definition, - определение
heterogeneous treatment effect, - неоднородный эффект воздействия
monotonicity assumption, - предпосылка о монотонности
selection on unobservables, - самоотбор по ненаблюдаемым показателям
Wald estimator, - Вальда оценка
see also ATE; ATET; MTE - см. также ATE, ATET, MTE
local linear regression estimator, - локальная линейная оценка регрессии
local polynomial regression estimator, - локальная полиномиальная оценка регрессии
local running average estimator, - локальная оценка скользящего среднего
local weighted average estimator, - локальная оценка взвешенного среднего
logistic distribution, - логистическое распределение
logistic regression. See logit model - логистическая регрессия. См. логит-модель
logit model, - логит-модель
application, - приложения
as ARUM, - как ARUM модель
clustered data, - кластеризованные данные
definition, - определение
for discrete-time duration data, - для дискретных данных времени жизни
GLM, - обобщенная линейная модель
imputation example, - пример восстановления данных
index function model, - модель индексной функции
marginal effects, - предельные эффекты
measurement error example, - пример с ошибками измерения
ML estimator, - оценка метода максимального правдоподобия
multinomial logit, - мультиномиальная логит-модель
nested logit, - вложенная логит-модель
ordered logit, - упорядоченная логит-модель
panel data, - панельные данные
probit model comparison, - сравнение с логит-моделью
random parameters logit, - логит-модель со случайными параметрами
see also binary outcome models - см. также модели бинарного выбора
log-likelihood function. See likelihood function - логарифмическая функция правдоподобия. См. функция правдоподобия
length-biased sampling, - выборка основанная на длительности
log-logistic distribution, - лог-логистическое распределение
log-normal distribution, - лог-нормальное распределение
log-normal model, - лог-нормальная модель

log-odds ratio, - логарифм отношения шансов
log-sum, - логарифм суммы
log-Weibull distribution. See type 1 extreme value - лог-Вейбулла распределение. См. распределение экстремальных значений 1-го типа
long panel, - длинная панель
longitudinal data. See panel data - лонгитюдные данные. См. панельные данные
loss function, - функция потерь
absolute error, - абсолютная ошибка
asymmetric expected error, - асимметричная ожидаемая ошибка
Bayesian decision analysis, - Байесовское принятие решений
expected, - ожидаемая
KLIC, - KLIC, Кульбака-Лейблера информационный критерий
squared error, - квадратичная ошибка
step, - шаг
Lowess regression estimator, - LOWESS оценка регрессии
application, - приложения
LR test. See likelihood ratio test - LR тест. См. тест отношения правдоподобия
LS estimators. See least squares
LSDV. See least-squares dummy variable - LSDV. См. МНК-оценка с фиктивными переменными
LSMS. See Living Standards Measurement Study - Обследование уровня жизни Всемирного Банка
MAR. See missing at random - MAR. См. случайные пропуски
marginal analysis of panel data, - предельный анализ в панельных данных
marginal effects, - предельные эффекты
in binary outcome models, - в моделях бинарного выбора
calculus method, - аналитические производные
computing, - вычисления
definition, - определение
example, - пример
finite-difference method, - метод конечных разностей
in fixed effects model, - в моделях с фиксированными эффектами
in multinomial models, - в моделях со случайными эффектами
population-weighted, - взвешенный по генеральной совокупности
in sample selection models, - в моделях самоотбора выборки
in single-index models, - в одноиндексных моделях
in Tobit model, - в тобит модели
see also coefficient interpretation - см. также интерпретация коэффициентов
marginal likelihood, - рыночные данные
marginal treatment effects (MTE) estimator, - оценка предельного эффекта воздействия
market-level data, - данные рыночного уровня
Markov chain Monte Carlo (MCMC) methods, - метод Монте Карло по схеме Марковской цепи
convergence, - сходимость
in data augmentation, - при пополнении данных
examples, - примеры
Gibbs sampler, - сэмплирование по Гиббсу
Metropolis algorithm, - Метрополиса алгоритм
Metropolis-Hastings algorithm, - Метрополиса-Гастингса алгоритм
Markov LLN, - ЗБЧ Маркова
Marshall-Olkin method, - метод Маршалла-Олкина
matching assumption, - предпосылка сопоставления
see also overlap assumption - см. также предположение о пересечении
matching estimators, - оценки сопоставления
application, - приложения
assumptions, - предположения
ATE matching estimator, - оценка ATE с помощью сопоставления
ATET matching estimator, - оценка ATET с помощью сопоставления
balancing condition, - балансирующее условие
caliper matching, - циркульное сопоставление
counterfactuals, - контр-фактические значения
exact matching, - точное сопоставление
inexact matching, - неточное сопоставление
interval matching, - интервальное сопоставление
kernel matching, - ядерное сопоставление
nearest-neighbor matching, - сопоставление по методу ближайших соседей
propensity score matching, - сопоставление мер склонности
radius matching, - радиальное сопоставление
selection on observables only, - самоотбор по наблюдаемым показателям
stratification matching, - сопоставление со стратификацией, стратифицированное сопоставление
variance computation, - вычисление дисперсии
maximum empirical likelihood (MEL) estimator, - оценка эмпирического максимального правдоподобия
maximum likelihood (ML) estimator, - оценка максимального правдоподобия
asymptotic distribution, - асимптотическое распределение
conditional ML estimator, - оценка условного максимального правдоподобия
consistency, - состоятельность
definition, - определение
endogenous stratification, - эндогенная стратификация
example, - пример
exogenous stratification, - экзогенная стратификация
MSL estimator, - оценка симуляционного максимального правдоподобия
quasi-ML estimator, - оценка квази-максимального правдоподобия
regularity conditions, - условия регулярности
restricted, - ограниченная
unrestricted, - неограниченная
variance matrix estimation, - оценка ковариационной матрицы
weighted ML estimator, - взвешенная оценка максимального правдоподобия
see also quasi-ML estimator - см. также оценка квази-максимального правдоподобия
maximum rank correlation estimator, - метод максимальной ранговой корреляции
maximum score estimator, - оценка максимального счета
maximum simulated likelihood (MSL) estimator, - оценка симуляционного правдоподобия
asymptotic distribution, - асимптотическое распределение
bias-adjusted MSL, - оценка симуляционного правдоподобия, скорректированная на смещение
compared to MSM, - сравнение с оценкой симуляционного метода моментов
count model examples, - примеры счетных моделей
definition, - определение
example, - пример
multinomial probit model, - мультиномиальная пробит модель
number of simulations, - количество симуляций
random parameters logit model, - логит-модель со случайными параметрами
MCAR. See missing completely at random - MCAR, См. полностью случайные пропуски
MD estimator. See minimum distance estimator - MD оценка. См. оценка минимального расстояния
mean-differenced estimator, - оценка модели отклонения от среднего
mean-differenced model, - модель отклонения от среднего
mean imputation, - заполнение средним
mean integrated squared error (MISE), - средняя интегрированная квадратичная ошибка
mean-scaling estimator, - оценка, масштабирующая по среднему
mean-square convergence, - сходимость в среднем квадратичном
mean substitution. See mean imputation - замещение средним. См. заполнение средним
measurement error - ошибка измерения
in cohort-level data, - в данных уровня когорт
in dependent variable, - в зависимой переменной
in microdata, - в микроданных
in panel data, - в панельных данных
in regressors, - в регрессорах
measurement error models - модели ошибки измерения
measurement error model estimators, - оценки для модели ошибки измерения
attenuation bias, - смещение смягчения
bounds identification, - интервальная идентификация
corrected score estimator, - скорректированная скор-оценка
IV estimator, - оценка инструментальных переменных
linear models, - линейные модели
nonlinear models, - нелинейные модели
OLS estimator inconsistency, - несостоятельность МНК оценки
using additional moment restrictions, - с использованием дополнительных моментных ограничений
using instruments, - с использованием инструментов
using known measurement error variance, - с использованием известной дисперсии ошибки измерения
using replicated data, - с использованием повторных данных
using validation sample, - с использованием тестовой выборки
measurement error models, - модели ошибки измерения
attenuation bias, - смещение смягчения
classical measurement error model, - классическая модель ошибки измерения
dependent variable measured with error, - зависимая переменная измеренная с ошибкой
examples, - примеры
identification, - идентифицируемость
linear models, - линейные модели
multiple regressors, - множественные регрессоры
nonclassical measurement error, - неклассическая ошибка измерения
nonlinear models, - нелинейные модели
panel models, - панельные модели
scalar regressor, - скалярный регрессор
serial correlation, - автокорреляция
variance inflation, - вздутие дисперсии
median regression. See LAD estimator - медианная регрессия. См. оценка наименьших абсолютных отклонений
MEL. See maximum empirical likelihood - Эмпирический метода максимального правдоподобия
m-estimator, - М-оценка
asymptotic distribution, - асимптотическое распределение
clustered data, - кластеризованные данные
definition, - определение
sequential two-step, - последовательная двухшаговая
simulated m-estimator, - симуляционная М-оценка
tests based on, - тесты основанные на
weighted m-estimator, - взвешенная М-оценка
see also extremum estimators - см. также экстремальные оценки

method of moments (MM) estimator - оценка метода моментов
asymptotic distribution, - асимптотическое распределение
definition, - определение
examples, - примеры
see also estimating equations estimator; GMM estimator - см. также оценка оценивающих уравнений, оценка обощенного метода моментов
method of scoring, - метод скоринга
method of simulated moments (MSM) estimator, - оценка симуляционного метода моментов
asymptotic distribution, - асимптотическое распределение
compared to MSL, - сравнение с методом симуляционного правдоподобия
definition, - определение
example, - пример
MNP model, - мультиномиальная пробит-модель
number of simulations, - количество симуляций
method of simulated scores (MSS)  - метод скоринга
estimator for MNP model, - оценка для мультиномиальной пробит модели
method of steepest ascent, - метод наискорейшего подъема
Metropolis algorithm, - Метрополиса алгоритм
Metropolis-Hastings algorithm, - Метрополиса-Гастингса алгоритм
microdata sets, - наборы микроданных
handling, - обработка
leading examples, - основные примеры
microeconometrics overview, - обзор микроэконометрики
midpoint rule, - правило прямоугольников
minimum chi-square estimator, - оценка минимума хи-квадрат
see also Berkson’s minimum chi-square estimator  - см. также Берксона оценка минимум хи-квадрат
minimum distance (MD) estimator, - оценка минимального расстояния
asymptotic distribution, - асимптотическое распределение
bootstrap for, - бутстрэп для
covariance structures, - ковариационные структуры
definition, - определение
equally-weighted, - с равными весами
generalized, - обобщенная
indirect inference, - косвенные статистические выводы
OIR test, - тест на сверх-идентифицирующие ограничения
optimal, - оптимальная
panel data, - панельные данные
relation to GMM, - связь с обобщенным методом моментов
misclassification, - неправильная классификация
MISE. See mean integrated squared error - MISE. См. средняя интегрированная квадратичная ошибка
missing at random (MAR), - случайность пропусков (MAR)
definition, - определение
and ignorable missingness, - и игнорируемые пропуски
relation to MCAR, - связь с MCAR
missing completely at random (MCAR), - полная случайность пропусков (MCAR)
definition, - определение
and ignorable missingness, - и игнорируемые пропуски
relation to MCAR, - связь с MCAR
missing data, - пропущенные данные
deletion methods, - методы удалени
examples, - примеры
ignorable assumption, - предположение об игнорируемости
imputation with models, - восстановление данных, основанное на моделях
imputation without models, - восстановление данных без использования моделей
MAR assumption, - предпосылка случайности пропусков
MCAR assumption, - предпосылка полной случайности пропусков
nonignorable missingness, - неигнорируемые пропуски в данных
see also imputation methods - см. также методы восстановления
misspecification tests. See specification tests - тесты на неправильную спецификацию. См. тесты на спецификацию
mixed estimator, - смешенная оценка
mixed linear model, - смешанная линейная модель
Bayesian methods, - Байесовские методы
FGLS estimator, - оценка доступного обобщенного МНК
fixed parameters, - фиксированные параметры
ML estimator, - ML оценка, оценка максимального правдоподобия
random parameters, - случайные параметры
restricted ML estimator, - ограниченная оценка максимального правдоподобия
nonstationary panel data, - нестационарные панельные данные
prediction, - прогнозирование
see also hierarchical linear model  - см. также иерархическая линейная модель
mixed logit model, - смешенная логит-модель
example, - пример
definition, - определение
see also RPL model - см. также логит-модель со случайными параметрами
mixed proportional hazards (MPH) model, - смешанная модель пропорциональных рисков
Weibull-gamma mixture, - смесь Вейбулла и гамма-распределения
see also mixture models - см. также модели смеси
mixture hazard function, - смешанная функция рисков
mixture models, - модели смеси распределений
application, - приложения
counts, - счетные данные
durations, - время жизни
identification, - идентифицируемость
MSL estimator, - оценка метода симуляционного максимального правдоподобия
multinomial outcomes, - мультиномиальные исходы
multiplicative heterogeneity, - мультипликативная неоднородность
specification tests, - тесты на спецификацию
see also finite mixture models; unobserved heterogeneity. См. также конечные модели смеси, ненаблюдаемая гетерогенность
ML estimator. See maximum likelihood - ML оценка. См. метод максимального правдоподобия
MM estimator. See method of moments - ММ оценка. См. метод моментов
MNL estimator. See multinomial logit - оценка мультиномиальной логит-модели. См. мультиномиальная логит-модель
MNP estimator. See multinomial probit model - оценка мультиномиальной пробит-модели. См. мультиномиальная пробит-модель
diagnostics, - диагностические тесты
binary outcome models, - модели бинарного выбора
duration models, - модели времени жизни
example, - пример
multinomial outcome models, - мультиномиальные модели
pseudo-R2 measures, - псевдо R2 критерии
residual analysis, - анализ остатков
see also model selection methods - см. также методы выбора модели
model misspecification, - неправильная спецификация (мисспецификация) модели
see also endogeneity; functional form misspecification; heterogeneity; omitted values; pseudo-true value. См. также эндогенность; неверная спецификация функциональной формы; гетерогенность, опущенные значения; псевдо-истинное значение
model selection methods - методы выбора модели
Bayesian, - Байесовские
nested models, - вложенные модели
nonnested models, - невложенные модели
order of testing, - порядок тестирования
see also model diagnostics; specification tests - см. также диагностика модели, тесты на спецификацию
moment-based simulation estimators, - симуляционные оценки, основанные на моментах
see MSL estimator; MSM estimator - см. оценка метода симуляционного правдоподобия, оценка симуляционного метода моментов
moment-based tests, - тесты, основанные на методе моментов
See m-tests moment matching. См. соотношение M-тестов
See indirect inference. См. неявные выводы
Monte Carlo integration, - интегрирование с использованием Монте-Карло
direct, - прямое
example, - пример
importance sampling, - сэмплирование по важности
simulators, - симуляторы, вспомогательные оценки
see also quadrature - см. также квадратура (численное интегрирование)
Monte Carlo studies, - исследования с использованием метода Монте-Карло
example, - пример
moving average estimator, - оцека скользящего среднего
moving blocks bootstrap, - блочный бутстрэп
MPH model. See mixed proportional hazards, - модель смешанные пропорциональных рисков. См. смешенные пропорциональные риски
MSL estimator. See maximum simulated likelihood, - Оценка симуляционного максимального правдоподобия. См. метод симуляционного максимального правдоподобия
MSM estimator. See method of simulated moments, - Оценка симуляционного метода моментов. См. симуляционный метод моментов 
MSS estimator. See method of simulated scores, - оценка метода скоринга. См. метод скоринга
MTE. See marginal treatment effects - MTE. См. предельный эффект воздействия
m-tests, - М-тесты
asymptotic distribution, - асимптотическое распределение
auxiliary regressions, - вспомогательная регрессия
bootstrap, - бутстрэп
chi-square goodness of fit, - хи-квадрат критерий согласия
conditional moment test, - тест условных моментов
CM test interpretation, - интерпретация теста условных моментов
computation, - вычисление
definition, - определение
Hausman test, - Хаусмана тест
information matrix tests, - критерий (тест) информационной матрицы
outer-product-of-the-gradient form, - вариант с внешним произведением градиента
overidentifying restrictions test, - тест на сверх-идентифицирующие ограничения
power, - мощность
rank, - ранг
multicollinearity, - мультиколлинеарность
in multinomial probit model, - в мультиномиальной пробит-модели
in panel model, - в панельных моделях
in sample selection model, - в моделях самоотбора выборки
multilevel models. See hierarchical models - многоуровневые модели. См. иерархические модели
multinomial logit (MNL) model, - мультиномиальная логит-модель
application, - приложения
as additive random utility model, - как аддитивная модель случайной полезности
definition, - определение
marginal effects, - предельные эффекты
ML estimator, - оценка метода максимального правдоподобия
panel data, - панельные модели
see also multinomial outcome models - см. также мультиномиальные модели
multinomial outcome models, - мультиномиальные модели
application, - приложения
alternative-invariant regressors, - регрессоры, постоянные для альтернатив 
alternative-varying regressors, - регрессоры, изменяющиеся в зависимости от выбранной альтернативы
conditional logit, - условная логит-модель
definition, - определение
identification, - идентифицируемость
index function model, - модель индексных функций
marginal effects, - предельные эффекты
mixed logit, - смешанная логит-модель
ML estimator, - оценка метода максимального правдоподобия
multinomial logit, - мультиномиальная логит-модель
multinomial probit, - мультиномиальная пробит-модель
ordered models, - модели упорядоченного выбора
OLS estimator, - МНК оценка
panel data, - панельные данные
random parameters logit, - логит-модель со случайными параметрами
random utility model, - модель случайной полезности
semiparametric estimation, - полупараметрическое оценивание
multinomial probit (MNP) model, - мультиномиальная пробит-модель
Bayesian Methods, - Байесовские методы
definition, - определение
identification, - идентифицируемость
ML estimator, - оценка метода максимального правдоподобия
MSL estimator, - оценка метода симуляционного правдоподобия
MSM estimator, - оценка симуляционного метода моментов
MSS estimator, - оценка метода скоринга
see also multinomial outcome models - см. также мультиномиальные модели
multiple duration spells, - многократные длительные события
fixed effects, - фиксированные эффекта
lagged duration dependence, - лаговая зависимость от длительности
ML estimator, - оценка метода максимального правдоподобия
random effects, - случайные эффекты
recurrent spells, - повторяющиеся события
multiple imputation, - множественное восстановление
estimator, - оценка
examples, - примеры
relative efficiency, - относительная эффективность
variance of estimator, - дисперсия оценки
multiple treatments, - множественные воздействия
multiplicative errors - мультипликативная ошибка
multistage surveys, - многоэтапные опросы 
variance estimation, - оценка дисперсии
multivariate data - многомерные данные
binary outcomes, - бинарные исходы
counts, - счетные данные
durations, - длительности
see also systems of equations - см. также системы уравнений
multivariate-t distribution, - многомерное t-распределение
NA estimator. See Nelson-Aalen - Нельсона-Аалена оценка. См. Нельсон-Аален
National Longitudinal Survey (NLS), - Национальное Лонгитюдное Обследование
National Longitudinal Survey of Youth (NLSY), - Национальное Лонгитюдное Обследование Молодежи
National Supported Work (NSW) demonstration project, - Национальная программа поддержки занятости
natural conjugate pair, - сопряженное распределение
natural experiments, - естественные эксперименты
definition, - определение
differences-in-differences estimator, - оценка разность разностей
examples, - примеры
exogenous variation, - экзогенная изменчивость
identification, - идентифицируемость
instrumental variables, - инструментальные переменные
regression discontinuity design, - разрывный дизайн
ncp. See noncentrality parameter. См. параметр нецентральности
nearest neighbors (k-NN) estimator, - оценка ближайших соседей (k-NN оценка)
definition, - определение
example, - пример
symmetrized, - симметризованный
see also nonparametric regression - см. также непараметрическая регрессия
nearest-neighbor matching, - сопоставление по ближайшим соседям
negative binomial distribution, - отрицательное биномиальное распределение
negative binomial model, - отрицательная биномиальная модель
application, - приложения
bivariate, - двумерная
hurdle model, - модель преодоления порогов
ML estimator, - оценка максимального правдоподобия
MSL estimator, - оценка симуляционного максимального правдоподобия
NB1 variant, - NB1, отрицательная биномиальная модель 1
NB2 variant, - NB2, отрицательная биномиальная модель 2
panel data, 804, - панельные данные
negative hypergeometric distribution, - отрицательное гипергеометрическое распределение
neglected heterogeneity. See unobserved heterogeneity - См. ненаблюдаемая неоднородность

Nelson-Aalen (NA) estimator, - Нельсона-Аалена оценка
application, - приложения
confidence bands for, - доверительные интервалы
definition, - определения
tied data, - идентичные данные
nested bootstrap, - вложенный бутстрэп
nested logit model, - вложенная логит-модель
from ARUM, - для ARUM модели
definition - определение
different versions of, - различные версии
example, - пример
GEV model, - модель, основанная на обобщенном распределении экстремальных значений стандартной ошибки
ML estimator, - оценка метода максимального правдоподобия
sequential estimator, - последовательная оценка
welfare analysis, - анализ благосостояния
see also multinomial models  - см. также мультиномиальные модели
nested models - вложенные модели
see also nonnested models - см. также невложенные модели
neural network models, - модели нейронных сетей
Newey-West robust standard errors, - Ньюи-Веста робастные стандартные ошибки
definition, - определение
see also robust standard errors - см. также робастные стандартные ошибки
Newton-Raphson (NR) method, - Ньютона-Рафсона алгоритм
examples, - примеры
NLFIML estimator. See nonlinear full-information maximum likelihood - оценка нелинейного максимального правдоподобия с полной информацией
NLS estimator. See nonlinear least squares. См. оценка нелинейного МНК
NLSY. See National Longitudinal Survey of Youth, NL2SLS estimator. См. Национальное Лонгитюдное Обследование Молодежи, оценка нелинейного двухшагового МНК  
See nonlinear two-stage least squares. См. нелинейный двухшаговый МНК.
NL3SLS estimator. See nonlinear three-stage least squares - Оценка нелинейного трехшагового МНК
noise-to-signal ratio, - отношение шум-сигнал
noncentral chi-square distribution, - нецентрированное хи-квадрат распределение
noncentrality parameter (ncp), - параметр нецентральности
nonclassical measurement error, - неклассическая ошибка измерения
nongradient methods, - неградиентные методы
nonignorable missingness, - неигнорируемые пропуски в данных
attrition bias due to, - смещение истощения
selection bias due to, - смещение самоотбора
nonlinear estimators - нелинейные оценки
coefficient interpretation, - интерпретация коэффициентов
extremum estimator - экстремальная оценка
m-estimator, - М-оценка
GMM estimator, - ОММ оценка (оценка обобщенного метода моментов)
ML estimator, - оценка метода максимального правдподобия
NLS estimator, - оценка нелинейного МНК
overview, - обзор
panel models, - панельные модели
nonlinear full-information maximum likelihood (NLFIML) estimator, - оценка нелинейного максимального правдоподобия с полной информацией
nonlinear GMM estimator, - оценка нелинейного обобщенного метода моментов
asymptotic distribution, - асимптотическое распределение
definition, - определение
example, - пример
instrument choice, - выбор инструментов
NL2SLS estimator, - оценка нелинейного двухшагового МНК
optimal, - оптимальная
panel data, - панельные данные
nonlinear in parameters, - нелинейность по параметрам
nonlinear in variables, - нелинейность по переменным
nonlinear IV estimator. See nonlinear GMM - нелинейная оценка инструментальным переменных. См. нелинейный обобщенный метод моментов
nonlinear least squares (NLS) estimator, - оценка нелинейного МНК
asymptotic distribution, - асимптотическое распределение
consistency, - состоятельность
definition, - определение
example, - пример
time series, - временные ряды
variance matrix estimation, - оценка ковариационной матрицы
nonlinear panel estimators, - нелинейные панельные оценки
application, - приложения
conditional ML estimator, - оценка условного максимального правдоподобия
dummy variable estimator, - оценка с помощью дамми-переменных
first-differences estimator, - оценка в первых разностях
fixed effects estimator, - оценка модели с фиксированными эффектами
GEE estimator, - обобщенная оценка оценивающих уравнений
mean-differenced estimator, - оценка модели отклонения от среднего
mean-scaling estimator, - оценка, масштабирующая по среднему
ML estimator, - оценка метода максимального правдоподобия
NLS estimator, - оценка нелинейного МНК
panel GMM estimator, - панельная оценка обобщенного метода моментов
panel-robust inference, - статистические выводы робастные к панельным данным
quadrature, - квадратура (численное интегрирование)
quasi-differenced estimator, - оценка в квази-разностях
quasi-ML estimator, - оценка метода квази-максимального правдоподобия
random effects estimator, - оценка модели со случайными эффектами
selection models, - модели самоотбора
semiparametric, - полупараметрические
nonlinear panel models, - нелинейные панельные модели
application, - приложения
binary outcome models, - модели бинарного выбора
conditional mean models, - модели условного среднего
count models, - счетные модели
dynamic models, - динамические модели
endogenous regressors, - эндогенные регрессоры
exogeneity assumptions, - предположения об эндогенности
finite mixture models, - модели конечной смеси
fixed effects models, - модели с фиксированными эффектами
fixed versus random effects, - сравнение фиксированных и случайных эффектов
incidental parameters problem, - проблема мешающих параметров
individual-specific effects models, - модели с индивидуальными эффектами
parametric models, - параметрические модели
pooled models, - сквозные модели
random effects models, - модели со случайными коэффициентами
selection models, - модели самоотбора
semiparametric, - полупараметрические
Tobit models, - тобит модели
transition models, - модели перехода
nonlinear regression model, - нелинейные регрессионные модели
additive error, - аддитивные ошибки
nonadditive error, - неаддитивные ошибки
nonlinear systems of equations, - нелинейные системы уравнений
additive errors, - аддитивные ошибки
copulas, - копулы
mixtures, - смеси распределений
ML estimator, - оценка метода максимального правдоподобия
NLFIML estimator, - оценка нелинейного метода максимального правдоподобия с полной информацией
NL3SLS estimator, - оценка нелинейного трехшагового МНК
nonadditive errors, - неаддитивные ошибки
nonlinear panel model, - нелинейные панельные модели
nonlinear SUR model, - нелинейные внешне несвязанные уравнения
quasi-ML estimator, - оценка метода квази-максимального правдоподобия
seemingly unrelated regressions, - внешне несвязанные уравнения
simultaneous equations, - одновременные уравнения
systems FGNLS estimator,  - оценка доступного обобщенного нелинейного МНК для систем уравнений
systems GMM estimator, - оценка обобщенного метода моментов для систем уравнений
systems IV estimator, - оценка инструментальных переменных для систем уравнений
systems MM estimator, - оценка метода моментов для систем уравнений
systems NLS estimator, - оценка нелинейного МНК для систем уравнений
nonlinear three-stage least squares (NL3SLS) estimator, - оценка нелинейного трехшагового МНК
nonlinear two-stage least squares (NL2SLS) estimator - оценка нелинейного двухшагового МНК
asymptotic distribution, - асимптотическое распределение
definition, - определение
example, - пример
see also nonlinear GMM estimator  - см. также оценка нелинейного обобщенного метода моментов
nonnested models - невложенные модели
Cox LR test, - Кокса тест отношения правдоподобия
definition, - определение
example, - пример
information criteria comparison, - сравнение информационных критериев
overlapping, - пересекающиеся
strictly nonnested, - строго невложенные
Vuong LR test, - Вуонга тест отношения правдоподобия
nonparametric bootstrap. See paired bootstrap - непараметрический бутстрэп. См. парный бутстрэп
nonparametric density estimation. - непараметрическое оценивание плотности
See kernel density estimator - См. ядерная оценка функции плотности
nonparametric maximum likelihood (NPML) - непараметрический метод максимального правдоподобия
estimator, - оценка
nonparametric regression, - непараметрическая регрессия
convergence rate, - схорость сходимости
kernel, - ядро
local linear, - локальная линейная
local weighted average, - локальное взвешенное среднее
Lowess, - LOWESS
nearest-neighbors, - ближайших соседей
series, - ряды
statistical inference intuition, - интуиция за статистическими выводами
test against parametric model, - тестирование против параметрическое модели
see also semiparametric regression - см. также полупараметрическая регрессия
nonrandomly varying coefficient, - неслучайно изменяющийся коэффициент
normal copula, - нормальная копула
normal distribution, - нормальное распределение
truncated moments, - усеченные моменты
normal limit product rule, - нормальность предела произведения. See Cramer linear transformation. См. линейное преобразование Крамера
NPML estimator, оценка непараметрического метода максимального правдоподобия. See nonparametric maximum likelihood - См. непараметрический метод максимального правдоподобия
NR method. See Newton-Raphson method - См. Ньютона-Рафсона метод
NSW demonstration project. See National Supported Work nuisance parameters. See incidental parameters - мешающие параметры

numerical derivatives,  - численные производные
numerical integration. See quadrature - численное интегрирование (квадратура)
observational data, - данные наблюденй
biased samples, - смещенные выборки
clustering, - кластеризация
identification strategies, - стратегия идентификации
measurement error, - ошибка измерения
missing data, - пропущенные данные
population, - генеральная совокупность
sample attrition, - истощение выборки
sampling methods, - методы построения выборки
sampling units, - единицы выборки
sampling without replacement, - выборка без повторения
survey methods, - методы опросов
survey nonresponse, - отсутствие ответа на опрос
types of data, - типы данных
observational equivalence, - эквивалентность
odds ratio, - отношение шансов
see also posterior odds ratio - см. также апостериорное отношение шансов
OIR test. See overidentifying restrictions test - OIR тест. См. тест на сверх-идентифицирующие ограничения
OLS estimator. See ordinary least squares - МНК оценка. См. метод наименьших квадратов
omitted variables bias, - смещение из-за пропущенных переменных
LM tests for, - LM тест (тест множителей Лагранжа)
one-step GMM estimator, - одношаговая оценка методом моментов
panel, - панель
see also two-stage least squares - см. также двухшаговый метод наименьших квадратов
one-way individual-specific effects model. See individual-specific effects model - модель с односторонними индивидуальными эффектами. См. модель с индивидуальными эффектами
on-site sampling, - отбор на месте
optimal Bayesian estimator, - оптимальная Байесовская оценка
optimal GMM estimator, - оптимальная оценка обобщенного метода моментов
compared to 2SLS, - сравнение с двухшаговым МНК
optimal MD estimator, - оптимальная оценка минимального расстояния
OPG. See outer-product of the gradient - См. внешнее произведение градиента
Orbit model, - орбит-модель
order of magnitude, - порядок малости
ordered logit model, - упорядоченная логит-модель
ordered multinomial models, - упорядоченные мультиномиальные модели
ordered probit model, - упорядоченная пробит-модель
ordinary least squares (OLS) estimator, - МНК оценка (оценка метода наименьших квадратов)
asymptotic distribution, - асимптотическое распределение
bias in standard errors with clustering, - смещение стандартных ошибок при кластеризованных данных
binary data, - бинарные данные
clustered data, - кластеризованные данные
coefficient interpretation in misspecified model, - интерпретация коэффициентов в неправильно специфицированной модели
consistency - состоятельность
definition, - определение
example, - пример
finite-sample distribution, - распределение в конечных выборках
heteroskedasticity-robust standard errors, - стандартные ошибки устойчивые к гетероскедастичности
identification, - идентификация
inconsistency, - несостоятельность
inefficiency, - неэффективность
nonlinear, - нелинейная
panel data, - панельные данные
see also least squares estimators - см. также оценки метода наименьших квадратов
orthogonal polynomials, - ортогональные многочлены
definition - определение
orthogonal regression approach, - подход ортогональной регрессии
orthonormal polynomials, - ортонормальные многочлены
outcome equation, - уравнение результата
outer product (OP) estimate, - оценка в внешним произведением
outer-product of the gradient (OPG) version - вариант с внешним произведением градиента
LM test, - LM тест (тест множителей Лагранжа)
m-test, - М-тест
small-sample performance, - результативность в малых выборках
overdispersion, - избыточная дисперсия
measurement error, - ошибка измерения
panel data, - панельные данные
tests for, - тесты на
overidentification, - сверх-идентифицируемость
see also GMM estimator - см. также оценка обобщенного метода моментов
overidentifying restrictions (OIR) test - тест на сверх-идентифицирующие ограничения
asymptotic distribution, - асимптотическое распределение
bootstrap, - бутстрэп
definition, - определение
panel data, - панельные данные
overlap assumption, - предположение о пересечении
in RD design, - при разрывном дизайне
oversampling, - избыточная выборка
paired bootstrap, - парный бутстрэп
pairwise deletion, - попарное удаление
biased standard errors, - смещенные стандартные ошибки
panel attrition, - истощение панели
panel bootstrap, - панельные бутстрэп
panel data, - панельные данные

panel data models and estimators, - панельные данные и оценки
comparison to clustered data, - сравнение с кластеризованными данными
see also linear panel; nonlinear panel GMM estimators. См. также линейные модели анализа панельных данных; нелинейные панельные оценки обобщенного метода моментов
application, - приложения
Arellano-Bond estimator, - Ареллано-Бонда оценка
asymptotic distribution, - асимптотическое распределение
bootstrap, - бутстрэп
compared to MD estimator, - сравнение с оценкой минимального расстояния
computation, - вычисление
definition, - определение
efficiency, - эффективность
exogeneity assumptions, - предположения об экзогенности
instruments, - инструменты
IV estimators for FE model, - оценки инструментальных переменных для модели с фиксированными эффектами
IV estimators for RE model, - оценки инструментальных переменных для модели со случайными эффектами
just-identified, - точно идентифицированные
nonlinear, - нелинейные
OIR test, - тест на сверх-идентифицирующие ограничения
one-step GMM estimator, - одношаговая ОММ оценка (одношаговая оценка обобщенного метода моментов)
overidentified, - сверх-идентифицированные
2SLS estimator, - двухшаговая МНК оценка
two-step GMM estimator, - двухшаговая ОММ оценка (двухшаговая оценка обобщенного метода моментов)
variance matrix estimation, - оценивание ковариационной матрицы

panel GMM model, - панельная модель с использованием обобщенного метода моментов
application, - приложения
dynamic, - динамическая
with individual-specific effects, - с индивидуальными эффектами
without individual-specific effects, - без индивидуальных эффектов
see also panel GMM estimators - см. также панельные оценки обобщенного метода моментов
panel IV estimators. See panel GMM estimators - панельные оценки инструментальных переменных. См. панельные оценки обобщенного метода моментов
panel-robust statistical inference, - статистические выводы, робастные для панельных данных
for Hausman test, - для теста Хаусмана
Panel Study in Income Dynamics (PSID), - Панельное исследование динамики доходов
parametric bootstrap, - параметрический бутстрэп
Pareto distribution - Парето распределение
of the first kind, - первого типа
of the second kind, - второго типа
partial additive model, - частичная аддитивная модель
partial equilibrium analysis, - анализ частичного равновесия
see also SUTVA - см. также предположение о стабильности величины воздействия
partial F-statistic, - частичная F-статистика
partial likelihood estimator, - оценка частичного метода максимального правдоподобия
partial ML estimator, - оценка частичного метода максимального правдоподобия
partial R-squared, - частичный R-квадрат
partially linear model, - частично линейная модель
participation equation, - уравнение участия
Pearson chi-square goodness-of-fit test, - Пирсона критерий согласия
Pearson residual, - Пирсона остатки
peer-effects model, - модель со взаимными эффектами
percentile, - перцентиль
percentile method, - перцентильный метод
percentile-t method, - $t$-перцентильный метод
PH model, - модель пропорциональных рисков. See proportional hazards. См. модель пропорциональных рисков
piecewise constant hazard model, - модель кусочно постоянной функции риска 
Pitman drift, - Питмана дрейф
PML estimator. See pseudo-ML estimator - оценка псевдо-максимального правдоподобия
Poisson distribution, - Пуассона распределение
Poisson-gamma mixture, - смесь Пуассоновского и гамма-распределений
Poisson-IG mixture, - смесь Пуассоновсокго и обратного гамма-распределений

Poisson regression model, - Пуассоновская регрессионная модель
application, - применения
asymptotic distribution of estimators, - асимптотическое распределение оценок
bivariate, - двумерная
censored MLE, - цензурированные оценки максимального правдоподобия
with clustered data, - с кластеризованными данными
coefficient interpretation, - интерпретация коэффициентов
definition, - определение
equidispersion, - равенство математического ожидания и дисперсии
example, - пример
LEF density, - плотность из линейного экспоненциального семейства
measurement error, - ошибка измерения
mixtures, - смеси 
ML estimator, - оценка максимального правдоподобия
overdispersion, - избыточная дисперсия

panel data, - панельные данные
quasi-ML estimator, - оценка квази-максимального правдоподобия
truncated MLE, - усеченная оценка максимального правдоподобия
underdispersion, - недостаточная дисперсия
zero-truncated, - усеченная в нуле
see also count models - см. также счетные модели
polynomial baseline hazard, - полиномиальный базовый риск
pooled cross-section time series model. See pooled model - сквозная модель
pooled estimators, - сквозная оценка
application, - приложения
FGLS estimator, - оценка доступного обобщенного МНК
GEE estimator, - обобщенная оценка оценивающих уравнений
NLS estimator, - оценка нелинейного МНК
OLS estimator, - МНК оценка
WLS estimator, - оценка взвешенного МНК
pooled model, - сквозная модель
pooling tests, - тесты на объединение
population-averaged model, - модель, усредненная по генеральной совокупности. See pooled model - Сквозная модель
population moment conditions - условия на теоретические моменты 
for estimation, - для оценивания
for testing, - для тестирования
see also GMM estimator; MM estimator; m-tests - см. также оценка обобщенного метода моментов, оценка метода моментов, М-тесты
posterior distribution, - апостериорное распределение
asymptotic behavior, - асимптотическое поведение
conditional posterior, - условное апостериорное распределение
definition, - определение
expected posterior loss, - ожидаемые апостериорные потери
expected posterior risk, - ожидаемый апостериорный риск
full conditional distribution, - полное условное распределение
highest posterior density interval, - интервал наивысшей апостериорной плотности
highest posterior density region, - область наивысшей апостериорной плотности
marginal posterior, - частное апостериорное распределение
observed-data posterior, - апостериорная плотность по наблюдаемым данным
posterior density interval, - апостериорный доверительный интервал
posterior mean, - среднее апостериорного распределения
posterior mode, - мода апостериорного распределения
posterior moments, - апостериорные моменты
posterior precision, - апостериорная точность
see also Bayesian methods - см. также Байесовские методы
posterior odds ratio, - апостериорное отношение шансов
posterior (P) step, - апостериорный шаг, P-шаг 
potential outcome model, - модель потенциального результата
see also treatment effects; treatment evaluation - см. также эффект воздействия, оценка эффекта воздействия
power of tests, - модность тестов
bootstrapped tests, - бутстрэп тесты
conditional moment test, - тест условных моментов
example, - пример
Hausman test, - Хаусмана теста
local alternative hypotheses, - локальные альтернативные гипотезы
uniformly most powerful test, - равномерно наиболее мощный тест
Wald tests, - Вальда тесты
precision parameter, - параметр точности
predetermined instruments. See weak exogeneity - предопределенный инструменты. См. слабая экзогенность
prediction, - прогноз
best linear, - наилучший линейный
conditional, - условный
error, - ошибка
in linear panel models, - в линейных моделях
in mixed linear model, - в смешанных линейных моделях
optimal, - оптимальный
rotation groups, - ротационные группы
in structural model, - в структурной модели
weighted, - взвешенные
pretest estimator, - претест оценка
primary sampling units (PSUs), - первичная единица выборки (ПЕВ)
prior distribution, - априорное распределение
conjugate prior, - сопряженное априорное распределение
definition, - определение
Dickey’s prior, - априорное распределение Дики
diffuse prior, - неинформативное априорное распределение
flat prior, - плоское априорное распределение
hierarchical priors, - иерархическое априорное распределение
improper prior, - несобственное априорное распределение
informative prior, - информативное априорное распределение
Jeffreys’ prior, - априорное распределение Джеффри
noninformative prior, - неинформативное априорное распределение
normal-gamma prior, - нормальное-гамма априорное распределение
sensitivity analysis for, - анализ чувствительности
see also Bayesian methods - см. также Байесовские методы
probit model, - пробит-модель
application, - приложения
as additive random utility model, - как аддитивная модель случайной полезности
bivariate probit, - двумерная пробит-модель
bootstrap example, - пример бутстрэпа
definition, - определение
discrete-time duration data, - данные продолжительности жизни в дискретном времени
as GLM, - как обобщенная линейная модель
index function model, - модель индексных функций
logit model comparison, - сравнение с логит-моделью
marginal effects, - предельные эффекты
ML estimator, - оценка метода максимального правдоподобия
Monte Carlo study example, - пример исследования с методом Монте-Карло
multinomial probit, - мультиномиальная пробит-модель
ordered probit, - упорядоченная пробит-модель
panel data, - панельные данные
simultaneous equations probit, - система одновременных пробит-моделей
see also binary outcome models - см. также модели бинарного выбора
probit selection equation, - пробит уравнение самоотбора
product copula, - копула произведения
product integral, - произведение интегралов
product rule, - правило произведения
see also Cramer linear transformation program evaluation. See treatment evaluation projection pursuit model, - модель поиска наилучшей проекции
propensity score, - мера склонности
application, - приложения
balancing condition,  - балансирующее условие
conditional independence assumption, - предположение об условной независимости
definition, - определение
matching, - сопоставление
see also treatment evaluation - см. также оценка воздействия
proportional hazards (PH) model, - модель пропорциональных рисков
application, - приложения
baseline survivor function estimator, - оценка базовой функции выживания 
coefficient interpretation, - интерпретация коэффициентов
competing risks model, - модель конкурирующих рисков 
definition, - определение
discrete-time model, - модель в дискретном времени
leading examples, - основные примеры
mixed PH, - смешанные пропорциональные риски
panel data, - панельные данные
partial likelihood estimator, - оценка частичного правдоподобия
pseudo-ML estimator (PML). See quasi-ML estimator - метод псевдо-максимального правдоподобия

pseudo panels, - псевдо панели
cohort, - когорта
cohort fixed effects, - фиксированные эффекты когорт
measurement error, - ошибка измерения
pseudo-random number generators, - генератор псевдо-случайных чисел
accept-reject methods, - методы принятия-отбрасывания
composition methods, - методы комбинирования
inverse transformation method, - метод обратного преобразования
leading distributions, - основные законы распределения
multivariate normal, - многомерное нормальное распределение
transformation method, - метод преобразования
uniform variates, - равномерно распределенные псевдослучайные значения
see also MCMC methods pseudo R-squared measures
for binary outcome models, - для моделей бинарного выбора
definitions, - определения
example, - пример
for multinomial outcome models, - для мультиномиальных моделей
pseudo-true value, - псевдо-истинное значение
PSID. See Panel Study in Income Dynamics - Панельное исследование динамики доходов
PSUs. See primary sampling units - См. первичная единица выборки (ПЕВ)
pure exogenous sampling, - чисто экзогенная выборка
p-value, - P-значение

quadrature, - численное интегрирование
Gaussian, - методом Гаусса
multidimensional, - многомерное
in nonlinear panel models, - в нелинейных панельных моделях
see also Monte Carlo integration - см. также интегрирование с использованием Монте-Карло
qualititative response models. See binary outcomes, multinomial outcomes
quantile, - квантиль
quantile regression, - квантильная регрессия
application, - приложения
asymmetric absolute loss, - асимметричные абсолютные потери
asymptotic distribution, - асимптотическое распределение
bootstrap, - бутстрэп
computation, - вычисление
definition, - определение
IV estimator, - оценка инструментальных переменных
multiplicative heteroskedasticity, - мультипликативная гетероскедастичность
quasi-difference, - квази-разность
quasi-experiment. See natural experiment - квази-эксперимент. См. естественный эксперимент
quasi-maximum likelihood (QML) estimator, - оценка квази-максимального правдоподобия
asymptotic distribution, - асимптотическое распределение
in binary outcome models, - в моделях бинарного выбора
in clustered models, - в моделях кластеризованных данных
definition, - определение
in LEF, - для линейного экспоненциального семейства
with multivariate dependent variable, - с многомерной зависимой переменной
in nonlinear systems, - в нелинейных системах
in panel models, - в панельных моделях
in Poisson model, - в модели Пуассона
quasi-random numbers. See pseudo-random numbers - квази-случайные числа. См. псевдо-случайные числа
QML estimator. See quasi-ML estimator - См. оценка квази-максимального правдоподобия
random assignment, - случайное назначение
see also sampling schemes - см. также план выборки
random coefficients model, - модель со случайными коэффициентами
see also hierarchical models - см. также иерархические модели
random effects (RE) estimator, - оценка RE модели (оценка модели со случайными эффектами)
application, - применения
asymptotic distribution, - асимптотическое распределение
clustered data, - кластеризованные данные
consistency, - состоятельность
definition, - определение
error components 2SLS estimator, - оценка двухшагового МНК со ставной ошибкой    
error components 3SLS estimator, - оценка трехшагового МНК со ставной ошибкой    
FGLS estimator, -  оценка доступного обобщенного МНК
GEE estimator, - обобщенная оценка метода оценивающих уравнений
Hausman test, - тест Хаусмана
incidental parameters, - мешающие параметры
IV estimators, - оценки инструментальных переменных
ML estimator, - оценка максимального правдоподобия
NLS estimator, - нелинейная МНК оценка
quasi-ML estimator, - оценка квази максимального правдоподобия
two-way effects model, - модель с двусторонними эффектами
versus fixed effects, - сравнение с фиксированными эффектами
random effects (RE) model, - модель со случайными эффектами (RE модель)
binary outcome models, - модель бинарного выбора
Chamberlain model, - Чемберлена модель
clustered data, - кластеризованные данные
count models, - счетные модели
definition, - определение
dynamic models, - динамические модели
duration models, - модели времени жизни
endogenous regressors, - эндогенные регрессоры
Mundlak model, - модель Мундлака
nonlinear models, - нелинейные модели
selection models, - модели самоотбора
Tobit model, - тобит модель
two-way effects model, - модель с двусторонними эффектами
versus random effects, - сравнение со случайными эффектами
see also hierarchical models; random effects
estimator
random number generators. See pseudo-random numbers - генераторы случайных чисел. См. псевдослучайные числа
random parameters logit (RPL) model, - модель логит со случайными параметрами
Bayesian methods, - Байесовские методы
definition, - определение
ML estimator, - оценка метода максимального правдоподобия
random parameters model. See random coefficients model - модель со случайными параметрами. См. модель со случайными коэффициентами
random utility models. See ARUM - модели случайной полезности. См. ARUM
randomization bias, - смещение рандомизации
randomized experiment, - рандомизированный эксперимент
National Supported Work demonstration project, - проект по Национальной поддержке работы
randomized trials, - рандомизированные эксперименты
randomly varying coefficient, - случайно изменяющийся коэффициент
rank condition for identification, - условие ранга для идентифицируемости
rank-ordered logit model, - логит-модель с ранжированными исходами
rank-ordered probit model,  - пробит-модель с ранжированными исходами
raw residual, - исходные остатки
RD design. See regression discontinuity design - разрывный дизайн
receiver operators characteristics (ROC) curve, - кривая рабочей характеристики приемника (ROC-кривая)
reduced form, - приведенная форма
see also structural model - см. также структурные модели
RE estimator. See random effects - оценка со случайными эффектами. См. случайные эффекты
regression-based imputation, - восстановление основанное на регрессии
EM algorithm, - EM алгоритм
nonignorable missingness, - неигнорируемые пропуски в данных
regression discontinuity (RD) design, - разрывный дизайн
fuzzy RD design, - нечеткий разрывный дизайн
heterogeneous treatment effects, - неоднородный эффект воздействия
RD estimator, - оценка разрывного дизайна
sharp RD design, - четкий разрывный дизайн
treatment assignment mechanism, - механизм назначения воздействия
regressors, - регрессоры
alternative-varying, - изменяющиеся в зависимости от альтернативы
endogenous, - эндогенные
fixed, - фиксированные (неслучайные)
irrelevant, - лишние
omitted, - пропущенные
stochastic, - стохастические
time-varying, - меняющиеся во времени
see also endogenous regressors - см. также эндогенные регрессоры
regularity conditions for ML, - условия регулярности для максимального правдоподобия
relative risk, - относительный риск
reliability ratio, - отношение стабильности
renewal function, - функция восстановления
renewal process, - процесс восстановления
repeated cross section data, - повторные пространственые данные
see also differences-in-differences - см. также разность разностей
repeated measures. See panel data - повторные измерения. См. панельные данные
replicated data, - реплицированные данные
RESET test, - RESET тест
residual analysis - анализ остатков
definitions, - определения
duration data, - данные времени жизни
example, - пример
panel data, - панельные данные
small-sample correction, - корректировка для выборок малого размера
residual bootstrap, - бутстрэп остатков
response-based sampling, - выборка на основе ответов
restricted ML estimator, - ограниченная оценка максимального правдоподобия
revealed preference data, - данные выявленных предпочтений
ridge regression estimator, - оценка ридж-регрессии
Robinson difference estimator, - оценка разностей Робинсона
robust sandwich variance matrix estimate. See sandwich variance matrix - робастная оценка ковариционной матрицы в сэндвич форме. См. оценка ковариационной матрицы в сэндвич-форме
robust standard errors - робастные стандартные ошибки
bootstrap, - бутстрэп
Eicker-White, - Эйкера-Уайта
for extremum estimator, - для экстремальной оценки
Huber-White, - Хубера-Уайта
Newey-West, - Ньюи-Веста
see also cluster-robust; heteroskedasticity-robust; panel-robust; systems-robust - см. также робастные к кластерам, робастные к гетероскедастичности, робастные к панельным данным, робастные к системам уравнений
ROC curve. See receiver operators characteristics curve - ROC-кривая. См. кривая рабочей характеристики приёмника

rotating panels, - чередующиеся панели
Roy model, - Роя модель
definition, - определение
dummy endogenous variable, - эндогенная дамми-переменная
Heckman two-step estimator, - двухшаговая оценка Хекмана
ML estimator, - оценка метода максимального правдоподобия
panel semiparametric estimation, - полупараметрическое оценивание для панельных данных
as treatment effects model, - как модель эффекта воздействия
RPL model. See random parameters logit - логит-модель со случайными параметрами
R-squared, - R-квадрат
pseudo, - псевда
uncentered, - нецентрированный
running mean estimator, - оценка скользящего среднего
SA method. See simulated annealing - метод имитация отжига
sample attrition, - истощение выборки
sample moment conditions - условия на выборочные моменты
see population moment conditions - см. условия на теоретические моменты
sample selection bias, - смещение самоотбора
sample weights, - выборочные веса
see also weighting sampling schemes - см. также взвешенные выборки
assumptions for OLS, - предпосылки МНК
case-control, - случай-контроль
choice-based sampling, - самоотбор выборки, отбор при построении выборки
endogenous sampling, - эндогенная выборка
endogenous stratified sampling, - выборка с эндогенной стратификацией
exogenous stratified sampling, - выборка с экзогенной стратификацией
fixed in repeated samples,  - фиксированные регрессоры в повторяющихся выборках
flow sampling, - выборка типа поток
multi-stage surveys, - многоэтапные опросы
on-site sampling, - отбор на месте
simple random sampling, - простая случайная выборка
stock sampling, - выборка типа запас
with replacement, - с повторениями
without replacement, - без повторений
sandwich variance matrix - ковариационная матрица в сэндвич-форме
clustered data, - кластеризованные данные
extremum estimator, - экстремальная оценка
GMM estimator, - оценка обобщенного метода моментов
ML estimator, - оценка метода максимального правдоподобия
NLS estimator, - оценка нелинейного МНК
OLS estimator, - МНК оценка
panel data, - панельные данные
for Wald test, - для теста Вальда
see also robust standard errors - см. также робастные стандартные ошибки
Sargan test, - Саргана тест
see also overidentifying restrictions test - см. также тест на сверх-идентифицирующие ограничения
scale parameter, - параметр масштаба
scanner data, - данные о сканированных штрих-кодах
Schwarz criterion. See BIC - Шварца критерий. См. BIC
SCLS estimator. See symmetrically censored least squares - Симметрично цензурированный МНК
score test, see Lagrange multiplier test - скор-тест, см. тест множителей Лагранжа
score vector, - скор-вектор
secondary sampling units (SSUs), - вторичная единица выборки (ВЕВ)
seed, - зерно генератора случайных чисел
seemingly unrelated regressions (SUR) model, - система одновременных уравнений
Bayesian MCMC example, - Байесовский MCMC пример
count data, - счетные данные
error components, - состовляющие ошибки
nonlinear, - нелинейная
selection bias, - смещение самоотбора
nonignorable missingness, - неигнорируемые пропуски в данных
treatment effects models, - модели эффективности воздействия
see also selection models - см. модели самоотбора
selection models, - модели самоотбора
bivariate sample selection model, - двумерная модель самоотбора
count models, - счетные модели
example, - пример
panel data, - панельные данные
Roy model, - Роя модель
sample selection, - самоотбор выборки
self selection, - самоотбор
semiparametric estimation, - полупараметрическое оценивание
structural models, - структурные модели
treatment effects model, - модель эффективности воздействия
versus selection on observables only, - сравнение с самоотбором по наблюдаемым показателям
versus two-part models, - сравнеие с двухчастной моделью
see also Tobit models - см. также тобит модели
selection on observables only, - самоотбор по наблюдаемым показателям
compared to selection models, - сравнение с моделями самоотбора
conditional independence assumption, - предпосылка об условной независимости
control function estimator, - оценка контрольных функций
definition, - определение
DID estimator, - оценка разность разностей
RD design estimator, - оценка при разрывном дизайне
treatment effects model, - модель эффекта воздействия
selection on unobservables, - самоотбор по ненаблюдаемым показателям
definition, - определение
in treatment effects model, - в модели эффекта воздействия
IV estimators, - оценки инструментальных переменных
Roy model, - Роя модель
selection bias, - смещение самоотбора
selection model, - модель самоотбора
self-weighting sample, - выборка с самовзвешиванием
SEM. See simultaneous equations model 
seminonparametric ML estimator, - оценка полунепараметрического метода максимального правдоподобия
semiparametric efficiency bounds, - полупараметрические границы эффективности
semiparametric estimators, - полупараметрические оценки
adaptive, - адаптивные
application, - приложения
average derivative estimator, - оценка средней производной
efficiency bounds, - границы эффективности
nonparametric FGLS, - непараметрический доступный обобщенный МНК
Robinson difference estimator, - оценка разностей Робинсона
semiparametric least squares, - полупараметрический МНК
seminonparametric ML estimator, - оценка полунепараметрического метода максимального правдоподобия
see also semiparametric models - см. также полупараметрические модели
semiparametric heterogeneity model, - полупараметрическая модель для неоднородных данных
see also finite mixture models - см. также модели смеси распределений
semiparametric least squares, - полупараметрический МНК
semiparametric ML estimator, - полупараметрическая оценка максимального правдоподобия
semiparametric models, - полупараметрические модели
additive models, - аддитивные модели
binary outcome models, - модель бинарного выбора
censored models, - цензурированные модели
count models, - счетные модели
definition, - определение
duration models, - модели времени жизни
flexible parametric models, - гибкие параметрические модели
heteroskedastic linear model, - гетероскедастичная линейная модель
identification, - идентифицируемость
leading examples, - основные примеры
multinomial outcome models, - мультиномиальные модели
panel data models, - модели панельных данных
partially linear model, - частично линейная модель
selection models, - модели самоотбора
single-index models, - одноиндексные модели
see also semiparametric estimators - см. также полупараметрические оценки
sequential limits, - последовательное взятие пределов
sequential multinomial models, - последовательные мультиномиальные модели
sequential two-step m-estimator, - двухшаговая М-оценка
bootstrap for, - бутстрэп
sequence of random variables, - последовательность случайных величин
serial correlation. See autocorrelation - См. автокорреляция
set identification, - идентификация множества
series estimator, - оценка с использованием рядов
for binary outcomes, - для моделей бинарного выбора
shared frailty model, - модель с уязвимостью
short panel - короткая панель
definition, - определение
statistical inference in, - статистические выводы
shrinkage estimator, - сжимающая оценка
Silverman’s plug-in estimate, - оценка Сильвермана
simple random sampling (SRS), - простая случайная выборка
simple stratified sampling, - простая стратифицированная выборка
Simpson’s rule, - правило Симпсона (правило парабол)
simulated annealing (SA) method, - метод имитации отжига
simulated m-estimator, - симуляционная М-оценка

simulation-based estimation methods, - методы основанные на симуляциях
motivating examples, - мотивационные примеры
see MSL, MSM, indirect inference, simulators  - см. метод симуляционного правдоподобия, метод симуляционных моментов, непрямые выводы, вспомогательные оценки
antithetic sampling, - антитетическое сэмплирование
direct, - прямые
frequency, - частнотная вспомогательная оценка
GHK, - GHK-симулятор, Гевеке-Хадживасилу-МакФаддена симулятор
Halton sequences, - последовательности Гальтона
importance sampling, - сэмплирование по важности
smooth, - гладкие
subsimulator, - вспомогательная оценка, симулятор
unbiased, - несмещенный
see also quadrature - см. также численное интегрирование

simultaneous equations model (SEM), - системы одновременных уравнений
causal interpretation, - причинная интерпретация
error components, - компоненты ошибки
extension to nonlinear models, - обобщение на нелинейные модели
FIML estimator, - оценка метода максимального правдоподобия с полной информацией

identification, - идентифицируемость
LIML estimator, - оценка метода максимального правдоподобия с ограниченной информацией
nonlinear, - нелинейная
order condition, - условие порядка
rank condition, - условие ранга
reduced form, - приведенная форма
single-equation models,  - модели с одним уравнением
structural form, - структурная форма
structural model, - структурная модель
2SLS estimator, - двухшаговая МНК оценка
3SLS estimator, - трехшаговая МНК оценка
simultaneous equations probit, - система одновременных пробит-моделей
simultaneous equations Tobit, - системе одновременных тобит-моделей
single-index models, - одноиндексные модели
definition, - определение
identification, - идентифицируемость
marginal effects, - предельные эффекты
nonlinear panel model, - нелинейные панельные модели
semiparametric estimators, - полупараметрические оценки
SIPP. See Survey of Income and Program Participation - SIPP. Обследование доходов и участия в программах поддержки
size of test, - размер теста
nominal size, - номинальный размер теста
size-corrected test, - тест с корректировкой на размер 
true size, - истинный размер теста
Sklar’s theorem, - Склара теорема
Slutsky’s Theorem, - Слуцкого теорема
alternative version, - альтернативная версия
small-sample bias. See finite-sample bias - смещение малых выборок. См. смещение конечных выборок
smooth maximum score estimator, - гладкая оценка максимального счета
smoothing parameters, - параметры сглаживания
smoothing spline estimator, - оценка сглаженных сплайнов
social experiments, - социальные эксперименты
advantages, - преимущества
examples, - примеры
limitations, - ограничения
randomization, - рандомизация
span, - ширина окна
specific to general test, - тесты от частного к общему
specification tests, - тесты на спецификацию
for clustered data, - для кластеризованных данных
for duration models, - для моделей времени жизни
for endogeneity, - на эндогенность
for exogeneity, - на экзогенность
for heteroskedasticity, - на гетероскедастичность
for individual-specific effects, - на индивидуальные эффекты
for omitted variables, - на пропущенные переменные
for overdispersion, - на избыточную дисперсию
for pooling, - на объединение
for unobserved heterogeneity, - на ненаблюдаемую гетероскедастичность
for Tobit model, - для тобит модели
see also m-tests; model diagnostics - см. также М-тесты, диагностика модели
spherical errors, - сферические ошибки
split-sample IV estimator, - оценки инструментальных переменных с делением выборки
SRS. See simple random sampling - простая случайная выборка
SSUs. See secondary sampling units - ВЕВ. См. вторичная единица выборки
stable family of distributions, - устойчивое семейство распределений
stable unit treatment value assumption (SUTVA), - предположение о стабильности величины воздействия
standard errors. See robust standard errors - стандартные ошибки. См. робастные стандартные ошибки
starting values, - стартовые значения
state dependence. See true state dependence - зависимость от состояния. См. истинная зависимость от состояния
stated preference data, - данные о заявленных предпочтениях
stationary population, - стационарная генеральная совокупность
statistical packages, - статистические пакеты
step size adjustment, - корректировка величины шага
stochastic order of magnitude, - стохастический порядок малости
stock sampling, - выборка типа запас
strata, - страта
see also sampling schemes; - см. также план выборки
weighting stratification matching, - взвешенное стратифицированное сопоставление
stratified random sampling, - стратифицированная случайная выборка
use of Liapounov CLT, - использование ЦПТ Ляпунова
use of Markov LLN, - использование ЗБЧ Маркова
see also sampling schemes; weighting - см. также план выборки, взвешивание
strict exogeneity. See strong exogeneity  - строгая экзогенность. См. сильная экзогенность
strong consistency, - сильная состоятельность
strong exogeneity, - сильная экзогенность
in panel models, - в панельных моделях
structural approach - структурный подход
to measurement error, - к ошибке измерения
to weighting, - к взвешиванию
structural economic models, - структурные экономические модели
with selection, - с самоотбором
structural form, - структурная форма
structural model, - структурная модель
based on economic model, - основанная на экономической модели
exogeneity, - экзогенность
full information, - полная информация
limited information, - ограниченная информация
reduced form, - приведенная форма
structural form, - структурная форма
structure, - структура
see also simultaneous equations model - см. также системы одновременных уравнений
structural selection models, - структурные модели самоотбора
based on utility maximization, - основанные на максимизации полезности
endogenous regressors, - эндогенные регрессоры
simultaneous equations Tobit, - система одновременных уравнений в тобит-модели
studentized statistic, - стьюдентизированная статистика
subsampling method, - метод подвыборок
substitution bias, - смещение замещения
sufficient statistic, - достаточная статистика
definition, - определение
summation assumption, - предположение о сумме
superpopulation, - суперсовокупность
supersmoother, - супер-сглаживатель
SUR model. See seemingly unrelated regressions - модели внешне не связанных уравнений
survey methods, - методы опросов
survey nonresponse, - отсутствие ответа на опрос
see also attrition bias; imputation methods - см. также смещение истощения, методы восстановления данных
Survey of Income and Program Participation (SIPP), - Обследование доходов и участия в программах поддержки
survival analysis. See duration models. - анализ выживания. См. модели времени жизни
survival function - функция выживания
See survivor function - см. функция выживания
survivor function - функция выживания
aggregate survivor function, - агрегированная функция выживания
definition, - определение
estimator in PH model, - оценка в PH модели (модели пропорциональных рисков)
Kaplan-Meier estimator, - Каплан-Мейера оценка
in mixture models, - в моделях смеси
multivariate, - многомерная
parametric examples, - параметрические примеры
SUTVA. See stable unit treatment value assumption - предположение о стабильности величины воздействия
switching regressions model. - модель переключающихся регрессий
See Roy model - См. Роя модель
symmetrically censored least squares (SCLS) - симметрично цензурированный МНК
estimator, - оценка
synthetic panels. See pseudo panels - синтетические панели. См. псевдо-панели
systems of equations, - системы уравнений
linear systems, - линейные системы
nonlinear systems, - нелинейные системы
seemingly unrelated regression, - внешне несвязанные уравнения
simultaneous equations model, - системы одновременных уравнений
systems-robust standard errors, - робастные стандартные ошибки для систем одновременных уравнений
target density, - целевое распределение
tests. See hypothesis tests, m-tests, specification tests - тесты. См. тесты гипотез, М-тесты, тесты на спецификацию
three-stage least squares (3SLS) estimator, - трехшаговая МНК оценка
3SLS estimator. See three-stage least squares - См. трехшаговая МНК оценка
time series data - временные ряды
bootstrap, - бутстрэп
NLS estimator, - оценка нелинейного МНК
Newey-West standard errors, - стандартные ошибки в форме Ньюи-Веста
time-varying regressors - регрессоры, меняющиеся во времени
in duration models, - в моделях времени жизни
in panel data models, - в моделях панельных данных
Tobit model, - тобит модель
Bayesian methods, - Байесовские методы
censored mean, - цензурированное среднее
censoring mechanism, - механизм цензурирования
consistency of MLE, - состоятельность оценок максимального правдоподобия
definition, - определение
example, - пример
generalized, - обобщенная
Heckman two-step estimator, - двухшаговая оценка Хекмана
identification, - идентифицируемость
as imputation method, - как метод восстановления
inverse-Mills ratio, - обратное отношение Миллса
marginal effects, - предельные эффекты
measurement error in dependent variable, - ошибки измерения зависимой переменной
ML estimator, - оценка максимального правдоподобия
NLS estimator, - оценка нелинейного МНК
OLS estimator, - МНК оценка
panel data, - панельные данные
simultaneous equations, - системы одновременных уравнений
specification tests, - тесты на спецификацию модели
with stochastic thresholds, - со стохастическими порогами
with truncated data, - с усеченными данными
truncated mean, - усеченное средние
two-limit, - с двумя границами
type 2, - 2-го типа
type 5, - 5-го типа
see also selection models - см. также модели самоотбора
top-coded data, - данный цензурированные сверху
transformation methods, - методы преобразования
transformation theorem, - теорема о преобразовании
transformed ML estimator, - преобразованная оценка метода максимального правдоподобия

transition data. See duration models - См. модели времени жизни
trapezoidal rule, - правило трапеций
treatment-control comparison - сравнение контрольной и экспериментальной группы
application, - приложения
treatment effects framework, - подход связанный с оценкой эффекта воздействия
balancing condition,  - балансирующее условие
binary treatment variable, - бинарная переменная воздействия
conditional independence assumption, - предпосылка об условной независимости
conditional mean independence assumption, - предпосылка о независимости условного среднего
heterogeneous treatment effects, - неоднородные эффекты воздействия
multiple treatments, - множественные воздействи
overlap assumption, - предположение о пересечении (предположение о сопоставлении)
propensity score, - мера склонности
Roy model, - Роя модель
stable unit treatment value assumption, - предположение о стабильности величины воздействия
see also treatment evaluation - см. также оценка воздействия
treatment evaluation, - оценка воздействия
application, - приложения
IV estimators, - оценка инструментальных переменных
matching estimators, - оценки сопоставления
DID estimators, - оценка разность разностей
selection bias, - смещение самоотбора
selection on observables, - самоотбор по наблюдаемым показателям
selection on unobservables, - самоотбор по ненаблюдаемым показателям
regression discontinuity design, - разрывный дизайн
see also treatment effects framework 
treatment group, - экспериментальная группа, группа, подвергнутая воздействию
trimming, - усечение
trivariate reduction, - трехмерная редукция
true state dependence - истинная зависимость от состояния
duration models, - модели времени жизни
dynamic panel models, - динамические панельные модели
see also unobserved heterogeneity - см. также ненаблюдаемая неоднородность
truncated models, - модели усеченных переменных
conditional mean, - условное среднее
count models, - счетные модели
definition, - определение
examples, - примеры
ML estimator, - ML оценка, оценка максимального правдоподобия
see also Tobit model; selection models - см. также тобит-модель; модели самоотбора
truncated moments of standard normal, - усеченные моменты стандартного нормального распределения
truncation mechanisms, - механизмы усечения
truncation from above, - усечение сверху
truncation from below, - усечение снизу
2SLS estimator. See two-stage least squares - См. двухшаговая МНК оценка
two-limit Tobit model, - тобит модель с двумя границами
two-part model, - двухчастная модель
application, - приложения
compared to selection models, - сравнение с моделями самоотбора
definition, - определение
example, - пример
see also hurdle model - см. также модель преодоления порогов
two-stage IV estimator, - двухшаговая оценка инструментальных переменных
two-stage least squares (2SLS) estimator, - двухшаговая оценка метода наименьших квадратов
alternatives to, - альтернативы
Basmann’s approach, - подход Басманна (интерпретация Басманна)
compared to optimal GMM, - сравнение с оптимальным обобщенным методом моментов
as GLS in transformed model, - как обобщенный МНК в преобразованной модели
as GMM estimator, - как оценка обобщенного метода моментов
nonlinear, - нелинейная
panel data, - панельные данные
in SEM, - в системах одновременных уравнений
Theil’s interpretation, - интерпретация Тейла
two-stage sampling, - двухэтапная выборка
two-step estimators - двухшаговые оценки
GMM, - обобщенный метод моментов, ОММ
Heckman, - Хекман
sequential m-estimator, - последовательная М-оценка
two-step GMM estimator, - двухшаговая оценка обобщенного метода моментов
panel, - панельные данные
two-way effects model, - модель с двусторонними эффектами
type I error, - ошибка I типа
type II error, - ошибка II типа
type 1 extreme value distribution, - распределение экстремальных значений 1-го типа
duration model error, - ошибка в модели времени жизни
multinomial logit model, - мультиномиальная логит-модель
type 2 Tobit. See bivariate sample selection model - тобит модель 2-го типа. См. двумерная модель самоотбора
type 5 Tobit. See Roy model - тобит модель 5-го типа. См. модель Роя
ultimate sampling units (USUs), - конечная единицы выборки (КЕВ)
unbalanced panels, - несбалансированные панели
uncentered explained sum of squares (ESS), - нецентрированная объясненная сумма квадратов
uncentered R-squared, - нецентрированный R-квадрат
unconfoundedness assumption. See conditional independence assumption - предположение о несмешиваемости. См. предположение об условной независимости
underrecording, - недозапись наступивших событий
undersmoothing, - недосглаживание
uniform convergence in probability, - равномерная сходимость по вероятности
uniform number generators, - генератор равномерно распределенных случайных чисел
uniformly most powerful (UMP) test, - равномерно наиболее мощный тест
unit roots, - единичные корни
universal logit model, - универсальная логит-модель

unobserved heterogeneity - ненаблюдаемая неоднородность
application, - приложения
in competing risks model, - в модели конкурирующих рисков
in count models, - в счетных моделях
distributions for, - законы распределения для описания
in duration models, - в моделях времени жизни
finite mixture models for, - моделирование с помощью моделей смеси распределений
identification, - идентифицируемость
IM test for, - тест информационных матриц
individual-specific effects, - индивидуальные эффекты
mixture models for, - модели смеси
MSL example, - пример с использованием симуляционного максимального правдоподобия
MSM example, - пример с использованием симуляционного метода моментов
multiplicative, - мультипликативная
in nonlinear systems, - в нелинейных системах
specification tests for, - тесты на спецификацию
variance inflation, - вздутие дисперсии
versus true state dependence, - сравнение с истинной зависимостью от состояния

USUs. See ultimate sampling units - КЕВ. См. конечная единица выборки
validation sample, - тестовая выборка
variance components, - компоненты дисперсии
variance matrix estimation - оценка ковариационной матрицы
BHHH estimate, - BHHH оценка
degrees-of-freedom adjustment, - поправка на количество степеней свободы
expected Hessian estimate, - оценка математического ожидания матрицы Гессе
for extremum estimator, - для экстремальной оценки
for GMM estimator, - для оценки обобщенного метода моментов
Hessian estimate, - оценка матрицы Гессе
for NLS estimator, - для оценки нелинейного МНК
OPG estimate, - оценка с помощью внешнего произведения градиента
robust estimate, - робастная оценка
sandwich estimate, - оценка в сэндвич-форме
for weighted estimators, - для взвешенных оценок
see also robust standard errors - см. также робастные стандартные ошибки
variance reduction for simulation, - снижение дисперсии для симуляций
Wald estimator - Вальда оценка
in treatment effects models, - в моделях эффективности воздействия
Wald test, - Вальда тест
asymptotic distribution, - асимптотическое распределение
comparison with LM and, LR tests, - сравнение с LM тестом (тестом множителей Лагранжа) и LR тестом (тестом отношения правдоподобия)
definition, - определение
examples, - примеры
exclusion restrictions, - исключающие ограничения, ограничения исключения
F-test version, - версия с помощью F-теста
introduction, - введение
lack of invariance, - отсутствие инвариантности
likelihood based, - основанный на методе максимального правдоподобия
linear models, - линейные модели
linear restrictions, - линейные ограничения
in misspecified models, - в неправильно специфицированных моделях
nonlinear restrictions, - нелинейные ограничения
power, - мощность
of statistical significance, - на значимость
t-test version, - версия с помощью t-теста
see also hypothesis tests - см. также тестирование гипотез
weak consistency, - слабая состоятельность
weak exogeneity, - слабая экзогенность
in panel data, - в панельных данных
weak instruments, - слабые инструменты
application, - приложения
definition, - определение
finite sample bias, - смещение малых выборок
GMM estimator, - ОММ оценка (оценка обобщенного метода моментов)
inconsistency, - несостоятельность
indicators - признаки
panel data, - панельные данные
Weibull distribution, - Вейбулла распределение
Weibull-gamma regression model, - Вейбулла-гамма регрессионная модель
Weibull regression model, - Вейбулла регрессионная модель
weighted estimation - взвешенное оценивание
endogenous stratification, - эндогенная стратификация
exogenous stratification, - экзогенная стратификация
weighted exogenous sampling ML (WESML) estimator, - взвешенная оценка максимального правдоподобия для экзогенного отбора
weighted least squares (WLS) estimator, - оценка взвешенного МНК
asymptotic distribution, - асимптотическое распределение
contrasted with GLS, - сравнение с обобщенным МНК
definition, - определение
example, - пример
in pooled model, - в сквозной модели
see also FGLS estimator - см. также доступная оценка обобщенного МНК
weighted maximum likelihood (WML) estimator, - взвешенная оценка метода максимального правдоподобия
weighted semiparametric least squares (WSWL) estimator, - взвешенная оценка полупараметрического МНК
for binary outcome models, - для моделей бинарного выбора
weighting, - взвешивание
descriptive versus structural approach, - сравнение описательного и структурного подходов
with endogenous stratification, - с эндогенной стратификацией
sample weights, - выборочные веса
variance estimation, - оценка дисперсия
weighted prediction, - взвешенное прогнозирование
weighted regression, - взвешенная регрессия
whether to weight, - необходимость взвешивания
welfare analysis - анализ благосостояния
with ARUM, - с помощью модели ARUM
with nested logit model, - с помощью вложенных логит-моделей
WESML estimator. See weighted exogenous sampling ML - взвешенная оценка максимального правдоподобия для экзогенного отбора
White standard errors. See robust standard errors - Уайта стандартные ошибки. См. робастные стандартные ошибки
wild bootstrap, - дикий бутстрэп
window width, - ширина окна сглаживания
Wishart distribution, - Уишарта распределение
see also inverse-Wishart distribution - см. также обратное распределение Уишарта
within estimator. See fixed effects estimator - оценка within. См. оценка с фиксированными эффектами
within model. See fixed effects model - модель within. См. модель с фиксированными эффектами
within-group variation, - внутригрупповая диспрерсия
with-zeros model, - модель с нулями
WLS estimator. See weighted least squares - Оценка взвешенного МНК. См. взвешенный метод наименьших квадратов
WML estimator. See weighted maximum likelihood - Оценка взвешенного максимального правдоподобия
WNLS estimator, - оценка взвешенного нелинейного МНК
asymptotic distribution, - асимптотическое распределение
definition, - определение
example, - пример
as GLM, - в обобщенной линейной модели
working matrix - рабочая матрица
definition, - определение
for GLM estimator, - для оценки обобщенной линейной модели
for pooled GEE estimator, - для сквозной обобщенной оценки оценивающих уравнений
for pooled WLS estimator, - для сквозной оценки взвешенного МНК
for WLS estimator, - для оценки взвешенного МНК
WSLS estimator. See weighted semiparametric least squares - Оценка взвешенного полупараметрического МНК

zero-inflated count model, - модель счетных данных с раздутым нулём

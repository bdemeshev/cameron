accelerated failure time (AFT) model, 591–2 coefficient interpretation, 606–7 definition, 592
leading examples, 585
accept-reject methods, 413–4, 445 ACD. See average completed duration acronyms, 17
AD estimator. See average derivative adaptive estimator, 323, 328, 684 adding-up constraints, 210
additive model, - аддитивная модель
additive random utility model (ARUM)
binary outcome models, 476–8 generalized random utility models, 515–6 identification, 504
multinomial outcome models, 504–7 nested logit model, 509, 526–7
RPL model, 513
welfare analysis in, 506–7
admissible estimator, - допустимая оценка
AFT. See accelerated failure time aggregated data
binary outcomes, 480–2
cohort-level, 772
nonlinear models, 482, 487 multinomial outcomes, 513 time-aggregated durations, 578, 600–3 see also discrete-time duration data
AIC. See Akaike information criterion - AIC. См. Акаике информационный критерий
AID. See average interrupted duration
Akaike information criterion (AIC), - Акаике информационный критерий
almost sure convergence, - сходимость почти наверное 
analog estimator, 135
analogy principle, - принцип аналогии
and method of moments estimators, 167 analysis of covariance, 733
analysis of variance, 733
Anscombe residual, - Анскомба остатки
antithetic sampling, 408–9, 445 applications with data
competing risks models, 658–62
duration models, 603–8, 632–6
IV estimation, 110–2
kernel regression, 295–7, 300
logit and probit models, 464–6, 486
multinomial and nested logit models, 491–5, 511 Poisson and negative binomial models, 671–4, 690 panel fixed and random effects estimation, 708–15 panel GMM linear estimation, 754–6
panel nonlinear estimation, 792–5 quantile regression, 88–90
selection and two-part models, 553–6, 565 survival function, 574–5, 582
treatment evaluation estimation, 889–96
see also data sets used in applications Archimedean family, 654 Arellano-Bond estimator, 765–6, 777
application, 754–6 nonlinear models, 791 unit roots, 768
ARMA. See autoregressive moving average artificial nesting, 283
ARUM. See additive random utility model asymptotic distribution, 953–4
1006
asymptotic efficiency, - асимптотическая эффективность
asymptotic normal distribution, - асимптотически нормальное распределение
definition, - определение
estimated asymptotic variance, - оценка асимптотической дисперсии
of extremum estimators, - экстремальных оценок
of FGLS estimator, - 
of FGNLS estimator, 156–7
of first-differences estimator, - оценки в первых разностях
of fixed effects estimator, - оценки в модели с фиксированными эффектами
of GMM estimator, - GMM оценки, оценки обобщенного метода моментов
of Hausman test statistic, - статистики Хаусмана
of kernel density estimator, - ядерной оценки функции плотности
of kernel regression estimator, - ядерной оценкии линии регрессии 
of LM test statistic, - LM статистики, статистики множителей Лагранжа
of LR test statistic, - LR статистики, статистики отношения правдоподобия
of m-estimators, 119–21
of MD estimator, 292
of ML estimator, - оценки метода максимального правдоподобия
of MM estimator, - оценки метода моментов
of MSL estimator, 394–5
of MSM estimator, 400–2 of m-test statistics, 260, 263 of NLS estimator, 152–4
of NL2SLS estimator, 195–6
of OIR test statistic, 181, 183
of OLS estimator, - МНК оценки 
of panel GMM estimator, 745–6 
of quasi-ML estimator, 146
of random effects estimator, 735 
of Wald test statistic, 226–8
see also asymptotic theory
asymptotic efficiency, - асимптотическая эффективность
of optimal GMM, 177
asymptotic refinement, 359, 371–2
by bootstrap, 256, 363–7, 371–2, 378–9 definition, 359
by Edgeworth expansion, 371–2
by nested bootstrap, 374, 379
asymptotic theory definitions, - асимптотические определения
asymptotic distribution, - асимптотическое распределение 
asymptotic variance, - асимптотическая дисперсия
central limit theorems, - центральные предельные теоремы 
consistency, - состоятельность
convergence in distribution, - сходимость по распределениею 
convergence in probability - сходимость по вероятности
laws of large numbers, - законы больших чисел 
limit distribution - предельное распределение
limit variance - предельная дисперсия
stochastic order of magnitude, - стохастический порядок малости 
summary of definitions and theorems, - список определений и теорем
asymptotic variance, - асимптотическая дисперсия 
estimated asymptotic variance,  - оценка асимптотической дисперсии
see also asymptotic distribution - см. также асимптотическое распределение
asymptotically pivotal statistic, 359–60, 363–4, 366, 372, 374, 379–80
ATE. See average treatment effect
ATET. See average treatment effect on the treated attenuation bias, 903–5, 911, 915, 919–20 attrition bias, - смещение истощения выборки
augmented regression model, 429
autocorrelation - автокорреляция
in panel model errors, - в ошибках панельных моделей 
dynamic panel models, - динамические панельные модели
see also panel-robust inference 
autoregressive moving average (ARMA) errors - ARMA ошибки
definition, - определение
NLS estimator, - оценка методом нелинейных наименьших квадратов
panel data, - в панельных данных
auxiliary model, - вспомогательная модель 
auxiliary regression - вспомогательная регрессия 
bootstrapping, - бутстрэп 
example, - пример 
Hausman test, - тест Хаусмана
LM test, - LM тест 
m-test, 261–4, 544
available case analysis. See pairwise deletion average completed duration (ACD), 626 average derivative (AD) estimator
definition, 326
uses, 317, 483
average interrupted duration (AID), 626 average selection bias, 868
average squared error, 315
average treatment effect (ATE), 33–4, 866–71
definition, 866
difficulties estimating, 866
local ATE, 883–6
matching estimators, 871–8 potential outcome model, 33–4 selection on observables only, 868–9 selection on unobservables, 868–71 see also ATET; LATE; MTE
average treatment effect on the treated (ATET), 866–78
application, 889–6
definition, - определение
difficulties estimating, 866
matching estimators, 871–8, 894–6 selection on observables only, 868–9 selection on unobservables, 868–71 see also ATE; LATE; MTE
averaged data. See aggregated data
backward recurrence time, 626
balanced bootstrap, 374
balanced repeated replication, 855
balancing condition, 864, 893–4
bandwidth, 299, 307, 312
bandwidth choice for kernel density estimator, - выбор ширины окна для ядерных оценок функции плотности
cross validation, - кросс-валидация
example, - пример
optimal, - оптимальный
Silverman’s plug-in estimate, - оценка Сильвермана
bandwidth choice for kernel regression estimator, - выбор ширины окна для ядерных оценок линии регрессии
cross validation, - кросс-валидация 
example, - пример 
optimal, - оптимальный 
plug-in estimate, 314
baseline hazard, 591
in AFT model, 592
identification in mixture models, 618–20 in multiple spells models, 655–6
in PH model, 591, 596–7, 601–2
Bayes factors, 456–8
Bayes rule. See Bayes theorem - Байеса правило, см. Байеса теорема
Bayes theorem, - Байеса теорема 
example, - пример
Bayesian central limit theorem, - Байесовская центральная предельная теорема
Bayesian information criterion (BIC), - Байесовский информационный критерий
see also AIC - см. также AIC
Bayesian methods, - Байесовские методы
Bayes 1764 example, - пример Байеса 1764 года
Bayesian approach, - Байесовский подход
binary outcome models, 
compared to non-Bayesian, 164, 424–5, 432–41,
439–41
count models, - счетные модели
data augmentation, 454–5, 932–3, 935–9 decision analysis, 434–5
examples, 452–4
hierarchical linear model, - иерархическая линейная модель
importance sampling, 443–5
linear regression, 435–43, 449–50, 452–4 
Markov chain Monte Carlo simulation, - метод Монте-Карло по схеме марковской цепи
measurement error model, 915
mixed linear model, 775
model selection, 456–8
multinomial outcome models, 514, 519 panel data, 775, 809
posterior distribution, - апостериорное распределение
prior distribution, - априорное распределение
Tobit model, - тобит модель
BCA method. See bias-corrected and accelerated before-after comparison
application, 890–1
Berkson error model, 920
Berkson’s minimum chi-square estimator, 480–1 Berndt, Hall, Hall, and Hausman (BHHH) estimate,
138, 241, 395
Berndt, Hall, Hall, and Hausman (BHHH) iterative method, - итерационный алгоритм Берндта, Берндта, Холла и Хаусмана, BHHH итерационный алгоритм
Bernoulli distribution, - распределение Бернулли 
Bernstein-von Mises Theorem, - Бернштейна-фон Мизеса теорема
best linear unbiased predictor, - наилучшая линейная несмещенная оценка
between estimator, 702, 736, 841
application, 710–3
between-group variation, 709, 733
between model, 702
BFGS algorithm. See Boyden, Fletcher, Goldfarb, and Shannon - BFGS алгоритм, см. Бойден, Флетчер, Голдфарб и Шеннон 
BHHH estimate. See Berndt, Hall, Hall, and Hausman BHHH method. See Berndt, Hall, Hall, and Hausman bias-corrected and accelerated (BCA) bootstrap
method, 360
biased sampling, - выборка со смещение
see also sample selection; endogenous stratification BIC. See Bayesian information criterion
binary endogenous variable, 562
binary outcome models, 463–89
additive random utility model, 476–8 aggregated data, 480–2 alternative-invariant regressors, 478
alternative-varying regressors, 478 choice-based samples, 478–9 corrected score estimator, 916–8 definition, 466
example, 464–5
identification, 476, 483
index function model, 475–6
marginal effects, 467, 470–1
measurement error in dependent variable, 914 measurement error in regressors, 919
ML estimator, - ML оценка, оценка максимального правдоподобия
model misspecification, - неправильная спецификация модели, мисспецификация
multiple imputation example, 937–8
OLS estimator, - МНК-оценка
panel data, - панельные данные
semiparametric estimation, - полупараметрическое оценивание
see also logit models; probit models
binding function, 404–5
bivariate counts, 215, 685–7
bivariate negative binomial distribution, - двумерное отрицательное биномиальное распределение 
bivariate ordered probit model, - двумерная упорядоченная пробит модель
bivariate Poisson distribution, - двумерное распределение Пуассона
bivariate Poisson-lognormal mixture, 686 
bivariate probit model, - двумерная пробит модель
bivariate sample selection model, 547–53
application, 553–5
bounds, 566
conditional mean, 548–50 conditional variance, 549–50 definition, 547
Heckman two-step estimator, - двушаговая оценка Хекмана 
identification, - идентифицируемость
marginal effects, 552
ML estimator, 548
outcome equation, 547 participation equation, 547 semiparametric estimator, 565–6 versus two-part model, 546, 552–3
Bonferroni test, 230 bootstrap hypothesis tests
asymptotic refinement, 363–4, 366–7, 371–2, 378–9
bootstrap critical value, 256, 363 bootstrap p-value, 256, 363 example, 366–8
nonsymmetrical test, 363, 380 power, 372–3
symmetrical test, 363
without asymptotic refinement, 363, 367–8,
378
bootstrap methods, 357–83
asymptotic refinement, 359, 366–7 bias estimate, 365
bias-corrected estimator, 365, 368 clustered data, 363, 377–8, 845 confidence intervals, 364–5, 368 consistency, 369–70
critical value, 363
examples, 254–6, 366–8
for functions of parameters, 363
general algorithm, 360
for GMM, 379–80
heteroskedastic data, 363, 376–7 introduction, 254–6
for nonsmooth estimators, 373, 380–1 number of bootstrap samples, 361–2
panel data, 363, 377–8, 708, 746, 751 p-value, 363
recentering, 374, 379
rescaling, 374
sampling methods for, 360
smoothness requirements, 370
standard error estimate, 362, 366
time series data, 381
variance estimate, 362
without asymptotic refinement, 358, 367–8 see also bootstrap hypothesis tests
bounds identification, 29
in measurement error models, 906–8
bounds in selection model, 566
Boyden, Fletcher, Goldfarb, and Shannon (BFGS)
algorithm, 344
CAIC. See consistent Akaike information criterion calibrated bootstrap, 374
caliper matching, 874, 895
canonical link function, 149, 469, 783 case-control analysis, 479, 823
causality, 18–38
examples, 69–70, 98
Granger causality, 22
identification frameworks and strategies,
35–3
in linear regression model, 68–9
in potential outcome models, 32–4, 862–5 in simultaneous equations model, 26–7
in single-equation model, 31
and weighting, 820–1
see also endogeneity
cdf. See cumulative distribution function
censored least absolute deviations (CLAD) estimator,
564–5, 808
censored models, 530–44, 579–80
conditional mean, 535
count models, 680
definitions, 532, 579–80
examples, 530–1, 535
ML estimator, 533–4
semiparametric estimation, - полупараметрическое оценивание
see also duration model; selection models; Tobit
models; truncated models
censored normal regression model. See Tobit model censoring mechanisms, 532, 579–80
censoring from above, 532, 579 censoring from below, 532, 579 left censoring, 532, 579, 588
independent censoring, 580 interval censoring, 579, 588 noninformative censoring, 580 random censoring, 579
right censoring, 532, 579, 581, 589 sample selection, 44–5, 547
type 1 censoring, - цензурирование 1-го типа
type 2 censoring, - цензурирование 2-го типа 
census coefficient, 819
central limit theorem (CLT), - центральная предельная теорема
Cramer linear transformation, 952 Cramer-Wold device, 951 definition, 950
examples of use, - примеры использования 
Liapounov CLT, - ЦПТ Ляпунова 
Lindeberg-Levy - ЦПТ Линдеберга-Леви 
multivariate, - многомерная ЦПТ
sample average, - выборочное среднее
sampling scheme, 131, 950
CGF tests. See chi-square goodness-of-fit characteristic function, 370, 913, 950
chatter, 394, 410
Chebychev’s inequality, - неравенство Чебышева
chi-square goodness-of-fit (CGF) tests, 266–7, 270–1,
474
choice-based samples, 823
binary outcome models, - модель бинарного исхода
see also endogenous stratification Choleski decomposition, 416, 448
CL model. See conditional logit
CLAD estimator. See censored least absolute
deviations
Clayton copula, - Клейтона копула
CLT. See central limit theorem clustered data, 829–53
application, 848–53
cluster bootstrap, 363, 377–8, 845 cluster-robust inference, 707, 834, 842,
845
cluster sampling, 41–2
cluster-specific effects, 830–2, 837–45 comparison to panel data, 831–2 diagnostic tests, 841
dummy variables model, 840
fixed effects estimator, 840–1, 843–5 hierarchical models, 845–8
large clusters, 832
nonlinear models, 841–5
OLS estimator, - МНК-оценка
quasi-ML estimator, - оценка квази-максимального правдоподобия
random effects estimator, - оценка модели со случайными эффектами
small clusters, 832
see also panel data - см. также панельные данные
cluster-robust standard errors bootstrap, 363, 377–8, 845 clustered data, 834, 842
panel data, 706–7, 745–6, 789 see also robust standard errors
cluster-specific fixed effects (CSFE) estimator, 839–41, 843–4
application, 848–53 between estimator, 840–1 nonlinear models, 843–4 within estimator, 140–1
cluster-specific fixed effects (CSFE) model, 831, 843 cluster-specific random effects (CSRE) estimator,
837–9, 843–4 application, 848–53
cluster-specific random effects (CSRE) model, 831, 843–4
cluster variable, 707
CM tests. See conditional moment coefficient interpretation
in binary outcome models, 467, 473 
in competing risks model, 646
in count model, 669
in duration models, 606–7
in misspecified linear model, - в неправильно специфицированных линейных моделях
in multinomial outcome models, - в мультиномиальных моделях
in nonlinear models, - в нелинейных моделях
in Tobit model, - в тобит модели
see also marginal effects
coherency condition, 562
cohort-level data. See pseudo panels cointegration, 382, 767
common parameters, 801
compensating variation, 500–7, 512 competing risks model (CRM), 642–8, 658–62
application, 658–62 censoring, 642
coefficient interpretation, 646 definitions, 642–4
dependent risks, 647–8
exit route, 643 identification, 646 independent risks, 644–6 ML estimator, 644–5 proportional hazards, 645–6 spell duration, 643
with unobserved heterogeneity, 647, 659 complementary log-log model, 466–7, 603 complete case analysis. See listwise deletion complex surveys, 41–2, 814–6, 853–6 composition methods, 415
computational difficulties, 350–2
concentration parameter, 109
conditional analysis, 717
conditional expectations, 955–6
conditional independence assumption, 23, 863, 865
definition, - определение
for participation, 863
given propensity score, 865 selection on observables only, 868 unconfoundedness, 863
conditional likelihood, 139–40, 824
panel models, 731–2, 782–3, 796–9, 805
conditional logit (CL) model, 500–3, 524–5 application, 491–4
definition, 500
fixed effects binary logit, - логит модель с фиксированными эффектами 
marginal effects, - предельные эффекты
ML estimator, - оценка максимального правдоподобия
from ARUM, 505
see also multinomial outcome models
conditional ML estimator, 731–2, 782–3, 796–9, 805, 824
conditional moment (CM) tests, 264–5, 267–9, 319 consistent CM test, 268
in duration models, 632
example, 269–71
in Tobit model, 544
see also m-tests conditional mean
squared error loss, 67–9 conditional mode
step loss, 68 condition number, 350 conditional quantile
asymmetric absolute loss, 68
confidence intervals, 231–2, 316, 364–5, 368 consistent Akaike information criterion (CAIC), 278 consistent test statistic, 248
consistency definition, - определение состоятельности
of extremum estimators, - экстремальных оценок
of GMM estimator, - GMM оценки, оценки обобщенным методом моментов
of m-estimator, - М-оценки
of ML estimator, - оценки максимального правдоподобия
of NLS estimator, - оценки нелинейного МНК
of OLS estimator, - МНК-оценки
strong consistency, - сильная состоятельность
weak consistency, - слабая состоятельность
see also asymptotic distribution; identification;
pseudo-true value
constant coefficients model. See pooled model contagion, 612
contamination bias, 903–4
contemporaneous exogeneity assumption, 748–9, 752,
781
continuous mapping theorem, - теорема о непрерывном отображении 
control function approach, 37
control function estimator, 869–70, 890 control group, 49
conventions, 16–17
convergence criteria, 339–40, 458 convergence in distribution, 948–9
continuous mapping theorem, 949 definition, 948
limit distribution, 948 transformation theorem, 949 vector random variables, 949
see also central limit theorem convergence in probability, 944–7
1010
alternative modes of convergence, 945
consistency, 945
definition, 945
probability limit, 945
Slutsky’s theorem, - Слуцкого теорема 
uniform convergence, - равномерная сходимость 
vector random variables, - векторная случайная величина 
see also law of large numbers - см. также закон больших чисел
copulas, - копулы
count example, 687 
definition, - опреление
dependence parameter, 653–4 leading examples, 654
ML estimator, 655
survival copulas, 652
correlated random effects model, 719, 786 counterfactual, 32, 555, 861, 871
see also potential outcome model count data, 665
examples, 665 heteroskedasticity, 665 right-skewness, 665 see also count models
count models, 665–93
censored, 680
application, - приложение
endogenous regressors, - эндогенные регрессоры 
endogenous sampling, 823
finite mixture models, - модели смеси распределений
hurdle models, 680–1
measurement error in dependent variable, - ошибка измерения зависимой переменной 
measurement error in regressors, - ошибка измерения регрессоров 
mixture models, 675–7
multivariate, - многомерные
OLS estimator, - МНК-оценка
negative binomial model, - отрицательная биномиальная модель
NLS estimator, - оценка нелинейного МНК
panel data, - панельные данные
Poisson model, - модель Пуассона
sample selection, 680
semiparametric regression, - полупараметрическая регрессия
truncated, 679–80
zero-inflated, - с раздутым нулем
covariance matrix. See variance matrix covariance structures, - ковариационная матрица, см. структуры ковариационной матрицы 
covariates. See regressors
Cox CRM model. See competing risks
Cox PH model. See proportional hazards Cox-Snell residual, 289, 631, 633–6 CPS. See Current Population Survey Cramer linear transformation, 952 Cramer-Rao lower bound, 143, 954
see also semiparametric efficiency bound Cramer’s theorem, 949
Cramer-Wold device, 130, 951
CRM. See competing risks model cross-equation parameter restrictions, 210 cross-section data, 47
cross-validation, 304, 314–6, 318, 321
CSFE estimator. See cluster-specific fixed effects CSRE. See cluster-specific random effects cumulant, 370
cumulative distribution function (cdf ), 576 cumulative hazard function
definition, - определение
in competing risks model, 644–5
as diagnostic tool, 631–2
in likelihood function, - в функции правдоподобия
Nelson-Aalen estimator, - Нельсона-Аалена оценка 
in proportional hazards model, 590
Current Population Survey (CPS), 58, 814–5 curse of dimensionality
in Bayesian methods, 419–20
multivariate kernel density estimator, 306 multivariate kernel regression estimator, 319 
high-dimensional integrals, - многомерные интегралы
data augmentation, 454–5, 932 imputation step, 455, 932 for missing data, 932–8 prediction step, 455, 933 regression example, 933
data-generating process (dgp), 72–3, 124 misspecified, 90, 132
data mining, 285–6
data sets. See microdata - наборы данных, см. микроданные
data sets used in applications - наборы данных, используемые в приложениях
Current Population Survey Displaced Workers Supplement (McCall), 603–8, 632–6, 658–62
fishing-mode choice data (Kling and Herriges), 463–6, 486, 491–5
National Longitudinal Survey (Kling), 110–2 National Supported Work demonstration project
(Dehejia and Wahba), 889–95
Panel Survey of Income Dynamics cross-section
sample, 295–7, 300
Panel Survey of Income Dynamics panel sample
(Ziliak), 708–15, 754–6
patents-R&D panel data (Hausman, Hall, and
Griliches), 792–5
Rand Health Insurance Experiment expenditures,
553–6, 565
Rand Health Insurance Experiment medical doctor
contacts, 671–4, 692
strike duration data (Kennan), 574–5, 582 Vietnam World Bank Livings Standards Survey,
88–90, 848–53
see also applications with data
data structures, 39–62
data sources, 58–9 handling microdata, 59–61 natural experiments, 54–8 observational data, 40–8 social experiments, 48–54
data summary approach to regression, 820
Davidon, Fletcher, and Powell (DFP) algorithm - DFP алгоритм, алгоритм Дэвидона-Флетчера-Паулла
decomposition of variance, - разложение дисперсии
degenerate distribution, 948
degrees-of-freedom adjustment, 75, 102, 138, 185–6,
278, 841
delta method, - дельта-метод
bootstrap alternative, 363
density kernel, 421
density-weighted average derivative (DWAD)
estimator, - оценка
dependent variable, - зависимая переменная
descriptive approach to regression, 820
deviance, 149, 244
deviance residual, 289, 291
DFP algorithm. See Davidon, Fletcher, and Powell
algorithm - DFP алгоритм, см. алгоритм Дэвидона-Флетчера-Паулла
dgp. See data-generating process - процесс порождающий данные
diagnostic tests. See specification tests - диагностические тесты, см. тесты на спецификацию
DID estimator. See differences-in-differences differences-in-differences (DID) estimator, 55–7,
768–70, 878–9 application, 890–1 consistency, 770 definition, 768 introduction, 55–7 natural experiments, 878 with controls, 878–9 without controls, 878
direct regression, 906 disaggregated data
contrasted with aggregated data, 5–10 discrete factor models, 678
see also finite mixture models
discrete outcomes. See binary outcomes; counts;
multinomial outcomes
discrete-time duration data, 577–8, 600–3
cumulative hazard function, 578 discrete-time proportional hazards, 600–3 gamma heterogeneity, 620
hazard function, 578
logit model, - логит модель
ML estimator, - оценка максимального правдподобия
nonparametric estimation, - непараметрическая оценка
probit model, - пробит модель
survivor function, 578
dissimilarity parameter, 509
disturbance term. See error term - случайная составляющая, см. ошибка
double bootstrap, 374
dummy endogenous variable model, 557 dummy variable estimator, 784–5, 800, 805, 840
see also LSDV estimator duration data, 573–664
different types, 626, 641 duration models, 573–664
accelerated failure time, 591–2
applications, - приложения
censoring, 579–82, 587–9, 595, 642 competing risks, 642–8, 658–62
cumulative hazard function, 577–8 discrete time, 577–8, 600–3 generalized residual, 631
hazard function, 576, 578
key concepts, - ключевые понятия
mixture models, 613–25
ML estimator, - оценка максимального правдоподобия
multiple spells, 655–8 multivariate, 648–55 nonparametric estimators, 580–4 OLS estimator, 590–1
panel data, - панельные данные
parametric models, - параметрические модели
proportional hazards, 592–7
risk set, 581, 594
semiparametric estimation, - полупараметрическое оценивание 
specification tests, - тесты на спецификацию
survivor function, 576, 578
time-varying regressors, - регрессоры, меняющиеся во времени
see also proportional hazards model
DWAD estimator. See density-weighted average derivative
dynamic panel models, 763–8, 791–2, 797–9, 806–7
Arellano-Bond estimator, - Ареллано-Бонда оценка
binary outcome models, 806–7
count models, - счетные модели
covariance structures, - структуры ковариационной матрицы
inconsistency of standard estimators, - несостоятельность обычных оценок 
initial conditions, - начальные условия
IV estimators, - оценки методом инструментальных переменных
linear models, - линейные модели
MD estimator, 767
nonlinear models, - нелинейные модели 
nonstationary data, - нестационарные данные
transformed ML estimator, 766 true state dependence, 763–4 unobserved heterogeneity, 764 
weak exogeneity, - слабая экзогенность
EDF bootstrap. See empirical distribution function bootstrap
Edgeworth expansions, - разложение Эджуорта
efficient score, 141
Eicker-White robust standard errors, - Эйкера-Уайта робастные стандартные ошибки 
see also heteroskedasticity robust-standard errors - см. также стандартные ошибки, устойчивые к гетероскедастичности
EM algorithm see expectation maximization empirical Bayes method, 442
empirical distribution function (EDF) bootstrap, 360
see also paired bootstrap
empirical likelihood, - эмпирический метод максимального правдоподобия
empirical likelihood bootstrap, 379–80 encompassing principle, 283 endogeneity
definition, - определение
due to endogenous stratification, 78, 824–5 
Hausman test for,  - тест Хаусмана на эндогенность
identification frameworks and strategies, 35–7
see also endogenous regressors; exogeneity endogenous regressors, 78
binary, 557, 562
in count models, 683–4, 687–9
in discrete outcome models, 473
in duration models, 598
dummy, 557, 562
inconsistency of OLS, - несостоятельность МНК
in linear panel models, 744–63
in linear simultaneous equations model, 23–30 in nonlinear panel models, 792
in potential outcome model, 30–3 returns-to-schooling example, 69–70
in selection models, 559–62
in single-equation models, 30
see also GMM estimator; IV estimator
endogenous sampling, 42–5, 78, 822–9, 856 consistent estimation, 827–9
leading examples, 823
see also censored models; endogenous
stratification; sample selection models endogenous stratification, 820, 826–7, 856 equation-by-equation OLS, 210 equicorrelated errors, 701, 722–4, 804 equidispersion, 668, 670
error components model. See RE model error components SEM, 762
error components SUR model, 762 error components 2SLS estimator, 760 error components 3SLS estimator, 762 
error term, - ошибка, случайная составляющая
additive, - аддитивная
nonadditive, - неаддитивная
errors-in-variables. See measurement error estimated asymptotic variance, 954
see also asymptotic distribution
estimated prediction error. See cross-validation estimating equations estimator, 13–5
asymptotic distribution, 134–5, 174 clustered data, 842
computation, 339
definition, - определение
generalized, 134, 790, 794, 804 variance matrix estimation, 137–9 weighted, 829
see also MM estimator
Euler conditions, - Эйлера условия
exact identification. See just identification exchangeable errors, 701, 804
exhaustive sampling, 815–6
exogeneity, 22–3
conditional independence, 23
Granger causality, - причинность по Грэнджеру
of instrument, 106
overidentifying restrictions test for, 277 panel data assumptions, 700, 748–52, 754,
strong exogeneity, - сильная экзогенность
weak exogeneity, - слабая экзогенность
exogenous sampling, 42–3
exogenous stratified sampling, 42, 78, 814–5, 820,
825, 856
exogenous regressor. See exogeneity - экзогенный регрессор, см. экзогенность
expectation maximization (EM) algorithm, - EM-алгоритм, алгоритм максимизации ожидания
for data imputation, 930–2
E (Expectation) step, - Е-шаг
for finite mixture model, 623–5 
M (Maximization) step, - М-шаг
compared to NR algorithm, 625
expected elapsed duration, 626 
experimental data, - экспериментальные данные
control group, - контрольная группа
natural experiments, - естественные эксперименты 
social experiments, - социальные эксперименты 
treatment group, 49
explanatory variables. See regressors exponential conditional mean, 124, 155, 669 
coefficient interpretation, 124, 162–3, 669
exponential distribution, 140, 584–6
for generalized (Cox-Snell) residual, 631
exponential family density, - экспоненциальное семейство распределений 
conjugate prior for, - сопряженное априорное распределение
see also linear exponential family - см. также экспоненциальное семейство распределений
exponential-gamma regression model, - экспоненциальная-гамма регрессионная модель
exponential-IG regression model, 634 
exponential regression model - экспоненциальная регрессионная модель
application with censored data, - применение к цензурированным данным
example with uncensored data, - пример с нецензурированными данными
extreme value distribution. See type 1 extreme value extremum estimator, 124–39
asymptotic distribution, - асимптотическое распределение 
consistency, - состоятельность
definition, - определение
formal proofs, - формальные доказательства
informal approach, - неформальный подход 
statistical inference, 135–9 variance matrix estimation, 137–9
factor analysis, 650
factor loadings, 517, 650–1, 689
factor model, 517, 648, 686 Fairlee-Gumble-Morgenstern copula, 654
fast simulated annealing (FSA) method, 347–8
FD estimator. See first-differences
FE estimator. See fixed effects
feasible generalized least squares (FGLS) estimator, - оценка доступного обобщенного МНК
asymptotic distribution, - асимптотическое распределение 
definition, - определение
example, - пример
in fixed effects model, - в модели с фиксированными эффектами 
in mixed linear model, 775 nonlinear, 155–8
in pooled model, - в сквозной модели
feasible generalized least squares (cont.)
in random effects model, - в модели со случайными эффектами
as sequential two-step m-estimator, 201 systems FGLS, 208–9
feasible generalized nonlinear least squares (FGNLS) estimator, 155–8
asymptotic distribution, - асимптотическое распределение 
definition, - определение
example, - пример
as optimal GMM estimator, 180–1 systems FGNLS, 217
FGLS estimator. See feasible generalized least squares FGNLS estimator. See feasible generalized nonlinear
least squares
FIML estimator. See full information maximum
likelihood
finite mixture models, 621–5
counts, 678–9
definition, - определение
EM algorithm, - ЕМ-алгоритм
latent class interpretation, 623 number of components, 624–5 
panel data, - панельные данные
see also mixture models
finite-sample bias
of GMM estimator, 177 of IV estimator, 108–12 of tests, 250–4, 262
finite-sample correction term
for sampling without replacement, 817
first-differences (FD) estimator, 704–5, 729–31 application, 710–11, 714
asymptotic distribution, - асимптотическое распределение
compared to FE estimator, - сравнение с оценкой модели с фиксированными эффектами
consistency, 730, 764 definition, 704–5, 730 IV estimator, 758
first-differences (FD) model, 704, 729–31, 758 first-differences (FD) transformation, 783–4 fixed effects (FE) estimator, 704, 726–9, 756–9,
781–5, 791–2
application, 710–3, 792–5
asymptotic distribution, - асимптотическое распределение
binary outcome models, 796–9
clustered data, - кластеризованные данные
compared to DID estimator, 768
compared to FD estimator, 729
as conditional ML estimator, - как оценка условного максимального правдоподобия
consistency, - состоятельность
count models, - счетные модели
definition, - определение
duration models, 802
dynamic models, - динамические модели 
as FGLS estimator, 729
Hausman test for, - тест Хаусмана
identification, - идентифицируемость
incidental parameters,
inconsistency, - несостоятельность 
IV estimators, - оценки методом инструментальных переменных
as LSDV estimator, 733 
multinomial outcome models, 798 selection models, 801
Tobit model, - тобит модель
versus random effects, - сравнение со случайными эффектами
fixed effects (FE) model, - модель с фиксированными эффектами
cohort-level, 772
clustered data, - кластеризованные данные
definition, - определение
dynamic models, - динамические модели 
endogenous regressors, - эндогенные регрессоры
identification, - идентифицируемость
incidental parameters, 704, 726
marginal effects, - предельные эффекты
nonlinear models, - нелинейные модели 
time-varying regressors, - регрессоры, меняющиеся во времени
versus random effects, - сравнение со случайными эффектами
see also fixed effects estimators - см. также оценку модели с фиксированными эффектами
fixed coefficient, 846
fixed design. See fixed in repeated samples fixed in repeated samples, 76–7
bootstrap sampling method, 360 
in kernel regression, 312 
Liapounov CLT, - Ляпунова ЦПТ
Markov LLN, - Маркова ЗБЧ
Monte Carlo sampling method, 251
fixed regressors. See fixed in repeated samples flexible parametric models
count models, 674–5 hazard models, 592 selection models, 563
flow sampling, 44, 626
forward orthogonal deviations IV estimator, 759 forward orthogonal deviations model, 759 forward recurrence time, 626
Fourier flexible functional form, 321
frailty, 612, 662
see also unobserved heterogeneity Frank copula, 654
Frechet bounds, 653–4
frequentist approach, 421–2, 424, 439–40 FSA method. See fast simulated annealing full conditional distributions, 431
see also Gibbs sampler - см. также сэмплирование по Гиббсу
full information maximum likelihood (FIML) - метод максимального правдоподобия с полной информацией
estimator, - оценка
nested logit model, 510–2 nonlinear models, 219
functional approach
to measurement error, 901
functional form misspecification, 91–2 diagnostics for, 272–3, 277–8
gamma distribution, 585–6, 614 gamma function, 586
Gaussian quadrature, 389–90, 393, 809 Gauss-Hermite quadrature, 389–90 Gauss-Laguerre quadrature, 389–90 Gauss-Legendre quadrature, 389–90
Gauss-Newton (GN) algorithm, 345 example, 348
GEE estimator. See generalized estimating equations general to specific tests, 285
generalized additive model, 323, 327
generalized cross-validation, 315
generalized estimating equations (GEE) estimator, 790, 794, 804, 809
generalized extreme value (GEV) distribution, 508 see also nested logit model
generalized information matrix equality, 142, 145, 264 generalized inverse, 261
generalized IV estimator, - обобщенная оценка инструментальных переменных
generalized least squares (GLS) estimator, 81–5
asymptotic distribution, - асимптотическое распределение 
definition, - определение
as efficient GMM, 179 
example, - пример
nonlinear, 155–8
generalized linear models (GLMs), 149–50, 155
count data, - счетные данные
conditional ML estimator, - оценка условного максимального правдоподобия 
GEE estimator, 791
quasi-ML estimator, - оценка квази-максимального правдоподобия 
see also LEF models - см. также модели экспоненциального семейства распределений
generalized method of moments (GMM) estimator, - GMM-оценка, оценка обобщенного метода моментов
asymptotic distribution, - асимптотическое распределение
based on additional moment restrictions, 169,
178–9
based on moment conditions from economic theory,
171
based on optimal conditional moment, 179–80 bootstrap for, 379–80
computation, - вычисление
definition, - определение
endogenous counts, 683–4, 687–9
with endogenous stratification, - с эндогенной стратификацией
with exogenous stratification, - экзогенной стратификацией
examples, - примеры
finite-sample bias, - смещение из-за малого размера выборки
identification, - идентификация
linear IV, - линейный метод инструментальных переменных
linear systems, - линейные системы
nonlinear IV, - нелинейный метод инструментальных переменных
one-step GMM estimator, - одношаговая оценка обобщенного метода моментов 
optimal GMM, 176
optimal moment condition, 179–81, 188 
optimal weighting matrix, - оптимальная матрица весов
panel data, - панельные данные
practical considerations, - практические замечания
test based on, 245
two-step, 176, 187, 746, 755
variance matrix estimation, - оценка ковариационной матрицы
weak instruments, - слабые инструменты
see also panel GMM estimator - см. также оценка обобщенного метода моментов для панельных данных
generalized nonlinear least squares (GNLS) estimator.
See feasible generalized nonlinear least squares generalized partially linear model, 323
generalized random utility models, 515–6 generalized residual, 289–90
in duration models, 631 in LM test, 239–40 plots of, 633–6
generalized Tobit model, 548
generalized Weibull distribution, 584–6
genetic algorithms, 341
GEV distribution. See generalized extreme value Geweke, Hajivassiliou, Keane (GHK) simulator,
407–8
for MNP model, 518
GHK simulator. See Geweke, Hajivassiliou, Keane simulator
Gibbs sampler, - сэмплирование по Гиббсу
data augmentation, 454–5, 933 example, 452–4
in latent variable models, 514, 519, 563 see also Markov chain Monte Carlo
GLMs. See generalized linear models
GLS estimator. See generalized least squares
GMM estimator. See generalized method of moments GN algorithm. See Gauss-Newton
GNLS estimator. See feasible generalized nonlinear
least squares
Gompertz distribution, 585–6 Gompertz regression model, 606–8 gradient methods, 337–48
see also iterative methods Granger causality, 22
grid search methods, 337, 351 grouped data. See aggregated data
Halton sequences, 409–10 Hausman test, - тест Хаусмана
applications, - приложения
asymptotic distribution, - асимптотическое распределение
auxiliary regressions, - вспомогательная регрессия
bootstrap, - бутстрэп
computation, - вычисление 
definition, - определение
for endogeneity, - на эндогенность
for fixed effects, - для модели с фиксированными эффектами 
for multinomial logit model, - для мультиномиальной логит модели 
power, - мощность
robust versions, - робастные версии 
Hausman-Taylor IV estimator, 761 
Hausman-Taylor model, - Хаусмана-Тейлора модель 
Hawthorne effect, 53
hazard function
baseline in PH model, 591 cumulative hazard, 577–8, 582–4 definition, 576, 578
hazard function (cont.)
in mixture models, 616–8 multivariate, 649
nonparametric estimator, 581, 583 parametric examples, 585 piecewise constant, 591
see also duration models
Health and Retirement Study (HRS), 58
Heckit estimator. See Heckman two-step estimator - хекит оценка, см. Хекмана двушаговая оценка 
Heckman two-step estimator - двушаговая оценка Хекмана 
application, - приложения
in Roy model, - в модели Роя
in selection model, 550–1 
semiparametric estimator, - полупараметрическая оценка 
in Tobit model, - в тобит моделях
Hessian matrix - матрица Гессе
estimate, - оценка
Newton-Raphson algorithm, - Ньютона-Рафсона алгоритм 
singular, - вырожденная
heterogeneous treatment effects, 882, 885–7 IV estimator, 886–7
LATE estimator, 885
RD design, 882
heterogeneity
within-cell, 480
see also unobserved heterogeneity
heteroskedastic errors
adaptive estimation, 323, 328 conditional heteroskedasticity, 78 definition, 78
in GLMs, 149–50
in linear model, 84–5, 94–5 multiplicative, 84–5, 86–7
in nonlinear model, 157–63 residuals, 289–90
tests for, 241, 267, 275
Tobit MLE inconsistency, 538 working matrix for, 82–3, 156–8
heteroskedasticity-robust standard errors bootstrap, 379–80
clustered data, 834
example, - пример
for extremum estimator, 137, 164 
intuition, - интуиция
for NLS estimator, 155, 164
for OLS estimator, 74–5, 80–1, 112 panel data, 705
for WLS estimator, 83
see also robust standard errors hierarchical linear models (HLMs), 845–8
Bayesian analysis, 847
clustered data, 845
coefficient types, 846–7 individual-specific effects, 848 mixed linear models, 774–6, 847 panel data, 847–8
random coefficients model, 847 two-level model, 846
hierarchical models, 429
Bayesian analysis, 441–2, 447, 450, 514 see also hierarchical linear models
histogram, - гистограмма
see also kernel density estimator
HLM. See hierarchical linear model
hot deck imputation, 929, 940
HRS. See Health and Retirement Study Huber-White robust standard errors, 137, 144, 146
see also robust standard errors hurdle model, 680–1, 690
see also two-part model hyperparameters, 428, 847 hypothesis tests, 223–58
based on extremum estimator, 224–33 based on ML estimator, 233–43
based on GMM estimator, 245
based on m-estimator, 244
bootstrap, 254–6, 363–8, 372–3, 378–9
for common misspecifications, 274–7, 670–1 examples, 236, 241–3, 252–4, 254–6, 372–3 induced test, 230
joint versus separate, 230–1, 285, 629–30 power, 247–50, 253–4
size, 246–7, 251–3
see also LM tests; LR test; Wald tests, m-tests
identification
in additive random utility models, 504
in binary outcome models, 476, 483
bounds identification, 29
definitions, 29–31
in fixed effects model, 702
of GMM estimator, 173, 182
just identification, 31, 214
in linear regression model, 71–2
in measurement error models, 905–14
in mixture models, 618–20
in multinomial probit model, 517
in natural experiments, 57–8
observational equivalence, 29
order condition, - условие порядка
over identification, - сверх идентифицированность
rank condition, - условие ранга
in sample selection model, 551, 565, 566
set identification, 29
in simultaneous equations model, 29–31, 213–4 in single-index models, 325
and singular Hessian, 351
weak identification, - слабая идентифицируемость
see also identification strategies
identification strategies, 36–7
control function approach, 37 exogenization, 36
incidental parameter elimination, 36–7 instrumental variables, 37
matching, 37 reweighting, 37

identified reduced form, 36
IG distribution. See inverse-Gaussian ignorable missingness, 927
estimator consistency if MCAR, 927 estimator inconsistency if MAR only, 927 problems if nonignorable, 940
weak exogeneity, 927
ignorability assumption, 863
see also conditional independence assumption
importance sampling, 407–8, 443–5, 518 accelerated, 409
GHK simulator, 407–8
importance sampling density, 444 importance sampling estimator, 444 importance weight, 445
target density, 444 imputation methods, 928–39
data augmentation, 454–5, 932–4 example, 936–8
hot deck imputation, 929
listwise deletion, 928
mean imputation, 928–9
multiple imputation, 934–5 pairwise deletion, 928 regression-based imputation, 930–2
imputation (I) step, 455, 932
IM test. See information matrix test IMSE. See integrated mean squared error incidental parameters, 36
clustered data FE model, 832, 840, 844
panel data FE model, 704, 726, 781–2, 805 inclusive value, 510–1
incomplete gamma function, 586
incomplete panels. See unbalanced panels independence of irrelevant alternatives, 503, 505, 527 independent variables. See regressors independently-weighted IV estimator, 192 independently-weighted optimal GMM estimator, 177 index function model
binary outcome model, 475–6, 482–3 bivariate probit model, 522–3 ordered multinomial model, 519–20 Tobit model, 536
see also single-index model indicator function, 298
indirect inference, 404–5 individual-specific effects model
additive, 780
binary outcome models, 795–6 cluster-specific effects, 830 count models, 802–3 definitions, 700, 780
duration models, 802 multiplicative, 780, 793 one-way, 700
parametric, 780
selection models, 801 single-index, 780
Tobit models, 800–1
two-way, 738
see also FE models; RE models
induced test, 230
information criteria, 278–9, 283–4
Akaike, 278–9, 284, 624
Bayesian, 278, 284
consistent Akaike, 278 Kullback-Liebler, 147, 169, 278, 280 Schwarz, 278, 284
information matrix, 142 block-diagonal, 144, 240, 329
information matrix equality, 141–2, 145 generalized, 142, 145
see also BHHH estimate; OPG version
information matrix (IM) test, 265–6 
bootstrap, - бутстрэп
computation, - вычисления
definition, - определение
example, - пример
power, - мощность
instrumental variables (IV) estimator - оценка метода инструментальных переменных
alternative estimators, - альтернативные оценки
application, - приложения
definition, - определение
example, - пример
finite-sample bias, - смещение из-за малого размера выборки 
identification, - идентификация 
independently-weighted IV estimator, 192 jackknife IV estimator, 192
LIML estimator, - оценка метода максимального правдоподобия с ограниченной информацией
in linear model, - в линейных моделях
linear IV as GMM estimator, 170, 186
local average treatment effects estimator, 883–9 in measurement error models, 908–10, 912–3
in natural experiments, - в естественных экспериментах
in nonlinear models, - в нелинейных моделях
in panel models, - в моделях панельных данных
quantile regression, - квантильная регрессия
in selection models, 559
split-sample estimator, 191–2
systems IV estimator, 211–2, 218–9
in treatment effects models, 883–9
two-stage IV estimator, - двушаговая оценка метода инструментальных переменных
two-stage least squares estimator, 101–2, 187–91 Wald estimator, 98–9
see also GMM estimator; panel GMM estimator
instruments
definition, - определение
examples, - примеры
by exclusion restriction, 106
by functional form restriction, 106
invalid, 100, 105–7
optimal, 180
for panel data, 750–1, 754–6
relevance, 108
weak, 100, 104–12, 177–8, 191–2, 196, 751–2, 756 see also instrumental variables estimator
integrated hazard function. See cumulative hazard function
integrated mean squared error (IMSE), 303 integrated squared error (ISE), 302, 314 interval data models
definition, - определение
ML estimator, - оценка метода максимального правдоподобия
interruption bias, 626
intraclass correlation, 816, 831, 835–8 inverse-Gaussian (IG) distribution, 614–5, 677 inverse law of probability, 421
inverse-Mills ratio, 540–1, 553–4
inverse transformation method, 409, 412–3 inverse-Wishart distribution, 443, 453, 514 irrelevant regressors, 93
ISE. See integrated squared error
iterated bootstrap, 374
iterative methods, 337–48
BFGS, - BFGS
BHHH, - BHHH
convergence criteria, - критерий сходимости
DFP, 344, 350–1
expectation maximization, - максимизация ожидания
fast simulated annealing, 347–8
Gauss-Newton, 345, 348
line search, 338
Newton-Raphson, 338–9, 341–3, 348
numerical derivatives, 340
simulated annealing, 347
starting values, 340, 351
step size adjustment, 338
IV estimator. See instrumental variables - IV оценка, см. оценка метода инструментальных переменных
jackknife, 374–6
bias estimate, 375
bias-corrected estimator, 375 example, 376
IV estimator, - IV оценка
standard error estimate, - оценка стандартной ошибки
Jensen’s inequality, - неравенство Йенсена
jittered data, 290
joint duration distributions, 648–55
copulas, - копулы
mixtures, - смеси
multivariate hazard function, 649 multivariate survivor function, 649–50
joint limits, 767
joint versus separate tests, 230–1, 285, 629–30 just identification, 31, 100, 173
Kaplan-Meier (KM) estimator, - КМ-оценка, оценка Каплан-Мейера 
application, - приложение
for baseline hazard, 596–7 
confidence bands for, - доверительные интервалы 
definition, - определение
tied data, 582
kernel density estimator, - ядерная оценка функции плотности
alternatives to, - альтернативы
application, - приложения
asymptotic distribution, - асимптотическое распределение
bandwidth choice, - выбор ширины окна
bias, - смещение
confidence interval for, - доверительный интервал
consistency, - состоятельность
convergence rate, - скорость сходимости
definition, - определение
derivative estimator, 305
examples, - примеры
multivariate, 305–6
Nadaraya-Watson kernel regression estimator, 312 optimal bandwidth, 303
optimal kernel, - оптимальное ядро
variance, - дисперсия
kernel functions, - ядерные функции 
comparison, - сравнение
definition, - определение
higher-order, - высокого порядка 
leading examples, - основные примеры
optimal for density estimation, 303 
properties, - свойства
kernel matching, 875, 895–6 
kernel regression estimator, - ядерная оценка регрессии
alternatives to, - альтернативы
asymptotic distribution, - асимптотическое распределение
bandwidth choice, - выбор ширины окна
bias, - смещение
bootstrap confidence interval for, - бутстрэп доверительные интервалы
boundary problems, 309, 320–1
conditional moment estimator, 317–8
confidence interval for, - доверительные интервалы
consistency, - состоятельность
convergence rate, - скорость сходимости
definition, - определение
derivative estimator, 317
introduction to nonparametric regression, - введение в непараметрическую регрессию
multivariate, - многомерная 
optimal bandwidth, - оптимальная ширина окна
optimal kernel, - оптимальное ядро
undersmoothing, - недосглаживание
variance, - диспрерсия
see also nonparametric regression - см. также непараметрическая регрессия
Khinchine’s theorem, 948
KLIC. See Kullback-Liebler information criterion KM estimator. See Kaplan-Meier
k-NN estimator. See nearest neighbors estimator Kolmogorov LLN, 80, 111, 947
Kolmogorov test, - Колмогора тест
Kullback-Liebler information criterion (KLIC), - KLIC, Кульбака-Лейблера информационный критерий
LAD estimator. See least absolute deviations Lagrange multiplier (LM) test
1018
asymptotic distribution, 235, 237–8 based on GMM-estimator, 245 based on m-estimator, 244 bootstrap, 379
comparison with LR and Wald tests, 238–9 computation, 239–41, 256, 274
definition, - определение
examples, - примеры
for heteroskedasticity, 241, 267, 275 in duration models, 632 interpretation, 239–40
for omitted variables, 274
OPG version, 240–1
for random effects, 737, 841
score test, 234–5
in Tobit model, 544
for unobserved heterogeneity, 630, 636 see also hypothesis tests
Laplace approximation, - Лапласа приближение
Laplace distribution, - Лапласа распределение
Laplace transform, - Лапласа преобразование
LATE estimator. See local average treatment effects latent class model, 622
see finite mixture models latent variable, 475, 532 latent variable models
additive random utility model, 476–8, 504–7 binary outcomes, 475–8
endogenous, 560–1
ordered multinomial model, 519–20
see also censored models; truncated models law of iterated expectations, 955
law of large numbers (LLN), 947–8
definition, 947
examples of use, 80, 129 Khinchine’s theorem, 948 Kolmogorov LLN, 947 Markov LLN, 948
sampling schemes, 131, 948 strong law, 947
weak law, 947
least absolute deviations (LAD) estimator application, 88–90
asymptotic distribution, 88
binary outcome models, 484
bootstrap, 381
censored LAD, 564–5, 808 definition, 87
two-stage LAD, 190
see also quantile regression
least-squares dummy variable (LSDV) estimator, 704, 732–3, 840
least-squares dummy variable (LSDV) model, 704, 732, 840
least squares (LS) estimators
clustered data, 833–7
feasible generalized LS, 81–3, 155–8 generalized LS, 81–5, 155–8
linear, 70–85
nonlinear LS, 150–9
ordinary LS, 70–81
panel data, 211, 702–3, 720–5
systems of equations, 207–8, 211, 217
see also FGLS; FGNLS; OLS; NLS leave-one-out estimate, 192, 304, 315, 375 LEF. See linear exponential family length-biased sampling, 43–4, 626 Liapounov CLT, 80, 131, 950 likelihood-based hypothesis tests, 233–43
comparisons of, 235–6, 238–9 definitions, 234–5
examples, 236–7, 241–3
see also LM tests; LR tests; Wald tests
likelihood function, 139–41
conditional likelihood function, 139, 731–2, 824 definition, 139
joint, 19, 824–7
leading examples, 140–1
marginal, 432, 595
partial, 594–6
likelihood principle, 139, 420, 433 likelihood ratio (LR) test
asymptotic distribution, 235, 237
based on GMM-estimator, 245
based on m-estimator, 244
comparison with LM and Wald tests, 238–9 definition, 234
examples, 236, 241–3
nonnested models, 279–83 quasi-LR test statistic, 244 uniformly most powerful test, 237 see also hypothesis tests
LIML estimator. See limited information maximum likelihood
limit distribution, 948
see also asymptotic distribution
limit variance matrix, 952–3
definition, 952
replacement by consistent estimate, 952 sandwich form, 953
limited information maximum likelihood (LIML) estimator, 191, 214
Lindeberg-Levy CLT, 80, 131, 950
line search, 338
linear exponential family (LEF) models, 147–9
conjugate priors, 427–8 conditional ML estimator, 782 consistency, 148
leading examples, 148 pseudo-R2, 288
residuals, 289–90
tests based on, 240, 268, 274–5 see also generalized linear models
linear panel estimators, 695–778
application, 708–15, 725
Arellano-Bond estimator, 764–5
between estimator, 703
covariance estimator, 733
conditional ML estimator, 731–2 differences-in-differences estimator, 768–70
linear panel estimators (cont.)
error components 2SLS estimator, 760
error components 3SLS estimator, 762
first differences estimator, 704–5, 729–31
first differences IV estimator, 758
fixed effects estimator, 704, 726–9
fixed effects IV estimators, 757–9
forward orthogonal deviations IV estimator, 759 Hausman-Taylor IV estimator, 761
LSDV estimator, 704, 732–3
MD estimator, 753, 76–7
panel bootstrap, 708, 377–8, 708, 746, 751 panel GMM estimators, 744–68
panel-robust inference, 705–8, 722, 745–6, 751 pooled OLS estimator, 702–3, 720–5
random effects estimator, 705, 734–6
random effects IV estimator, 759–60
within estimator, 704, 726–9
within IV estimator, 758
linear panel models, 695–778 analysis-of-covariance model, 733 application, 708–15, 725
between model, 702
dynamic models, 763–8
endogenous regressors, 744–63
first differences model, 704, 730, 758
fixed effects model, 700–2, 726–34, 757–9 fixed versus random effects, 701–2, 715–9 forward orthogonal deviations model, 759 Hausman-Taylor model, 760–2
incidental parameters problem, 704, 726 individual dummies, 699
individual-specific effects model, 700
LSDV model, 704, 732
minimum distance estimator, 753, 766–7 mean-differenced model, 758
measurement error, 739, 905
mixed linear models, 774–6
pooled model, 699, 720–5
random effects differenced model, 760–1 random effects model, 700–2, 734–6, 759–60 residual analysis, 714–5
strong exogeneity, 700, 749–50, 752
time dummies, 699
time-invariant regressors, 702, 749–51 time-varying regressors, 702, 749–51 two-way effects model, 738
unbalanced data, 739
weak exogeneity, 749, 752, 758
within model, 704, 758
see also linear panel estimators
linear probability model, 466–7 linear programming methods, 341 linear regression model
definition, 16–17, 70–1
linear systems of equations, 207–14
panel data models as, 211
seemingly unrelated regressions, 2
simultaneous equations, 22–31, 213–4 systems FGLS estimator, 208
systems GLS estimator, 208
systems GMM estimator, 208
systems ML estimator, 214 systems OLS estimator, 211 systems 2SLS estimator, 212
linearization method, 855 link function, 149, 469, 783 listwise deletion, 60, 928
consistency under MCAR, 928 example, 936–8
inconsistency under MAR only, 928
Living Standards Measurement Study (LSMS), 59, 88–90, 848–53
LLN. See law of large numbers
LM test. See Lagrange multiplier test
local alternative hypotheses, 238, 247–8, 254 local average treatment effects (LATE) estimator,
883–9
assumptions, 884–5
comparison with IV estimator, 885 definition, 884
heterogeneous treatment effect, 885 monotonicity assumption, 885 selection on unobservables, 883 Wald estimator, 886
see also ATE; ATET; MTE
local linear regression estimator, 320–1, 333 local polynomial regression estimator, 320–1 local running average estimator, 308, 320 local weighted average estimator, 307–8 logistic distribution, 476–7
logistic regression. See logit model logit model, 469–70
application, 464–5
as ARUM, 477, 486–7
clustered data, 844
definition, 469
for discrete-time duration data, 602 GLM, 149
imputation example, 937–9
index function model, 476
marginal effects, 470
measurement error example, 919
ML estimator, 468–9
multinomial logit, 494–5, 500–3, 525 nested logit, 509–12, 526–7
ordered logit, 520
panel data, 795–9
probit model comparison, 471–3 random parameters logit, 512–6
see also binary outcome models
log-likelihood function. See likelihood function length-biased sampling, 43–4
log-logistic distribution, 585–6, 592 log-normal distribution, 585–6, 592 log-normal model, 533, 545–6

log-odds ratio, 470, 472
log-sum, 510
log-Weibull distribution. See type 1 extreme value long panel, 723–5, 767
longitudinal data. See panel data
loss function, 66–69
absolute error, 67
asymmetric expected error, 67 Bayesian decision analysis, 434–5 expected, 66
KLIC, 68, 147, 168, 278–9 squared error, 67–9, 156
step, 67–8
Lowess regression estimator, 320–1 application, 297, 309–10, 712–5
LR test. See likelihood ratio test
LS estimators. See least squares
LSDV. See least-squares dummy variable
LSMS. See Living Standards Measurement Study
MAR. See missing at random
marginal analysis of panel data, 717, 787 marginal effects, 122–4
in binary outcome models, 466–5, 467, 470–1 calculus method, 123
computing, 122–4
definition, 122
example, 162–3
finite-difference method, 123
in fixed effects model, 702, 788
in multinomial models, 493–4, 501–3, 519–23, 525 population-weighted, 821
in sample selection models, 552
in single-index models, 123
in Tobit model, 541–2
see also coefficient interpretation
marginal likelihood, 432, 595
marginal treatment effects (MTE) estimator, 886 market-level data, 482, 513
Markov chain Monte Carlo (MCMC) methods,
445–54
convergence, 449, 458
in data augmentation, 933
examples, 452–4, 512, 687, 936–9
Gibbs sampler, 448–50, 514, 519, 563 Metropolis algorithm, 450–1 Metropolis-Hastings algorithm, 451–2, 512
Markov LLN, 77, 131, 948 Marshall-Olkin method, 649–51, 686 matching assumption, 864
see also overlap assumption matching estimators, 871–8, 889–96
application, 889–96
assumptions, 863–5
ATE matching estimator, 877
ATET matching estimator, 874, 877, 894–6 balancing condition, 893
caliper matching, 874
counterfactuals, 871
exact matching, 872, 891
inexact matching, 873
interval matching, 875–6
kernel matching, 875, 895–6 nearest-neighbor matching, 875, 894–6 propensity score matching, 873–8, 892 radius matching, 876, 895–6
selection on observables only, 871 stratification matching, 875–6, 893–6 variance computation, 877–8, 895
maximum empirical likelihood (MEL) estimator, 206 maximum likelihood (ML) estimator, 139–46
asymptotic distribution, 142–3
conditional ML estimator, 731–2, 782–3, 796–9 consistency, 142, 824
definition, 141
endogenous stratification, 824–7
example, 143–4
exogenous stratification, 824
MSL estimator, 393–8
quasi-ML estimator, 146–50
regularity conditions, 141, 145–6
restricted, 233
unrestricted, 233
variance matrix estimation, 144
weighted ML estimator, 828
see also quasi-ML estimator
maximum rank correlation estimator, 485 maximum score estimator, 341, 381, 483–4, 800 maximum simulated likelihood (MSL) estimator,
393–8
asymptotic distribution, 394–5 bias-adjusted MSL, 396–7
compared to MSM, 402–3
count model examples, 677–8, 687, 689 definition, 394
example, 397–8
multinomial probit model, 518
number of simulations, 396
random parameters logit model, 522
MCAR. See missing completely at random
MD estimator. See minimum distance estimator mean-differenced estimator, 783, 805–6 mean-differenced model, 758, 783
mean imputation, 928, 936–8
mean integrated squared error (MISE), 303, 314 mean-scaling estimator, 783, 805–6 mean-square convergence, 946
mean substitution. See mean imputation measurement error
in cohort-level data, 772–3
in dependent variable, 913–4
in microdata, 46, 60
in panel data, 739, 905
in regressors, 899–922
see also measurement error model estimators;
measurement error models
measurement error model estimators, 899–922 attenuation bias, 903–5, 911, 915, 919–20 bounds identification, 906–8
corrected score estimator, 916–8
IV estimator, 908–10, 912–3
linear models, 900–11
nonlinear models, 911–20
OLS estimator inconsistency, 902–4
using additional moment restrictions, 909–10 using instruments, 908–9
using known measurement error variance, 902–3, 910
using replicated data, 910–1, 913
using validation sample, 911 measurement error models, 899–922
attenuation bias, 903–5, 911, 915, 919–20 classical measurement error model, 901–2 dependent variable measured with error, 913–4 examples, 919–20
identification, 905–14
linear models, 900–11
multiple regressors, 904
nonclassical measurement error, 904, 920 nonlinear models, 911–20
panel models, 905
scalar regressor, 903
serial correlation, 909
variance inflation, 904, 916
see also measurement error model estimators
median regression. See LAD estimator MEL. See maximum empirical likelihood m-estimator, 118–22
asymptotic distribution, 120 clustered data, 842–3 definition, 118–9
sequential two-step, 200–2 simulated m-estimator, 398–9 tests based on, 244, 263–4 weighted m-estimator, 829, 856 see also extremum estimators
method of moments (MM) estimator asymptotic distribution, 134, 174 definition, 172
examples, 167
see also estimating equations estimator; GMM estimator
method of scoring, 343, 348
method of simulated moments (MSM) estimator,
399–404
asymptotic distribution, 400–2 compared to MSL, 402–3 definition, 400
example, 403
MNP model, 497, 518
number of simulations, 399
method of simulated scores (MSS) estimator for MNP model, 519
method of steepest ascent, 344
Metropolis algorithm, 450–1 Metropolis-Hastings algorithm, 451–2, 512 microdata sets, 58–61
handling, 59–61
leading examples, 58–9 microeconometrics overview, 1–17 midpoint rule, 388, 391–2 minimum chi-square estimator, 203
see also Berkson’s minimum chi-square estimator minimum distance (MD) estimator, 202–3, 753, 766–7
asymptotic distribution, 202 bootstrap for, 379–80 covariance structures, 766–7 definition, 202 equally-weighted, 202 generalized, 222
indirect inference, 404–5 OIR test, 203
optimal, 202, 753
panel data, 753, 766–7 relation to GMM, 203, 753
misclassification, 914
MISE. See mean integrated squared error missing at random (MAR), 926–7
definition, 926
and ignorable missingness, 927, 932 relation to MCAR, 927
missing completely at random (MCAR), 926–7
definition, 927
and ignorable missingness, 927 relation to MCAR, 927
missing data, 923–41
deletion methods, 928
examples, 924
ignorable assumption, 927 imputation with models, 929–41 imputation without models, 928–9 MAR assumption, 926–7
MCAR assumption, 927 nonignorable missingness, 927, 940 see also imputation methods
misspecification tests. See specification tests mixed estimator, 439–41
mixed linear model, 774–6
Bayesian methods, 775
FGLS estimator, 775
fixed parameters, 774
ML estimator, 776
random parameters, 774 restricted ML estimator, 776 nonstationary panel data, 767–8 prediction, 776
see also hierarchical linear model mixed logit model, 500–3
1022
example, 495 definition, 500
see also RPL model
mixed proportional hazards (MPH) model, 611–25
Weibull-gamma mixture, 615
see also mixture models mixture hazard function, 616–8 mixture models, 611–25
application, 623–6
counts, 675–9
durations, 611–25
identification, 618–20
MSL estimator, 393–8, 687 multinomial outcomes, 515–6 multiplicative heterogeneity, 613 specification tests, 628–32
see also finite mixture models; unobserved heterogeneity
ML estimator. See maximum likelihood MM estimator. See method of moments MNL estimator. See multinomial logit MNP estimator. See multinomial probit model diagnostics, 287–91
binary outcome models, 473–4 duration models, 628–32 example, 290–1
multinomial outcome models, 499 pseudo-R2 measures, 287–9, 291 residual analysis, 289–91
see also model selection methods model misspecification, 90–4
see also endogeneity; functional form misspecification; heterogeneity; omitted values; pseudo-true value
model selection methods Bayesian, 456–8
nested models, 278–81 nonnested models, 278–84 order of testing, 285
see also model diagnostics; specification tests moment-based simulation estimators,
398–404
see MSL estimator; MSM estimator
moment-based tests. See m-tests moment matching. See indirect inference Monte Carlo integration, 391–2
direct, 391
example, 392
importance sampling, 407, 443–5 simulators, 393–4, 406–10
see also quadrature
Monte Carlo studies, 250–4 example, 251–4
moving average estimator, 308
moving blocks bootstrap, 373, 381
MPH model. See mixed proportional hazards
MSL estimator. See maximum simulated likelihood MSM estimator. See method of simulated moments MSS estimator. See method of simulated scores MTE. See marginal treatment effects
m-tests, 260–71
asymptotic distribution, 260, 263 auxiliary regressions, 261–3
bootstrap, 261, 379
chi-square goodness of fit, 266–7, 270–1,
474
conditional moment test, 264–5, 267–9, 319 CM test interpretation, 268
computation, 261–3
definition, 260
Hausman test, 271–4, 717–9
information matrix tests, 265–6, 270 outer-product-of-the-gradient form, 262 overidentifying restrictions test, 181, 183, 267,
747 power, 268 rank, 261
multicollinearity, 350–1
in multinomial probit model, 517
in panel model, 752
in sample selection model, 542, 551
multilevel models. See hierarchical models multinomial logit (MNL) model, 500–3, 525
application, 494–5
as additive random utility model, 505 definition, 500
marginal effects, 494, 501–3, 525 ML estimator, 501
panel data, 798
see also multinomial outcome models
multinomial outcome models, 490–528 application, 491–5 alternative-invariant regressors, 498 alternative-varying regressors, 497 conditional logit, 500–3, 524–5 definition, 496–7
identification, 504
index function model, 519–20 marginal effects, 501–3, 524–5 mixed logit, 500–3
ML estimator, 496, 501 multinomial logit, 500–3, 525 multinomial probit, 516–9 ordered models, 519–20
OLS estimator, 471
panel data, 798
random parameters logit, 512–6 random utility model, 504–7 semiparametric estimation, 523–4
multinomial probit (MNP) model, 516–9 Bayesian Methods, 519
definition, 516–7
identification, 517
ML estimator, 518
MSL estimator, 518
MSM estimator, 518
MSS estimator, 518
see also multinomial outcome models
multiple duration spells, 655–8 fixed effects, 656
lagged duration dependence, 657 ML estimator, 658
random effects, 657
recurrent spells, 655 multiple imputation, 934–9
estimator, 934
examples, 935–9
relative efficiency, 935 variance of estimator, 934–5
multiple treatments, 860
multiplicative errors
multistage surveys, 41–2, 814–6, 853–6
variance estimation, 853 multivariate data
binary outcomes, 521–3 counts, 685–7
durations, 640–64
see also systems of equations
multivariate-t distribution, 442
NA estimator. See Nelson-Aalen
National Longitudinal Survey (NLS), 58, 110–2 National Longitudinal Survey of Youth (NLSY),
58–9
National Supported Work (NSW) demonstration
project, 889–95
natural conjugate pair, 427–8 natural experiments, 32, 54–8
definition, 54
differences-in-differences estimator, 55–7, 768–70,
878–9
examples, 54
exogenous variation, 54–5 identification, 57–8
instrumental variables, 54–5
regression discontinuity design, 879–83
ncp. See noncentrality parameter
nearest neighbors (k-NN) estimator, 319–20
definition, 319
example, 308–9
symmetrized, 308, 320
see also nonparametric regression
nearest-neighbor matching, 875, 894–6 negative binomial distribution, 675 negative binomial model, 675–7
application, 690 bivariate, 215, 686–7 hurdle model, 681 ML estimator, 677 MSL estimator, 677–8 NB1 variant, 676 NB2 variant, 676 panel data, 804, 806
negative hypergeometric distribution, 806 neglected heterogeneity. See unobserved
heterogeneity

Nelson-Aalen (NA) estimator, 582–4 application, 605–6, 662 confidence bands for, 584 definition, 582
tied data, 582
nested bootstrap, 374, 379
nested logit model, 507–12, 526–7
from ARUM, 526–7 definition 510–1
different versions of, 511–2 example, 511
GEV model, 508, 526 ML estimator, 510 sequential estimator, 510 welfare analysis, 510
see also multinomial models nested models 278, 281
see also nonnested models
neural network models, 322
Newey-West robust standard errors, 137, 175,
723
definition, 175
see also robust standard errors
Newton-Raphson (NR) method, 341–3 examples, 338–9, 348
NLFIML estimator. See nonlinear full-information maximum likelihood
NLS estimator. See nonlinear least squares
NLSY. See National Longitudinal Survey of Youth NL2SLS estimator. See nonlinear two-stage least
squares
NL3SLS estimator. See nonlinear three-stage least
squares
noise-to-signal ratio, 903
noncentral chi-square distribution, 248 noncentrality parameter (ncp), 248 nonclassical measurement error, 904, 920 nongradient methods, 337, 341, 347–8 nonignorable missingness, 927, 940
attrition bias due to, 940
selection bias due to, 927, 932, 940 nonlinear estimators
coefficient interpretation, 122–4 extremum estimator m-estimator, 118–22
GMM estimator, 166–222
ML estimator, 139–46 NLS estimator, 150–9 overview, 117–22 panel models, 779–810
nonlinear full-information maximum likelihood (NLFIML) estimator, 219
nonlinear GMM estimator, 192–9 asymptotic distribution, 194–5 definition, 194–5
example, 197–8, 199, 688 instrument choice, 196 NL2SLS estimator, 196

optimal, 195
panel data, 789–90
nonlinear in parameters, 27
nonlinear in variables, 27
nonlinear IV estimator. See nonlinear GMM nonlinear least squares (NLS) estimator, 150–9
asymptotic distribution, 152–4 consistency, 152–3
definition, 151
example, 155, 159–64
time series, 158–9
variance matrix estimation, 154–5 nonlinear panel estimators, 779–810
application, 792–5
conditional ML estimator, 781–2, 805
dummy variable estimator, 784–5, 800, 805 first-differences estimator, 783–4
fixed effects estimator, 783–5, 794, 796–802, 805–8 GEE estimator, 790, 794, 804
mean-differenced estimator, 783, 805–6 mean-scaling estimator, 783, 805–6
ML estimator, 785–6
NLS estimator, 787, 794
panel GMM estimator, 789–90
panel-robust inference, 788–91
quadrature, 785–6, 796, 800
quasi-differenced estimator, 783–4
quasi-ML estimator, 791
random effects estimator, 785–6, 794–6, 800–1,
803–4
selection models, 801 semiparametric, 808
nonlinear panel models, 779–810 application, 792–5
binary outcome models, 795–6 conditional mean models, 780–1
count models, 792–5, 802–6
dynamic models, 791–2, 797–9, 806–7 endogenous regressors, 792 exogeneity assumptions, 781
finite mixture models, 786
fixed effects models, 781–5, 791–2
fixed versus random effects, 788
incidental parameters problem, 781–2, 805 individual-specific effects models, 780–1 parametric models, 780, 782–3, 785–7, 792 pooled models, 787, 794
random effects models, 785–6, 792 selection models, 801
semiparametric, 808
Tobit models, 800–1
transition models, 801–2
nonlinear regression model, 151 additive error, 168, 193, 217 nonadditive error, 168, 193, 218
nonlinear systems of equations, 214–9 additive errors, 217
copulas, 651–5
mixtures, 650–1
ML estimator, 215–6
NLFIML estimator, 219
NL3SLS estimator, 219
nonadditive errors, 217–8
nonlinear panel model, 216 nonlinear SUR model, 216 quasi-ML estimator, 150
seemingly unrelated regressions, 216 simultaneous equations, 219
systems FGNLS estimator, 217 systems GMM estimator, 219 systems IV estimator, 218–9 systems MM estimator, 218
systems NLS estimator, 217
nonlinear three-stage least squares (NL3SLS) estimator, 219
nonlinear two-stage least squares (NL2SLS) estimator asymptotic distribution, 195–6
definition, 195–6
example, 199
see also nonlinear GMM estimator nonnested models
Cox LR test, 279–80
definition, 278
example, 283–4
information criteria comparison, 278–9 overlapping, 281
strictly nonnested, 281
Vuong LR test, 280–3
nonparametric bootstrap. See paired bootstrap nonparametric density estimation. See kernel density
estimator
nonparametric maximum likelihood (NPML)
estimator, 622 nonparametric regression, 307–22
convergence rate, 311, 314 kernel, 311–9
local linear, 320
local weighted average, 307–8 Lowess, 320
nearest-neighbors, 308–9, 319–20 series, 321
statistical inference intuition, 309–11 test against parametric model, 319 see also semiparametric regression
nonrandomly varying coefficient, 846 normal copula, 654
normal distribution, 140
truncated moments, 540, 566–7
normal limit product rule. See Cramer linear
transformation
NPML estimator. See nonparametric maximum
likelihood
NR method. See Newton-Raphson method
NSW demonstration project. See National Supported
Work
nuisance parameters. See incidental parameters

numerical derivatives, 340, 350 numerical integration. See quadrature
observational data, 40–8, 814–7 biased samples, 42–5 clustering, 42
identification strategies, 36–7 measurement error, 46 missing data, 46
population, 40
sample attrition, 47
sampling methods, 40–4, 815–7 sampling units, 41, 815
sampling without replacement, 816–7 survey methods, 41–2, 814–7
survey nonresponse, 45–6
types of data, 47–8
observational equivalence, 29 odds ratio, 470
see also posterior odds ratio
OIR test. See overidentifying restrictions test OLS estimator. See ordinary least squares omitted variables bias, 92–3, 700, 716
LM tests for, 274
one-step GMM estimator, 187, 196
panel, 746, 755
see also two-stage least squares
one-way individual-specific effects model. See
individual-specific effects model
on-site sampling, 43, 823
optimal Bayesian estimator, 434
optimal GMM estimator, 176, 179–81, 187, 195
compared to 2SLS, 187–8
optimal MD estimator, 202, 753
OPG. See outer-product of the gradient
Orbit model, 914
order of magnitude, 954
ordered logit model, 520, 682
ordered multinomial models, 519–20
ordered probit model, 520, 535
ordinary least squares (OLS) estimator, 70–81
asymptotic distribution, 73–4, 80–1
bias in standard errors with clustering, 836–7 binary data, 471
clustered data, 833–7
coefficient interpretation in misspecified model,
91–2
consistency 72, 80
definition, 71
example, 84–5
finite-sample distribution, 79 heteroskedasticity-robust standard errors, 74–5, 81 identification, 71–2
inconsistency, 91, 95–6
inefficiency, 80
nonlinear, 150–9
panel data, 702–3, 720–5
see also least squares estimators
orthogonal polynomials, 321, 329, 390 definition 390
orthogonal regression approach, 920 orthonormal polynomials, 321, 329, 390 outcome equation, 547, 867
outer product (OP) estimate, 138, 241, 395 outer-product of the gradient (OPG) version
LM test, 240–1
m-test, 262–4
small-sample performance, 262
overdispersion, 670–1, 674–6, 690 measurement error, 915–6 panel data, 794, 806
tests for, 671
overidentification, 31, 100, 173, 176, 379–80, 747 see also GMM estimator
overidentifying restrictions (OIR) test asymptotic distribution, 181, 183 bootstrap, 379–80
definition, 181, 267, 277
panel data, 747, 756 overlap assumption, 864, 871
in RD design, 881 oversampling, 41, 478–9, 814, 872
paired bootstrap, 360, 366–8, 376, 378 pairwise deletion, 928
biased standard errors, 928
panel attrition, 739, 801
panel bootstrap, 377, 707, 746, 751, 789 panel data, 47
panel data models and estimators, 695–810
comparison to clustered data, 831–2 see also linear panel; nonlinear panel panel GMM estimators, 744–68, 789–90
application, 754–6
Arellano-Bond estimator, 765–6 asymptotic distribution, 745–6 bootstrap, 389–90
compared to MD estimator, 753 computation, 751–2
definition, 745
efficiency, 747, 756
exogeneity assumptions, 748–52 instruments, 744, 747–51
IV estimators for FE model, 757–9 IV estimators for RE model, 759–60 just-identified, 745
nonlinear, 789–90
OIR test, 747, 756
one-step GMM estimator, 746, 755 overidentified, 745
2SLS estimator, 746, 755
two-step GMM estimator, 746, 755 variance matrix estimation, 751
panel GMM model, 744–66 application, 754–6 dynamic, 763–6

with individual-specific effects, 750–62 without individual-specific effects, 744–53 see also panel GMM estimators
panel IV estimators. See panel GMM estimators panel-robust statistical inference, 377, 705–7, 722,
746, 751, 788–90 for Hausman test, 718
Panel Study in Income Dynamics (PSID), 58, 889 parametric bootstrap, 360
Pareto distribution
of the first kind, 609
of the second kind, 616
partial additive model, 323
partial equilibrium analysis, 53, 862, 972
see also SUTVA
partial F-statistic, 105, 109, 111
partial likelihood estimator, 594–6
partial ML estimator, 140
partial R-squared, 104–5, 111
partially linear model, 323–5, 327, 565, 684 participation equation, 547, 551
Pearson chi-square goodness-of-fit test, 266 Pearson residual, 289, 291
peer-effects model, 832
percentile, 86
percentile method, 364–5, 367–8 percentile-t method, 364, 366–7
PH model. See proportional hazards piecewise constant hazard model, 591 Pitman drift, 248
PML estimator. See pseudo-ML estimator Poisson distribution, 668
Poisson-gamma mixture, 675
Poisson-IG mixture, 677
Poisson regression model, 666–74
application, 671–4, 690, 792–5, 850–3 asymptotic distribution of estimators, 668–9 bivariate, 686
censored MLE, 535
with clustered data, 844, 850–3 coefficient interpretation, 669 definition, 668
equidispersion, 668
example, 117–8, 121–2 LEF density, 148 measurement error, 915–8 mixtures, 675–9
ML estimator, 668
overdispersion, 670–1
panel data, 792–5, 802–6 quasi-ML estimator, 668–9, 682–3 truncated MLE, 535 underdispersion, 671 zero-truncated, 680
see also count models
polynomial baseline hazard, 591, 636
pooled cross-section time series model. See pooled
model
pooled estimators, 702–3, 720–5 application, 710–2, 725 FGLS estimator, 720–1
GEE estimator, 790, 794
NLS estimator, 794
OLS estimator, 211, 702–3, 720–5 WLS estimator, 702–3, 721
pooled model, 699, 720–5, 787–8
pooling tests, 737
population-averaged model. See pooled model population moment conditions
for estimation, 172
for testing, 260
see also GMM estimator; MM estimator; m-tests
posterior distribution, 421, 430–4 asymptotic behavior, 432–4 conditional posterior, 431 definition, 421
expected posterior loss, 434
expected posterior risk, 434
full conditional distribution, 431 highest posterior density interval, 431 highest posterior density region, 431 marginal posterior, 430 observed-data posterior, 930 posterior density interval, 431 posterior mean, 423, 434
posterior mode, 433
posterior moments, 430
posterior precision, 423
see also Bayesian methods
posterior odds ratio, 456
posterior (P) step, 455, 933
potential outcome model, 30–4, 861–5
see also treatment effects; treatment evaluation power of tests, 247–50, 253–4
bootstrapped tests, 372–3 conditional moment test, 267–9 example, 253–4
Hausman test, 273–4
local alternative hypotheses, 247–8 uniformly most powerful test, 237 Wald tests, 248–50
precision parameter, 423
predetermined instruments. See weak exogeneity prediction, 66–70
best linear, 70
conditional, 66
error, 66–70
in linear panel models, 738 in mixed linear model, 774–6 optimal, 66–70
rotation groups, 814 in structural model, 28 weighted, 821
pretest estimator, 285
primary sampling units (PSUs), 41, 815,
1027
845–55

prior distribution, 425–30 conjugate prior, 427 definition, 420 Dickey’s prior, 439 diffuse prior, 426
flat prior, 426
hierarchical priors, 428–9, 441–2 improper prior, 426
informative prior, 437–9 Jeffreys’ prior, 426 noninformative prior, 425, 435–7 normal-gamma prior, 437 sensitivity analysis for, 429–30 see also Bayesian methods
probit model, 470–71
application, 465–6
as additive random utility model, 477 bivariate probit, 522–3
bootstrap example, 254–6
definition, 470
discrete-time duration data, 602
as GLM, 149
index function model, 476
logit model comparison, 471–3
marginal effects, 467, 471
ML estimator, 470
Monte Carlo study example, 251–4 multinomial probit, 516–9
ordered probit, 520, 535
panel data, 795–6
simultaneous equations probit, 523, 560–1 see also binary outcome models
probit selection equation, 548 product copula, 654
product integral, 578
product rule, 949
see also Cramer linear transformation program evaluation. See treatment evaluation projection pursuit model, 323
propensity score, 864–5
application, 893–4
balancing condition, 864, 893–4 conditional independence assumption, 865 definition, 864
matching, 873–8, 892
see also treatment evaluation
proportional hazards (PH) model, 592–7 application, 605–7
baseline survivor function estimator, 596–7 coefficient interpretation, 606–7
competing risks model, 645–6 definition, 591
discrete-time model, 600–3 leading examples, 585
mixed PH, 611–25
panel data, 802
partial likelihood estimator, 
pseudo-ML estimator (PML). See quasi-ML estimator

pseudo panels, 771–3 cohort, 771
cohort fixed effects, 772–3 measurement error, 772–3
pseudo-random number generators, 410–6, 957–9 accept-reject methods, 413–4
composition methods, 415
inverse transformation method, 413
leading distributions, 957–9 multivariate normal, 416 transformation method, 413 uniform variates, 412
see also MCMC methods pseudo R-squared measures
for binary outcome models, 473–4 definitions, 287–9
example, 290–1
for multinomial outcome models, 499
pseudo-true value, 94, 132, 146, 281
PSID. See Panel Study in Income Dynamics PSUs. See primary sampling units
pure exogenous sampling, 825
p-value, 226, 229, 234, 286, 363
quadrature, 388–90
Gaussian, 389–90
multidimensional, 393
in nonlinear panel models, 785–6, 796, 800 see also Monte Carlo integration
qualititative response models. See binary outcomes, multinomial outcomes
quantile, 86–7
quantile regression, 85–90
application, 88–90
asymmetric absolute loss, 68, 85 asymptotic distribution, 88
bootstrap, 381
computation, 341
definition, 87
IV estimator, 190
multiplicative heteroskedasticity, 86–7
quasi-difference, 783–4
quasi-experiment. See natural experiment quasi-maximum likelihood (QML) estimator, 146–50
asymptotic distribution, 146
in binary outcome models, 469
in clustered models, 842–3
definition, 146
in LEF, 147–9
with multivariate dependent variable, 150 in nonlinear systems, 216
in panel models, 768, 786
in Poisson model, 668–9, 682–3
quasi-random numbers. See pseudo-random numbers QML estimator. See quasi-ML estimator
random assignment, 49–50, 862 see also sampling schemes
random coefficients model, 94, 385, 774–6, 786 see also hierarchical models
random effects (RE) estimator, 705, 734–6, 759–62, 785–6
application, 710–1, 725 asymptotic distribution, 735 clustered data, 837–9, 843–4 consistency, 699, 764 definition, 705, 734
error components 2SLS estimator, 760 error components 3SLS estimator, 762 FGLS estimator, 734–6
GEE estimator, 790, 794, 804 Hausman test, 717–9
incidental parameters, 704, 726
IV estimators, 759–60
ML estimator, 736, 785–6, 794–7, 800–1, 803–4 NLS estimator, 787, 794
quasi-ML estimator, 791
two-way effects model, 738
versus fixed effects, 701–2, 715–9
random effects (RE) model, 700–2, 734–6, 759–62, 785–6
binary outcome models, 795–6 Chamberlain model, 719, 786 clustered data, 831, 843–4 count models, 794, 803–4 definition, 700, 734
dynamic models, 792
duration models, 801–2
endogenous regressors, 756–7, 759–62 Mundlak model, 719
nonlinear models, 785–6
selection models, 801
Tobit model, 800–1
two-way effects model, 738
versus random effects, 701–2, 715–9
see also hierarchical models; random effects
estimator
random number generators. See pseudo-random
numbers
random parameters logit (RPL) model, 512–6
Bayesian methods, 514 definition, 513
ML estimator, 513–4
random parameters model. See random coefficients model
random utility models. See ARUM randomization bias, 53, 867 randomized experiment, 50–3
National Supported Work demonstration project, 889
randomized trials, 49–53
randomly varying coefficient, 847–8
rank condition for identification, 31, 182, 214 rank-ordered logit model, 521
rank-ordered probit model, 521
raw residual, 289, 291
RD design. See regression discontinuity design receiver operators characteristics (ROC) curve, 474 reduced form, 21, 25, 213
see also structural model
RE estimator. See random effects regression-based imputation, 930–2
EM algorithm, 932
nonignorable missingness, 932
regression discontinuity (RD) design, 879–83
fuzzy RD design, 882
heterogeneous treatment effects, 882
RD estimator, 882–3
sharp RD design, 880–1
treatment assignment mechanism, 879–81
regressors, 71
alternative-varying, 478, 497–8 endogenous, 23–33
fixed, 76–7
irrelevant, 93
omitted, 92–3
stochastic, 77
time-varying, 597–600, 702, 749–51 see also endogenous regressors
regularity conditions for ML, 141–2, 151–6 relative risk, 470, 503
reliability ratio, 903
renewal function, 626
renewal process, 626, 638
repeated cross section data, 47, 770–3
see also differences-in-differences repeated measures. See panel data replicated data, 910–1, 913
RESET test, 277–8
residual analysis
definitions, 289–90
duration data, 633–6 example, 290–1
panel data, 714–5 small-sample correction, 289
residual bootstrap, 361
response-based sampling, 43
restricted ML estimator, 233, 776
revealed preference data, 498, 516
ridge regression estimator, 440
Robinson difference estimator, 324–5, 565 robust sandwich variance matrix estimate. See
sandwich variance matrix robust standard errors
bootstrap, 362–3, 376–8
Eicker-White, 74–5, 80–1, 112, 137
for extremum estimator, 137–9
Huber-White, 137, 144, 146
Newey-West, 137, 175, 723
see also cluster-robust; heteroskedasticity-robust;
panel-robust; systems-robust
ROC curve. See receiver operators characteristics
curve
rotating panels, 739
Roy model, 555–7, 562
definition, 556
dummy endogenous variable, 557 Heckman two-step estimator, 556 ML estimator, 556
panel semiparametric estimation, 808 as treatment effects model, 867
RPL model. See random parameters logit R-squared, 287
pseudo, 287–9
uncentered, 241, 263 running mean estimator, 308
SA method. See simulated annealing sample attrition, 47
sample moment conditions
see population moment conditions sample selection bias, 44–5
sample weights, 817–21, 853–6
see also weighting sampling schemes
assumptions for OLS, 76–78
case-control, 479, 823
choice-based sampling, 43, 478–9, 823 endogenous sampling, 42–5, 78, 822–9, 856 endogenous stratified sampling, 78, 820, 825–6,
856
exogenous stratified sampling, 42, 78, 814–5, 820,
825, 856
fixed in repeated samples, 76–7
flow sampling, 44, 626
multi-stage surveys, 41–2, 814–6, 853–6 on-site sampling, 43, 823
simple random sampling, 41, 76–7, 816 stock sampling, 44, 626–7
with replacement, 816
without replacement, 816–7
sandwich variance matrix clustered data, 834, 842 extremum estimator, 132, 137–9 GMM estimator, 175
ML estimator, 144, 148
NLS estimator, 150
OLS estimator, 74
panel data, 705–7, 722, 746, 751 for Wald test, 277
see also robust standard errors Sargan test, 277
see also overidentifying restrictions test scale parameter, 509
scanner data, 499
Schwarz criterion. See BIC
SCLS estimator. See symmetrically censored least squares
score test, see Lagrange multiplier test
score vector, 141
secondary sampling units (SSUs), 41, 815, 854 seed, 411
seemingly unrelated regressions (SUR) model, 209–10, 216
Bayesian MCMC example, 452–4 count data, 685
error components, 762
nonlinear, 216
selection bias, 445
nonignorable missingness, 927, 932, 940 treatment effects models, 867–71
see also selection models
selection models, 546–62
bivariate sample selection model, 547–53
count models, 680
example, 553–5
panel data, 801
Roy model, 555–7, 867
sample selection, 546
self selection, 546
semiparametric estimation, 565–6
structural models, 558–62
treatment effects model, 862–4
versus selection on observables only, 552–3, 864,
868–71
versus two-part models, 546, 552–3 see also Tobit models
selection on observables only, 552–3, 862–4, 868–9, 878–3, 889–96
compared to selection models, 552–3, 864, 871 conditional independence assumption, 868 control function estimator, 869
definition, 868–9
DID estimator, 878–9
RD design estimator, 879–83
treatment effects model, 862–4, 889–96
selection on unobservables, 552–3, 865–71, 883–9 definition, 868
in treatment effects model, 862–4
IV estimators, 883–9
Roy model, 867 selection bias, 867–71 selection model, 552–3
self-weighting sample, 818
SEM. See simultaneous equations model seminonparametric ML estimator, 328–9, 485 semiparametric efficiency bounds, 323, 329–30, 485 semiparametric estimators, 322–30
adaptive, 323
application, 565
average derivative estimator, 326
efficiency bounds, 323, 329–30 nonparametric FGLS, 328
Robinson difference estimator, 324–5, 565 semiparametric least squares, 327, 483 seminonparametric ML estimator, 328–9, 485 see also semiparametric models
semiparametric heterogeneity model, 622 see also finite mixture models
semiparametric least squares, 327, 483
semiparametric ML estimator, 328–9, 485 semiparametric models, 322–30
additive models, 327
binary outcome models, 482–6 censored models, 563–5
count models, 684–5
definition, 322
duration models, 594–600, 601–2 flexible parametric models, 563 heteroskedastic linear model, 323, 328 identification, 325–6
leading examples, 322
multinomial outcome models, 523–4 panel data models, 808
partially linear model, 324–5
selection models, 565–6
single-index models, 325–7
see also semiparametric estimators
sequential limits, 767
sequential multinomial models, 520–1 sequential two-step m-estimator, 200–2
bootstrap for, 362
sequence of random variables, 943, 945 serial correlation. See autocorrelation set identification, 29
series estimator, 321
for binary outcomes, 483 shared frailty model, 662 short panel
definition, 700
statistical inference in, 705–8, 721–2, 746, 751, 768 shrinkage estimator, 440
Silverman’s plug-in estimate, 304
simple random sampling (SRS), 41, 76–7, 816
simple stratified sampling, 818
Simpson’s rule, 388–9
simulated annealing (SA) method, 347 simulated m-estimator, 398–9 simulation-based estimation methods, 364–418
motivating examples, 385–6
see MSL, MSM, indirect inference, simulators simulators, 393–4, 406–10
antithetic sampling, 408–9 direct, 393
frequency, 406
GHK, 407–8
Halton sequences, 409–10 importance sampling, 407 smooth, 407 subsimulator, 394 unbiased, 394, 400
see also quadrature
simultaneous equations model (SEM), 22–31, 213–4,
219
causal interpretation, 26
error components, 762
extension to nonlinear models, 27 FIML estimator, 214
identification, 29–31, 213–4 LIML estimator, 214 nonlinear, 219
order condition, 213
rank condition, 214 reduced form, 25, 213 single-equation models, 31 structural form, 25, 213 structural model, 24
2SLS estimator, 214
3SLS estimator, 214
simultaneous equations probit, 523, 560–1 simultaneous equations Tobit, 560–1 single-index models, 123, 323, 325–7
definition, 123
identification, 325
marginal effects, 123
nonlinear panel model, 780 semiparametric estimators, 325–7
SIPP. See Survey of Income and Program Participation size of test, 246–7, 251–3
nominal size, 251 size-corrected test, 251 true size, 251–3
Sklar’s theorem, 652 Slutsky’s Theorem, 945–6 alternative version, 949
small-sample bias. See finite-sample bias smooth maximum score estimator, 484 smoothing parameters, 307
smoothing spline estimator, 321
social experiments, 32, 48–54 advantages, 50–2 examples, 51, 889 limitations, 52–4 randomization, 49–50
span, 320
specific to general test, 285 specification tests, 259–78
for clustered data, 840
for duration models, 628–32
for endogeneity, 275–6
for exogeneity, 277
for heteroskedasticity, 275
for individual-specific effects, 737 for omitted variables, 274
for overdispersion, 670–1
for pooling, 737
for unobserved heterogeneity, 628–32 for Tobit model, 543–4
see also m-tests; model diagnostics
spherical errors, 78
split-sample IV estimator, 191–2
SRS. See simple random sampling
SSUs. See secondary sampling units
stable family of distributions, 621
stable unit treatment value assumption (SUTVA), 872 standard errors. See robust standard errors
starting values, 340, 351
state dependence. See true state dependence stated preference data, 498, 516
stationary population, 40
statistical packages, 349
step size adjustment, 338
stochastic order of magnitude, 954–5
stock sampling, 44, 626–7
strata, 41, 815
see also sampling schemes; weighting stratification matching, 875–6, 893–6 stratified random sampling, 76–7, 814–5
use of Liapounov CLT, 951
use of Markov LLN, 948
see also sampling schemes; weighting
strict exogeneity. See strong exogeneity strong consistency, 947
strong exogeneity, 22
in panel models, 700, 749–50, 752, 781 structural approach
to measurement error, 901
to weighting, 820–1
structural economic models, 28, 171
with selection, 558–60 structural form, 20, 25, 223 structural model, 20–31, 35–6
based on economic model, 28 exogeneity, 22–3
full information, 35
limited information, 35 reduced form, 21, 25, 223 structural form, 20, 25, 223 structure, 20
see also simultaneous equations model structural selection models, 558–62
based on utility maximization, 558–60 endogenous regressors, 561–2 simultaneous equations Tobit, 560–1
studentized statistic, 359 subsampling method, 373 substitution bias, 53, 867
sufficient statistic, 732, 782, 799, 805
definition, 782
summation assumption, 748, 752 superpopulation, 40, 816
supersmoother, 321
SUR model. See seemingly unrelated regressions survey methods, 41–2, 84–7, 814–8, 853–6 survey nonresponse, 45–6, 60, 739
see also attrition bias; imputation methods
Survey of Income and Program Participation (SIPP),
59
survival analysis. See duration models survival function. See survivor function survivor function
aggregate survivor function, 619 definition, 576–8
estimator in PH model, 596–7
Kaplan-Meier estimator, 581–2, 604–5 in mixture models, 615–6 multivariate, 649–50
parametric examples, 585
SUTVA. See stable unit treatment value assumption switching regressions model. See Roy model symmetrically censored least squares (SCLS)
estimator, 565
synthetic panels. See pseudo panels systems of equations, 206–19
linear systems, 206–14
nonlinear systems, 214–9
seemingly unrelated regression, 209–10, 216 simultaneous equations model, 22–31, 213–4, 219
systems-robust standard errors, 208–9, 212, 219
target density, 444
tests. See hypothesis tests, m-tests, specification tests three-stage least squares (3SLS) estimator, 214 3SLS estimator. See three-stage least squares
time series data
bootstrap, 381
NLS estimator, 158–9
Newey-West standard errors, 137, 175, 727
time-varying regressors
in duration models, 597–9
in panel data models, 702, 749–51
Tobit model, 536–44
Bayesian methods, 563 censored mean, 538–41 censoring mechanism, 532, 579 consistency of MLE, 538 definition, 536
example, 530–1
generalized, 548
Heckman two-step estimator, 543, 567–8 identification, 536
as imputation method, 932
inverse-Mills ratio, 540–1
marginal effects, 541–2
measurement error in dependent variable, 914 ML estimator, 537–8
NLS estimator, 542
OLS estimator, 543
panel data, 800–1
simultaneous equations, 560–1
specification tests, 543–4
with stochastic thresholds, 547
with truncated data, 538
truncated mean, 538–41, 566–7
two-limit, 536
type 2, 547
type 5, 557
see also selection models
top-coded data, 532–3, 541, 563 transformation methods, 413 transformation theorem, 949 transformed ML estimator, 766

transition data. See duration models trapezoidal rule, 388 treatment-control comparison
application, 890–1
treatment effects framework, 862–5, 871–8, 889–96
balancing condition, 864, 893–4
binary treatment variable, 862
conditional independence assumption, 863, 865 conditional mean independence assumption, 864 heterogeneous treatment effects, 882, 885 multiple treatments, 860
overlap assumption, 864, 871
propensity score, 864–5
Roy model, 867
stable unit treatment value assumption, 872
see also treatment evaluation
treatment evaluation, 860–98
application, 889–96
IV estimators, 883–9
matching estimators, 871–8
DID estimators, 878–9
selection bias, 865–71
selection on observables, 862–4, 878–3, 889–96 selection on unobservables, 865–71, 883–9 regression discontinuity design, 879–83
see also treatment effects framework treatment group, 49, 862
trimming, 316, 333
trivariate reduction, 686
true state dependence
duration models, 612, 630, 636 dynamic panel models, 763–4, 798, 802 see also unobserved heterogeneity
truncated models, 530–44 conditional mean, 535 count models, 679–80 definition, 532 examples, 530–1, 535 ML estimator, 534
see also Tobit model; selection models truncated moments of standard normal, 540, 566–7 truncation mechanisms, 532
truncation from above, 532
truncation from below, 532
2SLS estimator. See two-stage least squares two-limit Tobit model, 536
two-part model, 544–6
application, 553–5
compared to selection models, 546, 552–3 definition, 545
example, 545–6
see also hurdle model
two-stage IV estimator, 187
two-stage least squares (2SLS) estimator, 101–2,
187–91
alternatives to, 190–2
Basmann’s approach, 190–1 compared to optimal GMM, 187–8
as GLS in transformed model, 188–9 as GMM estimator, 187
nonlinear, 195–6, 199
panel data, 746, 755
in SEM, 214
Theil’s interpretation, 189–90 two-stage sampling, 41, 818 two-step estimators
GMM, 176, 187
Heckman, 543, 550–1, 556, 567–8 sequential m-estimator, 200–2
two-step GMM estimator, 176, 187 panel, 746, 755
two-way effects model, 738
type I error, 246–7
type II error, 246–7
type 1 extreme value distribution, 477, 486–7
duration model error, 590
multinomial logit model, 505
type 2 Tobit. See bivariate sample selection model type 5 Tobit. See Roy model
ultimate sampling units (USUs), 41, 815 unbalanced panels, 739
uncentered explained sum of squares (ESS), 241 uncentered R-squared, 241, 263 unconfoundedness assumption. See conditional
independence assumption underrecording, 915
undersmoothing, 305, 333, 380
uniform convergence in probability, 126, 301 uniform number generators, 412
uniformly most powerful (UMP) test, 247 unit roots, 382, 767–8
universal logit model, 500
unobserved heterogeneity
application, 632–6
in competing risks model, 647
in count models, 675–7, 686
distributions for, 614–5, 620–1
in duration models, 611–25
finite mixture models for, 621–5
identification, 618–20
IM test for, 267
individual-specific effects, 700, 764
mixture models for, 613–21
MSL example, 397–8
MSM example, 403
multiplicative, 613, 686
in nonlinear systems, 215
specification tests for, 629–32
variance inflation, 614
versus true state dependence, 612, 630, 636, 763–4,
798, 802
USUs. See ultimate sampling units
validation sample, 911 variance components, 735, 845
variance matrix estimation
BHHH estimate, 138
degrees-of-freedom adjustment, 75, 102, 138,
185–6, 278, 841
expected Hessian estimate, 138 for extremum estimator, 137–9 for GMM estimator, 174–5 Hessian estimate, 138
for NLS estimator, 154–5
OPG estimate, 138
robust estimate, 137
sandwich estimate, 137, 144 for weighted estimators, 854–6 see also robust standard errors
variance reduction for simulation, 478
Wald estimator
in treatment effects models, 886
Wald test, 136–7, 224–33
asymptotic distribution, 226–8 comparison with LM and, LR tests, 238–9 definition, 136
examples, 236, 241–3
exclusion restrictions, 227
F-test version, 226
introduction, 136–7
lack of invariance, 232–3
likelihood based, 234, 241–3
linear models, 224–5
linear restrictions, 136–7
in misspecified models, 229–30
nonlinear restrictions, 224, 229
power, 248–50
of statistical significance, 228
t-test version, 226–8
see also hypothesis tests
weak consistency, 947 weak exogeneity, 22
in panel data, 749, 752, 758 weak instruments, 100, 104–12
application, 110–2
definition, 104
finite sample bias, 108–12, 177–8, 191–2, 196 GMM estimator, 177–8
inconsistency, 105–7
indicators 104–5, 756
panel data, 751–2, 756
Weibull distribution, 584–6
Weibull-gamma regression model, 615
Weibull regression model, 143–4, 589, 606–8, 635 weighted estimation
endogenous stratification, 828–9 exogenous stratification, 818–20
weighted exogenous sampling ML (WESML) estimator, 828
weighted least squares (WLS) estimator, 81–5 asymptotic distribution, 83
contrasted with GLS, 83
definition, 83
example, 84–5
in pooled model, 702–3, 721 see also FGLS estimator
weighted maximum likelihood (WML) estimator, 828
weighted semiparametric least squares (WSWL) estimator, 327
for binary outcome models, 485 weighting, 817–21, 827–9, 853–6
descriptive versus structural approach, 820 with endogenous stratification, 827–9 sample weights, 817–8
variance estimation, 853–6
weighted prediction, 821 weighted regression, 818–20 whether to weight, 820–1
welfare analysis
with ARUM, 506–7
with nested logit model, 512
WESML estimator. See weighted exogenous sampling ML
White standard errors. See robust standard errors wild bootstrap, 377–8
window width, 299, 307, 312
Wishart distribution, 443
see also inverse-Wishart distribution
within estimator. See fixed effects estimator
within model. See fixed effects model
within-group variation, 709, 733
with-zeros model, 681
WLS estimator. See weighted least squares
WML estimator. See weighted maximum likelihood WNLS estimator, 156–7
asymptotic distribution, 156 definition, 156
example, 159–63
as GLM, 158
working matrix
definition, 82
for GLM estimator, 158
for pooled GEE estimator, 794 for pooled WLS estimator, 721 for WLS estimator, 82–3
WSLS estimator. See weighted semiparametric least squares

zero-inflated count model, - модель счетных данных с раздутым нулём


accelerated failure time (AFT) model, 591–2 
coefficient interpretation, - интерпретация коэффициентов
definition, - определение
leading examples, - основные примеры
accept-reject methods, 413–4, 445 ACD. See average completed duration acronyms, 17
AD estimator. See average derivative adaptive estimator, 323, 328, 684 adding-up constraints, 210
additive model, - аддитивная модель
additive random utility model (ARUM)
binary outcome models, - модели бинарного выбора
generalized random utility models, - обобщенные модели случайной полезности
identification, - идентифицируемость
multinomial outcome models, - мультиномиальные модели
nested logit model, - вложенная логит-модель
RPL model, 513
welfare analysis in, - в анализе благосотояния
admissible estimator, - допустимая оценка
AFT. See accelerated failure time aggregated data
binary outcomes, - бинарные исходы
cohort-level, 772
nonlinear models, - нелинейные модели
multinomial outcomes, - мультиномиальные исходы
513 time-aggregated durations, 578, 600–3 see also discrete-time duration data
AIC. See Akaike information criterion - AIC. См. Акаике информационный критерий
AID. See average interrupted duration
Akaike information criterion (AIC), - Акаике информационный критерий
almost sure convergence, - сходимость почти наверное 
analog estimator, 135
analogy principle, - принцип аналогии
and method of moments estimators, 167 analysis of covariance, 733
analysis of variance, 733
Anscombe residual, - Анскомба остатки
antithetic sampling, 408–9, 445 applications with data
competing risks models, 658–62
duration models, - модели времени жизни
IV estimation, - оценивание с помощью инструментальных переменных
kernel regression, - ядерная регрессия
logit and probit models, - логит и пробит модели
multinomial and nested logit models, - мультиномиальные и вложенные логит модели
Poisson and negative binomial models, 671–4, 690 panel fixed and random effects estimation, 708–15 panel GMM linear estimation, 754–6
panel nonlinear estimation, - нелинейное оценивание в панельных данных
quantile regression, - квантильная регрессия
selection and two-part models, 553–6, 565 survival function, 574–5, 582
treatment evaluation estimation, - оценивание эффекта воздействия
see also data sets used in applications Archimedean family, 654 
Arellano-Bond estimator, - Ареллано-Бонда оценка
application, - применения
nonlinear models, - нелинейные корни 
unit roots, - единичные корни
ARMA. See autoregressive moving average artificial nesting, 283
ARUM. See additive random utility model asymptotic distribution, 953–4
1006
asymptotic efficiency, - асимптотическая эффективность
asymptotic normal distribution, - асимптотически нормальное распределение
definition, - определение
estimated asymptotic variance, - оценка асимптотической дисперсии
of extremum estimators, - экстремальных оценок
of FGLS estimator, - оценки доступного обобщенного МНК
of FGNLS estimator, - оценки доступного обобщенного нелинейного МНК
of first-differences estimator, - оценки в первых разностях
of fixed effects estimator, - оценки в модели с фиксированными эффектами
of GMM estimator, - GMM оценки, оценки обобщенного метода моментов
of Hausman test statistic, - статистики Хаусмана
of kernel density estimator, - ядерной оценки функции плотности
of kernel regression estimator, - ядерной оценки линии регрессии 
of LM test statistic, - LM статистики, статистики множителей Лагранжа
of LR test statistic, - LR статистики, статистики отношения правдоподобия
of m-estimators, - М-оценок
of MD estimator, 292
of ML estimator, - оценки метода максимального правдоподобия
of MM estimator, - оценки метода моментов
of MSL estimator, - оценки симуляционного максимального правдподобия
of MSM estimator, - оценки симуляционного метода моментов
of m-test statistics, 260, 263 
of NLS estimator, - оценки нелинейного МНК
of NL2SLS estimator, 195–6
of OIR test statistic, 181, 183
of OLS estimator, - МНК оценки 
of panel GMM estimator, - панельной оценки обобщенного метода моментов
of quasi-ML estimator, - оценки квази-максимального правдоподобия
of random effects estimator, - оценки модели со случайными эффектами 
of Wald test statistic, - статистики Вальда
see also asymptotic theory - см. также асимптотическая теория
asymptotic efficiency, - асимптотическая эффективность
of optimal GMM, - оптимального обобщенного метода моментов
asymptotic refinement, - асимптотическое уточнение
by bootstrap, - с помощью бутстрэпа 
definition, - определение
by Edgeworth expansion, - с помощью разложения Эджуорта
by nested bootstrap, - с помощью вложенного бутстрэпа
asymptotic theory definitions, - асимптотические определения
asymptotic distribution, - асимптотическое распределение 
asymptotic variance, - асимптотическая дисперсия
central limit theorems, - центральные предельные теоремы 
consistency, - состоятельность
convergence in distribution, - сходимость по распределениею 
convergence in probability - сходимость по вероятности
laws of large numbers, - законы больших чисел 
limit distribution - предельное распределение
limit variance - предельная дисперсия
stochastic order of magnitude, - стохастический порядок малости 
summary of definitions and theorems, - список определений и теорем
asymptotic variance, - асимптотическая дисперсия 
estimated asymptotic variance,  - оценка асимптотической дисперсии
see also asymptotic distribution - см. также асимптотическое распределение
asymptotically pivotal statistic, 359–60, 363–4, 366, 372, 374, 379–80
ATE. See average treatment effect
ATET. See average treatment effect on the treated attenuation bias, 903–5, 911, 915, 919–20 attrition bias, - смещение истощения выборки
augmented regression model, 429
autocorrelation - автокорреляция
in panel model errors, - в ошибках панельных моделей 
dynamic panel models, - динамические панельные модели
see also panel-robust inference - см. также статистические выводы робастные к панельным данным
autoregressive moving average (ARMA) errors - ARMA ошибки
definition, - определение
NLS estimator, - оценка методом нелинейных наименьших квадратов
panel data, - в панельных данных
auxiliary model, - вспомогательная модель 
auxiliary regression - вспомогательная регрессия 
bootstrapping, - бутстрэп 
example, - пример 
Hausman test, - тест Хаусмана
LM test, - LM тест 
m-test, - М-тест
available case analysis. See pairwise deletion average completed duration (ACD), 626 average derivative (AD) estimator
definition, - определение
uses, - использование
average interrupted duration (AID), 626 average selection bias, 868
average squared error, - среднеквадратичная ошибка
average treatment effect (ATE), 33–4, 866–71
definition, - определение
difficulties estimating, - трудности при оценивании
local ATE, 883–6
matching estimators, 871–8 
potential outcome model, - модель потенциального результата
selection on observables only, 868–9 selection on unobservables, 868–71 
see also ATET; LATE; MTE
average treatment effect on the treated (ATET), 866–78
application, - приложения
definition, - определение
difficulties estimating, - трудности при оценивании
matching estimators, - оценки методом сопоставления 
selection on observables only, 868–9 selection on unobservables, 868–71 
see also ATE; LATE; MTE
averaged data. See aggregated data
backward recurrence time, 626
balanced bootstrap, - сбалансированный бутстрэп
balanced repeated replication, 855
balancing condition, 864, 893–4
bandwidth, - ширина окна
bandwidth choice for kernel density estimator, - выбор ширины окна для ядерных оценок функции плотности
cross validation, - кросс-валидация
example, - пример
optimal, - оптимальный
Silverman’s plug-in estimate, - оценка Сильвермана
bandwidth choice for kernel regression estimator, - выбор ширины окна для ядерных оценок линии регрессии
cross validation, - кросс-валидация 
example, - пример 
optimal, - оптимальный 
plug-in estimate, 314
baseline hazard, 591
in AFT model, 592
identification in mixture models, 618–20 in multiple spells models, 655–6
in PH model, 591, 596–7, 601–2
Bayes factors, - Байесовские факторы
Bayes rule. See Bayes theorem - Байеса правило, см. Байеса теорема
Bayes theorem, - Байеса теорема 
example, - пример
Bayesian central limit theorem, - Байесовская центральная предельная теорема
Bayesian information criterion (BIC), - Байесовский информационный критерий
see also AIC - см. также AIC
Bayesian methods, - Байесовские методы
Bayes 1764 example, - пример Байеса 1764 года
Bayesian approach, - Байесовский подход
binary outcome models, - модели бинарного выбора
compared to non-Bayesian, - сравнение с не-Байесовским подходом
count models, - счетные модели
data augmentation, - пополнение данных
decision analysis, 434–5
examples, - примеры
hierarchical linear model, - иерархическая линейная модель
importance sampling, - сэмплирование по важности
linear regression, - линейная регрессия
Markov chain Monte Carlo simulation, - метод Монте-Карло по схеме марковской цепи
measurement error model, - модель ошибки измерения
mixed linear model, 775
model selection, - выбор модели
multinomial outcome models, - мультиномиальная модель
panel data, - панельные данные
posterior distribution, - апостериорное распределение
prior distribution, - априорное распределение
Tobit model, - тобит модель
BCA method. See bias-corrected and accelerated before-after comparison
application, - приложения
Berkson error model, 920
Berkson’s minimum chi-square estimator, - Берксона метод минимизации хи-квадрат
Berndt, Hall, Hall, and Hausman (BHHH) estimate, - Берндта, Холла, Холла и Хаусмана оценка (BHHH)
Berndt, Hall, Hall, and Hausman (BHHH) iterative method, - итерационный алгоритм Берндта, Берндта, Холла и Хаусмана, BHHH итерационный алгоритм
Bernoulli distribution, - распределение Бернулли 
Bernstein-von Mises Theorem, - Бернштейна-фон Мизеса теорема
best linear unbiased predictor, - наилучшая линейная несмещенная оценка
between estimator, - оценка between
application, - приложения
between-group variation, - межгрупповая дисперсия
between model, - модель between
BFGS algorithm. See Boyden, Fletcher, Goldfarb, and Shannon - BFGS алгоритм, см. Бойден, Флетчер, Голдфарб и Шеннон 
BHHH estimate. See Berndt, Hall, Hall, and Hausman BHHH method. See Berndt, Hall, Hall, and Hausman 
bias-corrected and accelerated (BCA) bootstrap
method, 360
biased sampling, - выборка со смещением
see also sample selection; endogenous stratification BIC. See Bayesian information criterion
binary endogenous variable, - бинарные эндогенные переменные
binary outcome models, - модели бинарного выбора
additive random utility model, - аддитивная модель случайной полезности
aggregated data, - агрегированные данные
alternative-invariant regressors, 478
alternative-varying regressors, 478 choice-based samples, 478–9 
corrected score estimator, 916–8 
definition, - определение
example, - пример
identification, - идентифицируемость
index function model, - модель индексных функций
marginal effects, - предельные эффекты
measurement error in dependent variable, - ошибка измерения в зависимой переменной
measurement error in regressors, - ошибка измерения в регрессорах
ML estimator, - ML оценка, оценка максимального правдоподобия
model misspecification, - неправильная спецификация модели, мисспецификация
multiple imputation example, 937–8
OLS estimator, - МНК-оценка
panel data, - панельные данные
semiparametric estimation, - полупараметрическое оценивание
see also logit models; probit models - см. также логит-модели: пробит-модели
binding function, 404–5
bivariate counts, 215, 685–7
bivariate negative binomial distribution, - двумерное отрицательное биномиальное распределение 
bivariate ordered probit model, - двумерная упорядоченная пробит модель
bivariate Poisson distribution, - двумерное распределение Пуассона
bivariate Poisson-lognormal mixture, 686 
bivariate probit model, - двумерная пробит модель
bivariate sample selection model, - двумерная модель самоотбора выборки
application, - применения
bounds, 566
conditional mean, - условное среднее
conditional variance, - условная дисперсия
definition, - определение
Heckman two-step estimator, - двухшаговая оценка Хекмана 
identification, - идентифицируемость
marginal effects, - предельные эффекты
ML estimator, - ML оценка, оценка максимального правдоподобия
outcome equation, 547 participation equation, 547 semiparametric estimator, 565–6 versus two-part model, 546, 552–3
Bonferroni test, 230 bootstrap hypothesis tests
asymptotic refinement, - асимптотическое уточнение
bootstrap critical value, - бутстрэп критическое значение
bootstrap p-value, - бутстрэп P-значение
example, - пример
nonsymmetrical test, - несимметрические тест
power, - мощность
symmetrical test, - симметричный тест
without asymptotic refinement, - без асимптотического уточнения
bootstrap methods, - бутстрэп методы
asymptotic refinement, - асимптотическое уточнение
bias estimate, - оценка смещения
bias-corrected estimator, - оценка скорректированная на смещение 
clustered data, - кластеризованные данные
confidence intervals, - доверительные интервалы
consistency, - состоятельность
critical value, - критическое значение
examples, - примеры
for functions of parameters, - для функций от параметров
general algorithm, - общий алгоритм
for GMM, - для обобщенного метода моментов
heteroskedastic data, - гетероскедастичные данные
introduction, - введение
for nonsmooth estimators, 373, 380–1 number of bootstrap samples, 361–2
panel data, - панельные данные
p-value, - P-значение
recentering, 374, 379
rescaling, 374
sampling methods for, 360
smoothness requirements, 370
standard error estimate, - оценка стандартной ошибки
time series data, - временные ряды
variance estimate, - оценка дисперсии
without asymptotic refinement, - без асимптотического уточнения
see also bootstrap hypothesis tests - см. также тестирование гипотез с помощью бутстрэп
bounds identification, 29
in measurement error models, - в моделях ошибки измерения
bounds in selection model, 566
Boyden, Fletcher, Goldfarb, and Shannon (BFGS) algorithm, - BFGS-алгоритм, алгоритм Бойдена-Флетчера-Голдфарба-Шеннона
CAIC. See consistent Akaike information criterion calibrated bootstrap, 374
caliper matching, 874, 895
canonical link function, 149, 469, 783 case-control analysis, 479, 823
causality, - причинность
examples, - примеры
Granger causality, - причинность по Грейнджеру
identification frameworks and strategies, - стратегии идентификации
in linear regression model, - в линейных регрессионных моделях
in potential outcome models, - в моделях потенциального результата
in simultaneous equations model, - в моделях одновременных уравнений
in single-equation model, - в моделях с одним уравнением
and weighting, 820–1
see also endogeneity - см. также эндогенность
cdf. See cumulative distribution function - функция распределение
censored least absolute deviations (CLAD) estimator,
564–5, 808
censored models, - цензурированные модели
conditional mean, - условное среднее
count models, - счетные модели
definitions, - определения
examples, - примеры
ML estimator, - ML оценка, оценка максимального правдоподобия
semiparametric estimation, - полупараметрическое оценивание
see also duration model; selection models; Tobit models; truncated models - см. также модели времени жизни, модели самоотбора, тобит-модели, модели с усечением
censored normal regression model. See Tobit model censoring mechanisms, - цензурированная модель регрессии с нормальными ошибками. См. также механизм цензурирования в тобит модели
censoring from above, - цензурирование сверху
censoring from below, - цензурирование снизу
left censoring, - цензурирование слева
independent censoring, - независимое цензурирование
interval censoring, - интервальное цензурирование
noninformative censoring, - неинформативное цензурирование
random censoring, - случайное цензурирование 
right censoring, - цензурирование справа
sample selection, - самоотбор выборки
type 1 censoring, - цензурирование 1-го типа
type 2 censoring, - цензурирование 2-го типа 
census coefficient, 819
central limit theorem (CLT), - центральная предельная теорема
Cramer linear transformation, - линейное преобразование Крамера
Cramer-Wold device, - Крамера-Вольда теорема
definition, - определение
examples of use, - примеры использования 
Liapounov CLT, - ЦПТ Ляпунова 
Lindeberg-Levy - ЦПТ Линдеберга-Леви 
multivariate, - многомерная ЦПТ
sample average, - выборочное среднее
sampling scheme, 131,
CGF tests. See chi-square goodness-of-fit characteristic function, 370, 913, 950
chatter, 394, 410
Chebychev’s inequality, - неравенство Чебышева
chi-square goodness-of-fit (CGF) tests, 266–7, 270–1,
474
choice-based samples, 823
binary outcome models, - модель бинарного исхода
see also endogenous stratification - см. также эндогенная стратификация
Choleski decomposition, - Холецкого разложение
CL model. See conditional logit - условная логит-модель
CLAD estimator. See censored least absolute
deviations
Clayton copula, - Клейтона копула
CLT. See central limit theorem clustered data, 829–53
application, 848–53
cluster bootstrap, 363, 377–8, 845 cluster-robust inference, 707, 834, 842,
845
cluster sampling, 41–2
cluster-specific effects, 830–2, 837–45 comparison to panel data, 831–2 diagnostic tests, 841
dummy variables model, 840
fixed effects estimator, 840–1, 843–5 hierarchical models, 845–8
large clusters, 832
nonlinear models, 841–5
OLS estimator, - МНК-оценка
quasi-ML estimator, - оценка квази-максимального правдоподобия
random effects estimator, - оценка модели со случайными эффектами
small clusters, 832
see also panel data - см. также панельные данные
cluster-robust standard errors bootstrap, 363, 377–8, 845 clustered data, 834, 842
panel data, 706–7, 745–6, 789 see also robust standard errors
cluster-specific fixed effects (CSFE) estimator, 839–41, 843–4
application, 848–53 between estimator, 840–1 nonlinear models, 843–4 within estimator, 140–1
cluster-specific fixed effects (CSFE) model, 831, 843 cluster-specific random effects (CSRE) estimator,
837–9, 843–4 application, 848–53
cluster-specific random effects (CSRE) model, 831, 843–4
cluster variable, 707
CM tests. See conditional moment coefficient interpretation
in binary outcome models, 467, 473 
in competing risks model, 646
in count model, 669
in duration models, 606–7
in misspecified linear model, - в неправильно специфицированных линейных моделях
in multinomial outcome models, - в мультиномиальных моделях
in nonlinear models, - в нелинейных моделях
in Tobit model, - в тобит модели
see also marginal effects
coherency condition, 562
cohort-level data. See pseudo panels cointegration, 382, 767
common parameters, 801
compensating variation, 500–7, 512 competing risks model (CRM), 642–8, 658–62
application, 658–62 censoring, 642
coefficient interpretation, 646 definitions, 642–4
dependent risks, 647–8
exit route, 643 identification, 646 independent risks, 644–6 
ML estimator, - ML оценка, оценка максимального правдоподобия
proportional hazards, 645–6 spell duration, 643
with unobserved heterogeneity, 647, 659 complementary log-log model, 466–7, 603 complete case analysis. See listwise deletion complex surveys, 41–2, 814–6, 853–6 composition methods, 415
computational difficulties, 350–2
concentration parameter, 109
conditional analysis, 717
conditional expectations, 955–6
conditional independence assumption, 23, 863, 865
definition, - определение
for participation, 863
given propensity score, 865 selection on observables only, 868 unconfoundedness, 863
conditional likelihood, 139–40, 824
panel models, 731–2, 782–3, 796–9, 805
conditional logit (CL) model, 500–3, 524–5 application, 491–4
definition, 500
fixed effects binary logit, - логит-модель с фиксированными эффектами 
marginal effects, - предельные эффекты
ML estimator, - оценка максимального правдоподобия
from ARUM, 505
see also multinomial outcome models
conditional ML estimator, - условная ML оценка, оценка условного максимального правдоподобия
conditional moment (CM) tests, 264–5, 267–9, 319 consistent CM test, 268
in duration models, 632
example, 269–71
in Tobit model, 544
see also m-tests conditional mean
squared error loss, 67–9 conditional mode
step loss, 68 condition number, 350 conditional quantile
asymmetric absolute loss, 68
confidence intervals, 231–2, 316, 364–5, 368 consistent Akaike information criterion (CAIC), 278 consistent test statistic, 248
consistency definition, - определение состоятельности
of extremum estimators, - экстремальных оценок
of GMM estimator, - GMM оценки, оценки обобщенным методом моментов
of m-estimator, - М-оценки
of ML estimator, - оценки максимального правдоподобия
of NLS estimator, - оценки нелинейного МНК
of OLS estimator, - МНК-оценки
strong consistency, - сильная состоятельность
weak consistency, - слабая состоятельность
see also asymptotic distribution; identification;
pseudo-true value
constant coefficients model. See pooled model contagion, 612
contamination bias, 903–4
contemporaneous exogeneity assumption, 748–9, 752,
781
continuous mapping theorem, - теорема о непрерывном отображении 
control function approach, 37
control function estimator, 869–70, 890 control group, 49
conventions, 16–17
convergence criteria, 339–40, 458 convergence in distribution, 948–9
continuous mapping theorem, 949 definition, 948
limit distribution, 948 transformation theorem, 949 vector random variables, 949
see also central limit theorem convergence in probability, 944–7
1010
alternative modes of convergence, 945
consistency, 945
definition, 945
probability limit, 945
Slutsky’s theorem, - Слуцкого теорема 
uniform convergence, - равномерная сходимость 
vector random variables, - векторная случайная величина 
see also law of large numbers - см. также закон больших чисел
copulas, - копулы
count example, 687 
definition, - опреление
dependence parameter, 653–4 leading examples, 654
ML estimator, - ML оценка, оценка максимального правдоподобия
survival copulas, 652
correlated random effects model, 719, 786 counterfactual, 32, 555, 861, 871
see also potential outcome model - см. также модель потенциального результата
count data, 665
examples, 665 heteroskedasticity, 665 right-skewness, 665 see also count models
count models, 665–93
censored, 680
application, - приложение
endogenous regressors, - эндогенные регрессоры 
endogenous sampling, 823
finite mixture models, - модели смеси распределений
hurdle models, модели преодоления порогов
measurement error in dependent variable, - ошибка измерения зависимой переменной 
measurement error in regressors, - ошибка измерения регрессоров 
mixture models, 675–7
multivariate, - многомерные
OLS estimator, - МНК-оценка
negative binomial model, - отрицательная биномиальная модель
NLS estimator, - оценка нелинейного МНК
panel data, - панельные данные
Poisson model, - модель Пуассона
sample selection, 680
semiparametric regression, - полупараметрическая регрессия
truncated, 679–80
zero-inflated, - с раздутым нулем
covariance matrix. See variance matrix covariance structures, - ковариационная матрица, см. структуры ковариационной матрицы 
covariates. See regressors
Cox CRM model. See competing risks
Cox PH model. See proportional hazards Cox-Snell residual, 289, 631, 633–6 CPS. See Current Population Survey Cramer linear transformation, 952 Cramer-Rao lower bound, 143, 954
see also semiparametric efficiency bound Cramer’s theorem, 949
Cramer-Wold device, 130, 951
CRM. See competing risks model cross-equation parameter restrictions, 210 cross-section data, 47
cross-validation, 304, 314–6, 318, 321
CSFE estimator. See cluster-specific fixed effects CSRE. See cluster-specific random effects cumulant, 370
cumulative distribution function (cdf ), 576 cumulative hazard function
definition, - определение
in competing risks model, 644–5
as diagnostic tool, 631–2
in likelihood function, - в функции правдоподобия
Nelson-Aalen estimator, - Нельсона-Аалена оценка 
in proportional hazards model, 590
Current Population Survey (CPS), 58, 814–5 curse of dimensionality
in Bayesian methods, 419–20
multivariate kernel density estimator, 306 multivariate kernel regression estimator, 319 
high-dimensional integrals, - многомерные интегралы
data augmentation, 454–5, 932 imputation step, 455, 932 for missing data, 932–8 prediction step, 455, 933 regression example, 933
data-generating process (dgp), 72–3, 124 misspecified, 90, 132
data mining, 285–6
data sets. See microdata - наборы данных, см. микроданные
data sets used in applications - наборы данных, используемые в приложениях
Current Population Survey Displaced Workers Supplement (McCall), 603–8, 632–6, 658–62
fishing-mode choice data (Kling and Herriges), 463–6, 486, 491–5
National Longitudinal Survey (Kling), 110–2 National Supported Work demonstration project
(Dehejia and Wahba), 889–95
Panel Survey of Income Dynamics cross-section
sample, 295–7, 300
Panel Survey of Income Dynamics panel sample
(Ziliak), 708–15, 754–6
patents-R&D panel data (Hausman, Hall, and
Griliches), 792–5
Rand Health Insurance Experiment expenditures,
553–6, 565
Rand Health Insurance Experiment medical doctor
contacts, 671–4, 692
strike duration data (Kennan), 574–5, 582 Vietnam World Bank Livings Standards Survey,
88–90, 848–53
see also applications with data
data structures, 39–62
data sources, 58–9 handling microdata, 59–61 natural experiments, 54–8 observational data, 40–8 social experiments, 48–54
data summary approach to regression, 820
Davidon, Fletcher, and Powell (DFP) algorithm - DFP алгоритм, алгоритм Дэвидона-Флетчера-Паулла
decomposition of variance, - разложение дисперсии
degenerate distribution, 948
degrees-of-freedom adjustment, 75, 102, 138, 185–6,
278, 841
delta method, - дельта-метод
bootstrap alternative, 363
density kernel, 421
density-weighted average derivative (DWAD)
estimator, - оценка
dependent variable, - зависимая переменная
descriptive approach to regression, 820
deviance, 149, 244
deviance residual, 289, 291
DFP algorithm. See Davidon, Fletcher, and Powell
algorithm - DFP алгоритм, см. алгоритм Дэвидона-Флетчера-Паулла
dgp. See data-generating process - процесс порождающий данные
diagnostic tests. See specification tests - диагностические тесты, см. тесты на спецификацию
DID estimator. See differences-in-differences differences-in-differences (DID) estimator, 55–7,
768–70, 878–9 application, 890–1 consistency, 770 definition, 768 introduction, 55–7 natural experiments, 878 with controls, 878–9 without controls, 878
direct regression, 906 disaggregated data
contrasted with aggregated data, 5–10 discrete factor models, 678
see also finite mixture models
discrete outcomes. See binary outcomes; counts;
multinomial outcomes
discrete-time duration data, 577–8, 600–3
cumulative hazard function, 578 discrete-time proportional hazards, 600–3 gamma heterogeneity, 620
hazard function, 578
logit model, - логит-модель
ML estimator, - оценка максимального правдподобия
nonparametric estimation, - непараметрическая оценка
probit model, - пробит модель
survivor function, 578
dissimilarity parameter, 509
disturbance term. See error term - случайная составляющая, см. ошибка
double bootstrap, 374
dummy endogenous variable model, 557 dummy variable estimator, 784–5, 800, 805, 840
see also LSDV estimator duration data, 573–664
different types, 626, 641 duration models, 573–664
accelerated failure time, 591–2
applications, - приложения
censoring, 579–82, 587–9, 595, 642 competing risks, 642–8, 658–62
cumulative hazard function, 577–8 discrete time, 577–8, 600–3 generalized residual, 631
hazard function, 576, 578
key concepts, - ключевые понятия
mixture models, 613–25
ML estimator, - оценка максимального правдоподобия
multiple spells, 655–8 multivariate, 648–55 nonparametric estimators, 580–4 OLS estimator, 590–1
panel data, - панельные данные
parametric models, - параметрические модели
proportional hazards, 592–7
risk set, 581, 594
semiparametric estimation, - полупараметрическое оценивание 
specification tests, - тесты на спецификацию
survivor function, 576, 578
time-varying regressors, - регрессоры, меняющиеся во времени
see also proportional hazards model
DWAD estimator. See density-weighted average derivative
dynamic panel models, 763–8, 791–2, 797–9, 806–7
Arellano-Bond estimator, - Ареллано-Бонда оценка
binary outcome models, 806–7
count models, - счетные модели
covariance structures, - структуры ковариационной матрицы
inconsistency of standard estimators, - несостоятельность обычных оценок 
initial conditions, - начальные условия
IV estimators, - оценки методом инструментальных переменных
linear models, - линейные модели
MD estimator, 767
nonlinear models, - нелинейные модели 
nonstationary data, - нестационарные данные
transformed ML estimator, - преобразованная ML оценка, преобразованная оценка максимального правдоподобия
true state dependence, 763–4 
unobserved heterogeneity, 764 
weak exogeneity, - слабая экзогенность
EDF bootstrap. See empirical distribution function bootstrap
Edgeworth expansions, - разложение Эджуорта
efficient score, 141
Eicker-White robust standard errors, - Эйкера-Уайта робастные стандартные ошибки 
see also heteroskedasticity robust-standard errors - см. также стандартные ошибки, устойчивые к гетероскедастичности
EM algorithm see expectation maximization empirical Bayes method, 442
empirical distribution function (EDF) bootstrap, 360
see also paired bootstrap
empirical likelihood, - эмпирический метод максимального правдоподобия
empirical likelihood bootstrap, 379–80 encompassing principle, 283 endogeneity
definition, - определение
due to endogenous stratification, 78, 824–5 
Hausman test for,  - тест Хаусмана на эндогенность
identification frameworks and strategies, 35–7
see also endogenous regressors; exogeneity endogenous regressors, 78
binary, 557, 562
in count models, 683–4, 687–9
in discrete outcome models, 473
in duration models, 598
dummy, 557, 562
inconsistency of OLS, - несостоятельность МНК
in linear panel models, 744–63
in linear simultaneous equations model, 23–30 in nonlinear panel models, 792
in potential outcome model, - в моделях потенциального результата
returns-to-schooling example, 69–70
in selection models, 559–62
in single-equation models, 30
see also GMM estimator; IV estimator
endogenous sampling, 42–5, 78, 822–9, 856 consistent estimation, 827–9
leading examples, 823
see also censored models; endogenous
stratification; sample selection models endogenous stratification, 820, 826–7, 856 equation-by-equation OLS, 210 equicorrelated errors, 701, 722–4, 804 equidispersion, 668, 670
error components model. See RE model error components SEM, 762
error components SUR model, 762 error components 2SLS estimator, 760 error components 3SLS estimator, 762 
error term, - ошибка, случайная составляющая
additive, - аддитивная
nonadditive, - неаддитивная
errors-in-variables. See measurement error estimated asymptotic variance, 954
see also asymptotic distribution
estimated prediction error. See cross-validation estimating equations estimator, 13–5
asymptotic distribution, 134–5, 174 clustered data, 842
computation, 339
definition, - определение
generalized, 134, 790, 794, 804 variance matrix estimation, 137–9 weighted, 829
see also MM estimator
Euler conditions, - Эйлера условия
exact identification. See just identification exchangeable errors, 701, 804
exhaustive sampling, 815–6
exogeneity, 22–3
conditional independence, 23
Granger causality, - причинность по Грейнджеру
of instrument, 106
overidentifying restrictions test for, 277 panel data assumptions, 700, 748–52, 754,
strong exogeneity, - сильная экзогенность
weak exogeneity, - слабая экзогенность
exogenous sampling, 42–3
exogenous stratified sampling, 42, 78, 814–5, 820,
825, 856
exogenous regressor. See exogeneity - экзогенный регрессор, см. экзогенность
expectation maximization (EM) algorithm, - EM-алгоритм, алгоритм максимизации ожидания
for data imputation, 930–2
E (Expectation) step, - Е-шаг
for finite mixture model, 623–5 
M (Maximization) step, - М-шаг
compared to NR algorithm, 625
expected elapsed duration, 626 
experimental data, - экспериментальные данные
control group, - контрольная группа
natural experiments, - естественные эксперименты 
social experiments, - социальные эксперименты 
treatment group, 49
explanatory variables. See regressors exponential conditional mean, 124, 155, 669 
coefficient interpretation, 124, 162–3, 669
exponential distribution, 140, 584–6
for generalized (Cox-Snell) residual, 631
exponential family density, - экспоненциальное семейство распределений 
conjugate prior for, - сопряженное априорное распределение
see also linear exponential family - см. также экспоненциальное семейство распределений
exponential-gamma regression model, - экспоненциальная-гамма регрессионная модель
exponential-IG regression model, 634 
exponential regression model - экспоненциальная регрессионная модель
application with censored data, - применение к цензурированным данным
example with uncensored data, - пример с нецензурированными данными
extreme value distribution. See type 1 extreme value extremum estimator, 124–39
asymptotic distribution, - асимптотическое распределение 
consistency, - состоятельность
definition, - определение
formal proofs, - формальные доказательства
informal approach, - неформальный подход 
statistical inference, 135–9 variance matrix estimation, 137–9
factor analysis, 650
factor loadings, 517, 650–1, 689
factor model, 517, 648, 686 Fairlee-Gumble-Morgenstern copula, 654
fast simulated annealing (FSA) method, 347–8
FD estimator. See first-differences
FE estimator. See fixed effects
feasible generalized least squares (FGLS) estimator, - оценка доступного обобщенного МНК
asymptotic distribution, - асимптотическое распределение 
definition, - определение
example, - пример
in fixed effects model, - в модели с фиксированными эффектами 
in mixed linear model, 775 nonlinear, 155–8
in pooled model, - в сквозной модели
feasible generalized least squares (cont.)
in random effects model, - в модели со случайными эффектами
as sequential two-step m-estimator, 201 systems FGLS, 208–9
feasible generalized nonlinear least squares (FGNLS) estimator, 155–8
asymptotic distribution, - асимптотическое распределение 
definition, - определение
example, - пример
as optimal GMM estimator, 180–1 systems FGNLS, 217
FGLS estimator. See feasible generalized least squares FGNLS estimator. See feasible generalized nonlinear least squares
FIML estimator. See full information maximum likelihood - Оценка максимального правдподобия с полной информацией
finite mixture models, 621–5
counts, 678–9
definition, - определение
EM algorithm, - ЕМ-алгоритм
latent class interpretation, 623 
number of components, 624–5 
panel data, - панельные данные
see also mixture models
finite-sample bias - смещение в малой выборке, смещение в конечной выборке
of GMM estimator, - ОММ оценки, оценки обобщенного метода моментов
of IV estimator, - оценки инструментальных переменных
of tests, - тестовых статистик
finite-sample correction term - корректировка на конечный размер выборки
for sampling without replacement, - для выборки без повторений
first-differences (FD) estimator, - оценка в первых разностях
application, - приложения
asymptotic distribution, - асимптотическое распределение
compared to FE estimator, - сравнение с оценкой модели с фиксированными эффектами
consistency, - состоятельность
definition, - определение
IV estimator, - оценка инструментальных переменных
first-differences (FD) model, - модель в первых разностях
first-differences (FD) transformation, - взятие первой разности
fixed effects (FE) estimator, - FE-оценка, оценка модели с фиксированными эффектами
application, - приложения
asymptotic distribution, - асимптотическое распределение
binary outcome models, - модели бинарного выбора
clustered data, - кластеризованные данные
compared to DID estimator, 768
compared to FD estimator, 729
as conditional ML estimator, - как оценка условного максимального правдоподобия
consistency, - состоятельность
count models, - счетные модели
definition, - определение
duration models, - модели времени жизни
dynamic models, - динамические модели 
as FGLS estimator, 729
Hausman test for, - тест Хаусмана
identification, - идентифицируемость
incidental parameters,
inconsistency, - несостоятельность 
IV estimators, - оценки методом инструментальных переменных
as LSDV estimator, 733 
multinomial outcome models, 798 selection models, 801
Tobit model, - тобит модель
versus random effects, - сравнение со случайными эффектами
fixed effects (FE) model, - модель с фиксированными эффектами
cohort-level, 772
clustered data, - кластеризованные данные
definition, - определение
dynamic models, - динамические модели 
endogenous regressors, - эндогенные регрессоры
identification, - идентифицируемость
incidental parameters, 704, 726
marginal effects, - предельные эффекты
nonlinear models, - нелинейные модели 
time-varying regressors, - регрессоры, меняющиеся во времени
versus random effects, - сравнение со случайными эффектами
see also fixed effects estimators - см. также оценку модели с фиксированными эффектами
fixed coefficient, 846
fixed design. See fixed in repeated samples fixed in repeated samples, 76–7
bootstrap sampling method, 360 
in kernel regression, 312 
Liapounov CLT, - Ляпунова ЦПТ
Markov LLN, - Маркова ЗБЧ
Monte Carlo sampling method, 251
fixed regressors. See fixed in repeated samples flexible parametric models
count models, 674–5 hazard models, 592 selection models, 563
flow sampling, 44, 626
forward orthogonal deviations IV estimator, 759 forward orthogonal deviations model, 759 forward recurrence time, 626
Fourier flexible functional form, 321
frailty, 612, 662
see also unobserved heterogeneity Frank copula, 654
Frechet bounds, 653–4
frequentist approach, 421–2, 424, 439–40 FSA method. See fast simulated annealing full conditional distributions, 431
see also Gibbs sampler - см. также сэмплирование по Гиббсу
full information maximum likelihood (FIML) - метод максимального правдоподобия с полной информацией
estimator, - оценка
nested logit model, 510–2 nonlinear models, 219
functional approach
to measurement error, 901
functional form misspecification, 91–2 diagnostics for, 272–3, 277–8
gamma distribution, 585–6, 614 gamma function, 586
Gaussian quadrature, 389–90, 393, 809 Gauss-Hermite quadrature, 389–90 Gauss-Laguerre quadrature, 389–90 Gauss-Legendre quadrature, 389–90
Gauss-Newton (GN) algorithm, 345 example, 348
GEE estimator. See generalized estimating equations general to specific tests, 285
generalized additive model, 323, 327
generalized cross-validation, 315
generalized estimating equations (GEE) estimator, 790, 794, 804, 809
generalized extreme value (GEV) distribution, 508 see also nested logit model
generalized information matrix equality, 142, 145, 264 generalized inverse, 261
generalized IV estimator, - обобщенная оценка инструментальных переменных
generalized least squares (GLS) estimator, 81–5
asymptotic distribution, - асимптотическое распределение 
definition, - определение
as efficient GMM, 179 
example, - пример
nonlinear, 155–8
generalized linear models (GLMs), 149–50, 155
count data, - счетные данные
conditional ML estimator, - оценка условного максимального правдоподобия 
GEE estimator, 791
quasi-ML estimator, - оценка квази-максимального правдоподобия 
see also LEF models - см. также модели экспоненциального семейства распределений
generalized method of moments (GMM) estimator, - GMM-оценка, оценка обобщенного метода моментов
asymptotic distribution, - асимптотическое распределение
based on additional moment restrictions, 169,
178–9
based on moment conditions from economic theory,
171
based on optimal conditional moment, 179–80 bootstrap for, 379–80
computation, - вычисление
definition, - определение
endogenous counts, 683–4, 687–9
with endogenous stratification, - с эндогенной стратификацией
with exogenous stratification, - экзогенной стратификацией
examples, - примеры
finite-sample bias, - смещение из-за малого размера выборки
identification, - идентификация
linear IV, - линейный метод инструментальных переменных
linear systems, - линейные системы
nonlinear IV, - нелинейный метод инструментальных переменных
one-step GMM estimator, - одношаговая оценка обобщенного метода моментов 
optimal GMM, 176
optimal moment condition, 179–81, 188 
optimal weighting matrix, - оптимальная матрица весов
panel data, - панельные данные
practical considerations, - практические замечания
test based on, 245
two-step, 176, 187, 746, 755
variance matrix estimation, - оценка ковариационной матрицы
weak instruments, - слабые инструменты
see also panel GMM estimator - см. также оценка обобщенного метода моментов для панельных данных
generalized nonlinear least squares (GNLS) estimator.
See feasible generalized nonlinear least squares generalized partially linear model, 323
generalized random utility models, 515–6 generalized residual, 289–90
in duration models, - в моделях времени жизни
in LM test, 239–40 plots of, 633–6
generalized Tobit model, 548
generalized Weibull distribution, 584–6
genetic algorithms, 341
GEV distribution. See generalized extreme value Geweke, Hajivassiliou, Keane (GHK) simulator,
407–8
for MNP model, 518
GHK simulator. See Geweke, Hajivassiliou, Keane simulator
Gibbs sampler, - сэмплирование по Гиббсу
data augmentation, 454–5, 933 example, 452–4
in latent variable models, 514, 519, 563 see also Markov chain Monte Carlo
GLMs. See generalized linear models
GLS estimator. See generalized least squares
GMM estimator. See generalized method of moments GN algorithm. See Gauss-Newton
GNLS estimator. See feasible generalized nonlinear
least squares
Gompertz distribution, 585–6 Gompertz regression model, 606–8 gradient methods, 337–48
see also iterative methods Granger causality, 22
grid search methods, 337, 351 grouped data. See aggregated data
Halton sequences, 409–10 Hausman test, - тест Хаусмана
applications, - приложения
asymptotic distribution, - асимптотическое распределение
auxiliary regressions, - вспомогательная регрессия
bootstrap, - бутстрэп
computation, - вычисление 
definition, - определение
for endogeneity, - на эндогенность
for fixed effects, - для модели с фиксированными эффектами 
for multinomial logit model, - для мультиномиальной логит-модели 
power, - мощность
robust versions, - робастные версии 
Hausman-Taylor IV estimator, 761 
Hausman-Taylor model, - Хаусмана-Тейлора модель 
Hawthorne effect, 53
hazard function
baseline in PH model, 591 cumulative hazard, 577–8, 582–4 definition, 576, 578
hazard function (cont.)
in mixture models, 616–8 multivariate, 649
nonparametric estimator, 581, 583 parametric examples, 585 piecewise constant, 591
see also duration models - см. также модели времени жизни
Health and Retirement Study (HRS), 58
Heckit estimator. See Heckman two-step estimator - хекит оценка, см. Хекмана двухшаговая оценка 
Heckman two-step estimator - двухшаговая оценка Хекмана 
application, - приложения
in Roy model, - в модели Роя
in selection model, 550–1 
semiparametric estimator, - полупараметрическая оценка 
in Tobit model, - в тобит моделях
Hessian matrix - матрица Гессе
estimate, - оценка
Newton-Raphson algorithm, - Ньютона-Рафсона алгоритм 
singular, - вырожденная
heterogeneous treatment effects, 882, 885–7 
IV estimator, - оценка инструментальных переменных 
LATE estimator, - оценка LATE, оценка локального среднего эффекта воздействия
RD design, - разрывный дизайн
heterogeneity - неоднородность
within-cell, 480
see also unobserved heterogeneity - см. также ненаблюдаемая неоднородность
heteroskedastic errors - гетероскедастичные ошибки
adaptive estimation, - адаптивное оценивание
conditional heteroskedasticity, - условная гетероскедастичность 
definition, - определение
in GLMs, - в обобщенных линейных моделях
in linear model, - в линейных моделях 
multiplicative, - мультипликативная
in nonlinear model, - в нелинейных моделях 
residuals, - остатки
tests for, - тесты
Tobit MLE inconsistency, - несостоятельность ML оценок в тобит модели 
working matrix for, - рабочая матрица
heteroskedasticity-robust standard errors - стандартные ошибки устойчивые к гетероскедастичности
bootstrap, - бутстрэп
clustered data, - кластеризованные данные
example, - пример
for extremum estimator, - для экстремальных оценок
intuition, - интуиция
for NLS estimator, - для оценки нелинейным МНК
for OLS estimator, - для МНК оценки 
panel data, - панельные данные
for WLS estimator, - для оценки взвешенного МНК
see also robust standard errors - см. также робастные стандартные ошибки
hierarchical linear models (HLMs), - иерархические линейные модели
Bayesian analysis, - Байесовский анализ
clustered data, - кластеризованные данные
coefficient types, - типы коэффициентов 
individual-specific effects, - индивидуальные эффекты 
mixed linear models, 774–6, 847 
panel data, - панельные данные
random coefficients model, - модель со случайными коэффициентами 
two-level model, - двухуровневая модель 
hierarchical models, - иерархические модели
Bayesian analysis, - Байесовский анализ 
see also hierarchical linear models - см. также иерархические линейные модели
histogram, - гистограмма
see also kernel density estimator - см. также ядерная оценка функции плотности
HLM. See hierarchical linear model - см. иерархические линейные модели
hot deck imputation, - метод карточной колоды (для восстановления пропущенных наблюдений)
HRS. See Health and Retirement Study - 
Huber-White robust standard errors, - робастные стандартные ошибки Хубера-Уайта
see also robust standard errors - см. также робастные стандартные ошибки
hurdle model, - модель преодоления порогов
see also two-part model - см. также двухчастная модель
hyperparameters, - гиперпараметры 
hypothesis tests, - тестирование гипотез
based on extremum estimator, - основанное на экстремальных оценках
based on ML estimator, - основанное на оценках методом максимального правдоподобия
based on GMM estimator, - основанное на оценках обобщенных методом моментов
based on m-estimator, - основанное на М-оценках
bootstrap, - бутстрэп
for common misspecifications, 274–7, 670–1 
examples, - примеры 
induced test, 230
joint versus separate, 230–1, 285, 629–30 
power, - мощность
size, 246–7, 251–3
see also LM tests; LR test; Wald tests, m-tests - см. также LM-тест (тест множителей Лагранжа), LR-тест (тест отношения правдоподобия), Вальда тест, М-тест

identification - идентифицируемость
in additive random utility models, - в аддитивных моделях со случайной полезносью
in binary outcome models, - в моделях бинарного выбора
bounds identification, 29
definitions, - определения
in fixed effects model, - в моделях с фиксированными эффектами
of GMM estimator, - GMM-оценки, оценки обобщенного метода моментов
just identification, - точная идентифицируемость
in linear regression model, - в линейных регрессионных моделях
in measurement error models, - в моделях с ошибкой измерения
in mixture models, - в моделях смеси распределений
in multinomial probit model, - в мультиномиальной пробит модели
in natural experiments, - в естественных экспериментах
observational equivalence, - 
order condition, - условие порядка
over identification, - сверх идентифицированность
rank condition, - условие ранга
in sample selection model, - в моделях самоотбора выборки
set identification, 29
in simultaneous equations model, - в моделях одновременных уравнений 
in single-index models, - в одноиндексных моделях
singular Hessian, - вырожденная матрица Гесси
weak identification, - слабая идентифицируемость
see also identification strategies - см. также стратегии идентификации
identification strategies, - стратегии идентификации 
control function approach, - подход контрольных функций 
exogenization, 36
incidental parameter elimination, 36–7 
instrumental variables, - инструментальные переменные
matching, - сопоставление 
reweighting, 37

identified reduced form, - идентифицируемая приведенная форма
IG distribution. See inverse-Gaussian ignorable missingness, 927
estimator consistency if MCAR, 927 estimator inconsistency if MAR only, 927 problems if nonignorable, 940
weak exogeneity, - слабая экзогенность
ignorability assumption, 863
see also conditional independence assumption - см. также предположение об условной независимости
importance sampling, - сэмплирование по важности 
accelerated, 409
GHK simulator, GHK-симулятор, Гевеке-Хадживасилу-МакФаддена симулятор
importance sampling density, 444 
importance sampling estimator, 444 
importance weight, 445
target density, - целевая функция плотности 
imputation methods, - методы восстановления данных
data augmentation, - пополнение данных 
example, - пример
hot deck imputation, - восстановление пропущенных данных по методу горячей колоды
listwise deletion, - полное удаление наблюдений с пропусками
mean imputation, 928–9
multiple imputation, 934–5 pairwise deletion, 928 regression-based imputation, 930–2
imputation (I) step, - I-шаг, шаг пополнения
IM test. See information matrix test - IM-тест, см. тест информационной матрицы
IMSE. See integrated mean squared error incidental parameters, 36
clustered data FE model, - модель с фиксированными эффектами для кластеризованных данных
panel data FE model, - модель с фиксированными эффектами для панельных данных 
inclusive value, 510–1
incomplete gamma function, - неполная гамма-функция
incomplete panels. See unbalanced panels  - неполные панели. См. несбалансированные панели
independence of irrelevant alternatives, - независимость от посторонних альтернатив 
independent variables. See regressors - независимые переменных. См. регрессоры
independently-weighted IV estimator, 192 independently-weighted optimal GMM estimator, 177 index function model
binary outcome model, - модель бинарного выбора 
bivariate probit model, - двумерная пробит модель  
ordered multinomial model, упорядоченная мультиномиальная модель 
Tobit model, - тобит модель
see also single-index model - см. также одноиндексная модель
indicator function, - функция-индикатор, индикатор
indirect inference, - косвенные статистические выводы
individual-specific effects model - модель индивидуальных эффектов
additive, - аддитивная
binary outcome models, - модели бинарного выбора 
cluster-specific effects, 830 
count models, - счетные модели 
definitions, - определение
duration models, - модели времени жизни
multiplicative, - мультипликативная
one-way, - с односторонними эффектами
parametric, - параметрическиая
selection models, 801 
single-index, 780
Tobit models, - тобит модели
two-way, - с двусторонними эффектами
see also FE models; RE models - см. также модели с фиксированными эффектами, модели со случайными эффектами
induced test, 230
information criteria, - информационный критерий
Akaike, - Акаике
Bayesian, - Байесовский
consistent Akaike, - состоятельный критерий Акаике 
Kullback-Liebler, - Кульбака-Лейблера 
Schwarz, - Шварца
information matrix, - информационная матрица 
block-diagonal, - блочно-диагональная
information matrix equality, - равенство информационных матриц 
generalized, - обобщенное
see also BHHH estimate; OPG version
information matrix (IM) test, - тест информационной матрицы
bootstrap, - бутстрэп
computation, - вычисления
definition, - определение
example, - пример
power, - мощность
instrumental variables (IV) estimator - оценка метода инструментальных переменных
alternative estimators, - альтернативные оценки
application, - приложения
definition, - определение
example, - пример
finite-sample bias, - смещение из-за малого размера выборки 
identification, - идентификация 
independently-weighted IV estimator, 192 
jackknife IV estimator, - джекнайф оценки инструментальных переменных
LIML estimator, - оценка метода максимального правдоподобия с ограниченной информацией
in linear model, - в линейных моделях
linear IV as GMM estimator, - линейная оценка инструментальных переменных как GMM-оценка (оценка обобщенного метода моментов)
local average treatment effects estimator, - оценка локального среднего эффекта воздействия
in measurement error models, - в моделях с ошибкой измерения
in natural experiments, - в естественных экспериментах
in nonlinear models, - в нелинейных моделях
in panel models, - в моделях панельных данных
quantile regression, - квантильная регрессия
in selection models, 559
split-sample estimator, 191–2
systems IV estimator, - оценка инструментальных переменных для систем уравнений
in treatment effects models, 883–9
two-stage IV estimator, - двухшаговая оценка метода инструментальных переменных
two-stage least squares estimator, 101–2, 187–91 
Wald estimator, - оценка Вольда
see also GMM estimator; panel GMM estimator - см. также GMM оценка (оценка обобщенного метода моментов), GMM оценка для панельных данных
instruments - инструменты
definition, - определение
examples, - примеры
by exclusion restriction, 106
by functional form restriction, 106
invalid, 100, 105–7
optimal, 180
for panel data, - для панельных данных
relevance, 108
weak, 100, 104–12, 177–8, 191–2, 196, 751–2, 756 see also instrumental variables estimator
integrated hazard function. See cumulative hazard function
integrated mean squared error (IMSE), 303 integrated squared error (ISE), 302, 314 
interval data models
definition, - определение
ML estimator, - оценка метода максимального правдоподобия
interruption bias, 626
intraclass correlation, 816, 831, 835–8 inverse-Gaussian (IG) distribution, 614–5, 677 inverse law of probability, 421
inverse-Mills ratio, - обратное отношение Миллса
inverse transformation method, - метод обратного преобразования 
inverse-Wishart distribution, - обратное распределение Уишарта 
irrelevant regressors, 93
ISE. See integrated squared error
iterated bootstrap, - итерационный бутстрэп
iterative methods, - итерационные методы
BFGS, - BFGS
BHHH, - BHHH
convergence criteria, - критерий сходимости
DFP, 344, 350–1
expectation maximization, - максимизация ожидания
fast simulated annealing, - быстрый алгоритм имитации отжига
Gauss-Newton, - Гаусса-Ньютона
line search, - поиск на прямой
Newton-Raphson, - Ньютона-Рафсона
numerical derivatives, - численные производные
simulated annealing, - алгоритм имитации отжига
starting values, - стартовые значения
step size adjustment, - корректировка величины шага
IV estimator. See instrumental variables - IV оценка, см. оценка метода инструментальных переменных
jackknife, - джекнайф
bias estimate, - оценка смещения
bias-corrected estimator, 375 
example, - пример
IV estimator, - IV оценка
standard error estimate, - оценка стандартной ошибки
Jensen’s inequality, - неравенство Йенсена
jittered data, 290
joint duration distributions, - совместное распределение времени жизни
copulas, - копулы
mixtures, - смеси
multivariate hazard function, - многомерная функция риска
multivariate survivor function, - многомерная функция выживания
joint limits, 767
joint versus separate tests, 230–1, 285, 629–30 
just identification, - точная идентифицируемость
Kaplan-Meier (KM) estimator, - КМ-оценка, оценка Каплан-Мейера 
application, - приложение
for baseline hazard, 596–7 
confidence bands for, - доверительные интервалы 
definition, - определение
tied data, 582
kernel density estimator, - ядерная оценка функции плотности
alternatives to, - альтернативы
application, - приложения
asymptotic distribution, - асимптотическое распределение
bandwidth choice, - выбор ширины окна
bias, - смещение
confidence interval for, - доверительный интервал
consistency, - состоятельность
convergence rate, - скорость сходимости
definition, - определение
derivative estimator, 305
examples, - примеры
multivariate, 305–6
Nadaraya-Watson kernel regression estimator, 312 optimal bandwidth, 303
optimal kernel, - оптимальное ядро
variance, - дисперсия
kernel functions, - ядерные функции 
comparison, - сравнение
definition, - определение
higher-order, - высокого порядка 
leading examples, - основные примеры
optimal for density estimation, 303 
properties, - свойства
kernel matching, - ядерное сопоставление
kernel regression estimator, - ядерная оценка регрессии
alternatives to, - альтернативы
asymptotic distribution, - асимптотическое распределение
bandwidth choice, - выбор ширины окна
bias, - смещение
bootstrap confidence interval for, - бутстрэп доверительные интервалы
boundary problems, 309, 320–1
conditional moment estimator, 317–8
confidence interval for, - доверительные интервалы
consistency, - состоятельность
convergence rate, - скорость сходимости
definition, - определение
derivative estimator, 317
introduction to nonparametric regression, - введение в непараметрическую регрессию
multivariate, - многомерная 
optimal bandwidth, - оптимальная ширина окна
optimal kernel, - оптимальное ядро
undersmoothing, - недосглаживание
variance, - диспрерсия
see also nonparametric regression - см. также непараметрическая регрессия
Khinchine’s theorem, - Хинчина теорема
KLIC. See Kullback-Liebler information criterion - KLIC. См. Кульбака-Лейблера информационный критерий
KM estimator. See Kaplan-Meier - КМ-оценка. См. Каплан-Мейера оценка
k-NN estimator. See nearest neighbors estimator 
Kolmogorov LLN, - Колмогорова ЗБЧ, Колмогорова закон больших чисел
Kolmogorov test, - Колмогора тест
Kullback-Liebler information criterion (KLIC), - KLIC, Кульбака-Лейблера информационный критерий
LAD estimator. See least absolute deviations - оценка минимума абсолютных отклонений
Lagrange multiplier (LM) test - тест множителей Лагранжа, LM тест
asymptotic distribution, - асимптотическое распределение 
based on GMM-estimator, - основанный на оценке обобщенного метода моментов
based on m-estimator, - основанный на М-оценке
bootstrap, - бутстрэп
comparison with LR and Wald tests, - сравнение с LR тестом (тестом отношения правдоподобия) и тестом Вальда 
computation, - вычисление
definition, - определение
examples, - примеры
for heteroskedasticity, - на гетероскедастичность 
in duration models, - в моделях времени жизни 
interpretation, - интерпретация
for omitted variables, - на пропущенные переменные
OPG version, 240–1
for random effects, - на случайные эффекты
score test, 234–5
in Tobit model, - в тобит модели
for unobserved heterogeneity, - на ненаблюдаемую неоднородность 
see also hypothesis tests - см. также тестирование гипотез
Laplace approximation, - Лапласа приближение
Laplace distribution, - Лапласа распределение
Laplace transform, - Лапласа преобразование
LATE estimator. See local average treatment effects - LATE оценка. См. локальный средний эффект воздействия
latent class model, - модель латентных классов, модель скрытых классов
see finite mixture models - см. модели конечной смеси распределений
latent variable, - латентная переменная, скрытая переменная
latent variable models - модели с латентными переменными
additive random utility model, - аддитивная модель случайной полезности 
binary outcomes, - бинарные исходы
endogenous, - экзогенные
ordered multinomial model, - упорядоченная мультиномиальная модель 
see also censored models; truncated models
law of iterated expectations, 955
law of large numbers (LLN), - закон больших чисел (ЗБЧ)
definition, - определение
examples of use, - примеры использования 
Khinchine’s theorem, - Хинчина теорема
Kolmogorov LLN, - Колмогорова ЗБЧ 
Markov LLN, - Маркова ЗБЧ
sampling schemes, 131, 948 
strong law, - сильный закон больших чисел
weak law, - слабый закон больших чисел
least absolute deviations (LAD) estimator - оценка наименьших абсолютных значений
application, - применения
asymptotic distribution, - асимптотическое распределение
binary outcome models, - модели бинарного выбора
bootstrap, - бутстрэп
censored LAD, - цензурированная оценка наименьших абсолютных значений
definition, - определение
two-stage LAD, - двухшаговая оценка наименьших абсолютных значений
see also quantile regression - см. также квантильная регрессия
least-squares dummy variable (LSDV) estimator, 704, 732–3, 840
least-squares dummy variable (LSDV) model, 704, 732, 840
least squares (LS) estimators - МНК оценки, оценки метода наименьших квадратов
clustered data, - кластеризованные данные
feasible generalized LS, - доступный обобщенный МНК 
generalized LS, - обобщенный МНК
linear, - линейный МНК
nonlinear LS, - нелинейный МНК
ordinary LS, - МНК
panel data, - панельные данные
systems of equations, - системы одновременных уравнений
see also FGLS; FGNLS; OLS; NLS leave-one-out estimate, 192, 304, 315, 375 LEF. See linear exponential family length-biased sampling, 43–4, 626 
Liapounov CLT, - Ляпунова ЦПТ 
likelihood-based hypothesis tests, - тестирование гипотез с помощью функции правдоподобия
comparisons of, 235–6, 238–9 definitions, 234–5
examples, - примеры
see also LM tests; LR tests; Wald tests - см. LM тест (тест множителей Лагранжа), LR тест (тест отношения правдоподобия), Вальда тест
likelihood function, - функция правдоподобия
conditional likelihood function, - условная функция правдоподобия 
definition, - определение
joint, - совместная
leading examples, - основные примеры
marginal, 43
partial, 594–6
likelihood principle, - принцип максимального правдоподобия
likelihood ratio (LR) test - LR тест (тест отношения правдоподобия)
asymptotic distribution, - асимптотическое распределение
based on GMM-estimator, - основанные на оценке обобщенного метода моментов
based on m-estimator, - основанный на М-оценке
comparison with LM and Wald tests, - сравнение с LM тестом (тестом множителей Лагранжа) и тестом Вальда
definition, - определение
examples, - примеры
nonnested models, - невложенные модели
quasi-LR test statistic, - статистика теста квази-отношения правдоподобия
uniformly most powerful test, - равномерно наиболее мощный тест
237 see also hypothesis tests - см. также тесты 
LIML estimator. See limited information maximum likelihood
limit distribution, - предельное распределение
see also asymptotic distribution - см. также асимптотическое распределение
limit variance matrix, - предельная ковариационная матрица
definition, - определение
replacement by consistent estimate, 952 
sandwich form, - сэндвич-форма
limited information maximum likelihood (LIML) estimator, - оценка максимального правдоподобия с ограниченной информацией
Lindeberg-Levy CLT, - Линдеберга-Леви ЦПТ
line search, 338
linear exponential family (LEF) models, 147–9
conjugate priors, - сопряженное априорное распределение
conditional ML estimator, - оценка условного максимального правдоподобия
consistency, - состоятельность
leading examples, - основные примеры
pseudo-R2, - псевдо R2
residuals, - остатки
tests based on, - тесты основанные на
see also generalized linear models - см. также обобщенные линейные модели
linear panel estimators, 695–778
application, 708–15, 725
Arellano-Bond estimator, - Ареллано-Бонда оценка
between estimator, 703
covariance estimator, 733
conditional ML estimator, - оценка условного максимального правдоподобия
differences-in-differences estimator, - оценка разность разностей
linear panel estimators (cont.)
error components 2SLS estimator, 760
error components 3SLS estimator, 762
first differences estimator, - оценка в первых разностях
first differences IV estimator, - оценка инструментальных переменных в первых разностях
fixed effects estimator, - оценка для модели с фиксированными эффектами
fixed effects IV estimators, - оценка инструментальных переменных для модели с фиксированными эффектами
forward orthogonal deviations IV estimator, 759 
Hausman-Taylor IV estimator, - Хаусмана-Тейлора оценка инструментальных переменных
LSDV estimator, 704, 732–3
MD estimator, 753, 76–7
panel bootstrap, - панельные данные
panel GMM estimators, - панельные оценки обобщенного метода моментов
panel-robust inference, - статистические выводы робастные к панельным данным
pooled OLS estimator, - МНК оценка сквозной регрессии
random effects estimator, - оценка для модели со случайными эффектами
random effects IV estimator, - оценка инструментальных переменных для модели со случайными эффектами
within estimator, 704, 726–9
within IV estimator, 758
linear panel models, - линейные панельные модели
analysis-of-covariance model, 733 
application, - приложения
between model, 702
dynamic models, - динамические модели
endogenous regressors, - эндогенные регрессоры
first differences model, - модель в первых разностях
fixed effects model, - модель с фиксированными эффектами 
fixed versus random effects, 701–2, 715–9 forward orthogonal deviations model, 759 Hausman-Taylor model, 760–2
incidental parameters problem, 704, 726 individual dummies, 699
individual-specific effects model, - модель с индивидуальными эффектами
LSDV model, 704, 732
minimum distance estimator, 753, 766–7 
mean-differenced model, 758
measurement error, - ошибка измерения
mixed linear models, 774–6
pooled model, - сквозная модель
random effects differenced model, 760–1 random effects model, 700–2, 734–6, 759–60 residual analysis, 714–5
strong exogeneity, - сильная экзогенность
time dummies, 699
time-invariant regressors, - регрессоры, постоянные во времени 
749–51 time-varying regressors, - регрессоры, меняющиеся во времени 
749–51 two-way effects model, 738
unbalanced data, 739
weak exogeneity, - слабая экзогенность
within model, - модель within
see also linear panel estimators
linear probability model, 466–7 
linear programming methods, - методы линейного программирования 
linear regression model - линейная регрессионная модель
definition, - определение
linear systems of equations, - системы линейных уравнений
panel data models as, - как способ представления моделей панельных данных
seemingly unrelated regressions, - внешне несвязанные уравнения
simultaneous equations, - системы одновременных уравнений 
systems FGLS estimator, - оценка доступного обобщенного МНК для систем уравнений
systems GLS estimator, - оценка обобщенного МНК для систем уравнений
systems GMM estimator, - оценка обобщенного метода моментов для систем уравнений
systems ML estimator, - оценка  метода максимального правдоподобия для систем уравнений 
systems OLS estimator, - оценка МНК для систем уравнений
systems 2SLS estimator, - оценка двухшагового МНК для систем уравнений
linearization method, - линеаризация
link function, 149, 469, 783 
listwise deletion, 60, 928
consistency under MCAR, 928 
example, - пример
inconsistency under MAR only, 928
Living Standards Measurement Study (LSMS), 59, 88–90, 848–53
LLN. See law of large numbers - ЗБЧ. См. закон больших чисел
LM test. See Lagrange multiplier test - LM тест. См. тест множителей Лагранжа
local alternative hypotheses, 238, 247–8, 254 
local average treatment effects (LATE) estimator, - LATE-оценка, оценка локального среднего эффекта воздействия
assumptions, - предпосылки
comparison with IV estimator, - сравнение с оценкой метода инструментальных переменных
definition, - определение
heterogeneous treatment effect, 885 
monotonicity assumption, - предпосылка о монотонности
selection on unobservables, 883 
Wald estimator, - Вальда оценка
see also ATE; ATET; MTE
local linear regression estimator, 320–1, 333 local polynomial regression estimator, 320–1 local running average estimator, 308, 320 local weighted average estimator, 307–8 
logistic distribution, - логистическое распределение
logistic regression. See logit model - логистическая регрессия. См. логит-модель
logit model, - логит-модель
application, - приложения
as ARUM, - как ARUM модель
clustered data, - кластеризованные данные
definition, - определение
for discrete-time duration data, 602 
GLM, 149
imputation example, 937–9
index function model, - модель индексной функции
marginal effects, - предельные эффекты
measurement error example, - пример с ошибками измерения
ML estimator, - оценка метода максимального правдоподобия
multinomial logit, - мультиномиальная логит-модель
nested logit, - вложенная логит-модель 
ordered logit, - упорядоченная логит-модель
panel data, - панельные данные
probit model comparison, - сравнение с логит-моделью
random parameters logit, - логит-модель со случайными параметрами
see also binary outcome models - см. также модели бинарного выбора
log-likelihood function. See likelihood function - логарифмическая функция правдоподобия. См. функция правдоподобия
length-biased sampling, 43–4
log-logistic distribution, - лог-логистическое распределение
log-normal distribution, - лог-нормальное распределение
log-normal model, - лог-нормальная модель

log-odds ratio, - логарифм отношения шансов
log-sum, - логарифм суммы
log-Weibull distribution. See type 1 extreme value - лог-Вейбулла распределение. См. распределение экстремальных значений 1-го типа
long panel, - длинная панель
longitudinal data. See panel data - лонгитюдные данные. См. панельные данные
loss function, - функция потерь
absolute error, - абсолютная ошибка
asymmetric expected error, - асимметричная ожидаемая ошибка
Bayesian decision analysis, - Байесовское принятие решений
expected, 66
KLIC, - KLIC, Кульбака-Лейблера информационный критерий 
squared error, 67–9, 156
step, 67–8
Lowess regression estimator, 320–1 
application, - приложения
LR test. See likelihood ratio test - LR тест. См. тест отношения правдоподобия
LS estimators. See least squares
LSDV. See least-squares dummy variable
LSMS. See Living Standards Measurement Study
MAR. See missing at random
marginal analysis of panel data, 717, 787 marginal effects, 122–4
in binary outcome models, 466–5, 467, 470–1 calculus method, 123
computing, - вычисления
definition, - определение
example, - пример
finite-difference method, - метод конечных разностей
in fixed effects model, - в моделях с фиксированными эффектами
in multinomial models, - в моделях со случайными эффектами
population-weighted, 821
in sample selection models, - в моделях самоотбора выборки
in single-index models, - в одноиндексных моделях
in Tobit model, - в тобит модели
see also coefficient interpretation - см. также интерпретация коэффициентов
marginal likelihood, - рыночные данные
marginal treatment effects (MTE) estimator, 886 
market-level data, 482, 513
Markov chain Monte Carlo (MCMC) methods, - метод Монте Карло по схеме Марковской цепи
convergence, - сходимость
in data augmentation, - при пополнении данных
examples, - примеры
Gibbs sampler, - сэмплирование по Гиббсу 
Metropolis algorithm, - Метрополиса алгоритм 
Metropolis-Hastings algorithm, - Метрополиса-Гастингса алгоритм 
Markov LLN, - ЗБЧ Маркова 
Marshall-Olkin method, 649–51, 686 
matching assumption, - предпосылка сопоставления
see also overlap assumption matching estimators, 871–8, 889–96
application, - приложения
assumptions, - предположения
ATE matching estimator, 877
ATET matching estimator, 874, 877, 894–6 balancing condition, 893
caliper matching, 874
counterfactuals, 871
exact matching, - точное сопоставление
inexact matching, - неточное сопоставление
interval matching, - интервальное сопоставление
kernel matching, - ядерное сопоставление
nearest-neighbor matching, 875, 894–6 
propensity score matching, - сопоставление мер склонности
radius matching, - радиальное сопоставление
selection on observables only, 871 
stratification matching, - сопоставление со стратификацией, стратифицированное сопоставление
variance computation, - вычисление дисперсии
maximum empirical likelihood (MEL) estimator, - оценка эмпирического максимального правдоподобия
maximum likelihood (ML) estimator, - оценка максимального правдоподобия
asymptotic distribution, - асимптотическое распределение
conditional ML estimator, - оценка условного максимального правдоподобия
consistency, - состоятельность
definition, - определение
endogenous stratification, - эндогенная стратификация
example, - пример
exogenous stratification, - экзогенная стратификация
MSL estimator, - оценка симуляционного максимального правдоподобия
quasi-ML estimator, - оценка квази-максимального правдоподобия
regularity conditions, - условия регулярности
restricted, - ограниченная
unrestricted, - неограниченная
variance matrix estimation, - оценка ковариационной матрицы
weighted ML estimator, - взвешенная оценка максимального правдоподобия
see also quasi-ML estimator - см. также оценка квази-максимального правдоподобия
maximum rank correlation estimator, 485 maximum score estimator, 341, 381, 483–4, 800 
maximum simulated likelihood (MSL) estimator, - оценка симуляционного правдоподобия
393–8
asymptotic distribution, - асимптотическое распределение
bias-adjusted MSL, - оценка симуляционного правдоподобия, скорректированная на смещение
compared to MSM, - сравнение с оценкой симуляционного метода моментов
count model examples, - примеры счетных моделей
definition, - определение
example, - пример
multinomial probit model, - мультиномиальная пробит модель
number of simulations, - количество симуляций
random parameters logit model, - логит-модель со случайными параметрами
MCAR. See missing completely at random
MD estimator. See minimum distance estimator mean-differenced estimator, 783, 805–6 mean-differenced model, 758, 783
mean imputation, 928, 936–8
mean integrated squared error (MISE), 303, 314 mean-scaling estimator, 783, 805–6 
mean-square convergence, - сходимость в среднем квадратичном
mean substitution. See mean imputation measurement error
in cohort-level data, 772–3
in dependent variable, - в зависимой переменной
in microdata, - в микроданных
in panel data, - в панельных данных
in regressors, - в регрессорах
see also measurement error model estimators;
measurement error models - модели ошибки измерения
measurement error model estimators, - оценки для модели ошибки измерения
attenuation bias, 903–5, 911, 915, 919–20 
bounds identification, 906–8
corrected score estimator, 
IV estimator, - оценка инструментальных переменных
linear models, - линейные модели
nonlinear models, - нелинейные модели
OLS estimator inconsistency, - несостоятельность МНК оценки
using additional moment restrictions, 909–10 
using instruments, 908–9
using known measurement error variance, 902–3, 910
using replicated data, 910–1, 913
using validation sample, 911 
measurement error models, - модели ошибки измерения
attenuation bias, 903–5, 911, 915, 919–20 
classical measurement error model, - классическая модель ошибки измерения
dependent variable measured with error, - зависимая переменная измеренная с ошибкой
examples, - примеры
identification, - идентифицируемость
linear models, - линейные модели
multiple regressors, 904
nonclassical measurement error, - неклассическая ошибка измерения
nonlinear models, - нелинейные модели
panel models, - панельные модели
scalar regressor, - скалярный регрессор
serial correlation, - автокорреляция
variance inflation, - вздутие дисперсии
see also measurement error model estimators
median regression. See LAD estimator MEL. See maximum empirical likelihood m-estimator, 118–22
asymptotic distribution, - асимптотическое распределение
clustered data, - кластеризованные данные
definition, - определение
sequential two-step, 200–2 
simulated m-estimator, - симуляционная М-оценка
tests based on, - тесты основанные на
weighted m-estimator, 829, 856 see also extremum estimators
method of moments (MM) estimator asymptotic distribution, 134, 174 
definition, - определение
examples, - примеры
see also estimating equations estimator; GMM estimator
method of scoring, 343, 348
method of simulated moments (MSM) estimator, - оценка симуляционного метода моментов
asymptotic distribution, - асимптотическое распределение
compared to MSL, - сравнение с методом симуляционного правдоподобия
definition, - определение
example, - пример
MNP model, - мультиномиальная пробит-модель
number of simulations, - количество симуляций
method of simulated scores (MSS) estimator for MNP model, 519
method of steepest ascent, 344
Metropolis algorithm, - Метрополиса алгоритм
Metropolis-Hastings algorithm, - Метрополиса-Гастингса алгоритм
microdata sets, - наборы микроданных
handling, - обработка
leading examples, - основные примеры
microeconometrics overview, 1–17 midpoint rule, 388, 391–2 minimum chi-square estimator, 203
see also Berkson’s minimum chi-square estimator minimum distance (MD) estimator, 202–3, 753, 766–7
asymptotic distribution, - асимптотическое распределение
bootstrap for, 379–80 
covariance structures, - ковариационные структуры
definition, - определение
equally-weighted, 202 generalized, 222
indirect inference, - косвенные статистические выводы 
OIR test, 203
optimal, 202, 753
panel data, - панельные данные 
relation to GMM, 203, 753
misclassification, - неправильная классификация
MISE. See mean integrated squared error missing at random (MAR), 926–7
definition, - определение
and ignorable missingness, 927, 932 relation to MCAR, 927
missing completely at random (MCAR), 926–7
definition, - определение
and ignorable missingness, 927 relation to MCAR, 927
missing data, - пропущенные данные
deletion methods, 928
examples, - примеры
ignorable assumption, 927 imputation with models, 929–41 imputation without models, 928–9 MAR assumption, 926–7
MCAR assumption, 927 nonignorable missingness, 927, 940 see also imputation methods
misspecification tests. See specification tests - тесты на неправильную спецификацию. См. тесты на спецификацию
mixed estimator, 439–41
mixed linear model, 774–6
Bayesian methods, - Байесовские методы
FGLS estimator, - оценка доступного обобщенного МНК
fixed parameters, - фиксированные параметры
ML estimator, - ML оценка, оценка максимального правдоподобия
random parameters, - случайные параметры
restricted ML estimator, - ограниченная оценка максимального правдоподобия
nonstationary panel data, - нестационарные панельные данные
prediction, - прогнозирование
see also hierarchical linear model mixed logit model, 500–3
1022
example, - пример
definition, - определение
see also RPL model
mixed proportional hazards (MPH) model, 611–25
Weibull-gamma mixture, 615
see also mixture models mixture hazard function, 616–8 
mixture models, - модели смеси распределений
application, - приложения
counts, 675–9
durations, 611–25
identification, - идентифицируемость
MSL estimator, - оценка метода симуляционного максимального правдоподобия
multinomial outcomes, - мультиномиальные исходы
multiplicative heterogeneity, - мультипликативная неоднородность
specification tests, - тесты на спецификацию
see also finite mixture models; unobserved heterogeneity
ML estimator. See maximum likelihood - ML оценка. См. метод максимального правдоподобия
MM estimator. See method of moments - ММ оценка. См. метод моментов
MNL estimator. See multinomial logit - оценка мультиномиальной логит-модели. См. мультиномиальная логит-модель
MNP estimator. See multinomial probit model - оценка мультиномиальной пробит-модели. См. мультиномиальная пробит-модель
diagnostics, 287–91
binary outcome models, - модели бинарного выбора 
duration models, - модели времени жизни 
example, - пример
multinomial outcome models, - мультиномиальные модели 
pseudo-R2 measures, - псевдо R2 критерии
residual analysis, - анализ остатков
see also model selection methods model misspecification, 90–4
see also endogeneity; functional form misspecification; heterogeneity; omitted values; pseudo-true value
model selection methods - методы выбора модели
Bayesian, - Байесовские
nested models, - вложенные модели 
nonnested models, - невложенные модели 
order of testing, - порядок тестирования
see also model diagnostics; specification tests moment-based simulation estimators,
398–404
see MSL estimator; MSM estimator - см. оценка метода симуляционного правдоподобия, оценка симуляционного метода моментов
moment-based tests. See m-tests moment matching. See indirect inference 
Monte Carlo integration, - интегрирование с использованием Монте-Карло
direct, - прямое
example, - пример
importance sampling, - сэмплирование по важности 
simulators, - симуляторы
see also quadrature - см. также квадратура (численное интегрирование)
Monte Carlo studies, - исследования с использованием метода Монте-Карло
example, - пример
moving average estimator, - оцека скользящего среднего
moving blocks bootstrap, 373, 381
MPH model. See mixed proportional hazards - модель смешанные пропорциональных рисков. См. смешенные пропорциональные риски
MSL estimator. See maximum simulated likelihood MSM estimator. See method of simulated moments MSS estimator. See method of simulated scores MTE. See marginal treatment effects
m-tests, 260–71
asymptotic distribution, - асимптотическое распределение
auxiliary regressions, - вспомогательная регрессия
bootstrap, - бутстрэп
chi-square goodness of fit, - хи-квадрат критерий согласия
conditional moment test, - тест условных моментов 
CM test interpretation, - интерпретация теста условных моментов
computation, - вычисление
definition, - определение
Hausman test, - Хаусмана тест
information matrix tests, - критерий (тест) информационной матрицы
outer-product-of-the-gradient form, 262 
overidentifying restrictions test, 181, 183, 267,
747 
power, - мощность
rank, - ранг
multicollinearity, - мультиколлинеарность
in multinomial probit model, - в мультиномиальной пробит-модели
in panel model, - в панельных моделях
in sample selection model, - в моделях самоотбора выборки
multilevel models. See hierarchical models - многоуровневые модели. См. иерархические модели
multinomial logit (MNL) model, - мультиномиальная логит-модель
application, - приложения
as additive random utility model, - как аддитивная модель случайной полезности
definition, - определение
marginal effects, - предельные эффекты
ML estimator, - оценка метода максимального правдоподобия
panel data, - панельные модели
see also multinomial outcome models - см. также мультиномиальные модели
multinomial outcome models, - мультиномиальные модели
application, - приложения
alternative-invariant regressors, 498 alternative-varying regressors, 497 
conditional logit, - условная логит-модель
definition, - определение
identification, - идентифицируемость
index function model, - модель индексных функций
marginal effects, - предельные эффекты 
mixed logit, - смешанная логит-модель
ML estimator, - оценка метода максимального правдоподобия
multinomial logit, - мультиномиальная логит-модель
multinomial probit, - мультиномиальная пробит-модель
ordered models, - модели упорядоченного выбора
OLS estimator, - МНК оценка
panel data, - панельные данные
random parameters logit, - логит-модель со случайными параметрами
random utility model, - модель случайной полезности
semiparametric estimation, - полупараметрическое оценивание
multinomial probit (MNP) model, - мультиномиальная пробит-модель
Bayesian Methods, - Байесовские методы
definition, - определение
identification, - идентифицируемость
ML estimator, - оценка метода максимального правдоподобия
MSL estimator, - оценка метода симуляционного правдоподобия
MSM estimator, - оценка симуляционного метода моментов
MSS estimator, 518
see also multinomial outcome models - см. также мультиномиальные модели
multiple duration spells, 655–8 
fixed effects, - фиксированные эффекта
lagged duration dependence, 657 
ML estimator, - оценка метода максимального правдоподобия
random effects, - случайные эффекты
recurrent spells, 655 
multiple imputation, - множественное восстановление
estimator, - оценка
examples, - примеры
relative efficiency, - относительная эффективность 
variance of estimator, - дисперсия оценки
multiple treatments, - множественные воздействия
multiplicative errors - мультипликативная ошибка
multistage surveys, 41–2, 814–6, 853–6
variance estimation, - оценка дисперсии 
multivariate data - многомерные данные
binary outcomes, - бинарные исходы 
counts, 685–7
durations, 640–64
see also systems of equations - см. также системы уравнений
multivariate-t distribution, - многомерное t-распределение
NA estimator. See Nelson-Aalen - Нельсона-Аалена оценка. См. Нельсон-Аален
National Longitudinal Survey (NLS), 58, 110–2 National Longitudinal Survey of Youth (NLSY),
58–9
National Supported Work (NSW) demonstration
project, 889–95
natural conjugate pair, - сопряженное распределение 
natural experiments, - естественные эксперименты
definition, - определение
differences-in-differences estimator, - оценка разность разностей
examples, - примеры
exogenous variation, - экзогенная изменчивость 
identification, - идентифицируемость
instrumental variables, - инструментальные переменные
regression discontinuity design, - разрывный дизайн 
ncp. See noncentrality parameter
nearest neighbors (k-NN) estimator, - оценка ближайших соседей (k-NN оценка)
definition, - определение
example, - пример
symmetrized, 308, 320
see also nonparametric regression - см. также непараметрическая регрессия
nearest-neighbor matching, 875, 894–6 
negative binomial distribution, - отрицательное биномиальное распределение
negative binomial model, - отрицательная биномиальная модель
application, - приложения 
bivariate, - двумерная
hurdle model, - модель преодоления порогов
ML estimator, - оценка максимального правдоподобия
MSL estimator, - оценка симуляционного максимального правдоподобия 
NB1 variant, - NB1, отрицательная биномиальная модель 1
NB2 variant, - NB2, отрицательная биномиальная модель 2
panel data, 804, - панельные данные
negative hypergeometric distribution, - отрицательное гипергеометрическое распределение 
neglected heterogeneity. See unobserved heterogeneity - См. ненаблюдаемая неоднородность

Nelson-Aalen (NA) estimator, - Нельсона-Аалена оценка
application, - приложения
confidence bands for, 584 
definition, - определения
tied data, 582
nested bootstrap, - вложенный бутстрэп
nested logit model, - вложенная логит-модель
from ARUM, - для ARUM модели
definition - определение
different versions of, - различные версии 
example, - пример
GEV model, 508, 526 
ML estimator, - оценка метода максимального правдоподобия
sequential estimator, 510 
welfare analysis, - анализ благосостояния
see also multinomial models  - см. также мультиномиальные модели
nested models - вложенные модели
see also nonnested models - см. также невложенные модели
neural network models, - модели нейронных сетей
Newey-West robust standard errors, - Ньюи-Веста робастные стандартные ошибки
723
definition, - определение
see also robust standard errors - см. также робастные стандартные ошибки
Newton-Raphson (NR) method, - Ньютона-Рафсона алгоритм
examples, - примеры
NLFIML estimator. See nonlinear full-information maximum likelihood - оценка нелинейного максимального правдоподобия с полной информацией
NLS estimator. See nonlinear least squares - оценка нелинейного МНК
NLSY. See National Longitudinal Survey of Youth NL2SLS estimator. See nonlinear two-stage least
squares
NL3SLS estimator. See nonlinear three-stage least squares - Оценка нелинейного трехшагового МНК
noise-to-signal ratio, - отношение шум-сигнал
noncentral chi-square distribution, - нецентрированное хи-квадрат распределение
noncentrality parameter (ncp), 248 nonclassical measurement error, 904, 920 
nongradient methods, - неградиентные методы
nonignorable missingness, 927, 940
attrition bias due to, 940
selection bias due to, 927, 932, 940 
nonlinear estimators - нелинейные оценки
coefficient interpretation, - интерпретация коэффициентов
extremum estimator m-estimator, 118–22
GMM estimator, - ОММ оценка (оценка обобщенного метода моментов)
ML estimator, - оценка метода максимального правдподобия
NLS estimator, - оценка нелинейного МНК
overview, - обзор
panel models, - панельные модели
nonlinear full-information maximum likelihood (NLFIML) estimator, - оценка нелинейного максимального правдоподобия с полной информацией
nonlinear GMM estimator, - оценка нелинейного обобщенного метода моментов
asymptotic distribution, - асимптотическое распределение
definition, - определение
example, - пример 
instrument choice, - выбор инструментов 
NL2SLS estimator, - оценка нелинейного двухшагового МНК
optimal, - оптимальная
panel data, - панельные данные
nonlinear in parameters, - нелинейность по параметрам
nonlinear in variables, - нелинейность по переменным
nonlinear IV estimator. See nonlinear GMM nonlinear least squares (NLS) estimator, 150–9
asymptotic distribution, - асимптотическое распределение 
consistency, - состоятельность
definition, - определение
example, - пример
time series, - временные ряды
variance matrix estimation, - оценка ковариационной матрицы
nonlinear panel estimators, - нелинейные панельные оценки
application, - приложения
conditional ML estimator, - оценка условного максимального правдоподобия
dummy variable estimator, - оценка с помощью дамми-переменных 
first-differences estimator, - оценка в первых разностях
fixed effects estimator, - оценка модели с фиксированными эффектами
GEE estimator, - обобщенная оценка оценивающих уравнений
mean-differenced estimator, 783, 805–6 
mean-scaling estimator, 783, 805–6
ML estimator, - оценка метода максимального правдоподобия
NLS estimator, - оценка нелинейного МНК
panel GMM estimator, - панельная оценка обобщенного метода моментов
panel-robust inference, - статистические выводы робастные к панельным данным
quadrature, - квадратура (численное интегрирование)
quasi-differenced estimator, - оценка в квази-разностях
quasi-ML estimator, - оценка метода квази-максимального правдоподобия
random effects estimator, - оценка модели со случайными эффектами
selection models, - модели самоотбора
semiparametric, - полупараметрические
nonlinear panel models, - нелинейные панельные модели
application, - приложения
binary outcome models, - модели бинарного выбора
conditional mean models, - модели условного среднего
count models, - счетные модели
dynamic models, - динамические модели 
endogenous regressors, - эндогенные регрессоры 
exogeneity assumptions, - предположения об эндогенности
finite mixture models, - модели конечной смеси
fixed effects models, - модели с фиксированными эффектами
fixed versus random effects, - сравнение фиксированных и случайных эффектов 
incidental parameters problem, - проблема мешающих параметров
individual-specific effects models, - модели с индивидуальными эффектами 
parametric models, - параметрические модели
pooled models, - сквозные модели
random effects models, - модели со случайными коэффициентами
selection models, - модели самоотбора
semiparametric, - полупараметрические
Tobit models, - тобит модели
transition models, 801–2
nonlinear regression model, - нелинейные регрессионные модели
additive error, - аддитивные ошибки
nonadditive error, - неаддитивные ошибки
nonlinear systems of equations, - нелинейные системы уравнений
additive errors, - аддитивные ошибки
copulas, - копулы
mixtures, - смеси распределений
ML estimator, - оценка метода максимального правдоподобия
NLFIML estimator, - оценка нелинейного метода максимального правдоподобия с полной информацией
NL3SLS estimator, - оценка нелинейного трехшагового МНК
nonadditive errors, - неаддитивные ошибки
nonlinear panel model, - нелинейные панельные модели
nonlinear SUR model, - нелинейные внешне несвязанные уравнения
quasi-ML estimator, - оценка метода квази-максимального правдоподобия
seemingly unrelated regressions, - внешне несвязанные уравнения
simultaneous equations, - одновременные уравнения
systems FGNLS estimator, 217 - оценка доступного обобщенного нелинейного МНК для систем уравнений
systems GMM estimator, - оценка обобщенного метода моментов для систем уравнений 
systems IV estimator, - оценка инструментальных переменных для систем уравнений
systems MM estimator, - оценка метода моментов для систем уравнений
systems NLS estimator, - оценка нелинейного МНК для систем уравнений
nonlinear three-stage least squares (NL3SLS) estimator, - оценка нелинейного трехшагового МНК 
nonlinear two-stage least squares (NL2SLS) estimator - оценка нелинейного двухшагового МНК
asymptotic distribution, - асимптотическое распределение
definition, - определение
example, - пример
see also nonlinear GMM estimator  - см. также оценка нелинейного обобщенного метода моментов
nonnested models - невложенные модели
Cox LR test, - Кокса тест отношения правдоподобия
definition, - определение
example, - пример
information criteria comparison, 278–9 
overlapping, 281
strictly nonnested, - 
Vuong LR test, 280–3
nonparametric bootstrap. See paired bootstrap nonparametric density estimation. See kernel density
estimator
nonparametric maximum likelihood (NPML) - непараметрический метод максимального правдоподобия
estimator, - оценка
nonparametric regression, - непараметрическая регрессия
convergence rate, - схорость сходимости
kernel, - ядро
local linear, - локальная линейная
local weighted average, 307–8 
Lowess, 320
nearest-neighbors, - ближайших соседей
series, 321
statistical inference intuition, - интуиция за статистическими выводами
test against parametric model, - тестирование против параметрическое модели
see also semiparametric regression - см. также полупараметрическая регрессия
nonrandomly varying coefficient, 846 
normal copula, - нормальная копула
normal distribution, - нормальное распределение
truncated moments, - усеченные моменты
normal limit product rule. See Cramer linear
transformation
NPML estimator. See nonparametric maximum likelihood - См. непараметрический метод максимального правдоподобия
NR method. See Newton-Raphson method - См. Ньютона-Рафсона метод
NSW demonstration project. See National Supported
Work
nuisance parameters. See incidental parameters - мешающие параметры

numerical derivatives, 340, 350 - численные производные
numerical integration. See quadrature - численное интегрирование (квадратура)
observational data, - данные наблюденй
biased samples, - смещенные выборки
clustering, - кластеризация
identification strategies, - стратегия идентификации
measurement error, - ошибка измерения
missing data, - пропущенные данные
population, - генеральная совокупность
sample attrition, - истощение выборки
sampling methods, 40–4, 815–7 
sampling units, - единицы выборки
sampling without replacement, - выборка без повторения
survey methods, 41–2, 814–7
survey nonresponse, 45–6
types of data, - типы данных
observational equivalence, 29 
odds ratio, - отношение шансов
see also posterior odds ratio - см. также апостериорное отношение шансов
OIR test. See overidentifying restrictions test OLS estimator. See ordinary least squares omitted variables bias, 92–3, 700, 716
LM tests for, - LM тест (тест множителей Лагранжа)
one-step GMM estimator, - одношаговая оценка методом моментов
panel, - панель
see also two-stage least squares - см. также двухшаговый метод наименьших квадратов
one-way individual-specific effects model. See individual-specific effects model - модель с односторонними индивидуальными эффектами. См. модель с индивидуальными эффектами
on-site sampling, 43, 823
optimal Bayesian estimator, - оптимальная Байесовская оценка
optimal GMM estimator, - оптимальная оценка обобщенного метода моментов
compared to 2SLS, - сравнение с двухшаговым МНК
optimal MD estimator, 202, 753
OPG. See outer-product of the gradient
Orbit model, - орбит-модель
order of magnitude, - порядок малости
ordered logit model, - упорядоченная логит-модель
ordered multinomial models, - упорядоченные мультиномиальные модели
ordered probit model, - упорядоченная пробит-модель
ordinary least squares (OLS) estimator, - МНК оценка (оценка метода наименьших квадратов)
asymptotic distribution, - асимптотическое распределение
bias in standard errors with clustering, - смещение стандартных ошибок при кластеризованных данных
binary data, - бинарные данные
clustered data, - кластеризованные данные
coefficient interpretation in misspecified model, - интерпретация коэффициентов в неправильно специфицированной модели
consistency - состоятельность
definition, - определение
example, - пример
finite-sample distribution, - распределение в конечных выборках
heteroskedasticity-robust standard errors, - стандартные ошибки устойчивые к гетероскедастичности
identification, - идентификация
inconsistency, - несостоятельность
inefficiency, - неэффективность
nonlinear, - нелинейная
panel data, - панельные данные
see also least squares estimators - см. также оценки метода наименьших квадратов
orthogonal polynomials, - ортогональные многочлены
definition - определение
orthogonal regression approach, 920 
orthonormal polynomials, - ортонормальные многочлены 
outcome equation, 547, 867
outer product (OP) estimate, 138, 241, 395 outer-product of the gradient (OPG) version
LM test, - LM тест (тест множителей Лагранжа)
m-test, - М-тест
small-sample performance, 262
overdispersion, - избыточная дисперсия 
measurement error, - ошибка измерения 
panel data, - панельные данные
tests for, - тесты на
overidentification, - сверх-идентифицируемость 
see also GMM estimator - см. также оценка обобщенного метода моментов
overidentifying restrictions (OIR) test 
asymptotic distribution, - асимптотическое распределение 
bootstrap, - бутстрэп
definition, - определение
panel data, - панельные данные
overlap assumption, 864, 871
in RD design, - при разрывном дизайне
oversampling, 41, 478–9, 814, 872
paired bootstrap, - парный бутстрэп
pairwise deletion, - попарное удаление
biased standard errors, - смещенные стандартные ошибки
panel attrition, - истощение панели
panel bootstrap, - панельные бутстрэп 
panel data, - панельные данные
panel data models and estimators, - панельные данные и оценки
comparison to clustered data, - сравнение с кластеризованными данными 
see also linear panel; nonlinear panel panel GMM estimators, 744–68, 789–90
application, - приложения
Arellano-Bond estimator, - Ареллано-Бонда оценка
asymptotic distribution, - асимптотическое распределение
bootstrap, - бутстрэп
compared to MD estimator, 753 
computation, - вычисление
definition, - определение
efficiency, - эффективность
exogeneity assumptions, - предположения об экзогенности
instruments, - инструменты
IV estimators for FE model, 757–9 IV estimators for RE model, 759–60 just-identified, 745
nonlinear, 789–90
OIR test, 747, 756
one-step GMM estimator, - одношаговая ОММ оценка (одношаговая оценка обобщенного метода моментов)
overidentified, 745
2SLS estimator, - двухшаговая МНК оценка
two-step GMM estimator, - двухшаговая ОММ оценка (двухшаговая оценка обобщенного метода моментов)
variance matrix estimation, - оценивание ковариационной матрицы
panel GMM model, - панельная модель с использованием обобщенного метода моментов
application, - приложения
dynamic, - динамическая

with individual-specific effects, 750–62 without individual-specific effects, 744–53 see also panel GMM estimators
panel IV estimators. See panel GMM estimators panel-robust statistical inference, 377, 705–7, 722,
746, 751, 788–90 for Hausman test, 718
Panel Study in Income Dynamics (PSID), 58, 889 parametric bootstrap, 360
Pareto distribution - Парето распределение
of the first kind, - первого типа
of the second kind, - второго типа
partial additive model, 323
partial equilibrium analysis, 53, 862, 972
see also SUTVA
partial F-statistic, 105, 109, 111
partial likelihood estimator, 594–6
partial ML estimator, 140
partial R-squared, 104–5, 111
partially linear model, 323–5, 327, 565, 684 participation equation, 547, 551
Pearson chi-square goodness-of-fit test, - Пирсона критерий согласия
Pearson residual, - Пирсона остатки
peer-effects model, 832
percentile, - перцентиль 
percentile method, - перцентильный метод
percentile-t method, - $t$-перцентильный метод
PH model. See proportional hazards piecewise constant hazard model, 591 Pitman drift, 248
PML estimator. See pseudo-ML estimator Poisson distribution, 668
Poisson-gamma mixture, 
Poisson-IG mixture, 677
Poisson regression model, - Пуассоновская регрессионная модель
application, - применения
asymptotic distribution of estimators, - асимптотическое распределение оценок
bivariate, - двумерная
censored MLE, - цензурированные оценки максимального правдоподобия
with clustered data, - с кластеризованными данными
coefficient interpretation, - интерпретация коэффициентов
definition, - определение
equidispersion, 668
example, - пример
LEF density, 148 
measurement error, - ошибка измерения
mixtures, 675–9
ML estimator, 668
overdispersion, - избыточная дисперсия
panel data, 792–5, 802–6 quasi-ML estimator, 668–9, 682–3 truncated MLE, 535 underdispersion, 671 zero-truncated, 680
see also count models - см. также счетные модели
polynomial baseline hazard, 591, 636
pooled cross-section time series model. See pooled
model
pooled estimators, 702–3, 720–5 
application, - приложения
FGLS estimator, - оценка доступного обобщенного МНК
GEE estimator, - обобщенная оценка оценивающих уравнений
NLS estimator, - оценка нелинейного МНК
OLS estimator, - МНК оценка
WLS estimator, - оценка взвешенного МНК
pooled model, 699, 720–5, 787–8
pooling tests, 737
population-averaged model. See pooled model population moment conditions
for estimation, 172
for testing, 260
see also GMM estimator; MM estimator; m-tests
posterior distribution, - апостериорное распределение
asymptotic behavior, 432–4 
conditional posterior, - условное апостериорное распределение
definition, - определение
expected posterior loss, - ожидаемые апостериорные потери
expected posterior risk, - ожидаемый апостериорный риск
full conditional distribution, 431 highest posterior density interval, 431 highest posterior density region, 431 marginal posterior, 430 observed-data posterior, 930 
posterior density interval, 431 
posterior mean, - среднее апостериорного распределения
posterior mode, - мода апостериорного распределения
posterior moments, - апостериорные моменты
posterior precision, 423
see also Bayesian methods - см. также Байесовские методы
posterior odds ratio, - апостериорное отношение шансов
posterior (P) step, 455, 933
potential outcome model, - модель потенциального результата
see also treatment effects; treatment evaluation power of tests, 247–50, 253–4
bootstrapped tests, 372–3 
conditional moment test, - тест условных моментов
example, - пример
Hausman test, - Хаусмана теста
local alternative hypotheses, 247–8 
uniformly most powerful test, - равномерно наиболее мощный тест
Wald tests, - Вальда тесты
precision parameter, - параметр точности
predetermined instruments. See weak exogeneity 
prediction, - прогноз
best linear, - наилучший линейный
conditional, - условный
error, - ошибка
in linear panel models, - в линейных моделях
in mixed linear model, 774–6 optimal, 66–70
rotation groups, 814 in structural model, 28 weighted, 821
pretest estimator, - претест оценка
primary sampling units (PSUs), - первичная единица выборки (ПЕВ)
prior distribution, - априорное распределение
conjugate prior, - сопряженное априорное распределение 
definition, - определение
Dickey’s prior, 439 
diffuse prior, 426
flat prior, 426
hierarchical priors, 428–9, 441–2 
improper prior, - несобственное априорное распределение
informative prior, 437–9 Jeffreys’ prior, 426 
noninformative prior, 425, 435–7 normal-gamma prior, 437 sensitivity analysis for, 429–30 see also Bayesian methods
probit model, - пробит-модель
application, - приложения
as additive random utility model, 477 
bivariate probit, - двумерная пробит-модель
bootstrap example, - пример бутстрэпа
definition, - определение
discrete-time duration data, - данные продолжительности жизни в дискретном времени
as GLM, 149
index function model, - модель индексных функций
logit model comparison, - сравнение с логит-моделью
marginal effects, - предельные эффекты
ML estimator, - оценка метода максимального правдоподобия
Monte Carlo study example, 251–4 
multinomial probit, - мультиномиальная пробит-модель
ordered probit, - упорядоченная пробит-модель
panel data, - панельные данные
simultaneous equations probit, 523, 560–1 see also binary outcome models
probit selection equation, 548 
product copula, 654
product integral, 578
product rule, - правило произведения
see also Cramer linear transformation program evaluation. See treatment evaluation projection pursuit model, 323
propensity score, - мера склонности
application, - приложения
balancing condition, 864, 893–4 conditional independence assumption, 865 
definition, - определение
matching, - сопоставление
see also treatment evaluation
proportional hazards (PH) model, 592–7 application, 605–7
baseline survivor function estimator, 596–7 coefficient interpretation, 606–7
competing risks model, 645–6 definition, 591
discrete-time model, 600–3 leading examples, 585
mixed PH, 611–25
panel data, 802
partial likelihood estimator, 
pseudo-ML estimator (PML). See quasi-ML estimator - метод псевдо-максимального правдоподобия

pseudo panels, - псевдо панели
cohort, - когорта
cohort fixed effects, 772–3 
measurement error, - ошибка измерения
pseudo-random number generators, - генератор псевдо-случайных чисел 
accept-reject methods, 413–4
composition methods, 415
inverse transformation method, - метод обратного преобразования
leading distributions, 957–9 multivariate normal, 416 
transformation method, - метод преобразования
uniform variates, 412
see also MCMC methods pseudo R-squared measures
for binary outcome models, - для моделей бинарного выбора
definitions, - определения
example, - пример
for multinomial outcome models, - для мультиномиальных моделей
pseudo-true value, 94, 132, 146, 281
PSID. See Panel Study in Income Dynamics PSUs. 
See primary sampling units - См. первичная единица выборки (ПЕВ)
pure exogenous sampling, 825
p-value, - P-значение
quadrature, 388–90
Gaussian, 389–90
multidimensional, 393
in nonlinear panel models, - в нелинейных панельных моделях 
see also Monte Carlo integration - см. также интегрирование с использованием Монте-Карло
qualititative response models. See binary outcomes, multinomial outcomes
quantile, - квантиль
quantile regression, - квантильная регрессия
application, - приложения
asymmetric absolute loss, - асимметричные абсолютные потери 
asymptotic distribution, - асимптотическое распределение
bootstrap, - бутстрэп
computation, - вычисление
definition, - определение
IV estimator, - оценка инструментальных переменных
multiplicative heteroskedasticity, - мультипликативная гетероскедастичность
quasi-difference, 783–4
quasi-experiment. See natural experiment - квази-эксперимент. См. естественный эксперимент
quasi-maximum likelihood (QML) estimator, - оценка квази-максимального правдоподобия
asymptotic distribution, - асимптотическое распределение
in binary outcome models, - в моделях бинарного выбора
in clustered models, - в моделях кластеризованных данных
definition, - определение
in LEF, 147–9
with multivariate dependent variable, - с многомерной зависимой переменной
in nonlinear systems, - в нелинейных системах
in panel models, - в панельных моделях
in Poisson model, - в модели Пуассона
quasi-random numbers. See pseudo-random numbers - квази-случайные числа. См. псевдо-случайные числа
QML estimator. See quasi-ML estimator - См. оценка квази-максимального правдоподобия
random assignment, - случайное назначение 
see also sampling schemes - см. также план выборки
random coefficients model, - модель со случайными коэффициентами
see also hierarchical models - см. также иерархические модели
random effects (RE) estimator, - оценка RE модели (оценка модели со случайными эффектами)
application, - применения
asymptotic distribution, - асимптотическое распределение
clustered data, - кластеризованные данные
consistency, - состоятельность
definition, - определение
error components 2SLS estimator, 760 error components 3SLS estimator, 762 
FGLS estimator, -  оценка доступного обобщенного МНК
GEE estimator, - обобщенная оценка метода оценивающих уравнений 
Hausman test, - тест Хаусмана
incidental parameters, 704, 726
IV estimators, - оценки инструментальных переменных
ML estimator, - оценка максимального правдоподобия
NLS estimator, - нелинейная МНК оценка
quasi-ML estimator, - оценка квази максимального правдоподобия
two-way effects model, - модель с двусторонними эффектами
versus fixed effects, - сравнение с фиксированными эффектами
random effects (RE) model, - модель со случайными эффектами (RE модель)
binary outcome models, - модель бинарного выбора
Chamberlain model, - Чемберлена модель
clustered data, - кластеризованные данные
count models, - счетные модели 
definition, - определение
dynamic models, - динамические модели
duration models, - модели времени жизни
endogenous regressors, - эндогенные регрессоры
Mundlak model, - модель Мундлака
nonlinear models, - нелинейные модели
selection models, - модели самоотбора
Tobit model, - тобит модель 
two-way effects model, - модель с двусторонними эффектами
versus random effects, - сравнение со случайными эффектами
see also hierarchical models; random effects
estimator
random number generators. See pseudo-random numbers - генераторы случайных чисел. См. псевдослучайные числа
random parameters logit (RPL) model, - модель логит со случайными параметрами
Bayesian methods, - Байесовские методы
definition, - определение
ML estimator, 513–4
random parameters model. See random coefficients model - модель со случайными параметрами. См. модель со случайными коэффициентами
random utility models. See ARUM randomization bias, 53, 867 
randomized experiment, - рандомизированный эксперимент
National Supported Work demonstration project, 889
randomized trials, - рандомизированные эксперименты
randomly varying coefficient, - случайно изменяющийся коэффициент
rank condition for identification, 31, 182, 214 rank-ordered logit model, 521
rank-ordered probit model, 521
raw residual, - исходные остатки
RD design. See regression discontinuity design receiver operators characteristics (ROC) curve, 474 reduced form, 21, 25, 213
see also structural model
RE estimator. See random effects regression-based imputation, 930–2
EM algorithm, - EM алгоритм
nonignorable missingness, 932
regression discontinuity (RD) design, 879–83
fuzzy RD design, 882
heterogeneous treatment effects, 882
RD estimator, 882–3
sharp RD design, 880–1
treatment assignment mechanism, 879–81
regressors, - регрессоры
alternative-varying, 478, 497–8 
endogenous, - эндогенные
fixed, - фиксированные (неслучайные)
irrelevant, 93
omitted, - пропущенные
stochastic, - стохастические
time-varying, - меняющиеся во времени
see also endogenous regressors - см. также эндогенные регрессоры
regularity conditions for ML, - условия регулярности для максимального правдоподобия
relative risk, - относительный риск
reliability ratio, 903
renewal function, - функция восстановления
renewal process, - процесс восстановления
repeated cross section data, - повторные пространственые данные
see also differences-in-differences repeated measures. See panel data replicated data, 910–1, 913
RESET test, - RESET тест
residual analysis - анализ остатков
definitions, - определения
duration data, 633–6 
example, - пример
panel data, - панельные данные
small-sample correction, - корректировка для выборок малого размера
residual bootstrap, - бутстрэп остатков
response-based sampling, 43
restricted ML estimator, 233, 776
revealed preference data, 498, 516
ridge regression estimator, 440
Robinson difference estimator, 324–5, 565 robust sandwich variance matrix estimate. See
sandwich variance matrix robust standard errors
bootstrap, 362–3, 376–8
Eicker-White, 74–5, 80–1, 112, 137
for extremum estimator, 137–9
Huber-White, 137, 144, 146
Newey-West, 137, 175, 723
see also cluster-robust; heteroskedasticity-robust;
panel-robust; systems-robust
ROC curve. See receiver operators characteristics
curve
rotating panels, 739
Roy model, 555–7, 562
definition, 556
dummy endogenous variable, 557 Heckman two-step estimator, 556 ML estimator, 556
panel semiparametric estimation, 808 as treatment effects model, 867
RPL model. See random parameters logit R-squared, 287
pseudo, 287–9
uncentered, 241, 263 running mean estimator, 308
SA method. See simulated annealing sample attrition, 47
sample moment conditions
see population moment conditions sample selection bias, 44–5
sample weights, 817–21, 853–6
see also weighting sampling schemes - см. также взвешенные выборки
assumptions for OLS, 76–78
case-control, 479, 823
choice-based sampling, 43, 478–9, 823 endogenous sampling, 42–5, 78, 822–9, 856 endogenous stratified sampling, 78, 820, 825–6,
856
exogenous stratified sampling, - выборка с экзогенной стратификацией
fixed in repeated samples,  - фиксированные регрессоры в повторяющихся выборках
flow sampling, 44, 626
multi-stage surveys, - многоэтапные опросы
on-site sampling, 43, 823
simple random sampling, - простая случайная выборка
stock sampling, 44, 626–7
with replacement, - с повторениями
without replacement, - без повторений
sandwich variance matrix - ковариационная матрица в сэндвич-форме
clustered data, - кластеризованные данные
extremum estimator, - экстремальная оценка
GMM estimator, - оценка обобщенного метода моментов
ML estimator, - оценка метода максимального правдоподобия
NLS estimator, - оценка нелинейного МНК
OLS estimator, - МНК оценка
panel data, - панельные данные
for Wald test, 277
see also robust standard errors 
Sargan test, - Саргана тест
see also overidentifying restrictions test scale parameter, 509
scanner data, 499
Schwarz criterion. See BIC - Шварца критерий. См. BIC
SCLS estimator. See symmetrically censored least squares
score test, see Lagrange multiplier test
score vector, 141
secondary sampling units (SSUs), 41, 815, 854 seed, 411
seemingly unrelated regressions (SUR) model, - система одновременных уравнений
Bayesian MCMC example, - Байесовский MCMC пример
count data, - счетные данные
error components, - состовляющие ошибки
nonlinear, - нелинейная
selection bias, - смещение самоотбора
nonignorable missingness, 927, 932, 940 
treatment effects models, - модели эффективности воздействия
see also selection models - см. модели самоотбора
selection models, - модели самоотбора
bivariate sample selection model, - двумерная модель самоотбора
count models, - счетные модели
example, - пример
panel data, - панельные данные
Roy model, - Роя модель
sample selection, - самоотбор выборки
self selection, - самоотбор
semiparametric estimation, - полупараметрическое оценивание
structural models, - структурные модели
treatment effects model, - модель эффективности воздействия
versus selection on observables only, 552–3, 864,
868–71
versus two-part models, 546, 552–3 
see also Tobit models - см. также тобит модели
selection on observables only, 552–3, 862–4, 868–9, 878–3, 889–96
compared to selection models, - сравнение с моделями самоотбора
conditional independence assumption, - предпосылка об условной независимости
control function estimator, - оценка контрольных функций
definition, - определение
DID estimator, 878–9
RD design estimator, 879–83
treatment effects model, 862–4, 889–96
selection on unobservables, 552–3, 865–71, 883–9 definition, 868
in treatment effects model, 862–4
IV estimators, - оценки инструментальных переменных
Roy model, - Роя модель
selection bias, - смещение самоотбора
selection model, - модель самоотбора
self-weighting sample, 818
SEM. See simultaneous equations model seminonparametric ML estimator, 328–9, 485 semiparametric efficiency bounds, 323, 329–30, 485 
semiparametric estimators, - полупараметрические оценки
adaptive, - адаптивные
application, - приложения
average derivative estimator, 326
efficiency bounds, - границы эффективности
nonparametric FGLS, - непараметрический доступный обобщенный МНК
Robinson difference estimator, 324–5, 565 
semiparametric least squares, - полупараметрический МНК
seminonparametric ML estimator, 328–9, 485 
see also semiparametric models - см. также полупараметрические модели
semiparametric heterogeneity model, - полупараметрическая модель для неоднородных данных
see also finite mixture models - см. также модели смеси распределений
semiparametric least squares, - полупараметрический МНК
semiparametric ML estimator, - полупараметрическая оценка максимального правдоподобия
semiparametric models, - полупараметрические модели
additive models, - аддитивные модели
binary outcome models, - модель бинарного выбора
censored models, - цензурированные модели
count models, - счетные модели
definition, - определение
duration models, - модели времени жизни
flexible parametric models, - гибкие параметрические модели 
heteroskedastic linear model, - гетероскедастичная линейная модель
identification, - идентифицируемость
leading examples, - основные примеры
multinomial outcome models, 523–4 panel data models, 808
partially linear model, - частично линейная модель
selection models, - модели самоотбора
single-index models, - одноиндексные модели
see also semiparametric estimators - см. также полупараметрические оценки
sequential limits, 767
sequential multinomial models, 520–1 
sequential two-step m-estimator, - двухшаговая М-оценка
bootstrap for, - бутстрэп
sequence of random variables, - последовательность случайных величин
serial correlation. See autocorrelation - См. автокорреляция
set identification, 
series estimator, 321
for binary outcomes, - для моделей бинарного выбора
shared frailty model, 662 
short panel - короткая панель
definition, - определение
statistical inference in, - статистические выводы 
shrinkage estimator, 440
Silverman’s plug-in estimate, 304
simple random sampling (SRS), 41, 76–7, 816
simple stratified sampling, - простая стратифицированная выборка
Simpson’s rule, - правило Симпсона (правило парабол)
simulated annealing (SA) method, - метод симуляции отжига
simulated m-estimator, - симуляционная М-оценка
simulation-based estimation methods, - методы основанные на симуляциях
motivating examples, - мотивационные примеры
see MSL, MSM, indirect inference, simulators simulators, 393–4, 406–10
antithetic sampling, 408–9 
direct, 393
frequency, 406
GHK, 407–8
Halton sequences, 409–10 importance sampling, 407 smooth, 407 subsimulator, 394 unbiased, 394, 400
see also quadrature
simultaneous equations model (SEM), 22–31, 213–4,
219
causal interpretation, 26
error components, 762
extension to nonlinear models, 27 FIML estimator, 214
identification, 29–31, 213–4 LIML estimator, 214 nonlinear, 219
order condition, 213
rank condition, 214 reduced form, 25, 213 single-equation models, 31 structural form, 25, 213 structural model, 24
2SLS estimator, 214
3SLS estimator, 214
simultaneous equations probit, 523, 560–1 simultaneous equations Tobit, 560–1 single-index models, 123, 323, 325–7
definition, 123
identification, 325
marginal effects, 123
nonlinear panel model, 780 semiparametric estimators, 325–7
SIPP. See Survey of Income and Program Participation size of test, 246–7, 251–3
nominal size, 251 size-corrected test, 251 true size, 251–3
Sklar’s theorem, 652 Slutsky’s Theorem, 945–6 alternative version, 949
small-sample bias. See finite-sample bias smooth maximum score estimator, 484 
smoothing parameters, 307
smoothing spline estimator, 321
social experiments, - социальные эксперименты
advantages, - преимущества
examples, - примеры
limitations, - ограничения
randomization, - рандомизация
span, 320
specific to general test, 285 
specification tests, - тесты на спецификацию
for clustered data, - для кластеризованных данных
for duration models, - для моделей времени жизни
for endogeneity, - на эндогенность
for exogeneity, - на экзогенность
for heteroskedasticity, - на гетероскедастичность
for individual-specific effects, - на индивидуальные эффекты
for omitted variables, - на пропущенные переменные
for overdispersion, - на избыточную дисперсию
for pooling, - 
for unobserved heterogeneity, - на ненаблюдаемую гетероскедастичность
for Tobit model, - для тобит модели
see also m-tests; model diagnostics - см. также М-тесты, диагностика модели
spherical errors, - сферические ошибки
split-sample IV estimator, 191–2
SRS. See simple random sampling - простая случайная выборка
SSUs. See secondary sampling units
stable family of distributions, 621
stable unit treatment value assumption (SUTVA), 872 standard errors. See robust standard errors
starting values, - стартовые значения
state dependence. See true state dependence stated preference data, 498, 516
stationary population, 40
statistical packages, - статистические пакеты
step size adjustment, - корректировка величины шага
stochastic order of magnitude, - стохастический порядок малости
stock sampling, 44, 626–7
strata, 41, 815
see also sampling schemes; 
weighting stratification matching, - взвешенное стратифицированное сопоставление
stratified random sampling, - стратифицированная случайная выборка
use of Liapounov CLT, - использование ЦПТ Ляпунова
use of Markov LLN, - использование ЗБЧ Маркова
see also sampling schemes; weighting - см. также план выборки
strict exogeneity. See strong exogeneity  - строгая экзогенность. См. сильная экзогенность
strong consistency, - сильная состоятельность
strong exogeneity, - сильная экзогенность
in panel models, - в панельных моделях 
structural approach - структурный подход 
to measurement error, - к ошибке измерения
to weighting, - к взвешиванию
structural economic models, - структурные экономические модели
with selection, 558–60 
structural form, - структурная форма
structural model, - структурная модель 
based on economic model, - основанная на экономической модели
exogeneity, - экзогенность
full information, - полная информация
limited information, - ограниченная информация
reduced form, - приведенная форма 
structural form, - структурная форма
structure, - структура
see also simultaneous equations model - см. также системы одновременных уравнений
structural selection models, - структурные модели самоотбора
based on utility maximization, - основанные на максимизации полезности
endogenous regressors, - эндогенные регрессоры
simultaneous equations Tobit, - система одновременных уравнений в тобит-модели
studentized statistic, 359 subsampling method, 373 substitution bias, 53, 867
sufficient statistic, - достаточная статистика
definition, - определение
summation assumption, 748, 752 
superpopulation, - суперсовокупность
supersmoother, 321
SUR model. See seemingly unrelated regressions survey methods, - См. методы для внешне несвязанных уравнений
survey nonresponse, - отсутствие ответа на опрос
see also attrition bias; imputation methods - см. также смещение истощения, методы восстановления данных
Survey of Income and Program Participation (SIPP),
59
survival analysis. See duration models survival function. See survivor function survivor function
aggregate survivor function, 619 
definition, - определение
estimator in PH model, 596–7
Kaplan-Meier estimator, - Каплан-Мейера оценка
in mixture models, - в моделях смеси
multivariate, - многомерная
parametric examples, - параметрические примеры
SUTVA. See stable unit treatment value assumption switching regressions model. See Roy model symmetrically censored least squares (SCLS)
estimator, - оценка
synthetic panels. See pseudo panels systems of equations, 206–19
linear systems, - линейные системы
nonlinear systems, - нелинейные системы
seemingly unrelated regression, - внешне несвязанные уравнения
simultaneous equations model, - системы одновременных уравнений
systems-robust standard errors, - робастные стандартные ошибки для систем одновременных уравнений
target density, - целевое распределение
tests. See hypothesis tests, m-tests, specification tests - тесты. См. тесты гипотез, М-тесты, тесты на спецификацию
three-stage least squares (3SLS) estimator, - трехшаговая МНК оценка
3SLS estimator. See three-stage least squares - См. трехшаговая МНК оценка
time series data - временные ряды
bootstrap, - бутстрэп
NLS estimator, - оценка нелинейного МНК
Newey-West standard errors, - стандартные ошибки в форме Ньюи-Веста
time-varying regressors - регрессоры, меняющиеся во времени
in duration models, - в моделях времени жизни
in panel data models, - в моделях панельных данных
Tobit model, - тобит модель
Bayesian methods, - Байесовские методы 
censored mean, - цензурированное среднее
censoring mechanism, - механизм цензурирования
consistency of MLE, - состоятельность оценок максимального правдоподобия
definition, - определение
example, - пример
generalized, - обобщенная
Heckman two-step estimator, - двухшаговая оценка Хекмана 
identification, - идентифицируемость
as imputation method, 932
inverse-Mills ratio, - обратное отношение Миллса
marginal effects, - предельные эффекты
measurement error in dependent variable, - ошибки измерения зависимой переменной
ML estimator, - оценка максимального правдоподобия
NLS estimator, - оценка нелинейного МНК
OLS estimator, - МНК оценка
panel data, - панельные данные
simultaneous equations, - системы одновременных уравнений
specification tests, - тесты на спецификацию модели
with stochastic thresholds, - со стохастическими порогами
with truncated data, - с усеченными данными
truncated mean, - усеченное средние
two-limit, - с двумя границами
type 2, - 2-го типа
type 5, - 5-го типа
see also selection models - см. также модели самоотбора
top-coded data, 532–3, 541, 563 
transformation methods, - методы преобразования
transformation theorem, - теорема о преобразовании
transformed ML estimator, - преобразованная оценка метода максимального правдоподобия

transition data. See duration models 
trapezoidal rule, 388 treatment-control comparison
application, - приложения
treatment effects framework, 862–5, 871–8, 889–96
balancing condition, 864, 893–4
binary treatment variable, - бинарная переменная воздействия
conditional independence assumption, 863, 865 conditional mean independence assumption, 864 heterogeneous treatment effects, 882, 885 
multiple treatments, - множественные воздействи
overlap assumption, 864, 871
propensity score, 864–5
Roy model, - Роя модель
stable unit treatment value assumption, 872
see also treatment evaluation - см. также оценка воздействия
treatment evaluation, - оценка воздействия
application, - приложения
IV estimators, - оценка инструментальных переменных
matching estimators, 871–8
DID estimators, 878–9
selection bias, - смещение самоотбора
selection on observables, 862–4, 878–3, 889–96 selection on unobservables, 865–71, 883–9 regression discontinuity design, 879–83
see also treatment effects framework treatment group, 49, 862
trimming, 316, 333
trivariate reduction, 686
true state dependence
duration models, - модели времени жизни
dynamic panel models, - динамические панельные модели
see also unobserved heterogeneity - см. также ненаблюдаемая неоднородность
truncated models, - модели усеченных переменных
conditional mean, - условное среднее
count models, - счетные модели
definition, - определение
examples, - примеры
ML estimator, - ML оценка
see also Tobit model; selection models truncated moments of standard normal, 540, 566–7 truncation mechanisms, - механизмы усечения
truncation from above, - усечение сверху
truncation from below, - усечение снизу
2SLS estimator. See two-stage least squares - См. двухшаговая МНК оценка
two-limit Tobit model, - тобит модель с двумя границами
two-part model, - двухчастная модель
application, - приложения
compared to selection models, - сравнение с моделями самоотбора
definition, - определение
example, - пример
see also hurdle model - см. также модель преодоления порогов
two-stage IV estimator, - двухшаговая оценка инструментальных переменных
two-stage least squares (2SLS) estimator, - двухшаговая оценка метода наименьших квадратов
alternatives to, - альтернативы
Basmann’s approach, - подход Басманна (интерпретация Басманна)
compared to optimal GMM, - сравнение с оптимальным обобщенным методом моментов
as GLS in transformed model, 
188–9 
as GMM estimator, - как оценка обобщенного метода моментов
nonlinear, - нелинейная
panel data, - панельные данные
in SEM, - в системах одновременных уравнений
Theil’s interpretation, - интерпретация Тейла 
two-stage sampling, - 
two-step estimators - двухшаговые оценки
GMM, - обобщенный метод моментов, ОММ
Heckman, - Хекман 
sequential m-estimator, - последовательная М-оценка
two-step GMM estimator, - двухшаговая оценка обобщенного метода моментов 
panel, - панельные данные
two-way effects model, - модель с двусторонними эффектами
type I error, - ошибка I типа
type II error, - ошибка II типа
type 1 extreme value distribution, - распределение экстремальных значений 1-го типа
duration model error, - ошибка в модели времени жизни
multinomial logit model, - мультиномиальная логит-модель
type 2 Tobit. See bivariate sample selection model - тобит модель 2-го типа. См. двумерная модель самоотбора
type 5 Tobit. See Roy model - тобит модель 5-го типа. См. модель Роя
ultimate sampling units (USUs), 41, 815 
unbalanced panels, - несбалансированные панели
uncentered explained sum of squares (ESS), - нецентрированная объясненная сумма квадратов 
uncentered R-squared, - нецентрированный R-квадрат
unconfoundedness assumption. See conditional independence assumption - 
underrecording, 915
undersmoothing, - недосглаживание
uniform convergence in probability, - равномерная сходимость по вероятности
uniform number generators, - генератор равномерно распределенных случайных чисел
uniformly most powerful (UMP) test, - равномерно наиболее мощный тест
unit roots, - единичные корни
universal logit model, - универсальная логит-модель
unobserved heterogeneity - ненаблюдаемая неоднородность
application, - приложения
in competing risks model, 647
in count models, - в счетных моделях
distributions for, 614–5, 620–1
in duration models, - в моделях времени жизни
finite mixture models for, - моделирование с помощью моделей смеси распределений
identification, - идентифицируемость
IM test for, 267
individual-specific effects, - индивидуальные эффекты
mixture models for, 613–21
MSL example, 397–8
MSM example, 403
multiplicative, - мультипликативная
in nonlinear systems, - в нелинейных системах
specification tests for, 629–32
variance inflation, - вздутие дисперсии
versus true state dependence, 612, 630, 636, 763–4,
798, 802
USUs. See ultimate sampling units
validation sample, - тестовая выборка
variance components, 735, 845
variance matrix estimation - оценка ковариационной матрицы
BHHH estimate, - BHHH оценка
degrees-of-freedom adjustment, - поправка на количество степеней свободы
expected Hessian estimate, 138 for extremum estimator, 137–9 for GMM estimator, 174–5 Hessian estimate, 138
for NLS estimator, - для оценки нелинейного МНК
OPG estimate, 138
robust estimate, - робастная оценка
sandwich estimate, 137, 144 for weighted estimators, 854–6 see also robust standard errors
variance reduction for simulation, 478
Wald estimator - Вальда оценка
in treatment effects models, 886
Wald test, - Вальда тест
asymptotic distribution, - асимптотическое распределение
comparison with LM and, LR tests, - сравнение с LM тестом (тестом множителей Лагранжа) и LR тестом (тестом отношения правдоподобия)
definition, - определение
examples, - примеры
exclusion restrictions, 227
F-test version, - версия с помощью F-теста
introduction, - введение
lack of invariance, - отсутствие инвариантности
likelihood based, - основанный на методе максимального правдоподобия
linear models, - линейные модели
linear restrictions, - линейные ограничения
in misspecified models, - в неправильно специфицированных моделях
nonlinear restrictions, - нелинейные ограничения
power, - мощность
of statistical significance, - на значимость
t-test version, - версия с помощью t-теста
see also hypothesis tests - см. также тестирование гипотез
weak consistency, - слабая состоятельность
weak exogeneity, - слабая экзогенность
in panel data, - в панельных данных
weak instruments, - слабые инструменты
application, - приложения
definition, - определение
finite sample bias, - смещение малых выборок 
GMM estimator, - ОММ оценка (оценка обобщенного метода моментов)
inconsistency, - несостоятельность
indicators - признаки
panel data, - панельные данные
Weibull distribution, - Вейбулла распределение
Weibull-gamma regression model, - Вейбулла-гамма регрессионная модель
Weibull regression model, - Вейбулла регрессионная модель
weighted estimation - взвешенное оценивание
endogenous stratification, - эндогенная стратификация
exogenous stratification, - экзогенная стратификация
weighted exogenous sampling ML (WESML) estimator, 828
weighted least squares (WLS) estimator, 81–5 asymptotic distribution, 83
contrasted with GLS, - сравнение с обобщенным МНК
definition, - определение
example, - пример
in pooled model, - в сквозной модели 
see also FGLS estimator - см. также доступная оценка обобщенного МНК
weighted maximum likelihood (WML) estimator, - взвешенная оценка метода максимального правдоподобия
weighted semiparametric least squares (WSWL) estimator, - взвешенная оценка полупараметрического МНК
for binary outcome models, - для моделей бинарного выбора
weighting, - взвешивание
descriptive versus structural approach, - сравнение описательного и структурного подходов
with endogenous stratification, - с эндогенной стратификацией
sample weights, - выборочные веса
variance estimation, - оценка дисперсия
weighted prediction, - взвешенное прогнозирование 
weighted regression, - взвешенная регрессия 
whether to weight, - необходимость взвешивания
welfare analysis - анализ благосостояния
with ARUM, - с помощью модели ARUM
with nested logit model, - с помощью вложенных логит-моделей
WESML estimator. See weighted exogenous sampling ML
White standard errors. See robust standard errors - Уайта стандартные ошибки. См. робастные стандартные ошибки
wild bootstrap, 377–8
window width, - ширина окна сглаживания
Wishart distribution, - Уишарта распределение
see also inverse-Wishart distribution - см. также обратное распределение Уишарта
within estimator. See fixed effects estimator
within model. See fixed effects model
within-group variation, - внутригрупповая диспрерсия
with-zeros model, 681
WLS estimator. See weighted least squares
WML estimator. See weighted maximum likelihood WNLS estimator, 156–7
asymptotic distribution, - асимптотическое распределение
definition, - определение
example, - пример
as GLM, 158
working matrix - рабочая матрица
definition, - определение
for GLM estimator, 158
for pooled GEE estimator, 794 
for pooled WLS estimator, 721 
for WLS estimator, 82–3
WSLS estimator. See weighted semiparametric least squares

zero-inflated count model, - модель счетных данных с раздутым нулём


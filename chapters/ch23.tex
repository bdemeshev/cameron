
\chapter{Нелинейные модели панельных данных} 

\subsection{Введение}
Эта глава является продолжением глав 21 и 22, посвященных линейным моделям панельных данных. Здесь рассматриваются нелинейные модели регрессии, представленные в главах 14 --- 20. Во внимание берутся короткие панели и модели с фиксированными или случайными индивидуальными эффектами, не меняющимися во времени. Рассматриваются и статическая, и динамические модели.

Не существует какого-то единого предписания для нелинейных моделей с индивидуальными эффектами. Если используются фиксированные индивидуальные эффекты, а панель короткая, то получение состоятельных оценок для параметров наклона возможно только для некоторых нелинейных моделей. Если же используются случайные индивидуальные эффекты, то состоятельные оценки можно получить уже для более широкого ряда моделей.

В разделе 23.2 представлены общие подходы, которые могут или не могут быть применены для конкретных моделей. В разделе 23.3 описан пример применения нелинейных моделей с мультипликативными индивидуальными эффектами. Особенности основных классов нелинейных моделей --- моделей дискретных данных, моделей выбора, моделей с данными о переходах и счетными данными представлены в разделах 23.4-23.7. Полупараметрические модели оценивания исследуются в разделе 23.8.

\section{Общие результаты}

В этом разделе представлены общие подходы расширения методов линейных моделей. Во-первых, мы представляем различные модели --- с фиксированными и случайными эффектами, модели сквозной регрессии, проводя различие между параметрическими  моделями и моделями условного среднего. Затем представлены методы оценки этих моделей и способы получения робастных стандартных ошибок для панельных данных. Дальнейшие подробности спецификаций нелинейных моделей панельных данных даны в следующих разделах.

\subsection{Модели с индивидуальными эффектами}

В линейной модели с индивидуальными эффектами (см. раздел 21.2.1) зависимая переменная $y_{it}$ зависит от индивидуальных эффектов $\alpha_i$, обычных регрессоров $\x_{it}$ и параметров регрессии $\be$. Модель записывается как $y_{it}=\alpha_i+\x'_{it}\be + u_{it}$, где $u_{it}$ --- ошибка.

Для нелинейных моделей, таких как логит-модели или модели Пуассона, аддитивная ошибка $u_{it}$ не вводится. Вместо этого, более естественно напрямую моделировать условную плотность или условное среднее. В линейном случае это выражается как $\E[y_{it}|\alpha_i, \x_{it}]=\alpha_i+\x'_{it}\be$.

{\centering Параметрические модели \\
}

Общим для многих нелинейных моделей является полностью непараметрический подход. Особенно это касается моделей бинарного и множественного выбора, а также цензурированных выборок, описанных в главах 14-16.

Стандартные модели для пространственных данных  --- одноиндексные модели, или одноиндексные модели с дополнительными коэффициентами масштаба. В \textbf{параметрических моделях с индивидуальными эффектами}, представленных в последующих разделах, специфицируется условная плотность
\begin{align}
f(y_{it}|\alpha_i, \x_{it})=f(y_{it}, \alpha_i+\x'_{it}\be, \bm\gamma)
\label{Eq:23.1}
\end{align}
где $\bm\gamma$  обозначает дополнительные параметры, например, дисперсию. Эта модель называется одноиндексной моделью с регрессорами $\x_{it}$  и индивидуальными эффектами $\alpha_i$.

Обычно предполагается, что $y_{it}|\x_{it}, \alpha_i$ независимы по $i$ и по $t$.  Предположение может быть ослаблено. Возможна зависимость по $t$ для данного $i$ (см. раздел 23.2.6).

 {\centering Модели условного среднего \\}

Общая нелинейная модель для условного среднего с ненаблюдаемыми не меняющимися во времени индивидуальными эффектами:
\begin{align}
&\E[y_{it}|\alpha_i, \x_{it}]=\mathrm g(\alpha_i, \x_{it}, \be),
& i=1, \dots, N &
&t=1,  \dots, T,
\label{Eq:23.2}
\end{align}
для заданной функции $\mathrm g(\cdot)$. Три распространенных спецификации  --- это \textbf{модель с аддитивными индивидуальными эффектами}
\begin{align}
\mathrm g(\alpha_i, \x_{it}, \be)=\alpha_i+ \mathrm g(\x_{it}, \be),
\label{Eq:23.3}
\end{align}
\textbf{модель с мультипликативными индивидуальными эффектами}
\begin{align}
\mathrm g(\alpha_i, \x_{it}, \be)=\alpha_i \mathrm g(\x_{it}, \be),
\label{Eq:23.4}
\end{align}
и \textbf{одноиндексная модель с индивидуальными эффектами}
\begin{align}
\mathrm g(\alpha_i, \x_{it}, \be)= \mathrm g(\alpha_i+\x_{it}, \be).
\label{Eq:23.5}
\end{align}
Функция $\mathrm g(\cdot)$ специфицирована в каждом случае. Регрессоры $\x_{it}$ могут быть изменяющимися или неизменными во времени, а могут включать временную дамми переменную.

Модель с аддитивными эффектами подходит для случаев, когда значения $y_{it}$  не ограничены, как это неявно предполагалось в случае линейной регрессии. Модель с мультипликативными эффектами подходит для случаев, когда $y_{it}$ не принимает отрицательные значения. Например, это счетные данные, для которых $\alpha_i>0$ и  $\mathrm g(\cdot)>0$. Одноиндексная модель --- это естественное начало для пробит-модели, например, с $\mathrm g(\alpha_i+\x'_{it})=\Phi(\ali+\x'_{it}\be)$, где $\Phi(\cdot)$ --- это функция стандартного нормального распределения. Одноиндексная модель сводится к аддитивной модели, если $\mathrm g(\cdot)$ --- это тождественное отображение. 
Она сводится к мультипликативной модели, если $\mathrm g(\cdot)$  --- экспоненциальная функция. В таком случае $\exp (\ali+\x'_{it}\be)=\exp (\ali)\exp (\x'_{it}\be)$.

Моментное тождество \ref{Eq:23.2} обуславливается только текущим периодом $\x_{it}$ и предполагает, что регрессоры \textbf{одновременно экзогенны} (см. раздел 22.2.4). Для устранения индивидуальных эффектов $\alpha_i$ могут потребоваться более строгие предположения об экзогенности. Регрессоры \textbf{слабо экзогенны}, если 
\begin{align}
\E[y_{it}|\ali, \x_{i1}, \dots, \x_{it}]=\mathrm g(\ali, \x_{it},\be)
\label{Eq:23.6}
\end{align}
и \textbf{сильно экзогенны} или \textbf{строго экзогенны}, если
\begin{align}
\E[y_{it}|\ali, \x_{i1}, \dots, \x_{iT}]=\mathrm g(\ali, \x_{it},\be).
\label{Eq:23.7}
\end{align}

Нелинейная модель с аддитивными эффектами усложняется относительно незначительно. В частности, если модель панельных данных имеет вид $y_{it}=\ali+\mathrm g(\x_{it},\be)+u_{it}$, то подходы глав 21 и 22 должны претерпеть некоторые изменения, включая использование нелинейных МНК оценок и нелинейных оценок инструментальных переменных вместо линейных.

В этой главе внимание акцентируется на моделях с неаддитивными индивидуальными эффектами, такими как в моделях \ref{Eq:23.4} и \ref{Eq:23.5}. Эти эффекты могут рассматриваться как фиксированные или как случайные эффекты.

\subsection{Модели с фиксированными эффектами}

В \textbf{модели с фиксированными эффектами} индивидуальные эффекты $\ali$ фигурируют в качестве ненаблюдаемых случайных переменных, которые могут быть коррелированы с регрессорами $\x_{it}$. В коротких панелях совместное оценивание фиксированных эффектов $\alpha_i, \dots, \alpha_N$ и других параметров модели $\be$ и, возможно, $\bm\gamma$, вообще приводит к несостоятельным оценкам всех параметров. Вместо этого, было предложено множество методов для устранения фиксированных эффектов в некоторых специальных постановках модели, которые позволяют состоятельно оценивать другие параметры модели.

 {\centering Проблема второстепенных параметров\\}
 
Нейман и Скотт (1948) рассматривали статистические выводы, когда некоторые параметры являются общими для всех наблюдений, но в то же время имеется бесконечное число параметров, каждый из которых зависит от конечного количества наблюдений. Нас интересуют \textbf{общие параметры}. Последние же параметры называются \textbf{второстепенными параметрами}.

Здесь $\be$ и $\bm\gamma$ --- общие параметры, а $\alpha_1, \dots, \alpha_N$ --- второстепенные параметры. Это соответствует короткой панели, когда каждый $\ali$  зависит от фиксированного количества $T$ наблюдений и есть бесконечно много $\ali$, так как $N \rightarrow \infty$. Оценки второстепенных параметров несостоятельны при $N \rightarrow \infty$, так как используется только $T$ наблюдений для оценки каждого параметра. \textbf{Проблема второстепенных параметров} состоит в том, что это загрязняет оценивание общих параметров. В общем случае и оценки общих параметров также несостоятельны, даже если их количество ограничено, и их оценивание производится с помощью $NT \rightarrow \infty$ наблюдений.

Просто проиллюстрировать загрязнение второстепенными параметрами можно, предполагая, что $y_{it} \thicksim \mathcal{N}[\ali, \sigma^2]$. Оценка ММП дает $\hat{\ali}=\bar{y_i}, i=1, \dots, N$ и $\hat{\sigma}^2=(NT)^{-1}\sum_i \sum_t (y_{it}-\bar{y}_i)^2$. Тогда $\E[\hat{\sigma}^2]=\sigma^2(T-1)/T$, т.е. $\hat{\sigma}^2$ --- несостоятельная оценка для $\sigma^2$ при  $N \rightarrow \infty$ в коротких панелях с фиксированным $T$. Несостоятельность может быть значительной. Например, $\hat{\sigma}^2 \xrightarrow{\text{p}} 0.5 \sigma^2$, когда $T=2$.

В общем случае при возникновении проблемы второстепенных параметров, необходимы альтернативные методы оценивания, которые на первом шаге позволяют избавиться от второстепенных параметров. Для некоторых известных моделей, особенно пробит-модели панельных данных, существует решение проблемы второстепенных параметров. Даже когда  существуют методы для получения состоятельных оценок $\be$, эти методы, скорее, будут более специфическими, как отметил Ланкастер (2000). Не существует единого решения проблемы второстепенных параметров.

 {\centering Условная функция правдоподобия \\}

Статистика $t$  называется \textbf{достаточной} для параметра $\theta$, если распределение выборки при данном $t$ не зависит от $\theta$. В моделях панельных данных с индивидуальными эффектами, если существует достаточная статистика для вспомогательного параметра $\ali$, то вспомогательный параметр уничтожается, когда мы включаем в условие эту достаточную статистику. Результирующая условная плотность зависит от общих параметров, что позволяет их состоятельно оценивать.

Пусть $\mathbf{y}_i=[y_{i1}, \dots, y_{iT}]'$ --- это вектор зависимой переменной для индивидуального наблюдения $i$ размерности $T \times 1$ по всем $T$ периодам, и пусть $\mathbf{X}_i=[\x_{it}, \dots, \x_{iT}]'$ обозначает соответствующую матрицу регрессоров размерности $T \times K$. Для статической модели плотность $\mathbf{y}_i$ равна
\begin{align}
f(\mathbf{y}_i|\mathbf{X}_i, \ali, \be, \gamma)= \prod^T_{t=1} f(y_{it}|\x_{it}, \ali, \be, \gamma).
\label{Eq:23.8}
\end{align}

Оценивание методом максимального правдоподобия, основанное на этой плотности в общем случае приводит к несостоятельному оцениванию $\be$ в коротких панелях из-за проблемы второстепенных параметров.

Предположим, что существует \textbf{достаточная статистика} $\mathbf{s}_i$ для $\alpha_i$. Тогда использование ее в качестве условия в дополнение к обычным регрессорам дает \textbf{условную плотность}
\begin{align}
f(\mathbf{y}_i|\mathbf{X}_i, \ali, \be, \bm\gamma,\mathbf{s}_i)= f(\mathbf{y}_{it}|\mathbf{X}_{i}, \be, \bm\gamma, \mathbf{s}_i),
\label{Eq:23.9}
\end{align}
где исчезает $\ali$. Например, для линейной регрессионной модели в предположении нормальности $\mathbf{s}_i=\bar{y}_i$ (см. раздел 21.6.3). Тогда \textbf{оценка условного МП} максимизирует условную функцию правдоподобия
\begin{align}
\mathrm{ln L_{COND}}(\be,\bm\gamma)= \sum^N_{i=1} ln f(\mathbf{y}_{it}|\mathbf{X}_{i}, \be, \bm\gamma, \mathbf{s}_i).
\label{Eq:23.10}
\end{align}
Прилагательное <<условная>> добавлено для обозначения того, что берется функция  не только при условии$\mathbf{X}_i$, но и $\mathbf{s}_i$.

Андерсен (1970) проделал детальный анализ оценки условного ММП. Он показал, что оценка условного ММП будет состоятельна, если плотность $f(\mathbf{y}_i|\mathbf{X}_i, \ali, \be)$  правильно специфицирована, что информационно матричное равенство выполняется для условной функции правдоподобия. Однако вообще происходит потеря эффективности, так как оценка условного ММП необязательно достигает  нижнюю границу неравенства Рао-Крамера. Для нормального распределения и распределения Пуассона, однако, потеря эффективности не происходит.

Для данного подхода требуется, чтобы существовала подходящая достаточная статистика. Это выполняется только для нескольких моделей, в основном для моделей из семейства линейных экспоненциальных распределений. Андерсен останавливал свое внимание на моделях без регрессоров, используя в качестве примеров нормальное распределение, распределение Пуассона, биномиальное и гамма распределения. С регрессорами найти подходящую достаточную статистику становится еще сложнее. Общее обсуждение данного подхода можно найти в работе МакКуллах и Нелдер (1989). Диггл и др. (2002) в своей работе уделяли внимание специализированным обобщенным линейным моделям с каноническими связующими функциями. 

Основные примеры моделей, когда достаточные статистики доступны, --- линейные модели в условиях нормальности (см. раздел 21.6.2), логит-модели (но не пробит-модели) для бинарных данных (см. раздел 23.4.3), модели распределения с одним параметром гамма (включая экспоненциальное) и особые параметризации Пуассона и отрицательные биномиальные модели для счетных данных (см. раздел 23.7.3).

{\centering Преобразование <<отклонение от среднего>> \\}

Для некоторых моделей условных средних с аддитивными или мультипликативными эффектами индивидуальные эффекты $\ali$ могут быть устранены с помощью подходящего взятия разностей. Это позволяет получить моментные тождества, которые можно использовать для метода моментов или ОММ оценивания, что подробно представлено в разделе 23.2.6.

\textbf{Преобразование <<отклонение от среднего>>} --- это обобщение преобразования within для линейной модели, описанного в разделе 21.2.2, которое позволяет избавиться от $\ali$ посредством вычитания индивидуальных \textbf{средних}. Для этого преобразования необходимы строго экзогенные регрессоры, см. \ref{Eq:23.7}.

Для модели с аддитивными эффектами \ref{Eq:23.3} со строго экзогенными регрессорами
\begin{align}
\E[(y_{it}-\bar{y}_i)-(\mathrm g(\x'_{it}\be)-\bar{g}_i(\be))|\x_{i1},\dots,\x_{iT}]=0,
\label{Eq:23.11}
\end{align}
где $\bar{g}_i(\be)=T^{-1}\sum^T_{t=1}\mathrm g(\x'_{it}\be)$, и для получения результата используется $\E[\bar{y}_i|\x_{i1}, \dots, \x_{iT}]=\ali+\bar{g}_i(\be)$. Для линейных моделей \ref{Eq:23.11} значительно упрощается, так как в случае линейности $\mathrm g(\x'_{it}\be)-\bar{g}_i(\be)=(\x_{it}-\bar{x}_i)'\be$.

Для модели с мультипликативными эффектами \ref{Eq:23.4} алгебраические преобразования приводят к 
\begin{align}
\E\left[y_{it}-\frac{\mathrm g(\x'_{it}\be)}{\bar{g}_i(\be)}\times \bar{y}_i|\x_{i1},\dots, \x_{iT} \right]=0.
\label{Eq:23.12}
\end{align}
В ходе преобразований используется тот факт, что $\E[\bar{y}_i|\x_{i1}, \dots, \x_{iT}]=\ali\bar{g}_i(\be)$. Для простоты мы называем это преобразованием \textbf{<<отклонение от среднего>>}, хотя строго говоря это \textbf{квази-разность}. Это преобразование также называется (условным) \textbf{преобразованием, масштабирующим по среднему}, так как оно эквивалентно
\begin{align}
\E \left[y_{it}-\frac{\bar{y}_i}{\bar{g}_i(\be)}\mathrm g(\x'_{it}\be)|\x_{i1}, \dots, \x_{iT} \right] = 0.
\nonumber
\end{align}

{\centering Преобразование <<первые разности>> \\}

\textbf{Преобразование <<первые разности>>} --- обобщение преобразования <<взятие первых разностей>> для линейной модели, представленной в разделе 21.2.2, который устраняет $\ali$  посредством вычитания первого лага. Мы предполагаем, что регрессоры слабо экзогенны (см. \ref{Eq:23.6}).

Для модели с аддитивными эффектами
\begin{align}
\E[(y_{it}-y_{i,t-1})-(\mathrm g(\x'_{it}\be)-\mathrm g(\x'_{i,t-1}\be))|\x_{i1},\dots,\x_{i,t-1}]=0,
\label{Eq:23.13}
\end{align}
где мы использовали $\E[y_{i,t-1}|\x_{i1}, \dots, \x_{i,t-1}]=\ali+\mathrm g(\x'_{i,t-1}\be)$.

Для моделей с мультипликативными эффектами \ref{Eq:23.4}
\begin{align}
\E\left[ y_{it}-\frac{\mathrm g(\x'_{it}\be)}{\mathrm g(\x'_{i,t-1}\be)}\times y_{i,t-1}|\x_{i1}, \dots, \x_{i,t-1} \right]=0,
\label{Eq:23.14}
\end{align}
где мы использовали $\E[y_{i,t-1}|\x_{i1}, \dots, \x_{i,t-1}]=\ali \mathrm g(\x'_{i,t-1}\be)$. Для простоты мы называем это \textbf{преобразованием <<первые разности>>}, хотя строго говоря, это \textbf{квази-разность}.

Преобразование <<первые разности>> основывается на более слабых предположениях, в качестве условий используются только периоды меньше $t$. Это позволяет оценивать динамические модели в том числе и нелинейные модели в продолжение моделей раздела 22.5. Для динамических мультипликативных эффектов Вулдридж (1997) и Чемберлин (1992) предложили использовать версию \ref{Eq:23.14}
\begin{align}
\E\left[ \frac{\mathrm g(\x'_{i,t-1}\be)}{\mathrm g(\x'_{it}\be)}y_{it}-y_{i,t-1}|\x_{i1}, \dots, \x_{i,t-1} \right]=0.
\label{Eq:23.15}
\end{align}

{\centering Оценивание модели с фиктивными переменными \\}

Если игнорировать проблему второстепенных параметров, можно попытаться оценить все параметры, включая индивидуальные эффекты. Возьмем набор $N$ дамми переменных $d_{j,it}$, равных 1, если $i=j$ и 0 иначе. Затем совместно оцениваем индивидуальные параметры $\alpha_1, \dots, \alpha_N$ наряду с другими параметрами модели.

Эта оценка доступна с вычислительной точки зрения, несмотря на большое число параметров из-за большого $N$. Но результирующие оценки $\be$ и, возможно, $\bm\gamma$ несостоятельны. Здесь мы рассматриваем только параметрические модели, хотя есть сходства с моделями условного среднего.

Итак, рассмотрим параметрическую модель с индивидуальными эффектами \ref{Eq:23.1}. Тогда способ получения оценок ММП $\be$, $\gamma$ и  $\bm\alpha=[\alpha_1 \dots \alpha_N]'$ --- максимизировать полную функцию правдоподобия в логарифмах
\begin{align}
\mathrm{ln L}_{FE}(\be,\bm\gamma,\alpha)=\sum^N_{i=1} \sum^T_{t=1} \mathrm{ln} f(y_{it}, \mathbf{d}'_{it} \bm\alpha + \x'_{it} \be, \bm\gamma),
\label{Eq:23.16}
\end{align}
где $\mathbf{d}_{it}=[d_{1,it} \dots, d_{N,it}]'$. Условия первого порядка для $\bm\delta = [\be' \; \bm\gamma]'$ и $\alpha$:
\begin{align}
\sum^N_{i=1} \sum^T_{t=1} \partial \mathrm{ln} f (y_{it}, \mathbf{d}'_{it}\bm\alpha + \x'_{it} \be, \bm\gamma) / \partial \delta = \mathbf{0}, 
\nonumber \\
\sum^T_{t=1} \partial \mathrm{ln} f (y_{it}, \ali + \x'_{it} \be, \bm\gamma) / \partial \ali = 0,&
& i=1, \dots, N.
\nonumber
\end{align}

Эту оценку просто посчитать несмотря на  большое количество параметров $N$ плюс размерность $\bm\delta$. Как подробно описано у Грина (2004b), обратить матрицу Гессе легко с помощью разбиения по $\bm\delta$  и $\bm\alpha$  и применения стандартной формулы обращения блочной матрицы. Используя тот факт, что $\partial \mathrm{ln L}(\bm\delta, \bm\alpha)/\partial\ali \partial\alpha_j=0$ для $j \neq i$,  получить обратную матрицу к блоку размерности $N \times N$, соответствующей $(\bm\alpha, \bm\alpha)$, довольно просто.


В двух особых случаях проблемы второстепенных параметров не возникает. Во-первых, если $y_{it} \thicksim \mathcal{N} [\ali + \x'_{it}\be, \sigma^2]$ тогда из раздела 21.6.4, оценка ММП для $\be$  --- оценка within, которая состоятельна для $\be$ даже для конечных $T$. Здесь проблема второстепенных параметров возникает при оценивании $\sigma^2$, но не $\be$. Во-вторых, для $y_{it} \thicksim  \mathcal{P} [\exp (\ali+\x'_{it}\be)]$ также нет проблемы второстепенных параметров при оценивании $\be$ (см. раздел 23.7.3).

Однако вообще проблема второстепенных параметров имеет место. Для взятия производной по $\ali$ необходимо только $T$, а не $NT$ наблюдений. Это обычно приводит к несостоятельности $\hat{\be}_{ML}$ и $\hat{\bm\gamma}_{ML}$ в коротких панелях. Возможно, несостоятельность умеренна в панелях средней длины с $T=10$ или $T=20$. С помощью симуляций Грин (2004a) показал, что природа и степень смещения сильно зависят от конкретной нелинейной модели. Развитие методов, робастных к присутствию фиксированных эффектов, хотя и несостоятельных для коротких панелей, является актуальной темой исследований.

\subsection{Модели со случайными эффектами}

В \textbf{модели со случайными эффектами} индивидуальные эффекты $\ali$ рассматриваются как случайные переменные с заданным распределением, и $\ali$ уничтожаются посредством взятия интеграла по этому распределению. Случайные эффекты обычно используют в случае параметрических моделей.

{\centering Параметрические модели \\}

Предположим $i$-е наблюдение $\mathbf{y}_i$ имеет безусловную совместную плотность $f(\mathbf{y}_i|\mathbf{X}_i, \ali, \be, \bm\gamma)$ \ref{Eq:23.8} и случайный эффект имеет плотность
\begin{align}
\ali \thicksim \mathrm g(\ali|\bm\eta),
\label{Eq:23.17}
\end{align}
где $\mathrm g(\ali|\bm\eta)$ не зависит от наблюдаемых переменных. Тогда безусловная совместная плотность для $i$-го наблюдения будет иметь вид
\begin{align}
f(\mathbf{y}_i|\mathbf{X}_i, \be, \bm\gamma, \bm\eta) = \int \left[ \prod^T_{t=1} f(y_{it} | \x_{it}, \ali, \be, \bm\gamma) \right] \mathrm g(\ali|\bm\eta)d \ali,
\label{Eq:23.18}
\end{align}
где под безусловным подразумевается то, что в условии больше не содержится $\ali$. \textbf{Оценки ММП случайных эффектов} $\be$, $\bm\gamma$ и $\bm\eta$ максимизируют логарифмическую функцию правдоподобия.
\begin{align}
\mathrm{ln L}_{\mathrm{RE}}(\be, \bm\gamma, \bm\eta)=\sum^N_{i=1} \mathrm{ln}\left( \int \left[ \prod^T_{t=1} f(y_{it}|\x_{it}, \ali, \be) \right] \mathrm g(\ali| \bm\gamma) d \ali \right).
\label{Eq:23.19}
\end{align}

В некоторых случаях этот интеграл можно выразить аналитически, в основном когда $\prod_t f(y_{it}|\ali)$ и $\mathrm g(\ali)$ --- сопряженные распределения (см. таблицу 23.2). Например, если оба распределения --- нормальные, то на выходе получается нормальное распределение. Или одно --- распределение Пуассона, а другое --- гамма-распределение для счетных данных, то на выходе получается отрицательное биномиальное распределение.

Во многих случаях аналитические результаты не доступны, но работают численные методы или методы, основанные на симуляциях, так как это однократный интеграл. Обычный подход состоит в том, чтобы для $f(y_{it})$ использовать плотность, наиболее адекватно описывающую данные  в отсутствие индивидуальных эффектов, а для $\mathrm g(\ali)$ --- нормальную плотность. Тогда мы получим однократный интеграл по случайной переменной, имеющей нормальное распределение. При малом $T$ для приближенного вычисления интеграла можно использовать метод Гаусса (см. раздел 12.3.1), которая приближает интеграл по нормальной плотности с помощью взвешенной суммы. Батлер и Моффитт (1982) подробно описывают пробит-модель со случайными эффектами. Скрондал и Рэйб-Хаскет (2004) используют метод Гаусса. В качестве альтернативы основой для оценивания методом симуляционного максимального правдоподобия могут быть  повторяющиеся выборки из $\mathrm g(\ali)$ (см. раздел 12.4.2).

В предыдущем обсуждении предполагалась независимость по $t$ для данного $i$. Если вместо этого $y_{it}$ и $y_{is}$ коррелированы по $i$, тогда более эффективно будет заменить $\prod_t f(y_{it}|\x_{it}, \ali, \be, \bm\gamma)$ на $ f(\mathbf{y}_i|\mathbf{X}_i, \ali, \be, \bm\gamma)$ в \ref{Eq:23.18} и \ref{Eq:23.19}.

{\centering Модель со случайными коэффициентами \\}

Модель со случайными эффектами может быть расширена до \textbf{модели со случайными коэффициентами}, т.е. со случайными коэффициентами наклона и свободным членом подобно линейному случаю раздела 22.8.

Естественная модель --- это одноиндексная модель с условной плотностью $f(y_{it}, \x'_{it}(\be+\ali),\bm\gamma)$ или условным средним $\mathrm g(y_{it}, \x'_{it}(\be+\ali))$ и однократный интеграл по скаляру $\ali$  станет многократным интегралом по вектору $\bm\alpha_i$. Обычно предполагается нормальное распределение.

{\centering Модель с коррелированными случайными эффектами \\}
  
Основной недостаток модели со случайными эффектами состоит в сильном предположении о независимых от регрессоров случайных коэффициентах.  Чемберлин (1980, 1982) решил ослабить это предположение и предложил \textbf{модель с коррелированными случайными эффектами}. Обсуждение см. в разделе 21.4.4, где $\ali$ специфицируется следующим образом
\begin{align}
\ali=\x'_{it} \bm\pi_1+ \dots + \x'_{Ti}\bm\pi_T+\xi_i.
\label{Eq:23.20}
\end{align}
Затем функция правдоподобия максимизируется по $\be, \bm\gamma, \bm\pi$, и параметрам плотности $\xi$. В отличие от линейных моделей эта модель  приводит к оценкам, отличным от тех, которые получаются с помощью спецификации Мундлака (1978), 
\begin{align}
\ali=\bar{\x}'_i\bm\pi+\xi_i.
\label{Eq:23.21}
\end{align}
Уравнение \ref{Eq:23.20} можно рассматривать как пример иерархической модели. В более общих иерархических моделях коэффициенты наклона могут быть случайными. Эти модели оцениваются классическими или байесовскими методами. В разделе 22.8 представлены подробности для линейной  модели.

{\centering Модель смеси распределений \\} 

Модель смеси распределений (см. раздел 18.5.1) --- это альтернативная модель для ненаблюдаемых индивидуальных эффектов. Если есть $m$ \textbf{латентных классов} или типов индивидуумов и для $j$-го типа $\ali=\alpha_j$ тогда \ref{Eq:23.18}
\begin{align}
f(\mathbf{y}_i|\mathbf{X}_i, \be, \bm\gamma, \pi) = \sum^m_{j=1} \left[\sum^T_{t=1}f(y_{it}|\x_{it}, \ali, \be, \bm\gamma) \right] \bm\pi_j.
\nonumber
\end{align}
Эта модель наиболее часто используется для моделей длительности состояний (см. раздел 18.5.2).

\subsection{Модели сквозной регрессии}

В модели сквозной регрессии индивидуальные эффекты не фигурируют явно. Сейчас расширим линейную сквозную регрессию (см. раздел 21.5) на нелинейный случай.

{\centering Модели условного среднего \\}

Для моделей условного среднего \textbf{модель сквозной регрессии} имеет вид
\begin{align}
\E[y_{it}|\x_{it}]=\mathrm g(\x_{it}, \be),
\label{Eq:23.22}
\end{align}
для специфицированной функции $\mathrm g(\cdot)$.

Модель \ref{Eq:23.22} можно оценить напрямую с помощью нелинейного МНК и использовать статистические выводы, основанные на робастных стандартных ошибках для панельных данных, которые учитывают возможную условную гетероскедастичность и условную корреляцию между $y_{it}$ и  $y_{is}$. Более эффективные оценки можно получить, только моделируя гетероскедастичность и корреляцию. Подробности см. в разделе 23.2.6.

{\centering Модели сквозной регрессии против модели со случайными эффектами \\}

Каковы издержки игнорирования индивидуальных случайных эффектов?

Модель с аддитивными эффектами $\E[y_{it}|\ali, \x_{it}]=\ali \mathrm g(\x_{it}, \be)$ сводится к \ref{Eq:23.22}, если $\E[\ali|\x_{it}]=0$. Модель с мультипликативными эффектами $\E[y_{it}|\ali, \x_{it}]=\ali \mathrm g(\x_{it}, \be)$ подразумевает, что   \ref{Eq:23.22}, если $\E[\ali|\x_{it}]=1$. Модель сквозной регрессии будет давать состоятельные оценки $\be$ в модели со случайными эффектами, если эффекты являются аддитивными или мультипликативными, и в модели использовалось стандартное нормирование  среднего $\ali$.

Иначе в модели сквозной регрессии вряд ли получатся те же оценки параметров, что и в модели со случайными эффектами. Например, рассмотрим пробит-модель со случайными эффектами, для которой $\E[y_{it}|\ali, \x_{it}]=\Phi(\ali+\x_{it}'\be)$, где $\ali \thicksim N[0,\sigma^2_{\alpha}]$. Тогда можно показать, что $\E[y_{it}|\x_{it}]=\Phi(\x_{it}'\be/\sqrt{1+\sigma^2_{\alpha}})$, что отличается от естественной пробит-модели сквозной регрессии, где $\E[y_{it}|\x_{it}]=\Phi(\x_{it}'\be)$. В отличие от линейной модели главы 21, если истинная модель содержит индивидуальные эффекты, то при игнорировании этих эффектов будут получены несостоятельные оценки параметра $\be$.

В статистической литературе часто используется сквозная регрессия для таких версий \textbf{обобщенных линейных моделей} панельных данных, как, например, модели бинарных данных и счетных данных. Результирующие оценки параметров называются \textbf{усредненными по генеральной совокупности}, так как случайные эффекты устранены. Такой подход называет \textbf{маргинальным анализом}, так как $\E[y_{it}|\x_{it}]$ --- это модель, которая является предельно по отношению к случайным эффектам.

{\centering Параметрические модели \\}

Отправной точкой в случае с \textbf{моделями сквозной регрессии} обычно выступает
\begin{align}
f(y_{it}|\x_{it})=f(y_{it}, \x'_{it}\be, \be, \bm\gamma)
\label{Eq:23.23}
\end{align}
для плотности $f(\cdot)$. Эту модель можно оценить ММП, используя для статистических выводов робастные стандартные ошибки, которые учитывают условную гетероскедастичность и корреляцию (см. раздел 23.2.6).

В общем оценки $\be$ и $\bm\gamma$ параметрической модели сквозной регрессии, скорее всего, не будут соответствовать оценкам, полученным из параметрической модели со случайными эффектами. Аргументы те же, что и для модели условного среднего.

\subsection{Фиксированные эффекты против случайных эффектов}

Важным результатом является то, что оценки модели со случайными эффектами и модели сквозной регрессии несостоятельны в нелинейных моделях, если в модели присутствуют индивидуальные эффекты, и они коррелированы с регрессорами. В связи с этим модели с фиксированными эффектами используются с большей предпочтительностью, хотя с другой стороны возникает проблема потери эффективности в оценивании. Проверить, необходимо ли использовать модель с фиксированными эффектами, можно с помощью теста Хаусмана (см. раздел 21.4.4), если возможно состоятельно оценить модель с фиксированными эффектами.

Другие методы сравнения линейных моделей фиксированных и случайных эффектов  (см. раздел 21.4) нужно видоизменить, чтобы адаптировать к нелинейным моделям.

Из-за проблемы второстепенных параметров не все нелинейные модели с фиксированными эффектами допускают состоятельные оценки параметров. Поэтому, моделирование фиксированных эффектов не всегда доступно.

Если нелинейную модель с фиксированными эффектами можно оценить состоятельно, то, в отличие от линейного случая, коэффициенты регрессоров, не меняющихся во времени, идентифицируемы. Чтобы это продемонстрировать, рассмотрим преобразование <<отклонение от среднего>> для модели с аддитивными эффектами. Для линейной модели $\E[(y_{it}-\bar{y}_i)-(\x_{it}-\bar{\x_i})'\be|\x_{i1},\dots,\x_{iT}]=\mathbf{0}$, и очевидно присутствуют проблемы с не меняющимися во времени регрессорами, так как для $j$-го регрессора  $x_{itj}-\bar{x}_{ij}=x_{ij}-x_{ij}=0$. В общем виде, из \ref{Eq:23.11}
\begin{align}
\E[(y_{it}-\bar{y}_i)-(\mathrm g(\x'_{it}\be-\bar{\mathrm{g}}_i(\be))|\x_{i1}]=\mathbf{0},
\nonumber
\end{align}
Такое упрощение возможно для нелинейного $\mathrm g(\cdot)$ только, если все $K$ компоненты $\x_{it}$ не меняются во времени.

В моделях с фиксированными эффектами с неаддитивными эффектами  не возможно предсказать изменения зависимой переменной при изменении регрессоров. Для общей модели \ref{Eq:23.2} \textbf{предельный эффект} $\partial \E[y_{it}|\x_{it},\ali,\be]/\partial \x_{it} = \partial \mathrm g(\x_{it}, \ali, \be)/\partial\x_{it}$ зависит от $\ali$.

Предельный эффект может быть измерен в двух частных случаях. Для аддитивных эффектов (см. \ref{Eq:23.3}) предельный эффект составляет $\partial \mathrm g(\x_{it}, \be)/\partial \x_{it}$. Для мультипликативных эффектов (см. \ref{Eq:23.4}) предельный эффект составляет $\ali \partial \mathrm g(\x_{it}, \be)/\partial \x_{it}$. Тогда возможно измерить размер предельных эффектов для изменений разных регрессоров. В частности, если $\E [y_{it}|\x_{it}, \ali, \be]=\ali \exp (\x_{it}' \be)$, то $(\partial \E[y_{it}]/\partial x_{it})/(\partial \E[y_{it}]/\partial x_{itk}) = \beta_j/\beta_k$.

\subsection{Оценивание и робастные статистические выводы}

В предыдущем анализе акцент был на устранении второстепенных параметров $\ali$. Сейчас, когда $\ali$ устранены, мы подробно будем рассматривать оценку параметров моделей с индивидуальными эффектами.

Мы предполагаем, что используется короткая панель, и наблюдения независимы по $i$.
Зависимая переменная $y_{it}$ может быть условно гетероскедастичной и условно коррелированной по $t$ для данного $i$. Ситуация похожа на ситуацию в разделе 21.2.3, за исключением того, что нелинейные оценки используются вместо обычного линейного МНК оценивания. По стандартным результатам выдаваемым статистическими пакетами, в которых игнорируется гетероскедастичность или коррелированность, можно сделать неверные статистические выводы. Далее мы представим выражения для робастных оценок ковариационной матрицы оценок параметров для панельных данных. В качестве альтернативы можно использовать панельный бутстрэп (см. раздел 11.6.2).

{\centering ОMM оценивание \\}

ОММ оценивание подходит для моделей, основанных на условном среднем. Ключевой является спецификация моментных тождеств, что составляет основу ОММ оценивания. Согласно разделу 22.2.1, естественной отправной точкой будет
\begin{align}
&\E[\mathbf{Z}'_i \mathbf{u}_i (\mathbf{\theta})]=\mathbf{0},
&i=1, \dots, N
\label{Eq:23.24}
\end{align}
где $\mathbf{Z}_i$ --- это матрица размерности $T \times r$, которая зависит от регрессоров, $\mathbf{u}_i(\mathbf{\theta})$ --- вектор остатков размерности $T \times 1$, и $\mathbf{\theta}$ --- вектор параметров размерности  $q \times 1$. В различных моделях панельных данных $\mathbf{u}_i$ и $\mathbf{Z}_i$. Ниже дан пример. В качестве основной отправной точки главы 22 мы берем тот факт, что остатки $\mathbf{u}_i(\mathbf{\bm\theta})$ будут нелинейны по $\bm\theta$.

Если $r=q$, то моментных тождеств такое же количество, как и параметров для оценки, и мы можем использовать \textbf{оценку метода моментов для панельных данных} $\mathbf{\theta}_{\mathrm{MM}}$, которая является решением 
\begin{align}
\frac{1}{N} \sum^N_{i=1} \mathbf{Z}'_i \mathbf{u}_i (\hat{\bm\theta})=\mathbf{0}.
\label{Eq:23.25}
\end{align}
Используя результаты раздела 6.10.3, посвященного нелинейным системам оценивания, эта оценка асимптотически нормальна и имеет её  ковариационную матрицу можно состоятельно оценить
\begin{align}
\hat{\mathbf{V}} [\hat{\bm\theta}]=\left[ \sum^N_{i=1} \hat{\mathbf{D}}'_i \mathbf{Z}_i \right]^{-1} \sum^N_{i=1} \mathbf{Z}'_i \mathbf{\hat{u}}_i \mathbf{\hat{u}}'_i \mathbf{Z}_i 
\left[ \sum^N_{i=1} \mathbf{Z}'_i \hat{\mathbf{D}}_i  \right]^{-1}
\label{Eq:23.26}
\end{align}
где $\mathbf{\hat{D}}_i=\partial \mathbf{u}_i/ \partial \bm\theta'|_{\hat{\bm\theta}}$ и $\mathbf{\hat{u}}_i=\mathbf{u}_i(\hat{\bm\theta})$. Такая оценка дисперсии дает робастные стандартные ошибки для коротких панелей.

Если $r > q$, то требуется ОММ оценивание, и мы используем \textbf{ОММ оценку для панельных данных} $\hat{\bm\theta}_{\mathrm{GMM}}$, которая минимизирует 
\begin{align}
Q_{N}(\bm\theta)=\left[ \frac{1}{N} \sum^N_{i=1} \mathbf{Z}'_i \mathbf{u}_i (\bm\theta) \right]' \mathbf{W}_N \left[ \frac{1}{N} \sum^N_{i=1} \mathbf{Z}'_i \mathbf{u}_i (\bm\theta) \right],
\label{Eq:23.27}
\end{align}
где $\mathbf{W}_N$ --- это взвешивающая матрица размерности $r \times r$. Асимптотическая ковариационная матрица для этой оценки может быть получена напрямую из результатов для  оценки инструментальных переменных нелинейных систем, представленной в разделе 6.10.4. При моментных тождествах \ref{Eq:23.24} самая эффективная оценка использует $\mathbf{W}_N=[N^{-1} \sum_i \mathbf{Z}'_i \mathbf{\hat{u}}_i 
\mathbf{\hat{u}}'_i \mathbf{Z}_i ]^{-1}$.

Более эффективные оценки возможны при использовании альтернативных моментных тождеств. В частности, если отправной точкой является отдельное условное моментное тождество, то оптимальное безусловное моментное тождество для ОММ оценивания дано в разделе 6.3.7. Этому подходу соответствует оценка обобщенных оценивающих уравнений, описанная ниже. Более подробно это описано в Авери, Хансен, и Хотц (1983) и Брейтунг и Лехнер (1999).


{\centering Пример ОMM оценивания \\}

В качестве особенного примера рассмотрим преобразование <<первые разности>> в применении к модели с мультипликативными фиксированными эффектами. Начнем с условного моментного ограничения \ref{Eq:23.14}. Это приводит к большому количеству безусловных моментных тождеств, одно из которых:
\begin{align}
 \E \left[ \x_{it} \left( y_{it} - \frac{\mathrm g(\x'_{it}\be)}{\mathrm g(\x'_{i,t-1}\be)}\times y_{i,t-1} \right) \right]  = \mathbf{0}, &
& t=1, \dots, T, &
& i=1, \dots, N.
\nonumber 
\end{align}
Предположим, что нам известны данные $(y_{it},\x_{it})$ для $(T+1)$ периодов. Начальный период выпадает в связи с взятием первых разностей. Расположив в столбец $T$ наблюдений для данного $i$, получаем \ref{Eq:23.24} с $\mathbf{Z}'_i=[\x_{i1}, \dots, \x_{iT}]$ и $\mathbf{u}'_i=[u_{i1}, \dots, u_{iT}]$, где $u_{it}=y_{it}-[\mathrm g(\x'_{it}\be)/\mathrm g(\x'_{i,t-1}\be)]y_{i,t-1}$. Здесь $\mathbf{Z}'_i\mathbf{u}_i=\sum_t \x_{it}u_{it}$, поэтому оценка методом моментов $\hat{\be}$ получается из решения
\begin{align}
\sum_{i=1}^N \sum^T_{t=1} \x_{it} \left[ y_{it} - \frac{\mathrm g(\x'_{it}\be)}{\mathrm g(\x'_{i,t-1}\be)} y_{i,t-1} \right]=\mathbf{0}.
\nonumber
\end{align}
Очевидно, что можно использовать дополнительные моментные тождества, например, $\E[\x_{it-1}u_{it}]=\mathbf{0}$, что приводит к сверхиндентифицируемости модели и оценивании с помощью ОММ. Это подробно обсуждалось для линейной модели в разделе 22.2.

{\centering Оценивание  с помощью обобщенных оценивающих уравнений \\}

В модели сквозной регрессии для условного среднего $\E[y_{it}|\x_{it}]=\mathrm g(\x_{it}, \be)$ (см. разедел 23.2.4). Эту модель можно оценить с помощью уже известных ОММ методов. Но мы пойдем дальше и рассмотрим эффективное ОММ оценивание.

Расположив в столбец все $T$ наблюдений для данного $i$ получаем условное моментное тождество
\begin{align}
\E[\mathbf{y}_i-\mathbf{g}_i(\be)|\mathbf{X}_i]=\mathbf{0,}
\label{Eq:23.28}
\end{align}
где $\mathbf{g}_i (\be)=[\mathrm g(\x_{i1},\be), \dots, \mathrm g(\x_{iT},\be)]'$  и $\mathbf{X}_i=[\x_{i1}, \dots, \x_{iT}]'$. Тогда оптимальное условное моментное тождество 
\begin{align}
\E \left[ \frac{\partial \mathbf{g}'_i(\be)}{\partial\be} \{ \mathbf{V}[\mathbf{y}_i|\mathbf{X}_i] \}^{-1} (\mathbf{y}_i - \mathbf{g}_i(\be))\right]=\mathbf{0},
\label{Eq:23.29}
\end{align}
которое получено после применения общего результата раздела 6.3.7. Получаем \textbf{оценку обобщенных оценивающих уравнений} (Generalized Estimating Equations, GEE) $\hat{\be}_{\mathrm{GEE}}$, которая является решением
\begin{align}
\bm\sum^N_{i=1} \frac{\partial \mathbf{g}'_i(\be)}{\partial \be} \Sigma^{-1}_i (\mathbf{y}_i-\mathbf{g}_i(\be))=\mathbf{0},
\label{Eq:23.30} 
\end{align}
где $\bm\Sigma_i$ --- это рабочая ковариационная матрица для $\mathbf{V}[\mathbf{y}_i|\mathbf{X}_i]$. Выражение для асимптотической ковариационной матрицы $\hat{\be}_{\mathrm{GEE}}$ дано в \ref{Eq:23.26} с $\hat{\mathbf{u}}_i=\mathbf{y}_i-\mathbf{g}_i(\hat{\be})$ и $\mathbf{Z}_i=\partial \mathbf{g}'_i(\be)/ \partial\be |_{\hat{\be}} \times \hat{\bm\Sigma}_i$. Для панельных данных эта оценка дисперсии является робастной. Кроме того она робастна к неправильной спецификации $\bm\Sigma_i$.

Согласно Лянг и Цегер (1986) оценка обобщенных оценивающих уравнений широко используется в статистической литературе для обобщенных линейных моделей панельных данных. Разные обобщенные линейные модели соответствуют разным функциям условного среднего $\mathbf{g}_i(\be)$ и действующим ковариационным матрицам $\bm\Sigma_i$.

{\centering Оценивание методом максимального правдоподобия  \\}

Для моделей, основанных на максимальном правдоподобии отправная точка --- совместная плотность для всех $T$ индивидуальных наблюдений, $f(\mathbf{y}_i|\mathbf{X}_i, \bm\theta)$. Для параметрических моделей сквозной регрессии $\bm\theta'= [\bm\beta', \bm\gamma', \bm\eta']$ (см. \ref{Eq:23.18}).

Согласно стандартному подходу, $f(\mathbf{y}_i|\mathbf{X}_i,\bm\theta)=\prod^T_{t=1} f(y_{it}|\x_{it}, \bm\theta)$, где $f(y_{it}|\x_{it}, \bm\theta)$ --- это плотность для $(i,t)$-го наблюдения. Неявное предположение независимости по $t$  для данного $i$ обычно неуместно, особенно для моделей сквозной регрессии, которые не включают случайный эффект, в которых может присутствовать корреляция во времени.  Тем не менее,  оценки $\bm\theta$ состоятельны, даже  если $f(\mathbf{y}_i|\mathbf{X}_i, \bm\theta)$ неправильно специфицирована при правильно специфицированом $f(y_{it}|\x_{it}, \bm\theta)$. Чтобы получать робастные стандартные ошибки для панельных данных для оценки ковариационной матрицы следует использовать сэндвич-форму. Оценка ММП --- это, строго говоря, оценка квази-максимального правдоподобия. Подробное обсуждение см. в разделе 5.7.5. Вообще этот подход является примером статистических выводов с кластеризованными данными (см. раздел 24.5).

Можно использовать более сложную модель для $f(\mathbf{y}_i|\mathbf{X}_i, \bm\theta)$, которая учитывают корреляцию во времени. Однако ненормальные многомерные распределения $\mathbf{y}_i$ сложны для работы. Для обобщенных линейных моделей сквозной регрессии можно использовать оценку обобщенных оценивающих уравнений.

\subsection{Динамические модели}

Динамические модели с индивидуальными эффектами представляют особый интерес, так как они позволяют проводит различие между истинной зависимостью от состояния и кажущейся зависимостью, которая обусловлена ненаблюдаемой гетерогенностью (см. раздел 22.5.1).

Для нелинейных моделей не всегда очевидно, как включать лаговые зависимые переменные в качестве регрессоров, так как некоторые типы данных не всегда объясняются стандартными моделями временных рядов. Это проиллюстрировано в разделе 23.7.4 для модели Пуассона. Когда определена подходящая спецификация, стандартные оценки с фиксированными эффектами будут несостоятельны и в оценках со случайными эффектами нужно учесть начальные условия так, как это было для линейных моделей панельных данных.

{\centering Модели сквозной регрессии \\}

Модели сквозной регрессии игнорируют случайные эффекты и оценивают обычные модели пространственных данных, где в регрессоры включены лаговые зависимые переменные. См. обсуждение в разделе 23.2.4.

{\centering Модели с фиксированными эффектами \\}

Вопросы моделей с фиксированными эффектами близки к представленным в разделе 22.5. Сейчас регрессоры не строго, а слабо экзогенны. Обычные оценки с фиксированными эффектами несостоятельны.

Для моделей с аддитивными эффектами или мультипликативными эффектами можно получить состоятельные оценки, используя преобразование <<взятие первых разностей>> (см. раздел 23.2.2), а также лаги зависимых переменных более высокого порядка в качестве инструментов. Для моделей с аддитивными эффектами это приводит к нелинейной версии оценки Ареллано-Бонда, представленной в разделе 22.5.3. Для мультипликативных эффектов преобразование <<взятие первых разностей>> подробно описано в разделе 23.7.4. Для  динамической логит-модели с фиксированными эффектами см. раздел 23.4.3.

{\centering Параметрические модели со случайными эффектами \\}

Для параметрических моделей со случайными эффектами имеют значение начальные условия лаговых зависимых переменных. Обычно в коротких панелях оценки несостоятельны. При этом несостоятельность уменьшается при увеличении $T$.

Рассмотрим самый простой случай, когда в модели присутствует лаг только первого периода, поэтому в качестве регрессоров выступают $\x_{it}$ и $y_{it-1}$. Плотность случайных эффектов \ref{Eq:23.1} будет $f(y_{it}|y_{it-1}, \x_{it}, \ali, \bm\delta)$ для $t=2, \dots, T$. Однако похожая модель для $y_{i1}$ не может быть использована, так как $y_{i0}$ не наблюдаем. Один из способов --- считать $y_{i1}$ экзогенной переменной, так что мы моделируем условное распределение только для $T-1$ наблюдений $y_{it}, \dots, y_{i2}$. Альтернативный подход предполагает статическую модель для $y_{i1}$, которое зависит от регрессоров $\x_{i1}$ и возможно от предельного эффекта $\ali$. Тогда совместная условная плотность $\mathbf{y}_i$ 
\begin{align}
f(\mathbf{y}_i|\x_{i1}, \dots, \x_{iT}, \ali, \bm\delta, \bm\delta_1, \bm\gamma) \nonumber \\
& =\int \left[ \prod^T_{t=2} f(y_{it}|y_{it-1}, \x_{it}, \ali, \bm\delta) \right] f_1(y_{i1}|\x_{i1}, \ali, \bm\delta_1) \mathrm g(\ali|\bm\gamma) d \ali,
\nonumber
\end{align}
а не \ref{Eq:23.18}, где $f_1(y_{i1}|\x_{i1}, \ali, \bm\delta_1)$ --- это предполагаемая плотность для первого наблюдения.

В анализе чисто временных рядов начальные условия не имеют значения в асимптотике, так как $T \rightarrow \infty$. В коротких панелях, однако, они имеют очень большое значение, так как $T$ маленькое и асимптотика обусловлена тем, что $N \rightarrow \infty$. 

\subsection{Эндогенные регрессоры}

Эндогенные переменные  в нелинейных моделях учитываются схожим образом, как было представлено в главе 22.

Естественный подход --- это панельный ОММ. Для начала определяется условное моментное ограничение $\E[\mathbf{u}_i(\bm\theta)|\mathbf{Z}_i]=\mathbf{0}$ для соответствующим образом определенных остатков $\mathbf{u}_i(\bm\theta)$ и инструментов $\mathbf{Z}_i$. Это приводит к безусловному моментному тождеству \ref{Eq:23.24}, которое лежит в основе ОММ оценивания. Возможными кандидатами на инструменты могут быть в том числе экзогенные регрессоры других периодов, как это было описано в разделе 22.2 и 22.4 для линейной модели.

\section{Пример нелинейной модели панельных данных: Патенты и НИОКР}

Мы будем моделировать зависимость между патентами и затратами на НИОКР, используя американские данные о 346 фирмах для 1975-1979 гг. из Холл, Грилихес, и Хаусман (1986). Зависимая переменная $y_{it}$ --- это количество патентов, определенное как количество поданных заявок на патенты в течение года, которые были в конце концов одобрены. Для простоты мы рассмотрим только одну объясняющую переменную $x_{it}$, реальные затраты на НИОКР в течение года (в ценах 1972 г.).

Рисунок 23.1: Патенты и затраты на НИОКР: сквозная регрессия. Натуральный логарифм количества успешных заявок против натурального логарифма затрат на НИОКР для 346 фирм в 1975-79 гг. Нулевое количество патентов было записано как 0.5.

Очевидно, первая модель --- модель в логарифмах с $\E[\mathrm{ln} y_{it}|x_{it}]=\ali+\beta \mathrm{ln} x_{it}$, так как в таком случае $\beta$ равно эластичности количества патентов по затратам на НИОКР. Эту модель здесь нельзя применять, так как $y_{it}=0$ для рассматриваемого количества наблюдений и $\mathrm{ln 0}$ не определено. Здесь делаем корректировку: перед взятием логарифмов $y_{it}=0$ заменяем на $y_{it}=0.5$.

На рисунке 23.1 изображен график зависимости ln(Patents) от ln(R\& D). Наряду с наблюдениями изображена МНК регрессия (с оцененным коэффициентом наклона 0.834) и кривая непараметрической регрессии. Использовались данные обо всех фирмах. Очевидно, число патентов увеличивается по мере увеличения затрат на НИОКР. С помощью анализа панельных данных, в особенности модели с фиксированными эффектами, можно разделить эту взаимосвязь на пространственную компоненту и временную компоненту. Заметим, что число патентов меняется значительно в зависимости от наблюдения, т.е. фирмы. Среднее число патентов --- 36.3, стандартное отклонение --- 74.5. Число патентов ранжируется от 0 до 608.

Мы оцениваем модель с мультипликативными индивидуальными эффектами для условного среднего с 
\begin{align}
\E[y_{it}|x_{it}, \ali] = \ali \exp (\beta \mathrm{ln} x_{it})=\exp  (\gamma_i+\beta \mathrm{ln} x_{it})
\label{Eq:23.31}
\end{align}
где $\gamma_i=\mathrm{ln} \ali$. Тогда $\beta$ является оценкой эластичности патентов по затратам на НИОКР, так как из \ref{Eq:23.31} следует $\partial \mathrm{ln} \E[y_{it}|x_{it}]/\partial \mathrm{ln} x_{it}=\beta$. В отличие от модели в логарифмах, нулевые значения $y_{it}$ не составляют проблему.

В более сложных параметрических моделях зависимая переменная счетная. Начнем с модели Пуассона:
\begin{align}
y_{it}|x_{it}, \gamma_i \thicksim \mathcal{P}[\exp (\gamma_i + \beta \mathrm{ln} x_{it})].
\label{Eq:23.32}
\end{align}
Эта модель, подробно описанная в разделе 23.7, имеет то же условное среднее для $y_{it}$, как и в \ref{Eq:23.31}.

В таблице \ref{Tab:23.1} представлены оценки для этих данных. Все оценки состоятельны при условии, что условное среднее в \ref{Eq:23.31} со случайным эффектом $\ali$, независимым от $\x_{it}$ и имеет постоянное среднее. Все оценки за исключением последней состоятельны при условии, что $\ali$ --- это фиксированный эффект, который коррелирован с $\x_{it}$. Имеется три вида оценок стандартных ошибок: вычисляемые программой по умолчанию, робастные оценки для панельных данных (где возможно) и оценки бутстрэпа (без уточнения). Более подробно для каждой колонки:

\begin{table}[ht]
\caption{{\it Патенты и затраты на НИОКР: Оценки нелинейных моделей панельных данных}$^a$} 
\centering
\begin{tabular}{c|c|c|c|c|c}
\hline \hline
  & \textbf{NLS} & \textbf{Пуассон} & \textbf{GEE} & \textbf{Пуассон-RE} & \textbf{Пуассон-FE} \\ 
\hline 
$\gamma=\mathrm{ln}\alpha$ & 2.529 & 1.712 & 2.068 & 2.313 & --- \\ 
$\beta$ & .509 & .693 & .560 & .349 & -0.038 \\  
Panel с.о.$^\beta$ & (.055) & (.043) & (.033) & (.033) & (.033) \\ 
Бутстреп с.о. & [.054] & [.047] & [.107] & [.119] & {.107} \\ 
Обычные с.о. & {.011} & {.002} & {.004} & {.033} & {.033} \\ 
Сумма $\beta$ & --- & .486 & .460 & .546 & .313 \\ 
N & 1730 & 1730 & 1730 & 1730 & 1620 \\ 
\hline \hline
\multicolumn{6}{p{15cm}}{${}^a$ В таблице представлены оценки нелинейного МНК сквозной регрессии, Пуассона сквозной регрессии, GEE оценка сквозной регрессии, Пуассона со случайными эффектами (RE) и Пуассона с фиксированными эффектами для нелинейной регрессии \ref{Eq:23.31} ln(Patents) на ln(R\&D). В круглых скобках представлены робастные стандартные ошибки для коэффициентов наклона, в квадратных скобках --- стандартные ошибки бутстрэп, в фигурных скобках --- обычные стандартные ошибки, которые предполагают независимые и одинаково распределенные ошибки. В предпоследней строчке дана сумма коэффициентов $\beta$ в расширенной модели, которая включает до пяти лагов $\mathrm{ln}(R \& D)$ в качестве регрессоров.} \\
\multicolumn{6}{p{15cm}}{$^b$ с.о. - стандартные ошибки}
\end{tabular} 
\label{Tab:23.1}
\end{table}

\textbf{NLS сквозной регрессии}: Оценки нелинейного МНК в первой колонке получаются в результате оценивания \ref{Eq:23.31} при $\ali=\alpha$. Нелинейный МНК описан в  разделе 5.8. Стандартные ошибки, вычисленные по умолчанию (их значение равно 0.011), в предположении о независимых и одинаково распределенных ошибках намного меньше, чем правильные робастные стандартные ошибки 0.054.

\textbf{Пуассона сквозной регрессии:} Оценки Пуассона во второй колонке --- для модели Пуассона \ref{Eq:23.32} c $\ali=\alpha$ с помощью ММП Пуассона в предположении о независимости по $i$ и $t$. Эластичность равна 0.693 по сравнению с оценкой нелинейного МНК равной 0.509. Стандартные ошибки, вычисленные по умолчанию (равные 0.002), удовлетворяют равенству дисперсии и среднего в распределении Пуассона (см. раздел 20.2.2). После учёта избыточной дисперсии с помощью оценки ковариационной матрицы в сэндвич-форме  (см. также раздел 20.2.2) оценка стандартной ошибки увеличилась до 0.020, что подтверждает необходимость учета избыточной дисперсии в счетных данных. Дополнительный учет корреляции по $t$ для данного $i$ приводит даже к более высоким робастным оценкам стандартных ошибок 0.043.

\textbf{GEE сквозной регрессии:} Оценка обобщенных оценивающих уравнений сквозной регрессии --- это решение \ref{Eq:23.30}, где $\mathrm g(\x_{it}, \be)$ определено в \ref{Eq:23.32} с $\ali=\alpha$. Используемая здесь спецификация действующей матрицы $\bm\Sigma_i$ дана после \ref{Eq:23.55}. Оцененная эластичность равна 0.560 со стандартной ошибкой 0.033, полученной с помощью робастной оценки, обсуждаемой после \ref{Eq:23.30}.

\textbf{Пуассон-RE}: Оценка Пуассона со случайными эффектами предполагает, что $\ali=\mathrm{ln} \gamma_i$ имеет гамма-распределение (см. раздел 23.7.2). Оцененная эластичность равна 0.349 со стандартными ошибками равными 0.033.

\textbf{Пуассон-FE}: Оценка Пуассона с фиксированными эффектами предполагает, что $\ali=\mathrm{ln} \gamma_i$ --- это фиксированный эффект, и он оценивается как в разделе 23.7.3. Оцененная эластичность сейчас отрицательна и равна -0.038, а стандартные ошибки по умолчанию равны 0.033. Для модели Пуассона с фиксированными эффектами исключены фирмы с $\sum_t y_{it}=0$, что привело к потере $22 \times 5=110$ наблюдений.

Есть большая разница между результатами с фиксированными и случайными эффектами, причем фиксированные эффекты более предпочтительны. Оцененная эластичность в модели с фиксированным эффектом имеет неожиданно отрицательный знак из-за того, что модель слишком проста. В частности, затраты на НИОКР влияют на патентную активность с лагом. Если заменить $\beta \mathrm{ln} \x_{it}$ в \ref{Eq:23.31} и \ref{Eq:23.32} на $\sum^5_{l=0} \beta_l \mathrm{ln} \x_{i,t-1}$, то оцененная эластичность будет равна $\sum^5_{l=0} \hat{\beta}_l$, она приведена в предпоследнем ряду таблицы 23.1. Оценка с фиксированными эффектами, равная 0.313, все еще меньше других оценок, но уже ненамного.

\section{Данные бинарного выбора}

Рассмотрим бинарный выбор, в котором $y_{it}$ принимает значения 0 или 1. Например, нам может быть известно, занят ли индивидуум на рынке труда в каждый из наблюдаемых периодов или нет. Ключевым результатом здесь является то, что оценивание фиксированных эффектов возможно только с помощью логит-модели, и невозможно с помощью пробит.

\subsection{Модели бинарного выбора с индивидуальными эффектами}

Естественный способ расширить модель бинарного выбора для пространственных данных (см. раздел 14.3) до модели панельных данных с индивидуальными эффектами --- специфицировать, что $y_{it}$ принимает только значения 0 и 1.
\begin{align}
\P [y_{it}=1|\x_{it}, \be, \ali]= 
\begin{cases}
& F (\ali+\x'_{it} \be) \text{в общем случае} \\
& \Lambda(\ali+\x'_{it}\be) \text{для логит-модели} \\
& \Phi(\ali+\x'_{it}\be) \text{для пробит-модели}
\end{cases}
\label{Eq:23.33}
\end{align}
где $F(\cdot)$ --- это  функция распределения, $\Lambda(\cdot)$  --- логистическая  функция распределения с $\Lambda(\mathrm{z})=e^{\mathrm{z}}/(1+e^{\mathrm{z}})$ и $\Phi(\cdot)$ --- функция стандартного нормального распределения. Зная \ref{Eq:23.33} и предполагая условную независимость, совместная вероятность для $i$-го наблюдения $\mathbf{y}_i=(y_{i1}, \dots, y_{iT})$ будет иметь вид

\begin{align}
f(\mathbf{y}_i|\mathbf{X}_i, \ali, \be)=\prod^T_{t=1} F(\ali+\x'_{it}\be)^{y_{it}} (1-F(\ali+\x'_{it}\be))^{1-y_{it}}.
\label{Eq:23.34}
\end{align}

Для бинарных данных условная вероятность --- это также условное среднее, т.е.
\begin{align}
\E[y_{it}|\ali, \x_{it}]=F(\ali+\x'_{it}\be).
\label{Eq:23.35}
\end{align}
Это одноиндексная модель с индивидуальными эффектами (см. \ref{Eq:23.5}), которая не упрощается ни до модели с аддитивными эффектами, ни до модели с мультипликативными эффектами. Модели с аддитивными или мультипликативными эффектами не являются подходящими, так как они не ограничивают условное среднее и условную вероятность так, чтобы они лежали между 0 и 1.

В моделях бинарного выбора большое значение придается параметрической модели \ref{Eq:23.34}, так как бинарные данные должны иметь распределение Бернулли. Модель условного среднего \ref{Eq:23.35} используется достаточно редко, хотя она применяется в случае эндогенных регресоров. 

\subsection{Модели бинарного выбора со случайными эффектами}

Оценка максимального правдоподобия со случайным эффектом предполагает, что индивидуальные эффекты нормально распределены с $\ali \thicksim \mathcal{N} [0, \sigma^2_{\alpha}]$. Для получения \textbf{оценки максимального правдоподобия со случайным эффектом} $\be$ и $\sigma^2_{\alpha}$ максимизируется логарифм функции правдоподобия $\sum^N_{i=1} \mathrm{ln} f(\mathbf{y}_i|\mathbf{X}_i, \be, \sigma^2_{\alpha})$, где 
\begin{align}
f(\mathbf{y}_i|\mathbf{X}_i, \be, \sigma^2_{\alpha}) = \int f(\mathbf{y}_i|\mathbf{X}_i, \ali, \be) \frac{1}{\sqrt{2 \pi \sigma^2_{\alpha}}} \exp  \left( \frac{-\ali}{2\sigma^2_{\alpha}} \right) ^2 d \ali,
\label{Eq:23.36}
\end{align}
где $f(\mathbf{y}_i|\mathbf{X}_i, \ali, \be)$ дано в \ref{Eq:23.34} с $F=\Lambda$ для логит-модели и $F=\Phi$ для пробит-модели. Не существует явной формулы для интеграла \ref{Eq:23.36}. Обычно его вычисляют с помощью численных методов, используя, например, \textbf{метод Гаусса}.

Если фиксированных эффектов нет, то в качестве альтернативной модели со случайными эффектами  можно использовать модель бинарного выбора сквозной регрессии, в которой $\P  [\mathbf{y}_{it}=1|\x_{it}]=F(\x'_{it}\be)$. Статистические выводы тогда должны основываться на робастных стандартных ошибках (см. раздел 23.2.6). Получить более эффективные оценки можно, используя ОММ подход (см. Авери и др., 1983) или подход обобщенных оценивающих уравнений (см. Лянг и Цегер, 1986).

\subsection{Логит-модель с фиксированными эффектами}

Оценить фиксированные эффекты для логит-модели панельных данных можно, используя оценку условного максимального правдоподобия. Однако этот метод не подходит для других моделей бинарных данных, таких как пробит-модель панельных данных.

После алгебраических преобразований раздела 23.4.5 совместная плотность $\mathbf{y}_i=(y_{i1}, \dots, y_{iT})$ для логит-модели равна
\begin{align}
f(\mathbf{y}_i|\ali, \x_i, \be) = \frac{\exp  \left(\ali \sum_t y_{it} \right) \exp  \left( \left( \sum_t y_{it} \x'_{it} \right) \be \right)}{\prod_t [1+ \exp (\ali+\x'_{it} \be) ]}.
\label{Eq:23.37}
\end{align}
Плотность зависит от $\ali$, которое должно быть устранено из модели. Для наблюдения $i$ есть $\sum_t y_{it}$ исходов равных 1 в $T$ периодах. Определим набор $\mathbf{B}_c=\{\mathbf{d}_i|\sum_t d_{it}=\sum_t y_{it}=c\}$ как набор всех возможных последовательностей нулей и единиц, для которых сумма $T$ бинарных исходов $\sum_t y_{it}=c$. Тогда, если мы включим в условие $\sum_t y_{it}=c$, то, как показано в разделе 23.4.6,  $\ali$ будет устранено и 
\begin{align}
f(\mathbf{y}_i|\sum_t y_{it} = c, \x_i, \be) = \frac{\exp \left(\left(\sum_t y_{it} \x'_{it} \right) \be \right)}{\sum_{\mathbf{d} \in \mathbf{B}_c} \mathbf{exp} \left( \left( \sum_t d_{it} \x'_{it} \right) \be \right)},
\label{Eq:23.38}
\end{align}
согласно результату, полученному в работе Чемберлина (1980). Плотность \ref{Eq:23.38} --- это основа для оценивания условным ММП. Единственная сложность состоит в том, что наборов $\mathbf{B}_c$ и последовательностей внутри них много.

Во-первых, невозможно использовать $\sum_t y_{it}=0$ как условие, так как оно выполняется только в том случае, когда все $y_{it}=0$. Аналогично для  $\sum_t y_{it}=T$. Это может означать большую потерю наблюдений если, например, большинство людей заняты во все периоды времени.

Проиллюстрируем пример, где можно считать условную вероятность. Предположим, что $T=2$ и $\sum_t y_{it}=1$. Тогда возможны две последовательности: ${0, 1}$ и ${1, 0}$, и из условной вероятности в \ref{Eq:23.38} следует, например, что
\begin{align}
\P [y_{i1}=0, y_{i2}=1 | y_{i1}+ y_{i2}=1]
&=\frac{\exp (\x'_{i1}\be}{\exp (\x'_{i1}\be)+\exp (\x'_{i2}\be)} \nonumber \\
& = \frac{\exp ((\x'_{i1}-\x_{i0})'\be}{1+\exp ((\x_{i1}-\x_{i0}'\be)}.
\nonumber
\end{align} 
Если $T=3$, то мы можем поставить условие $\sum_t y_{it}=1$ с возможными последовательностями ${0, 0, 1}$, ${0, 1, 0}$ и ${1, 0, 0}$, либо $\sum_t y_{it}=2$ с возможными последовательностями ${0, 1, 1}$, ${1, 0, 1}$ и ${1, 1, 0}$. Для большого $T$ возможно много последовательностей, и условная вероятность может оказаться очень сложной.

Это условная вероятность --- вероятность из условной логит-модели, где параметры неизменны, а регрессоры меняются в зависимости от выбранной альтернативы. Количество альтернатив меняется в зависимости от индивидуума, где для индивидуума $i$ каждая альтернатива --- это специфическая последовательность нулей и единиц, сумма элементов которой равна $\sum_t y_{it}$. Самый простой путь --- это использовать компьютерный код, созданный специально для данной задачи. Даже в такой случае может быть большое количество альтернатив. Например, если $T=10$ и $\sum_t y_{it}=5$, то у нас 252 альтернативы. Получить состоятельные, но менее эффективные оценки можно, удалив некоторые наблюдения, например, для индивидуумов с большим числом альтернатив из-за большого $\sum_t y_{it}$. Либо можно уменьшить число периодов.

Из-за устранения индивидуальных эффектов $\ali$ интерпретировать коэффициенты регрессии, используя модель первоначальную модель \ref{Eq:23.37}, невозможно. Вместо этого, мы используем условную модель \ref{Eq:23.38}. Например, предположим, что у нас один регрессор и $\beta=0.2$. Тогда, если мы рассматриваем два временных периода и условие $\sum_t y_{it}=1$, то
\begin{align}
\P [y_{i1}=0, y_{i2}=1| y_{i1}+y_{i2}=1]= 
\frac{\exp (\beta(x_{i1}-x_{10}))}{1+\exp (\beta(x_{i1}-x_{10}))}.
\nonumber
\end{align}
Если $x_{i1}$ и $x_{i2}$ различаются на единицу, то условная вероятность этой последовательности равна $\exp (\beta)/[1+\exp  (\beta)]$, если $x_{i1}=x_{i2}$, то вероятность равна одной второй. 

\subsection{Динамические модели бинарного выбора}

Предположим, что у нас есть логит-модель временных рядов, марковский процесс первого порядка, где имеется только один регрессор --- лаг зависимой переменной:
\begin{align}
\P [y_{it}=1|\ali, y_{it-1}]= \frac{\exp (\ali+\gamma y_{it-1})}{1+\exp (\ali+\gamma y_{it-1})}.
\label{Eq:23.39}
\end{align}
Выполнив некоторые алгебраические преобразования раздела 23.4.5, получаем
\begin{align}
f(\mathbf{y}_{it}|y_{i1}, y_{iT}, \sum^{T-1}_{t=2} y_{it}, \gamma)= 
\frac{\exp (\gamma \sum^{T-1}_{t=2} y_{it} y_{it-1})}{\sum_{\mathbf{d} \in \mathbf{C}_i}\exp (\gamma \sum^{T-1}_{t=2} d_{it} d_{it-1})},
\label{Eq:23.40}
\end{align}
где набор $\mathbf{C}_i=\{ \mathbf{d}_i|y_{i1}, y_{iT}, \sum_t d_{it}=\sum_t y_{it} \}$ --- это набор всех возможных последовательностей нулей и единиц, для которой сумма $T$ бинарных исходов равна $\sum_t y_{it}$, первое значение --- $y_{i1}$, последнее значение --- $y_{iT}$.

Оценивание условным ММП, основанным на \ref{Eq:23.40}, даст состоятельные оценки $\gamma$. Минимальное необходимое количество временных периодов --- четыре. Например, если $\mathbf{y}_i$ --- это последовательность $\{ 0, 1, 0, 1\}$, то набор $\mathbf{C}_i$ состоит из последовательностей $\{0, 1, 0, 1\}$ и $\{0, 0, 1, 1\}$. Этот подход предложил Чемберлин (1985). Он рассматривал марковскую модель второго порядка. Чей, Хойнес, и Хислоп (2001) применяли этот метод к данным по административным округам Калифорнии о благосостоянии и выяснили, что учет ненаблюдаемой гетерогенности позволяет выявить истинную зависимость благосостояния.

Предыдущие результаты и обсуждение относились к моделям чистых временных рядов. Оноре и Кирьязиду (2000) предложили метод, который позволяет включать в качестве регрессоров не только лаговые зависимые переменные. Так \ref{Eq:23.39} принимает вид
\begin{align}
\P [y_{it}=1|\ali, y_{it-1}, \x_{it}]=
\frac{\exp (\ali+\x'_{it}\be + \gamma y_{it-1})}{1+\exp (\ali+\x'_{it}\be+\gamma y_{it-1})}.
\label{Eq:23.41}
\end{align}
Рассмотрим четыре временных периода и последовательности с одинаковыми значениями в первом и четвертом периоде, $d_1$ и $d_4$. Тогда вероятность того, что это последовательность $\{d_1, 0, 1, d_4 \}$ сейчас зависит от $\ali$, при условии, что это либо $\{d_1, 0, 1, d_4 \}$, либо $\{d_2, 0, 1, d_4 \}$. Однако зависимость от $\ali$ исчезает, если $x_{3i}=x_{4i}$. Из-за того, что для некоторых наблюдений $x_{3i}=x_{4i}$, особенно в случае с непрерывными данными, Оноре и Кирьязиду (2000) предложили использовать методы ядерного сглаживания с весами, которые зависят от $(x_{3i}-x_{4i})$. Чей и Хислоп (2000) применяли этот и другие методы для динамических моделей бинарного выбора.

\subsection{Модель множественного выбора}

Оценка с фиксированным эффектом может быть расширена до логит-модели множественного выбора, так как эта модель заключает в себе логит-модель бинарного выбора для попарного сравнения альтернатив (см. раздел 15.4.3). Чемберлин (1980) кратко описывает статическую модель, М.-Дж. Ли (2002) представляет более подробное описание. Маньяк (2000), используя динамическую логит-модель с фиксированными эффектами с лаговыми зависимыми в качестве регрессоров, подробно описывает применение этой модели к анализу индивидуального карьерного движения по шести различным ступеням на французском рынке труда. Оноре и Кирьязиду (2000) рассматривают логит-модель множественного выбора.

Для других моделей множественного выбора необходимо использовать случайные эффекты. Эти модели, такие как смешанные логит-модель или пробит-модель множественного выбора, достаточно сложны даже в случае пространственных данных. Более подробное описание см. Трейн (2003).

\subsection{Выводы для логит-модели фиксированных эффектов}

Для простоты опустим индекс $i$. Для логит-модели совместная вероятность  $\mathbf{y}=(y_1, \dots, y_T)$, которая дана в \ref{Eq:23.34}, равна
\begin{align}
f(\mathbf{y}|\bm\alpha)
& =\prod^T_{t=1} \left( \frac{\exp (\alpha+\x'_t \be)}{1+\exp (\alpha+\x'_t\be)} \right) ^{y_t} 
\left( \frac{1}{1+\exp (\alpha+\x'_t\be)}\right)^{1-y_t} 
\label{Eq:23.42}\\
& = \frac{\exp (\sum_t y_t(\alpha+\x'_t \be))}{\prod_t [1+\exp (\alpha+\x'_t \be)]} \nonumber \\
& = \frac{\exp (\alpha \sum_t y_t ) \exp ((\sum_t y_t \x'_t)\be)}{\prod_t [1+\exp (\alpha+\x'_t \be)]}, \nonumber
\end{align}
что в результате дает \ref{Eq:23.37}.

Можно показать, что количество $\sum_t y_t$ может быть достаточной статистикой для $\alpha$. Предположим, у нас есть наблюдение для $\mathbf{y}$, такое что $\sum_t y_t = c$. Определим набор $\mathbf{B}_c = \{ \mathbf{d}| \sum_t d_t = c\}$ как набор всех возможных последовательностей нулей и единиц, для которых сумма $T$ бинарных исходов  равна $c$ при условии $\sum_t y_t = c$. Тогда
\begin{align}
f(\mathbf{y}|\sum_t y_t = c)
& = \frac{\P [\mathbf{y}, \sum_t y_t = c]}{\P [\sum_t y_t = c]}
\label{Eq:23.43} \\
& = \frac{\P [\mathbf{y}]}{\P [\sum_t y_t = c]} \nonumber \\
& = \frac{\P [\mathbf{y}]}{\sum_{\mathbf{d} \in \mathbf{B}_c} \P [\mathbf{d}]} \nonumber \\
& = \frac{\exp ((\sum_t y_t \x'_t) \be )}{\sum_{\mathbf{d} \in \mathbf{B}_c} \exp ((\sum_t d_t \x'_t) \be ) }  \nonumber ,
\end{align}
где для получения первого равенства используется правило Байеса, для второго равенства используется тот факт, что знание $\sum_t y_t$ не несет дополнительной при известном $\mathbf{y}$. В третьем равенстве используется тот факт, что $\P  [\sum_t y_t =c]$ равна сумме вероятностей конкретных комбинаций из нулей и $c$ единиц. В четвертом используется определение $f(\mathbf{y})$ и упрощение, которое базируется на том, что $\sum_t y_t = \sum_t d_t$, когда $\mathbf{d} \in \mathbf{B}_c$.

Сейчас рассмотрим динамическую модель. Заменяя $\x'_t \be$ в \ref{Eq:23.42} на $\gamma y_{t-1}$, получаем 
\begin{align}
f(\mathbf{y})
&= \frac{\exp (\alpha \sum^T_{t=2} y_t) \exp ( \sum^T_{t=2}\gamma y_{t-1} y_t)}{\prod_t [1+\exp (\alpha+\gamma y_{t-1})]} \nonumber \\
&= \frac{\exp (\alpha \sum^T_{t=2} y_t )\exp (\sum^T_{t=2} \gamma y_{t-1} y_t )}{[1+\exp (\alpha)]^{\sum^T_{t=2} (1-y_{t-1})}[1+\exp (\alpha+\gamma)]^{\sum^T_{t=2} y_{t-1}}} \nonumber \\
&= \frac{\exp (\alpha \sum^T_{t=2} y_t) \exp (\sum^T_{t=2} \gamma y_{t-1} y_t)}{[1+\exp (\alpha)]^{(T-1+y_1-y_T+\sum^T_{t=2} y_t)+}[1+\exp (\alpha+\gamma)]^{y_1-y_T+\sum^T_{t=2} y_{t}}}, \nonumber
\end{align}
где во втором равенстве используются некоторые алгебраические преобразования и тот факт, что $y_{t-1}$ принимает значение либо 0, либо 1, а в последнем выражении используется $\sum^T_{t=2} y_{t-1} = y_1 - y_T + \sum^T_{t=2} y_t$. Здесь используются те же алгебраические преобразования, что и в \ref{Eq:23.43}, за исключением того, что помимо условия $\sum^T_{t=2} y_t$ нужно включить в условие  еще $y_1$ и $y_T$ , которые появляются в знаменателе. Мы можем включить в условие $\sum^T_{t=1} y_t$, $y_1$, $y_T$, что будет эквивалентно. Таким образом, получаем
\begin{align}
f(\mathbf{y})=\frac{\exp (\sum^T_{t=2} \gamma y_{t-1} y_t)}{\sum_{\mathbf{d} \in \mathbf{C}_c} \exp (\sum^T_{t=2} \gamma d_{t-1} d_t)},
\nonumber
\end{align}
где $\mathbf{C}=\{ \mathbf{d}| d_1 = y_1, d_T = y_T, \sum^T_{t=1} d_t = \sum^T_{t=1} y_t \}$ --- это набор всех возможных последовательностей нулей и единиц, для которых сумма бинарных исходов $T$ равна $\sum_t y$, первый исход --- $y_1$, последний исход --- $y_T$.

\section{Тобит-модель и модели самоотбора}


Мы рассмотрим цензурированные и урезанные выборки, а также модели самоотбора, когда в нашем распоряжении имеются панельные данные, а не данные пространственного типа.

Анализ сквозной регрессии просто повторяет анализ случая с пространственными данными, лишь с той корректировкой, что нужно использовать робастные стандартные ошибки для панельных данных (см. раздел 23.2.8). Например, см. Грасдал (2001), где рассматриваются модели самоотбора, появившегося в результате истощения панели.

Здесь мы фокусируем внимание на моделях панельных данных с индивидуальными эффектами. Затем можно оценить модели со случайными эффектами при условии выполнения сильного предположения о чисто случайных эффектах. Трудность при оценке моделей со случайными эффектами может состоять лишь в численных вычислениях. Нет простых состоятельных оценок для моделей с фиксированными эффектами, кроме  обычного микроэконометрического случая короткой панели. Более сложные полупараметрические оценки, которые позволяют включать фиксированные эффекты в тобит-модель и обощенную тобит-модель, представлены в разделе 23.8.

\subsection{Модели с цензурированными и урезанными выборками}

Для пространственных данных тобит-модель с цензурированной выборкой представлена в разделе 16.3.1. Версия для панельных данных с индивидуальными эффектами:
\begin{align}
y^*_{it}=\ali+\x'_{it}\be+\e_{it},
\label{Eq:23.44}
\end{align}
где $\e_{it} \thicksim \mathcal{N}[0,\sigma^2_{\e}]$, мы наблюдаем $y_{it}=y^*_{it}$, если $y^*_{it} > 0$ и $y_{it}=0$ или наблюдаем пропуск, если $y^*_{it} \leq 0$. Совместная плотность для $i$-го наблюдения $\mathbf{y}_i=(y_{i1}, \dots, y_{iT})$ можно переписать как
\begin{align}
f(\mathbf{y}_i| \mathbf{X}_i, \ali, \be, \sigma^2_{\e})=\prod^T_{t=1} [\frac{1}{\sigma_{\e}} \phi_{it}]^{d_{it}} [1-\Phi_{it}]^{1-d_{it}},
\label{Eq:23.45}
\end{align}
где $\phi_{it}=\phi((y_{it}-\ali-\x'_{it}\be)/\sigma_{\e})$, $\Phi_{it}=\Phi ((\ali+\x'_{it}\be)/\sigma_{\e})$, и $\phi(\cdot)$ и $\Phi(\cdot)$ обозначают соответственно функцию плотности и функцию распределения стандартной нормальной величины.

Оценка максимального правдоподобия с фиксированным эффектом, основанная на плотности \ref{Eq:23.45}, максимизирует логарифм функции максимального правдоподобия по $\be, \sigma^2_{\e}$ и $\alpha_1, \dots, \alpha_N$. В коротких панелях оценка $\be$ несостоятельна ввиду проблемы второстепенных параметров, и не существует  простого метода взятия разностей или введения условия, который может обеспечить состоятельную оценку. Хекман and МакКарди (1980) применяют оценку максимального правдоподобия с фиксированным эффектом к анализу рынка труда женщин. Хотя они признают несостоятельность оценки, они показывают, что при $T=8$ несостоятельность может быть невелика. Грин (2004a) изучает оценку максимального правдоподобия тобит-модели с фиксированным эффектом с помощью метода Монте Карло.

Оценка со случайным эффектом чаще используется ввиду несостоятельности оценки с фиксированным эффектом. При выполнении предположения, что $\ali \thicksim \mathcal{N} [0, \sigma^2_{\alpha}]$ \textbf{оценка ММП со случайным эффектом} $\be, \sigma^2_{\e}$ и $\sigma^2_{\alpha}$ максимизирует логарифм функции правдоподобия $\sum^N_{i=1} \mathrm{ln} f(\mathbf{y}_i |\mathbf{X}_i, \be, \sigma^2_{\e}, \sigma^2_{\alpha})$, где
\begin{align}
f(\mathbf{y}_i|\mathbf{X}_i, \be, \sigma^2_{\e}, \sigma^2_{\alpha})=
\int f(\mathbf{y}_i | \mathbf{X}_i, \ali, \be, \sigma^2_{\e}) 
\frac{1}{\sqrt{2 \pi \sigma^2_{\alpha}}} \exp  (\frac{-\ali}{2 \sigma^2_{\alpha}})^2 d \ali,
\label{Eq:23.46}
\end{align}
для $f(\mathbf{y}_i |\mathbf{X}_i, \ali, \be, \sigma^2_{\e})$, представленного в \ref{Eq:23.45}. Этот однократный интеграл можно вычислить с помощью \textbf{метода Гаусса}.

Этот подход можно расширить до других моделей, в которых используются цензурированные и урезанные выборки. Например, модель Пуассона со случайными эффектами с цензурированной справа выборкой в разделе 23.7.2 можно использовать, если число событий выше 10 записывается только как 10 или выше.

У полного параметрического подхода есть два недостатка. Во-первых, как и в случае с пространственными данными, зависимость от предположений о распределении намного выше, когда выборка урезана или цензурирована. Во-вторых, предположение о чисто случайных эффектах, независимых от регрессоров, может оказаться чересчур сильным. 

\subsection{Модели самоотбора}

Модели самоотбора могут использоваться для панельных данных по тем же причинам, что и в случае пространственных данных (см. раздел 16.5). Обобщение \textbf{тобит-модели второго типа} в разделе 16.5.1  до линейной модели панельных данных с индивидуальными эффектами $\lambda_i$ и $\delta_i$: 
\begin{align}
y^*_{it} = \ali + \x'_{it} be +\e_{it} 
\label{Eq:23.47} \\
d^*_{it} = \delta_i+\mathbf{z}'_{it} \gamma + v_{it}, \nonumber
\end{align}
где $y_{it}=y^*_{it}$ наблюдается, если $d^*_{it} >0$, и $y_{it}$ не наблюдается в противном случае.

Для формулирования случайных эффектов предполагается, что четыре ненаблюдаемые переменные имеют нормальное распределение. 
Хаусман и Уайз (1979) предложили оценивание ММП, которое включает в себя вычисление двойного интеграла, так как $\ali$ может коррелировать с $\delta_i$, и $\e_{it}$ может коррелировать с $v_{it}$.

Оценка с фиксированными эффектами состоятельна в коротких панелях.  Заметим, однако, что если $d^*_{it}=\delta_i$, то выбор объясняется неизменными во времени характеристиками индивидуума, которые могут быть наблюдаемыми или не наблюдаемыми. В таком случае оценка с фиксированным эффектом в модели $y_{it} = \ali+\x'_{it} \be + \e_{it}$ состоятельна. Модель с фиксированными эффектами учитывает самоотбор выборки в той мере, в которой он зависит характеристик, не меняющихся во времени.

Вербик и Нейман (1992) более подробно обсуждают ключевые предположения, необходимые для состоятельного оценивания этих моделей и предлагают тест на наличие смещения, связанного  самоотбором. Вулдридж (1995) описывает похожий анализ при более слабых предположениях и показывает предположения, которые могут быть не такими ограничительными в некоторых приложениях, и позволяют состоятельно оценить модель с фиксированным эффектом. В работе Велла (1998) можно найти обзор и дополнительные справки.

Методы для учёта самоотбора выборки могут быть расширены до работы с \textbf{истощением панели} (см. раздел 21.8.5), которое приводит к \textbf{смещению в связи с истощением}, если наблюдения зависимой переменной потеряны не случайным образом. Тогда все данные для $i$-го наблюдения ненаблюдаемы, если $d^*_{it} \leq 0$, из-за чего $\mathbf{z}_{it}$ в \ref{Eq:23.47} нужно заменить на переменные других периодов. Пример можно найти у Хаусмана и Уайза (1979), а также у Грасдала (2001). Ссылки на другую литературу можно найти у Бальтаджи (2001) и Хсяо (2003).

\section{Данные о переходах}

Для конкретности возьмем панельные данные о продолжительности благосостояния. Особый интерес представляет измерение индивидуального постоянства в периоды благосостояния, и определение меры, в которой оно объясняется истинной зависимостью от состояния, а не  индивидуальной предрасположенностью к благосостоянию. Так как индивидуальная предрасположенность может частично зависеть от ненаблюдаемых наблюдений, следует использовать модели с индивидуальными эффектами. Для данных о длительности состояний существует необыкновенно широкий набор подходов моделирования, так как возможно несколько типов панельных данных о переходах между состояниями. Здесь мы остановим внимание на моделях с фиксированными эффектами. 

Могут быть доступны данные о том, находится или нет индивидуум в определенном состоянии, например, положении хорошего благосостояния, в несколько различных моментов времени.  В таком случае можно использовать модель бинарного выбора для панельных данных (см. раздел 23.4), такую как динамическая логит-модель с фиксированными эффектами.

Более сложные данные дают нам информацию о продолжительностях индивидуальных состояний. Начнем с \textbf{модели пропорционального риска панельных данных}
\begin{align}
\lambda (t_{ij} | \x_{ij}) = \lambda_j (t_{ij}, \gamma_j) \exp (\x'_{ij} \be) \ali 
\label{Eq:23.48}
\end{align}
где $t_{ij}$ --- продолжительность завершенного периода для $j$-го периода $i$-го индивидуума, $\ali$ --- индивидуальный эффект. Это смешанная модель пропорционального риска, которая обсуждается для данных о продолжительности в главе 18. В ряд условий для непараметрического оценивания модели пропорциональных рисков с данными о продолжительности одного периода (см. раздел 18.3) входит предположение, что $\ali$ распределены независимо от регрессоров. Это позволяет избавиться от фиксированных эффектов. Однако как только нам доступны данные о продолжительности нескольких периодов, $\ali$ может быть фиксированным эффектом, если $\x_{ij}$ не меняется по $j$, как показал  Оноре (1992) (см. раздел 19.4.1). Дальнейшее обсуждение модели \ref{Eq:23.48}, включая динамическую модель длительности состояний с функцией риска для второго периода, зависимого от продолжительности первого, см. в разделе 19.4.1.

Чемберлин (1985) предложил несколько подходов для устранения $\ali$ в различных моделях продолжительности состояний панельных данных. Для модели пропорциональных рисков с базовым риском $\lambda_j(\cdot)$ вероятность того, что второй период будет длиннее, чем первый, не зависит от $\ali$. Для гамма моделей продолжительности состояний можно применять условный ММП, так как гамма --- это распределение из экспоненциального семейства. Для моделей Вейбулла, гамма и лог-нормальной плотность $t_{i1}/t_{i2}$ не зависит от $\ali$ .

Актуальные ссылки и подробные обсуждения, включая чувствительность данных о продолжительности нескольких периодов к цензурированию, см. у Ван ден Берга (2001).

\section{Счетные данные}

Хаусман и др. (1984) представляет модели с фиксированными и случайными эффектами как для моделей Пуассона, так и для отрицательных биномиальных моделей панельных данных. В более современной работе рассматриваются фиксированные эффекты в моделях с мультипликативными эффектами,  которые позволяют получать оценку статической и динамической моделей при относительной слабых предположениях о распределении.

\subsection{Счетные модели с индивидуальными эффектами}

Мы остановимся на модели Пуассона, подробно описанной для пространственных данных в разделе 20.2, хотя панельные версии отрицательных биномиальных моделей тоже кратко рассмотрим.

В \textbf{модели Пуассона с индивидуальными эффектами} 
$y_{it} \thicksim \mathcal{P}[\ali \exp (\x'_{it}\be)]$.
Тогда, предполагая условную независимость, совместная вероятность 
для $i$-го наблюдения $\mathbf{y}_i=(y_{i1}, \dots, y_{iT})$ равна
\begin{align}
f(\mathbf{y}_i|\mathbf{X}_i, \ali, \be)=\prod^T_{t=1} \exp [-\ali \exp (\x'_{it}\be)]
[-\ali \exp (\x'_{it}\be)]^{y_{it}}/y_{it!}.
\label{Eq:23.49}
\end{align}

В менее параметрическом подходе условное среднее моделируется просто как 
\begin{align}
\E[y_{it}|\ali, \x_{it}] = \ali \exp (\x'_{it} \beta) = 
\label{Eq:23.50} \\
= \exp (\gamma_i+\x'_{it}\beta). \nonumber
\end{align}
Это относится и к одноиндексной модели с индивидуальными эффектами, и к модели с мультипликативными эффектами. Так как это мультипликативные эффекты, индивидуальные эффекты $\ali$ могут быть устранены  с помощью вычитания среднего или взятия первых разностей. Заметим, что у модели Пуассона панельных данных \ref{Eq:23.49} условное среднее равно \ref{Eq:23.50}.

\subsection{Счетные модели со случайными эффектами}

Предположение о том, что случайные эффекты имеют гамма-распределение, позволяет получить трактуемое решение для вероятности в модели со случайными эффектами. Предположим, что $\ali$ имеет $\mathcal{G}[\eta, \eta]$ распределение со средним 1 и дисперсией $1/\eta$ и плотностью $\mathrm g(\ali|\eta)=\eta^{\eta} \ali^{\eta-1} e^{-\ali\eta} / \Gamma(\eta)$. Тогда \ref{Eq:23.18} для модели Пуассона \ref{Eq:23.49} будет следующим:
\begin{align}
f(\mathbf{y}_i |\mathbf{X}_i, \be, \eta) = 
\left[ \prod_t \frac{\lambda^{y_{it}}_{it}}{y_{it}!} \right]
\times \left( \frac{\eta}{\sum_t \lambda_{it}+\eta} \right)^{\eta} 
\label{Eq:23.51} \\
\times \left( \sum_t \lambda_{it} \right)^{-\sum_t y_{it}}
\frac{\Gamma (\sum_t y_{it} + \eta)}{\Gamma(\eta)}, \nonumber
\end{align}
где $\lambda_{it}=\exp (\x'_{it}\be)$, а преобразования и выводы даны в разделе 23.7.5. Получившееся условие первого порядка для оценки Пуассона случайных эффектов $\hat{\be}$ можно выразить как
\begin{align}
\sum^N_{i=1} \sum^T_{t=1} \x_{it} \left(y_{it} - \lambda_{it} \frac{\bar{y}_i + \eta/T}{\bar{\lambda}_i + \eta/T} \right) = \mathbf{0},
\label{Eq:23.52}
\end{align}
где $\bar{\lambda}_i = T^{-1} \sum_t \exp (\x'_{it}\be)$.

Ожидаемое значение выражения слева от знака равенства в \ref{Eq:23.52} равно 0, если среднее при условии регрессоров всех периодов равно $\E [y_{it}|\ali, \x_{i1}, \dots, \x_{iT}]=\ali \exp  (\x'_{it} \be)$. Так что несмотря на все параметрические предположения, оценка Пуассона со случайным эффектом состоятельна для $\be$ при выполнении слабого предположения о том, что условное среднее равно \ref{Eq:23.50}, и что регрессоры строго экзогенны. Для вероятности \ref{Eq:23.51} $\E [y_{it}|\x_{it}]=\lambda_{it}$ и $\mathrm{V}[y_{it}|\x_i]=\lambda_{it}+\lambda^2_{it}/\delta$, т.е. избыточная дисперсия имеет вторую форму отрицательного биномиального распределения. Оценка  ковариационной матрицы в сэндвич форме позволит использовать более гибкие модели избыточной дисперсии и условной корреляции. Условия первого порядка для $\eta$, которые мы не приводим, довольно сложны, хотя информационная матрица имеет блочно-диагональный вид по $\be$ и $\eta$.

Для случайных эффектов доступно несколько альтернативных оценок. Во-первых, оценка Пуассона сквозной регрессии игнорирует случайные эффекты и предполагает, что $y_{it} | \x_{it} \thicksim \mathcal{P}[\exp (\x'_{it} \be)]$. Условия первого порядка:
\begin{align}
\sum^N_{i=1} \sum^T_{t=1} \x_{it} (y_{it} - \lambda_{it})= \mathbf{0},
\label{Eq:23.53}
\end{align}
где $\lambda_{it}=\exp (\x'_{it} \be)$. Эта оценка состоятельна, если условное среднее имеет вид \ref{Eq:23.50} c $\E [\ali | \x_{it}]=1$. Поэтому обычная оценка ММП Пуассона для пространственных данных состоятельна, если истинная модель --- это модель с мультипликативными эффектами. Однако, как было показано в разделе 23.3, следует использовать робастные стандартные ошибки. Здесь \ref{Eq:23.26} будет равно
\begin{align}
\hat{V} [\hat{\be}_{pool} ] = 
\left[ \sum_{i,t} \hat{\lambda}_{it} \x_{it} \x'_{it} \right]^{-1}
\sum_{i,t,s} \hat{u}_{it} \hat{u}_{is} \x_{it} \x'_{it}
\left[ \sum_{i,t} \hat{\lambda}_{it} \x_{it} \x'_{it} \right]^{-1},
\label{Eq:23.54}
\end{align}
где $\hat{\lambda}_{it}=\exp (\x'_{it}\hat{\be})$, $\hat{u}_{it}=y_{it}-\hat{\lambda}_{it}$, $\sum_{i,t}$ обозначает $\sum^N_{i=1} \sum^T_{t=1}$, и $\sum_{i,t,s}$ обозначает $\sum^N_{i=1} \sum^T_{t=1} \sum^T_{s=1}$. Альтернативная оценка сквозной регрессии, основанная на \ref{Eq:23.50}, --- это оценка, полученная с помощью нелинейного МНК. В этом случае \ref{Eq:23.53}  будет выглядеть как $\sum_i \sum_t \x_{it} \lambda_{it} (y_{it}-\lambda_{it})=\mathbf{0}$.

Во-вторых, более эффективную оценку сквозной регрессии можно получить, используя  подход обобщенных оценивающих уравнений раздела 23.2.8, в котором предполагается условная корреляция. Общий результат для $\mathrm{g}_{it}=\lambda_{it}=\exp (\x'_{it}\be)$ теперь примет вид
\begin{align}
\sum^N_{i=1} \mathbf{Z}'_i \Sigma^{-1}_i (\mathbf{y}_i-\bm\lambda_i)=\mathbf{0},
\label{Eq:23.55}
\end{align}
где $\mathbf{Z}_i$  --- это матрица размерности $T \times K$ с $t$-м рядом  $\lambda_{it}\x'_{it}$ и $\bm\lambda_i$ --- это вектор размерности $T \times 1$ c $t$-м элементом $\lambda_{it}$. Возможно несколько видов ковариационных матриц $\bm\Sigma_i$ для $\mathrm{V}[\mathbf{y}_i|\mathbf{X}_i]$. Выбор матрицы вида $\bm\Sigma_i=\mathrm{Diag}[\lambda_{it}]$ приводит к оцениванию уравнений \ref{Eq:23.53} с помощью модели Пуассона. Если $\Sigma_{i,tt}=\lambda_{it}$ и $\Sigma_{i,ts}=\lambda_{is}=\phi \sqrt{\lambda_{it}\lambda_{is}}$ для $s \neq t$ , то возможно включить корреляцию по $t$. Речь идёт о  \textbf{равнокоррелированности} или \textbf{перестановочной корреляции}, так как для $s \neq t$ корреляция  равна константе $\phi$.

В-третьих, более эффективное оценивание возможно, если использовать ММП с отрицательной биномиальной моделью, а не моделью Пуассона. Предположим, $y_{it}$ независимая и одинаково распределенная отрицательная биномиальная величина с функцией избыточной дисперсии в форме NB2 с параметрами $\ali\lambda_{it}$ и $\Phi$ (см. раздел 20.4.1). Среднее $y_{it}$ равно $\ali \lambda_{it}/\phi_i$, а дисперсия $(\ali\lambda_{it}/\phi_i) \times (1+\ali/\phi)$. Если $(1+\ali/\phi_i)^{-1}$ --- это случайная величина, имеющая бета-распределение с параметрами $(\eta_1, \eta_2)$, то после некоторых алгебраических преобразований \ref{Eq:23.18} упрощается до 
\begin{align}
f(\mathbf{y}_i|\mathbf{X}_i, \be, \eta) = 
\left( \prod_t \frac{\Gamma (\lambda_{it}+y_{it})!}{\Gamma(\lambda_{it})! |Gamma(y_{it}+1)!} \right)
\label{Eq:23.56} \\
\times \frac{\Gamma(\eta_1+\eta_2)\Gamma (\eta_1+\sum_t \lambda_{it}) \Gamma (\eta_2+\sum_t y_{it})}{\Gamma(\eta_1) \Gamma(\eta_2) \Gamma (\eta_1+\eta_2+\sum_t \lambda_{it} + \sum_t y_{it})}. \nonumber \\
\end{align}
где $\lambda_{it}=\exp (\x'_{it}\be)$. Это база для оценивания параметров $\be$, $\eta_1$ и $\eta_2$ с помощью ММП. Эта модель базируется на более сильных предположениях, чем модель Пуассона со случайными эффектами.

В-четвертых, анализ нельзя ограничивать параметрическими моделями с решением в аналитическом виде для $f(\mathbf{y}_i|\mathbf{X}_i, \be, \eta)$. Крепон и Дюжё (1997a) используют методы симуляционного  максимального правдоподобия для оценки модели преодоления препятствий панельных на счетных данных и модели для панельных счетных данных с избыточным количеством нулей с нормально распределёнными случайными эффектами. 



\subsection{Счетные модели с фиксированными эффектами}

Оценку с фиксированным эффектом для модели Пуассона панельных данных \ref{Eq:23.50} можно вывести несколькими способами.

Во-первых, оценка максимального правдоподобия Пуассона одновременно оценивает $\be$ и $\alpha_1, \dots, \alpha_N$. Логарифм функции правдоподобия, основанной на вероятности \ref{Eq:23.49}, равен
\begin{align}
\mathrm{ln L}(\be, \bm\alpha) 
& = \mathrm{ln} 
\left[ \prod_i \prod_t \{\exp (-\ali \lambda_{it}) (\ali \lambda_{it})^{y_{it}} / y_{it} ! \} \right] 
\label{Eq:23.57} \\
& = \sum_i \left[ -\ali \sum_t \lambda_{it} + \mathrm{ln} \ali \sum_t y_{it} + \sum_t y_{it} \mathrm{ln} \lambda_{it} - \sum_t \mathrm{ln} y_{it}! \right], 
\nonumber 
\end{align}
где $\lambda_{it}=\exp (\x'_{it} \be)$. Взятие производной по $\ali$ и приравнивание ее к нулю дает $\hat{\ali}=\sum_t y_{it}/ \sum_t \lambda_{it}$. Подставим это обратно в \ref{Eq:23.57}, в результате чего получим \textbf{краткую функцию правдоподобия}. Опуская члены, которые не содержат $\be$, получаем
\begin{align}
\mathrm{ln L}_{\mathrm{conc}}(\be) \varpropto 
\sum_i \sum_t \left[ y_{it} \mathrm{ln} \lambda_{it} - y_{it} \mathrm{ln} \left( \sum_s \lambda_{is} \right) \right].
\label{Eq:23.58}
\end{align}
Отсюда следует, что для модели Пуассона с фиксированными эффектами отсутствует проблема второстепенных параметров. Состоятельные оценки $\be$ для фиксированных $T$ и $N \rightarrow \infty$ могут быть получены посредством максимизации $\mathrm{ln L_{conc}}(\be)$ в \ref{Eq:23.58}. При взятии производной \ref{Eq:23.58} по $\be$  получаем условия первого порядка
\begin{align}
\sum_i \sum_t 
\left[ y_{it} \x_{it} - y_{it} \left[ \sum_s \lambda_{is} \x_{is} \right] /
\left[ \sum_s \lambda_{is} \right] \right] = \mathbf{0},
\nonumber
\end{align}
которое можно перезаписать как
\begin{align}
\sum^N_{i=1} \sum^T_{t=1} \x_{it} \left( y_{it} - \frac{\lambda_{it}}{\bar{\lambda}_i}\bar{y}_i \right) = \mathbf{0},
\label{Eq:23.59}
\end{align}
где $\lambda_{it}=\exp (\x'_{it} \be)$  и $\bar{\lambda}_i=T^{-1} \sum_t \exp (\x'_{it} \be)$, см. Бланделл, Гриффит и Виндмайер (1995). Модель Пуассона \ref{Eq:23.49} и линейная модель панельных данных раздела 21.6 необычны тем, что одновременное оценивание $\be$ и $\bm\alpha$  дает состоятельные оценки $\be$ в коротких панелях. Так что \textbf{проблема второстепенных параметров} здесь отсутствует.

Во-вторых, оценивание условным ММП устраняет фиксированные эффекты посредством включения в условие достаточной статистики $\ali$. Для модели Пуассона это $\sum_t y_{it}$. Выполнение алгебраических преобразований, представленных в разделе 23.7.5, дает условную функцию правдоподобия, которая пропорциональна краткой логарифмической функции правдоподобия в \ref{Eq:23.58}. Оценка условного ММП для $\be$ в модели Пуассона с фиксированными эффектами является решением \ref{Eq:23.59}. Это был обычный вывод оценки Пуассона с фиксированными эффектами $\be$, который проделали  Палмгрен (1981) и Хаусман и др. (1984).

В-третьих, выполнение преобразования <<отклонение от среднего>> \ref{Eq:23.14} для модели с мультипликативными эффектами \ref{Eq:23.50} дает $\E [y_{it} - (\lambda_{it} / \bar{\lambda}_i) \bar{y}_i |\x_{i1}, \dots, \x_{iT}]=0$, и поэтому
\begin{align}
\E[\x_{it}(y_{it}-(\lambda_{it}/\bar{\lambda}_i) \bar{y}_i )] = \mathbf{0}.
\label{Eq:23.60}
\end{align}
Используя соответствующие выборочные моментные тождества, получаем оценку $\be$, которая является решением \ref{Eq:23.59}.

Одна и та же оценка была получена тремя различными способами. Из третьего вывода понятно, что важное предположение, необходимое для состоятельности оценки Пуассона  с фиксированным эффектом, состоит в том, что регрессоры строго экзогенны и \ref{Eq:23.50} правильно специфицирована. Статистические выводы должны основываться на робастных стандартных ошибках для панельных данных. В частности, если использовать обычные выводы стандартного и условного ММП, то исходя из первых двух выводов, стандартные ошибки могут быть значительно недооценены из-за того, что в счетных данных не была учтена избыточная дисперсия. Оценка с фиксированным эффектом приводит к потере данных, так как наблюдения $i$, для которых $\sum_t y_{it} = 0$, не дополняет сумму в \ref{Eq:23.59}.

Получить состоятельные оценки $\be$ в присутствии фиксированных эффектов тоже возможно, если использовать особую параметризацию отрицательной биномиальной модели. Хаусман и др. (1984) предполагал, что $y_{it}$ --- это независимая одинаково распределенная NB1 величина с параметрами $\ali \lambda_{it}$ и $\phi_i$, где $\lambda_{it}=\exp (\x'_{it}\be)$, так что $y_{it}$ имеет среднее $\ali \lambda_{it} / \phi$ и дисперсию $(\ali \lambda_{it}/\phi_i) \times (1+\ali /\phi_i)$. Параметры $\ali$ и $\phi_i$ могут быть идентифицируемы вплоть до отношения $\ali/\phi_i$, и эта дробь выпадает из выражения условной совместной плотности для $i$-го наблюдения. Выражение для совместной плотности упрощается до 
\begin{align}
f(y_{i1}, \dots, y_{iT} | \sum_t y_{it})=
& \left( \prod_t \frac{\Gamma(\lambda_{it}+y_{it})}{\Gamma(\lambda_{it}) \Gamma (y_{it} + 1)} \right) 
\label{Eq:23.61} \\
& \times \frac{\Gamma (\sum_t \lambda_{it}) \Gamma (\sum_t y_{it} + 1)}{\Gamma(\sum_t \lambda_{it}+\sum_t y_{it})}.
\nonumber
\end{align}
Целое $\lambda_{it}$  имеет \textbf{отрицательное гипергеометрическое распределение}. Оценка отрицательной биномиальной модели с фиксированными эффектами условного ММП $\be$ максимизирует логарифм функции  правдоподобия, основанную на \ref{Eq:23.61}. Модель Пуассона с фиксированными эффектами более часто используется, так как оценка этой модели будет состоятельно при гораздо более слабых предположениях о распределении.  


\subsection{Динамические счетные модели}


Есть несколько способов учитывать динамику в моделях счетных данных. Чистые модели временных рядов исследуются в Кэмерон и Триведи (1998). Для простоты рассмотрим включение в модель одной лаговой зависимой переменной. Очевидно будет использовать модель $\E [y_t| y_{t-1}, \x_t]=\exp (\gamma y_{t-1}+\x'_t\be)$, но это может привести к <<взрывному поведению>> из-за того, что берется экспонента от $y_{t-1}$. Более стабильную модель можно получить, используя $\exp (\gamma \mathrm{ln} y_{t-1} + \x'_t \be)$, но тогда возникают проблемы при $y_{t-1}=0$. По этой причине используется линейная модель обратной связи $\E [y_t|y_{t-1}, \x_t]=\gamma y_{t-1}+\exp (\x'_t\be)$. Модель Пуассона AR(1) обладает этим свойством и в случае чистых временных рядов имеет корреляционную функцию $\mathrm{Cor}[y_t, y_{t-k}]=\gamma^k$, похожую на корреляционную функцию для модели AR(1) (см. Аль-Ош и Алзаид, 1987).

Таким образом, Бланделл, Гриффит, и Виндмайер (1995, 2002) рассматривали динамическую модель панельных данных с фиксированными эффектами с
\begin{align}
\E [y_{it} | \ali, y_{i,t-1}, \x_{it} ] = \gamma y_{i,t-1} + \ali \exp (\x'_t \be).
\nonumber
\end{align}
Применяя преобразование <<взятие первых разностей>> \ref{Eq:23.15}, получаем условные моментные ограничения
\begin{align}
\E \left[ \frac{\exp (\x'_{i,t-1} \be)}{\exp (\x'_{it}\be)}
(y_{it} - \gamma y_{i,t-1}) - (y-{i,t-1} - \gamma y_{i,t-2}) | y_{i1}, \dots, y_{i,t-2}, \x_{i1}, \dots, \x_{i,t-1} \right] = 0.
\nonumber
\end{align}
Они приведут к большому количеству моментных тождеств (см. в разделе 22.5.3 похожее обсуждение для линейной модели), которые могут служить основой для ОММ оценивания, как в разделе 23.2.6. Крепон и Дюжё (1997b), Монталво (1997), и Бланделл, Гриффит, и Ван Ринен (1995, 1999) используют схожие методы взятия квази-разностей для анализа зависимости количества патентов и затрат на НИОКР.

Бёкенхольт (1999) использует более параметрическую модель, оценивая AR(1) модель Пуассона для целых значений с ненаблюдаемой гетерогенностью, для моделирования которой используется смесь распределений (см. раздел 18.5). 

\subsection{Выводы для моделей Пуассона со случайными и фиксированными эффектами}

Во-первых, рассмотрим модель Пуассона со случайными эффектами с гамма-распределением случайных эффектов. Для простоты опустим индекс $i$ и пусть $\lambda_t=\exp (\x'_t\be)$. Из общей формулы \ref{Eq:23.18} для модели Пуассона \ref{Eq:23.49} и плотности случайных эффектов $\mathrm g(\alpha|\gamma)$ следует, что
\begin{align}
f(y_1, \dots, y_T | \x_t) 
& = \int^{\infty}_0 \left[ \prod_t (e^{-\alpha \lambda_t} (\alpha \lambda_t)^{y_t}/ y_t !) \right]  \mathrm g(\alpha | \gamma) d \alpha 
\nonumber \\
& = \int^{\infty}_0 \left[ \prod_t \lambda_t^{y_t}/y_t ! \right] \left( e^{-\alpha \sum_t \lambda_t} \alpha^{\sum_t y_t} \right) 
\mathrm g(\alpha|\gamma) d \alpha
\nonumber \\
& = \left[ \prod_t \lambda_t^{y_t} /y_t ! \right]
\times \int^{\infty}_0 \left( e^{-\alpha \sum_t \lambda_t} \alpha^{\sum_t y_t} \right) 
\mathrm g(\alpha|\gamma) d \alpha. \nonumber
\end{align}
Для $\mathrm g(\ali|\eta)=\eta^{\eta} \alpha^{\eta-1} e^{\alpha \eta}/\Gamma(\eta)$ используя похожие алгебраические преобразования, что и в разделе 20.4.1, получаем выражение для плотности \ref{Eq:23.51}.

Во-вторых, выведем условную плотность для модели Пуассона с фиксированными эффектами для наблюдений во все временные периоды для данного индивидуума, где для простоты опущен индивидуальный индекс $i$. В общем случае вероятность для $y_1, \dots, y_T$ при $\sum_t y_t$ равна
\begin{align}
f(y_1, \dots, y_T| \sum_t y_t) 
& = f(y_1, \dots, y_T | \sum_t y_t)/f(\sum_t y_t) \nonumber \\
& = f(y_1, \dots, y_T )/f(\sum_t y_t) \nonumber \\
& = \frac{\prod_t (\exp (-\mu_t)\mu_t^{y_t}/y_t !)}{\exp (-\sum_t \mu_t) (\sum_t \mu_t)^{\sum_t y_t}/(\sum_t y_t)!} \nonumber \\
& = \frac{\exp (-\sum_t \mu_t)\prod_t \mu_t^{y_t}/\prod_t y_t !}{\exp (-\sum_t \mu_t) \prod_t (\sum_s \mu_s)^{y_t}/(\sum_t y_t)!} \nonumber \\
& = \frac{(\sum_t y_t)!}{\prod_t y_t !} \times \prod_t \left( \frac{\mu_t}{\sum_s \mu_s} \right)^{y_t}, \nonumber 
\end{align}
где во второй строке используется тот факт, что знание $\sum_t y_t$ ничего не добавляет к знанию $y_1, \dots, y_T$. В  третьем равенстве используется то, что $y_t$ независимо и одинаково распределены с $\mathcal{P}[\mu_t]$ и поэтому $\sum_t y_t$ имеет распределение $\mathcal{P}[\sum_t \mu_t]$. В четвертой и пятой строке выражение упрощается. Условная вероятность --- это вероятность для мультиномиальной модели для $\sum_t y_t$ испытаний, где $t$-й из $T$ исходов появляется в любом испытании с вероятностью $\mu_t / \sum_s \mu_s$. Установив, что $\mu_{it}=\ali \exp (\x'_{it}\be)$, и взяв логарифмы, получаем условное правдоподобие, которое пропорционально функции правдоподобия в краткой форме \ref{Eq:23.58}.

\section{Полупараметрическое оценивание}

В литературе, посвященной \textbf{полупараметрическому оцениванию} панельных данных, главным образом рассматриваются модели ограниченных зависимых переменных, так как, как и для случая панельных данных, предположения о параметрическом распределении становится очень важным, когда используются цензурированные, урезанные выборки или модели самоотбора. Внимание сосредоточено на моделях с фиксированными эффектами. Мы проведем их краткий обзор.

Для данных бинарного выбора Мански (1987) расширил свою оценку, основанную на методе максимального счёта, для модели панельных данных с фиксированными эффектами, представленной в \ref{Eq:23.33}, где функция $F(\cdot)$ уже не специфицирована. Хотя эта оценка состоятельна, она сходится со скоростью меньше, чем $\sqrt{N}$ и не является асимптотически нормальной. 

Для тобит-модели Оноре (1992) расширил подход наименьших абсолютных отклонений для цензурированных выборок, который разработал Пауэл (1986a), до модели панельных данных с фиксированными эффектами \ref{Eq:23.45}, где распределение ошибки $\e_{it}$ не специфицировано. Данные урезаны искусственным образом, так что фиксированный эффект последовательно устранен подходящим преобразованием взятия разностей. Оценка состоятельна со скоростью $\sqrt{N}$ и асимптотически нормальна.

Для панельных данных с самоотбором выборки Кирьязиду (1997) рассматривал тобит-модель типа 2 с фиксированными эффектами, где распределение ошибок $\e_{it}$ и $v_{it}$  не специфицировано. Она представила двухшаговую оценку вида Хекмана. С помощью сглаженной оценки, основанной на максимального счёта Мански (1987), устраняются фиксированные эффекты в уравнении самоотбора, хотя на втором шаге используется довольно сложное преобразование взятия разностей для устранения фиксированных эффектов в результирующем уравнении. Этот подход может быть расширен до обобщенных тобит-моделей. Шарлье, Меленберг, и ван Соест (2001) привели пример применения модели Роя или тобит-модели типа 5 для панельных данных.

Цензурирование часто используется в моделях продолжительности жизни. В разделе 23.6 рассматриваются модели панельных данных с завершенными периодами. Когда для индивидуума наблюдаются и завершенные, и незавершенные периоды, методы частичного правдоподобия не подходят, так как в присутствии не меняющихся во времени фиксированных эффектов цензурирование выборки проводится не независимо. Хоровиц и Ли (2004) предложили состоятельную оценку для  модели пропорциональных рисков \ref{Eq:23.43} с незавершенными периодами, для которых не требуется спецификация базового риска.


\section{Практические соображения}
Как и для случая линейных моделей, если используются панельные данные, как минимум, статистические выводы должны основываться на робастных стандартных ошибках. Вычисление таких ошибок не предусмотрено компьютерными программами для пространственных данных, если не используется опция  кластеризации стандартных ошибок. Кластеризация в этом случае проводится по индивидуальным наблюдениям.

Более эффективное оценивание возможно благодаря использованию моделей, в которых учитывается корреляция во времени. Эконометристы придают особое значение случайным эффектам. Несколько пакетов оценивают модели с нормально распределенными случайными эффектами, используя метод Гаусса для учёта эффектов, а также более специализированные  модели счетных данных со случайными эффектами, имеющие явные аналитические решения. Статистики же придает значение подходу обобщенных оценивающих уравнений для обобщенных линейных моделей, доступному во многих статистических и некоторых эконометрических пакетах.

Эти предшествующие методы дают несостоятельные оценки, если случайные эффекты коррелированы с регрессорами. Поэтому эконометристы уделяют особое внимание использованию фиксированных эффектов. Из-за проблемы второстепенных параметров состоятельные оценки в коротких панелях получаются только у некоторых нелинейных моделей. Доступны эконометрические пакеты для условного ММП оценивания этих моделей, логит-моделей и счетных моделей с фиксированными эффектами. Если модель с фиксированными эффектами недоступна, то нужно использовать модель с более сложными случайными эффектами, а не простейшими независимыми и одинаково распределенными эффектами.

Можно также оценивать динамические модели панельных данных. Они позволяют провести различие между постоянством, вызванным ненаблюдаемой гетерогенностью, и постоянством, вызванным истинной зависимостью от состояния. При оценивании этих моделей может потребоваться написание собственных программ.


\section{Литература}

В этой главе предоставлен обзор обширной и разнообразной литературы. По необходимости пропущено множество деталей. В монографиях о панельных данных, написанных Ареллано (2004), Бальтаджи (2001), Хсяо (2003), и М.-Дж. Ли (2002), рассматриваются модели панельных данных для данных бинарного выбора, цензурированных выборок, а также модели самоотбора. Кэмерон и Триведи (1998), а также М.-Дж. Ли (2002) описали модели панельных данных для счетных данных. Вулдридж (2002) описывает методы для оценки бинарных данных, цензурированных выборок, а также счетных данных. Фармайер и Тутц (1994) и Диггл и др. (1994, 2002) провели обзор статистической литературы для различных обобщенных линейных моделей. Матьяс и Севестр (1995) рассматривают нелинейные модели панельных данных. М.-Дж. Ли (2002) заостряет внимание на ОММ оценивании. Ареллано и Оноре (2001) придают важное значение полупараметрическим методам для нелинейных моделе панельных данных. Куп (2003) рассматривал байесовские методы оценки моделей панельных данных.

\textbf{23.2} Проблему второстепенных параметров обсуждает Ланкастер (2002). Ключевые работы, посвященные методам взятия разностей ---  Чемберлин (1992) and Вулдридж (1997a), а условному ММП --- Андерсен (1970). Для моделей со случайными эффектами Батлер и Моффитт (1982) для устранения нормально распределенных случайных эффектов используют метод Гаусса, в то время как в статистической литературе главное предпочтение отдается подходу, который разработали Лянг и Цегер (1986).

\textbf{23.4} Основные работы, в которых рассматриваются модели с фиксированными эффектами --- Чемберлин (1980) для статических моделей, Чемберлин (1985) для динамических моделей чистых временных рядов, и Оноре и Кирьязиду (2000) для динамических моделей с дополнительными регрессорами. См. также Хсяо (1995).

\textbf{23.5} Исследование моделей выбора панельных данных --- Велла (1998), а также учебники Бальтаджи (2001) и Вулдридж (2002).

\textbf{23.6} Чемберлин (1985) описывает несколько способов устранения фиксированных эффектов в различных моделях длительности состояний. Хорошее обсуждение и ссылки см. в Ван ден Берг (2001, раздел 6). \textbf{Анализ исторических событий} с использованием индивидуальных данных с несколькими промежутками намного сложнее, чем любой анализ панельных данных, так как такие модели являются динамическими по своей природе.

\textbf{23.7} Классическая литература, посвященная моделям счетных панельных данных, --- это Хаусман и др. (1984), а также Бланделл и др. (2002) для динамических моделей.

\textbf{23.8} Исследования полупараметрических методов оценивания моделей панельных данных --- Ареллано и Оноре (2001), а также Л.-Ф. Ли (2001).

{\centering \textbf{Упражнения} \\}

\textbf{23-1} Рассмотрим нелинейную модель панельных данных $y_{it} = \ali+ \exp (\x'_{it}\be)+u_{it}$, где $\be$ --- параметры для оценки, $\ali, i=1, \dots, N$ --- индивидуальные эффекты, $u_{it}$ --- независимые одинаково распределенные ошибки с параметрами $[0, \sigma^2_{\e}]$. Используется короткая панель.
\begin{itemize}
\item[{\bf (a)}] Предположим, что $\ali=0$ для любого $i$. Можно ли состоятельно оценить $\be$? Если да, то напишите формулу или целевую функцию для состоятельной оценки. Если нет, кратко объясните, почему невозможно получить состоятельную оценку для $\be$.
\item[{\bf (b)}] Предположим, что индивидуальные эффекты $\ali$ случайны, независимы и одинаково распределены с параметрами $[0, \sigma^2_{\alpha}]$ независимо от регрессоров. Можно ли состоятельно оценить $\be$? Если да, то напишите формулу или целевую функцию для состоятельной оценки. Если нет, кратко объясните, почему невозможно получить состоятельную оценку для $\be$.
\item[{\bf (с)}] Предположим, что индивидуальные эффекты $\ali$ случайны, но коррелированы с регрессорами. Можно ли состоятельно оценить $\be$? Если да, то напишите формулу или целевую функцию для состоятельной оценки. Если нет, кратко объясните, почему невозможно получить состоятельную оценку для $\be$.
\end{itemize}

\textbf{23-2} (Чемберлин, 1980) Покажите, что оценка ММП в логит-модели бинарного выбора несостоятельна, предел равен $2\beta$ в простой модели, где $T=2$.

\textbf{23-3}  Используйте ту же модель, что и модель Патенты-НИОКР в разделе 23.3. Выбирайте зависимую переменную и модель так, как предложено ниже. В каждом случае оценивайте модели со случайными эффектами и модель с фиксированными эффектами, если это возможно.

\begin{itemize}
\item[{\bf (a)}] Используйте логит-модель с зависимой переменной, показывающей, имеет ли фирма патент.
\item[{\bf (b)}] Используйте тобит-модель с урезанной выборкой, где в качестве зависимой переменной стоит логарифм количества патентов и из выборки удалены фирмы с нулевым количеством патентов.  
\item[{\bf (с)}] Используйте модель Пуассона для количества патентов.
\end{itemize}





\chapter{Структура микроэкономических данных}

\section{Введение}

Данная глава исследует вопрос, связанный с потенциальной полезностью и ограничением различных типов микроэкономических данных. До сих пор наиболее распространенной структурой данных, используемой в микроэконометрики являются данные обследования или переписи. Эти данные, как правило, называются данные наблюдений, чтобы отличить их от экспериментальных данных.


В этой главе обсуждается потенциальное ограничение вышеупомянутых структур данных. Ограничения, присущие данным наблюдений могут еще более усугубляться, способом сбора данных, такими как, на сгенерированной выборке, простой случайной выборке по сравнению с стратифицированной случайной выборки и одномоментной выборке. Поэтому мы также обсуждаем способы сбора данных в связи с использованием данных наблюдений. 
	
	
Микроэконометрика выходит за рамки анализа данных обследования в предположениях простой случайной выборки. В этой главе эти вопросы рассматриваются более детально. Раздел 3.2 описывает структуру многоступенчатого выборочного обследования и некоторы распространенные формы ухода от случайной выборки; более д	етальный анализ их статистических последствий обсуждается в следующих главах. В ней также рассматриваются некоторые часто встречающиеся проблемы, в результате которых выборка не является представителем генеральной совокупности. Учитывая недостатки данных наблюдений при оценке параметров причинности, исследователи чаще используют экспериментальные и квази-экспериментальные данные. Раздел 3.3 рассматривает преимущество данных из социальных экспериментов, а раздел 3.4 --- возможности моделирования с использованием особого типа данных наблюдения, генерируемых в квази-экспериментальных условиях. Раздел 3.5 охватывает практические вопросы использования микроданных.
	
\section{Данные наблюдений}

	
Основным источником микроэкономических данных наблюдения являются обследования домашних хозяйств, фирм и государственных административных учреждений. Данные переписи могут также использоваться для создания выборки. Также  источником данных могут являться маркетинговые опросы, интернет-аукционы и т.д. 


Существует огромная литература по выборочным обследованиям с точки зрения как статистиков, так и пользователей данных. В первой части обсуждается как правильно получить выборку от генеральной совокупности, а также результаты различных способов конструкций выборки, а вторая занимается вопросами оценки и выводов, которые возникают, когда данные обследования собираются с использованием различных способов конструкций выборки. Ключевым вопросом является то, насколько хорошо выборка отражает генеральную совокупность. Эта глава фактически является введением в данную теорию, более детальное рассмотрение методов приведены в главе 24.
	

\subsection{Природа данных обследования}


Термин данные наблюдений обычно относится к данным обследования, собранных путем дискретизации соответствующей генеральной совокупности без контролирования характеристики выборки данных. Обозначим через $t$ время, а через $w$  набор переменных, представляющих интерес. При этом  $t$ может быть точкой в интервале времени или периодом времени. Пусть $S_{t}$ обозначает выборку из функции распределения вероятностей генеральной совокупности $F(w_{t}|\theta_{t}$; $S_{t}$ выводится из $F(w_{t}|\theta_{t}$, где $\theta$ --- вектор параметров. Генеральная совокупность должна рассматриваться как множество точек с интересующими нас характеристиками, а для простоты будем считать, что форма функции $F$ известна. Случайная выборка предполагает равновероятность включения того или иного элемента из генеральной совокупности в выборку, более сложные способы составления выборки будут рассмотрены позже.


Абстрактное понятие стационарности генеральной совокупности является очень полезной. Если моменты характеристик генеральной совокупности постоянны, то мы можем записать, что $\theta_{t}=\theta$, для всех $t$, это предположение является сильным. Например, возрастное распределение по полу должно быть постоянным, однако, некоторые характеристики генеральной совокупности не могут быть постоянным. Чтобы справиться с этим, необходимо, чтобы параметры каждой генеральной совокупности можно было рассмотреть как вывод из суперпопуляции с постоянными характеристиками. В частности, каждое $\theta_{t}$ выводится  из функции распределения вероятностей с постоянными гиперпараметрами $\theta$. Термины суперпопуляция и гиперпараметры часто встречаются в литературе по иерархическим моделям, которые обсуждаются в главе 24. Использование иерархических моделей, обсуждаемых в главах 13 и 26, обеспечивает один подход к моделированию связи между гиперпараметрами и субпопуляционными характеристиками.


\subsection{Простая случайная выборка}

В качестве ориентира для последующего обсуждения, рассмотрим простую случайную выборку, в которой вероятность единицы выборки $i$, из генеральной совокупности размера $N$, составляет $1/N$ для всех наблюдений. Предположим, наш цель заключается в моделировании $y$, вектор возможных результатов, обусловлен экзогенной ковариацией вектора $x$, чье совместное распределение обозначается $f_{J}(y,x)$, которое всегда может быть разделено на произведение условного распределения $f_{C}(y|x,\theta)$ и предельного распределения $f_{M}(x)$:
\begin{equation}
f_{J}(y,x)=f_{C}(y|x,\theta)f_{M}(x)
\end{equation}

Суть простой случайно выборки заключается в равномерном отборе $(Y,X)$ из всей совокупности.


\subsection{Многошаговые опросы}

Одной из альтернатив является многошаговая стратифицированная кластаризация выборки, также называемая сложный метож выборки. Такой подход используется в крупномасштабных обследованиях. 
	
	
Данный подход имеет свои преимущества. Во-первых он является более эффективным с точки зрения затрат, поскольку он сокращает географическую разбросанность. Например, избыточность малых субпопуляций демонстрирует некоторые соответстующие характеристики становятся возможными, в то время как случайная выборка генеральной совокупности будет давать слишком мало наблюдений для поддержки надежных результатов. Недостатки данного метода заключаются в том, что стратифицированная выборка уменьшает различия между индивидами, которые имеют важное значение для большей точности.


Большое количество литературы про выборочное обследование  фокусируется на многоступенчатых опросах, которые последовательно разбивают генеральную совокупность на следующие категории:

\begin{enumerate}
\item Страта: Непересекающиеся подгруппы, которые заполняют всю генеральную совокупность.
\item Первичная выборочная единица (ПВЕ): Непересекающиеся подмножество страты.
\item Вторичная выборочная единица (ВВЕ): включается в ПВЕ.
\item Окончательная выборочная единица (ОВУ): Финальная единица, выбранная для опроса, может быть как домохозяйство, так и группа домохозяйств.
\end{enumerate}


В качестве примера, стратой может быть различные штаты или провинции в стране, ПВЕ может быть регионом в штате или провинции, а ОВУ может быть небольшой группой домохозяйств в том же районе.
	
	
Обычно опрашиваются все страты, так что, например, все штаты будут включены в выборку. Но не все их ПВЕ и ВВЕ включаются в выборку. В двухступенчатой выборке обследуемые из ПВЕ взяты случайным образом, как и обследуемые из ОВУ, а в многоступенчатой выборке появляются промежуточные единицы выборки.


Следствием этих методов отбора является то, что разные домохозяйств будут иметь различные вероятности попадания в выборку. Поэтому выборка не является репрезентативной для генеральной совокупности. Многие обследования дают веса выборки, которые предназначены, чтобы быть обратно пропорциональна вероятности в выборку, в этом случае эти веса могут быть использованы для получения объективной оценки характеристик генеральной совокупности.
	
	
Данные исследования могут быть сгруппированы в связи с выбором большого количества домохозяйств в том же самом регионе. Наблюдения в одном кластере могут быть зависимы или коррелируемы, поскольку они могут зависеть от некоторых наблюдаемых или ненаблюдаемых факторов, которые могут влиять на все наблюдения в страте. Например, в пригороде могут доминировать домохозяйства с высоким доходом или домохозяйства, которые являются относительно однородными в некотором измерении их предпочтений. Данные из этих домохозяйств будут иметь тенденцию к коррелируемости. Статистический вывод при условии игнорирования корреляции между выборочными наблюдениями приводит к ошибочным оценкам дисперсии, которые меньше, чем в случае правильной формулой. Эти вопросы рассматриваются более подробно в разделе 24.5. Двух-шаговая  много-шаговая выборка усложняет вычисление стандартных ошибок.
	
	
Таким образом, (1) стратификация с различными частотами страт означает, что выборка не репрезентативная; (2) веса выборки, которые обратно пропорциональны вероятности попадания в выборку, могут быть использованы для получения объективной оценки характеристик населения, и (3) кластеризация может привести к корреляции наблюдений и занижению истинных оценок стандартной ошибки.
	
	
\subsection{Смещение выборки}

Если выборка случайна, то распределение вероятности данных такое же, как распределение генеральной совокупности. Некоторые отклонения от случайной выборки вызывают расхождения между этими двумя распределениями, это и называют смещением выборки. Распределение данных отличается от распределения генеральной совокупности таким образом, что зависит от характера отклонения от случайной выборки. Отклонение от случайной выборки происходит потому, что иногда более удобно или экономически эффективно получить данные из генеральной совокупности даже если они не являются репрезентативными. Рассмотрим теперь несколько примеров таких отклонений, начиная со случая, в котором нет никаких отклонений от случайности.

\begin{center}
Экзогенная выборка
\end{center}


Экзогенные выборки на основе данных обследований возникают тогда, когда аналитик разделил имеющуюся выборку на подвыборки основаные лишь на множестве экзогенных переменных $x$, но не на переменной отклика. Например, в исследовании госпитализаций в Германии, Geil и соавт. (1997) разделили данные на две категории: имеющие хронические заболевания и не имеющие. Также, классификация по категориям доходов является обычным явлением, как и сегментирование существующей выборки по полу, здоровью, социально-экономическому статусу. В допущение экзогенной выборки входит то, что распределение вероятностей экзогенных переменных не зависит от $y$ и не содержит информацию об интересующих параметров генеральной совокупности $\theta$. Таким образом, можно не учитывать предельные распределения экзогенных переменных и основываться на оценке условного распределения $f(y|x,\theta)$. Конечно, предположение может быть ошибочным и наблюдаемое распределение результата переменной может зависеть от выбранной сегментации, которая может коррелировать с результатом, что вызывает отклонение от экзогенной выборки.



\begin{center}
Выборка на основе ответов
\end{center}

Выборка на основе ответов происходят, если вероятность включения отдельного индивида в выборку зависит от ответа,  сделанного данным индивидом. В этом случае выборка происходит по правилам, определенным в условиях эндогенных переменных в стадии изучения.


Можно привести три примера: (1) Исследование влияния отрицательного подоходного налога или помощи семьям с детьми-иждивенцами на предложение труда тех, кто за чертой бедности из числа опрошенных. (2) Исследование определяющих факторов выбора общественного транспорта, из числа опрошенных пользователей общественного транспорта (субпопуляции). (3) Исследование факторов влияющих на количество посещений сайта отдыха, из тех, кто по крайней мере имел одно посещение.


Снижение затрат на исследования дают важную мотивацию для использования выборки на основе предпочтение, чем  простой случайной выборки. Будет необходимо очень большая случайная выборка для создания достаточного числа наблюдения (информации) о результатах относительно редких выборов, а значит, дешевле собрать выборку от тех, кто на самом деле сделал выбор.


Практическое значение этого в том, что последовательные оценки параметров генеральной совокупности $\theta$ больше не могут быть осуществлен с использованием условной плотностью $f(y|x)$. Эффект от выбора схемы сбора данных также должен быть принят во внимание. Эта тема обсуждается далее в разделе 24.4.


\begin{center}
Length-Biased Sampling
\end{center}

Length-Biased Sampling показывает, как смещения могут возникнуть в результате выборки из одной генеральной совокупности, чтобы сделать выводы о другой генеральной совокупности. Строго говоря, это не столько примером случайности выборки, а скорее выборка  из <<неправильной>> генеральной совокупности.


Эконометрические исследования модели переходов время, проведенное в состоянии $j$ индивида $i$ до перехода в другую точку назначения $s$. Например, когда $j$ соответствует безработице, а $s$ занятости. Данные, используемые в таких исследованиях могут исходить из одного или нескольких возможных источников. Одним из источников является выборка лиц, которые являются безработными на определенную дату, другой является выборка тех, кто в составе рабочей силы, независимо от их текущего состояния, а третий является выбор лиц, которые либо входят, либо выходят из стадии безработицы в течение определенного периода времени. Каждый тип схемы выборки основан на различных концепциях соответствующей группы населения. В первом случае человек соответствующей группы населения являются безработным, во втором --- рабочей силой, а в третьем --- переходящим в статус занятости. Эта тема обсуждается далее в разделе 18.6.


Предположим, что целью исследования является вычислить показатель средней продолжительности безработицы. Это средняя продолжительность, случайно выбранного человека, в состоянии безработицы, если он или она становятся безработными. Ответ на этот простой вопрос по-видимому может меняться в зависимости от того, каким способом получены данные. Распределение потоков завершенной длительности в целом довольно сильно отличается от распределения запасов. Когда мы выбираем запасы, вероятность нахождения в выборке выше у лиц с более длительной продолжительностью. Когда мы выбираем потоки, вероятность не зависит от времени, проведенного в данным состоянии. Это хорошо известный пример Length-Biased Sampling, в котором оценка полученная на основе выборки запасов является смещенной оценкой средней продолжительности безработицы.

Следующая простая схема может прояснить идею:

\[ \bullet \: \circ \: \longmapsto \: \begin{aligned} \bullet &\: \bullet \\ \bullet &\: \circ \end{aligned}  \: \longmapsto \: \circ \: \circ \: \bullet \]

Здесь мы используем символ $\bullet$ для обозначения тех, кто медленно переходит из одного состояния в другое, и символ $\circ$ для обозначения тех, кто быстро переходит. Предположим, что два типа в равной степени представлены в потоке, но $\bullet$ остаётся дольше, чем $\circ$. Тогда население имеет более высокую долю тех, кто переходит медленно. Наконец, на выходе из состояния мы имеем более высокую долю тех, кто быстро переходит. 


Суть этого примера не в том, что выборка потоков является лучше, чем выборка запасов. Скорее суть в том, что в зависимости от вопроса, выборка потоков может не дать случайную выборку соответствующей генеральной совокупности.



\subsection{Смещение в сторону случайной выборки}

Рассмотрим следующую задачу. Исследователь заинтересован в измерении эффекта обучения, обозначается $z$, на  заработной платы, обозначается $y$, при заданных характеристиках работника, обозначаемая $x$. Переменной $z$ принимает значение 1, если работник прошел обучение и 0 в противном случае. Наблюдения доступны на $(x, D)$ для всех работников, а на $y$ только для тех, кто прошел подготовку $(D = 1)$. Необходимо сделать вывод о средней эффективности обучения на заработную плату случайно выбранного работник с известными характеристиками, который в настоящее время не подготовленные $(D = 0)$. Проблема выборки касается трудности нахождения правильного вывода.


Манский (1995), который рассматривает это как проблему идентификации, определяет проблему выбора, формально следующим образом:


Это проблема идентификации условного распределения вероятностей случайных данных из выборки, в котором реализация условных переменных всегда наблюдаема, но реализации результатов подвергаются цензурированию.


Пусть $y$ результат, который должен быть предсказан и условные переменные обозначаются через $x$. Переменной $z$ является показателем того, что принимает значение 1, если результат наблюдается и 0 в противном случае. Переменные $(D, x)$ всегда наблюдается, а $y$ наблюдается только при $D = 1$, поэтому Манский считает это цензорным процессом отбора. Этот процесс отбора не идентифицирует $Pr[y|x]$, как видно из

\begin{equation}
Pr[y|x]=Pr[y|x,D=1]Pr[D=1|x]+Pr[y|x,D=0]Pr[D=0|x].
\end{equation}

Процесс отбора можно выделить три из четырех слагаемых в правой части, но не дает никакой информации о  
$Pr[y|x, D = 0]$. Потому что

\[
E[y|x]=E[y|x,D=1]Pr[D=1|x]+E[y|x,D=0]Pr[D=0|x],
\]
всякий раз, когда вероятность $Pr[D=0|x]$ является положительным, имеющиеся эмпирические данные не накладывают никаких ограничений на $E[y|x]$. Следовательно, процесс отбора может определить $Pr[y|x]$ только по неизвестному значению $Pr[y|x,D=0]$. Чтобы узнать что-нибудь о $E[y|x]$, ограничения должны быть размещены на $Pr[y|x]$.


Альтернативные подходы в решении этой проблемы обсуждаются в разделе 16.5.


\subsection{Качество данных наблюдений}

Качество данных зависит не только от конструкции выборки и инструмента обследования, но ответов индивидов. Эта черта особенно относится к данным наблюдений. Мы рассмотрим несколько способов, в которых качество выборки данных может быть нарушено. Некоторые из этих проблем (например, истощение) также могут возникнуть при использовании других типов данных. Эта тема пересекается со смещением выборки.


\begin{center}
Проблема в случае, когда нет ответов на опрос
\end{center}


Опросы, как правило, добровольны, и стимул для участия может варьироваться в соответствии с систематической характеристиками домохозяйств и типа поставленного вопроса. Физические лица могут отказаться отвечать на некоторые вопросы. Если есть связь между систематическим отказом отвечать на вопрос и характеристиками индивида, то возникает вопрос о репрезентативности опроса после поправки на не полученные ответы. Если неполучение ответов игнорируется, а  анализ проводится только с использованием данных от респондентов, как это будет влиять на оценку интересующих параметров?


Неполучения ответов является частным случаем проблемы отбора, упомянутых в предыдущем разделе. Оба связаны со смещением выборки. Чтобы проиллюстрировать, как это приводит к искажению выводов рассмотрим следующую модель:

\begin{equation}
\begin{bmatrix} y_{1} \\ y_{2} \end{bmatrix} \Biggm| x,z \sim N \Biggm( \begin{bmatrix} x'\beta \\ z'\gamma \end{bmatrix}, \begin{bmatrix} \sigma^2_{1} & \sigma_{12} \\ \sigma_{12} &  \sigma^2_{2}\end{bmatrix}\Biggm),
\end{equation}
где $y_{1}$ является непрерывной случайной величиной (например, расходы), которые зависят от $x$, а $y_{2}$ является скрытой переменной, которая измеряет <<склонность к участию>> в опросе и зависит от $z$. Домохозяйства участвует, если $y_{2}>0$, в противном случае не участвуют. Переменных $x$ и $z$ предполагаются экзогенными. Формулировка позволяет , чтобы $y_{1}$ и $y_{2}$  коррелировали. 


Предположим, мы оцениваем $\beta$ из данных, предоставленных участниками, методом наименьших квадратов. Является ли эта оценка несмещенной в случае, когда есть незаполненные опросы? Ответ в том, что если неучастие является случайным и независимым от $y_{1}$, то нет никакой смещённости, а иначе будет.

Аргумент в следующем:

\[
\begin{aligned}
&{\beta}=[X'X]^{-1}X'y_{1}, \\
E[\hat{\beta}-\beta]&=E\Bigm[ [X'X]^{-1}X'E[y_{1}-X\beta|X,Z,y_{2}>0] \Bigm],
\end{aligned}
\]

где первая строка даёт формулу для оценки $\beta$, а вторая --- дает смещённость. Если $y_{1}$ и $y_{2}$ независимы, при условии $X$, $Z$ и $\sigma_{12}=0$, то

\[
E[y_{1}-X\beta|X,Z,y_{2}>0]=E[y_{1}-X\beta|X,Z]=0,
\]
и здесь нет смещённости.

\begin{center}
Пропущенные или неправильно внесённые данные
\end{center}

Обследование респондентов, занимает обширную анкету и не обязательно, что все отвечают на все вопросы, и даже если они это делают, ответы могут быть умышленно или случайно ложными. Предположим, что выборочное обследование пытается получить вектор ответов обозначенный как $x_{i}=(x_{i1},\dots,x_{iK})$ от $N$ участников, $i= 1,\dots,N$. Предположим теперь, что если человек не в состоянии предоставить информацию об одном или более элементов $x_{i}$, то весь вектор отбрасывается. Первая проблема, в результате отсутствия данных является то, что размер выборки уменьшается. Второй потенциально более серьезной проблемой является то, что недостающие данные могут привести к смещение как и в случае неправильной выборки. Если отсутствуют данные на систематической основе, то оставшаяся выборка не может быть репрезентативной. Форма смещения отбора может быть вызвана любым систематическим характером пропущенных данных. Например, респонденты с высоким доходом систематически не отвечают на вопросы о доходах. Глава 27 обсуждает проблему пропущенных данных и её решение.


Измерение ошибки в ответах являются широко распространенной проблемой. Они могут возникнуть в результате различных причин, в том числе неправильные ответы, связанные с небрежностью, намеренный неправильный ответ, ошибочная интерпретация данных. Более важным источником погрешности измерений является несовершенное прокси для соответствующей теоретической концепции. Последствия таких ошибок измерения является основной темой главы 26.


\begin{center}
Sample Attrition
\end{center}

В случае панельных данных обследование включает в себя повторные наблюдения на множество индивидов. В этом случае мы можем имеем:

\begin{itemize}
\item полные ответы во все периоды (полное участие),
\item нет ответов в первом периоде и во всех последующих периодах (полное неучастие), или
\item частичный ответ в смысле ответа на начальных периодов, но отсутствие ответа в более поздние периоды (неполное участие) - ситуация Sample Attrition
\end{itemize}


Sample Attrition приводит к недостающим данным, наличие любой формы неслучайной <<missingness>> приведет к проблеме выборки, которая уже упоминалась. Это можно интерпретировать как особый случай задачи отбора данных. Пример этого кратко обсуждается в разделах 21.8.5 и 23.5.2.

\subsection{Типы данных наблюдения}

Кросс-секционные данные получены при наблюдении $w$, для выборки $S_{t}$ для некоторых $t$. Хотя это обычно нецелесообразно собирать все домохозяйства в одну выборку в один момент времени, кросс-секционные данные предоставляют характеристики каждого элемента подмножества генеральной совокупности, которая будет использоваться, чтобы сделать вывод о генеральной совокупности. Если генеральная совокупность стационарная, то выводы сделанные о $\theta_{t}$ с использованием $S_{t}$ могут быть справедливыми и для $t'\neq t$. Если существует значительная зависимость между прошлым и текущим поведением, то лонгитюдные данные, необходимые для идентификации отношений, представляющих интерес. Например, ранее принятые решения могут повлиять на текущие результаты; инерции или привычка могут стимулировать покупки, но такая зависимость не может быть смоделирована, если история покупок недоступна. Это одно из ограничений, налагаемых кросс-секционными данными.


Повторные кросс-секционные данные получены последовательностью независимых выборок $S_{t}$, взятых из $F(w_{t},\theta_{t}), t=1,\dots,T$. Поскольку выборка не пытается сохранить ту же единицу в себе, то сведения о динамических зависимостях теряется. Если генеральная совокупность стационарная, тогда повторные кросс-секционные данные получаются путем обработки выборок похожей на схему с возвращением от постоянной генеральной совокупности. Если же она нестационарная, повторные кросс-секции связаны таким образом, что зависят от того, как население меняется с течением времени. В таком случае целью является, сделать выводы о лежащих в основе постоянных (гипер)параметров. Анализ повторяющихся данных обсуждается в разделе 22.7.


Панель или лонгитюдные данные получают сначала выбрав образец $S$, а затем собрав наблюдения для последовательности периодов времени, $t=1,\dots,T$. Это может быть достигнуто путем опроса и сбора как нынешних, так и прошлых данных в то же время, или путем отслеживания предметов, когда они были введены в опросе. Это создает последовательность данных векторов ${w_{1},\dots,w_{T}}$, которые используются, чтобы сделать вывод о любом поведении населения или конкретной выборки лиц. Соответствующие методологии в каждом случае могут быть не одинаковыми. Если данные взяты из нестационарного населения, соответствующей целью должен быть вывод о (гипер)параметрах суперпопуляции.


Некоторые из недостатков этих типов данных очевидны. Кросс-секционные и повторные кросс-секционные данные не обеспечивают в целом подходящие данные для моделирования межвременной зависимости результатов. Такие данные пригодны только для моделирования статических отношений. В отличие от лонгитюдных данных, особенно если они охватывают достаточно длительный период времени, являются подходящими для моделирования статических и динамических отношений.


Лонгитюдные данные также имеют проблемы. Первая проблема в  не репрезентативности панели. Проблемы в выводах о генеральной совокупности на основе лонгитюдных данных становятся всё более сложными, если совокупность не является стационарной. Для анализа динамики поведения, сохраняя оригинальные домашние хозяйства в панели как можно дольше является привлекательным вариантом. На практике лонгитюдные наборы данных страдают от проблемы <<sample attrition>>, возможно, из-за <<sample fatigue>>. Это просто означает, что респонденты не продолжают предоставлять ответы на вопросы. Это создает две проблемы: (1) панели становятся несбалансированными и (2) существует опасность того, что домохозяйство не может быть <<типичной>> и, что образец становится не репрезентативным для генеральной совокупности. Когда имеющиеся данные выборки не являются случайной выборкой из совокупности, результаты, основанные на данных различных типов будут восприимчивы к смещению в разной степени. Проблема <<sample fatigue>> возникает потому, что с течением времени становится все более трудно сохранить индивидов внутри панели или они могут быть <<потеряны>> по некоторым другим причинам, таким как изменение местоположения. Эти вопросы рассматриваются далее в книге. Анализ лонгитюдных данных тем не менее может предоставить информацию о некоторых аспектах поведения единиц выборки, хотя экстраполяция поведения генеральной совокупности не может быть простой.


\section{Данные социальных экспериментов}

Наблюдаемые и экспериментальные данные различны, так как экспериментальная среда в принципе может быть тщательно контролируема и управляема. В отличие от этого, наблюдаемые данные создаются в неконтролируемой среде, оставляя открытой возможность наличие сопутствующих факторов, которые усложнять определение причинно-следственных связей. Например, при попытке изучения заработка  от образованности по данным наблюдений, надо признать, что количество лет обучения отдельного индивида является результатом принятия решения, и, следовательно, никто не может рассматривать уровень школьного образования, если в экспериментальных условиях.


В социальных науках, данные аналогичные экспериментальным данным получают либо от социальных экспериментов, определены и описаны более подробно ниже, или в  <<лабораторных>> экспериментах на небольших группах добровольцев , которые имитируют поведение экономических агентов в реальной жизни. Социальные эксперименты относительно редкое явление, и все же экспериментальные концепции, методы и данные служат основой для оценки эконометрических исследований, основанных на данных наблюдений.


В этом разделе представлен краткий обзор методологии социальных экспериментов, характер данных, вытекающих из них, и некоторые проблемы и вопросы эконометрической методологии, которые они производят.


Главной особенностью экспериментальной методологии является сравнение между результатами случайно выбранной экспериментальной группы, которая подвергается <<воздействию>> с теми, кто в контрольной группе. В хорошем эксперименте уделяется много времени для сравнения контрольных и экспериментальных групп, чтобы избежать возможных смещений в результатах. Такие условия не могут быть реализованы в наблюдательной среды, что ведет к возможному отсутствию идентификации причинных параметров, представляющих интерес. Иногда, экспериментальные условия приближенно можно повторить в данных наблюдений. Рассмотрим, например, два соседних региона или государства, в которых разная политика установления минимальной зарплаты, создавая условия естественного эксперимента, в котором наблюдения в состоянии <<воздействия>> можно сравнить с теми, которые в состоянии <<контроля>>. Структура данных естественного эксперимента также привлекает внимание эконометристов.


Социальный эксперимент включает экзогенные изменения в экономической среде перед набором испытуемых, которая разбита на одно подмножество, которое получает экспериментальное воздействие и другое, которое служит в качестве контрольной группы. В отличие от наблюдательных исследований, в которых изменения в экзогенных и эндогенных факторов часто смятение, хорошо продуманный социальный эксперимент направлен на выделение воздействованных переменных. В некоторых экспериментальных конструкциях может не быть никакой контрольной группы, но применяются разные уровни воздействия, в этом случае становится возможным в принципе оценить всю поверхность экспериментальных результатов.


Основной задачей социального эксперимента является оценка влияния фактических или потенциальных социальных программ. Потенциальная модель результата раздел 2.7 обеспечивает соответствующую подготовку для моделирования влияния социальных экспериментов. Несколько альтернативных мер воздействия были предложены и будут обсуждаться в главе, посвященной оценке этих программ (глава 25).


Burtless (1995) обобщает случай социальных экспериментов, отмечая при этом некоторые потенциальные ограничения. В сопутствующей статье Хекмана и Смит (1995) сосредотачиваются на ограниченности реальных социальных экспериментов, которые были реализованы. 


\subsection{Характеристики социального эксперимента}


Социальные эксперименты мотивированы политическим вопросам о том, как предметы будут реагировать на тип политики, который никогда не был опробован и, следовательно, для которого нет данных наблюдений. Идея социального эксперимента заключается в привлечении группы участников, некоторые из которых случайным образом распределены в группу воздействия, а остальные в контрольную группу. Разница между ответами тех, кто в группе воздействия, и тех, кто в контрольной группе, представляет собой оценку влияния политики. Схема стандартной экспериментальной конструкции,  показано на рисунке 3.1.


Термин "экспериментальные данные" относится к группе, получавшей воздействие, "контрольный" к группе не получавшее воздействие, а "рандомизация" к процессу деления лиц на две группы.


Рандомизированные исследования были введены в статистике Р. Фишером (1928) и его сотрудниками. Типичным сельскохозяйственным экспериментом является испытание, в котором новые методы воздействия, такие как удобрения будут применяться для растений, растущих на случайно-выбранных частях земли, а затем результаты будут сравниваться с теми растениями, которые входили в контрольную группу. Если эффект всех других различий между экспериментальной и контрольной группой может быть устранен, по оценкам, разница между этими двумя наборами результатов может быть отнесены к воздействию. В простейшей ситуации можно сконцентрировать внимание на сравнение средних результатов, воздействованной и контрольной группы.


Хотя в сельском хозяйстве и биомедицинских науках, методологии рандомизированных экспериментов давно установлены, в области экономики и социальных наук это явление новое. Более того, оно привлекательно для изучения результатов политические изменения, для которых нет наблюдательных данных. Рандомизированные эксперименты также приводят к более сильному изменению в политике переменных и параметров, чем их наличие в данных наблюдений, тем самым облегчая для выявления и изучения ответов на политические изменения.


Социальные эксперименты все еще довольно редки за пределами Соединенных Штатов, отчасти потому, что они дорогие. В США число таких экспериментов росло с начала 1970-х. Таблица 3.1 суммирует особенности некоторых относительно известных примеров; для более широкого охвата см. Burtless (1995).


Эксперимент может производиться либо на кросс-секционных или на лонгитюдных данных, из соображения стоимости, не используются временные ряды. В случае, когда эксперимент длится несколько лет и имеет несколько этапов и / или географических мест, промежуточный анализ на основе <<неполны>> встречается очень часто (Ньюхаусом соавт., 1993).



\subsection{Плюсы социального эксперимента}


Burtless (1995) исследовал преимуществ социальных экспериментов с большой ясностью. Главное преимущество проистекает из рандомизированных исследований, которые устраняют любые корреляции между наблюдаемыми и не наблюдаемыми характеристиками участников эксперимента.

\begin{table}[h]
\begin{center}
\caption{\label{tab:pred}Черты некоторых экспериментов}
\begin{tabular}[t]{llcll|}
\hline
\bf{Эксперимент} & \bf{Тестирумые переменные} & \bf{Целевая аудитория} \\
\hline
Rand Health Insurance Experiment (RHIE), 1974–1982 & Планы медицинского страхования с различной ставкой и с разным уровнем максимальных расходы за свой счет  & Индивиды и домохозяйства со средним и низким уровнем дохода \\
Negative Income Tax (NIT), 1968–1978 & Планы NIT с альтернативными гарантиями дохода и налоговой ставкой & Индивиды и домохозяйства со средним и низким уровнем дохода \\
Job Training Partnership Act (JTPA), (1986–1994)& Ассистенты по поиску работы, тренинги, финансируемый JTPA & Абитуриенты и безработные люди\\
\hline
\end{tabular}
\end{center}
\end{table}

Следовательно, вклад воздействия по итогам разницы между обработанной и контрольной группой может быть оценена без вмешивающихся факторов, даже если человек не может контролировать вмешивающиеся переменные. Наличие корреляции между воздействованными и вмешивающимися переменными часто страдает от наблюдательных исследований и усложняет выводы причинно-следственных результатов. С другой стороны, экспериментальные исследования, проведенные в идеальных условиях могут производить последовательную оценку средней разницы в результатах обработанной и необработанной группы без особой вычислительной сложности.


Если, однако, результат зависит от воздействия также, как и от других наблюдаемых факторов, то контролирование будет улучшать точность влияния оценок.


Даже если данные наблюдений доступны, создание и использование экспериментальных данных имеет большую привлекательность, потому что это дает возможность экзогеннизации политической переменной, а рандомизация воздействованных переменных может привести к большому упрощению статистического анализа. Выводы основанные на данных наблюдений часто не хватает общности, поскольку они основаны на неслучайной выборки из генеральной совокупности --- проблема смещения отбора. Примером может служить вышеупомянутое исследование RHIE, в котором основной акцент делается на чувствительность к ценам спроса на медицинские услуги. Наличие медицинской страховки влияет на цену медицинских услуг и тем самым на её использование. Можно, конечно, использовать данные наблюдений для моделирования отношения между спросом на медицинские услуги и уровнем страхования. Тем не менее, такой анализ подлежит критике, что уровень медицинского страхования не должен рассматриваться как экзогенный. Теоретический анализ показывает, что спрос на страхование здоровья и здравоохранения определяется совместно, поэтому причинно-следственная связь разно направлена. Этот факт может потенциально сделать затруднительным определение роли медицинского страхования. Медицинское страхование как экзогенная переменная смещает оценку чувствительности к ценам. Однако в экспериментальной установке участвующие домохозяйства могут иметь страховой полис, что делает его экзогенной переменной. Как только ключевые переменные становятся экзогенными, направление причинно-следственной становится очевидными, и влияние воздействия может быть изучено однозначно. Кроме того, если эксперимент не содержит некоторые из проблем, о которых мы говорим далее, это значительно упрощает статистический анализ относительно того, что часто возникает необходимость в данных обследования.




\subsection{Ограничения социального эксперимента}

Применение негуманной методологии, вызывает оживленную дискуссию в литературе. См. особенно Хекмана и Смит (1995), которые утверждают, что многие социальные эксперименты могут страдать от ограничений, которые применяются к наблюдательных исследований. Эти вопросы касаются общих моментов, таких как достоинства экспериментальных наблюдений по сравнению с наблюдательной методологией, а также конкретные вопросы, касающиеся предубеждения и проблем, связанных с использованием человека в экспериментах. Некоторые вопросы рассматриваются более подробно в последующих главах.


Социальные эксперименты являются очень дорогостоящими для запуска. Иногда, они не соответствуют <<чистым>> случайным  исследованиям. Таким образом, результаты таких экспериментов не всегда однозначны и легко интерпретируемы или свободны от предубеждений. Если воздействуемая переменная имеет много альтернативных интересов, или если экстраполяция является важной задачей, то должна быть собрана очень большая выборка, чтобы обеспечить достаточное изменение данных и точно измерить эффект воздействия вариации. В этом случае стоимость эксперимента также будет увеличиваться. Если фактор стоимости не позволяет провести достаточно большой эксперимент, его полезность по сравнению с наблюдательным исследованием может быть сомнительной; см. работы Розена и Стаффорд в Хаусман и Вайса (1985).


К сожалению, проектирование некоторых социальных экспериментов является неправильным. Хаусман и Вайс (1985) утверждают, что данные из Нью-Джерси эксперимент отрицательного подоходного налога был подвержены эндогенным изменениям, которые они описывают следующим образом:

$\dots$ Причиной эксперимента, по рандомизации, является устранение корреляции между переменными воздействия и другими детерминантами реакции переменной, которая находится в стадии изучения. В каждой эксперименте, в котором исследуется доход, выборка производится отчасти на основе зависимой переменной. В общем, группы, имеющие право на выбор - на основе семейного положения, расы, возраста главы семьи, и т.д. - была стратифицирована на основе дохода (и других переменных), а индивиды, были отобраны из этих слоев. (Хаусман и Вайс, 1985, стр. 190-191)


Авторы приходят к выводу, что, в присутствии эндогенного стратификации, объективной оценки результатов лечения не совсем просто получить. К сожалению, полностью рандомизированное исследование, в котором назначение лечения в случайно выбранный из экспериментальной группы населения не зависит от дохода будет гораздо более дорогостоящим и может оказаться невозможным.


Есть несколько других вопросов, которые вытекают тз идеальной простоты рандомизированных экспериментов. Во-первых, если экспериментальные участки выбираются случайно, сотрудничество администраторов и потенциальных участников на этом этапе не потребуется. Если этого не последует, то альтернативные места воздействия, где такое сотрудничество может быть получено будут заменены, тем самым ставя под угрозу принцип случайного назначения, см. Хотц (1992).


Второй проблемой является проблема отбора, потому что участие в эксперименте является добровольным. По этическим причинам есть много экспериментов, которые просто не могут быть совершены (например, случайное распределение студентов в годы обучения). В отличие от медицинских экспериментов, которые могут использовать официальные проток, в социальных экспериментов экспериментуемые знают, являются ли они группой воздействия или контрольной группой. Если решение об участии не коррелирует с $x$ или $\epsilon$, анализ экспериментальных данных упрощается.


Третья проблема заключается sample attrition вызванное тем, что испытыемый выбывает из эксперимента после ее начала. Даже если первоначальная выборка была случайной на истощение неслучайных вполне может привести к проблеме похожей на истощение смещения в панелях. Наконец, существует проблема эффект Hawthorne. Термин происходит в социальной психологии исследования, проведенного совместно Гарвардской высшей школы делового администрирования и управления Западной Электрической компании в Хоторне; Hawthorne, который работал в Чикаго с 1926 по 1932 год. Человек, в отличие от неодушевленных предметов, может изменять или адаптировать своё поведение во время участия в эксперименте. В этом случае изменение ответа, наблюдаемого в экспериментальных условиях не может быть отнесено исключительно к воздействию.


Хекман и Смит (1995) упоминают ряд других трудностей в осуществлении рандомизированного воздействия. Потому, что администрация социального эксперимента включает в себя бюрократию, есть возможность для предубеждений. Рандомизированное смещение происходит, если задание представляет систематическое различие между экспериментальными участника и участника в процессе ее эксплуатации. Хекман и Смит документировали возможности такого смещения в реальных экспериментах. Другой тип смещения, называется погрешностью замещения, когда вводимое управления может получать некоторые формы лечения, которое заменяет экспериментальное воздействие. Наконец, анализ социальных экспериментов неизбежно несёт в себе характер частичного равновесия. 


В частности, ключевым вопросом является возможность экстраполяции результатов эксперимента, на всю генеральную совокупность. Если эксперимент проводится в качестве экспериментальной программы в небольших масштабах, но есть намерение предсказать влияние политики, более широкое применение, то очевидным ограничением является то, что пилотная программа не может включать более широкого воздействия на переменные. Широко применимое воздействие может изменить экономическую среду, достаточно признать недействительным прогнозы от частичного равновесия. 


Таким образом, социальные эксперименты, в принципе, являются данными, которые проще анализировать и понимать с точки зрения причинно-следственных связей, чем данные наблюдений. Однако это зависит от дизайна эксперимента. Плохо сконструированный эксперимент создаёт свои статистические сложности, которые влияют на точность выводов. Социальные эксперименты принципиально отличаются от тех, которые в биологии и сельском хозяйстве, потому что человеку как правило, активный и и вперед смотрящий агент с личными предпочтениями, что усложняет процесс эксперимента.



\section{Данные естественного эксперимента}


Иногда, исследователь может иметь в наличии данные <<естественного эксперимента>>. Данный эксперимент происходит, когда подмножество генеральной совокупности подвергается экзогенным изменения в переменной, возможно, в результате изменения политики, которое обычно является эндогенным изменением. В идеале, источник вариации известен.


В микроэконометрики широко используются два способа применения идеи естественного эксперимента. Для конкретности рассмотрим простую регрессионную модель:

\begin{equation}
y=\beta_{1}+\beta_{2}x+u,
\end{equation}

где $x$ эндогенный объясняющие переменные коррелируемые с $u$.


Предположим, что существует экзогенные последствия, которые измененяют $x$. Примерами такого внешнего вмешательства могут быть административные правила, непредвиденное законодательство, природные явления (см. таблицу 3.2). Экзогенные вмешательство создает возможность для оценки ее воздействия путем сравнения поведения влияние группы как до, так и после вмешательства. То есть, <<естественное>> сравнение генерируется по событию, которое облегчает оценку $\beta_{2}$. Оценка упрощается, так как $x$ можно рассматривать как экзогенный.


Второй способ, в котором естественный эксперимент может помочь с выводом является создания естественных инструментальных переменных. Пусть $z$ это переменная, которая коррелирует с $x$, или, возможно, причинно связана с $x$, и не коррелирует с $u$. Тогда инструментальные переменные оценки $\beta_{2}$ выражаются через ковариацию

\begin{equation}
\hat{\beta_{2}}=\frac{Cov[z,y]}{Cov[z,x]}
\end{equation}
(см. раздел 4.8.5). В наблюдаемых данных сложно найти инструментальные переменные, но они легко возникают в случае естественного эксперимента. Мы рассмотрим первый случай в следующем разделе; тема естественной генерации инструментов будет рассмотрена в главе 25.

\subsection{Естественное экзогенное воздействие}


Такие данные являются менее дорогими для сбора и они также позволяют исследователю оценить роль некоторых специфических факторов, как в контролируемом эксперименте, потому что <<природа>> содержит постоянные изменения, связанные с другими факторами, которые не имеют непосредственного интереса. Такие естественные эксперименты привлекательны тем, что они создают группу воздействия и контрольную группу без каких-либо издержек и в реальных условиях. Способность естественного эксперимента поддерживать устойчивые выводы зависит, в частности, от того, предполагается ли экзогенное вмешательство, или его влияние достаточно велико, чтобы быть измеримыми.


Исследования основанные на естественных экспериментах имеют несколько потенциальных ограничений, важность которых в той или иной степени можно оценить только путем тщательного рассмотрения соответствующей теории, фактов и институциональных установок. Следуя Кэмпбеллу (1969) и Мейеру (1995), эти ограничения делатся на группы, влияющие на внутреннюю валидность исследования (т.е. выводы о политике воздействия взяты из исследования), и  на те, которые влияют на внешнюю валидность исследования (т.е. обобщение выводов на других членов генеральной совокупности).


Рассмотрим исследование изменения в политике, в котором делаются выводы из сравнения до и после вмешательств с использованием метода регрессии кратко описанной ниже и более подробно в главе 25. В любом исследовании будут опущены переменные, которые могут также изменяться в интервале времени между изменением политики и ее воздействием. Характеристики выборки лиц, таких как возраст, состояние здоровья, и их фактических или ожидаемых экономических условий также может меняться. Эти опущенные факторы будут непосредственно влиять на измеренное воздействие изменения политики. Можно ли обобщить  результаты на других членов совокупности будет зависеть от отсутствие предвзятости из-за неслучайной выборки, наличия существенных эффектов взаимодействия между изменением политики и ее установками, отсутствием исторических факторов, которые могут также на результат.


\subsection{Разница в разнице}


Одним из простых регрессионных методов является сравнение результатов в одной группе до и после вмешательства. К примеру, рассмотрим

\[
y_{it}=\alpha+\beta D_{t} +\epsilon_{it}, i=1, \dots, N, t=0,1,
\]

где $D_{t}=1$ в первом периоде (после вмешательства), $D_{t}=0$ в периоде 0 (до вмешательства), и $y_{it}$ измеряет результат. Регрессии оценивается по обобщенным данным, и даст оценку $\beta$. Легко показать, что она будет равна  средней разнице результата до и после вмешательства,


\[
\hat{\beta}=N^{-1}\sum_{i}(y_{i1}-y_{i0})=\bar{y_{1}}-\bar{y_{0}}.
\]

Сильным предположением является сопоставимость группы с течением времени. Это необходимо для идентифицируемости $\beta$. Если, например, мы допускаем изменение $\alpha$ между этими двумя периодами, $\beta$ больше не может быть идентифицирована. Изменения в $\alpha$ смешиваются с политикой воздействия.

Одним из способов улучшения предыдущей модели является включение дополнительной необработанной группы сравнения, то есть ту, которая не влияет на политику, и для которых имеются данные в обоих периодах. Используя обозначение Майера (1995), соответствующая регрессия выглядит так


\[
y^{j}_{it}=\alpha+\alpha_{1}D_{t}+\alpha^{1}D^{j}+\beta D^{j}_{t} +\epsilon^{j}_{it}, i=1, \dots, N, t=0,1,
\]
где $j$ является группой индексов, $D^{j}=1$, если $j=1$ и $D^{j}=0$, если $j=0$.  $D^{j}_{t}=1$, если $j$ и $t$ равняется 1 и$D^{j}_{t}=0$ в противном случае, а $\epsilon$ случайное возмущение с нулевым математическим ожиданием и постоянной дисперсией. Уравнение не включает ковариацию, но она может быть добавлена, и то, что не изменяется можно отнести к $\alpha$. Это означает, что для группы воздействия, модель будет выглядеть до вмешательства

\[
y^{1}_{i0}=\alpha+\alpha^{1}D^{1}+\epsilon^{1}_{i0},
\]

и после вмешательства

\[
y^{1}_{i1}=\alpha+\alpha_{1}+\alpha^{1}D^{1}+\beta +\epsilon^{1}_{i1}.
\]

Результат, поэтому такой

\begin{equation}
y^{1}_{i1}-y^{1}_{i0}=\alpha_{1}+\beta+\epsilon^{1}_{i1}-\epsilon^{1}_{i0}.
\end{equation}

Соответствующее уравнение для контрольной группы такого

\[
y^{0}_{i0}=\alpha+\epsilon^{0}_{i0}, \qquad
y^{0}_{i1}=\alpha+\alpha_{1}+\epsilon^{0}_{i1},
\]
а разница
\begin{equation}
y^{1}_{i1}-y^{1}_{i0}=\alpha_{1}+\epsilon^{0}_{i1}-\epsilon^{1}_{i0}.
\end{equation}

Эти уравнения первой разности включают первый период влияние $\alpha_{1}$, которое может быть устранено путём вычитания из уравнения (3.6) и (3.7):

\begin{equation}
(y^{1}_{i1}-y^{1}_{i0})-(y^{1}_{i1}-y^{1}_{i0})=\beta+(\epsilon^{1}_{i1}-\epsilon^{1}_{i0})-(\epsilon^{0}_{i1}-\epsilon^{1}_{i0}).
\end{equation}
Предполагая что $E[(\epsilon^{1}_{i1}-\epsilon^{1}_{i0})-(\epsilon^{0}_{i1}-\epsilon^{1}_{i0})]=0$, мы можем получить несмещённую оценку $\beta$ как среднее $(y^{1}_{i1}-y^{1}_{i0})-(y^{1}_{i1}-y^{1}_{i0})$. Этот метод использует разницу в разнице. Если присутствуют меняющиеся во времени регрессоры, то они могут быть включены в соответствующие уравнения и их различия появятся в уравнение регрессии (3.8).


Для простоты, наш анализ игнорирует возможность того, что остаются наблюдаемые различия в распределении характеристик между группами воздействия и контроля. Если так, то такие различия, необходимо контролировать. Стандартным решением является включение таких контрольных переменных в регрессию.


Примером исследования, основанного на естественном эксперименте является то, что исследовали Эшенфельтер и Крюгер (1994). Они оценивали влияния образования на уровень заработной платы у идентичных близнецов с разным уровнем образования. В этом случае обычный эксперимент, в котором людям заданы различные уровни образования просто не возможен. Тем не менее, некоторые экспериментальные типы контроля необходимы. Как объясняют авторы:


Наша цель заключается в обеспечении,того что корреляция наблюдаемая между образованием и ставок заработной платы не появляется из-за корреляции между образованием и способностью работника или другим признакам. Мы делаем это, пользуясь тем, что однояйцевые близнецы генетически идентичны и имеют схожие характеристики.


Данные о близнецах послужили основой для ряда других эконометрических исследований (Розенцвейг и Wolpin, 1980; Bronars и Grogger, 1994). Поскольку вероятность двойников в совокупности не является высокой, важным вопросом  является репрезентативность данной выборки. Одним из источников таких данных является перепись. Другой источник <<фестиваль близнецов>>, которые проводятся в Соединенных Штатах. Эшенфельтер и Крюгер (1994, с. 1158) сообщают, что их данные были получены из интервью, проведенном на 16-ом ежегодном фестивале близнецов.


Привлекательность использования данных близнецов заключается в том, что наличие общих эффектов от наблюдаемых и не наблюдаемых факторов может быть устранено путем моделирования различий между результатами близнецов. Например, Эшенфельтер и Крюгер оценили регрессионную модель разницы в ставках заработной платы между близнецами. Первая разность исключает эффекты от возраста, пола, этнической принадлежности, и так далее. Остальные объясняющие переменные различаются между школьным уровнем, который является переменной интереса, и переменной, такие как различия в годы пребывания в должности и семейном положении.


\subsection{Идентификация естественного эксперимента}

Естественный эксперимент образования имел бы полезное влияние на практике. Поощряя оппортунистической эксплуатации квази-экспериментальных данных, и с помощью моделирования структур, таких как POM главы 2, эконометрическая практика устраняет разрыв между наблюдений и экспериментальными данными. Понятия идентификации параметров, заключённые в рамки SEM будут расширены, чтобы включить определение меры, которое интересна с точки зрения политики. Основное преимущество использования данных из естественного эксперимента является то, что политика переменной может быть обоснованно рассматриваться как экзогенная. Однако при использовании данных естественного эксперимента, как и в случае социального эксперимента, выбор контрольной группы играет важную роль в определении достоверности выводов. Несколько потенциальных проблем, которые влияют на социальный эксперимент, такие как селективность и истощение смещённости, также останутся в случае естественных экспериментов. Эксперимент может применяться только к небольшой части генеральной совокупности, а также условия, при которых он происходит, не распространяются легко. Пример, приведенный в разделе 22.6 иллюстрирует этот момент в контексте разности в расностях.

\section{Практические соображения}


Хотя существует огромный спрос на микроданные, количество имеющихся баз данных можно пересчитать по пальцам. Мы предоставляем очень неполный список некоторых из очень известных американских баз. Для получения дополнительной информации, см. соответствующие веб-сайты для этих наборов данных. Многие из них позволяют загружать данные напрямую.


\subsection{Некоторые источники микроданных}

{\bf Панель изучения динамики доходов в (PSID)}: Основываясь на Центр по обзору исследований в Университете штата Мичиган, PSID является национальным опросом, который проводится с 1968 года. Сегодня она охватывает более 40 000 индивидов и собирает экономические и демографические данные. Эти данные были использованы для поддержки широкого спектра микроэконометрических анализов. Браун, Дункан и Стаффорд (1996) подводят последние разработки в PSID данных.


{\bf Текущее обследование населения (CPS)}: Это ежемесячная национальное обследования около 50000 домохозяйств, которая предоставляет информацию о характеристиках рабочей силы. Исследование проводилось в течение более 50 лет. Основные дополнения в выборке следовали каждому из десятилетних переписей. Для получения дополнительной информации об этом исследовании см. раздел 24.2. Оно является важным источником микроданных, которые поддержали многочисленные исследования особенностей рынка труда. Опрос был переработан в 1994 году (Поливка, 1996).


{\bf Национальное лонгитюдное обследование (NLS)}: NLS имеет четыре оригинальных когорты: NLS пожилых мужчин, NLS молодых мужчин, NLS пожилых женщины, и NLS молодых женщин. Каждый из первоначальных когорт является национальным ежегодным опросом более 5000 лиц, которые были неоднократно проинтервьюированы, начиная с середины 1960---х годов. Опросы содержат информацию об опыте работы каждого респондента, образование, обучение, доход семьи, состав семьи, семейное положение, и здоровье. Дополнительные данные возраста, пола и т.д. также имеются.


{\bf Национальное лонгитюдное обследование молодежи (NLSY)}: NLSY является национальным ежегодного опросом 12686 молодых мужчин и женщин, в возрасте от 14 до 22 лет, когда они были впервые обследованных в 1979 году; содержит три подвыборки. Данные обеспечивают уникальную возможность для изучения всего жизненного цикла большой выборки молодых людей, которые являются представителями американских мужчин и женщин, родившихся в конце 1950---х и начале 1960---х. Вторая NLSY началась в 1997 году.


{\bf Обследование доходов и участия в программах (SIPP)}: SIPP --- лонгитюдное обследование около 8000 единиц домохозяйств в месяц. Оно охватывает источники доходов, участие в правовых программах, корреляция между этими предметами, и отдельных вложений на рынке труда с течением времени. Это многопанельное обследования, при котором новая панелью внедряется в начале каждого календарного года. Первая панель SIPP была начата в октябре 1983 года. По сравнению с CPS, SIPP имеет меньше занятых и больше безработных лиц.


{\bf Изучение здравоохранения и пенсионеров (HRS)}: HRS -- лонгитюдное национального обследование, которое состоит из интервью с членами 7600 домохозяйств в 1992 году (респондентов в возрасте от 51 до 61) с последующим интервью каждые два года в течение 12 лет. Данные содержат огромное количество экономической, демографической и медико---санитарной информации.


{\bf Изучение уровня жизни Всемирным Банком (LSMS)}: Данные Всемирного Банка содержат обследование домашних хозяйств  <<по многим аспектам благосостояния домохозяйств, которые могут быть использованы для оценки благосостояния домохозяйств, для понимания поведения домохозяйств, и оценивания влияния различной политики правительства на условий жизни населения>> во многих развивающихся странах. Многие примеры использования этих данных можно найти в Deaton (1997) и в экономической литературе. Грош и Glewwe (1998) подчёркивают характер данных и предоставляют ссылки на исследования, в которых использовались эти данные.


{\bf Данные расчетные палаты}: Межвузовский консорциум политических и социальных исследований (ICPSR) обеспечивает доступ ко многим базам данных, в том числе PSID, CPS, NLS, SIPP, Национальным обследованиям медицинских расходов (NMES), и многим другим. Американское Бюро статистики труда обрабатывает данные CPS и NLS. Американское Бюро переписи населения обрабатывает данные SIPP. Американский Национальный центр статистики здравоохранения обеспечивает доступ ко многим наборам данных по здоровью. Полезный канал к европейским данным является архив Совет Европейских Социальных Наук (CESSDA), который содержит ссылки на несколько европейских национальных архивов данных.


{\bf Данные из архива журналов}: Для некоторых целей, таких как копирование опубликованных результатов для работы в классе, вы можете получить данные из архивов журнала. Два архива можно загрузить посредством интернет-браузера. The Journal of Business and Economic Statistics архив данных, используе большинство, но не все статьи, опубликованные в этом журнале. The Journal of Applied Econometrics содержит данные, относящиеся к большинству статей, опубликованных с 1994 года.



\subsection{Обработка микроданных}


Микроэкономический наборы данных, как правило, очень большой. Выборка из нескольких сотен или тысяч наблюдений являются типичным явлением, и даже десятки тысяч наблюдений не так удивительно. Распределения результатов чаще всего не поддаётся нормальному распределению, потому что эти данные являются дискретными. Обработка больших наборов данных создает некоторые проблемы обобщения и описания важных особенностей данных. Часто бывает полезно использовать одну вычислительную среду (программу) для извлечения, восстановление и подготовки данных, а другую для  оценки моделей.


\subsection{Подготовка данных}

Самая основная особенность микроэконометрического анализа является то, что процесс получения выборки, которая должна будет использоваться в исследовании, скорее всего, будет долгим. Важно точно документировать решения и выбор, на основе которого делалась <<очистка>> данных. Рассмотрим несколько конкретных примеров.


Одним из наиболее общих черт данных выборочного обследования является неполученные или частичные ответы. Проблемы неполучение уже обсуждались. Частичный ответ обычно означает, что некоторые части опросных анкет остались без ответа. Если при этом, некоторая часть из необходимой информации недоступна, описываемые наблюдения, будут удалены. Это называется listwise удаления. Если эта проблема возникает в значительном числе случаев, то она должна быть тщательным образом проанализирована и сообщена, потому что это может привести к не репрезентативности выборки и погрешности в оценке; этот вопрос анализируется в главе 27. 


Вторая проблема заключается в погрешности измерения отчетных данных. Микроэкономические данные, как правило, noisy. Степень, тип и серьезность ошибки измерения зависит от типа обследования кросс---секции или панели, человека, который отвечает на опрос, и переменных, о котором запрашивается информация. Deaton (1997) исследовал некоторые из источников погрешностей измерений с особым упором на данные Всемирного Банка, хотя некоторые из поднятых вопросов имеют более широкое значение. Отклонения от измерения ошибки зависит от того, что делается на данных в терминах преобразований (например, первых разностей). Следовательно, чтобы сделать информативное заявления о серьезности смещения, возникающего в результате погрешности, нужно проанализировать четко определенные модели. В последующих главах будут приведены примеры влияния ошибки измерения в конкретных условиях.


\subsection{Проверка данных}

В больших наборов данных легко могут встретиться ошибочные данные, полученные от неправильного ввода с клавиатуры или кодирование ошибок. Поэтому следует применять некоторые элементарные способы проверки, позволяющие судить об существовании проблем. Прежде, чем исследовать некоторые описательной статистики, необходимо проверить данные. Во-первых, использовать сводные статистические данные (минимальное, максимальное, среднее и медианное), чтобы убедиться, что данные находятся в надлежащем отрезка и соответствующем масштабе. Например, категориальные переменные должно быть между нулем и единицей, численные должны быть больше или равны нулю. Иногда отсутствующие данные кодируются как -999, или каким---либо другим целым числом, поэтому позаботьтесь о том, чтобы этих чисел не было. Во-вторых, надо знать, какие изменения являются дробными, а какие процентными. В-третьих, можно использовать гистограмму для выявления проблемных наблюдений.  Проверка наблюдений до оценивания, может также подразумевать нормализацию и / или предположение о функции распределения, для моделирования определенного набора данных. Наконец, может быть важно, проверить шкалы измерения переменных. Для некоторых целей, таких как использование нелинейных оценок, желательно масштабировать переменные, так чтобы они имели приблизительно одинаковые масштабы. Описательные статистики могут быть использованы для проверки того, что средние, дисперсии и ковариации переменных имеют корректное масштабирование.

\subsection{Представление описательных статистик}

Так как микроданные, как правило, большие, очень важно, предоставить в начальной таблице описательной статистики, как правило, среднее, стандартное отклонение, минимум и максимум для каждой переменной. В некоторых случаях неожиданно большие или малые значения могут выявить наличие грубой ошибки записи или ошибочное включение неправильных значений  данных. Для дискретных переменных могут быть полезны гистограммы, а для непрерывных переменных информативными являются графики плотности распределения.




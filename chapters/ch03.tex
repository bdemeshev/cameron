

\chapter{Структуры микроэкономических данных}

\section{Введение}

В данной главе дается обзор потенциальной полезности и ограничений различных типов микроэкономических данных. Наиболее распространенной структурой данных, используемой в микроэконометрики являются данные обследования или переписи. Эти данные, как правило, называются данными наблюдений, чтобы отличить их от экспериментальных данных.


В этой главе обсуждается потенциальное ограничение вышеупомянутых структур данных. Ограничения, присущие данным наблюдений могут усиливаться способом сбора данных, то есть основой выборки (sample frame) (способ создания выборки), планом выборки (sample design) (простая случайная выборка или стратифицированная случайная выборка) и объемом выборки (sample scope) (пространственные или панельные данные). Поэтому мы также обсуждаем вопросы построения выборки в связи с использованием данных наблюдений. Все новые термины будут определены позже в данной главе.
	
	
Микроэконометрика выходит за рамки анализа данных обследования в предположении простой случайной выборки. В этой главе представлены различные обобщения. Раздел 3.2 описывает структуру многоэтапного выборочного обследования и некоторые распространенные отклонения от случайной выборки; более детальный анализ их статистических последствий обсуждается в следующих главах. Здесь также рассматриваются некоторые часто встречающиеся проблемы, в результате которых выборка не является репрезентативной для генеральной совокупности. Учитывая недостатки данных наблюдений при оценке причинно-следственных параметров, исследователи чаще используют экспериментальные и квази-экспериментальные данные. Раздел 3.3 рассматривает преимущество данных из социальных экспериментов. В Разделе 3.4 рассматриваются возможности моделирования с использованием особого типа данных наблюдения, получающихся в квази-экспериментальных условиях, когда естественным образом возникают индивиды получившие и неполучившие воздействие. Такие ситуации получили название естественного эксперимента. Раздел 3.5 охватывает практические вопросы использования микроданных.
	
\section{Данные наблюдений}

	
Основным источником микроэкономических данных наблюдения являются обследования домашних хозяйств, фирм и государственных административных учреждений. Данные переписи могут также использоваться для создания выборки. Также  источником данных могут являться маркетинговые опросы, интернет-аукционы и т.д. 

Существует огромная литература по выборочным обследованиям с точки зрения как статистиков, так и пользователей данных. В литературе для статистиков обсуждается как правильно получить выборку из генеральной совокупности, а также последствия использования различных планов выборки, а литература для пользователей занимается вопросами оценки параметров и статистических выводов, которые возникают, когда данные обследования собираются с использованием различных планов выборки. Ключевым вопросом является то, насколько репрезентативной является выборка. Эта глава фактически является введением в данную теорию с двух точек зрения. Дополнительные детали  приведены в главе 24.
	

\subsection{Природа данных обследования}


Термин <<данные наблюдений>> обычно относится к данным обследования, собранных путем выборки  индивидов из  генеральной совокупности без попыток контролировать характеристики выборки данных.
Обозначим через $t$ время, а через $w$  набор переменных, представляющих интерес. При этом  $t$ может быть моментом времени или периодом времени. Пусть $S_{t}$ обозначает выборку из генеральной совокупности с  функции распределения  $F(w_{t}|\theta_{t}$; $S_{t}$ --- это выборка из $F(w_{t}|\theta_{t}$, где $\theta$ --- вектор параметров. Генеральная совокупность должна рассматриваться как множество точек с интересующими нас характеристиками, а для простоты будем считать, что форма функции $F$ известна. Случайная выборка предполагает, что каждый элемент генеральной совокупности имеет равные шансы попасть в выборку. Более сложные схемы составления выборки будут рассмотрены позже.


Абстрактное понятие стационарности генеральной совокупности является очень полезным. Если моменты характеристик генеральной совокупности постоянны, то мы можем записать, что $\theta_{t}=\theta$, для всех $t$.  Это предположение является сильным, поскольку оно предполагает неизменность моментов характеристик генеральной совокупности во времени. Например, возрастное распределение по полу и возрасту должно быть постоянным. 
Более реалистичным было бы предположение, что некоторые характеристики генеральной совокупности могу меняться во времени. 
Чтобы допустить возможность изменений, можно предположить, что параметры каждой генеральной совокупности --- это случайная выборка из суперсовокупности (superpopulation) с постоянными характеристиками. В частности, каждое $\theta_{t}$ рассматривается как случайная выборка из  распределения  с постоянными гиперпараметрами $\theta$. 
Термины суперсовокупность и гиперпараметры часто встречаются в литературе по иерархическим моделям, которые обсуждаются в главе 24. 
Дополнительные трудности возникают, если в $\theta_t$ есть эволюционная составляющая, например, если параметры зависят от $t$, или если зависимы соседние значения.
Использование иерархических моделей, обсуждаемых в главах 13 и 26, --- это один из подходов к моделированию связи между гиперпараметрами и  характеристиками генеральной совокупности.


\subsection{Простая случайная выборка}

В качестве ориентира для последующего обсуждения, рассмотрим простую случайную выборку, в которой вероятность выбора объекта $i$ из генеральной совокупности большого размера $N$, составляет $1/N$ для любого объекта. Представим набор переменных $w$ в виде $[y:x]$. Предположим, наша цель заключается в моделировании $y$, вектора возможных исходов, обусловленного  экзогенными объясняющими переменными $x$, чье совместное распределение обозначается $f_{J}(y,x)$. Совместная плотность всегда может быть представлена в виде произведения  условного распределения $f_{C}(y|x,\theta)$ и частного распределения $f_{M}(x)$:
\begin{equation}
f_{J}(y,x)=f_{C}(y|x,\theta)f_{M}(x)
\end{equation}

Суть простой случайно выборки заключается в равномерном отборе $(Y,X)$ из всей совокупности.


\subsection{Многоэтапные опросы}

Одной из альтернатив является стратифицированная многоэтапная  кластерная выборка, также называемая сложной выборкой. Такой подход используется в крупномасштабных обследованиях, таких как Текущее обследование населения (Current population survey, CPS), Панельное исследование динамики доходов (Panel Study of Income Dynamics, PSID). В главе 24.2 мы приводим дополнительное описание структуры CPS.
	
	
Сложный план выборки имеет свои преимущества. Он является более эффективным с точки зрения затрат, поскольку сокращает географическую разбросанность, становится возможным сделать более интенсивной выборку из определенной подсовокупности.  Например, можно осуществить избыточную выборку из малых подсовокупностей, обладающих интересующими характеристиками, в то время как случайная выборка из генеральной совокупности будет давать слишком мало наблюдений для получения надежных результатов. Недостатки данного метода заключаются в том, что стратифицированная выборка уменьшает изменчивость характеристик между индивидами, что имеет важное значение для большей точности.


Большое количество литературы про выборочные обследования  посвящено многоэтапным опросам, которые последовательно разбивают генеральную совокупность на следующие категории:

\begin{enumerate}
\item Страта: Непересекающиеся подгруппы, которые исчерпывают всю генеральную совокупность.
\item Первичная единица выборки (ПЕВ, Primary sampling units, PSU): Непересекающиеся подмножества страты.
\item Вторичная единица выборки (ВЕВ, Secondary sampling units, SSU): часть ПЕВ, в свою очередь может быть разделена дальше.
\item Конечная единица выборки (КЕВ): Финальная единица, выбранная для опроса, может быть как домохозяйство, так и группа домохозяйств.
\end{enumerate}


В качестве примера, страта может быть штатом или провинцией в стране, ПЕВ может быть регионом в штате или провинции, а КЕВ может быть небольшой группой домохозяйств в том же районе.
	
	
Обычно опрашиваются все страты, так что, например, все штаты будут включены в выборку. Но не все их ПЕВ и ВЕВ включаются в выборку, также они могут выбираться с разной интенсивностью. В двухэтапной выборке обследуемые ПЕВ взяты случайным образом, а затем внутри отобранных ПЕВ случайным образом выбираются КЕВ. В многоэтапной выборке появляются промежуточные единицы выборки.


Следствием этих методов отбора является то, что разные домохозяйства будут иметь различные вероятности попадания в выборку. Поэтому выборка не является репрезентативной для генеральной совокупности. Многие обследования приводят веса, которые должны быть обратно пропорциональны вероятности попадания объекта в выборку, в этом случае эти веса могут быть использованы для получения несмещенной оценки характеристик генеральной совокупности.
	
	
Данные исследования могут быть кластеризованы в связи с выбором большого количества домохозяйств в том же самом районе. Наблюдения в одном кластере могут быть зависимы или коррелируемы, поскольку они могут зависеть от некоторых наблюдаемых или ненаблюдаемых факторов, которые могут влиять на все наблюдения в кластере. 
Например, в пригороде могут преобладать домохозяйства с высоким доходом или домохозяйства, которые являются относительно однородными по некоторым своим предпочтениям. Данные по этим  домохозяйствам скорее всего будут коррелированы, по крайней мере безусловно. Хотя вполне возможно, что корреляция будет пренебрежимо мала после учета наблюдаемых характеристик домохозяйств. 
Статистические выводы при  игнорировании корреляции между выборочными наблюдениями приводят к ошибочным оценкам дисперсии, которые меньше, чем в случае правильной формулы. Эти вопросы рассматриваются более подробно в разделе 24.5. Двухэтапная и многоэтапная выборка усложняют вычисление стандартных ошибок.
	
	
Таким образом, (1) стратификация с различными частотами внутри страт означает, что выборка не репрезентативная; (2) веса, обратно пропорциональны вероятности попадания в выборку, могут быть использованы для получения несмещенной оценки характеристик генеральной совокупности, и (3) кластеризация может привести к корреляции наблюдений и занижению истинной стандартной ошибки, если не внесена соответствующая поправка.
	
	
\subsection{Смещенные выборки}

Если выборка случайна, то закон распределения любой характеристики данных такой же, как закон  распределения в генеральной совокупности. Некоторые отклонения от случайной выборки вызывают расхождения между этими двумя распределениями, это и называют смещением выборки. Распределение данных отличается от распределения генеральной совокупности, и отличие  зависит от характера отклонения от случайной выборки. Отклонение от случайной выборки происходит потому, что иногда более удобно или экономически эффективно делать выборку из подсовокупности, даже если она не является репрезентативной. Рассмотрим теперь несколько примеров таких отклонений, начиная со случая, в котором нет никаких отклонений от случай выборки.

\begin{center}
Экзогенная выборка
\end{center}


Экзогенная выборка на основе данных обследований возникает тогда, когда аналитик разделяет имеющуюся выборку на подвыборки основанные лишь на  экзогенных переменных $x$, но не на зависимой переменной. Например, в исследовании госпитализаций в Германии Гейл и соавт. (1997) разделили данные на две категории: имеющие хронические заболевания и не имеющие. 
Также распространена классификация по категориям доходов. 
Более точно было бы называть этот тип выборки экзогенной подвыборкой, так как она осуществляется на базе уже сделанной выборки.
Сегментирование существующей выборки по полу, здоровью, социально-экономическому статусу также очень популярно. 
При экзогенной выборке предполагается, что распределение  экзогенных переменных не зависит от $y$ и не содержит информации об интересующих параметрах генеральной совокупности $\theta$. Таким образом, можно не учитывать частные распределения экзогенных переменных и строить оценивание на базе условного распределения $f(y|x,\theta)$. Конечно, предположение может быть ошибочным и наблюдаемое распределение объясняемой переменной может зависеть от конкретной сегментирующей переменной. 


\begin{center}
Выборка на основе ответов
\end{center}

Выборка на основе ответов имеет место, если вероятность включения отдельного индивида в выборку зависит от ответа,  сделанного данным индивидом. В этом случае самоотбор происходит по правилам, определенным с помощью изучаемой эндогенной переменной.


Можно привести три примера: (1) В исследовании влияния отрицательного подоходного налога или Помощи семьям с детьми-иждивенцами (Aid to Families with Dependent Children, AFDC) на предложение труда опрашивались только индивиды за чертой бедности. (2) В исследовании факторов  определяющих выбор общественного транспорта опрашивались только пользователи  общественного транспорта (субсовокупность). (3) В исследовании факторов влияющих на количество посещений места отдыха опрашивались только индивиды хотя бы с одним посещением.


Снижение затрат на исследования --- это важная причина использования выборки на основе ответов, а не  простой случайной выборки. Для того чтобы получить достаточное количество наблюдений (информации) по относительно редкому выбору,  необходима очень большая случайная выборка. Следовательно, дешевле собрать выборку из тех, кто сделал данный выбор.


Практический вывод состоит в том, что состоятельные оценки параметров генеральной совокупности $\theta$ больше не могут быть получены  с использованием только условной плотностью для генеральной совокупности $f(y|x)$. План выборки также должен быть принят во внимание. Эта тема обсуждается далее в разделе 24.4.


\begin{center}
Выборка основанная на длительности
\end{center}

Выборка основанная на длительности является примером того, как смещение может возникнуть в результате использования выборки из одной генеральной совокупности, чтобы сделать выводы о другой генеральной совокупности. Строго говоря, это не столько пример отклонения от простой случайной выборки, а скорее выборка  из <<неправильной>> генеральной совокупности.


Эконометрические  модели переходов моделируют время, проведенное в состоянии $j$ индивидом $i$ до перехода в другую состояние $s$. Например, при моделировании безработицы $j$ соответствует безработице, а $s$ занятости. Данные, используемые в таких исследованиях могут происходить из одного или нескольких возможных источников. 
Одним из возможных источников является выборка лиц, которые являются безработными на определенную дату, другим --- выборка лиц, находящихся в составе рабочей силы, независимо от их текущего состояния, а третьим --- выборка лиц, которые либо входят, либо выходят из стадии безработицы в течение определенного периода времени. Каждой схеме выборки соответствует своя генеральная совокупность. 
В первом случае генеральной совокупностью будут все безработные, во втором --- рабочая сили, а в третьем --- индивиды, меняющие статус занятости. Эта тема обсуждается далее в разделе 18.6.


Предположим, что целью исследования является вычислить показатель средней продолжительности безработицы. Речь идет о средней продолжительности безработицы, с которой столкнется случайно выбранный индивид, если когда-либо станет безработным. Ответ на этот казалось бы простой вопрос  может меняться в зависимости от того, каким способом получены данные. Распределение  завершенной длительности безработицы довольно сильно отличается, если измерять её у тех, кто находится в состоянии безработицы (запас безработных индивидов) и у тех, кто покидает состояние безработицы (исходящий поток). Когда мы строим выборку из запаса, вероятность нахождения в выборке выше у лиц с более длительной продолжительностью. Когда мы выбираем из потока, вероятность не зависит от времени, проведенного в данным состоянии. Это хорошо известный пример выборки на основе длительности, в которой оценка, полученная на основе выборки запасов является смещенной оценкой средней продолжительности безработицы.

Следующая простая схема может прояснить идею:

\vspace{3cm}

\[ \bullet \: \circ \: \longmapsto \: \begin{aligned} \bullet &\: \bullet \\ \bullet &\: \circ \end{aligned}  \: \longmapsto \: \circ  \: \bullet \]

Entry flow --- Входящий поток

Stock --- Запас

Exit flow --- Исходящий поток

Здесь мы используем символ $\bullet$ для обозначения тех, кто медленно переходит из одного состояния в другое, и символ $\circ$ для обозначения тех, кто быстро переходит. Предположим, что два типа в равной степени представлены во входящем потоке, но $\bullet$ остаётся в запасе дольше, чем $\circ$. Тогда среди индивидов в запасе более высока доля тех, кто переходит медленно. Наконец, в исходящем потоке из состояния снова будет равная доля тех, кто быстро и медленно переходит. Этот аргумент обобщается и на другие виды неоднородности.


Суть этого примера не в том, что выборка из потока  лучше, чем выборка из запаса. Скорее суть в том, что в зависимости от вопроса, выборка из запаса  может не быть случайной выборкой из нужной генеральной совокупности.



\subsection{Смещение самоотбора}

Рассмотрим следующую задачу. Исследователь заинтересован в измерении эффекта обучения (воздействия), обозначаемого $z$, на заработную плату, обозначаемую $y$, при заданных характеристиках работника, обозначаемых $x$. Переменная $z$ принимает значение 1, если работник прошел обучение и 0 в противном случае. Наблюдения по $(x, D)$ имеются для всех работников, а по $y$ только для тех, кто прошел подготовку $(D = 1)$. Необходимо оценить  среднее влияние обучения на заработную плату случайно выбранного работник с известными характеристиками, который в настоящее время не обучался $(D = 0)$. Проблема самоотбора выборки связана с  трудностью  нахождения данной оценки.


Мански (1995), который рассматривает это как проблему идентификации, определяет проблему самоотбора формально следующим образом:


Это проблема идентификации условного распределения вероятностей по случайной выборке, в которой  значения объясняющих переменных всегда наблюдаемы, а значения зависимой переменной подвергаются цензурированию.


Пусть $y$ --- объясняемая переменная, а объясняющие переменные обозначаются через $x$. Переменная $z$ является цензурирующей переменной, то есть принимает значение 1, если значение $y$ наблюдается и 0 в противном случае. Переменные $(D, x)$ всегда наблюдается, а $y$ наблюдается только при $D = 1$, поэтому Мански называет эту ситуацию цензурированной выборкой. Этот процесс получения выборки  не позволяет идентифицировать $\Pr[y|x]$, как видно из равенства

\begin{equation}
\Pr[y|x]=\Pr[y|x,D=1]\Pr[D=1|x]+\Pr[y|x,D=0]\Pr[D=0|x].
\end{equation}

Процесс построения выборки позволяет идентифицировать три из четырех слагаемых в правой части, но не дает никакой информации о  
$\Pr[y|x, D = 0]$. Потому что

\[
\E[y|x]=\E[y|x,D=1]\Pr[D=1|x]+\E[y|x,D=0]\Pr[D=0|x],
\]
всякий раз, когда вероятность цензурирования $\Pr[D=0|x]$ является положительной, имеющиеся эмпирические данные не накладывают никаких ограничений на $\E[y|x]$. Следовательно, цензурированный процесс построения выборки позволяет идентифицировать $\Pr[y|x]$ с точностью до неизвестного значения $\Pr[y|x,D=0]$. Чтобы узнать что-нибудь о $\E[y|x]$, ограничения должны быть наложены на $\Pr[y|x]$.


Альтернативные подходы к решению этой проблемы обсуждаются в разделе 16.5.


\subsection{Качество данных опросов}

Качество данных зависит не только от плана выборки и способа опроса, но и от ответов индивидов. Этот факт особенно относится к данным наблюдений. Мы рассмотрим несколько ситуаций, в которых качество выборочных данных может быть существенно ухудшено. Некоторые из этих проблем (например, истощение) также могут возникнуть при использовании других типов данных. Эта тема пересекается со смещением выборки.


\begin{center}
Проблема с отсутствием ответа
\end{center}


Опросы, как правило, добровольны, и стимул отвечать может варьироваться в зависимости от  характеристик домохозяйства и типа поставленного вопроса. Индивиды могут отказаться отвечать на некоторые вопросы. Если есть связь между  отказом отвечать на вопрос и характеристиками индивида, то возникает вопрос о репрезентативности опроса. Если неполучение ответов игнорируется, а  анализ проводится только с использованием данных от респондентов, как это будет влиять на оценку интересующих исследователя параметров?


Неполучения ответов является частным случаем проблемы самоотбора, упомянутой в предыдущем разделе. Оба случая связаны со смещением выборки. Чтобы проиллюстрировать, как это приводит к искажению статистических выводов рассмотрим следующую модель:

\begin{equation}
\begin{bmatrix} y_{1} \\ y_{2} \end{bmatrix} \Biggm| x,z \sim N \Biggm( \begin{bmatrix} x'\beta \\ z'\gamma \end{bmatrix}, \begin{bmatrix} \sigma^2_{1} & \sigma_{12} \\ \sigma_{12} &  \sigma^2_{2}\end{bmatrix}\Biggm),
\end{equation}
где $y_{1}$ является непрерывной случайной величиной (например, расходы), которые зависят от $x$, а $y_{2}$ является скрытой переменной, которая измеряет <<склонность к участию>> в опросе и зависит от $z$. Домохозяйство участвует в опросе, если $y_{2}>0$, а в противном случае не участвует. Переменные $x$ и $z$ предполагаются экзогенными. Данная спецификация допускает, что $y_{1}$ и $y_{2}$  коррелированы. 


Предположим, мы оцениваем $\beta$ по данным, предоставленным участниками опроса, методом наименьших квадратов. Является ли эта оценка несмещенной в случае, когда есть отказавшиеся участвовать? Ответ в том, что если неучастие является случайным и независимым от $y_{1}$, то нет никакой смещённости, а иначе будет смещение появляется:

Действительно:

\[
\begin{aligned}
&{\beta}=[X'X]^{-1}X'y_{1}, \\
\E[\hat{\beta}-\beta]&=E\Bigm[ [X'X]^{-1}X'\E[y_{1}-X\beta|X,Z,y_{2}>0] \Bigm],
\end{aligned}
\]

где первая строка содержит формулу для оценки $\beta$, а вторая ---  смещение. Если $y_{1}$ и $y_{2}$ независимы при фиксированных $X$ и $Z$, $\sigma_{12}=0$, то

\[
\E[y_{1}-X\beta|X,Z,y_{2}>0]=\E[y_{1}-X\beta|X,Z]=0,
\]
и здесь нет смещения.

\begin{center}
Пропущенные данные и ошибки измерения
\end{center}

Респонденты, заполняющие обширную анкету, не обязательно отвечают на все вопросы, и даже если они отвечают, ответы могут быть умышленно или случайно ложными. Предположим, что выборочное обследование пытается получить вектор ответов, обозначенный как $x_{i}=(x_{i1},\dots,x_{iK})$, от $N$ участников, $i= 1,\dots, N$. Предположим теперь, что если человек не в состоянии предоставить информацию об одном или более элементов $x_{i}$, то весь вектор отбрасывается. Первая проблема, возникающая в результате отсутствия данных состоит в том, что размер выборки уменьшается. Второй потенциально более серьезной проблемой является то, что недостающие данные могут привести к смещению подобному смещению самоотбора. Если данные отсутствуют  на систематической основе, то оставшаяся выборка не может быть репрезентативной. К смещению самоотбора может привезти любой систематический характер пропущенных данных. Например, респонденты с высоким доходом могут систематически не отвечать на вопросы о доходах. Наоборот, если данные пропущены случайным образом, то пропуски приводят к потере точности, но не к смещению. Глава 27 обсуждает проблему пропущенных данных и её решение.


Ошибки измерения в ответах являются широко распространенной проблемой. Они могут возникнуть в результате различных причин, в том числе это могут быть неправильные ответы, связанные с небрежностью, намеренные неправильные ответы, ошибочные воспоминания о прошлых событиях, неправильная интерпретация вопросов, ошибки обработки данных.  Более существенным источником ошибок измерения является тот факт, что измеряемая переменная может является далекой от совершенства прокси-переменной для исследуемого теоретического понятия. Последствия таких ошибок измерения является основной темой главы 26.


\begin{center}
Истощение выборки
\end{center}

В случае панельных данных обследование включает в себя повторные наблюдения множества индивидов. В этом случае мы можем иметь:

\begin{itemize}
\item полные ответы во все периоды (полное участие),
\item нет ответов в первом периоде и во всех последующих периодах (полное неучастие), или
\item частичный ответ в смысле наличия ответа на вопросы начальных периодов, но отсутствия ответа в более поздние периоды (неполное участие), эта ситуация называется истощением выборки
\end{itemize}


Истощение выборки приводит к недостающим данным, наличие любой формы неслучайности пропусков приводит к проблеме смещения самоотбора, которая уже упоминалась. Истощение выборки можно интерпретировать как особый случай самоотбора. Истощение выборки кратко обсуждается в разделах 21.8.5 и 23.5.2.

\subsection{Типы данных наблюдения}

Пространственные данные получаются при наблюдении переменных $w$, для выборки $S_{t}$ в некоторый момент времени $t$. Обычно нецелесообразно делать одномоментную выборку всех домохозяйств, поэтому пространственные данные предоставляют характеристики каждого элемента некоторого подмножества генеральной совокупности, которое будет использоваться, чтобы сделать вывод о генеральной совокупности. Если генеральная совокупность стационарная, то выводы сделанные о $\theta_{t}$ с использованием $S_{t}$ могут быть справедливыми и для $t'\neq t$. Если существует значительная зависимость между прошлым и текущим поведением, то для обнаружения интересующих нас соотношений необходимы панельные данные. Например, ранее принятые решения могут повлиять на текущие результаты; привычка может определять текущие покупки, но такая зависимость не может быть смоделирована, если история покупок неизвестна. Это одно из ограничений, налагаемых пространственными данными.


Повторные пространственные данные получаются с помощью последовательности независимых выборок $S_{t}$, взятых из генеральной совокупности с распределением $F(w_{t},\theta_{t}), t=1, \dots, T$. Поскольку при построении выборки нет попыток сохранить тех же индивидов, то сведения о динамических зависимостях теряются. Если генеральная совокупность стационарная, тогда повторные пространственные данные похожи на  выборку с возвращением из постоянной генеральной совокупности. Если же генеральная совокупность нестационарная, повторные пространственные данные  связаны между собой в соответствии с изменениями генеральной совокупности во времени. В таком случае целью являются выводы о постоянных гипер-параметрах. Анализ повторяющихся пространственных данных обсуждается в разделе 22.7.


Панельные или лонгитюдные данные получают, если изначально фиксируется выборка $S$, а затем наблюдения для данной выборки получают в периоды $t=1, \dots, T$. Это может быть достигнуто путем опроса индивидов и сбора в один момент времени как текущих, так и прошлых данных. Также панельные данные можно получить отслеживая во времени характеристики индивидов, попавших в выборку. Таким образом получается последовательность  векторов данных ${w_{1}, \dots, w_{T}}$, которая используется, чтобы сделать выводы о поведении населения или конкретной выборки лиц. Соответствующие методологии  могут быть различными. Если данные взяты из нестационарной генеральной совокупности, то целью исследования может быть оценивание гипер-параметров суперсовокупности.


Некоторые из недостатков этих типов данных очевидны. Пространственные и повторные пространственные данные плохо подходят для моделирования межвременной зависимости. Такие данные пригодны только для моделирования статических отношений. В отличие от них, панельные  данные, особенно если они охватывают достаточно длительный период времени, пригодны для моделирования статических и динамических отношений.


Панельные данные также не лишены недостатков. Первая проблема может быть связана с нерепрезентативностью панели. Трудности получения статистических выводов о генеральной совокупности на основе панельных данных становятся еще более сильными, если совокупность не является стационарной. 
Для анализа динамики поведения сохранение исходных домашних хозяйств в панели как можно дольше является очень привлекательным. На практике панельные данные страдают от проблемы <<истощения выборки>>, возможно вызываемой <<усталостью выборки>>. Это просто означает, что респонденты перестают предоставлять ответы на вопросы. Это создает две проблемы: (1) панели становятся несбалансированными и (2) существует опасность того, что остающиеся в выборке домохозяйства не являются <<типичными>> и, что выборка становится не репрезентативной для генеральной совокупности. 
Когда имеющиеся данные выборки не являются случайной выборкой из совокупности, результаты, основанные на данных различных типов будут восприимчивы к смещению в разной степени. Проблема <<истощения выборки>> возникает потому, что с течением времени становится все труднее сохранить индивидов внутри панели или они могут быть <<потеряны>> (цензурированы) по некоторым другим причинам, таким как изменение местоположения. Эти вопросы рассматриваются далее в книге. Анализ панельных данных тем не менее может предоставить информацию о некоторых аспектах выборки, хотя экстраполяция выводов на генеральную совокупность  может быть непростой.


\section{Данные социальных экспериментов}

Экспериментальные данные и данные наблюдений различны, так как экспериментальная среда в принципе может быть тщательно контролируема и управляема. В такой ситуации возможно изменять значение исследуемой причинной переменной, оставляя значения остальных переменной равными контрольным. В отличие от этого, данные наблюдений возникают  в неконтролируемой среде, в которой существует возможность наличия мешающих факторов, которые усложняют определение причинно-следственных связей. Например, при изучении воздействия образования на  заработок по данным наблюдений надо учитывать, что количество лет обучения отдельного индивида является результатом принятия решения.  Следовательно, нельзя рассматривать уровень школьного образования как величину, назначаемую экспериментатором.


В социальных науках данные аналогичные экспериментальным получают либо из  \textbf{социальных экспериментов}, которые определены и описаны подробно далее, или из  <<лабораторных>> экспериментов на небольших группах добровольцев, которые имитируют поведение экономических агентов в реальной жизни. Социальные эксперименты относительно редкое явление, и все же экспериментальные концепции, методы и данные служат основой для оценивания эконометрических исследований, основанных на данных наблюдений.


В этом разделе представлен краткий обзор методологии социальных экспериментов, характер  порождаемых данных, и некоторые возникающие проблемы и вопросы эконометрической методологии.


Главной особенностью экспериментальной методологии является сравнение  результатов случайно выбранной экспериментальной группы, которая подвергается <<воздействию>>, с результатами в контрольной группе. 
В хорошем эксперименте уделяется много внимания сравнению контрольной и экспериментальной групп, чтобы избежать возможных смещений в результатах. 
Подобные условия могут не выполняться в данных наблюдений, что может привести к невозможности идентифицировать причинные параметров, представляющие интерес. 
Иногда, экспериментальные условия могут примерно выполняться в данных наблюдений. Рассмотрим, например, два соседних региона или государства, в которых разная политика установления минимальной зарплаты, создаёт условия естественного эксперимента, в котором наблюдения подверженные <<воздействию>> можно сравнить с  <<контрольными>>. Структура данных естественного эксперимента также привлекает внимание эконометристов.


Социальный эксперимент подразумевает экзогенные изменения в экономической среде, при этом  индивиды разбиты на два подмножества:  одно подмножество, которое получает экспериментальное воздействие и другое подмножество, которое служит в качестве контрольной группы. 
В отличие от данных наблюдений, в которых изменения в экзогенных и эндогенных факторах часто смешены, хорошо продуманный социальный эксперимент 
изолирует  эффект переменных, подвергнутых <<воздействию>>. В некоторых экспериментах может не быть явной контрольной группы, но применяя разные уровни воздействия в принципе возможно  оценить всю поверхность отклика результата эксперимента.


Основной задачей социального эксперимента является оценка влияния фактических или потенциальных социальных программ. Модель потенциального результата из раздела 2.7 является базовым подходом для оценивания влияния социальных экспериментов. Несколько альтернативных мер воздействия были предложены и будут обсуждаться в главе, посвященной оценке экономических и социальных программ (глава 25).


Бартлесс (1995) приводит общее описание социальных экспериментов, отмечая при этом некоторые потенциальные ограничения. В сопутствующей статье Хекман и Смит (1995) сосредотачиваются на ограниченности реальных социальных экспериментов, которые были реализованы. Дальнейшее содержимое этого раздела основано на этих двух статьях. 


\subsection{Основные особенности социального эксперимента}


Социальные эксперименты мотивированы вопросом о том, как индивиды будут реагировать на тип политики, который никогда не был опробован и, следовательно, для которого нет данных наблюдений. Идея социального эксперимента заключается в привлечении группы добровольцев, некоторые из которых случайным образом попадают в группу получающую <<воздействие>>, а остальные --- в контрольную группу. Разница между ответами тех, кто оказался в тестовой группе, и тех, кто оказался  в контрольной группе, представляет собой оценку влияния политики. Схема типичного эксперимента показана на рисунке 3.1.


Термин <<экспериментальная>> относится к группе, получавшей воздействие, <<контрольная>> --- к группе не получавшее воздействие, а <<рандомизация>> к процессу деления лиц на две группы.


Рандомизированные исследования были введены в статистику Р. Фишером (1928) и его коллегами. В типичном сельскохозяйственном эксперименте новое воздействие, например, удобрение будет применяться для растений, растущих на случайно-выбранных участках земли. Затем результаты для экспериментальных растений будут сравниваться с результатами контрольных растений, которые во всех существенных характеристиках совпадают с экспериментальными растениями, кроме получения воздействия. 
Если эффект всех других различий между экспериментальной и контрольной группой может быть учтен, то оценка разницы зависимой переменной между этими двумя группами может трактоваться как влияние воздействия. В простейшем случае можно  сравнить средние результаты экспериментальной и контрольной групп.


Хотя в сельском хозяйстве и биомедицинских науках, методология рандомизированных экспериментов давно используется, в области экономики и социальных наук она является достаточно новой. Более того, она привлекательна для изучения эффекта смены политики, если отсутствуют данные наблюдений, например, из-за того, что интересующей смены политики ещё ни разу не было.  Рандомизированные эксперименты также допускают  более сильное изменение параметров  политики, чем то, что имеет место в данных наблюдений, тем самым облегчая выявление и изучение реакции на политические изменения. Во многих случаях в социальном эксперименте можно попробовать применить политику, которая еще ни разу не применялась, и данных наблюдений по которой вообще не существует.


Социальные эксперименты все еще довольно редки за пределами Соединенных Штатов, отчасти потому, что они дорогие. В США число таких экспериментов росло с начала 1970-х. В Таблице 3.1 приведены особенности некоторых относительно известных экспериментов; для более широкого описания см. работу Бартлесса (1995).


В результате эксперимента могут быть получены пространственные или панельные данные,  хотя  из соображений стоимости длина временных рядов обычно существенно ниже по сравнению с данными наблюдений. Если эксперимент длится несколько лет и имеет несколько этапов и / или географических мест, как, например, Эксперимент RAND по медицинскому страхованию (RHIE, Rand Health Insurance Experiment), промежуточный анализ на основе <<неполных>> встречается очень часто (Ньюхаус и др., 1993).



\subsection{Преимущества социального эксперимента}


Бартлесс (1995) описывает преимущества социальных экспериментов с большой ясностью. Главное преимущество проистекает из рандомизированных исследований, которые устраняют любые корреляции между наблюдаемыми и не наблюдаемыми характеристиками участников эксперимента.

\begin{table}[h]
\begin{center}
\caption{\label{tab:exppart}Особенности некоторых экспериментов}
\begin{tabular}[t]{llcll|}
\hline
\bf{Эксперимент} & \bf{Тестирумое воздествие} & \bf{Целевая аудитория} \\
\hline
Эксперимент RAND по медицинскому страхованию (Rand Health Insurance Experiment, RHIE), 1974–1982 & Планы медицинского страхования с различной ставкой и с разным уровнем максимальных расходы за свой счет  & Индивиды и домохозяйства со средним и низким уровнем дохода \\
Отрицательный подоходный налог (Negative Income Tax, NIT), 1968–1978 & Планы NIT с альтернативными гарантиями дохода и налоговой ставкой & Индивиды и семьи с непожилым главой семьи со средним и низким уровнем дохода \\
Закон о партнерстве по подготовке кадров (Job Training Partnership Act, JTPA), (1986–1994)& Помощь при поиске работы, тренинги, финансируемые JTPA & Молодые люди без образования и взрослые люди, столкнувшиеся с трудностями \\
\hline
\end{tabular}
\end{center}
\end{table}



Следовательно, вклад воздействия в разницу результирующей переменной между экспериментальной и контрольной группами может быть оценён без смещения, даже если не возможно контролировать мешающие переменные. Наличие корреляции между переменной воздействия и мешающими переменными --- частая проблема исследований по данным наблюдений, затрудняющая получение причинно-следственных выводов. Напротив, экспериментальные исследования, проведенные в идеальных условиях, могут позволить получить состоятельную оценку средней разницы в результатах экспериментальной  и контрольной группы без особых вычислительных сложностей.


Если, однако, результат зависит не только от переменной воздействия, но и от  других наблюдаемых факторов, то контролирование последних будет улучшать точность оценки влияния воздействия.



Даже если данные наблюдений доступны, создание и использование экспериментальных данных имеет большую привлекательность, потому что позволяет сделать переменную воздействия экзогенной.  Рандомизация воздействия может привести к существенному упрощению статистического анализа. Выводам основанным на данных наблюдений часто не хватает общности, поскольку они основаны на неслучайной выборке из генеральной совокупности, отсюда возникает проблема смещения самоотбора. Примером может служить упомянутый эксперимент Rand по медицинскому страхованию (RHIE), посвященный  чувствительности к цене спроса на медицинские услуги. 
Наличие медицинской страховки влияет на цену медицинских услуг и тем самым на их использование. Важным вопросом является величина избыточного использования медицинских услуг, возникающего при субсидировании страховки.  Можно, конечно, использовать данные наблюдений для моделирования связи между спросом на медицинские услуги и уровнем страхования. 
Такой анализ уязвим для критики, ведь уровень медицинского страхования не должен рассматриваться как экзогенный. Теоретический анализ показывает, что спрос на страхование здоровья и на медицинские услуги определяется одновременно, поэтому причинно-следственная связь не является однонаправленной. Этот факт может потенциально  затруднить определение влияния медицинского страхования. 
Если рассматривать медицинское страхование как экзогенную переменную, то мы получим смещенную оценку чувствительности к ценам. Однако в экспериментальных условиях участвующим домохозяйствам можно выдать заранее определенную страховку,  что сделает её экзогенной переменной. В этом случае влияние страховки будет идентифицируемым. Как только ключевые переменные становятся экзогенными, направление причинно-следственной становится очевидным, и влияние воздействия может быть прямо измерено. Кроме того, если эксперимент не порождает проблем, о которых мы говорим далее, то  значительно упрощается статистический анализ по сравнению с тем, что часто необходим при работе с данными наблюдений.



\subsection{Ограничения социального эксперимента}

Применение к людям методологии, которая изначально разрабатывалась и применялась к другим объектам исследования, вызывает оживленную дискуссию в литературе. См. работу Хекмана и Смита (1995), которые утверждают, что многие социальные эксперименты подвержены тем же ограничениям, что и исследования по данным наблюдений. Возникающие вопросы касаются как общих моментов, таких как достоинства экспериментальных наблюдений по сравнению с данными наблюдений, так и специальных сюжетов, связанных с участием людей  в экспериментах. Некоторые вопросы рассматриваются более подробно в последующих главах.

%%%
Социальные эксперименты являются очень дорогостоящими. Иногда, они не соответствуют <<чистым>> рандомизированным  исследованиям. Таким образом, результаты таких экспериментов не всегда однозначны и легко интерпретируемы или свободны от смещения. Если  переменная воздействия имеет много альтернативных значений, представляющих интерес, или нужна экстраполяция результатов, то должна быть собрана очень большая выборка, чтобы обеспечить достаточную изменчивость данных и точно измерить эффект изменения переменной воздействия. В этом случае стоимость эксперимента также будет увеличиваться. Если фактор стоимости не позволяет провести достаточно большой эксперимент, его полезность по сравнению с исследований по данным наблюдений может быть сомнительной; см. статьи Розена и Стаффорда в сборнике Хаусмана и Уайза (1985).


К сожалению, структура некоторых социальных экспериментов является неправильнай. Хаусман и Уайз (1985) утверждают, что данные эксперимента с отрицательным подоходным налогом в Нью-Джерси пострадали от эндогенной стратификации, которую они описывают следующим образом:

\ldots Экспериментальное исследование необходимо для устранения корреляции между переменной воздействия и другими детерминантами изучаемой зависимой переменной. В каждой эксперименте, связанным с доходом индивидов, однако, участвующие отбирались  отчасти на основе зависимой переменной, и, более того, деление участвующих на контрольную и экспериментальную группы также проводилось на базе зависимой переменной.  В целом, группа имеющих право на участие, отобранных на основе семейного положения, расы, возраста главы семьи, и т.д. --- была стратифицирована на основе дохода (и других переменных), а индивиды, были отобраны из этих страт. (Хаусман и Уайз, 1985, стр. 190-191)


Авторы приходят к выводу, что в присутствии эндогенной стратификации, получение несмещенной оценки влияния воздействия не является простой задачей.  К сожалению, полностью рандомизированное исследование, в котором назначение воздействия в случайно выбранной экспериментальной группе населения не зависит от дохода, будет гораздо более дорогостоящим и может оказаться невозможным.



Есть несколько других причин, которые могут привести к нарушению условий идеального рандомизированного эксперимента. Во-первых, если места проведения эксперимента выбираются случайно, то потребуется сотрудничество организаторов и потенциальных участников на местах. Если сотрудничества не возникнет, то будут использоваться альтернативные места проведения, где сотрудничество возможно,  альтернативные места воздействия, где такое сотрудничество может быть получено. При этом под  угрозу ставится принцип случайного назначения, см. Хотц (1992).



Второй является проблема самоотбора, так как участие в эксперименте является добровольным. По этическим причинам есть много экспериментов, которые просто не могут быть реализованы (например, рандомизированное назначение количества лет обучения студентам). В отличие от медицинских экспериментов, в которых можно использовать классические условия двойного слепого опыта, в социальных экспериментах организаторы и участники могут знать, принадлежит ли индивид к контрольной или к экспериментальной группе. Более того индивиды из контрольной группы могут получать воздействия другим способом (например, обучение). Если решение об участии не коррелирует с $x$ или $\epsilon$, анализ экспериментальных данных упрощается.


Третья проблема заключается в истощении выборки, вызванном тем, что индивиды выбывают из эксперимента после его начала. Даже если первоначальная выборка была рандомизированной, неслучайное истощение выборки может привести к проблеме похожей на смещение истощения  в панелях. Наконец, существует проблема эффекта Хоторна. Термин происходит из исследования по  социальной психологии, проведенного совместно Гарвардской Высшей Школой делового администрирования и управлением Западной Электрической компании, на которую  Хоторн  работал в Чикаго с 1926 по 1932 год. Человек, в отличие от неодушевленных предметов, может изменять или адаптировать своё поведение во время участия в эксперименте. В этом случае изменение зависимой переменной, наблюдаемое в экспериментальных условиях, не может быть отнесено исключительно к воздействию.


Хекман и Смит (1995) упоминают ряд других трудностей в проведении рандомизированного воздействия. Одной из причин смещения может стать бюрократия. \textbf{Смещение рандомизации}происходит, если при назначении воздействия создаётся  систематическая разница между индивидами, участвующими в эксперименте, и индивидами вне эксперимента.
Хекман и Смит приводят потенциальные примеры такого смещения в реальных экспериментах. Другой тип смещения, называемый \textbf{смещением замещения}, возникает, если контрольная группа получает какое-то воздействие, замещающее экспериментальное воздействие.  Наконец, анализ социальных экспериментов неизбежно связан с частичным равновесием. Нельзя с уверенностью экстраполировать эффект воздействия на всю генеральную совокупность, т.к. предпосылка \textit{при прочих равных} не будет выполнена для всей генеральной совокупности. 


Возможность экстраполяции результатов эксперимента на всю генеральную совокупность является ключевым вопросом. Если эксперимент проводится в качестве пилотной программы в небольших масштабах, а задача состоит в предсказании влияния политики, охватывающей большое количество индивидов, то очевидным ограничением пилотной программы является то, что она  может не учитывать широкого эффекта воздействия. Широкое применение воздействия может изменить экономическую среду, и сделать некорректными прогнозы полученные при анализе частичного равновесия. То есть пилотная программа может иметь не те эффекты, как политика, которую она имитирует.


Таким образом, социальные эксперименты, в принципе, могут порождать данными, которые проще анализировать и понимать с точки зрения причинно-следственных связей, чем данные наблюдений. Будет ли реализовано это потенциальное преимущество зависит от структуры эксперимента. Плохо организованный эксперимент создаёт свои статистические сложности, которые влияют на точность выводов. Социальные эксперименты принципиально отличаются от экспериментов в биологии и сельском хозяйстве. Участники эксперимента и организаторы как правило ведут себя активно, пытаются прогнозировать будущее и имеют личные предпочтения, а не пассивно выполняют протокол эксперимента.

\begin{table}[h]
\begin{center}
\caption{\label{tab:natfeature}Особенности некоторых естественных экспериментов}
\begin{tabular}[t]{p{6cm}p{4cm}p{4cm}}
\hline
\hline
\bf{Эксперимент} & \bf{Тестирумое воздействие} & \bf{Ссылка} \\
\hline
Результаты близнецов с разным уровнем обучения &  Разница в доходе от образования через корреляцию между уровнем образования и зарплатой  & Ашенфелтер и Крюгер (1994) \\
Переход к Национальной Системе Страхования (National Health Insurance, NHI) в Канаде, когда Саскетчеван переходит к NHI, и за ним через несколько лет следуют остальные штаты  & Эффекты перехода к NHI на рынок труда на основе сравнения провинций с и без NHI & Грубер и Ханрэтти (1995) \\
Нью-Джерси увеличивает минимальную заработную плату, а соседний штат Пенсильвания --- нет & Эффект минимальной заработной платы на занятость & Кард и Крюгер (1994) \\
\hline
\hline
\end{tabular}
\end{center}
\end{table}



\section{Данные естественного эксперимента}


Иногда, исследователь может располагать данными <<естественного эксперимента>>. Данный эксперимент возникает, если подмножество генеральной совокупности подвергается экзогенным изменения в переменной воздействия, возможно, в результате изменения политики.  В идеале, источник изменений известен.


В микроэконометрике широко используются два  применения идеи естественного эксперимента. Для примера рассмотрим простую регрессионную модель:

\begin{equation}
y=\beta_{1}+\beta_{2}x+u,
\end{equation}

где $x$ --- эндогенная объясняющая переменные, коррелированная с $u$.

Предположим, что существует экзогенное вмешательство, которое изменяет $x$. Примерами такого внешнего вмешательства могут быть административные правила, непредвиденное изменение законодательства, рождение двойни, природные явления или географические изменения, см. таблицу 3.2. Экзогенные вмешательство создает возможность для оценивания его воздействия путём сравнения  затронутой группы до и после вмешательства или путём сравнения с незатронутой группой после вмешательства. То есть, само событие порождает <<естественные>> контрольные группы, что облегчает оценку $\beta_{2}$. Оценивание упрощается, так как $x$ можно рассматривать как экзогенный регрессор.


Также естественный эксперимент может упростить статистические выводы, если порождает  естественные инструментальные переменные. Предположим, что $z$  коррелирует с $x$, или, возможно, причинно связана с $x$, и не коррелирует с $u$. Тогда оценку $\beta_{2}$ по методу инструментальных переменных можно представить в виде

\begin{equation}
\hat{\beta}_{2}=\frac{\Cov[z,y]}{\Cov[z,x]}
\end{equation}
(см. раздел 4.8.5). В  данных наблюдений сложно найти инструментальные переменные, которые легко возникают в случае естественного эксперимента при благоприятном стечении обстоятельств. В этом случае оценивание сильно упрощается. Мы рассмотрим первый случай в следующем разделе; тема естественных инструментов будет рассмотрена в главе 25.

\subsection{Естественное экзогенное воздействие}


Такие данные являются менее дорогими для сбора, и они также позволяют исследователю оценить влияние некоторых конкретных факторов, изолировав их от других. Подобно контролируемому эксперименту, <<природа>> может оставлять неизменными факторы, не представляющие интерес. Такие естественные эксперименты привлекательны тем, что они создают экспериментальную и контрольную группы без существенных издержек и в реальных условиях. Возможность использовать  естественный эксперимент для получения состоятельных выводов зависит, в частности, от того, насколько воздействие действительно является экзогенным, достаточно ли велико воздействие, чтобы его измерить, насколько хорошо выделены экспериментальная и контрольная группы. То, что изменение отражено в законодательстве, не означает автоматически экзогенности воздействия. В хорошей ситуации использование подобных данных может привести к получению ценных эмпирических выводов.


Исследования основанные на естественных экспериментах имеют несколько потенциальных ограничений, существенность которых  можно оценить только путем тщательного рассмотрения соответствующей теории, фактов и институциональных установок. Следуя Кэмпбеллу (1969) и Мейеру (1995), эти ограничения делятся на две группы. Во-первых, это ограничения, влияющие на внутреннюю валидность исследования, т.е. выводы о воздействии политики, сделанные на основании исследования. Во-вторых, это ограничение, которые влияют на внешнюю валидность исследования, т.е. обобщение выводов на исследования на генеральную совокупность.


Рассмотрим исследование изменения политики, в котором  выводы делаются из сравнения данных до и после воздействия с использованием метода описанного кратко ниже и более подробно в главе 25. В любом исследовании будут пропущенные переменные, некоторые из них  могут  измениться в период между  изменением политики и измерением эффекта политики. 
Характеристики индивидов, попавших в выборку, такие как возраст, состояние здоровья, и их фактические или ожидаемые экономические условия также могут меняться. 
Эти пропущенные факторы будут непосредственно влиять на измеренное воздействие изменения политики. Можно ли обобщать  результаты на всю генеральную совокупность, зависит от наличия смещения из-за неслучайной выборки, наличия существенного  взаимодействия между изменением политики и другими факторами, наличия исторических факторов, которые могут изменять эффект воздействия. 
Конечно, эти соображения относятся не только к естественным экспериментам, мы лишь подчёркиваем, что естественные эксперименты не свободны от данных проблем.


\subsection{Метод <<разность  разностей>>}



Одним из простых регрессионных методов является сравнение результатов в одной группе до и после воздействия. К примеру, рассмотрим

\[
y_{it}=\alpha+\beta D_{t} +\epsilon_{it}, i=1, \dots, N, t=0,1,
\]

где $D_{t}=1$ в первом периоде (после вмешательства), $D_{t}=0$ в периоде 0 (до вмешательства), и $y_{it}$ измеряет результат. Сквозная регрессии даёт оценку параметра воздействия политики  $\beta$. Легко показать, что она будет равна  средней разнице результата до и после вмешательства,


\[
\hat{\beta}=N^{-1}\sum_{i}(y_{i1}-y_{i0})=\bar{y}_{1}-\bar{y}_{0}.
\]

Использование одной и той же группы индивидов подразумевает сопоставимость групп до и после воздействия и  является сильным предположением. Данное предположение необходимо для идентифицируемости $\beta$. Если, например, мы допускаем изменение $\alpha$ между  двумя периодами, то величина $\beta$ больше не может быть идентифицирована. Изменения в $\alpha$ смешиваются с эффектом политики.

Одним из способов улучшения предыдущей модели является включение дополнительной контрольной группы, то есть такой, которая не затронута политикой, и для которой имеются данные в обоих периодах. Используя обозначение Мейера (1995), можно записать соответствующую регрессию в виде:


\[
y^{j}_{it}=\alpha+\alpha_{1}D_{t}+\alpha^{1}D^{j}+\beta D^{j}_{t} +\epsilon^{j}_{it}, i=1, \dots, N, t=0,1,
\]
где $j$ является индексом группы, $D^{j}=1$, если $j=1$ и $D^{j}=0$, если $j=0$.  Величина $D^{j}_{t}=1$, если $j$ и $t$ равны 1 и $D^{j}_{t}=0$ в противном случае, а $\epsilon$ --- случайное возмущение с нулевым математическим ожиданием и постоянной дисперсией. Уравнение не включает регрессоры, но они могут быть добавлены, а те, что не изменяются уже включены в  $\alpha$. Таким образом, для группы получившей воздействие, до воздействия модель примет вид:

\[
y^{1}_{i0}=\alpha+\alpha^{1}D^{1}+\epsilon^{1}_{i0},
\]

и вид 

\[
y^{1}_{i1}=\alpha+\alpha_{1}+\alpha^{1}D^{1}+\beta +\epsilon^{1}_{i1}.
\]

после воздействия.

Эффект воздействия, поэтому равен

\begin{equation}
y^{1}_{i1}-y^{1}_{i0}=\alpha_{1}+\beta+\epsilon^{1}_{i1}-\epsilon^{1}_{i0}.
\end{equation}

Соответствующее уравнения для контрольной группы имеют вид:

\[
y^{0}_{i0}=\alpha+\epsilon^{0}_{i0}, \qquad
y^{0}_{i1}=\alpha+\alpha_{1}+\epsilon^{0}_{i1},
\]
а разница равна
\begin{equation} % здесь ошибка в английском тексте (индекс 1, правильно 0)
y^{0}_{i1}-y^{1}_{i0}=\alpha_{1}+\epsilon^{0}_{i1}-\epsilon^{1}_{i0}.
\end{equation}

Оба уравнения (первые разности) включают коэффициент первого периода, $\alpha_{1}$, который может быть устранён путём вычитания из уравнения (3.6) уравнения (3.7):

\begin{equation}
(y^{1}_{i1}-y^{1}_{i0})-(y^{0}_{i1}-y^{0}_{i0})=\beta+(\epsilon^{1}_{i1}-\epsilon^{1}_{i0})-(\epsilon^{0}_{i1}-\epsilon^{0}_{i0}).
\end{equation}
Предполагая что $\E[(\epsilon^{1}_{i1}-\epsilon^{1}_{i0})-(\epsilon^{0}_{i1}-\epsilon^{0}_{i0})]=0$, мы можем получить несмещённую оценку $\beta$ используя $(y^{1}_{i1}-y^{1}_{i0})-(y^{0}_{i1}-y^{0}_{i0})$. Этот метод называется <<разность разностей>>. Если присутствуют меняющиеся во времени регрессоры, то они могут быть включены в соответствующие уравнения и их разницы появятся в уравнение регрессии (3.8).


Для простоты, наш анализ игнорировал возможность того, что остаются наблюдаемые различия в распределении характеристик между экспериментальной и контрольной группами. Если такие различия есть, то их необходимо учитывать. Стандартным решением является включение таких контрольных переменных в регрессию.


Примером исследования, основанного на естественном эксперименте является работа Ашенфелтера и Крюгера (1994). Они оценивали влияния образования на уровень заработной платы используя данные по идентичным близнецам с разным уровнем образования. В этом случае обычный эксперимент, в котором людям назначают различные уровни образования, просто невозможен. Тем не менее, некоторые контрольные переменные необходимы. Как объясняют авторы:


Мы должны быть уверены в том, что корреляция наблюдаемая нами между образованием и размером заработной платы не появляется из-за корреляции между образованием и способностью работника или другим признаками. Для этого мы используем то преимущество, что однояйцевые близнецы генетически идентичны и имели примерно одинаковые условия в семье, в которой родились.


Данные о близнецах послужили основой для ряда других эконометрических исследований (Розенцвейг и Волпин, 1980; Бронарс и Гроггер, 1994). Поскольку вероятность рождения двойни  не является высокой, важным вопросом  является получение достаточно большой репрезентативной выборки, при этом не стоит забывать про возможность отказа отвечать на вопросы. Одним из источников таких данных является перепись населения. Другой источник --- <<фестивали близнецов>>, которые проводятся в Соединенных Штатах. Ашенфелтер и Крюгер (1994, с. 1158) сообщают, что их данные были получены из интервью, проведенном на 16-ом Ежегодном Фестивале Близнецов, Твинсбург, штат Огайо, в августе 1991. Данный фестиваль является самой крупной встречей однояйцевых близнецов в мире.


Привлекательность использования данных близнецов заключается в том, что наличие общих эффектов от наблюдаемых и не наблюдаемых факторов может быть устранено путем моделирования разницы между результатами близнецов. Например, Ашенфелтер и Крюгер оценили регрессионную модель разницы логарифмов заработной платы между близнецами. Первая разность исключает эффекты от возраста, пола, этнической принадлежности, и так далее. Остальные объясняющие переменные  помимо разницы между уровнем образования (основной переменной исследования) включают, например, разницу опыта работы и семейного положения.


\subsection{Идентифицируемость при естественном эксперименте}

Идеология естественных экспериментов оказала полезное влияние на практику эконометрики. Используя квази-экспериментальные данные и модели потенциального результата главы 2, эконометрическая практика устраняет разрыв между данными наблюдений и экспериментальными данными. 
Понятие идентифицируемости параметров систем одновременных уравнений расширяется, и включает идентифицируемость мер, которые интересны с точки зрения политики. Основное преимущество  естественного эксперимента состоит в том, что  переменная воздействия может обоснованно рассматриваться как экзогенная. 
Однако при использовании данных естественного эксперимента, например, социального эксперимента, выбор контрольной группы сильно влияет на  достоверность выводов. Некоторые из потенциальных проблем социального эксперимента, например, смещение самоотбора или смещение истощения выборки,  также присущи естественным экспериментам. Далеко не все интересные вопросы оценивания политики поддаются анализу в рамках естественного эксперимента. Эксперимент может относиться только к небольшой части генеральной совокупности, а условия, при которых он происходит, могут быть трудно повторимыми. Пример, приведенный в разделе 22.6, иллюстрирует этот вопрос в контексте метода разности разностей.

\section{Практические соображения}


Хотя существует огромный спрос на микроданные, количество имеющихся баз данных можно пересчитать по пальцам. Мы приводим очень неполный список некоторых  очень известных американских баз. Для получения дополнительной информации, см. соответствующие веб-сайты этих наборов данных. Многие из них позволяют напрямую скачивать данные.


\subsection{Некоторые источники микроданных}


{\bf Панельное исследование динамики доходов (Panel Study of Income Dynamics, PSID)}: Проводимое  Центром Обследований (Survey Research Center) Мичиганского Университете, PSID является национальным опросом, проводимым с 1968 года. Сегодня оно охватывает более 40 000 индивидов и включает сбор экономических и демографических данные. Эти данные были использованы в большом количестве микроэконометрических работ. Браун, Дункан и Стаффорд (1996) приводят обзор современного состояние PSID.


{\bf Текущее обследование населения (Current Population Survey, CPS)}: Это ежемесячное национальное обследование около 50000 домохозяйств, которое предоставляет информацию о характеристиках рабочей силы. Исследование проводилось в течение более 50 лет. Существенный пересмотр выборки происходит каждые десять лет. Дополнительная информация об этом исследовании находится в разделе 24.2. На базе этого обследования рассчитываются многие официальные показатели доходов и занятости. Обследование является важным источником микроданных и послужило основой многочисленных исследований, в особенности, рынка труда. Обследование был переработано в 1994 году (Поливка, 1996).


{\bf Национальное лонгитюдное обследование (National Longitudinal Survey, NLS)}: NLS проводится по четырём  когортам: NLS пожилых мужчин, NLS молодых мужчин, NLS пожилых женщины, и NLS молодых женщин. Каждая из исходных когорт является национальным ежегодным опросом более 5000 индивидов, которые повторно опрашивались начиная с середины 1960-х годов. Опросы содержат информацию об опыте работы каждого респондента, уровне образования, повышении квалификации, доходе и составе семьи, семейном положении и здоровье. Доступны также данные о возрасте, поле и т.д. 


{\bf Национальное лонгитюдное обследование молодежи (National Longitudinal Survey of Youth, NLSY)}: NLSY является национальным ежегодным опросом 12686 молодых мужчин и женщин в возрасте от 14 до 22 лет на момент первого обследования 1979 года. Содержит три подвыборки. Данные предоставляют уникальную возможность  изучения  жизненного пути большой выборки молодых людей, которые являются представителями американских мужчин и женщин, родившихся в конце 1950-х --- начале 1960-х. Второе обследование NLSY началось в 1997 году.


{\bf Обследование доходов и участия в программах поддержки (Survey of Income and Program Participation, SIPP)}: SIPP --- лонгитюдное обследование около 8000 домохозяйств в месяц. Оно посвящено источникам дохода, участию в  программах поддержки населения, корреляции между этими величинами, и индивидуальным характеристикам на рынке труда, также тому как показатели меняются с течением времени. Это многопанельное обследования, при котором новая панелью появляется в начале каждого календарного года. Первая панель SIPP была образована в октябре 1983 года. По сравнению с CPS, SIPP имеет меньше занятых и больше безработных лиц.


{\bf Исследование здоровья и выхода на пенсию (Health and Retirement Study, HRS)}: HRS -- лонгитюдное национальное обследование. Первое обследование респондентов в возрасте от 51 до 61 из 7600 домохозяйств состоялось в 1992 году  с последующими повторениями каждые два года в течение 12 лет. Данные содержат огромное количество экономических, демографических показателей и показателей здоровья.


{\bf Обследование уровня жизни Всемирного Банка (World Bank’s Living Standards Measurement Study, LSMS)}: Обследование домохозяйств Всемирного Банка содержит данные  <<по многим аспектам благосостояния домохозяйств, которые могут быть использованы для оценки уровня жизни домохозяйств, для понимания поведения домохозяйств, и оценивания влияния различных государственных мер на условия жизни населения>> во многих развивающихся странах. Много примеров использования этих данных можно найти у Дитона (1997) и в экономической литературе по развитию. Грош и Глевве (1998) описывают особенности данных и приводят ссылки на исследование, в которых использовались эти данные.


{\bf Организации предоставляющие доступ к базам данных}: Межвузовский консорциум политических и социальных исследований (Interuniversity Consortium for Political and Social Research, ICPSR) предоставляет доступ ко многим базам данных, в том числе PSID, CPS, NLS, SIPP, Национальныму обследованию расходов на здравоохранение (National Medical Expenditure Survey, NMES), и многим другим. Американское Бюро статистики труда обрабатывает данные CPS и NLS. Американское Бюро переписи населения обрабатывает данные SIPP. Американский Национальный центр статистики здравоохранения обеспечивает доступ ко многим базам данных по здравоохранению.  Совет европейских архивов данных по социальным наукам (Council of European Social Science Data Archives, CESSDA) предоставляет ссылки на несколько европейских национальных архивов данных.

{\bf Данные из архива журналов}: Для некоторых целей, например, для воспроизведения  опубликованных результатов при обучении студентов, можно  использовать данные из архивов журналов. Два архива предлагают очень удобную процедуру загрузки и скачивания данных через браузер. Журнал деловой и экономической статистики (The Journal of Business and Economic Statistics) ведёт архив данных практически ко всем своим статьям.  Аналогично устроен архив Журнала прикладной эконометрики (The Journal of Applied Econometrics), он содержит данные, относящиеся к большинству статей, опубликованных с 1994 года.



\subsection{Обработка микроэконометрических данных}


Микроэкономические наборы данных, как правило, имеют очень большой размер. Выборка из нескольких сотен или тысяч наблюдений является довольно типичной, а порою встречаются выборки в десятки тысяч наблюдений. Распределения интересующих исследователя переменных чаще всего не имеет нормального распределения, иногда потому что эти переменные являются дискретными. Обработка больших наборов данных, не имеющих нормального распределения,  создает некоторые проблемы при описании важных особенностей данных. Часто бывает полезно использовать одну вычислительную среду (программу) для извлечения, и первичной подготовки данных, а другую для  оценки моделей. 


\subsection{Подготовка данных}

Основная особенность микроэконометрического анализа состоит в том, что процесс получения выборки, которая должна будет использоваться в исследовании, скорее всего, будет долгим. Важно точно документировать решения, принятые исследователем во время <<очистки>> данных. Рассмотрим несколько конкретных примеров.

Одной из наиболее общих черт данных выборочного обследования являются отсутствующие или частичные ответы. Проблемы отсутствия ответа уже обсуждались. Частичный ответ обычно означает, что некоторые части опросных анкет остались без ответа. Если при этом в ответах отсутствует важная информация, то можно, например, не учитывать имеющиеся частичные ответы. Такая стратегия называется полным удалением наблюдений с пропусками. Если эта проблема возникает в значительном числе случаев, то она должна быть тщательным образом проанализирована и описана, потому что это может привести к нерепрезентативности выборки и смещению оценок. Этот вопрос анализируется в главе 27. Рассмотрим, например, ситуацию, в которой индивиды с высокими доходами отказываются отвечать. В таком случае в выборке будет занижено количество индивидов с высокими доходами, что равносильно ситуации с полными ответами по нерепрезентативной выборке.





Вторая проблема заключается в ошибках измерения в получаемых данных. Микроэкономические данные, как правило, содержат много шума. Тип, уровень и существенность ошибок измерения зависит от типа обследования, пространственные данные или панельные, индивидов, которые отвечают на опрос, и переменной, значение которой спрашивается. Например, считается, что панельные данные о доходе содержат существенные коррелированные ошибки измерений. Напротив, обычно предполагают, что данные о расходах имеют меньшую ошибку измерений.
Дитон (1997) исследовал некоторые из источников ошибок измерений, уделяя особое внимание данным Обследования уровня жизни Всемирного Банка (Living Standards Measurement Survey), хотя некоторые из поднятых вопросов имеют более широкое значение. Смещение от ошибок измерения зависит от того, какие преобразования применяются к данным (например, взятие первых разностей), и используемого алгоритма оценивания. Следовательно, чтобы делать содержательные заключения о  смещении, возникающем в результате ошибок измерения, необходимо использовать четко определенные модели. В последующих главах будут приведены примеры влияния ошибок измерения в конкретных условиях.



\subsection{Проверка данных}

В больших наборах данных легко могут встретиться наблюдения с ошибками, возникающими в результате  неправильного набора или ошибок кодирования. Поэтому следует применять некоторые элементарные способы проверки, позволяющие определить наличие проблем. Прежде чем анализировать данные, можно проверить их используя описательные статистики. Могут оказаться полезными следующие идеи. 
Во-первых, используя описательные статистики (минимум, максимум, среднее и медиану), можно убедиться, что данные находятся в надлежащем интервале и имеют правильным масштаб. Например, характеристики бинарных переменных должны быть между нулем и единицей, характеристики счетных данных должны быть больше или равны нулю. Иногда пропущенные данные кодируются как $-999$, или каким-либо другим целым числом, поэтому позаботьтесь о том, чтобы подобные значения не обрабатывались как числа. 
Во-вторых, необходимо знать, как записаны изменения, в долях или в процентах. 
В-третьих, можно использовать диаграмму <<ящик с усами>> для выявления проблемных наблюдений.  
Например, подобная диаграмма может помочь обнаружить страну с отрицательным ростом населения (из-за войны) или страну с уровнем инвестиций превосходящим ВВП (если по методологии зарубежная помощь не включается в  ВВП). 
Проверка наблюдений до оценивания иногда помогает выбрать необходимую нормализацию и / или предположение о законе распределения для моделирования имеющегося набора данных. 
В-четвертых, % в английском тексте дважды "в третьих"
предварительный взгляд на данные может помочь выбрать подходящее преобразование данных. Например, по диаграмме <<ящик с усами>> иногда можно сделать вывод, что для данных подойдет логарифмирование или степенное преобразование.
Наконец, может быть важным проверить шкалы измерения переменных. Для некоторых целей, таких как использование нелинейных оценок, желательно масштабировать переменные так, чтобы они имели приблизительно одинаковые масштабы. Описательные статистики могут быть использованы для проверки сравнимости масштабов.


\subsection{Представление описательных статистик}

Так как микроданные обычно имеют большой объём, очень важно, предоставить таблицу описательной статистики. Как правило в описательную статистику включают среднее, стандартное отклонение, минимум и максимум для каждой переменной. В некоторых случаях неожиданно большие или малые значения могут выявить наличие грубой ошибки записи или ошибочное включение неправильных значений  данных. Диаграммы рассеяния как правило не очень полезны, в то время как таблицы сопряженности могут оказаться ценными. Для дискретных переменных могут быть полезны гистограммы, а для непрерывных переменных информативными являются графики плотности распределения.

\subsection{Библиографические заметки}

\begin{enumerate}
\item Дитон (1997) приводит введение в практику выборочных опросов, в частности, в развивающихся странах. Ссылки касающихся сложных опросов приведены в главе 24. Бекетти и др. (1988) исследуют вопрос репрезентативности Панельного исследования динамики доходов (Panel Study of Income Dynamics, PSID).
\item Коллективная работа под редакцией Хаусмана и Уайза (1985) содержит ряд статей по социальным экспериментам, включая эксперимент Rand по медицинскому страхованию, (Rand Health Insurance Experiment, RHIE), эксперимент с отрицательным подоходным налогом (Negative Income Tax, NIT) и эксперимент с тарифом, зависящим от времени суток (Time-of-Use).
Многие исследования ставят под вопрос полезность экспериментальных данных и существует объемная полемика о недостатках экспериментов, мешающих делать ясные выводы. Достоинства и недостатки социальных экспериментов по сравнению с данными наблюдений обсуждаются в замечательной паре статей Бартлесса (1995) и Хаусмана и Смита (1995).
\item Специальный выпуск Журнала деловой и экономической статистики (Journal of Business and Economic Statistics, 1995) посвящен статьям, использующим методологию естественных и квази-экспериметнов. В подборку входит статья Мейера, в которой можно найти обзор методологии и проблем эконометрических исследований, использующих данные естественных экспериментов. Частично основываясь на работе Кэмпбелла (1969), Мейер также приводит руководство по использованию естественной изменчивости для оценивания воздействия экономической политики. 
Ким и Сингал (1993) исследуют эффект изменения концентрации рыночной власти на цену, используя данные по слиянию авиакомпаний. Розенцвейг и Волпин (2000) составили обзор литературы по естественным экспериментам, в частности по использованию данных о близнецах. Исакссон (1999) использует данные о близнецах для оценивания влияния образования на зарплату в Швеции. Ангрист и Лейви (1999) исследуют влияние размера класса на результаты тестирования используя данные школ в которых применяется правило Маймонида, согласно которому размер класса не должен превышать 40 человек.  Правило порождает инструментальную переменную. Данный сюжет кратко обсуждается в разделе 25.6.



\end{enumerate}


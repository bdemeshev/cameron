
\chapter{Модели смеси и ненаблюдаемая гетерогенность}

% Тонкости перевода:
%
% 1. heterogeneity term
% я переводил словосочетание как "параметр гетерогенности", что может слегка запутать читателя, поскольку обычно "параметр" относится к некоторому распределению. Можно поменять на "переменную", или "элемент" гетерогенности? но как-то не звучит

% составляющая гетерогенности

\section{Введение}\label{sec:18.1}

\noindent
По вопросам ненаблюдаемой гетерогенности существует значительный объем статистической и эконометрической литературы. Ненаблюдаемая гетерогенность включает в себя те отличия между объектами, которые не могут быть измерены с помощью включенных регрессоров. Как правило, и наблюдаемые, и ненаблюдаемые отличия влияют на продолжительность пребывания в определенном состоянии. Если есть ненаблюдаемая гетерогенность, то коэффициенты риска для объектов могут различаться даже при одинаковых значениях объясняющих переменных. Если же эту гетерогенность проигнорировать, то её влияние можно спутать с составляющей базового риска.

Идею можно понять на следующем примере. Пусть агрегированная функция риска для наблюдений по безработице является убывающей функцией от продолжительности поиска работы. Если бы все объекты были одинаковыми, это означало бы негативную зависимость от длительности, то есть, более низкую вероятность найти работу для индивидов, которые дольше находятся в поисках. Предположим, однако, что объекты неоднородны и делятся, в частности, на два типа: быстрые F (fast) с постоянным коэффициентом риска, равным 0.4, и медленные S (slow) с коэффициентом риска 0.1. Совокупность объектов на 50\% состоит из 1-го типа и на 50\% из 2-го. Из 100 наблюдений типа F 40 переходов происходят в первом периоде, 24 перехода во втором и 14.4 --- в третьем. Для типа S мы наблюдаем, соответственно, 10, 9 и 8.1 переходов в первом, втором и третьем периодах. Тогда общие доли переходов будут равны $(40 + 10) / 200 = 0.25$, $(24 + 9) / 150 = 0.22$ и $(14.4 + 8.1) / 117 = 0.192$. Следовательно, убывающий коэффициент риска является следствием агрегирования двух неоднородных групп объектов с различными, но постоянными коэффициентами риска. Таким образом, правильность выводов зависит от наличия и учета ненаблюдаемой гетерогенности.

В моделях линейной регрессии ненаблюдаемая гетерогенность не считается серьезной проблемой, если она некореллирована с набором объясняющих факторов, поскольку в таком случае условное математическое ожидание неизменно, оценки несмещены, а ненаблюдаемые различия заключены в остатках модели. В моделях времени жизни последствия неучета таких различий однако могут быть более серьезными. Даже если предположить, что ненаблюдаемая гетерогенность некореллирована с регрессорами и не оказывает воздействия на ожидаемую длительность, математическое ожидание условного риска будет смещено. Заметим, что именно коэффициент риска представляет интерес для интерпретации при наличии цензурирования. Например, прежде чем определять направления политики, мы, возможно, захотим понять, как условная вероятность найти работу зависит от длительности поисков.

Ненаблюдаемая гетерогенность лежит в основе решения множества эмпирических загадок и несоответствий. Несмотря на то, что эта проблема представлена в контексте анализа выживаемости, рассматриваемые методы применяют и в других областях эконометрики, поскольку все эконометрические модели так или иначе содержат пропущенные ненаблюдаемые индивидуальные эффекты. Примерами моделей, учитывающих такие эффекты, в других областях являются логит-модели со случайными параметрами (см. раздел 15.7)% \ref{sec:15.7} # UNCOMMENT AFTER 15 CH
, модели с самоотбором выборки (см. раздел 16.4)% \ref{sec:16.4} # UNCOMMENT AFTER 16 CH
, модели конечной смеси для счетных данных (см. раздел 20.4)% \ref{sec:20.4} # UNCOMMENT AFTER 20 CH
, а также модели панельных данных с фиксированными и случайными индивидуальными эффектами (см. разделы 21-23)% \ref{ch:21}-\ref{ch:23}
. % Термины должны совпадать с терминами из глав 15, 16, 20, 21-23 !!!
Все эти модели относятся к проблеме ненаблюдаемой гетерогенности. В биостатистике также используется термин \textbf{уязвимость} \textit{(frailty)}, и соответствующие модели называются моделями с уязвимостью \textit{(frailty models)}. В актуарной математике с помощью (мультипликативной) ненаблюдаемой гетерогенности измеряют пропорциональное увеличение или уменьшение коэффициента риска (интенсивности смертности, \textit{``force of mortality''}) для определенного индивида по отношению к среднему индивиду. Допускается, что индивидуальная гетерогенность может зависеть от времени, но при анализе данных пространственного типа удобнее предполагать ее независимость.

Важно понимать последствия такой неизбежной недоспецификации. В моделях линейной множественной регрессии пропуск значимых переменных может приводить к смещению оценок учтенных переменных. В моделях времени жизни, нелинейных в отличие от предыдущего случая, анализ недоспецификации гораздо сложнее. В частности, существует целый класс моделей, называемый \textbf{моделями смеси} и посвященный проблеме ненаблюдаемой гетерогенности. Суть данной главы заключается в том, чтобы представить методы, позволяющие строить и анализировать модели смеси и последствия неучета гетерогенности.

Тот факт, что гетерогенность и истинную зависимость (от состояния) (\textit{true state dependence}, предшествующий опыт) необходимо различать, берет начало с обсуждения истинного и кажущегося \textbf{заражения} \textit{(contagion)}.\footnote{Убывающая функция риска как следствие агрегирования неоднородных групп относится к гетерогенности, а постоянные коэффициенты риска в этих группах являются истинной зависимостью.} В частности, Нейман был одним из первых, кто заметил, что на основе панельных данных возможно обнаружить эти различия эмпирическим путем. Если же доступны данные лишь пространственного типа, то необходимо формулировать предпосылки о распределении параметров. В последнее время в эмпирических исследованиях все чаще стараются уходить от параметрической спецификации и тестировать корректность таких предпосылок.

В первой части главы (разделы \ref{sec:18.2}--\ref{sec:18.4}) мы рассмотрим модели смеси, основанные на предположении о непрерывном распределении гетерогенности. Модели с дискретной гетерогенностью будут представлены в разделе \ref{sec:18.5}. В следующем разделе \ref{sec:18.6} мы определим взаимосвязь между различными понятиями длительностей, свойственных для данных типа запас и типа поток. Тесты на неправильную спецификацию и пропуск гетерогенности описаны в разделе \ref{sec:18.7}. Наконец, эмпирический пример в разделе \ref{sec:18.8} является иллюстрацией к идеям, представленным в этой главе.




\section{Ненаблюдаемая гетерогенность и дисперсия}
\label{sec:18.2}

\noindent
В данном разделе мы рассмотрим ненаблюдаемую гетерогенность в экспоненциальной и вейбулловской моделях. Спецификация в форме произведения (то есть, мультипликативно) позволяет исключить ненаблюдаемую гетерогенность при интегрировании, не изменяя условное математическое ожидание, но завышая дисперсию и, что наиболее важно, воздействуя на условную функцию риска. Здесь мы также представим известную модель Вейбулла с гамма-распределенной гетерогенностью.


\subsection{Смешанные модели}\label{sec:18.2.1}

\noindent
Рассмотрим наиболее простой случай, экспоненциальную модель времени жизни. При отсутствии гетерогенности распределение завершенных длительностей $t_i$ задается как условное относительно слабо экзогенных регрессоров $\x_i$. Это эквивалентно отсутствию случайной компоненты в условном матожидании: $\E[T|\x]=\exp{(\xb)}$. В моделях смеси мы задаем распределение $(t_i|\x_i,\nu_i)$, где $\nu_i$ обозначает составляющую ненаблюдаемой гетерогенности для наблюдения $i$. Другими словами, мы допускаем, что индивиды могут иметь случайные различия, неучтенные объясняющими факторами. Маргинальное распределение для $t_i$ получается с помощью усреднения по отношению к $\nu_i$.

Необходимо определить конкретную функциональную зависимость $t_i$ от $(\x_i,\nu_i)$. Чаще всего предполагают, что регрессоры представлены в виде экспоненты с \textit{мультипликативной} ошибкой. Например, модель пропорциональных рисков (PH, proportional hazard), обозначенная уравнениями \ref{eq:17.25} 
и \ref{eq:17.26} 
в разделе \ref{sec:17.8} 
с учетом ненаблюдаемой гетерогенности $\nu$ будет выглядеть следующим образом
    $$\la(t|\x,\nu)=\la_0(t)\exp(\xb)\nu, \nu>0.$$
Следовательно, можно найти интегральный базовый риск, который будет равен
    \begin{align}
    \label{eq:18.1}
    \la_0(t)                        &=\la(t|\x,\nu)\exp(-\xb)\nu^{-1},\\
    \int\la_0(u)du                  &=\exp(-\xb)\nu^{-1}\int\la(u|\x,\nu)du, \notag \\
    \ln\left[\int\la_0(u)du\right]  &=-\xb-\ln\nu+\e, \notag
    \end{align}
где предполагается, что $\e=\ln\int\la(x|\x,\nu)du$, и $\nu$ независима от регрессоров и моментов цензурирования. Часто используют нормализацию $\E[\nu]=1$ (по соображениям идентифицируемости). Если $\nu>1$, то коэффициент риска выше среднего, если же $\nu<1$, то ниже. Предпосылка о независимости является строгой, хотя и необязательно реалистична. Предпосылка о мультипликативном характере связи также сделана специально, поскольку она удобна для математических вычислений и гарантирует неотрицательность $t_i$. В рамках стандартного подхода сперва делаются предположения о распределении $\nu_i$, а затем выводится частное распределение для $t_i$.

Спецификация гетерогенности в мультипликативной форме имеет важное следствие. 
% Кэмерон, мы искали второе --- не нашли :)
Очевидно, что дисперсия распределения смеси (условие относительно наблюдаемых переменных) превышает дисперсию распределения всей совокупности (условие относительно наблюдаемых переменных и составляющей гетерогенности). Следовательно, дисперсия будет завышена. Рассмотрим экспоненциальный случай. Произведем замену $\mu_i=\exp(\xib)$ на
    \begin{align}
    \label{eq:18.2}
    \mu_i^*&=\E[t_i|\x_i,\nu_i]\\
            &=\exp(\xib)\nu_i \notag \\
            &=\exp(\xib)\exp(\e_i) \notag \\
            &=\exp(\beta_0+\e_i+\x_{1i}'\be_1), \notag
    \end{align}
где в третьей строке составляющая ненаблюдаемой гетерогенности, $\nu_i$, записана как $\exp(\e_i)$, а в последней строке $\xib$ разделен на свободный член и коэффициент наклона. Результат в последней строке можно интерпретировать как условное среднее со случайным свободным членом $(\beta_0+\e_i)$. Обычно предполагают, что $\nu_i$ независимо одинаково распределены ($iid$) и не зависят от $\x_i$; также иногда определяют конкретную форму распределения.

Предположим, что $\nu_i \sim \mathrm{iid}$ с $\E[\nu_i]=1$ и $\V[\nu_i]=\sigma_{\nu}^{2}$, где условие $\E[\nu_i]=1$ позволяет идентифицировать свободный член. Моменты экспоненциального распределения $t_i$ могут быть записаны как $\E[t_i|\x_i,\nu_i]=\mu_i\nu_i$. Используя теорему о декомпозиции дисперсии из раздела A.8, 
получим
    \begin{align}
    \label{eq:18.3}
    \V[t_i|\x_i]    &=\V_{\nu}[\E_{t|\nu,\x}(t_i|\nu_i,\x_i)] + \E_{\nu}[\V_{t|\nu,\x}(t_i|\nu_i,\x_i)] \\
                    &=\mu_i^{2}\V(\nu_i)+\mu_i^{2}(\V(\nu_i)+1) \notag\\
                    &=\mu_i^{2}[1+2\sigma_{\nu}^{2}] \notag\\
                    &>\mu_i^{2}. \notag
    \end{align}
То есть, ненаблюдаемая гетерогенность завышает безусловную дисперсию.


\subsection{Выбор распределения неоднородности}\label{sec:18.2.2}

\noindent
Чтобы изучить влияние ненаблюдаемой гетерогенности на распределение $t$, необходимо вывести частное распределение $t_i$ за счет исключения составляющей $\nu$ из $S(t|\x,\nu)$ с помощью интегрирования, где параметрическое распределение $\nu$ обычно задано. Но чем необходимо руководствоваться при выборе этого распределения?

Во-первых, необходимо принимать во внимание, что $\nu_i>0$. Следовательно, можно выбрать такое распределение, которое порождает только положительные значения случайной величины. Примерами являются гамма, обратное гауссовское и лог-нормальное распределения.

\textbf{Функция плотности гамма распределения} равна
    \begin{align}
        \label{eq:18.4}
        g(\nu;\de,k) = \frac{\de^k\nu^{k-1}\exp(-\de\nu)}{\Ga(k)}, \nu>0,
    \end{align}
c математическое ожиданием $\E[\nu]=k/\de$ и дисперсией $\V[\nu]=k/\de^2$. Нормализуя $\E[\nu]=1$, получим $k=\de$ и $\V[\nu]=1/\de$. Данное распределение удобно для математических вычислений и встроено в большинство популярных статистических пакетов для анализа времени жизни.

\textbf{Функция плотности обратного гауссовского распределения} равна
    \begin{align}
        \label{eq:18.5}
        g(\nu;\de,\ttt) = \de\pi^{1/2}\exp(2\de\ttt^{1/2})\nu^{-3/2}\exp(-\ttt\nu-\de^{2}/\nu), \nu>0,
    \end{align}
c математическое ожиданием $\E[\nu] = \de\ttt^{1/2}$ и дисперсией $\V[\nu]=\de\ttt^{-3/2}/2$. Нормализуя $\E[\nu]=1$, получим $\ttt=\de^2$ и $\V[\nu]=1/2\ttt$. По сравнению с гамма, обратное гауссовское распределение имеет более тяжелый хвост.

Заметим, что нет гарантии, что мы сможем аналитически найти частное распределение $t$. Например, некоторые смеси, такие как сочетание Вейбулла и гамма распределений, позволяют найти решения в аналитическом виде, в то время как другие --- нет. Наличие аналитического решения удобно лишь для математических вычислений и не является необходимостью. % не представляет интереса само по себе.
К сожалению, экономическая теория редко предсказывает данный аспект моделирования.

Во-вторых, при выборе распределения важно учитывать его универсальность и гибкость. Так, гамма обладает рядом привлекательных свойств, которые делают его довольно гибким по отношению к данным. Однако для распределений с более тяжелым правым хвостом, вероятно, лучше подойдет обратное гауссовское. Оба распределения являются однопараметрическими (после нормализации). Более гибкий двухпараметрический класс распределений представлен в работе Хугаард (1986), для которого гамма и обратное гауссовкое являются частными случаями. Дискретное (непараметрическое) представление, % ALT: дискретный случай
которое будет рассмотрено позже, также допускает % ALT: позволяет, допускает
значительную гибкость при оценивании.


\subsection{Смесь распределений Вейбулла--гамма}\label{sec:18.2.3}

\noindent
В данном разделе мы рассмотрим известную \textbf{смесь распределений Вейбулла и гамма}, для которой смесь экспоненциального и гамма является частным случаем. В силу своей универсальности (а именно, потому, что позволяет моделировать и убывающие, и возрастающие риски) она является наиболее важной \textbf{моделью смешанных пропорциональных рисков} (MPH).

В модели Вейбулла условная функция выживания относительно мультипликативного $\nu$ равна
    \begin{align}
        \label{eq:18.6}
        S(t|\nu) = \exp(-\mu t^\al\nu), \la>0, \al>0,
    \end{align}
где ранее в главе \ref{ch:17} 
на месте $\mu$ находилась $\al$.

Безусловную функцию выживания можно получить как среднюю функцию, взвешенную по плотности распределения неоднородной совокупности $\nu$, $g(\nu)$, или
    \begin{align}
        \label{eq:18.7}
        S(t) = \E_\nu[S(t|\nu)] = \int S(t|\nu)g(\nu)d\nu.
    \end{align}
Тип смеси зависит от выбора распределения $g(\nu)$, которое, при соответствующих изменениях в формуле, может быть как непрерывным, так и дискретным. Интеграл в \ref{eq:18.7} необязательно должен иметь аналитическое решение. Например, для лог-нормального распределения $g(\nu)$ решения нет, а для гамма --- есть. Для удобства математических вычислений мы будем работать с гамма распределением.

При гамма-распределенной гетерогенности безусловная функция выживания записывается как
    \begin{align}
        \label{eq:18.8}
        S(t) &= \int^{\infty}_{0}\exp(-\mu t^{\al}\nu) \frac{\de^k\nu^{k-1}\exp(\de\nu)}{\Ga(k)} d\nu \\
             &= \frac{\de^k}{\Ga(k)} \int^{\infty}_{0}\nu^{k-1}\exp(-\nu(\mu t^{\al} + \de)) d\nu. \notag
    \end{align}
Чтобы получить распределение смеси, необходимо взять интеграл. Пусть $\mu t^{\al} + \de = \beta$, тогда
    $$S(t) = \frac{\de^k}{\Ga(k)} \int^{\infty}_{0} \frac{(\nu\beta)^{k-1}}{\beta^{k-1}} \exp(-\nu\beta) d\nu.$$
Произведя замену $y = \nu\beta$ так, что $d\nu = \beta^{-1}dy$, получим
    \begin{align}
        \label{eq:18.9}
        S(t) &= \frac{\de^k}{\Ga(k)\beta^k} \int^{\infty}_{0} y^{k-1} \exp(-y) dy \notag\\
             &= \frac{\de^k}{\Ga(k)} \frac{\Ga(k)}{(\mu t^\al + \de)^k} \notag \\
             &= \de^k(\mu t^\al + \de)^{-k} \notag \\
             &= [1 + (\mu t^\al / \de)]^{-k},
    \end{align}
где во второй строке мы использовали определение функции распределения $\Ga(k)$ и произвели обратную замену для $\beta$.

Безусловную плотность распределения длительностей можно получить, взяв производную по $t$ и домножив на $-1$, то есть
    \begin{align}
        \label{eq:18.10}
        f(t) = \frac{k}{\de} \mu \al t^{\al-1} [1 + (\mu t^\al /\de)]^{-(k+1)}.
    \end{align}
Следовательно, безусловная функция риска $\la(t) = f(t) / S(t)$ равна
    \begin{align}
        \label{eq:18.11}
        \la(t) = \frac{k}{\de} \mu \al t^{\al-1} [1 + (\mu t^\al /\de)]^{-1}.
    \end{align}

Эти выражения, представленные в общем виде, можно упростить, положив, что $k = \de$, что равносильно нормализации $\E[\nu] = 1$. Тогда выражения для \textit{смеси Вейбулла--гамма} будут выглядеть следующим образом:
    \begin{align}
        \label{eq:18.12}
        S(t) = [1 + (\mu t^\al / \de)]^{-\de},
    \end{align}

    \begin{align}
        \label{eq:18.13}
        f(t) = -\frac{\pa S(t)}{\pa t} = \mu \al t^{\al-1}[1 + (\mu t^{\al} / \de)]^{-(\de + 1)},
    \end{align}

    \begin{align}
        \label{eq:18.14}
        \la(t) = -\frac{\pa \ln S(t)}{\pa t} = \mu \al t^{\al-1}[1 + (\mu t^{\al} / \de)]^{-1},
    \end{align}
где коэффициент риска стремится к коэффициенту риска в модели Вейбулла при дисперсии $1/\de$, стремящейся к нулю.

Хотя модель Вейбулла позволяет моделировать как возрастающие, так и убывающие риски, она также требует предпосылки об условно монотонных рисках на индивидуальном уровне. Тем не менее, из-за своих удобных свойств она получила широкое распространение в эконометрической литературе; см. Ланкастер (1979), Нарендранатан, Никелл и Стерн (1985).

\textbf{Смесь экспоненциального и гамма распределений} получается, если положить $\al = 1$. В этом случае соответствующие выражения равны $S(t) = [1 + (\mu t / \de)]^{-\de}$, $f(t) = \mu [1 + (\mu t / \de)]^{-(\de + 1)}$, и $\la(t) = \mu[1 + (\mu t / \de)]^{-1}$. Данная смесь также называется \textbf{распределением Парето} второго рода и имеет более тяжелые хвосты по сравнению с обычным экспоненциальным распределение из-за различия в дисперсии, $1/\de$. Заметим, что $r$-ый момент существует только для $\de > r$.


\subsection{Интерпретация функции риска в моделях смеси}\label{sec:18.2.4}

\noindent
В экономических приложениях ключевой интерес представляет наличие положительной или отрицательной зависимости от длительности, например, увеличение или снижение вероятности найти работу с продлением периода поисков. Вероятность найти работу может расти из-за роста альтернативных издержек быть безработным. Вероятность найти работу может падать, если рассматривать работника как <<плохой товар>>.
% по смыслу
% (e.g., owing to worker is reservation wage falling)
% (e.g., owing to the worker being viewed as damaged goods)
В случае, если объекты независимо и одинаково распределены ($iid$), истинную зависимость можно установить с помощью непараметрических методов оценивания. Если же объекты распределены неодинаково или зависимы ($non$-$iid$), убывающая функция риска может получиться в результате агрегирования как убывающих, так и постоянных индивидуальных рисков. Понять, какой из случаев на самом деле имеет место, бывает довольно трудно.

Обратимся к интерпретации функции риска в модели смеси экспоненциального и гамма распределений при наличии ненаблюдаемой гетерогенности. Заметим, что даже если индивидуальный риск (то есть, риск при условии $\nu$) постоянен при $\mu$, средний, или агрегированный, риск $\la(t)$ убывает по $t$. То есть, на основе значений агрегированного риска нельзя судить о характере взаимосвязи на индивидуальном уровне. Негативная зависимость наблюдается именно из-за агрегирования объектов, обладающих случайными различиями в функциях риска. Подобная ошибочная интерпретация может возникнуть и в смеси Вейбулла и гамма распределений, где наклон индивидуальной функции риска по-прежнему зависит от $\al$, но агрегированная функция риска подвержена воздействию гетерогенности. Поэтому неучет ненаблюдаемой гетерогенности может приводить к недооценке коэффициентов наклона функции риска. Результат является довольно общим (см. Ланкастер, 1990); подробное описание представлено ранее в работе Саланта (1977).

Результат основан на утверждении (см. например, Ланкастер, 1979; Хекман и Cингер, 1984a), что при наличии ненаблюдаемой гетерогенности оценки могут быть существенно смещены, что указывает на необходимость проведения тестов на наличие ненаблюдаемых эффектов. Рассмотрим эту идею в контексте модели смеси Вейбулла, где $S(t) = \int \exp(-\mu t^{\al}\nu) g(\nu) d\nu$. Агрегированная функция риска равна
    \begin{align}
        \la(t) &= -\int\frac{\pa \ln S(t|\nu)}{\pa t} g(\nu) d\nu \notag\\
               &= \al\mu t^{\al-1} \int \frac{\nu\exp(-\mu t^{\al}\nu)}{S(t|\nu)} g(\nu) d\nu \notag \\
               &= \al\mu t^{\al-1} \E[\nu|T\ge t]. \notag
    \end{align}
Поскольку математическое ожидание $\E[\nu|T\ge t]$ рассчитывается как среднее по $\nu$ среди доживших до момента $t$, со временем оно должно убывать, так как индивиды с высокими значениями $\nu$ покидают состояние быстрее, чем индивиды с низкими значениями $\nu$. Как следствие, наклон агрегированной функции риска меняется. Такой эффект можно воспринимать как форму \textbf{смещения отбора} \textit{(selectivity bias)}, % термин должен совпадать с термином из главы 16.5!!!
см. раздел 16.5. % \ref{sec:16.5} # UNCOMMENT AFTER 16 CH
Формально, среднее по $\nu$ при условии пройденного времени может быть записано как
    $$\E[\nu|T\ge t] = \int\frac{\nu\exp(-\mu t^\al\nu)}{S(t|\nu)} g(\nu) d\nu.$$
Тогда для модели смеси Вейбулла
    \begin{align}
    \label{eq:18.15}
        \frac{\pa\E[\nu|T\ge t]}{\pa t} &= -\al\mu t^{\al-1} \left[ \int\frac{\nu^2\exp(-\mu t^\al\nu)}{S(t|\nu)} g(\nu) d\nu \right] \notag\\
                                        &\hspace{0.43cm} +\al\mu t^{\al-1} \left[ \int\frac{\nu\exp(-\mu t^\al\nu)}{S(t|\nu)} g(\nu) d\nu \right]^2 \notag \\
                                        &= -\al\mu t^{\al-1} \{\E[\nu^2|T\ge t] - (\E[\nu|T\ge t])^2 \} \notag\\
                                        &= -\al\mu t^{\al-1} \V[\nu|T\ge t] \\
                                        &< 0. \notag
    \end{align}
В результате, неучет ненаблюдаемой гетерогенности приводит к тому, что оценка коэффициента риска падает быстрее (или же медленнее растет), чем истинный коэффициент.

Интересно также сопоставить пропорциональный эффект изменения регрессоров на коэффициент риска в модели при наличии и при отсутствии гетерогенности. При отсутствии, логарифм условной функции риска равен
    $$\ln\la(t|\mu) = \ln(\mu t^{\al-1}) + \ln\al,$$
и пропорциональный эффект изменения $x_j$ на $\mu$ составляет
    $$\frac{\pa\ln\la(t|\mu)}{\pa x_j} = \be_j,$$
что является одним из свойств модели пропорциональных рисков.

При наличии гетерогенности
    \begin{align}
        \ln\la(t|\mu) &= \ln(\mu t^{\al-1}) + \ln\al + \ln\E[\nu|T\ge t] \notag \\
                      &= \ln\al + \ln\mu + (\al-1)\ln t + \ln\E[\nu|T\ge t], \notag
    \end{align}
откуда, зная, что $\ln\mu = \xb$ и $\pa\E[\nu|T\ge t]/\pa x_j = -\mu t^{\al}\V[\nu|T\ge t]\beta_j$, для модели смеси Вейбулла получим
    \begin{align}
        \label{eq:18.16}
        \frac{\pa\ln\la(t|\mu,\nu)}{\pa x_j} &= \beta_j \left[ 1 - \frac{\mu t^\al \V[\nu|T\ge t]}{\E[\nu|T\ge t]} \right] \\
                                             &< \beta_j. \notag
    \end{align}
То есть, при данной гетерогенности пропорциональный эффект изменения $x_j$ меньше и зависит от $t$, и кроме того, больше не относится к типу пропорциональных рисков. Следовательно, полученные оценки могут быть неверными и вести к неправильным выводам, даже если составляющая  гетерогенности некоррелирована с наблюдаемыми регрессорами.

Аналогичные результаты для более общих, чем Вейбулла, моделей представлены в Ланкастер и Никелл (1980).




\section{Идентификация в моделях смеси}\label{sec:18.3}

\noindent
Моделям смеси свойственна общая \textbf{проблема идентификации}, которая относится к возможности (или невозможности) последовательного разложения индивидуальных вкладов на среднюю вероятность выживания базового риска, ненаблюдаемую гетерогенность и объясняющие переменные для наблюдаемого набора данных $(t,\x)$ по отношению к единственному состоянию. Точнее говоря, если модель неидентифицируема, такое разложение невозможно. Как и в аналогичных обсуждениях идентификации прочих моделей, формулировка
основывается на определенных ограничениях. Модели (смешанных) пропорциональных рисков в эконометрической литературе представлены детально. В частности, Хекман и Cингер (1984b) и Элберс и Риддер (1982) доказали идентифицируемость модели смешанных пропорциональных рисков (MPH) при определенных условиях. Эти и более поздние доказательства можно также найти в работе Ван ден Берга (2001).

Рассуждения об идентифицируемости модели смешанных пропорциональных рисков (MPH) следует начать с определения \textbf{средней}, или \textbf{агрегированной, функции выживания}
    \begin{align}
        \label{eq:18.17}
        S(t|\x) &= \E_\nu[S(t|\x,\nu)] \\
                &= \int\exp(-\nu\La_0(t)\phi(\x)) g(\nu)d\nu, \notag
    \end{align}
где предполагается, что риски пропорциональны как в \ref{eq:18.1}, а формулировки для пропорциональных рисков (PH) взяты из раздела \ref{sec:17.8} 
но без предположений о распределении параметров $\La_0, \phi$ или $g$. Здесь $\La_0(t) = \int^{T}_{0}\la_0(s)ds$. Модель называется непараметрически идентифицируемой, если для набора данных функции $\la_0, \phi$ и $g$ единственные; ``непараметрически'' --- ввиду отсутствия каких-либо предположений о функциональной форме.

Разброс наблюдаемых моментов выживания можно объяснить изменениями ковариат $\x$, $\nu$ и базового риска (функции зависимости от длительности). Идентифицируемость означает единственность такой декомпозиции, следовательно, доказательство должно основываться на том, что отдельные компоненты, в принципе, идентифицируемы. Большинство существующих доказательств предполагают использование сложных математических методов для того, чтобы показать, что функцию правдоподобия можно разложить единственным образом. Мелино и Суейоши (1990) предлагают более простое доказательство.

Для того чтобы модель была непараметрически идентифицируема, требуется, чтобы выполнялись следующие условия:
(i) Составляющая гетерогенности $\nu$ не зависит от времени и распределен независимо от набора переменных $\x$.
(ii) $g(\nu)$ невырождена с конечным математическим ожиданием $\E[\nu] < \infty$.
(iii) $\phi(\x) > 0$ для всех $\x$.
(iv) $\la_0(t)$ непрерывна и положительна в интервале $[0, \infty)$
(v) Наблюдаемые объясняющие переменные $\x$ линейно независимы и обладают достаточной изменчивостью.
Различные доказательства используют различные модификации перечисленных условий, но мы не будем вдаваться в подробности.

Тогда как непараметрическая идентифицируемость предполагает использование сложных математических инструментов, проблема актуальна и для параметрических моделей. Предположив определенную параметрическую форму для $\la_0(t|\al)$, $\phi(\x|\be)$ и $g(\nu|\ga)$, можем ли мы утверждать что она единственна? К сожалению, во многих случаях ответ на этот вопрос будет отрицательным. То есть, мы можем оценить определенную модель смеси без каких-либо вычислительных трудностей и получить вроде бы неплохие результаты со значимыми коэффициентами. Но в то же время при других предпосылках мы получим не менее хорошие результаты, но выводы на этот раз будут другими. Таким образом, наблюдаемая функция выживания согласуется с различными предположениями о базовом риске и распределении гетерогенности (Ланкастер, 1990, глава 4). В терминах из раздела 2.2 % \ref{sec:2.2} # UNCOMMENT AFTER 2 CH
это означает, что различные структурные модели при существенно разных выводах могут иметь одинаковую приведенную форму. Это, соответственно, поднимает вопрос о способах применения параметрических методов. Такими способами могут быть выбор гибкой параметрической формы, или же применение полупараметрического подхода к анализу методом частичного правдоподобия. Мы продолжим обсуждение в следующем разделе.




\section{Спецификация распределения неоднородности}\label{sec:18.4}

\noindent
Чувствительность оценок коэффициентов к различным предположениям о распределении гетерогенности изучена в литературе подробно. На основе предыдущих исследований можно выделить две, на первый взгляд, противоположные точки зрения:

\begin{enumerate}
    \item
Выбор параметрической формы ненаблюдаемой гетерогенности осуществляется зачастую произвольным образом, поэтому выводы о поведении функции риска могут быть существенно искажены. Следовательно, предпочтительно применение гибкой параметрической формы или же непараметрической спецификации. См. Хекман и Cингер (1984a).

    \item
Последствия выбора некорректной параметрической формы ненаблюдаемой гетерогенности относительно безобидны, если спецификация функции базового риска верна. Если же форма функции риска неоднозначна, то оценки на основе различных предпосылок о распределении гетерогенности могут приводить к различным оценкам маргинального распределения данных. См. Мантон, Сталлард и Вопель (1986).
\end{enumerate}

Кажущееся противоречие можно разрешить следующим образом. Спецификация функции риска воздействует на первый момент распределения $f(t)$, в то время как спецификация гетерогенности воздействует на второй момент, при условии, что гетерогенность независима от наблюдаемых ковариат. Если функция риска специфицирована верно, то выбор распределения гетерогенности будет оказывать эффект, в основном, на относительную эффективность оценок.




\subsection{Гамма гетерогенность для PH в дискретном времени}\label{sec:18.4.1} % гамма гетерогенность для пропорциональных рисков в дискретном времени

\noindent
На основе предыдущих рассуждений можно ожидать, что определенное распределение гетерогенности должно сочетаться с функцией риска произвольной формы в модели пропорциональных рисков. Хан и Хаусман (1990) и Мейер (1990) предложили объединить гамма-распределенную гетерогенность с моделью пропорциональных рисков в дискретном времени, представленной в разделе \ref{sec:17.10} % UNCOMMENT AFTER 17 CH.
Согласно полученным результатам, оценки слабо чувствительны к альтернативным спецификациям функциональной формы $g(\nu)$, если базовый риск не зависит от параметров.

Для определенности, уравнение \ref{eq:17.43} % UNCOMMENT AFTER 17 CH
с учетом гетерогенности можно перезаписать как
    $$\e_i = \ln \left( \int \la_0 (\tau) d\tau - \xib -\nu_i \right),$$
и подставить в выражение логарифма правдоподобия (17.44). % \ref{eq:17.44} # UNCOMMENT AFTER 17 CH
Составляющая гетерогенности следует исключить с помощью интегрирования. Решение в аналитическом виде для гамма гетерогенности представили Хан и Хаусман, где указали на относительно незначительную чувствительность по отношению к параметрическим распределениям при данной гибкой спецификации риска.


\subsection{Другие модели с гетерогенностью}\label{sec:18.4.2}

%%%%%% ПЕРЕСМОТРЕТЬ %%%%%%%
\noindent
Как уже упоминалось, модели смеси Вейбулла--Гамма удобны для математических вычислений, поскольку имеют аналитическое решение.

Однако если частное распределение имеет более тяжелые хвосты, нежели гамма или лог-нормальное, имеет смысл выбрать распределение из \textbf{семейства устойчивых распределений} Мандельброта. В частности, довольно широкий класс, включающий, в том числе, гамма и обратное гауссовское распределения, представлен в работе Хугаард (1986) (см. также Джаггиа, 1991b). Распределение называется строго устойчивым, если сумма его $p$ независимых реализаций подчиняется исходному распределению, умноженному на $p$.
% ALT: Распределение называется строго устойчивым, если сумма его $p$ независимых реализаций имеет то же распределение, что и исходное, умноженное на $p$.
Краткое изложение свойств можно найти в Хугаард (2000, аппендикс 3.3).

Хотя распределение гетерогенности с большим числом параметров в силу своей универсальности выглядит привлекательнее, его применение может быть связано с трудностями следующего характера. Во-первых, если объем данных относительно небольшой, степеней свободы может быть недостаточно для идентификации или точной оценки параметров. Обычно, трудно понять, что это так, не попытавшись сперва оценить модель.

Во-вторых, трудности могут возникнуть при вычислениях. Если плотность смеси не может быть выражена аналитически, она записывается в форме интеграла. В результате логарифм правдоподобия состоит из элементов, являющихся интегралами, и для максимизации требуется произвести значительный объем вычислений, используя численные методы интегрирования, или интегрирование по методу Монте-Карло (см. главу 12). % \ref{ch:12} # UNCOMMENT IN THE END OF THE BOOK
Примером модели смеси, требующей применения таких методов, является смесь Вейбулла и лог-нормального распределений, где ненаблюдаемая гетерогенность распределена лог-нормально. Оценивание моделей гетерогенности на основе симуляционных методов представлено в работе Gouri\'eroux и Monfort (1991, 1996) и рассмотрено в качестве примера в разделе 12.2. % \ref{sec:12.2} # UNCOMMENT AFTER 12 CH




\section{Дискретная гетерогенность и анализ латентных классов}\label{sec:18.5}

\noindent
До сих пор анализ подразумевал оценивание параметров непрерывного распределения ненаблюдаемой гетерогенности.

Альтернативный подход предполагает, что выборка объектов построена на основе генеральной совокупности, которая состоит из конечного числа \textbf{латентных классов} $q$, таким образом, что каждый каждый элемент в выборке принадлежит к одному из этих классов, или страт. Такая модель известна как модель конечной смеси, \textbf{полупараметрическая модель гетерогенности} (Хекман и Cингер, 1984a) или \textbf{модель латентных классов} (Айткен и Рубин, 1985). Результатом модели является гибкое параметрическое распределение, что является значительным преимуществом при ее оценивании. В моделировании длительностей сторонниками данного подхода можно назвать Хекмана и Cингер (1984a), которые занимались анализом и применением модели.

Несмотря на то, что эти модели представлены в контексте анализа времени жизни, их применение гораздо шире, в связи с чем мы будем использовать более общие обозначения; см. раздел 20.4. %\ref{sec:20.4}


\subsection{Модель конечной смеси}\label{sec:18.5.1}

\noindent
Рассмотрим следующую \textit{модель конечной смеси}. Если выборка является вероятностной смесью двух групп (классов) с плотностями распределения $f_1(t|\mu_1(\x))$ и $f_2(t|\mu_2(\x))$, то $\pi f_1(\cdot) + (1 - \pi)f_2(\cdot)$, где $0\le\pi\le1$, задает двухкомпонентную конечную смесь. То есть, с вероятностью $\pi$ мы наблюдаем объект из группы с плотностью $f_1$ и с вероятностью $1-\pi$ --- из группы с плотностью $f_2$. Следовательно, требуется оценить параметры $(\pi,\mu_1,\mu_2)$, где параметр $\pi$ может выступать как в роли экзогенной константы, так и в роли эндогенной переменной, для оценки которой можно использовать, например, логит-функцию. Тогда $\pi = \exp(\la)/[1+\exp(\la)]$, и далее с помощью наблюдаемых ковариат оценивается $\la$. Таким образом, мы подразумеваем два типа индивидов, наблюдаемых с вероятностями $f_1(\cdot)$ и $f_2(\cdot)$. Возможно, есть некоторая априорная теория, которая предсказывает принадлежность объектов к одной из этих групп, например, если существует некая латентная характеристика, которая разделяет выборку на две части. Иначе говоря, % ALT: интерпретация, альтернативная формулировка идеи
идея заключается в том, что линейная комбинация функций плотности лучше описывает наблюдаемое распределение длительностей $t$.

Обобщение модели до аддитивных смесей с тремя и более компонентами, в принципе, не связано ни с какими проблемами, за исключением возможной неидентифицируемости компонент. Этот случай будет рассмотрен в главе далее. % ALT: момент, обсуждение будет представлено и т.д.
Таким образом, в эмпирических приложениях желательно, чтобы компоненты были обоснованы с интуитивной точки зрения. Наиболее простым образом компоненты можно интерпретировать как ``типы'', но в некоторых ситуациях возможна более информативная интерпретация (см. Линдсей, 1995).

Иначе, модель конечной смеси можно сформулировать, представив гетерогенность совокупности объектов в дискретном виде. Пусть, совокупность состоит из $m$ гомогенных групп, обычно называемых \textbf{компонентами}. Параметрическая модель, например, экспоненциальная или Вейбулла, применяется к каждой из компонент. Предположим, доля $j$-ой компоненты в генеральной совокупности равна $\pi_j$, при этом $\sum\pi_j = 1$.

Формально, задача может быть представлена следующим образом: До сих пор носитель распределения ненаблюдаемой гетерогенности имел бесконечное число точек. Если непрерывное смешиваемое распределение $g(\nu_i)$ может быть аппроксимировано дискретным распределением, $\pi_j(j = 1, \ldots, m)$, с конечным числом точек $m$, то маргинальное распределение (смеси) равно
    \begin{align}
        \label{eq:18.18}
        h(t_i|\x_i,\pi_j,\be) = \sum^{m}_{j=1} f(t_i|\x_i, \nu_j, \be) \pi_j(\nu_j),
    \end{align}
где $\pi_j$ --- вероятность, соответствующая оценке точки определения $\nu_j$. В моделировании длительностей такая полупараметрическая форма ненаблюдаемой гетерогенности была рассмотрена Хекман и Cингер, 1984a. Подобной работой является работа Веделя и др. (1993), где гетерогенность сформулирована в виде латентных классов. Если смешиваемое распределение $\pi_j$ не зависит от параметрических предпосылок, то модель смеси называется \textbf{полупараметрической моделью смеси} для $t$.

Оценивание модели конечной смеси можно проводить, предполагая, что число компонент или известно, или же нет. Если известны доли $\pi_j$, то оценки распределений компонент можно получить методом максимального правдоподобия. Зачастую же доли $\pi_j$, $j=1,\ldots,m$, неизвестны, и требуется оценить не только $\pi_j$, но и параметры компонент. Функция оценки ММП в таком случае называется функцией оценки непараметрическим методом максимального правдоподобия (NPMLE), где непараметрической составляющей является число классов. Заметим, однако, что это строго полупараметрический метод, поскольку он совмещен с параметрической моделью для компонент. Обычно число компонент неизвестно, и поэтому в отношении выводов требуется аккуратность; см. раздел 18.5.4. % \ref{sec:18.5.4}

Применение моделей конечной смеси обосновано тем, что зачастую разумней и проще предполагать, что неоднородность совокупности состоит из небольшого числа латентных классов, а не из континуума ``типов'', как было представлено ранее в разделе \ref{sec:18.2}.




\subsection{Интерпретация в виде латентных классов}\label{sec:18.5.2} % ALT: Формулировка в виде латентных классов?

\noindent
Модель конечной смеси относится к \textbf{латентному анализу классов} (Айткен и Рубин, 1985; Ведел и др., 1993). Обозначим переменную $d_{i} = (d_{i1}, \ldots, d_{im})$, $d_{ij} = \textbf{1}(\sum_jd_{ij} = 1)$, как индикатор (дамми), что наблюдаемая длительность $t_i$ относится к $j$-ой латентной группе, или классу, для $i = 1, \ldots, N$. То есть, каждое наблюдение может принадлежать одной из $m$ латентных групп, классов, или же ``типов''. Далее будем предполагать, что модель идентифицирована.

Согласно модели, $(t_i|d_i,\bmu, \bpi)$ распределены независимо с плотностями распределения
    \begin{align}
        \label{eq:18.19}
        \sum^{m}_{j=1} d_{ij}f(t_i|\mu_j) = \prod^{m}_{j=1}f(t_i|\mu_j)^{d_{ij}},
    \end{align}
где $\mu_j = \mu(\x_j, \be_j)$, $\bmu = (\mu_1, \ldots, \mu_m)$ и $d_i|\bmu,\bpi$ независимо и одинаково распределены  c полиномиальным распределением
    \begin{align}
        \label{eq:18.20}
        \prod^{m}_{j=1}\pi_j^{d_{ij}}, 0<\pi_j<1, \sum^{m}_{j=1}\pi_j=1.
    \end{align}
Из последних двух выражений следует, что
    $$(t_i|\bmu,\bpi)\thicksim\!\!\!\!\!\!\!^{{}^{\textit{iid}}}\hspace{0.1cm}\sum^{m}_{j=1}\pi_j^{d_{j}}f_j(t|\mu_j)^{d_{ij}}.$$
Следовательно, функция правдоподобия равна
    \begin{align}
        \label{eq:18.21}
        \mL(\be,\bpi|\bt) = \prod^{N}_{i=1}\sum^{m}_{j=1}\pi_j^{d_{ij}}f_j(t;\mu_j)^{d_{ij}}.
    \end{align}


\subsection{EM алгоритм}\label{sec:18.5.3}

\noindent
Данную функцию правдоподобия можно максимизировать непосредственным вычислением, % ALT: напрямую
или же с помощью EM алгоритма, где переменные $\bd = (d_1, \ldots, d_n)$ воспринимаются как пропущенные % ALT: недостающие
данные, см раздел 10.3. % \ref{sec:10.3}
Если бы $\bd$ были наблюдаемы, то логарифм правдоподобия был бы равен
    \begin{align}
        \label{eq:18.22}
        \ln L(\bmu,\bpi|\bt,\bd) = \sum^{N}_{i=1}\sum^{m}_{j=1} d_{ij}\ln f_j (\bt_i; \mu_j) + \sum^{N}_{i=1}\sum^{m}_{j=1} d_{ij} \ln\pi_j.
    \end{align}
Если $\pi_j$, $j = 1,\ldots,m$ известны, то апостериорная вероятность, что $t_i$ принадлежит к группе $j$, $j = 1, 2, \ldots, m$, обозначенная как $\zeta_{ij}$, равна
    \begin{align}
        \label{eq:18.23}
        \zeta \equiv \Pr[y_i\in\textrm{population }j] = \frac{\pi_jf_j(y_i|\x_i,\be_j)}{\sum^{m}_{j=1}\pi_jf_j(y_i|\x_i,\be_i)}.
    \end{align}
Среднее по $i$ значение $\zeta_{ij}$ равно вероятности, что случайно выбранная длительность принадлежит к группе $j$, то есть, $\pi_j$:
    $$\E[\zeta_{ij}] = \pi_j.$$

Предположим, что оценка $\hat{\zeta}_{ij}$ для $\E[d_{ij}]$ известна. % логично: математическое ожидание индикатора равно вероятности, т.е. дзете.
Тогда, при условии, что мы ее знаем, получим
    \begin{align}
        \label{eq:18.24}
        \mathrm{EL}(\be_1, \ldots, \be_m,\bpi|\bt,\hat{\mathbf{z}},\x_1, \ldots, \x_m)) = \sum^{N}_{i=1}\sum^{m}_{j=1} \hat{\zeta}_{ij}\ln f_j(t_i, \mu(\x_j,\be_j)) + \sum^{N}_{i=1}\sum^{m}_{j=1} \hat{\zeta}_{ij}\ln\pi_j,
    \end{align}
что соответствует E-шагу EM алгоритма. На M-шаге мы максимизируем EL, решая условия первого порядка
    \begin{align}
        \label{eq:18.25}
        \hat{\pi}_j - N^{-1}\sum^{m}_{j=1} \hat{\zeta}_{ij} = 0, j = 1, \ldots, m,
    \end{align}
    \begin{align}
        \label{eq:18.26}
        \sum^{N}_{i=1}\sum^{m}_{j=1} \hat{\zeta}_{ij} \frac{\pa\ln f_j(t_i|\be_j)}{\pa\be_j} = \0.
    \end{align}
Далее, мы можем получить новые значения $\hat{\zeta}_{ij}$, используя уравнение \ref{eq:18.23}, и повторять шаги E и M до тех пор, пока процесс не сойдется.
Дисперсию можно рассчитать после, с помощью информационной матрицы или робастной формулы. % не одно и то же: что такое робастная формула, можно посмотреть, например, здесь, в разделе 4.2:
% http://www.actuaries.org/ASTIN/Colloquia/Berlin/Dupin_Montfort_Verle.pdf


\subsection{Выбор количества латентных классов}\label{sec:18.5.4}

\noindent
В связи с количеством компонент возникает два важных вопроса. Первый относится непосредственно к выбору числа компонент, $m$. Зачастую, мы не имеем теории, предсказывающей определенное количество групп, и выбор приходится осуществлять исходя из практических соображений. В частности, количество оцениваемых параметров может быть довольно большим, поскольку их размерность равна $m\dim[\be + m -1]$. Чтобы уменьшить число параметров, можно положить, что некоторые коэффициенты из набора $\be$ равны между собой. В качестве другого способа часто предполагают, что константа принимает различные значения, в то время как коэффициенты наклона между группами одинаковы (как в уравнении \ref{eq:18.18}). Однако, если допустить, что изменяются все параметры, логично выбрать малое число компонент $m$. Часто используют $m = 2$, даже если только константа может варьироваться между группами. Разумной стратегией будет начать с $m = 2$ и затем проверить качество модели с помощью диагностических тестов. Если качество плохое, добавляется еще одна компонента. Добавление компонент, которые невозможно однозначно различить, бессмысленно --- если межгрупповые различия незначительны, нет необходимости представлять выборку в виде конечной смеси. Желательно, чтобы компоненты имели интерпретацию. Выбор между моделями с различным числом компонент можно осуществить на основе штрафных критериев правдоподобия (AIC или BIC), см. раздел 8.5.1. % \ref{sec:8.5.1} % Термин должен совпадать!!!
Отношение правдоподобия \textit{(likelihood ratio)} неприменимо, поскольку параметр находится на границе проверяемой гипотезы.
Бейкер и Мелино (2000) провели эксперимент по методу Монте Карло, который указывает на возможные проблемы перепараметризации, приводящие к неверным выводам, в случае когда и базовый риск, и гетерогенность имеют гибкую форму в целях избежания проблемы неправильной спецификации.
Для выбора модели среди моделей с различным количеством латентных классов авторы рекомендуют использовать штрафные критерии правдоподобия с более высоким штрафом за большее число параметров.

Если модель перепараметризована, параметры невозможно идентифицировать. Проблема возникает из-за наличия нескольких максимумов, или же плоской поверхности функции правдоподобия. Как следствие, алгоритм будет показывать % ALT: выдавать, рассчитывать
различные оценки параметров в зависимости от начальных условий.

Заметим, что минимальное значение штрафного критерия не гарантирует качество подгонки выбранной модели, которое может быть проверено только с помощью соответствующих тестов. По сути, систематическая составляющая объясняет выборочную дисперсию достаточно хорошо, если отклонение между истинными и предсказанными длительностями несущественно.

        \begin{center}{Замечания по поводу вычисления}\end{center}
        \noindent
Второй из вопросов относится к выбору вычислительного алгоритма. Несмотря на то, что EM алгоритм помогает разобраться в структуре решения задачи, на практике он работает довольно медленно. Существует множество примеров, когда алгоритм Ньютона-Рафсона решает ту же задачу быстрее. Обзор работы других алгоритмов можно найти в Хоутон (1997). Заметим, что разницы в применении нет, если межгрупповые различия малы, поскольку поверхность правдоподобия будет содержать несколько локальных максимумов. В любом случае, нет гарантии, что максимум единственный.

Все модели конечной смеси неидентифицированы в том смысле, что распределение данных не изменится, если поменять компоненты местами (например, назвать ``компоненту 1'' ``компонентой 2'', и наоборот). Проблему можно решить, положив, что $\pi_j$ и $\mu_j$ не убывают. Желательно, чтобы компоненты имели содержательную интерпретацию.

Одно из замечаний к модели конечной смеси заключается в том, что, вводя дополнительные компоненты для учета гетерогенности, мы можем попросту найти выбросы. Однако, это необязательно плохо, поскольку такая информация также полезна. Чтобы понять, что компонента включает в себя выбросы, можно воспользоваться формулой \ref{eq:18.23} для расчета апостериорной вероятности. В случае выброса эта вероятность будет велика по отношению к одной из компонент и мала по отношению к остальным.




\section{Выборка типа поток и запас}\label{sec:18.6}

\noindent
Во многих практических ситуациях часто возникает вопрос: В чем заключается взаимосвязь между двумя и более доступными мерами длительностей? Например, различия известны между такими концепциями как средний возраст и ожидаемая продолжительность жизни в демографии, период продажи существующей собственности и ожидаемый период продажи новой в недвижимости. Часто говорят о первом типе, в то время как к делу больше подходит второй. В экономике возникает аналогичный вопрос о различных мерах длительности безработицы, которая публикуется государственными статистическими агентствами. С этими обсуждениями тесно связана ненаблюдаемая гетерогенность, так как она относится и к запасу (совокупности) безработных, и к потоку. Одной из ранних и значимых работ по этой теме является Салант (1977).

Рассмотрим идею на примере с длительностью безработицы.
Одним из показателей, измеряющих продолжительность поисков работы уже безработных индивидов, является \textbf{средняя прекращенная длительность} (\textit{average interrupted duration}, AID). Показатель равен средней длительности поисков индивидов, находящихся в текущем запасе безработных, и его можно воспринимать как переменную запаса.
Это оценка \textbf{ожидаемого пройденного времени} (\textit{expected elapsed duration}), то есть, времени, которое индивид, оставшийся без работы, может ожидать, что потратит на ее поиски.
Часто такую оценку также называют средней длительностью завершенного состояния безработицы (average completed duration, ACD), о чем и идет речь в этой и предыдущих главах. В свою очередь, это является оценкой ожидаемой \textbf{завершенной длительности} (\textit{completed duration}), которую можно воспринимать как переменную потока.
По сути, AID аналогична среднему возрасту, а ACD --- ожидаемой продолжительности жизни. Вопрос же заключается в том, каким образом они взаимосвязаны.

Для решения подобных вопросов используют такой статистический инструмент, как \textbf{теорию восстановления} (\textit{renewal theory}).
Примером \textbf{процесса восстановления} (\textit{renewal process}) является стационарный пуассоновский процесс с постоянной интенсивностью. Количество обновлений в интервале $dt$ является числом событий. Длительность равна промежутку между успешными наступлениями событий (то есть, обновлениями). Для объекта в определенном состоянии пройденное с момента обновления время называется \textbf{обратным временем повторения} (\textit{backward recurrence time}). \textbf{Прямое время повторения} (\textit{forward recurrence time}) обозначает продолжительность периода с текущего момента до момента перехода. Ожидаемое число событий $\E[N(t)]$ в промежутке $(0, t]$ называется \textbf{функцией восстановления} (\textit{renewal function}), а $\lim_{dt\rightarrow0}\mathrm{d}\E[N(t)]/dt$ --- \textbf{интенсивностью восстановления} (\textit{renewal intensity}), которая и определяет зависимость между ACD и средним обратным временем повторения. Далее будут представлены некоторые важные результаты.

Салант (1977) показал, что гетерогенность, присутствующая в коэффициентах риска, может играть ключевую роль в объяснении взаимосвязи между AID и ACD. Представленный им график помогает интуитивно понять два фактора, влияющих на расчет средних. На вертикальной оси рисунка \ref{fig:18.1} отображено календарное время, где горизонтальная ось соответствует дате начала исследования. \textbf{Выборка типа запас} строится на основе запаса объектов, находящихся в данном состоянии. \textbf{Выборка типа поток}, напротив, означает, что мы отбираем только тех, кто попадает в данное состояние в течение периода исследования. Продолжительности состояния объектов изображены вертикальными линиями. На момент начала наблюдения четыре из девяти объектов (S6, S7, S8 и S9) уже находились в исследуемом состоянии. Еще пять (S1, S2, S3, S4 и S5) начались и завершились в течение 12-месячного периода исследования. Пусть $u_j$ обозначает длительность $j$-го продолжающегося наблюдения, а $t_i$ --- длительность $i$-го завершенного наблюдения. Тогда в нашем примере $\mathrm{AID} = 1/4(\sum_ju_j)$, а $\mathrm{ACD} = 1/5(\sum_i t_i)$.

Заметим, что длительных наблюдений в выборке, вероятно, больше, чем коротких. Как следствие, оценка средней длительности будет завышена, и такая выборка будет \textbf{смещенной по длительностям} (\textit{length-biased sampling}). В результате, соотношение между показателями будет выглядеть как AID > ACD. Однако, поскольку выборка также включает неполные длительности, то среднее, вероятно, будет ниже, чем среднее по полностью наблюдаемым объектам. Такой эффект называется \textbf{смещением прекращенных длительностей} (\textit{interruption bias}). Итоговый эффект зависит от распределения длительностей, который, в свою очередь, зависит от распределения коэффициентов риска, где гетерогенность играет ключевую роль.

    \begin{figure}[ht!]\caption{Смещенная по длительностям выборка типа запас: примеры}\label{fig:18.1}
    \centering
%    \includegraphics[scale=0.7]{fig.png}
    \end{figure}

Основной предпосылкой является стационарность, то есть, когда количество переходов в состояние и количество выходов из него равны между собой. Пусть плотность распределения неполных длительностей равна $f(u)$, а плотность полных --- $g(t)$. Тогда распределение $u$ будет равно
    \begin{align}
        \label{eq:18.27}
        f(u) = \frac{\bar{G}(u)}{\int \bar{G}(u)du} = \frac{\bar{G}(u)}{\E[t]},
    \end{align}
где
    $$\bar{G}(u) = \int g(x)dx$$
соответствует функции выживания с плотностью $g(u)$, а $\E(t)$ --- математическое ожидание распределения полных длительностей. Полный вывод формулы со всеми предпосылками можно найти в работах Салант (1977) и Ланкастер (1990, раздел 5.3).

Таким образом, если $g(t)$ экспоненциальная функция плотности, то стохастический процесс наступления событий является процессом Пуассона, $f(u)$ также экспоненциальная, а средние для обеих мер равны.

На основе (\ref{eq:18.27}) можно получить общее соотношение между моментами распределений $u$ и $t$. В частности, математическое ожидание $u$ можно найти, зная математическое ожидание и дисперсию $t$:
    \begin{align}
        \label{eq:18.28}
        \E[u] = \frac{1}{2}\left( \E[t] + \frac{\V[t]}{E[t]} \right).
    \end{align}

Также мы можем определить взаимосвязь между $\E(t)$ и средней завершенной длительностью для \textit{постоянной генеральной совокупности} с продолжающимися (еще не завершенными) наблюдениями (то есть, средним значением по запасу незавершенных наблюдений). Такое соотношение будет равно % опечатка в Кэмероне

    \begin{align}
        \label{eq:18.29}
        \E[t^{S(t)}] = \E[t] + \frac{\V[t]}{E[t]} > \E[t],
    \end{align}
что означает, что средняя длительность по постоянной совокупности, обозначенная $\E[t^{S(t)}]$, превышает среднюю ожидаемую длительность нового объекта. Если $f(t)$ является экспоненциальным, то $\E[t^{S(t)}] = 2 \E[t]$ и $\E[u] = 1/2\E[t^{S(t)}]$. То есть, в среднем, прекращенная длительность будет равна половине завершенной.

Что, если коэффициент риска не постоянен? Если он возрастает с длительностью объекта (то есть, положительная зависимость от состояния), то $\E[u] < \E[t]$, если же он убывает (негативная зависимость от состояния), то $\E[u] > \E[t]$.

Несмотря на то, что мы получили эти результаты при предпосылке о постоянности совокупности, они помогают понять взаимосвязь между распространенными мерами средних длительностей. Причина наступления события при этом не играет роли. Следовательно, форму функции риска нужно моделировать аккуратно, принимая во внимание описанные рассуждения.




\section{Тестирование спецификации}\label{sec:18.7}

\noindent
В моделях времени жизни тестирование спецификации включает в себя:
\begin{itemize}
\item тесты на включение переменных (значимость коэффициентов),\\
\item тесты на функциональную форму функции выживания,\\
\item тесты на ненаблюдаемую гетерогенность,\\
\item совместные тесты на зависимость от состояния % state dependence, предшествующий опыт; то же самое, что и duration dependence
и ненаблюдаемую гетерогенность.
\end{itemize}


Первый тип не подразумевает ничего нового, поскольку такие тесты проводятся с помощью статистики Вальда.

Тесты с ограничением на функциональную форму не отличаются от тестов на ненаблюдаемую гетерогенность, где в качестве ограничения мы предполагаем отсутствие ненаблюдаемой гетерогенности. Поскольку неучет гетерогенности может служить причиной смещения оценок, как было показано в разделе \ref{sec:18.2}, желательно проведение диагностических тестов.

Стандартная задача заключается в том, чтобы проверить, отличается ли составляющая гетерогенности от нуля, или нет. Если для проверки гипотезы мы используем модель с ограничением на отсутствие гетерогенности, то применяется скор-тест. Отношение правдоподобия и тест Вальда, напротив, неприменимы, поскольку основываются на модели без ограничений, где проверяемый параметр находится на границе гипотезы.
Например, смесь Вейбулла--Гамма \ref{eq:18.9} при $1/\de = 0$ равна модели Вейбулла, но такая гипотеза будет граничной. % ALT?
Стандартный хи-квадрат тест с одной степенью свободы при нулевой гипотезе имеет взвешенное хи-квадрат распределение.


\subsection{Тестирование гипотез}\label{sec:18.7.1}

\noindent
Одним из типов тестов на спецификацию является скор-тест на ненаблюдаемую гетерогенность в экспоненциальной модели. Поскольку гетерогенность легко перепутать с зависимостью от длительности, желательно проводить совместный, а не отдельный тест. Сделать это можно в рамках модели Вейбулла с локальной гетерогенностью (Ланкастер, 1985).

\textbf{Локально гетерогенная плотность} получается с помощью разложения в ряд Тейлора около $\nu = 1$ с плотностью Вейбулла с мультипликативной гетерогенностью $\nu$, то есть
    \begin{align}
        S(t|\nu) &= e^{-\mu t^{\al}\nu} = e^{-\e\nu} \notag \\
                 &= e^{-\e}[1 + (-\e)(\nu - 1) + (\e^2/2)(\nu - 1)^2 + O(\e^3)], \notag
    \end{align}
где $\e = \mu t^{\al}$. Из второй строки получим
    $$\E[e^{-\e\nu}] = e^{-\e}[1 + (\e^2\sis/2)] \equiv S_m(t),$$
где $\sis$ равен параметру дисперсии в распределении гетерогенности.

Тогда
    \begin{align}
        f_m(t) &= - \frac{\pa S_m(t)}{\pa t} \notag \\
               &= \al\mu t^{\al-1}e^{-\e}[1 + (\e^2\sis/2)] - e^{-\e}[2\e(\al\mu t^{\al-1})\sis/2] \notag \\
               &= \al\mu t^{\al-1}e^{-\e}[1 + \sis(\e^2-2\e)/2]. \notag
    \end{align}
Используя последний результат и допуская наличие цензурированных наблюдений, логарифм правдоподобия можно записать как
    \begin{align}
        \ln\mL (\al, \be, \sis) &= \sum^{N}_{i=1}\ln\{[f_m(t)]^{\de_i}[S_m(t)]^{1-\de_i}\} \notag \\
                                &= \sum^{N}_{i=1}\de_i[\ln\al + (\al-1)\ln t_i + \ln\mu_i +\ln(1 + \sis(\e^{2}_{i}-2\e_i)/2) - \e_i \notag\\
                                &  + (1-\de_i)\ln(1 + \sis\e^{2}_{i}/2)], \notag
    \end{align}
где $\de_i$ --- индикатор (отсутствия) цензурирования, равный 1 для нецензурированных наблюдений и 0 в обратном случае, $\ln\mu_i = \beta_0 + \xib_1$, и $\e_i = \mu_i t_i^\al$ --- \textbf{обобщенная ошибка} (см. раздел \ref{sec:18.7.2}).

Нулевая гипотеза записывается как $H_0: \sis = 0$ и $\al = 1$, и проверяет одновременное отсутствие ненаблюдаемой гетерогенности и спецификацию экспоненциального распределения. Пусть $\bttt = (\bttt^{'}_1, \bttt^{'}_2)$, $\bttt^{'}_2 = (\beta_0, \be_1)$ и $\bttt^{'}_0 = (0,1,\beta_0,\be_1)$, а $\bttt'_0 = (0, 1, \beta_0, \be_1)$ обозначает вектор ограничения. Для простоты будем рассматривать только нецензурированные наблюдения. Тогда совместная скор-статистика равна
    \begin{align}
        \label{eq:18.30}
        \mathrm{LM}_{\mathrm{HD}} = \frac{1}{d}\bs^{'}\left[ \begin{array}{cc}
        \Psi'(1) & 1 \\
        1        & 1
        \end{array} \right] \bs,
    \end{align}
где $\bs^{'} = [\frac{1}{2}\sum_i(\e^{2}_{i} - 2\e_i)]$, $\sum_i(1+1(1-\e_i)\ln t_i)$ и $\Psi^{'}(r)$ соответствует первой производной дигамма функции $d\ln\Ga(r)/dr$ и $d = 1/(N(\Psi^{'}(1) - 1))$. Для проведения теста статистика $\mathrm{LM}_{\mathrm{HD}}$ оценивается при нулевой гипотезе (то есть, для экспоненциального распределения) и имеет асимптотическое $\chi^2(2)$ распределение (Джаггиа и Триведи, 1994).

Заметим, что матрица квадратичной формы в статистике $\mathrm{LM}_{\mathrm{HD}}$ не является диагональной, поскольку компоненты совместного теста коррелированы друг с другом. Отдельный тест на гетерогенность (зависимость от длительности) может также указывать на зависимость от длительности (гетерогенность). % ENG: has power against duration dependence ? correct?
Чтобы пояснить, рассмотрим два отдельных теста
    \begin{align}
        \label{eq:18.31}
        \mathrm{LM}_{\mathrm{H}} = \frac{1}{4N}(\sum_i(\e^{2}_{i} - 2\e_i))^2,
    \end{align}
    \begin{align}
        \label{eq:18.32}
        \mathrm{LM}_{\mathrm{D}} = \frac{1}{d}\sum_i(1 + (1-\e_i)\ln t_i))^2,
    \end{align}
каждый из которых имеет распределение $\chi^2(1)$ при нулевой гипотезе. Поскольку тесты коррелированы, см. (\ref{eq:18.30}.), каждый из них будет иметь мощность против другой нулевой гипотезы. Как следствие, выводы о характере неправильной пефицикации на основе отдельных тестов могут быть ошибочны.

Поскольку спецификация ненаблюдаемой гетерогенности тесно связана с зависимостью от состояния, тестирование гипотез по отдельности может приводить к неправильным результатам (Джаггиа и Триведи, 1994). Формально, тесты на зависимость от состояния без учета ненаблюдаемой гетерогенности, и наоборот, смещены. Джаггиа (1991c) проводит повторный анализ данных по длительностям забастовок, которые ранее были ошибочно представлены в экономической литературе. Джаггиа и Триведи (1994) предлагают несколько совместных тестов для класса параметрических моделей. Бера и Юун (1993) рассматривают более общие вопросы тестирования гипотез при неправильной спецификации моделей.

Эти тесты полезны в простых параметрических моделях, так что анализ можно начать с моделей Вейбулла, Вейбулла--Гамма или пропорциональных рисков. Тестирование ненаблюдаемой гетерогенности или любой другой ошибки спецификации возможно с помощью функции интегрального риска, поскольку интегральный риск без гетерогенности является нормированно экспоненциально распределенной случайной величиной. Далее мы представим обсуждение методов для оценки качества модели, основанных на интегральном риске.




\subsection{Графические способы выявления неправильной спецификации}\label{sec:18.7.2} % ALT: обнаружения

\noindent
В разделе 8.7.2 % \ref{sec:8.7.2}
мы обозначили понятие обобщенных остатков. В нелинейный моделях выбор такой меры обычно неочевиден, однако в данном контексте можно найти подходящий вариант.

        \begin{center}{Обобщенные остатки}\end{center}
        \noindent
Графические непараметрические тесты представляют собой практичный инструмент для оценки качества подгонки модели. Тесты используют обобщенный остаток как некоторую функцию от данных и оценок параметров. Для верно специфицированной модели остатки должны быть приблизительно независимо и одинаково распределены. Такому критерию удовлетворяет интегральный риск, и, следовательно, может быть использован для проведения теста на основе остатков. В контексте моделей времени жизни из раздела 17.3.1
    \begin{align}
        S(t|\mu) &= \exp[\La(t|\mu)], \notag \\
        f(t|\mu) &= \la(t|\mu)\exp[\La(t|\mu)], \notag
    \end{align}
\textbf{обобщенный остаток} равен
    \begin{align}
        \label{eq:18.33}
        \eps &= \La(t|\mu) \\
             &= -\ln (S(t|\mu)). \notag
    \end{align}
Якобиан преобразования может быть записан как
    \begin{align}
        |J| &= dt/d\eps \notag \\
            &= \frac{1}{d\La(t|\mu) / dt} \notag \\
            &= 1/\la(t|\mu). \notag
    \end{align}
Зная $f(t|\mu)$, преобразование (\ref{eq:18.33}) и Якобиан преобразования, получим плотность распределения $\eps$
    \begin{align}
        \label{eq:18.34}
        \la(t|\mu) \exp(-\eps)\frac{1}{\la(t|\mu)} = \exp(-\eps),
    \end{align}
которая не зависит от $\mu$ и имеет нормированное экспоненциальное распределение. Данный результат был использован в разделах
17.3.1 и % \ref{sec:17.3.1} # UNCOMMENT 17
17.6.7. % \ref{sec:17.6.7} # UNCOMMENT 17


        \begin{center}{Диагностические тесты на основе интегрального риска}\end{center}
        \noindent
Используя свойство нормированного экспоненциального распределения обобщенного остатка, можно построить диагностический тест при нулевой гипотезе, что спецификация верна. Функция выживания для обобщенного остатка равна $S(\eps) = \exp(-\eps)$, откуда следует, что $-\ln S(\eps) = \La(\eps) = \eps$. Для верно специфицированной модели зависимость между оценкой интегрального риска и обобщенным остатком $\hat{\eps}$ изображается прямой линией под углом $45^{\circ}$. Значительное отклонение от этой линии означает, что модель специфицирована неверно.

Например, \textbf{оценка интегрального риска} в модели Вейбулла равна $\hat{\eps} = \hat{\mu}t^{\hat{\al}}$, а функция выживания $\hat{S}(\hat{\eps}) = N^{-1}$ (количество наблюдений в выборке $\ge \hat{\eps}$). Формально, можно построить регрессию $-\ln S(\hat{\eps})$ на $\hat{\eps}$ и проверить совместную гипотезу, что константа равна 0, а коэффициент наклона --- 1.

Такой метод применим к любой параметрической модели, для которой можно выразить интегральный риск. Например, для смеси Вейбулла--Гамма (и смеси экспоненциального и гамма распределений при $\al = 1$) \textit{обобщенная ошибка} равна $\eps = k\ln[(k + \mu t^{\al})/k]$. Чтобы применить тест, нужно сперва найти $\hat{\eps}$ при данных $(\mu,\al,k)$, а затем построить графическую зависимость $\hat{\eps}$ от $-\ln \hat{S}(\hat{\eps})$.


        \begin{center}{Цензурированные данные}\end{center}
        \noindent
Если наблюдения цензурированы, длительность равна $t = \min[T,L]$, где $L$ обозначает предел цензурирования справа. Если длительность превышает $L$, то она оказывается цензурирована в момент $L$. В таком случае обобщенная ошибка $\eps(t)$ больше не подчиняется нормированному экспоненциальному закону распределения. Соответствующая поправка для цензурированных данных выглядит следующим образом
    \begin{align}
        \label{eq:18.35}
        \E[\eps(T)|T\ge L] &= \int^{\infty}_{\eps(L)}\frac{\eps f(\eps)}{S(\eps (L))} d\eps \notag \\
                           &= \frac{1}{e^{\eps(L)}}\left[ \int^{\infty}_{\eps(L)} \eps e^{-\eps} d\eps \right] \notag \\
                           &= \frac{1}{e^{\eps(L)}}[1 + \eps(L)e^{-\eps(L)} + e^{-\eps(L)} - 1] \notag \\
                           &= 1 + \eps(L),
    \end{align}
где требуется интегрирование по частям и упрощение.

Таким образом, обобщенная ошибка может быть оценена как $\tilde{\eps}(t) = \hat{\eps}(t)$ для нецензурированных наблюдений и как $\tilde{\eps}(t) = 1 + \hat{\eps}(L)$ для цензурированных. Исследования показывают, что такой метод работает достаточно хорошо в экспоненциальной модели с цензурированными данными, где доля таких наблюдений относительно невелика (Джаггиа и Триведи, 1994; Джаггиа, 1997).

\subsection{Тесты на условный момент}\label{sec:18.7.3} % !!!Термин должен совпадать с термином из главы 8!!! ок
\noindent

Для тестирования спецификации к обобщенным остаткам можно применить идею \textbf{условного момента} (см. раздел 8.2), % \ref{sec:8.2} # UNCOMMENT IN THE END
представив ее в контексте тестирования ненаблюдаемой гетерогенности.

Ранее было показано, что интегральная функция риска имеет нормированное экспоненциальное распределение с математическое ожиданием и дисперсией, равными 1. Тогда соответствующее ограничение на условный момент второго порядка равно $\E[(\eps - 1)]^2 = \V[\eps] = 1$, что эквивалентно
    $$\E[\eps^2 - 2] = 0.$$
Ограничения более высокого порядка можно также построить и проверить, как совместно, так и по отдельности. Детали можно найти в работе Джаггиа (1991a).




\section{Пример ненаблюдаемой гетерогенности: длительность безработицы}\label{sec:18.8}

\noindent
В этом разделе мы вернемся к примеру из раздела 17.11, % \ref{sec:17.11} # UNCOMMENT AFTER 17 CHAPTER
предположив, что в модели присутствует ненаблюдаемая гетерогенность, выраженная в аналитической форме.

\begin{figure}[ht!]\caption{Длительность безработицы: обобщенные остатки в экспоненциальной модели. Данные США по 3343 наблюдениям в 1986--92 гг., некоторые наблюдения неполные.}\label{fig:18.2}
\centering
%\includegraphics[scale=0.7]{fig.png}
\end{figure}

Как уже обсуждалось в разделе \ref{sec:18.7.2}, возможное наличие ненаблюдаемой гетерогенности можно проверить графически, построив предсказанные значения модели. Для верно специфицированной модели остатки должны следовать нормированному экспоненциальному распределению. Качество модели можно также оценить неформально, рассчитав эмпирическую функцию кумулятивного риска и сравнив ее с обобщенными остатками. Для верно специфицированной модели должна получиться почти прямая линия с наклоном, равным единице.

На рисунках \ref{fig:18.2} и \ref{fig:18.3} изображены обобщенные остатки в экспоненциальной модели с ненаблюдаемой гетерогенностью и без, соответственно. По графикам видно, что качество модели с учетом ненаблюдаемой гетерогенности незначительно выше.
    \begin{figure}[ht!]\caption{Длительность безработицы: обобщенные остатки в модели смеси экспоненциального и гамма распределений. Данные те же, что и на рисунке \ref{fig:18.2}.}\label{fig:18.3}
    \centering
    %\includegraphics[scale=0.7]{fig.png}
    \end{figure}

    \begin{table}[!htbp]\caption{\textit{Длительность безработицы: экспоненциальная модель с гамма и обратной гауссовской гетерогенностью}}\label{tab:18.1}
    \begin{center}
\begin{tabular}{lcccc}
\hline \hline
&\multicolumn{2}{c}{\textbf{Экспоненциальное--Гамма}}&\multicolumn{2}{c}{\textbf{Экспоненциальное--IG}}\\
\cmidrule(r){2-3}\cmidrule(r){4-5}
\textbf{Переменная} &\textbf{Коэффициент}   &$t$      &\textbf{Коэффициент}   &$t$ \\
\hline
RR                  &0.501  &0.817  &0.504  &0.821 \\
DR                  &-0.882 &-1.118 &-0.807 &1.032 \\
UI                  &-1.585 &-6.043 &-1.545 &-5.994 \\
RRUI                &1.091  &1.725  &1.057  &1.686 \\
DRUI                &0.057  &0.055  &-0.013 &-0.012 \\
LNWAGE              &0.379  &3.184  &0.373  &3.156 \\
CONS                &-4.095 &-4.507 &-4.097 &-4.545 \\
$\sis$              &0.232  &3.178  &0.207  &2.925 \\
$-\ln\textrm{L}$    &\multicolumn{2}{c}{2695.35}&\multicolumn{2}{c}{2696.48} \\
\hline \hline
\end{tabular}
    \end{center}
    \end{table}

Графические результаты могут быть подтверждены расчетами, представленными в таблице \ref{tab:18.1}, где также указаны оценки экспоненциальной модели с обратной гауссовской гетерогенностью (IG, \textit{inverse-Gaussian}). Несмотря на то, что мы знаем, что в данных присутствует ненаблюдаемая гетерогенность, оценки коэффициентов лишь незначительно отличаются от тех, что были получены в предыдущей главе. Значит, она должна оказывать существенный эффект на параметр зависимости от длительности, так как он не учитывается в экспоненциальной модели.

Следовательно, более интересный случай для нас представляет модель, учитывающая как ненаблюдаемую гетерогенность, так и зависимость от длительности, например, модель Вейбулла со смесью IG. Для удобства сопоставления, мы представим оценки в таблице \ref{tab:18.2} рядом с оценками без учета гетерогенности.

Согласно представленным расчетам, ненаблюдаемая гетерогенность оказывает существенный эффект на параметр зависимости от длительности, который возрастает с $1.129$ в таблице 17.8 % \ref{tab:17.8} # UNCOMMENT AFTER 17 CHAPTER
до $1.753$ в таблице \ref{tab:18.2}, что соответствует более крутому наклону коэффициента риска в модели с гетерогенностью. Вспомним из раздела \ref{sec:18.2.4}, что одно из последствий неучета ненаблюдаемой гетерогенности в модели пропорциональных рисков заключается в недооценке коэффициента риска; следовательно, данный результат согласуется с теорией. О наличии гетерогенности свидетельствует также то, что $t$-статистика для оценки параметра дисперсии $\sis$ превышает $11$. Наконец, качество модели, измеренное с помощью логарифма правдоподобия, также повысилось с $-2687.6$ до $-2616.6$. Хотя изменения в оценках коэффициентов и незначительны, эффекты значимых коэффициентов (UI, LNWAGE и CONS) стали более выражены.

    \begin{table}[!htbp]\caption{\textit{Длительность безработицы: модель Вейбулла с обратной гауссовской гетерогенностью и без.}}\label{tab:18.2}
    \begin{center}
\begin{tabular}{lcccc}
\hline \hline
&\multicolumn{2}{c}{\textbf{Weibull-IG}}&\multicolumn{2}{c}{\textbf{Weibull}}\\
\cmidrule(r){2-3}\cmidrule(r){4-5}
\textbf{Переменная} &\textbf{Коэффициент}   &$t$      &\textbf{Коэффициент}   &$t$ \\
\hline
RR                  &0.736  &0.812  &0.448  &0.70 \\
DR                  &-1.073 &-0.933 &-0.427 &-0.53 \\
UI                  &-2.575 &-6.698 &-1.496 &-5.67 \\
RRUI                &1.734  &1.857  &1.105  &1.57 \\
DRUI                &-0.061 &-0.039 &-0.299 &-0.28 \\
LNWAGE              &0.576  &3.259  &0.37   &2.99 \\
CONS                &-5.303 &-3.953 &-4.358 &-4.74 \\
$\al$               &1.753  &44.19  &1.129  &51.44 \\
$\sis$              &6.377  &11.149 &-      &- \\
$-\ln\textrm{L}$    &\multicolumn{2}{c}{2616.6}&\multicolumn{2}{c}{2687.6} \\
\hline \hline
\end{tabular}
    \end{center}
    \end{table}

Тем не менее, несмотря на повышение качества модели, полученная модель смеси по прежнему может быть мисспецифицирована. Графический инструментарий снова может быть использован в качестве неформального теста на спецификацию. Изображенные на рисунках \ref{fig:18.4} и \ref{fig:18.5} обобщенные остатки из моделей Вейбулла с ненаблюдаемой гетерогенностью и без указывают на неверную спецификацию модели смеси.

Заметим, что для ``улучшенной'' модели, учитывающей как ненаблюдаемую гетерогенность, так и зависимость от длительности, проще определить определить ошибочность спецификации по сравнению с более простой моделью, не учитывающей ни один из факторов; похожий результат представлен в работе Джаггиа (1991c). Объяснить такой результат можно с помощью взаимосвязи между гетерогенностью и зависимостью от длительности. В то время как модель Вейбулла предполагает монотонные риски, МакКолл (1996) показывает, что для этих же данных лучше подходит U-образная форма функции риска. В частности, он использует более гибкую полиномиальную спецификацию функции риска. Следовательно, основной вывод здесь заключается в том, что для модели, которая допускает наличие обоих факторов, обнаружить мисспецификацию проще.

\begin{figure}[ht!]\caption{Длительность безработицы: обобщенные остатки в модели Вейбулла. Данные те же, что и на рисунке \ref{fig:18.2}.}\label{fig:18.4}
\centering
%\includegraphics[scale=0.7]{fig.png}
\end{figure}

\begin{figure}[ht!]\caption{Длительность безработицы: обобщенные остатки в модели смеси Вейбулла и обратного гауссовского распределения. Данные те же, что и на рисунке \ref{fig:18.2}.}\label{fig:18.5}
\centering
%\includegraphics[scale=0.7]{fig.png}
\end{figure}

Наконец, мы проведем параметрический тест на наличие ненаблюдаемой гетерогенности, целью которого является показать, как можно применить теорию, описанную в разделе \ref{sec:18.7}. Однако скор-тест на ненаблюдаемую гетерогенность из раздела \ref{sec:18.7.1} предназначен для работы с нецензурированными наблюдениями. Поскольку используемые данные содержат наблюдения, цензурированные справа, мы воспользуемся скор-тестом, представленном в работе Джаггиа (1997) для работы с цензурированной выборкой.

Мы хотим проверить отсутствие ненаблюдаемой гетерогенности в экспоненциальной модели времени жизни, $H_0:\sis=0$. Обозначим параметры как $\bttt = (\sis, \be)$, а информационную и скор-матрицы как $\cI(\bttt_0)$ и $\bs(\bttt_0)$, рассчитанные при нулевой гипотезе. Используя логарифм правдоподобия, полученный в разделе \ref{sec:18.7.1}, мы можем записать $\bs(\bttt_0) = (\bs_1(\bttt_0),\bs_2(\bttt_0))$, где $\bs_1(\bttt_0) = \frac{\pa\cL}{\pa\sis}\big|_{H_0} = \frac{1}{2}\sum(\eps^{2}_{i} - 2C_i\eps_i)$ и $\cI(\bttt_0) = -\E\left[\frac{\pa^2\cL}{\pa\ttt\pa\ttt^{'}}\right]\big|_{H_0}$. Тогда тестовая статистика будет выглядеть следующим образом
    \begin{align}
        \label{eq:18.36}
            \mathrm{LM} = \bs^{'}_{1}(\tilde{\bttt}_0)\cI^{11}(\tilde{\bttt}_0)\bs_1(\tilde{\bttt}_0)\sim\chi^2(1),
    \end{align}
где $\cI^{11} = [\cI_{11} - \cI_{12}(\cI_{22})^{-1}\cI_{21}]^{-1}$ является первым диагональным элементом блочной матрицы, обратной от $\cI(\bttt)$, предложенной Джаггиа (1997), а волны над оценками соответствуют методу максимального правдоподобия с ограничениями.

Для нашей выборки статистика равна $\mathrm{LM} = 44.25$, что значительно превышает критическое значение $\chi^{2}(1)$. Следовательно, мы отвергаем нулевую гипотезу, что $\sis = 0$. Результат согласуется с моделями смеси Вейбулла--Гамма и Вейбулла--IG, где существенное повышение качества модели является результатом учета ненаблюдаемой гетерогенности. Как было сказано ранее, данный тест может также указывать на неправильную спецификацию зависимости от длительности.




\section{Практические соображения}\label{sec:18.9}

\noindent
Вопрос о взаимосвязи функции риска и ненаблюдаемой гетерогенности послужил поводом для многочисленных исследований. Одна достаточно изученная точка зрения заключается в том, что спецификация распределения гетерогенности не играет существенной роли, если функция риска специфицирована правильно (Мантон и др., 1986). Другими словами, вместо параметрического моделирования ненаблюдаемой гетерогенности можно попросту использовать робастные оценки дисперсии, при условии, что спецификация функции риска верна. Другие авторы утверждают, что выбор формы гетерогенности, напротив, имеет значение (Хекман и Cингер, 1984a), и поэтому следует применять непараметрическую спецификацию. Некоторые известные работы также используют дискретную модель с гибкой спецификацией риска в сочетании с параметрическими предпосылками о распределении гетерогенности (Мейер, 1990; Хан и Хаусман, 1990). Наконец, в качестве компромисса некоторые исследователи предлагают объединить предыдущие подходы, совместив дискретную модель Хан-Хаусман, или же функцию риска в виде полинома высшего порядка, и непараметрическую гетерогенность Хекман-Cингер. Однако, Бейкер и Мелино (2000) показали, что такая общая модель может привести к другой проблеме --- перепараметризации. В таком случае нужно действовать аккуратно и отдавать предпочтение моделям с относительно небольшим числом параметров.

Модель PH Кокса является центральным объектом анализа в биометрической литературе. Если для исследования не требуется определенная форма базового риска, то такая модель является довольно привлекательной в отношении функциональной формы риска. Обычно, удобнее начинать именно с нее. При этом, поскольку ненаблюдаемая гетерогенность имеет место во многих эконометрических приложениях, нельзя игнорировать ее роль.

Многие статистические пакеты умеют работать с набором стандартных параметрических моделей, которые могут быть комбинированы с любой из стандартных спецификаций гетерогенности (гамма, обратной гауссовской или лог-нормальной). Но несмотря на удобство таких моделей в применении, дискретные модели риска обычно являются более гибкими и лучше соответствуют данным.

EM алгоритм, применяемый для оценки моделей латентных классов, работает довольно медленно, и максимизация правдоподобия напрямую зачастую оказывается более эффективна.




\section{Библиографические заметки}\label{sec:18.10}

\noindent

\begin{itemize}
    \item[\textbf{18.2}]
Существует довольно много работ, в которых обсуждается спецификация распределения гетерогенности и последствия неправильной спецификации. Вопель и др. (1979) предлагает качественное описание свойств модели гамма. Хугаард (1984) рассматривает различные альтернативы этой модели. Работа Хугаард (1995) представляет собой обзор моделей гетерогенности. Хекман и Cингер (1984a) указывают на преимущества непараметрической спецификации и чувствительность результатов к неправильной спецификации. В работе Мантон и др. (1986) представлена попытка сравнить последствия неправильной спецификации риска и гетерогенности.

    \item[\textbf{18.3}]
Ван ден Берг (2001) предлагает детальное и понятное описание идентификации модели смешанных пропорциональных рисков MPH и ссылки на работы по этой теме.

    \item[\textbf{18.4}]
Хан и Хаусман (1990) и Мейер (1990) представляют собой качественные эмпирические исследования, в которых гибкая спецификация риска сочетается с параметрическими предпосылками о распределении гетерогенности.

    \item[\textbf{18.5}]
Хекман и Cингер (1984a) является одной из первых работ, где предлагается обсуждение дискретной модели гетерогенности. Модель конечной смеси также называется ``непараметрической моделью гетерогенности''. Бейкер и Мелино (2000) проводят эксперимент по методу Монте Карло для оценивания модели с более гибкой спецификацией базового риска с непараметрической гетерогенностью. Результат заключается в том, что наличие множества компонент конечной смеси приводит к существенным смещениям и недостоверным выводам. В этом случае может быть полезным применение критериев BIC и Ханнана-Куинна, штрафующих за перепараметризацию.

    \item[\textbf{18.6}]
Ланкастер (1990) и Салант (1977) являются
Ланкастер
Тэйлор и Карлин (1994).
    \item[\textbf{18.7}]
Кифер (1988)
Джаггиа (1991a)
Грин (2003)
Эндрюс (1997)
Кэмерон и Триведи (1998, 6)
Хосмер и Лемешов (1999, 196-240).
    \item[\textbf{18.8}]
Ланкастер (1979)
Джаггиа (1991c)
19.
\end{itemize}




\section{Упражнения}\label{sec:18.ex}

\noindent

\begin{itemize}
    \item[\textbf{18--1}]
(Сапра, 2002)
В разделе \ref{sec:18.2} представлены последствия неучета гетерогенности для безусловной, или средней, функции риска. Согласно приведенному анализу, неучет гетерогенности приводит к недооценке коэффициента наклона средней функции риска. Пусть условная функция риска равна $\la_C(t|\nu) = \la_0(t)$, где $\la_0$ обозначает базовый, или безусловный, риск. Покажите, что (i) безусловный риск $\la_U(t) < \la_0(t)$ и (ii) $\pa\la_U(t)/\pa t < 0$ для каждого из следующих пунктов.
        \item[\textbf{(a)}]
$\nu \sim \mathrm{Uniform}[0,1]$ и $\la_0(t) = 1$ $\forall$ $t$.
        \item[\textbf{(b)}]
$\nu$ подчиняется нормированному экспоненциальному закону распределения с функцией плотности $g(\nu) = e^{-\nu}$ и $\la_0(t) = \rho \exp(\ga t)$, $\rho > 0$, $\ga < 0$.

    \item[\textbf{18--2}]
Рассмотрим модель Вейбулла--Гамма из раздела \ref{sec:18.2.3}, где вместо гамма-распределенной гетерогенности предполагается лог-нормальное распределение с математическое ожиданием, равным единице.
        \item[\textbf{(a)}]
Проверьте, что выражение для безусловной функции риска невозможно представить в аналитическом виде.
        \item[\textbf{(b)}]
Подставьте безусловный риск в виде интеграла в функцию логарифма правдоподобия, представленную в разделе 17.6.3. % \ref{sec:17.6.3} # UNCOMMENT AFTER 17 CHAPTER
Используя основанный на симуляции метод максимального правдоподобия (см. раздел 12.4), % \ref{sec:12.4} # UNCOMMENT AFTER 12 CHAPTER
опишите последовательность действий, с помощью которых находится максимум правдоподобия.

    \item[\textbf{18--3}]
Рассмотрим смесь экспоненциального и гамма распределений, которая является частным случаем модели смешанных пропорциональных рисков, MPH. Для экспоненциальной модели условная функция выживания относительно мультипликативного $\nu$ равна $S(t|\nu) = \exp(-\mu t \nu)$, $\la > 0$. Безусловная функция выживания дана функция выживания, усредненная по плотности распределения неоднородной совокупности $g(\nu)$, то есть, $S(t) = \int^{\infty}_{0} S(t|\nu)g(\nu)d\nu$. Пусть $\nu$ имеет (двухпараметрическое) гамма распределение с функцией плотности $g(\nu) = \de^k\nu^{k-1}\exp(-\de\nu)/\Ga(k)$.
        \item[\textbf{(a)}]
Покажите, что для гамма-распределенной гетерогенности $S(t) = (1 + \mu t /\de)^{-k}$.
        \item[\textbf{(b)}]
Выведите выражения для безусловной плотности распределении длительности $f(t)$ и безусловной функции риска $\la(t)$. Эти выражения можно упростить, положив, что $\E[\nu] = 1$, или же $k = \de$. В результате должна получиться смесь экспоненциального и гамма распределений. Сравните математическое ожидание и дисперсию для экспоненциального распределения и распределения смеси.

        \item[\textbf{(c)}]
Пусть случайная величина $\nu$ может принимать только два значения: $\nu_1$ с вероятностью $\pi$ и $\nu_2$ с вероятностью $(1 - \pi)$. Как это повлияет на безусловную функцию риска? Объясните ответ.

    \item[\textbf{18--4}]
Используя выборку по данным МакКолла из упражнения 17--3, оцените заново модель Вейбулла для тех индивидов, кто переходит на полную ставку (CENSOR1 = 1), при условии, что ненаблюдаемая гетерогенность (в некоторых пакетах используется термин уязвимость, \textit{frailty}) имеет гамма распределение.
        \item[\textbf{(a)}]
Используя обобщенные остатки, представленные в разделе \ref{sec:18.7.2}, проверьте гипотезу о неверной спецификации модели.
        \item[\textbf{(b)}]
Описывает ли новая модель зависимость от длительности соответствующим образом? Подходит ли она лучше для описания данных? % качество подгонки?
Объясните ответ через взаимосвязь ненаблюдаемой гетерогенности и зависимости от длительности.
        \item[\textbf{(c)}]
Повторите пункт (a) при условии, что гетерогенность распределена лог-нормально. Значительны ли различия в результатах?
\end{itemize}


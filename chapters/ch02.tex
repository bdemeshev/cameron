
\chapter{Причинно-следственные и статистические модели}

\section{Введение}

	Микроэконометрика посвящена теориям и методам анализа микроданных, относящимся к отдельным лицам, домохозяйствам и компаниям. Более широкое определение может также включать данные на региональном или государственном уровне. Микроданные, как правило, представляют собой либо пространственную выборку, тогда они содержат характеристики объектов на  конкретную дату, либо панельные данные, тогда они содержат характеристики объектов за несколько периодов времени. Данные могут быть неэкспериментальными, например, данные из опросов и переписей или экспериментальными и квази-экспериментальными, например, данные социальных  экспериментов, проводимых властями с участием добровольцев. 
	
	
	Микроэконометрическая модель может полностью  специфицировать закон распределения для всех наблюдений; она также может частично специфицировать некоторые свойства распределения, например моменты, для части переменных. Математическое ожидание зависимой переменной при фиксированных регрессорах представляет особый интерес. 
	
	
	Целями микроэконометрики является как описание данных, так и нахождение причинно-следственных связей. 
	Первая цель может быть вольно определена как описание свойств моментов зависимой переменной и нахождение уравнений регрессий, описывающих статистическую, а не причинно-следственную связь. Вторая цель подразумевает исследование причинно-следственных связей и состоит в измерении силы, а также эмпирическом подтверждении или опровержении гипотез о микроэкономическом поведении. Тип и стиль эмпирических исследований, поэтому многообразен. 
	
	На одном конце спектра находятся высоко структурированные модели, полученные из подробной спецификации экономического поведения, с помощью которых анализируют причинные (поведенческие) или структурные отношения для взаимозависимых микроэкономических переменных. На другом конце находятся исследования приведенной формы, которые направлены на то, чтобы выявить корреляцию между переменными, не обязательно полагаясь на детальную спецификацию всех соответствующих взаимозависимостей. Оба подхода имеют общую цель в раскрытии важных и интересных отношений, которые могут быть полезны для понимания микроэкономического поведения, но они отличаются по степени, в которой они опираются на экономическую теорию. 
	

	Как отдельная дисциплина микроэконометрика моложе, чем макроэконометрика, которая занимается моделированием рыночных и агрегированных данных. Многие из ранних работ по прикладной эконометрике были основаны на агрегированных временных рядах, собранных государственными учреждениями. В большей части ранних работ по статистическому анализу спроса вплоть до приблизительно 1940 года использовались рыночные данные, а не индивидуальные или данные домашних хозяйств (Хендри и Морган, 1996). Книга Моргана (1990) по истории эконометрики не содержит ссылок на микроэконометрические работы до 1940 года, за одним важным исключением.  Это исключение --- работа по бюджетам домохозяйств, созданная при исследовании уровня жизни  менее обеспеченных слоев населения в разных странах. В результате были собраны данные по бюджетам домохозяйств,  которые явились основой для ряда ранних микроэконометрических исследований, например, работы Аллена и Боули (1935). Однако лишь в 1950 году микроэконометрика стала признанной и отдельной дисциплиной. Даже в 1960-х годах основная часть макроэконометрических работа была связана с анализом спроса на базе опросов домохозяйств. 
	
		
	После вручения Нобелевской премии по экономике в 2000 году Джеймсу Хекману и Даниэлу МакФаддену за их вклад в микроэконометрику, данная область добилась четкого признания в качестве отдельного раздела науки. Хекману вручили премию <<за развитие теории и методов анализа данных с самоотбором выборки>>, а МакФаддену <<за развитие теории и методов анализа моделей дискретного выбора>>. Примеры тем, с которыми сталкивается микроэконометрика, были упомянуты в цитате: <<\ldots какие факторы определяет, будет ли индивид работать, и если да, то сколько часов? Как экономические факторы влияют на решение индивида об образовании, роде деятельности и месте жительства? Как влияют различные образовательные программы и программы по трудоустройству на занятость и доход индивидов?>>.
	
		
	Применение микроэконометрических методов можно найти не только в  микроэкономике, но и в других родственных социальных науках, таких как политология, социология, и география. 
	
	
	Начиная с 1970-х и особенно в течение последних двух десятилетий произошли революционные изменения в анализе больших объемов данных и соответствующих вычислительных методах. Вместе с этим стало доступно огромное количество микроэкономических наборов данных, что позволило значительно расширить сферу микроэконометрики. 
	В результате, хотя эмпирический анализ спроса продолжает оставаться одной из наиболее важных областей применения микроэконометрических методов, его стиль и содержание были существенно изменены под влиянием новых методов и моделей. Сейчас микроэконометрические методы широко применяются в таких областях, как анализ экономического развития,  финансы, здравоохранение, теория отраслевых рынков, экономика труда, экономика государственного сектора, и эти приложения будут продемонстрированы в разных частях данной книги.
	
	
	Основное внимание в этой книге уделяется новым достижениям, которые появились в последние три десятилетия. Нашей целью является обзор концепций, моделей и методов, которые мы рассматриваем в качестве стандартных  инструментов современного исследователя. Конечно, понятие стандартных методов и моделей является субъективным, и зависит от предполагаемых читателей и уровня авторов книги. Также есть темы, которые мы считаем слишком сложными для такой вводной книги, как эта, но которые были бы по-другому оценены другими авторами.
	
		
	Микроэконометрика уделяет особое внимание нелинейным моделям и получению оценок, которые могут иметь структурную интерпретацию. Большая часть этой книги, особенно Части 2---4, рассматривает методы для нелинейных моделей. Эти нелинейные методы пересекаются со многими областями прикладной статистики, включая биостатистику. Отличительной особенностью эконометрики является именно моделирование причинно-следственных связей. В этой главе представлены основные понятия, связанные с причинно-следственным и описательным моделированием, концепции, которые относятся как к линейным, так и к нелинейным моделям. 
	
	
	В Разделах 2.2 и 2.3 вводятся ключевые понятия структурности и экзогенности. В Разделе 2.4 представлены линейные системы одновременных уравнений для иллюстрации структурной модели и указана их связь с моделями в приведенной форме. Идентифицируемость определяется в разделе 2.5.  Раздел 2.6 рассматривает структурные модели из одного уравнения. В Разделе 2.7 вводятся модели потенциального исхода и их причинно-следственная интерпретация сравнивается с моделями одновременных уравнений. Раздел 2.8 содержит краткое обсуждение стратегий моделирования и оценивания при наличии вычислительных трудностей.
	
	
\section{Структурные модели}

В описание структурной модели входят:
\begin{enumerate}
\item набор переменных $W$ разделённых для удобства на $[Y \, Z]$
\item закон распределения вероятностей для $W$, $F(W)$
\item априорное упорядочивание $W$ в соответствии с гипотетическими причинно-следственными связями и спецификация априорных ограничений на гипотетическую модель
\item параметрическая, полупараметрическая или непараметрическая спецификация функциональной формы и ограничений на параметры модели.
\end{enumerate}

Данное определение структурной модели соответствует определению, данному комиссией Коулса. Например, Сарган (1988, стр. 27) утверждает: \\


Модель --- это спецификация закона распределения  вероятностей для набора наблюдений. Структура это спецификация параметров данного распределения. Следовательно, структура --- это модель в которой параметрам присвоены  числовые значения. 


Рассмотрим случай, когда цель моделирования заключается в объяснении наблюдаемых значений векторной переменной $y$, $y'= (y_{1}, \ldots , y_{G})$. Каждый элемент $y$ является функцией некоторых других элементов $y$, объясняющих переменных $z$ и случайных ошибок $u$. Обратите внимание, что переменные $y$ считаются взаимозависимыми. Заметим, что  взаимозависимость между $z_{i}$ не моделируется. Наблюдение номер  $i$ удовлетворяет набору неявных уравнений:
	
\begin{equation}
g(y_{i},z_{i},u_{i}|\theta)=0,
\end{equation}

где $g$ --- известная функция. Мы называем такой тип уравнений структурной моделью, а $\theta$  --- структурными параметрами. Это соответствует свойству 4, которое приведено в начале раздела. 


Предположим, что существует единственное решение $y_{i}$ для каждой пары $(z_{i},u_{i})$. Тогда мы можем записать уравнения в явном виде, где  $y$ --- это функция от $(z, u)$:

\begin{equation}
y_{i}=f(z_{i},u_{i}|\pi).
\end{equation}


Это уравнение называется приведенной формой структурной модели, где $\pi$ --- вектор параметров приведенной формы, который является функцией от $\theta$. Приведенная форма получается путем решения уравнения структурной модели относительно эндогенных переменных $y_{i}$ при фиксированных $(z_{i},u_{i})$. 


Если целью моделирования является получение выводов об элементах $\theta$, то (2.1) обеспечивает прямой путь, который подразумевает оценку структурной формы. Однако, поскольку элементы $\pi$ являются функциями $\theta$, (2.2) также предоставляет косвенный путь для построения выводов относительно $\theta$. Если $f(z_{i},u_{i}|\pi)$ имеет известную функциональную форму, и если она аддитивно-сепарабельна по $z_{i}$ и $u_{i}$, то есть мы можем записать её в виде:
	
	
	
\begin{equation}
y_{i}=g(z_{i}|\pi)+u_{i}=E[y_{i}|z_{i}]+u_{i},
\end{equation}

тогда  регрессия $y$ на $z$ является естественной прогнозной функцией $y$ для заданного $z$. В этом смысле приведенная форма модели полезна для условных прогнозов $y_{i}$ при заданных $(z_{i},u_{i})$. Чтобы получить прогнозы объясняемой переменной левой части  для заданных значений объясняющих переменных правой части (2.2) требуется иметь оценки $\pi$, получение которых может быть вычислительно проще. 

Важным обобщением (2.3) является  модель с преобразованием, которая для скалярного $y$ принимает следующий вид:
	
\begin{equation}
\Lambda(y)=z'\pi+u,
\end{equation}

где $\Lambda(y) $ является преобразования (например,  $\Lambda(y)=ln(y)$ или $\Lambda(y)=y^{1/2}$). В некоторых случаях преобразование может зависеть от неизвестных параметров. Модель с преобразованием  отличается от регрессии, но она также может быть использована для оценки $E[y|x]$. Важным примером такой модели является модель ускоренная модель жизни (см. Главу 17). 


Одним из наиболее важных и спорных этапов в спецификации структурной модели является пункт 3, который априори располагает переменные  в порядке причинно-следственных связей.
В сущности мы проводим различия между теми переменными, чьё изменение объясняется моделью и теми, чьё изменение определяется внешними факторами и, следовательно, выходит за рамки исследования. В микроэконометрике, в качестве первых выступают количество лет обучения в школе и продолжительность рабочего времени; примерами переменных второго типа являются пол, этническая принадлежность, возраст, и аналогичные демографические показатели. Переменные первой группы, обозначаемые $y$, называются эндогенными, а переменные второй группы, обозначаемые $z$, называется экзогенными переменными. 
	
	
Экзогенность переменное является важным упрощением и по сути оправдывает решение трактовать данную переменную как вспомогательную, и отказаться от её моделирования, поскольку параметры модели для неё не оказывают прямого влияния на изучаемую переменную.  Экзогенность требует более формального определения, которое мы сейчас предоставим. 

	
\section{Экзогенность}

	
Мы начнем с рассмотрения общего случая конечномерного параметрического представления, в котором совместное распределение $W$, с параметрами $\theta$, разбитыми на $(\theta_{1},\theta_{2})$, раскладывается в произведение условной плотности распределения $Y$ при заданном $Z$, и в частной функции плотности $Z$:


\begin{equation}
f_{J}(W|\theta)=f_{C}(Y|Z,\theta)\times f_{M}(Z|\theta).
\end{equation}

Частным случаем является разложение вида:

\[
f_{J}(W|\theta)=f_{C}(Y|Z,\theta_{1})\times f_{M}(Z|\theta_{2}),
\]

где $ \theta_{1} $ и $ \theta_{2} $ функционально независимы. Тогда говорят, что $Z$ является экзогенной по отношению к  $ \theta_{1} $, а это означает, что знание $ f_{M}(Z|\theta_{2}) $  не требуется для статистических выводов о $ \theta_{1}$ и, следовательно, мы можем изучать распределение $Y$ при фиксированном  $Z$. 


Модели можно параметризовать по-разному. Поэтому далее рассмотрим случай, в котором модель описана с помощью параметров $\varphi$, которые находятся во  взаимно-однозначном соответствии с параметрами $\theta$, скажем $\varphi=h(\theta)$, где $\varphi$ разбивается на $(\varphi_{1},\varphi_{2})$. Такая параметризация может представлять интерес, если $\varphi_1$ инвариантны относительно некоторого класса реформ. Допустим, что цель состоит в оценивании $\varphi_1$. Следовательно нас интересует экзогенность $Z$ относительно $\varphi_1$. В таком случае условие экзогенности выглядит следующим образом:
	
\begin{equation}
f_{J}(W|\varphi)=f_{C}(Y|Z,\varphi_{1})\times f_{M}(Z|\varphi_{2}),
\end{equation}

где $\varphi_{1}$ и $\varphi_{2}$ независимы.

%%%%%
Наконец, рассмотрим случай, когда параметр $\lambda$ функцией от $\varphi$, скажем $h(\varphi)$. Тогда для экзогенности $Z$ относительно $\lambda$, нам нужны два условия: (i) $\lambda$ зависит только от $ \varphi_{1}) $, т.е. $\lambda=h(\varphi)$, (при этом интерес представляет только условное распределение) и (ii) $\varphi_{1}$ и $\varphi_{2}$ не имеют перекрестных ограничений, т.е. $(\varphi_{1}, \varphi_{2}) \in \Phi_{1}\times\Phi_{2} = \lbrace\varphi_{1} \in \Phi_{1}, \varphi_{2} \in \Phi_{2}\rbrace$. 

Разложение в (2.5) --- (2.6) играет важную роль в развитии концепции экзогенности. В этой книге акцентируется внимание на трех понятиях экзогенности: (1) слабая экзогенность; (2) отсутствие причинности по Грейнджеру; (3) сильная экзогенность.

{\bf Определение 2.1 (слабая экзогенность):}  $Z$ является слабо экзогенной по $\lambda$, если условия (i) и (ii) выполнены.\\

Если параметры модели неинформативны для статистических выводов о $\lambda$, то статистические выводы о $\lambda $  могут быть сделаны только на основе условного распределения $f(Y|Z,\varphi_{1})$. То есть на практике можно принять слабую экзогенность переменных как данность, если основной интерес состоит в получении выводов о $\lambda$ или $\varphi_1$. Это не означает, что нет никакой статистической модели для $Z$, однако параметры этой модели не играют никакой роли в статистических выводах относительно $\varphi$, и, следовательно, не имеют значения. 

%%% --->
\subsection{Условная независимость}


Первоначально концепция причинности по Грейнджеру была определена в контексте предсказания  временных рядов. В целом, она может быть интерпретирована как форма условной независимости (Холланд, 1986, с. 957). 


Разделяя вектор $z$ на два подмножества $z_{1}$ и $z_{2}$, обозначим с помощью $W=[y,z_{1},z_{2}]$ матрицу интересующих нас переменных. Переменные $z_{1}$ и $y$ условно независимы при заданном $z_{2}$ если

\begin{equation}
f(y|z_{1},z_{2})=f(y|z_{2})
\end{equation}

Это предположение сильнее, чем независимость условного среднего:
 
\begin{equation}
\E(y|z_{1},z_{2})=\E(y|z_{2})
\end{equation}
%%%
В этом случае $z_{1}$ не имеет ценности для прогнозирования $y$. То есть $z_{1}$ не является причиной $y$ по Грейнджеру.

В контексте временных рядов $z_1$ и $z_2$ --- это непересекающиеся подмножества лагированных значений переменной $y$.

{\bf Определение 2.2 (сильная экзогенность):}  $z_{1}$ сильно экзогенна для $\varphi$, если она слабо экзогенна для $\varphi$ и не является причиной по Грейнджеру для $y$, т.е. выполнено (2.8).

\subsection{Экзогенные переменные}

%%% --->
	Экзогенность является сильным предположением. Это свойство случайных величин по отношению к параметрам, представляющим интерес. Поэтому переменная может обоснованно считаться экзогенной в одной структурной модели, а в другой нет, ключевым вопросом является выбор целевых параметров. Необдуманное введение этого предположения будет иметь некоторые нежелательные последствия, которые будут обсуждаться в разделе 2.4. 
	
	
	Предположение об экзогенности может быть априори оправдано теорией, в этом случае оно является частью поддерживаемой гипотезы модели. В некоторых случаях экзогенность может быть хорошей аппроксимацией, в этом случае могут она может тестироваться способами, описанными в разделе 8.4.3. В анализе пространственных данных  данное предположение может быть оправдано как следствие естественного эксперимента или квази-эксперимента, в котором значение переменной определяется внешним вмешательством, например, правительство или регулирующий орган могут выбирать ставку налога или параметр вмешательства. Особый интерес представляет случай, когда внешнее вмешательство приводит к изменению значения важных переменных политики. Такой естественный эксперимент равносилен экзогенности некоторых переменных. Как мы увидим в главе 3, это создает квазиэкспериментальные возможности изучения влияния переменной при отсутствии других осложняющих факторов. 


\section{Линейная модель одновременных уравнений}


Важным частным случаем общей структурной модели описанной в (2.1) является система линейных одновременных уравнений, разработанная эконометристами Комиссии Коулса. Подробное изложение данной модели можно найти во многих учебниках (например, Сарган, 1988). В данном разделе предлагается краткое и выборочное изложение данной модели, см. также раздел 6.9.6. Цель состоит в том, чтобы обсудить нескольких ключевых идей и концепций, которые имеют более общий характер. Хотя анализ ограничен линейными моделями, много интересных идей применимо к нелинейным моделям.
  
\subsection{Система одновременный уравнений}

\textbf{Система линейных одновременных уравнений} (СЛОУ) представляется в виде
\[
\begin{aligned}
y_{1i}\beta_{11}+ &\dots + y_{Gi}\beta_{1G}+z_{1i}\gamma_{11} + \dots + &z_{Ki}\gamma_{1K}=  &u_{1i} \\
                  &\vdots                                               &\vdots           =  &\vdots \\
y_{1i}\beta_{G1}+ &\dots + y_{Gi}\beta_{GG}+z_{1i}\gamma_{G1} + \dots + &z_{Ki}\gamma_{GK}=  &u_{Gi},
\end{aligned}
\]

где $i$ --- обозначает номер наблюдения.

Четкое различие или упорядочивание сделано между эндогенными переменными, $y'_{i}=(y_{1i},\dots,y_{Gi})$, и экзогенными переменными, $z'_{i}=(z_{1i},\dots,z_{Ki})$. По определению экзогенные переменные не коррелируют со случайными ошибками $u_{1i},\dots,u_{Gi}$. В неограниченной форме каждая переменная входит в каждое уравнение. 

В матричной форме уравнение для $i$-го наблюдения системы из $G$ уравнений выглядит следующим образом:

\begin{equation}
y'_{i}B+z'_{i}\Gamma=u'_{i},
\end{equation}

где $y'_{i}$,$B$,$z'_{i}$,$\Gamma$ и $u'_{i}$ имеют размерности $G \times 1$ , $G \times G$ , $ K \times 1$, $K \times G$ и $G \times 1$, соответственно. Для конкретных значений $(B,\Gamma)$ и $(z_i,u_i)$ система из $G$ одновременных уравнений в принципе может быть решена относительно $y_i$.

Стандартные предположения для СЛОУ следующие:

\begin{enumerate}
\item $B$ невырожденна и имеет ранг $G$
\item $\rank Z = K$,   матрица $Z$ размера $N \times K$ составлена из $z'_{i}$, $i=1, \dots, N$, расположенных друг под другом.
\item $\plim N^{-1}Z'Z=\Sigma_{ZZ}$ --- симметричная положительно определенная матрица размера $K \times K$.
\item $u_{i}\sim N[0,\Sigma]$, то есть  $\E[u_{i}]=0$ и $\E[u_{i}u'_{i}]=\Sigma=[\sigma_{ij}]$, где $\Sigma$ --- симметричная положительно определенная матрица размера $G \times G$.
\item Ошибки в каждом уравнении не зависят от прошлых значений.
\end{enumerate}

В этой модели структурными параметрами являются  $(B,\Gamma,\Sigma)$. Введем обозначения

\[
Y=\begin{bmatrix} y'_{1} \\ \vdots \\ y'_{N} \end{bmatrix}, \quad  Z=\begin{bmatrix} z'_{1} \\ \vdots \\ z'_{N} \end{bmatrix}, \quad U=\begin{bmatrix} u'_{1} \\ \vdots \\ u'_{N} \end{bmatrix}
\]

Тогда можно представить структурную модель в более компактной записи:

\begin{equation}
YB+Z\Gamma=U,
\end{equation}

где $Y$,$B$, $Z$,$\Gamma$ и $U$ имеют размерности $N \times G$ , $G \times G$ , $ N \times K$, $K \times G$ и $N \times G$, соответственно. В таком случае, решение для всех эндогенных переменных в терминах экзогенных выглядит следующим образом:


\begin{equation}
Y+Z\Gamma B^{-1}=UB^{-1},
Y=Z\Pi+V,
\end{equation}

где $\Pi= -\Gamma B^{-1}$, с учётом предпосылки 4, $v_{i} \sim N[0,B^{-1^{'}} \Sigma B^{-1}]$. Данная форма записи СЛОУ называется \textbf{приведенной формой}.


	В рамках СЛОУ структурная модель  по нескольким причинам. Во---первых, сами уравнения имеют экономическую интерпретацию, такие как спроса или предложения, производственные функции, и так далее, и на них распространяются ограничения экономической теории. Следовательно, $B$ и $\Gamma$ являются параметрами, описывающие экономическое поведение. Следовательно априорная теория может быть использована для формирования гипотез относительно знака и размер отдельных коэффициентов. С другой стороны, неограниченная форма приведенных параметров является сложной функцией структурных параметров, и поэтому её трудно оценить.  

	
	Рассмотрим, без ограничения общности, первое уравнение в модели (2.11), с зависимой переменной $y_{i}$. Кроме того, некоторые из оставшихся $G-1$ эндогенных переменных и $K-1$ экзогенных переменных могут быть исключены их этого уравнения. Из (2.12) мы видим, что в целом эндогенные переменные $Y$ зависят стохастически от $V$, который в свою очередь является функцией структурных ошибок $U$, поэтому, в общем $plim N^{-1}Y'U\not=0$. Как правило, применение метода наименьших квадратов в системах линейных одновременных уравнений дает несовместимые оценоки. Это хорошо известный и основной результат в литература про системы одновременных уравнений, что часто называют проблемой смещения. Большая часть литературы по одновременным уравнениям занимается идентификацией и последовательной оценкой при методе наименьших квадратов см. Сарган (1988) и Шмидт (1976), а также раздел 6.9.6. 
	
	
	Сокращенная форма СЛОУ выражает каждую эндогенную переменную в виде линейной функции всех экзогенных переменных и всех структурных ошибок; (ошибки в приведенной форме являются линейными комбинациями ошибок структурной формы). Из приведенной формы для $i$---го наблюдения получаем:

\begin{equation}
E[y_{i}|z_{i}]=z'_{i}\Pi,
\end{equation}
\begin{equation}
V[y_{i}|z_{i}]=\Omega=B^{-1^{'}} \Sigma B^{-1}
\end{equation}
 
 
Параметры приведенной формы $\Pi$ являются производными параметрами, определенными в зависимости от структурных параметров. Если $\Pi$  можно последовательно оценить, то приведенная форма может быть использована, чтобы сделать прогноз об изменениях в $Y$ под влиянием внешних изменений в $Z$; это возможно, даже если $B$ и $\Gamma$ не известны. Учитывая экзогенность $Z$, полный набор приведенной формы является многомерной регрессионной моделью, которая может быть оценена методом наименьших квадратов. Сокращенная форма обеспечивает основу для принятия условных предсказаний $Y$ при заданных $Z$. 


	Ограниченная приведенная форма является неограниченной приведенной формой модели при заданных ограничениях. Если эти ограничения совпадают с ограничениями структуры, то структурная информация может быть извлечена из приведенной формы. 
	
	
	В рамках СЛОУ, неизвестные структурные параметры, ненулевые элементы $B$, $\Gamma$,$\Sigma$, играют ключевую роль, потому что они отражают причинную структуру модели. Взаимозависимость между эндогенными переменными описывается $B$, а реакция эндогенных переменных на экзогенные шоки в $Z$ отражается в параметрах матрицы $\Gamma$. Поэтому нас интересуют те параметры, которые измеряют предельное влияние изменения независимой переменной, $y_{j}$ или $z_{k}$ на результаты $y_{l}$, где $l\not=j$, функцию этих параметров и данных. Элементы $\Sigma$ описывают дисперсию и свойств зависимости случайных ошибок, следовательно, они измеряют некоторые свойства генерирования данных. 
	

\subsection{Причинная интерпретация в СЛОУ}


Простой пример может иллюстрировать причинную интерпретацию параметров в СЛОУ. Структурная модель имеет две непрерывные эндогенные переменные $y_{1}$ и $y_{2}$, одну непрерывную экзогенную переменную $z_{1}$, одно стохастическое соотношение, связывающее $y_{1}$ и $y_{2}$ и одно уравнение, связывающее все три переменные в модели:


\[
y_{1}=\gamma_{1}+\beta_{1}y_{2}+u_{1}, \quad 0<\beta_{1}<1, \\
y_{2}=y_{1}+z_{1}.
\]


В этой модели $u_{1}$ является стохастической ошибкой, независима от $z_{1}$, с четко определенным распределением. Параметра $\beta_{1}$ имеет ограничение, которое также является частью спецификации модели. Переменная $z_{1}$ является экзогенной и, следовательно, ее изменение индуцируется внешними источниками, которые мы можем рассматривать как вмешательства, они оказывают непосредственное влияние на $y_{2}$. Это влияние измеряется при помощи приведенной формы модели, которая выглядит следующим образом: 

\[
y_{1}=\frac{\gamma_{1}}{1-\beta_{1}}+\frac{\beta_{1}}{1-\beta_{1}}z_{1}+\frac{1}{1-\beta_{1}}u_{1} \\
= E[y_{1}|z_{1}]+v_{1}, \\
\]

\[
y_{2}=\frac{\gamma_{1}}{1-\beta_{1}}+\frac{1}{1-\beta_{1}}z_{1}+\frac{1}{1-\beta_{1}}u_{1} \\
= E[y_{2}|z_{1}]+v_{1},
\]

где $v_{1}=\frac{u_{1}}{1-\beta_{1}}$. Коэффициенты приведённой формы $\frac{\beta_{1}}{1-\beta_{1}}$ и $\frac{1}{1-\beta_{1}}$  имеют причинную интерпретацию: любые внешние индуцированных изменения $z_{1}$ вызовет количественное изменение $y_{1}$ и $y_{2}$ . Следует отметить, что в этой модели $y_{1}$ и $y_{2}$ также реагируют на изменения $u_{1}$, поэтому, чтобы не смешать влияние двух источников изменчивости эндогенных переменных, потребуем, чтобы $z_{1}$ и $u_{1}$ являлись независимыми. \\
	К тому же

\[
\frac{\partial y_{1}}{\partial y_{2}}=\beta_{1}=\frac{\beta_{1}}{1-\beta_{1}}\div\frac{1}{1-\beta_{1}} \\
=\frac{\partial y_{1}}{\partial z_{1}}\div\frac{\partial y_{2}}{\partial z_{1}}.
\]


В каком смысле $\beta_{1}$ измеряет причинно-следственное влияние на $y_{1}$ и $y_{2}$? Чтобы увидеть возможные трудности, заметим, что $y_{1}$ и $y_{2}$ являются взаимозависимыми или совместно определены, так что неясно, в каком смысле $y_{2}$ <<влияет на>> $y_{1}$. Хотя $z_{1}$ (и $u_{1}$) являются причиной изменение в приведённой форме, $y_{2}$ промежуточно влияет на $y_{1}$. То есть, первое структурное уравнение представляет собой срез влиянии на $y_{2}$ на  $y_{1}$, в то время как приведённая форма дает равновесное воздействие после учета всех взаимодействий  между эндогенными переменными. В рамках СЛОУ даже эндогенные переменные рассматриваются как причинные переменные и их коэффициенты как причинные параметров. Такой подход может вызвать недоумение для тех, кто видит причинность в экспериментальных условиях, когда независимые источники изменения являются причинными переменными. Подход СЛОУ имеет смысл, если $y_{2}$ имеет независимый и экзогенный источник изменений, который в этой модели представлен как $z_{1}$. Поэтому предельный коэффициент $\beta_{1}$ является мерой того, как $y_{1}$ и $y_{2}$ реагируют на изменение $z_{1}$. 


	В целом, мы можем задаться вопросом: при каких условиях параметры СЛОУ имеют значимую причинную интерпретацию. Мы вернемся к этому вопросу при обсуждении понятия идентификации в разделе 2.5.

%%% <---
\subsection{Нелинейный модели и модели скрытых переменных}


Если система одновременных уравнений нелинейна только по параметрам, то структурную модель можно записать в виде:



\begin{equation}
YB(\theta)+Z\Gamma(\theta)=U,
\end{equation}

где $B(\theta)$ и $\Gamma(\theta)$ матрицы, элементами которых зависят от структурных параметров $\theta$. Приведенная форма в явном виде может быть получена как раньше. 

Если модель нелинейна по переменным, то получение приведенной формы в явном (аналитическом) виде может быть невозможно. Впрочем, обычно зависимые переменные можно выразить численно или воспользоваться линейной аппроксимацией.


Во многих микроэконометрических моделях присутствуют скрытые или ненаблюдаемые переменные, помимо наблюдаемых эндогенных переменных. Например, в моделях поиска и моделях аукционов используется понятие минимального уровня оплаты труда или резервной цены, в моделях выбора --- косвенной функции полезности, и так далее. В случае таких моделей структурная модель (2.1), может быть заменена на

\begin{equation}
g(y^{\ast}_{i},z_{i},u_{i}|\theta)=0,
\end{equation}

где скрытая переменная $y^{\ast}_{i}$ заменяет наблюдаемую переменную $y_{i}$. Соответствующая приведенная форма выражает $y^{\ast}_{i}$ через $(z_{i},u_{i})$ в виде

\begin{equation}
y^{\ast}_{i}=f(z_{i},u_{i}|\pi).
\end{equation}


Эта приведенная форма имеет ограниченную пользу, так как $y^{\ast}_{i}$ не наблюдаются полностью. Тем не менее, если у нас есть функция $y_{i}=h(y^{\ast}_{i})$, которая связывает  наблюдаемые  переменные $y_{i}$ с соответствующими скрытыми, то приведенная форма может быть записана с помощью  наблюдаемых переменных  так:

\begin{equation}
y_{i}=h(f(z_{i},u_{i}|\pi)).
\end{equation}
Подробнее см. в разделе 16.8.2


Когда структурная форма модели  нелинейна по  переменным или, когда существуют скрытые переменные, приведённую форму может быть  трудно получить. В таких случаях на практике используется аппроксимация. Из соображений удобства аналитических выкладок или численных подсчетов, специальная функциональная может использоваться для того, чтобы выразить эндогенную переменную через  экзогенные. Данный результат называют приведенной формой записи модели.  



\subsection{Интерпретация структурных отношений}


Маршак (1953, стр. 26) в своей статье, оказавшей существенное влияние, дает следующее определение структуры: 


Структура определяется как набор условий, которые не меняются в то время, пока делаются наблюдения, но которые могут измениться в будущем. Если некоторое изменение структуры ожидаемо, то для прогнозирования переменных при осуществлении некоторой политики необходимо  знание прошлой структуры\ldots 
В экономике, условиями, определяющими структуру являются (1) соотношения, описывающие поведение человека или институциональную среду, как правило включающие случайные возмущения или случайные ошибки измерения, (2) совместное распределение этих случайных величин. 


Маршак утверждал, что структура имеет большое значение для количественной оценки или тестирования экономической теории, а выбор наилучшей политики требует знания структуры. 


В литературе про системам одновременных уравнений  структурная модель относится к <<автономным>> (не выведенным) взаимосвязям. Есть и другие близкие понятия структуры. Одна из таких концепций формулируется с помощью <<глубоких параметров>>, под которыми подразумеваются параметры технологии или предпочтений, не зависящие от внешних воздействий. 


В последние годы распространилось альтернативное использование термина структуры. Под структуро подразумевают эконометрические модели, основанные на гипотезе динамической стохастической оптимизации рациональными агентами. При таком подходе отправной точкой для любой структурной модели  является множество необходимых условий первого порядка, которое определяют оптимальное поведение агента. 
Например, в стандартном случае  решения задачи максимизации полезности поведенческими условиями являются будут детерминистические условия первого порядка на предельную полезность. Если соответствующие функциональные формы заданы в явной форме, и присутствуют стохастические ошибки оптимизации, то условия первого порядка определяют поведенческую модель, параметры которой характеризуют функцию полезности. Эти параметры называют  <<глубокими>> или инвариантными по отношению к политике. Примеры рассмотрены в разделах 6.2.7 и 16.8.1. 


Укажем две особенности этого сильно структурированного подхода. 
Во-первых, он весьма серьёзным образом априорно полагается на экономическую теорию. Экономическая теория  используется не просто для создания списка релевантных переменных, которые можно использовать в более или менее произвольной  функциональной форме. Скорее,  экономическая теория играет большую (но не исключительную) роль в спецификации, оценивании и статистических выводах. Второй особенностью является то, что идентификация, спецификация и оценивание полученной модели могут быть очень сложными, потому что сама задача оптимизации агента является потенциально сложной, особенно если это динамическая оптимизация в условиях неопределенности, с наличием дискретности и разрывов в данных, см. работу Раста (1994) .


\section{Идентификация}


Цель подхода систем одновременных уравнений заключается в получении состоятельной оценки $(B,\Gamma,\Sigma)$ и в проведении статистических выводов. Важной предпосылкой для состоятельности оценок является идентифицируемость модели. Мы кратко обсуждаем две важные  взаимосвязанные концепции  эквивалентности и идентифицируемости в контексте параметрических моделей.


Идентификация состоит в определении значений параметров при достаточном количестве наблюдений. В этом смысле она является асимптотической концепцией. Статистическая неопределенность неизбежно затрагивает любые выводы на основе конечного числа наблюдений. 
Предположив, что имеется достаточное количество наблюдений, мы встаем перед вопросом, возможно ли определить значение интересующего нас параметра. Речь идет как об определении точечного значения, так и об определении множества возможных  значений. 
Идентификация является одним из ключевых свойств, логически происходит до и отдельно от  статистического оценивания. Большая часть эконометрической литературы под идентификацией подразумевает точечную идентификацию. В этом разделе мы также работает с понятием точечной идентификации. Тем не менее, идентификации множества или идентификация границ, также является важным подходом, который будет использоваться в отдельных местах этой книги (например, главы 25 и 27, см. Мански, 1995).


{\bf Определение 2.3 (Эквивалентность):} Две структуры модели, определяемые с помощью закона распределения  $\Pr[x|\theta]$,$x\in W$, $\theta \in \Theta$,  эквивалентны, если $\Pr[x|\theta^{1}]=\Pr[x|\theta^{2}] \forall x \in W.$ 


Менее формально,  если две структурных  модели приводят к идентичным  совместным функциям   распределения, то они эквивалентны. Существование множества  эквивалентных структур равносильно отсутствию идентификации.

{\bf Определение 2.4 (Идентификация):} Структура $\theta^{0}$ идентифицируема, если не существует других эквивалентных ей структур в множестве $\Theta$.


Простой пример неидентифицируемости  возникает в случае жесткой мультиколлинеарности между регрессорами в линейной модели $y=X\beta+u$. Тогда мы можем идентифицировать линейную комбинацию $C\beta$, где $\rank[C] < \rank[\beta]$, но мы не можем идентифицировать $\beta$.


Это определение относится к уникальности структуры. В контексте систем одновременных уравнений  это определение означает, что существует единственная тройка $(B,\Gamma,\Sigma)$ соответствующая данным. 
В системах одновременных уравнений, как и в других случаях, идентификация подразумевает возможность получить уникальные оценки структурных параметров при известных выборочных моментах. Например, в случае приведенной формы (2.12), при указанных предположениях, метод наименьших квадратов позволяет получить единственные оценки матрицы $\Pi$, то есть $\hat{\Pi}=[Z'Z]^{-1}Z'Y$, а для идентифицируемости  $(B,\Gamma)$ необходимо, чтобы существовало решение относительно $B$ и $\Gamma$  уравнения $\Pi+\Gamma B^{-1}=0$, при некоторых  априорных ограничениях  модели. 
Единственность решения означает точную  идентифицируемость модели.

Модель в целом называется идентифицируемой,  если все параметры модели идентифицируемы. Вполне возможно, что в некоторых моделях только подмножество параметров идентифицируемо. В некоторых случаях важно иметь возможность идентифицировать заданную функцию параметров, и не обязательно все  параметры по отдельности. Идентифицируемость функции от параметров означает, что функция может быть однозначно восстановлена по $F(W|\Theta)$.


Как можно убедиться в том, что структуры альтернативных спецификаций модели могут быть исключены? В системах одновременных уравнений решение этой проблемы связано с добавлением к имеющимся наблюдениям априорных ограничений на $(B,\Gamma,\Sigma)$. Эти априорные ограничения должны внести в модель достаточное количество дополнительной информации, чтобы исключить существование других эквивалентных структур.
	
	
О необходимости априорных ограничений свидетельствует следующее рассуждение. Прежде всего заметим, что при предположениях раздела 2.4.1, приведенная форма, определяемая формулой $(\Pi,\Omega)$, всегда единственна. Изначально предположим, что нет никаких ограничений на $(B,\Gamma,\Sigma)$. Далее предположим, что существует два эквивалентных структуры $(B_{1},\Gamma_{1},\Sigma_{1})$ и $(B_{2},\Gamma_{2},\Sigma_{2})$. Тогда
	
\begin{equation}
\Pi=-\Gamma_{1} B^{-1}_{1}=-\Gamma_{2} B^{-1}_{2}.
\end{equation}


Пусть $H$ --- невырожденная  матрица размера $G \times G$, тогда
$\Gamma_{1} B^{-1}_{1}=\Gamma_{1} H H^{-1} B^{-1}_{1}=\Gamma_{2} B^{-1}_{2}$, что означает, что $\Gamma_{2}=\Gamma_{1}H$, $B_{2}=B_{1}H$. Таким образом, вторая структура является линейным преобразованием первой.


В системах одновременных уравнений решение данной проблемы заключается во введение ограничений на $(B,\Gamma,\Sigma)$ таких, что мы можем исключить существование линейных преобразований, которые приводят к  эквивалентным структурам. Другими словами, ограничения на $(B,\Gamma,\Sigma)$, должны быть такими, чтобы не существовало матрицы $H$, которая даёт другую структуру с такой же приведённой формой; при заданных $(\Pi,\Omega)$ будет существовать единственные решения уравнений $\Pi=-\Gamma B^{-1}$ и $\Omega=(B^{-1})'\Sigma B^{-1}$.
	
	
На практике могут быть наложены различные ограничения, в том числе (1) нормализация, например, можно приравнять диагональные элементы матрицы $B$ к единице, (2) исключение (зануление) отдельных коэффициентов, и линейные однородные и неоднородные ограничения и (3) ковариационные ограничения и ограничения в виде неравенств.  Подробную информация о необходимых и достаточных условиях  идентификации  линейных и нелинейных моделей можно найти во многих текстах, включая книгу Саргана (1988).
	
	
Исключающие ограничения по существу означают, что модель содержит некоторые переменные, которые имеют нулевое влияние на некоторые эндогенные переменные. То есть, определенные направления причинно-следственных связей априори невозможны. Это дает возможность идентифицировать другие направления причинно-следственных связей. Например, в простом случае с двумя переменными, приведенном выше, $z_{1}$ не входит в уравнение $y_{1}$, что позволяет оценить  прямое воздействие $y_{2}$ на $y_{1}$. Хотя исключающие ограничения являются самыми простыми для реализации, в параметрических моделях идентификация может быть достигнута за счет ковариационных ограничений или ограничений типа неравенств.
	
	
Если нет никаких ограничений на $\Sigma$, а диагональные элементы $B$ нормированы и равны 1, то необходимым условием для идентификации является условие порядка, согласно которому, количество исключенных из уравнения экзогенных переменных должно быть не меньше, чем  количество включенных эндогенных переменных. Часто приводимым в учебниках достаточным условием является условие ранга, которое гарантирует, что  параметры $j$-го уравнения $\Pi \Gamma_{j}=-B_{j}$ соответствуют единственному решению для $\Gamma_{j},B_{j}$ при заданном $\Pi$.

Если модель идентифицируема, то используют термин точно идентифицируема для случая, когда условие порядка выполнено как равенство, и  термин сверх-идентифицируема для случая, когда количество ограничений на систему уравнений превосходит количество, необходимое для точной идентификации.

Идентификация в нелинейных системах уравнений обсуждает Сарган (1988), он также приводит ссылки на более ранние работы.
	
\section{Модель с одним уравнением}


Без ограничения общности рассмотрим первое уравнение системы одновременных уравнений при нормализации $\beta_{11}=1$. Пусть $y=y_{1}$, а $\by_{1}$ обозначает эндогенные компоненты $\by$ отличные от $y_{1}$, и пусть $z_{1}$ обозначает экзогенные компоненты $z$, тогда


\begin{equation}
y=\by'_{1}\alpha+z'_{1}\gamma+u.
\end{equation}

Во многих работах пропускаются формальные шаги при переходе от системы уравнений к одному уравнению и начинают с написания уравнения регрессии

\[
y=x'\beta+u,
\]

где некоторые компоненты $x$ являются эндогенными (неявно $\by_{1}$), а другие экзогенными (неявно $z_{1}$).
Основная задача состоит в оценке влияния изменений  ключевых регрессоров, которые могут быть эндогенными или экзогенными, в зависимости от предположений. Инструментальные переменные или двухшаговый метод наименьших квадратов являются естественными способами оценивания (см. разделы 4.6, 6.4 и 6.5).


В системах одновременных уравнений естественно специфицировать по крайней мере некоторые из оставшихся уравнений в модели, даже если они не являются предметом исследования. Пусть $y_{1}$ имеет размерность 1. Тогда одна из возможностей состоит в том, что можно указать структурное уравнение для $y_{1}$ и других эндогенных переменных, которые могут появиться в  структурном  уравнении для $y_{1}$. Вторая возможность состоит в использовании уравнения для  $y_{1}$ в приведенной форме. В этой форме будут видны экзогенные переменные, которые влияют на $y_{1}$, но не влияют напрямую на $y$. Преимуществом является то, что при таком подходе инструментальные переменные возникают естественным путем. Тем не менее, в последних эмпирических исследованиях с использованием инструментальных переменных для одного уравнения формальный шаг с записью уравнения в приведенной форме для эндогенной переменной в правой части нередко пропускают.
	
	
\section{Модели потенциального результата}


Необходимость причинно-следственных рассуждений в эконометрических моделях особенно сильна, когда акцент делается на влияние государственной политики или индивидуальных решений на некоторый результат. Примерами могут служить  влияние социальных трансфертов на предложение труда, влияние количества учеников в классе на эффективность обучения, а также влияние наличия медицинской страховки на интенсивность пользования услугами здравоохранения. 
Во многих случаях причинные переменные сами отражают индивидуальные решения и, следовательно, потенциально эндогенны. Когда, как это обычно бывает, эконометрические оценки и выводы основаны на данных наблюдений, идентификация и выводы о причинных параметрах ставят перед исследователем немало проблем. 
Эти трудности можно быть преодолены, если для причинно-следственного моделирования  использовать данные, полученные в результате контролируемого социального эксперимента с продуманным планом. 
Хотя такие эксперименты существуют (см. примеры и подробности в разделе 3.4.), как правило, они дороги и их трудно организовать.
Поэтому гораздо удобнее оценивать причинно-следственные модели на данных полученных в результате  естественного эксперимента или в квази-экспериментальных условиях. 
В разделе 3.4 обсуждаются плюсы и минусы таких  данных. Для текущих задач о естественном эксперименте или о квази-экспериментальных условиях, можно думать как о ситуации, в которой некоторые причинные переменные изменяются экзогенно и независимо от других объясняющих переменных, что позволяет относительно легче  оценить причинно-следственные параметры.
	
	
Одним из основных препятствий для моделирования причинности является фундаментальная проблема причинно-следственных статистических выводов (Холланд , 1986). Пусть $X$ --- предполагаемая причина, а $Y$ --- результат. Изменяя значение $X$, мы можем изменить значение $Y$. Предположим, что значение $X$ изменяется с $x_{1}$ до $x_{2}$. Мера причинного  изменения $Y$ определяется путем сравнения двух значений $Y$: $y_{2}$, которое является результатом изменений, и $y_{1}$, которое было бы, если бы не было никаких изменений в $X$. Однако, если $X$ изменился, то значение $Y$, которое было бы в отсутствие изменений, не наблюдается. Следовательно, ничего больше нельзя сказать о причинном влиянии без некоторых гипотез о том, какое значение имел бы $Y$ в отсутствие изменений $X$. Это гипотетическое ненаблюдаемое значение называют контрфактическим. Иными словами, все причинно-следственные  выводы должны быть сделаны исходя из сравнения фактического и контрфактического результата. В обычной эконометрической модели (например системах одновременных уравнений) нет необходимости явно формулировать контрфактический результат.
	
	
Относительно новый сюжет в микроэконометрической литературе --- оценивание эффективности программы мер или оценивание воздействия --- задает статистическую парадигму для оценивания   причинно-следственных параметров. В статистической литературе этот подход также известен как причинно-следственная модель Рубина (Rubin causal model, RCM), названной в знак признания существенного вклада Рубина ( 1974, 1978 ), который  в свою очередь, ссылается на Р.А. Фишера, как создателя подхода. Согласно действующей традиции, мы называем данный подход причинно-следственной моделью Рубина, однако Нейман (Шплава-Нейман) также предложил похожую модель в статье опубликованной на польском языке в 1923 году, см. работу Неймана (1990). Модели с использованием контрфактических значений в эконометрике разрабатывались независимо после ключевой работы Роя (1951).  В оставшейся части этого раздела будут проанализированы характерные особенности модели Рубина. 


Оценивание причинно-следственных параметров на основе контрфактических ситуаций дает  статистически осмысленное и рабочее  определение причинности, которое в некоторых отношениях отличается от традиционного определения комиссии Коулса. Во-первых, в идеальных условиях этот подход приводит к использованию простых эконометрических методов. Во-вторых, при этом подходе внимание сосредоточено на небольшом количестве параметров, которые считаются релевантными для анализируемой политики. Здесь видно отличие с традиционным эконометрическим подходом, где внимание уделяется оценке всех структурных параметров. В-третьих, данный подход дает дополнительные сведения о свойствах причинно-следственных параметров, оцениваемых структурными методами.
	
\subsection{Причинно-следственная модель Рубина}



Мы используем термины <<воздействие>> и <<причина>> наравне. В медицинских исследованиях нового лекарства участвуют две группы людей: те, кто принимает лекарство и те, кто не принимает. Реакцию на препарат у тех, кто принимал, сравнивают с показателями у тех, кто не принимал лекарство. Мерой причинно-следственного воздействия является средняя разница показателей этих групп. В экономике термин воздействие используется очень широко. По существу он охватывает переменные, влияние которых на некоторый результат является объектом исследования. Примерами причинно-следственных связей являются: связь количества лет обучения с заработной платой, размер класса и успеваемость, обучение на рабочем месте и заработок. Следует отметить, что воздействие не обязательно должно быть экзогенным, а во многих случаях оно является эндогенной переменной.


В рамках  модели потенциального результата (potential outcome model, POM),  предполагается, что каждый индивид целевой группы населения потенциально может быть подвержен воздействию. Оценивание строится на базе тройки $(y_{1i},y_{0i},D_{i})$, $i=1,\dots,N$. Переменная $D$ принимает значения 1 и 0 соответственно, когда воздействие получено или не получено; $y_{1i}$ измеряет зависимую переменную для индивида $i$, получающего воздействие, а $y_{0i}$ --- для не получающего. То есть


\begin{equation}
y_{i}=
\begin{cases}
y_{1i}, & \text{ если }D_{i}=1 \\
y_{0i}, &  \text{ если }D_{i}=0.
\end{cases}
\end{equation}

Получение и неполучение воздействия являются взаимоисключающими для $i$-го индивида, только одна из двух зависимых переменных доступна для $i$-го индивида, недоступная величина является контрфактической. Эффект от воздействия $D$ измеряется как $(y_{1i}-y_{0i})$. Средний причинно-следственный эффект $D_{i}=1$, относительно $D_{i}=0$, измеряется с помощью среднего эффекта воздействия (average treatment effect, АТЕ):

\begin{equation}
ATE=\E[y|D=1]-\E[y|D=0],
\end{equation}

где ожидания берутся относительно вероятностного распределения по целевой группе населения. В отличие от обычной структурной модели, в которой важны  предельные эффекты, в модели потенциального результата подчеркивается роль среднего эффекта  воздействия и параметров, связанных с ним.


Экспериментальный подход к оценке параметров ATE состоит в случайном назначении воздействия с последующим сравнением результатов с индивидами, не получившими воздействия, которые служат в качестве контрольной группы. Этот план выборки подробнее обсуждается в главе 3. 
Случайное назначение воздействия  предполагает, что лица, подвергающиеся воздействию выбираются случайным образом, и, следовательно, назначение воздействия не зависит от результата и не коррелирует с характеристиками индивидов. 
В этом случае имеют место два существенных упрощения.
Индикатор воздействия может рассматриваться как экзогенная переменная. Коэффициент при нем в линейной регрессии не будет страдать от смещения вызванного пропуском переменных, даже при неизбежном пропуске  некоторых существенных переменных в регрессии. При определенных условиях, обсуждаемых более подробно в главах 3 и 25, средняя разница между результатами экспериментальной и контрольной групп даст оценку ATE. Выигрыш от хорошо продуманных экспериментов заключается в относительной простоте, с которой могут быть сделаны выводы о причинно-следственных связях. Конечно, чтобы достичь высокой точности оценки эффекта воздействия, необходимо учитывать в модели факторы, которые помимо самого воздействия могли повлиять на результат. 
	
	
Из-за того, что случайное назначение воздействия, как правило, не  возможно в экономике, оценивание ATE параметров должно быть основано на данных наблюдений, полученных при неслучайном назначении воздействия. 
В такой ситуации состоятельное оценивание ATE будет сопряжено с трудностями, которые включают, например, возможную корреляцию между гипотетическим результатом и назначением воздействия, пропущенные переменные, и эндогенность индикатора воздействия. Некоторые эконометристы считают, что отсутствие рандомизации является главным препятствием для убедительного статистического вывода о причинно-следственных связях.
	
	
Модель потенциального результата можно использовать для заключений о причинно-следственных связях, если контрфактические величины можно четко определить и работать с ними.
Ясное определение контрфактических величин с объяснением того, какие показатели нужно сравнивать, является важной составляющей этой модели. Если четко не определена разница  между наблюдаемыми и контрфактическими величинами,  как это может быть в случае с данными наблюдений, то ответ на вопрос о том, на кого происходит воздействие, остается неясным. 
Величина ATE взвешивает и комбинирует предельные эффекты разных групп генеральной совокупности. Для того, чтобы с контрфактическими величинами можно было работать необходимы определенные предположения. Для того, чтобы оценить ATE необходимо учесть результаты и группы подвергавшейся воздействию, и группы неподвергавшейся воздействию.
Например, необходимо определить группу индивидов, неподвергавшуюся воздействию, которая наиболее похожа по характеристикам на экспериментальную группу, если бы к ней воздействие не применялось. Не всегда данный шаг может быть реализован.
Способ отбора индивидов, получающих воздействие, связан с особенностями плана выборки, обсуждаемыми в главах 3 и 25.

	
	
Вторая полезная особенность моделей потенциального результата состоит в том, что они позволяют увидеть  возможности причинно-следственного моделирования, появившиеся в результате естественного или квази-эксперимента. Когда данные порождаются в таких условиях, и при наличии некоторых предположений,  моделирование причинно-следственных связей  может происходить и без всех сложностей систем одновременных уравнений. Эта тема развивается далее в главах 3 и 25.

%%%% --->
	В-третьих, в отличие от структурной формы систем одновременных уравнений, где все переменные кроме объясняемой могут считаться <<причины>>, в модели потенциального результата не все объясняющие переменные можно рассматривать как причинные. Многие из них просто характеристики индивида, которые необходимо включать в регрессионный анализ, а характеристики не являются причинами (Холланд, 1986). Причинные параметры должны быть связаны с переменными, на которые фактически или потенциально, прямо или косвенно, можно воздействовать.
	
	
	Наконец, идентифицируемость  параметров ATE может быть более легкой исследовательской целью  и, следовательно, возможна в ситуациях, когда  система одновременных уравнений в целом не идентифицируема (Ангрист, 2001). Вопрос о верности данного утверждения надо решать отдельно в каждом конкретном случае. Тем не менее, многие из имеющихся приложений модели потенциального результата обычно используют ограниченную, а не полную информацию. Тем не менее, даже в рамках систем одновременных уравнений использование ограниченной информации также возможно, как было описано ранее.



\section{Причинное моделирование и стратегии оценивания}


В этом разделе мы кратко расскажем о некоторых подходах, которые используют эконометристы для моделирования причинно-следственных связей. Эти подходы могут быть использованы и в системах одновременных уравнений, и в моделях потенциального результата, но, их как правило, связывают с системами одновременных уравнений.

\subsection{Подходы к идентификации}
\begin{center}
Структурные модели с полной информацией
\end{center}


Один из вариантов данного подхода основан на параметрической спецификации совместного распределения эндогенных переменных при фиксированных экзогенных. Взаимосвязи не обязательно являются следствием из  модели оптимального поведения. Параметрические ограничения накладываются для обеспечения идентификации параметров модели, которые являются целью статистического анализа. Вся модель оценивается одновременно с использованием метода максимального правдоподобия или метода моментов. Мы называем этот подход структурным подходом с полной информацией. Для хорошо специфицированных моделей этот подход является привлекательным, но в целом его потенциальный недостаток заключается в том, что модель может содержать плохо специфицированные  уравнения. При одновременном оценивании ошибка спецификации в одном месте может затронуть оценки остальных параметров. 


Этот подход статистически можно интерпретировать, как подход, в котором совместное распределение вероятностей эндогенных переменных, при фиксированных экзогенных переменных, лежит в основе вывода о причинности. Совместное распределение может быть следствием одновременной или динамической взаимозависимости между эндогенными переменными и/или случайными ошибками разных уравнений.
	
	
\begin{center}
Структурные модели с неполной информацией
\end{center}


Напротив, когда центральным объектом статистического вывода является оценка одного или двух ключевых параметров, может быть использован информационно---ограниченный подход. Особенностью данного подхода является то, что, хотя одно уравнение находится в центре внимания вывода, используется совместная зависимость между ним и другими эндогенными переменными. Это требует явные предположения о некоторых особенностях модели, которые не являются основным объектом вывода. Метод инструментальных переменных, последовательное многошаговый метод, и метод ограниченного максимального правдоподобия информации являются конкретными примерами такого подхода. Для реализации данного подхода обычно работают с одним (или более) структурным уравнением. К тому же, информационно---ограниченный  подход вычислительно более сложный, чем с информационно---полный подход.


	Статистически можно интерпретировать информационно---ограниченный подход как тот, в котором совместное распределение раскладывается в произведение условной модели эндогенной переменной, представляющие интерес, например $y_{1}$, и предельной модели для других эндогенных переменных, скажем, $y_{2}$, которые во множестве условных переменных, как в

\begin{equation}
f(y|x,\theta)=g(y_{1}|x,y_{2},\theta_{1})h(y_{2}|x,\theta_{2}), \quad \theta \in \Theta.
\end{equation}


Моделирование может быть основано на компоненте $g(y_{1}|x,y_{2},\theta_{1})$, не обращая внимание на $h(y_{2}|x,\theta_{2})$, если $\theta_{2}$ рассматриваются как мешающие параметры. Конечно, такое разложение не является уникальным, поэтому информационно-ограниченный подход может иметь несколько вариантов.


\begin{center}
Приведённая форма
\end{center}


Третий вариант подхода СЛОУ работает с приведенной восстановленной формой. Здесь тоже есть заинтересованность в структурных параметрах. Тем не менее, может быть удобно оценивать параметры от восстановленной формы с учетом ограничений. Во временном ряду выявленных авторегрессиях вектор привести пример.


\subsection{Стратегии идентификации}


Есть множество потенциальных способов, в которых определение ключевых параметров модели может оказаться под угрозой. Пропущенные переменные, неправильная спецификация функциональной формы, ошибки при измерении объясняющих переменных, использование нерепрезентативных данных населения, и игнорирование эндогенности объясняющих переменными. Микроэконометрика содержит много конкретных примеров того, как эти проблемы могут быть решены. Ангрист и Крюгер (2000) предоставляют комплексное обследование популярных стратегий идентификации в экономике труда, с акцентом на ПОМ. Большинство вопросов, рассматриваются в других частях книге, но кратко упоминается здесь.


\begin{center}
Exogenization
\end{center}


Данные иногда создаются в экспериментальных условиях. Идея заключается в том, что перменные могут экзогенно меняться  в некоторой выборке, в то время как в другой они остаются неизменными. Например,  минимальная заработной плата в одном государстве может измениться, а в соседнем нет. Если естественный эксперимент приближает рандомизированное воздействие, то использование таких данных для оценки структурных параметров может быть проще, чем оценка больших одновременных уравнений с эндогенными переменными воздействия. Возможно также, что воздействие переменной в естественном эксперименте можно рассматривать как экзогенные, но воздействие само по себе не является случайным.


\begin{center}
Ликвидация мешающих параметров
\end{center}


Идентификация может быть проблематична при наличии большого количества мешающих параметров. Например, в поперечном сечении регрессионной модели функция условного математического ожидания $E[y_{i}|x_{i}]$ может включать в себя отдельный конкретный фиксированный эффект $\alpha_{i}$, который коррелирует с ошибками регрессии. Этот эффект не может быть определен без большого количества наблюдений для каждого индивида (т.е. панельные данные). Тем не менее, с небольшой панелью она может быть устранена путем трансформации модели. Другим примером является наличие не зависящий от времени экзогенной переменной, которая может быть общим для групп индивидов.


\begin{center}
Контроль вмешивающихся факторов
\end{center}


Когда переменные исключены из регрессии, и когда исключенные факторы коррелируют с включенными переменными, то появляется confounding bias results. Например, в регрессии, где заработная плата является зависимой переменной, а обучение объясняющей переменной, индивидуальные способности можно рассматривать как не включённую переменную. Это означает, что потенциально коэффициент обучения не может быть определен. Одна из возможных стратегий является введение контрольных переменных в модель; это называется подходом функции управления. Эти переменные позволяют приблизить влияние не включённых переменных.Например, различные школьные достижения могут служить в качестве контроля индивидуальных способностей.



\begin{center}
Создание синтетических образцов
\end{center}


В рамках ПОМ параметр причинности может быть неизвестным из---за отсутствия необходимого сравнения или контрольной группы, что не может обеспечить оснований для оценки. Возможным решением является создание синтетического образца, который включается в группу сравнения, который являются прокси для контроля. Такой образец создается путем сопоставления (см. главу 25). Если обработанные образцы могут быть расширены с помощью хорошей контрольной группы, то идентификация причинных параметров может быть достигнута в том смысле, что параметр, связанный с ATE может быть оценён.


\begin{center}
Инструментальные переменные
\end{center}


Если идентификация находится под угрозой, потому что переменные воздействия являются эндогенными, то стандартное решение заключается в использовании инструментальных переменных. Выбор инструментальных переменных, а также интерпретация полученных результатов должна быть сделана аккуратно, потому что результаты могут быть чувствительны к выбору инструментов. Этот подход анализируется в разделах 4.8, 4.9, 6.4, 6.5 и 25.7, а также в других местах по мере необходимости.
\\
\\
\begin{center}
Reweighting Samples
\end{center}


Выводы о генеральной совокупности действительны только, если выборочные данные являются репрезентативными для неё. Проблема возникает тогда, когда выборочные данные не являются репрезентативными, и в этом случае параметры генеральной совокупности не определены. Эта проблема может быть связана с проблемой корректировкой выборки (Глава 16) или с той, которая требует перевзвешивания информации (Глава 24).













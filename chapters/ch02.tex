
\chapter{Модели}

\section{Введение}

	Микроэконометрика посвящена теориям и методам анализа микроданных отдельных лиц, домохозяйств и компаний. Более широкое определение может также включать данные на региональном и государственном уровне. Микроданных как правило либо одномоментная выборка, тогда они относятся к конкретной дате, либо панельные данные, тогда они отличаются единицами и временным интервалом. 
	
	
	Микроэконометрическая модель может быть полной спецификации распределения вероятности всех наблюдений; она также может иметь частичную спецификацию или некоторые свойства распределения, например, моменты какого---то порядка. Математическое ожидание зависимой переменной при условии регрессоров представляет особый интерес. 
	
	
	Целями микроэконометрики является как описание данных, так и нахождение причинно---следственных связей. Первая цель может быть определена достаточно широко, вторая целью, включает в себя причинно---следственные связи, которые направлены на измерение и / или эмпирическое подтверждение, опровержение гипотез и предложений в отношении микроэкономического поведения.Тип и стиль эмпирических исследований, поэтому многообразен. 
	
	
	На одном краю лежат высоко структурированные модели, полученные из подробной спецификации экономического поведения, которая анализирует причинные (поведенческий) или структурные отношения для взаимозависимых микроэкономических переменных. На другом конце сводятся формы исследования, которые направлены на то, чтобы выявить корреляцию между переменными, не обязательно полагаясь на детальную спецификацию всех соответствующих взаимозависимостей. Оба подхода имеют общую цель в раскрытии важных и стойких отношений, которые могут быть полезны для понимания микроэкономического поведения, но они отличаются по степени, в которой они опираются на экономическую теорию. 
	

	Как отдельная дисциплина микроэконометрика новее, чем макроэконометрика , которая занимается моделированием рыночных и агрегированных данных. Многое из ранних работ в прикладной эконометрики были основаны на агрегированных временных рядах, собранных государственными учреждениями. Большая часть в ранних работ по статистическому анализу спроса вплоть до приблизительно 1940 года использовались рыночные данные, а не индивидуальные или данные домашних хозяйств (Хендри и Morgan , 1996). Книга Моргана в 1990 по истории эконометрических идей не делает ссылок на микроэконометрические работы до 1940 года, но с одним важным исключением, это исключение составляет работа по данным семейного бюджета. Это привело к сбору данных семейного бюджета, которые являлись исследовательсим ресурсом для некоторых более ранних микроэкономистов, таких как Аллена и Боули (1935). Однако лишь в 1950 году микроэконометрика стала признанной и отдельной дисциплиной. 
	
		
	После вручения Нобелевской премии по экономике в 2000 году Джеймсу Хекману и Даниэлу Макфаддену за их вклад в микроэконометрику, данная область добилась четкого признания в качестве отдельного раздела науки. Хекмана вручили премию «за развитие теории и методов анализа селективных данных», а Макфаддену «за развитие теории и методов анализа дискретного выбора. 
	
		
	Применение микроэконометрического метода можно найти не только в областях микроэкономики, но и в других родственных социальных наук, таких как политология, социология, и география. 
	
	
	Начиная с 1970---х и особенно в течение последних двух десятилетий были совершенны революционные достижения в анализу больших объемов данных, более того это позволило значительно расширить сферу микроэконометрики. В результате, хотя эмпирический анализ спроса продолжает оставаться одной из наиболее важных областей применения микроэконометрических методов, его стиль и содержание были существенно изменены под влиянием новых методов и моделей. Кроме того, эти модели встречаются в приложениях экономического развития, финансов, здравоохранения, промышленной организации, трудовой и государственной экономики, и прикладной микроэкономики. 
	
	
	Основное внимание в этой книге сконцентрировано на новые достижения, которые появились в последние три десятилетия. Нашей целью является обзор концепций, моделей и методов, которые мы рассматриваем в качестве стандартных компонентов набора инструментов современного исследователя. Конечно, понятие стандартные методы и модели являются субъективными, но в книге будут встречаться более продвинутые темы, которые уже являются объективными. 
	
		
	Микроэконометрика фокусируется на нелинейных моделях и получению оценки, которое может быть предоставлена в структурной интерпретации. Большая часть этой книги, особенно Части 2---4, рассматривает методы для нелинейных моделей. Эти нелинейные методы пересекаются со многими областями прикладной статистики, включая биостатистику. В отличие от этого, отличительной особенностью эконометрики является акцент на причинное моделирование. В этой главе представлены основные понятия, связанные с причинным моделированием, концепции, которые уместны в линейных и нелинейных моделях. 
	
	
	Разделы 2.2 и 2.3 посвящены ключевым понятиям структурности и экзогенности. Раздел 2.4 использует линейную систему одновременных уравнений для иллюстрации структурной модели и связывает его с другими важными понятиями, такими как сокращенная модель. Идентификация определений даны в разделе 2.5. Раздел 2.6 рассматривает уравнения структурных моделей. Раздел 2.7 рассматривает полученные результаты моделей и сравнивает их между собой. Раздел 2.8 содержит краткое обсуждение моделирования и оценки стратегий предназначена для обработки данных и вычислительных задач.
	
	
\section{Структурные модели}

Структура состоит из:
\begin{enumerate}
\item набор переменных $W$ разделённых для удобства на $[Y X]$
\item функция распределения вероятности для $W$, $F(W)$
\item априорное упорядочение $W$ в соответствии с гипотетическими причинно---следственными связями и спецификация априорных ограничений на гипотетическую модель; к тому же
\item параметрическая, полупараметрическая и непараметрическая спецификация функциональной формы и ограничений модели.
\end{enumerate}


Sargan (1988, p. 27) утверждает: \\
Модель это набор спецификаций функции вероятности для набора наблюдений. Структура это распределение параметров данной спецификации. Следовательно, структура это модель в которой параметрам задаются числовые значения. \\
	Рассмотрим случай, когда цель моделирования заключается в объяснении наблюдаемых значений векторной переменной $y$, $y'= (y_{1}, ..., y_{G})$. Каждый элемент $y$ является функцией некоторых других элементов $y$ и объясняющих переменных $z$ и случайных ошибок $u$. Обратите внимание, что переменные $y$ считаются взаимозависимыми. В отличие от этого, взаимозависимость между $z_{i}$ не моделируется, более того $i$---ые наблюдения удовлетворяют набору неявных уравнений:
	
\begin{equation}
g(y_{i},z_{i},u_{i}|\theta)=0
\end{equation}

где $g$ известная функция. Мы называем такой тип уравнений структурной модель, а $\theta$ структурным параметром. Это соответствует свойству 4, который приведён в начале раздела. 


Предположим, что существует единственное решение $y_{i}$ для каждого $(z_{i},u_{i})$. Тогда мы можем записать уравнения в явном виде с $y$, как функция $(z, u)$:

\begin{equation}
y_{i}=f(z_{i},u_{i}|\pi).
\end{equation}


	Это уравнение приводит к сокращённой форме структурной модели, где $\pi$ вектор параметров сокращенной формы, которые являются функцией от $\theta$. Приведенная форма получается путем решения структурной модели эндогенных переменных $y_{i}$ для заданных $(z_{i},u_{i})$. 


	Если целью моделирования является получение информации об элементах $\theta$, то (2.1) обеспечивает прямой путь, к тому же это включает в себя оценку структурной модели. Однако, поскольку элементы из $\pi$ являются функциями $\theta$, (2.2) также предоставляет косвенные пути к выводу на $\theta$. Если $f(z_{i},u_{i}|\pi)$ имеет известную функциональную форму, и если она аддитивно---сепарабельна по $z_{i}$ и $u_{i}$, то мы можем записать:
	
	
	
\begin{equation}
y_{i}=g(z_{i}|\pi)+u_{i}=E[y_{i}|z_{i}]+u_{i},
\end{equation}

к тому же регрессия $y$ на $z$ является естественной прогнозной функцией $y$ для заданного $z$. В этом смысле приведенная форма уравнения имеет полезную роль для создания условных предсказаний $y_{i}$ при заданных $(z_{i},u_{i})$. Чтобы получить прогнозы объсняемых переменных для заданных значений объясняющих переменных (2.2) требуется оценить $\pi$, которое вычисляется достаточно просто.
	Важным дополнение к (2.3) является трансформационная модель, которая для скалярных значений принимает следующий вид:
	
\begin{equation}
\Lambda(y)=z'\pi+u,
\end{equation}

где $\Lambda(y) $ является функцией преобразования (например,  $\Lambda(y)=ln(y)$ или $\Lambda(y)=y^{1/2}$). В некоторых случаях функция преобразования может зависеть от неизвестных параметров. Трансформационная модель отличается от регрессионной, но она также может быть использована для оценки $E[y|x]$ (см. Глава 17). 


	Одним из наиболее важных и спорных этапов в спецификации структурной модели является свойство 3, в котором априори переменные находятся в порядке причинно---следственных значений. Поэтому необходимо проводить различия между теми переменными, чьи вариации можно объяснить в модели и теми, чьи вариации определяется внешними факторами и, следовательно, выходят за рамки исследования. В микроэконометрики такими примерами являются количество лет обучения в школе и продолжительность рабочего времени; примерами последнего являются пол, этническая принадлежность, возраст, и аналогичные демографические показатели. Более того, $y$ считается эндогенными, а $z$ называется экзогенными переменными. 
	
	
	Экзогенность переменных является важным упрощением и оно требует более формального определения, которое мы сейчас предоставим. 

	
\section{Экзогенность}

	
Мы начнем с рассмотрения общего случая конечномерного параметрического представления, в котором совместное распределение $W$, с параметрами $\theta$ разбит на $\theta_{1},\theta_{2}$, это учитывается в условная плотности распределения $Y$ при заданном $Z$, и в маржинальном распределении $Z$:


\begin{equation}
f_{J}(W|\theta)=f_{C}(Y|Z,\theta)\times f_{M}(Z|\theta).
\end{equation}

специальный случай при этом возникакет, когда:

\[
f_{J}(W|\theta)=f_{C}(Y|Z,\theta_{1})\times f_{M}(Z|\theta_{2}),
\]

где $ \theta_{1} $ и $ \theta_{2} $ функционально независимы. Тогда говорят, что $Z$ является экзогенной по отношению к  $ \theta_{1} $, а это означает, что знание $ f_{M}(Z|\theta_{2}) $  не требуется для вывода $ \theta_{1} $ и, следовательно, мы можем поставить условие $Z$ на распределение $Y$. 


	В моделях всегда можно избавиться от параметров. Поэтому далее рассмотрим случай, в котором модель депараметризированна с точки зрения параметров $\varphi$, который является взаимно---однозначным преобразованием $\theta$, скажем $\varphi=H(\theta)$, где $\varphi$ разбивается на $(\varphi_{1},\varphi_{2})$. В таком случае условие экзогенность выглядит следующим образом:
	
\begin{equation}
f_{J}(W|\varphi)=f_{C}(Y|Z,\varphi_{1})\times f_{M}(Z|\varphi_{2}),
\end{equation}

где $\varphi_{1}$ и $\varphi_{2}$ независимы.


Наконец, рассмотрим случай, когда параметр $\lambda$ функцией от $\varphi$, скажем $h(\varphi)$. Тогда для экзогенности $Z$ относительно $\lambda$, нам нужны два условия: (i) $\lambda$ зависит только от $ \varphi_{1}) $, т.е. $\lambda=h(\varphi)$, и (ii) $\varphi_{1}$ и $\varphi_{2}$ не имеют перекрестных ограничений, т. е. $(\varphi_{1}, \varphi_{2}) \in \Phi_{1}\times\Phi_{2} = \lbrace\varphi_{1} \in \Phi_{1}, \varphi_{2} \in \Phi_{2}\rbrace$. \\
Разложение в (2.5) --- (2.6) играет важную роль в развитии концепции экзогенности. В этой книге акцентруется внимание на три понятия экзогенности: (1) слабая экзогенность; (2) невзаимосвязаность по Грейнджеру; (3) сильная экзогенность.

{\bf Определение 2.1 (слабая экзогенность):}  $Z$ слабо экзогенная по $\lambda$, если условия (i) и (ii) выполнены.\\
	Если параметры модели неинформативны для вывода $\lambda$, то вывод $\lambda $  может быть выполнен только на основе условного распределения $f(Y|Z,\varphi_{1})$. Более того, нет причин полагать, что нет никакой статистической модели $Z$, однако параметры этой модели не играют никакой роли в выводе $\varphi$, и, следовательно, не имеют значения. \\


\subsection{Условная независимость}


Первоначально концепция причинности по Грейнджеру была определена в контексте предсказания в виде временного ряда. В целом, это может быть интерпретировано как форма условной независимости (Holland, 1986, с. 957). 


	Разделяя $z$ на два подмножества $z_{1}$ и $z_{2}$, предположим, что $W=[y,z_{1},z_{2}]$ матрица переменных. Таким образом $z_{1}$ и $y$ условно независимы при заданном $z_{2}$ если

\begin{equation}
f(y|z_{1},z_{2})=f(y|z_{2})
\end{equation}

Это предположение сильнее, чем:
 
\begin{equation}
E(y|z_{1},z_{2})=E(y|z_{2})
\end{equation}

поэтому $z_{1}$ не имеет предсказательной силы на $y$, что означает, что $z_{1}$ не взаимосвязано по Грейнджеру с $y$.

{\bf Определение 2.2 (сильная экзогенность):}  $z_{1}$ сильно экзогенна по $\varphi$, если она слабо экзогенна по $\varphi$ и выполнено (2.8).

\subsection{Экзогенные переменные}


	Экзогенность является сильным предположение. Это свойство случайных переменных относительно параметров, представляющих интерес. Поэтому переменная может обоснованно восприниматься экзогенной в одной структурной модели, а в другом нет. Необдуманное введение этого предположения будет иметь некоторые нежелательные последствия, которые будут обсуждаться в разделе 2.4. 
	
	
	Предположение об экзогенности может быть оправдано теорией, в этом случае оно является частью подтверждения гипотез в модели; в разделе 8.4.3 будут рассмотрены способы проверки на экзогенность. В анализе одномоментной выборки  данное предположение может быть оправдано как следствие естественного эксперимента или квази---эксперимента, в котором значение переменной определяется внешним вмешательством, например, правительство или регулирующий орган может выбирать ставку налога. Особый интерес представляет случай, когда внешнее вмешательство приводит к изменению значения некоторых переменных. Такой естественный эксперимент равносилен экзогенности некоторых переменных. Как мы увидим в главе 3, это создает квазиэкспериментальные возможности изучаения влияния переменной при отсутствии других осложняющих факторов. 


\section{Модель линейных одновременных уравнений}


Важным частным случаем общей структурной модели указанной в (2.1) является система линейных одновременных уравнений.В данном разделе предлагается краткой и выборочное решение данной модели, см. также раздел 6.9.6. Цель состоит в том, чтобы обсудить нескольких ключевых идей и концепций, которые имеют более общий характер. Хотя анализ ограничен линейными моделями, много интересных идей обычно применяется к нелинейным моделям.
  
\subsection{Система}

\[
\begin{aligned}
y_{1i}\beta_{11}+ &\dots + y_{Gi}\beta_{1G}+z_{1i}\gamma_{11} + \dots + &z_{Ki}\gamma_{1K}=  &u_{1i} \\
                  &\vdots                                               &\vdots           =  &\vdots \\
y_{1i}\beta_{G1}+ &\dots + y_{Gi}\beta_{GG}+z_{1i}\gamma_{G1} + \dots + &z_{Ki}\gamma_{GK}=  &u_{Gi},
\end{aligned}
\]

Четкое различие  сделан между эндогенными переменными, $y'_{i}=(y_{1i},\dots,y_{Gi})$, и экзогенными переменными, $z'_{i}=(z_{1i},\dots,z_{Ki})$. По определению экзогенные переменные не коррелируются с случайными ошибками $u_{1i},\dots,u_{Gi}$. В своей неограниченной форме каждая переменная входит в каждое уравнение. 

В матричной форме система выглядит следующим образом:

\begin{equation}
y'_{i}B+z'_{i}\Gamma=u'_{i},
\end{equation}

где $y'_{i}$,$B$,$z'_{i}$,$\Gamma$ и $u'_{i}$ имеют размерности $G \times 1$ , $G \times G$ , $ K \times 1$, $K \times G$ и $G \times 1$, соответственно. 

Стандартные предположения для СЛОУ (система линейных одновременных уравнений) следующие:

\begin{enumerate}
\item $B$ не сингулярна и имеет ранг $G$
\item $rank Z = K$, При этом $N \times K$ матрица $Z$ полученная из переменных $z'_{i}$, $i=1,\dots,N$.
\item $plim N^{-1}Z'Z=\Sigma_{ZZ}$ --- симметричная положительно определенная матрица $K \times K$.
\item $u_{i}\sim N[0,\Sigma]$, так что $E[u_{i}]=0$ и $E[u_{i}u'_{i}]=\Sigma=[\sigma_{ij}]$, где $\Sigma$ симметричная положительно определенная матрица $G \times G$.
\item Ошибки в каждом уравнении независимы друг от друга.
\end{enumerate}


В этой модели структурные параметры состоят из $(B,\Gamma,\Sigma)$. Записав

\[
Y=\begin{bmatrix} y'_{1} \\ \vdots \\ y'_{N} \end{bmatrix}, \quad  Z=\begin{bmatrix} z'_{1} \\ \vdots \\ z'_{N} \end{bmatrix}, \quad U=\begin{bmatrix} u'_{1} \\ \vdots \\ u'_{N} \end{bmatrix}
\]

можно представить структурную модель в более компактной записи:

\begin{equation}
YB+Z\Gamma=U,
\end{equation}

где $Y$,$B$, $Z$,$\Gamma$ и $U$ имеют размерности $N \times G$ , $G \times G$ , $ N \times K$, $K \times G$ и $N \times G$, соответственно. В таком случае, решение для всех эндогенных переменных в терминах экзогенных выглядит следующим образом:


\begin{equation}
Y+Z\Gamma B^{-1}=UB^{-1},
Y=Z\Pi+V,
\end{equation}

где $\Pi= -\Gamma B^{-1}$, с учётом предпосылки 4, $v_{i} \sim N[0,B^{-1^{'}} \Sigma B^{-1}]$.


	В рамках СЛОУ структурная модель  по нескольким причинам. Во---первых, сами уравнения имеют экономическую интерпретацию, такие как спроса или предложения, производственные функции, и так далее, и на них распространяются ограничения экономической теории. Следовательно, $B$ и $\Gamma$ являются параметрами, описывающие экономическое поведение. Следовательно априорная теория может быть использована для формирования гипотез относительно знака и размер отдельных коэффициентов. С другой стороны, неограниченная форма приведенных параметров является сложной функцией структурных параметров, и поэтому её трудно оценить.  

	
	Рассмотрим, без ограничения общности, первое уравнение в модели (2.11), с зависимой переменной $y_{i}$. Кроме того, некоторые из оставшихся $G-1$ эндогенных переменных и $K-1$ экзогенных переменных могут быть исключены их этого уравнения. Из (2.12) мы видим, что в целом эндогенные переменные $Y$ зависят стохастически от $V$, который в свою очередь является функцией структурных ошибок $U$, поэтому, в общем $plim N^{-1}Y'U\not=0$. Как правило, применение метода наименьших квадратов в системах линейных одновременных уравнений дает несовместимые оценоки. Это хорошо известный и основной результат в литература про системы одновременных уравнений, что часто называют проблемой смещения. Большая часть литературы по одновременным уравнениям занимается идентификацией и последовательной оценкой при методе наименьших квадратов см. Сарган (1988) и Шмидт (1976), а также раздел 6.9.6. 
	
	
	Сокращенная форма СЛОУ выражает каждую эндогенную переменную в виде линейной функции всех экзогенных переменных и всех структурных ошибок; (ошибки в сокращенной форме являются линейными комбинациями ошибок структурной формы). Из сокращенной формы для $i$---го наблюдения получаем:

\begin{equation}
E[y_{i}|z_{i}]=z'_{i}\Pi,
\end{equation}
\begin{equation}
V[y_{i}|z_{i}]=\Omega=B^{-1^{'}} \Sigma B^{-1}
\end{equation}
 
 
Параметры сокращенной формы $\Pi$ являются производными параметрами, определенными в зависимости от структурных параметров. Если $\Pi$  можно последовательно оценить, то приведенная форма может быть использована, чтобы сделать прогноз об изменениях в $Y$ под влиянием внешних изменений в $Z$; это возможно, даже если $B$ и $\Gamma$ не известны. Учитывая экзогенность $Z$, полный набор приведенной формы является многомерной регрессионной моделью, которая может быть оценена методом наименьших квадратов. Сокращенная форма обеспечивает основу для принятия условных предсказаний $Y$ при заданных $Z$. 


	Ограниченная приведенная форма является неограниченной приведенной формой модели при заданных ограничениях. Если эти ограничения совпадают с ограничениями структуры, то структурная информация может быть извлечена из приведенной формы. 
	
	
	В рамках СЛОУ, неизвестные структурные параметры, ненулевые элементы $B$, $\Gamma$,$\Sigma$, играют ключевую роль, потому что они отражают причинную структуру модели. Взаимозависимость между эндогенными переменными описывается $B$, а реакция эндогенных переменных на экзогенные шоки в $Z$ отражается в параметрах матрицы $\Gamma$. Поэтому нас интересуют те параметры, которые измеряют предельное влияние изменения независимой переменной, $y_{j}$ или $z_{k}$ на результаты $y_{l}$, где $l\not=j$, функцию этих параметров и данных. Элементы $\Sigma$ описывают дисперсию и свойств зависимости случайных ошибок, следовательно, они измеряют некоторые свойства генерирования данных. 
	

\subsection{Причинная интерпретация в СЛОУ}


Простой пример может иллюстрировать причинную интерпретацию параметров в СЛОУ. Структурная модель имеет две непрерывные эндогенные переменные $y_{1}$ и $y_{2}$, одну непрерывную экзогенную переменную $z_{1}$, одно стохастическое соотношение, связывающее $y_{1}$ и $y_{2}$ и одно уравнение, связывающее все три переменные в модели:


\[
y_{1}=\gamma_{1}+\beta_{1}y_{2}+u_{1}, \quad 0<\beta_{1}<1, \\
y_{2}=y_{1}+z_{1}.
\]


В этой модели $u_{1}$ является стохастической ошибкой, независима от $z_{1}$, с четко определенным распределением. Параметра $\beta_{1}$ имеет ограничение, которое также является частью спецификации модели. Переменная $z_{1}$ является экзогенной и, следовательно, ее изменение индуцируется внешними источниками, которые мы можем рассматривать как вмешательства, они оказывают непосредственное влияние на $y_{2}$. Это влияние измеряется при помощи приведенной формы модели, которая выглядит следующим образом: 

\[
y_{1}=\frac{\gamma_{1}}{1-\beta_{1}}+\frac{\beta_{1}}{1-\beta_{1}}z_{1}+\frac{1}{1-\beta_{1}}u_{1} \\
= E[y_{1}|z_{1}]+v_{1}, \\
\]

\[
y_{2}=\frac{\gamma_{1}}{1-\beta_{1}}+\frac{1}{1-\beta_{1}}z_{1}+\frac{1}{1-\beta_{1}}u_{1} \\
= E[y_{2}|z_{1}]+v_{1},
\]

где $v_{1}=\frac{u_{1}}{1-\beta_{1}}$. Коэффициенты приведённой формы $\frac{\beta_{1}}{1-\beta_{1}}$ и $\frac{1}{1-\beta_{1}}$  имеют причинную интерпретацию: любые внешние индуцированных изменения $z_{1}$ вызовет количественное изменение $y_{1}$ и $y_{2}$ . Следует отметить, что в этой модели $y_{1}$ и $y_{2}$ также реагируют на изменения $u_{1}$, поэтому, чтобы не смешать влияние двух источников изменчивости эндогенных переменных, потребуем, чтобы $z_{1}$ и $u_{1}$ являлись независимыми. \\
	К тому же

\[
\frac{\partial y_{1}}{\partial y_{2}}=\beta_{1}=\frac{\beta_{1}}{1-\beta_{1}}\div\frac{1}{1-\beta_{1}} \\
=\frac{\partial y_{1}}{\partial z_{1}}\div\frac{\partial y_{2}}{\partial z_{1}}.
\]


В каком смысле $\beta_{1}$ измеряет причинно---следственное влияние на $y_{1}$ и $y_{2}$? Чтобы увидеть возможные трудности, заметим, что $y_{1}$ и $y_{2}$ являются взаимозависимыми или совместно определены, так что неясно, в каком смысле $y_{2}$ <<влияет на>> $y_{1}$. Хотя $z_{1}$ (и $u_{1}$) являются причиной изменение в приведённой форме, $y_{2}$ промежуточно влияет на $y_{1}$. То есть, первое структурное уравнение представляет собой срез влиянии на $y_{2}$ на  $y_{1}$, в то время как приведённая форма дает равновесное воздействие после учета всех взаимодействий  между эндогенными переменными. В рамках СЛОУ даже эндогенные переменные рассматриваются как причинные переменные и их коэффициенты как причинные параметров. Такой подход может вызвать недоумение для тех, кто видит причинность в экспериментальных условиях, когда независимые источники изменения являются причинными переменными. Подход СЛОУ имеет смысл, если $y_{2}$ имеет независимый и экзогенный источник изменений, который в этой модели представлен как $z_{1}$. Поэтому предельный коэффициент $\beta_{1}$ является мерой того, как $y_{1}$ и $y_{2}$ реагируют на изменение $z_{1}$. 


	В целом, мы можем задаться вопросом: при каких условиях параметры СЛОУ имеют значимую причинную интерпретацию. Мы вернемся к этому вопросу при обсуждении понятия идентификации в разделе 2.5.


\subsection{Нелинейный модели и модели скрытых параметров}


Если одновременная модель нелинейна по параметрам, то структурную модель можно записать в виде:



\begin{equation}
YB(\theta)+Z\Gamma(\theta)=U,
\end{equation}

где $B(\theta)$ и $\Gamma(\theta)$ матрицы, элементами которых являются функции от структурных параметров $\theta$. Приведенная форма может быть получена, как раньше. 


Во многих микроэконометрических моделях существуют скрытые или ненаблюдаемые переменных, также как и эндогенные переменные. Например, модель поиска и подбора  использует понятие минимального уровня оплаты труда или резервной цены, модель выбора --- косвенную функцию полезности, и так далее. В случае таких моделей структурная модель (2.1), может быть заменена на

\begin{equation}
g(y^{\ast}_{i},z_{i},u_{i}|\theta)=0,
\end{equation}

где скрытая переменная $y^{\ast}_{i}$ заменяет $y_{i}$. Соответствующая сокращенная форма разрешима для $y^{\ast}_{i}$ через $(z_{i},u_{i})$

\begin{equation}
y^{\ast}_{i}=f(z_{i},u_{i}|\pi).
\end{equation}


Эта сокращенная форма бессмысленна, если $y^{\ast}_{i}$ не наблюдаются полностью. Тем не менее, если у нас есть функция $y_{i}=h(y^{\ast}_{i})$, которые соотносят наблюдаемые скрытые переменные с аналогами $y_{i}$, то сокращенная форма с точки зрения наблюдаемых переменных выглядит так:

\begin{equation}
y_{i}=h(f(z_{i},u_{i}|\pi)).
\end{equation}
подробнее см. раздел 16.8.2


Когда структурную модель включает нелинейность в переменных или, когда скрытые переменные участвуют в выводе функциональной формы, то приведённую форму трудно получить. В таких случаях используется аппроксимация. Для удобства, конкретная функциональная форма может быть использована для связи эндогенной переменной со всеми экзогенными переменными.



\subsection{Интерпретация структурных отношений}


Маршак (1953, стр. 26) в влиятельной статье дал следующее определение структуры: \\
Структура определяется как набор условий, которые не меняются в то время, когда были сделаны наблюдения, но которые могут измениться в будущем. Если указанное изменение структуры ожидаемо или прогнозируемо, то необходимо  некоторое знания прошлого структуры...В экономике, условия, которые соотносятся с структурой являются (1) наборами соотношений, описывающих поведение человека или учреждения, а также законы с ненаблюдаемыми случайными возмущениями и ненаблюдаемыми случайными ошибками измерения, (2) совместным распределением вероятностей этих случайных величин. 


	Маршак утверждал, что структура имеет большое значение для количественной оценки или тестирования экономической теории. 
	
	
	В литературе про СЛОУ структурная модель относится к «автономным» отношениям. Есть и другие тесно связанные понятия структуры. Одной из таких концепций является <<deep parameters>>, под которыми подразумеваются параметры не зависящие от внешних вмешательств. 
	
	
	В последние годы альтернативное использование термина структура стало, то, которое относится к эконометрическим моделям, основанным на гипотезе динамической стохастической оптимизации рациональными агентами. При таком подходе отправной точкой для любой способа оценивания структурной модели  является необходимых условий первого порядка, которое определяют оптимизационное поведение агента. Например, в стандартном случае  решения проблемы максимизации функции полезности при ограничениях используется условие первого порядка, а за поведенческое поведение берут предельную полезность. Если соответствующие функциональные формы явно указаны, и вводятся стохастические ошибки оптимизации, то условия первого порядка определяют поведенческие модели, параметры которого характеризуют функции полезности --- так называемые <<deep>> параметры; примеры рассмотрены в разделах 6.2.7 и 16.8.1. 


	Укажем две особенности этого весьма структурного подхода. Во-первых, он весьма серьёзным образом полагается на априорные экономические теории. Экономическая теория не используется просто для создания списка соответствующих переменных, которые можно использовать в более или менее произвольно заданной функциональной форме. Скорее,  экономическая теория имеет большую (но не исключительную) роль в спецификации, оценках и выводе. Второй особенностью является то, что идентификация, спецификация и оценка полученной модели может быть очень сложной, потому что сама задача оптимизации агента является потенциально сложной, особенно если это динамическая оптимизация в условиях неопределенности, с наличием дискретности и разрывов в данных, см. Rust (1994) .


\section{Идентификация}


Цель подхода СЛОУ заключается в последовательной оценке $(B,\Gamma,\Sigma)$ и статистических выводах. Важной предпосылкой для последовательной оценки является то, что модель должна быть определена. Мы кратко обсудим две важные  взаимосвязанные концепции наблюдательной эквивалентности и идентифицируемости в контексте параметрических моделей.


	Идентификация связана с определением параметров при достаточном количестве наблюдений. В этом смысле она является асимптотической концепцией. Статистическая неопределенность неизбежно затрагивает любого выводы на основе конечного числа наблюдений.  Идентификация является одним из основных понятий и логически происходит до и после статистической оценки. Во большинстве эконометрической литературы под идентификацией подразумевается идентификация точек. Мы также акцентируем своё внимание на этом. Тем не менее, идентификации данных или границ, также является важным подходом, который будет использоваться в отдельных местах этой книги (например , главы 25 и 27 , см. Манский , 1995).


{\bf Определение 2.3 (наблюдательная эквивалентность):} Две структуры модели определяются как совместная функция распределения вероятностей $Pr[x|\theta]$,$x\in W$, $\theta \in \Theta$ наблюдаемо эквивалентны, если $Pr[x|\theta^{1}]=Pr[x|\theta^{2}] \forall x \in W.$ 


Менее формально, если, с учетом данных, если две структурно идентичные модели подразумевают совместные функции   распределения вероятностей, то две структуры наблюдательно эквивалентны. Существование множества экспериментально эквивалентных структур подразумевает отказ от идентификации.

{\bf Определение 2.4 (идентификация):} Структура $\theta^{0}$ идентифицирована, если не существует других наблюдаемых схожих параметров в $\Theta$.


	Простой пример не идентификации происходит, когда есть коллинеарность между регрессорами в линейной регрессии $y=X\beta+u$. Тогда мы можем определить линейные комбинации $C\beta$, где $rank[C] < rank[\beta]$, но мы не можем идентифицировать $\beta$.
	
	
	Это определение относится к уникальности структуры. В контексте СЛОУ это определение означает, что существует единственная тройка $(B,\Gamma,\Sigma)$ в соответствии с данными. В СЛОУ, как и в других случаях, идентификация включает в себя возможность получить уникальные оценки структурных параметров. Например, в случае приведенной формы (2.12), при указанных предположениях, метод наименьших квадратов предоставляет уникальные оценки $\Pi$, то есть $\hat{\Pi}=[Z'Z^{-1}]Z'Y$, а также определяет $(B,\Gamma)$ так, что есть решение для неизвестных элементов $B$ и $\Gamma$ из уравнений $\Pi+\Gamma B^{-1}=0$, с учетом априорных ограничений на модели. Уникальное решение подразумевает только идентификацию модели.


	Полная модель будет идентифицирована, если все параметры модели определены. Вполне возможно, что в некоторых моделях только подмножество параметров выявлено. В некоторых случаях важно иметь возможность определить некоторые функции параметров, и не обязательно все индивидуальные параметры. Определение функции параметров означает, что функция может быть однозначно восстановлена от $F(W|\Theta)$.


	Как можно убедиться, структуры альтернативной спецификации модели могут быть <<исключены>>? В СЛОУ решение этой проблемы зависит от увеличения информации априорных ограничений на $(B,\Gamma,\Sigma)$. 
	
	
	О необходимости априорных ограничений свидетельствует следующее рассуждение. Прежде всего заметим, что при предположении раздела 2.4.1, сокращенная форма, определяемая формулой $(\Pi,\Omega)$, всегда уникальна. Изначально предполагается, что нет никаких ограничений на $(B,\Gamma,\Sigma)$. Далее предположим, что существует два наблюдательно эквивалентных структур $(B_{1},\Gamma_{1},\Sigma_{1})$ и $(B_{2},\Gamma_{2},\Sigma_{2})$. Тогда
	
\begin{equation}
\Pi=-\Gamma_{1} B^{-1}_{1}=-\Gamma_{2} B^{-1}_{2}.
\end{equation}


Допустим $H$ несингулярная матрица $G \times G$, тогда
$\Gamma_{1} B^{-1}_{1}=\Gamma_{1} H H^{-1} B^{-1}_{1}=\Gamma_{2} B^{-1}_{2}$, что означает, что $\Gamma_{2}=\Gamma_{1}H$, $B_{2}=B_{1}H$. Таким образом, вторая структура является линейной трансформацией первой.


	Решение СЛОУ для этой проблемы заключается в введение ограничений на $(B,\Gamma,\Sigma)$ таких, что мы можем исключить существование линейных преобразований, которые приводят к наблюдательно эквивалентным структурам. Другими словами, ограничения на $(B,\Gamma,\Sigma)$, должны быть такими, чтобы не существовало матрицы $H$, которая даёт другую структуру с такой же приведённой формой; с учётом $(\Pi,\Omega)$ будет существовать единственные решения уравнений $\Pi=-\Gamma B^{-1}$ и $\Omega=(B^{-1})'\Sigma B^{-1}$.
	
	
	На практике могут быть наложены различные ограничения в том числе (1) нормализации, такие как диагональные элементы матрицы $B$ должны быть равны 1, (2) линейных однородность и неоднородные ограничения, и (3) ковариация и неравенство ограничений. Подробная информация о необходимых и достаточных условиях для идентификации в линейных и нелинейных моделей можно найти во многих текстах, включая Сарган (1988).
	
	
	Исключение ограничений по существу утверждают, что модель содержит некоторые переменные, которые имеют нулевое влияние на некоторые эндогенные переменные. То есть, определенные направления причинности априори невозможны. Это дает возможность определить другие направления причинности. Например, в простом случае с двумя переменными, приведенном выше, $z_{1}$ не входил в уравнение $y_{1}$, позволяющее определить прямое воздействие $y_{2}$ на $y_{1}$.
	
	
	Если нет никаких ограничений на $\Sigma$, а диагональные элементы $B$ нормированы на 1, то необходимым условием для идентификации является условие порядка, в котором говорится, что число исключенных экзогенных переменных должно по крайней мере равняться количеству включенных эндогенных переменных. Достаточным условием является условие ранга, которое обеспечивает для $i$---го уравнения $\Pi \Gamma_{j}=-B_{j}$ уникальное решение для 
$\Gamma_{j},B_{j}$ при заданном $\Pi$.
	Идентификация в нелинейных СЛОУ обсуждалась в Сарган (1988), который также дает ссылки на более ранние работы.
	
\section{Модель с одним уравнением}


Без ограничения общности рассмотрим первое уравнение СЛОУ при нормализации $\beta_{11}=1$. Пусть $y=y_{1}$, а $y_{1}$ обозначают эндогенные компоненты $y$ отличные от $y_{1}$, и пусть $z_{1}$ обозначают экзогенные компоненты $z$


\begin{equation}
y=y'_{1}\alpha+z'_{1}\gamma+u.
\end{equation}

Во многих исследованиях пропускаются формальные шаги при переходе от системы уравнений к одному уравнению и начинают с написания уравнения регрессии

\[
y=x'\beta+u,
\]

где некоторые компоненты $x$ являются эндогенными (неявно $y_{1}$), а другие экзогенными (неявно $z_{1}$).
Основной упор мы делаем на оценке влияния изменений в ключевых регрессорах, которые могут быть эндогенными или экзогенными, в зависимости от предположений. Инструментальные переменные или двухшаговый метод наименьших квадратов является наиболее привлекательным способом оценивания (см. разделы 4.6, 6.4 и 6.5).


	В подходе СЛОУ естественно указывать по крайней мере некоторые из оставшихся уравнений в модели, даже если они не являются предметом исследования. Пусть $y_{1}$ имеет размерность 1. Тогда первая возможность такова, что можно указать структурное уравнение для $y_{1}$ и других эндогенных переменных, которые могут появиться в этом структурные уравнения. Вторая возможность заключается в определении приведенного уравнения форму для $y_{1}$. Это также покажет экзогенные переменные, которые влияют на $y_{1}$, но не влияют напрямую на $y$. Преимуществом является то, что в таких условиях инструментальные переменные возникают естественным путем. 
	
	
\section{Потенциальные результаты модели}


Необходимость причинного вывода в эконометрических моделях особенно сильна, когда акцент делается на влияние государственной политики и / или частных решений на некоторый конкретный результат. Примерами могут служить,  влияния социальных трансфертов на предложение труда, влияние количества учеников в классе на обучение студентов, а также влияние медицинского страхования на пользу здравоохранения. Во многих случаях причинные переменные сами отражают индивидуальные решения и, следовательно, потенциально эндогенны. Когда, как это обычно бывает, эконометрические оценки и выводы основаны на данных наблюдений, определения и выводы о причинности параметров несут в себе немало проблем. Таким образом, для реализации причинного моделирования лучше использовать данные, полученные естественным экспериментом или в квази---экспериментальных условиях. В раздел 3.4 обсуждаются плюсы и минусы таких  данных, но для наших целей необходимо понимать, что это условия, в которых некоторые причинные переменные изменяются экзогенно и независимо от других независимых переменных, что позволяет относительно легче  выявить причинно---следственные параметры.
	
	
	Одним из основных препятствий для моделирования причинности является фундаментальная проблема причинного вывода (Holland , 1986). Пусть $X$ гипотетическая причина, $Y$ результат. Манипулируя значением $X$ мы можем изменить значение $Y$. Предположим, что значение $X$ изменяется от $x_{1}$ до $x_{2}$. Мера причинного воздействия изменений $Y$ формируется путем сравнения двух значений $Y$ : $y_{2}$, которая является результатом изменений, и $y_{1}$, в случае когда не было бы никаких изменений в $x$. Однако, если бы $X$ изменился, то значение $Y$ , в отсутствие изменений не наблюдалось бы . Следовательно, ничего больше нельзя сказать о причинном влиянии без некоторых гипотеза о том, какое значение предположительно имело бы $Y$, в отсутствие изменений в $X$. Иными словами, все причинные выводы должны быть сделаны исходя из сравнения фактического и контрфактического результата. В обычной эконометрической модели (например СЛОУ) нет необходимости чётко формулировать контрфактический результат.
	
	
	Относительно новые нити в микроэконометрической литературе --- программное оценивание или treatment evaluation --- обеспечивает статистическую основу для оценки причинных параметров. В статистической литературе эти способы также известен как причинно---следственные модели Рубина (RCM) в знак признания Рубину ( 1974, 1978 ) который внес ключевой вклад, но в свою очередь, ссылается на Р.А. Фишера, как создателя подхода. В оставшейся части этого раздела будут проанализированы характерные особенности RCM. 


	Причинные параметры на основе гипотетических ситуаций обеспечивают статистически значимые и оперативные определения причинности, что в некоторых отношениях отличается от традиционного определения. Во---первых, идеальные параметры этой структуры происходят из---за значительной простоты эконометрических методов. Во---вторых, этот способ , обычно опирается на параметры имеющие меньшую причину, которые , как считается, имеют самое непосредственное отношение к политике, которое рассматривается. В---третьих, подход дает дополнительные сведения о свойствах причинных параметров.
	
\subsection{Причинно---следственная модель Рубина}



Термин <<воздействие>> используется наравне с <<причинность>>. В медицинских исследованиях новых лекарств, с участием людей, которые получили лечение и тех, кто не получил, реакцию на препарат сравнивают между двумя группами. Мерой воздействия причинности является средняя разница в результатах этих групп. В экономике термин воздействие используется очень широко. По существу он охватывает переменные, влияние которых на некоторые результаты является объектом исследования. Примерами являются, связь количества лет обучения с заработной платой , размер класса и успеваемость, профессиональная подготовка и заработок. Следует отметить, что воздействие не обязательно должно быть экзогенным, а во многих случаях оно является эндогенной переменной.


	В рамках потенциальной модели результата (ПОМ), который предполагает, что каждый элемент целевой группы населения потенциально подвержен воздействию, $(y_{1i},y_{0i},D_{i})$, $i=1,\dots,N$, образуют основу оценок. Переменная $D$ принимает значения 1 и 0 соответственно, когда воздействие получено или не получено; $y_{1i}$ измеряет ответ для индивидов $i$, получающих воздействие, а $y_{0i}$ --- не получают воздействие. То есть


\begin{equation}
y_{i}=
\begin{cases}
y_{1i}, & \text{если $D_{i}=1$,} \\
y_{0i}, &  \text{если $D_{i}=0$.}
\end{cases}
\end{equation}

Два этих состояния являются взаимоисключающими для $i$---го индивида. Эффект от причины $D$ измеряется как $(y_{1i}-y_{0i})$. Средний причинный эффект $D_{i}=1$, относительно $D_{i}=0$, измеряют средний эффект воздействия (АТЕ):

\begin{equation}
ATE=E[y|D=1]-E[y|D=0],
\end{equation}

где ожидания относительно вероятностного распределения на целевые группы населения. В отличие от обычной структурной модели, которая подчеркивает предельные эффекты, ПОМ подчеркивает среднее воздействие и параметры, связанные с ней.


	Экспериментальный подход к оценке АТС---параметров включает в себя случайное величины воздействия с последующим сравнением результатов с набором не воздействовавших случаев, которые служат в качестве контроля (см глава 3). Случайное распределение предполагает, что лица, подвергавшиеся воздействию выбираются случайным образом, и, следовательно, назначение воздействия не зависит от результата и не коррелирует с атрибутами воздействовавших предметов. Переменная воздействия может рассматриваться как экзогенная, а его коэффициент в линейной регрессии не будет страдать от исключения переменной систематической ошибки,  если некоторые соответствующие переменные неизбежно будут исключены из регрессии. При определенных условиях (более подробно в главах 3 и 25), средняя разница между результатами обработанной и контрольной группы даст оценку ATE. Выигрыш в хорошо продуманных экспериментов заключается в относительной простоте, с которой могут быть сделаны выводы о причинности. 
	
	
	Из---за того, что  рандомизации воздействия, как правило, не представляется возможным в экономике, оценка ATE параметров должны быть основаны на данных наблюдений, полученных при неслучайном воздействии. Затем оценки ATE будут подвергаться некоторым усложнениям, которые включают, например, возможную корреляцию между результатами и воздействием, пропущенных переменных, и эндогенностью обработанных переменных. Некоторые эконометристы предположили, что отсутствие рандомизации является главным препятствием для убедительного статистического вывода о причинно---следственных связях.
	
	
	Модель потенциального результата может привести к причинности, если контрфактические явления могут быть четко сформулированы и введены в действие . Ясное изложение контрфактического, с прояснением того, что нужно сравненить, является важной особенностью этой модели. Если, как может быть в случае с данными наблюдений, есть отсутствие четкого различия между наблюдаемыми и контрфактическими количествами, то ответ на вопрос о том, кто зависит от воздествия, остается неясным. Например, необходимо определить необработанную группу, которая является прокси к воздействующей группе, если воздействие не применялось. Не обязательно, чтобы этот шаг был всегда реализован.
	
	
	Вторая полезная функция ПОМ в том, что она определяет возможности для причинного моделирования, созданные естественным или квази---экспериментальным способом. Когда данные генерируются в таких условиях, и при наличии некоторых других условий, причинного моделирования может происходить без всех осложнений СЛОУ. Этот вопрос анализируется далее в главах 3 и 25.


	В---третьих, в отличие от структурной формы СЛОУ, где все переменные кроме объясняемой могут быть помечены как <<причины>>, в ПОМ не все объясняющие переменные можно рассматривать как причинные, многие из них просто характеристики индивида, которые необходимо включать в регрессионный анализ, а характеристики не являются причинами (Holland, 1986). Причинные параметры должны быть связаны с переменными, которые фактически или потенциально, прямо или косвенно, воздействуют на объясняемую переменную.
	
	
	Наконец, идентифицируемость ATE параметров может быть легкой исследовательской целью  и, следовательно, возможно в ситуациях, когда идентифицируемость СЛОУ не может быть выполнена (Angrist, 2001), однако в каждом конкретном случае это определяется по---своему. Тем не менее, многие из доступных приложений ПОМ обычно используют ограниченные, а не полные, информационные рамки. Тем не менее, даже в пределах СЛОУ использования ограниченной информации также возможно, как было описано ранее.



\section{Причинное моделирование и стратегии оценивания}


В этом разделе мы кратко расскажем о некоторых способах, которые используют эконометристы для моделирования причинно---следственных связей. Эти подходы могут быть использованы в структурах СЛОУ и ПОМ, но они, как правило, определены с первой.

\subsection{Идентификация структур}
\begin{center}
Информационно---полная структурная модель
\end{center}


Один из вариантов такого подхода основан на параметрической спецификации совместного распределения эндогенных переменных при условии экзогенных. Взаимосвязи не обязательно выходят из оптимизационной модели поведения. Параметрические ограничения накладываются для обеспечения идентификации параметров модели, которые являются целью статистического вывода. Вся модель оценивается одновременно с использованием метода максимального правдоподобия или метода моментов. Мы называем этот подход информационно---полная структурная подход. Для четко определенных моделей этот подход является привлекательным, но в целом его минус заключается в том, что он может содержать некоторые уравнения, которые плохо определены. 


	Этот подход статистически можно интерпретировать, как подход, в котором совместное распределение вероятностей эндогенных переменных, с учетом экзогенных переменных, лежит в основе вывод о причинности. Объединенность может происходить от одновременной или динамической взаимозависимости между эндогенными переменными и/или ошибками уравнений.
	
	
\begin{center}
Информационно---ограниченная структурная модель
\end{center}


Напротив, когда центральным объектом статистического вывода является оценка одного или двух ключевых параметров, может быть использован информационно---ограниченный подход. Особенностью данного подхода является то, что, хотя одно уравнение находится в центре внимания вывода, используется совместная зависимость между ним и другими эндогенными переменными. Это требует явные предположения о некоторых особенностях модели, которые не являются основным объектом вывода. Метод инструментальных переменных, последовательное многошаговый метод, и метод ограниченного максимального правдоподобия информации являются конкретными примерами такого подхода. Для реализации данного подхода обычно работают с одним (или более) структурным уравнением. К тому же, информационно---ограниченный  подход вычислительно более сложный, чем с информационно---полный подход.


	Статистически можно интерпретировать информационно---ограниченный подход как тот, в котором совместное распределение раскладывается в произведение условной модели эндогенной переменной, представляющие интерес, например $y_{1}$, и предельной модели для других эндогенных переменных, скажем, $y_{2}$, которые во множестве условных переменных, как в

\begin{equation}
f(y|x,\theta)=g(y_{1}|x,y_{2},\theta_{1})h(y_{2}|x,\theta_{2}), \quad \theta \in \Theta.
\end{equation}


Моделирование может быть основано на компоненте $g(y_{1}|x,y_{2},\theta_{1})$, не обращая внимание на $h(y_{2}|x,\theta_{2})$, если $\theta_{2}$ рассматриваются как мешающие параметры. Конечно, такое разложение не является уникальным, поэтому информационно---ограниченный подход может иметь несколько вариантов.


\begin{center}
Приведённая форма
\end{center}


Третий вариант подхода СЛОУ работает с сокращенной восстановленной формой. Здесь тоже есть заинтересованность в структурных параметрах. Тем не менее, может быть удобно оценивать параметры от восстановленной формы с учетом ограничений. Во временном ряду выявленных авторегрессиях вектор привести пример.


\subsection{Стратегии идентификации}


Есть множество потенциальных способов, в которых определение ключевых параметров модели может оказаться под угрозой. Пропущенные переменные, неправильная спецификация функциональной формы, ошибки при измерении объясняющих переменных, использование нерепрезентативных данных населения, и игнорирование эндогенности объясняющих переменными. Микроэконометрика содержит много конкретных примеров того, как эти проблемы могут быть решены. Angrist и Krueger (2000) предоставляют комплексное обследование популярных стратегий идентификации в экономике труда, с акцентом на ПОМ. Большинство вопросов, рассматриваются в других частях книге, но кратко упоминается здесь.


\begin{center}
Exogenization
\end{center}


Данные иногда создаются в экспериментальных условиях. Идея заключается в том, что перменные могут экзогенно меняться  в некоторой выборке, в то время как в другой они остаются неизменными. Например,  минимальная заработной плата в одном государстве может измениться, а в соседнем нет. Если естественный эксперимент приближает рандомизированное воздействие, то использование таких данных для оценки структурных параметров может быть проще, чем оценка больших одновременных уравнений с эндогенными переменными воздействия. Возможно также, что воздействие переменной в естественном эксперименте можно рассматривать как экзогенные, но воздействие само по себе не является случайным.


\begin{center}
Ликвидация мешающих параметров
\end{center}


Идентификация может быть проблематична при наличии большого количества мешающих параметров. Например, в поперечном сечении регрессионной модели функция условного математического ожидания $E[y_{i}|x_{i}]$ может включать в себя отдельный конкретный фиксированный эффект $\alpha_{i}$, который коррелирует с ошибками регрессии. Этот эффект не может быть определен без большого количества наблюдений для каждого индивида (т.е. панельные данные). Тем не менее, с небольшой панелью она может быть устранена путем трансформации модели. Другим примером является наличие не зависящий от времени экзогенной переменной, которая может быть общим для групп индивидов.


\begin{center}
Контроль вмешивающихся факторов
\end{center}


Когда переменные исключены из регрессии, и когда исключенные факторы коррелируют с включенными переменными, то появляется confounding bias results. Например, в регрессии, где заработная плата является зависимой переменной, а обучение объясняющей переменной, индивидуальные способности можно рассматривать как не включённую переменную. Это означает, что потенциально коэффициент обучения не может быть определен. Одна из возможных стратегий является введение контрольных переменных в модель; это называется подходом функции управления. Эти переменные позволяют приблизить влияние не включённых переменных.Например, различные школьные достижения могут служить в качестве контроля индивидуальных способностей.



\begin{center}
Создание синтетических образцов
\end{center}


В рамках ПОМ параметр причинности может быть неизвестным из---за отсутствия необходимого сравнения или контрольной группы, что не может обеспечить оснований для оценки. Возможным решением является создание синтетического образца, который включается в группу сравнения, который являются прокси для контроля. Такой образец создается путем сопоставления (см. главу 25). Если обработанные образцы могут быть расширены с помощью хорошей контрольной группы, то идентификация причинных параметров может быть достигнута в том смысле, что параметр, связанный с ATE может быть оценён.


\begin{center}
Инструментальные переменные
\end{center}


Если идентификация находится под угрозой, потому что переменные воздействия являются эндогенными, то стандартное решение заключается в использовании инструментальных переменных. Выбор инструментальных переменных, а также интерпретация полученных результатов должна быть сделана аккуратно, потому что результаты могут быть чувствительны к выбору инструментов. Этот подход анализируется в разделах 4.8, 4.9, 6.4, 6.5 и 25.7, а также в других местах по мере необходимости.
\\
\\
\begin{center}
Reweighting Samples
\end{center}


Выводы о генеральной совокупности действительны только, если выборочные данные являются репрезентативными для неё. Проблема возникает тогда, когда выборочные данные не являются репрезентативными, и в этом случае параметры генеральной совокупности не определены. Эта проблема может быть связана с проблемой корректировкой выборки (Глава 16) или с той, которая требует перевзвешивания информации (Глава 24).













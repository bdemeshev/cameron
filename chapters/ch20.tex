
    % Section 20.3. Эксперимент по страхованию здоровья:
    % http://www.d22d.ru/load/28-1-0-261

\chapter{ Модели счетных данных}


\section{Введение}\label{sec:20.1}

\noindent
Во многих экономических ситуациях зависимая переменная принимает целые неотрицательные значения, представляющие собой количество, число или подсчет произошедших событий, которое мы и хотим объяснить с помощью набора регрессоров. В отличие от классической модели регрессии зависимая переменная является дискретной, с ненулевой вероятностью принимающей только целые неотрицательные значения. Можно показать, что некоторые модели, представленные ранее в книге, такие как модель бинарного выбора и модель времени жизни, тесно связаны с моделью регрессии для счетных данных. Как и другие модели с дискретными переменными, например, логит и пробит, эта модель является нелинейной, свойства которой объясняются дискретностью и нелинейностью.

Рассмотрим несколько примеров из микроэконометрики на выборке независимых пространственных данных. В исследованиях по фертильности часто моделируют число рождений у женщины в определенном возрастном интервале как зависимость от таких факторов, как количество лет обучения, возраст и доход домохозяйства (Винкелманн, 1995). В некоторых моделях принятия решений в домохозяйстве в качестве объясняющей переменной используют количество детей, принимая во внимание ее эндогенность. В рамках анализа несчастных случаев часто исследуют взаимосвязь между безопасностью, измеренной как количество несчастных случаев на рейсах авиакомпании, и прибыльностью, а также другими финансовыми показателями авиакомпании (Роуз, 1990). Для оценки спроса на рекреационные услуги определяют ценность природных ресурсов, таких как национальные парки, измеренную с помощью количества путевок в зоны отдыха (Гурму и Триведи, 1996). Для оценки спроса на услуги здравоохранения обычно моделируют данные по количеству потребления таких услуг, например, посещений доктора или дней, проведенных в госпитале в прошлом году (Кэмерон и др., 1988). В случае если мы хотим определить взаимосвязь между такими переменными, как состояние здоровья и страхование на случай болезни, регрессии для счетных данных также подходят.

    \begin{table}[!ht]\caption{\textit{Доля нулевых значений в эмпирических исследованиях}}\label{tab:20.1}
    \begin{center}
\begin{tabular}{llcc}
\hline \hline
                          &               &Размер         &Доля нулевых\\
Исследование              &Переменная     &выборки        &значений\\
\hline
Кэмерон и др. (1988)     &Визиты к врачу&$5,190$&$0.798$\\
Полмейер и Улрих (1995) &Визиты к специалисту&$5,096$&$0.678$\\
Гротендорст (1995)       &Лекарства по рецепту&$5,743$&$0.224$\\
Деб и Триведи (1997)      &Число дней в госпитале&$4,406$&$0.806$\\
Гурму и Триведи (1996)    &Путевки в зоны отдыха&$659$&$0.632$\\
Гейл и др. (1997)        &Число госпитализаций&$30,590$&$0.899$\\
Грин (1997)             &Негативные отметки&$1,319$&$0.803$\\
                          &в кредитной истории&&\\
\hline \hline
    \end{tabular}
    \end{center}
    \end{table}

Основные подходы к моделированию представлены в разделах \ref{sec:20.2}--\ref{sec:20.5}. Детальное описание модели регрессии Пуассона можно найти в разделе \ref{sec:20.2}. Применение модели на примере данных RHIE рассмотрено в разделе \ref{sec:20.3}. Поскольку модель регрессии Пуассона довольно ограничительна, в разделе \ref{sec:20.4} мы рассмотрим другие, также широко используемые параметрические модели счетных данных. Здесь же можно найти менее известные альтернативные параметрические методы для счетных данных, такие как модель дискретного выбора. Частично параметрический подход к моделированию условных математического ожидания и дисперсии подробно представлен в разделе \ref{sec:20.5}. Раздел \ref{sec:20.6} представляет собой введение в многомерные модели счетных данных и модели с эндогенными регрессорами. Раздел \ref{sec:20.7} иллюстрирует применение различных моделей на данных RHIE. В обучающих целях мы подробно рассмотрим модель регрессии Пуассона на пространственных данных.  Другие модели, хотя и более подходящие, чем модель Пуассона, из соображений краткости представлены менее детально. Полное описание можно найти у Кэмерона и Триведи (1998), а также в работах, указанных в разделе \ref{sec:20.9}.




\section{Основные модели регрессии счетных данных}\label{sec:20.2}

\noindent
В некоторых случаях нас интересуют переменные в том виде, в котором они есть, например, число рождений. В других случаях интересующие нас переменные, такие как спрос на медицинские услуги и расходы на НИОКР, выраженные в долларах, являются непрерывными, однако доступные по ним данные --- счетные. Зачастую выборка состоит из \textbf{нескольких дискретных значений}, например, 0, 1 и 2. Это подтверждает таблица \ref{tab:20.1}, где представлена доля нулевых значений в некоторых опубликованных эконометрических исследованиях; заметим, что иногда эта доля может достигать 90\%. Также распределение данных может быть \textbf{скошено вправо}. Наконец, в данных по определению присутствует \textbf{гетероскедастичность}, где дисперсия возрастает вместе со средним.


\subsection{Регрессия Пуассона}\label{sec:20.2.1}

\noindent
Несмотря на то, что модель Пуассона зачастую плохо соответствует данным, изучение анализа счетных данных обычно начинают именно с нее. В разделах \ref{sec:20.2.1}--\ref{sec:20.2.3} мы представим модель регрессии Пуассона, рассмотренную ранее в разделе 5.2, % \ref{sec:5.2} # UNCOMMENT AFTER 5 CH
оценивание методом максимального правдоподобия, интерпретацию оценок коэффициентов и соответствующие модификации для урезанных и цензурированных данных. В разделе \ref{sec:20.2.3} мы также представим квази-ММП, основанный на распределении Пуассона с верно специфицированным условным математическим ожиданием, но с неверно специфицированной функцией дисперсии. Недостатки модели Пуассона, в частности, свойство равенства математического ожидания и дисперсии, будут рассмотрены в разделе \ref{sec:20.2.4}

    \begin{table}[!ht]\caption{\textit{Описательные статистики данных в некоторых исследованиях по НИОКР}}\label{tab:20.2}
    \begin{center}
\begin{tabular}{lccccc}
\hline \hline
                        &Размер         &           &Стандартная&Максимальное            &Доля нулевых\\
Исследование            &выборки        &Среднее    &ошибка     &кол-во патентов         &значений\\
\hline
Синсера (1997)          &$181$          &$60.8$     &$721.6$    &$925$                  &$<0.19$\\
Крепон и Дюге (1997b)   &$698$          &$11.6$     &na$^a$     &na                   &$0.441$\\
Крепон и Дюге (1997a)   &$451$          &$2.73$     &$11.45$    &na                   &$0.729$\\
Хаусман и др. (1984)   &$346$          &$32.1$     &$66.36$    &$515$                  &$0.220$\\
Ванг и др. (1998)      &$70$           &$23.46$    &$39.10$    &$173$                  &$0.186$\\
\hline \hline
\multicolumn{6}{l}{$^a$ \scriptsize{na, not available}}
    \end{tabular}
    \end{center}
    \end{table}

Следует оговориться, что в некоторых случаях высокая доля нулей в выборке может сочетаться с высокими ненулевыми значениями, что создает определенные сложности при моделировании. Это подтверждает таблица \ref{tab:20.2}, где представлены описательные статистики данных из пяти исследований о зависимости между количеством патентов и расходами на НИОКР (R\&D). Следует обратить внимание, что максимальное наблюдаемое значение существенно превышает среднее. Сложности же заключаются в выборе такой функциональной формы, которая будет учитывать высокую долю нулевых значений вместе с высоким средним. В других примерах, практически все данные являются однозначными числами (например, число рождений), поэтому среднее количество событий невелико.

Такие особенности требуют применения специальных методов и моделей счетных регрессий, среди которых можно выделить два подхода.

Первый подход является \textbf{полностью параметрическим}, описывающим распределение данных, которое учитывает целочисленность и неотрицательность $y$. Данный подход применялся в первых работах, в основном, в биостатистике, где счетные регрессии рассматривались как способ обобщения литературы на распределение независимых и идентичных наблюдений. Также он был представлен в известной работе Хаусмана и др. (1984).

Второй подход основан на спецификации \textbf{математического ожидания и дисперсии} (\textit{mean--variance approach}) и описывает неотрицательность условного математического ожидания и условную дисперсию как функцию от математического ожидания, что неплохо соответствует неотрицательности и гетероскедастичности данных, но не учитывает их дискретность. Данный подход был представлен Нелдером и Веддербёрном (1972) в рамках анализа не только счетных данных и впоследствии послужил основой для обобщенной линейной модели, широко используемой в статистике (МакКуллах и Нелдер, 1989). В эконометрике он был представлен Гурьеру, Монфортом и Троньоном (1984a,b), и чаще всего рассматривался как частный случай обобщенного метода моментов.


\subsection{ММП и квази-ММП Пуассона}\label{sec:20.2.2}

\noindent
Метод квази-максимального правдоподобия Пуассона (квази-ММП, \textit{QMLE}) был представлен и изучен в главе 5 % \ref{ch:5} # UNCOMMENT IN THE END OF THE BOOK
как пример оценивания $m$ плотностей. В данном разделе мы представим более полное описание.

Обычная стохастическая модель для счетных данных является точечным процессом Пуассона. Это подразумевает, что число наступивших событий описывается \textbf{распределением Пуассона} с функцией вероятности
    \begin{align}\label{eq:20.1}
    \Pr [Y=y]=\frac{e^{-\mu}\mu^y}{y!}, \hspace{0.5cm} y=0,1,2,\ldots ,
    \end{align}
где $\mu$ --- параметр интенсивности. Обозначим распределение как $\mathcal{P}[\mu]$, первые два момента которого равны
    \begin{align}\label{eq:20.2}
    \E[Y] = \mu, \\
    \V[Y] = \mu, \notag
    \end{align}
что соответствует свойству \textbf{равенства математического ожидания и дисперсии} (\textit{equidispersion}) в распределении Пуассона.

Добавив нижний индекс $i$ для $y$ и $\mu$, мы можем расширить данный подход на случай регрессии. \textbf{Модель регрессии Пуассона} получается из распределения Пуассона с помощью выражения среднего $\mu$ через набор регрессоров $\x$. Стандартной предпосылкой является моделирование среднего в виде экспоненты
    \begin{align}\label{eq:20.3}
    \mu = \exp{(\xib)}, \hspace{0.5cm} i=1,\ldots ,N,
    \end{align}
с $K$ линейно независимых ковариат, включая константу. Зная (\ref{eq:20.2}) и (\ref{eq:20.3}), можно найти дисперсию $\V[y_i|\x_i] = \exp(\xib)$, поэтому регрессия Пуассона гетероскедастична по определению.

Из (\ref{eq:20.1}) и (\ref{eq:20.3}), а также предположения о независимости наблюдений $(y_i|\x_i)$ следует, что подходящим методом оценивания является метод максимального правдоподобия. Логарифм функции правдоподобия записывается как
    \begin{align}\label{eq:20.4}
    \ln L(\be) = \sum^{N}_{i=1} {y_i\xib - \exp{(\xib) - \ln y_i !}}.
    \end{align}
\textbf{Оценки, полученные ММП Пуассона}, $\hat{\be}_P$, являются решением $K$ нелинейных уравнений, соответствующих условию первого порядка для нахождения максимума правдоподобия
    \begin{align}\label{eq:20.5}
    \sum^{N}_{i=1} (y_i - \exp{(\xib)})\x_i = \textbf{0}.
    \end{align}
Если $\x_i$ включает константу, то сумма остатков $y_i - \exp(\xib)$ равна нулю согласно (\ref{eq:20.5}). Функция правдоподобия является глобально вогнутой, следовательно, оценки параметров единственны. Сами оценки можно найти с помощью методов Гаусса--Ньютона или Ньютона--Рафсона.

В эконометрической литературе \textbf{псевдо-ММП} (\textit{PML}) или \textbf{квази-ММП} (\textit{QML}) относится к оценке максимального правдоподобия при мисспецификации функции плотности (Гурьеру и др., 1984a). Термины псевдо-ММП и квази-ММП взаимозаменяемы. Оценки не требуют столь же строгих предпосылок о процессе, генерирующем данные, что и обозначенная функция правдоподобия; см. раздел 5.7. % \ref{sec:5.7} # UNCOMMENT AFTER 5 CH
В статистической литературе квази-ММП часто относят к нелинейному обобщенному методу наименьших квадратов. В этом смысле квази-ММП эквивалентен стандартной максимизации правдоподобия для регрессии Пуассона.

Из (\ref{sec:20.5}) следует, что условия первого порядка для оценки псевдо-ММП, $\hat{\be_\mP}$, равны $\sum^{N}_{i = 1} (y_i - \exp(\xib))\x_i = \0$. Как уже упоминалось, левая часть уравнения имеет математическое ожидание, равное нулю, если $\E[y_i|\x_i] = \exp(\xib)$. Следовательно, оценки квази-ММП Пуассона состоятельны при более слабых предпосылках о корректной спецификации условного математического ожидания; то есть, необязательно, чтобы данные имели распределение Пуассона. Используя результаты, полученные в разделе 5.2.3, % \ref{sec:5.7} # UNCOMMENT AFTER 5 CH
запишем матрицу ковариаций в виде
    \begin{align}\label{eq:20.6}
    \V_{\mathrm{PML}}[\hat{\be_\mP}] = \left( \sum^{N}_{i = 1} \mu_i \x_i\x'_i \right)^{-1} \left( \sum^{N}_{i = 1} \omega_i \x_i\x'_i \right) \left( \sum^{N}_{i = 1} \mu_i \x_i\x'_i \right)^{-1},
    \end{align}
где $\omega_i = \V[y_i|\x_i]$ является условной дисперсией $y_i$.

Если ввести более строгие предпосылки о корректной параметрической спецификации регрессии Пуассона, что есть $\omega_i = \mu_i$, то $\hat{\be_\mP}$ будет состоятельной и асимптотически нормальной оценкой $\be$ с матрицей выборочных ковариаций
    \begin{align}\label{eq:20.7}
    \V[\hat{\be_\mP}] = \left( \sum^{N}_{i = 1} \mu_i \x_i \x'_i \right)^{-1},
    \end{align}
в случае, если $\mu_i$ имеет экспоненциальную форму вида (\ref{eq:20.3}).

Оценки, полученные ММП и псевдо-ММП Пуассона, идентичны, но обладают разной дисперсией. Практическое применение более робастных оценок вида (\ref{eq:20.6}) представлено в разделе \ref{sec:20.5.1}.


\subsection{Интерпретация коэффициентов регрессии}\label{sec:20.2.3}

\noindent
В линейных моделях с $\E[y|\x] = \xib$ коэффициенты $\be$ показывают, насколько изменяется условное математическое ожидание при изменении соответствующих регрессоров на одну единицу. В нелинейных моделях интерпретация будет отличаться; см. обсуждение в разделе 5.2.4. Взяв производную в любой модели с экспоненциальным условным математическим ожиданием, получим
    \begin{align}\label{eq:20.8}
    \frac{\pa\E[y|\x]}{\pa x_j} = \beta_j \exp(\xib),
    \end{align}
где скаляр $x_j$ обозначает $j$-ый регрессор. Например, если $\hat{\beta}_j = 0.25$ и $\exp(\x'_i\hat{\be}) = 3$, то изменение $j$-го регрессора на единицу приведет к изменению $y$ на $0.75$. Значит, частичный отклик зависит от выражения $\exp(\x'_i\hat{\be})$, которое принимает различные значения в зависимости от индивида. Легко показать, что $\beta_j$ равняется относительному изменению $\E[y|x]$ при изменении $x_j$ на единицу. Если $x_j$ выражен в логарифмах, то $\beta_j$ представляет собой эластичность.

Так как эффекты от изменения регрессоров индивидуальны, можно рассчитать средний отклик, $N^{-1} \sum_i \pa \E[y_i|\x_i]/\pa x_{ij} = \hat{\beta}_j \times N^{-1} \sum_i \exp(\x'_i\hat{\be})$. Для моделей регрессии Пуассона со свободным членом это выражение упрощается до $\hat{\beta}_j \bar{y}$.

Из (\ref{eq:20.8}) также следует, что если оценка $\beta_j$ в два раза превышает оценку $\beta_k$, то эффект от изменения $j$-го регрессора на единицу будет также вдвое больше эффекта от изменения $k$-го регрессора.


\subsection{Избыточная дисперсия}\label{sec:20.2.4}

\noindent
Зачастую модель регрессии Пуассона плохо соответствует счетным данным, в связи с чем в разделах \ref{sec:20.3} и \ref{sec:20.4} предлагаются альтернативные модели. Фундаментальная проблема заключается в том, что распределение Пуассона задается единственным параметром $(\mu)$, и, как следствие, все остальные моменты $y$ являются функциями от $\mu$. Для сравнения, нормальное распределение имеет различные параметры для среднего и разброса, $(\mu)$ и $(\sis)$, соответственно. По той же причине однопараметрическое экспоненциальное распределение хуже соответствует данным, чем двухпараметрическое распределение Вейбулла. Заметим, однако, что проблема отсутствует в бинарных данных. В этом случае в качестве распределения подходит однопараметрическое распределение Бернулли, которое определяет вероятность успеха $p$ и вероятность проигрыша $1 - p$, где $p$ необходимо объяснить с помощью набора регрессоров.

Одно из последствий такого однопараметрического моделирования заключается в том, что вероятность нулевых значений, предсказанная по модели Пуассона, значительно ниже, чем их доля в выборке, что называется проблемой \textbf{избыточных нулевых значений}.

Другим недостатком является то, что модель Пуассона подразумевает равенство дисперсии и математического ожидания (см. (\ref{eq:20.2})), в то время как в счетных данных дисперсия обычно превышает среднее, что соответствует проблеме \textbf{избыточной дисперсии}.

Избыточная дисперсия в модели Пуассона имеет те же последствия, что и нарушение предпосылки о гомоскедастичности в модели линейной регрессии. При условии, что спецификация условного математического ожидания (\ref{eq:20.3}) верна, оценки ММП Пуассона состоятельны, что следует из условий первого порядка (\ref{eq:20.5}), так как математическое ожидание левой части уравнения (\ref{eq:20.5}) равно нулю при $\E[y_i|\x_i] = \exp(\xib)$. В общем случае свойство состоятельности применяется к квази-ММП, когда заданная плотность принадлежит экспоненциальному семейству распределений. Хотя и распределение Пуассона, и нормальное распределение являются членами экспоненциального семейства распределений, рассмотренного ранее в разделе 5.7.3, учитывать избыточную дисперсию важно по ряду причин. Во-первых, при более сложной структуре данных, например, при наличии урезанных и цензурированных наблюдений, оценки больше не являются состоятельными. Во-вторых, даже для простой структуры данных избыточная дисперсия приводит к существенно завышенным стандартным ошибкам и $t$-статистикам по сравнению с обычным ММП, что указывает на важность робастного оценивания дисперсии. В-третьих, для оценивания вероятностей числа событий требуется больше дополнительных параметров, чем для оценивания условного математического ожидания.

Избыточная дисперсия может указывать на мисспецификацию, особенно при наличии урезанных и цензурированных наблюдений, если они не были учтены при оценивании. В таком случае условное математическое ожидание специфицировано неверно и наличие избыточной дисперсии приводит не только к неэффективности, но и к несостоятельности оценок ММП.

Поэтому сразу после оценивания регрессии Пуассона желательно провести тест на избыточную дисперсию. Во многих моделях счетных данных избыточная дисперсия имеет вид
    \begin{align}\label{eq:20.9}
    \V[[y_i|\x_i] = \mu_i + \al g(\mu_i)
    \end{align}
где $\al$ является неизвестным параметром, а $g(\cdot)$ --- известной функцией, обычно задаваемой как $g(\mu) = \mu^2$ или $g(\mu) =\mu$. Предполагается, что при обеих гипотезах спецификация математического ожидания верна, а при нулевой гипотезе также $\al = 0$, что соответствует равенству дисперсии и математического ожидания $\V[y_i|\x_i] = \mu_i$.
Простая \textbf{тестовая статистика} для проверки гипотезы об \textbf{избыточной дисперсии} $H_0: \al = 0$ против альтернативной $H_1: \al \ne 0$ или $H_1: \al > 0$ может быть рассчитана с помощью оценки вспомогательной МНК регрессии (без свободного члена) на основе предсказанных значений $\hat{\mu}_i = \exp(\x'_i\hat{\be})$ по модели Пуассона
    \begin{align}\label{eq:20.10}
    \frac{(y_i - \hat{\mu}_i)^2 - y_i}{\hat{\mu}_i} = \al\frac{g(\hat{\mu}_i)}{\hat{\mu}_i} + u_i,
    \end{align}
где $u_i$ обозначает ошибку. Рассчитанные $t$-статистики для коэффициента $\al$ асимптотически нормальны при нулевой гипотезе (Кэмерон и Триведи, 1990). Тест также может быть использован для проверки гипотезы о \textbf{недостаточной дисперсии}, $\al < 0$, то есть, когда условная дисперсия меньше условного математического ожидания. См. также Гурму и Триведи (1992).




\section{Пример на счетных данных: Визиты к врачу}\label{sec:20.3}

\noindent
Для иллюстрации методов и моделей, описанных выше, мы будем использовать данные из эксперимента по страхованию здоровья, проводимого корпорацией RAND (\textit{RAND Health Insurance Experiment}) с 1974 по 1982 г., ранее использованные в работе Деба и Триведи (2002). Авторы провели более глубокий анализ данных, чем это требуется и хотелось бы в рамках моделей, представленных в этой главе. Данный эксперимент является, пожалуй, наиболее длительным и контролируемым экспериментом в исследованиях медицинского обслуживания. Цель эксперимента заключалась в том, чтобы оценить влияние способов страхования здоровья, случайно распределенных (\textit{randomly assigned}) между пациентами, на потребление медицинских услуг.
% (\textit{fee-for-service})
% (\textit{health maintenance organizations, HMOs})
В ходе эксперимента были собраны данные о $8,000$ участниках из $2,823$ семей из шести городов США. Каждая семья принимала участие в одной из 14 различных программ страхования здоровья в течение трех или пяти лет. Программы варьировались от бесплатной медицинской помощи до 95\%-го сострахования (соплатежа) в пределах максимальных долларовых затрат (\textit{maximum dollar expenditure, MDE}), а также включали приписку к тому или иному медицинскому центру.

Ключевая идея заключается в том, что, поскольку программы страхования не выбирались участниками, а были распределены случайно, то в анализе отсутствует эндогенный эффект воздействия, что позволяет определить причинную связь.

Данные были собраны на основе потребления участниками медицинских услуг и их состояния здоровья в течение случайно распределенного периода участия, трех или пяти лет. Детали можно найти в работах Маннинга и др. (1987), Ньюхауза и др. (1993) и Деба и Триведи (2002). Выборка, используемая в данном примере, исключает программы страхования с бесплатной медициной.

Данные включают информацию о потреблении, затратах, демографических характеристиках, состоянии здоровья и типе страхования. Данные по затратам были проанализированы в разделе 16.6. % \ref{sec:16.6} # UNCOMMENT AFTER CH 16
Ставка сострахования в данной выборке может принимать четыре различных значения. Тем не менее, аналогично исследованиям RAND, мы будем рассматривать ее как непрерывную переменную. Итоговая выборка состоит из $20,186$ наблюдений; каждое наблюдение отображает информацию по соответствующему участнику эксперимента в определенном году. Для простоты мы не будем рассматривать кластеры данных; см. раздел
24.5. % \ref{sec:24.5} # UNCOMMENT AFTER CH 24

    \begin{table}[!htbp]\caption{\textit{Визиты к врачу: распределение частот}}\label{tab:20.3}
    \begin{center}
\begin{tabular}{lccccccccccc}
\hline \hline
Визиты                  &0&1&2&3&4&5&6&7&8&9&10\\
Относительная частота   &31.2&18.9&13.8&9.3&6.7&4.8&3.4&2.6&2.0&1.4&1.0\\
\hline
Визиты                  &11&12&13&14&15&16&\ldots &$>21$& Max&&\\
Относительная частота   &0.9&0.6&0.5&0.4&0.3&0.3& &1.0&77&&\\
\hline \hline
\end{tabular}
    \end{center}
    \end{table}

Потребление в данном примере измеряется в количестве визитов к врачу (MDU). Распределение относительных частот MDU в процентах представлено в таблице \ref{tab:20.3}. MDE обозначает максимальные долларовые затраты, то есть, максимальную сумму платежей, после достижения которой все медицинские счета оплачиваются страховой компанией. Заметим, что порядка 31\% наблюдений содержат нули. Тяжелый правый хвост распределения и существенное превышение дисперсии над средним указывают на то, что в данных присутствует (безусловная) избыточная дисперсия.

В данном разделе мы будем рассматривать оценки регрессии ММП и квази-ММП Пуассона, другие спецификации будут рассмотрены позже. Список объясняющих переменных представлен в таблице \ref{tab:20.4}.

    \begin{table}[!htbp]\caption{\textit{Визиты к врачу: описание переменных}}\label{tab:20.4} % ALT: Посещения врача
    \begin{center}
\begin{tabular}{llcc}
\hline \hline
\textbf{Переменная}&\textbf{Определение}&\textbf{Среднее}&\textbf{Ст. Откл.}\\
\hline
MDU     & Число визитов к врачу &2.861&4.505\\
LC      & $\ln(\textrm{сострахование} + 1)$, $0 \le \textrm{сострахование} \le 100$ &1.710&1.962\\
IDP     & 1, если план страхования с индивидуальной франшизой&0.220&0.414\\
        & 0, в противном случае&&\\
LPI     & $\ln(\max(1, \textrm{ежегодная стимулирующая выплата}$  &4.709&2.697\\
        & \textrm{за участие}))&&\\
FMDE    & 0, если IDP = 1  &3.153&3.641\\
        & $\ln(\max(1, \textrm{MDE}$/(0.01 \textrm{coinsurance}))), в противном случае&&\\
LINC    & $\ln(\textrm{доход домохозяйства})$ &8.708&1.228\\
LFAM    & $\ln(\textrm{размер домохозяйства})$  &1.248&0.539\\
AGE     & Возраст в годах  &25.718&16.678\\
FEMALE  & 1, если женщина  &0.517&0.500\\
CHILD   & 1, если возраст меньше 18  &0.402&0.490\\
FEMCHILD& FEMALE*CHILD  &0.194&0.395\\
BLACK   & 1, если глава домохозяйства афроамериканец &0.182&0.383\\
EDUCDEC & Количество лет обучения главы домохозяйства  &11.967&2.806\\
PHYSLIM & 1, если имеет ограничения физического характера  &0.124&0.322\\
NDISEASE& Количество хронических заболеваний  &11.244&6.742\\
HLTHG   & 1, если самооценка состояния здоровья хорошая  &0.362&0.481\\
HLTHF   & 1, если самооценка состояния здоровья средняя  &0.077&0.267\\
HLTHP   & 1, если самооценка состояния здоровья плохая  &0.015&0.121\\
        & \multicolumn{3}{l}{Пропущенная переменная --- отличная самооценка состояния здоровья}\\
\hline\hline
\end{tabular}
    \end{center}
    \end{table}

    \begin{table}[!htbp]\caption{\textit{Визиты к врачу: оценки модели счетных данных}}\label{tab:20.5}
    \begin{center}
\begin{tabular}{lccccc}
\hline \hline
&\multicolumn{2}{c}{\textbf{Poisson}}&\textbf{PPML}&\multicolumn{2}{c}{\textbf{NB2-PML}}\\
\cmidrule(r){2-3}\cmidrule(r){4-4}\cmidrule(r){5-6}
\textbf{Модель}&\textbf{Коэффициент}&\textbf{$t$-статистика}&\textbf{$t$-статистика}&\textbf{Коэффициент}&\textbf{$t$-статистика}\\
\hline
LC      &$-.0427$&$-7.030$  &$-2.835$   &$-0.0504$  &$-3.228$\\
IDP     &$-.1613$&$-13.881$ &$-5.773$   &$-0.1475$  &$-4.889$\\
LPI     &$0.0128$&$6.999$   &$2.912$    &$0.0158$   &$3.574$\\
FMDE    &$-.0206$&$-5.803$  &$-2.319$   &$-0.0213$  &$-2.351$\\
PHYSLIM &$0.2684$&$21.711$  &$8.240$    &$0.2751$   &$8.068$\\
NDISEASE&$0.0231$&$38.124$  &$13.487$   &$0.0259$   &$15.324$\\
HLTHG   &$0.0394$&$4.109$   &$1.699$    &$0.0065$   &$0.275$\\
HLTHF   &$0.2531$&$15.613$  &$5.894$    &$0.2368$   &$5.425$\\
HLTHP   &$0.5216$&$19.150$  &$6.966$    &$0.4256$   &$6.205$\\
$\al$   &$-$     &$-$       &$-$        &$1.1822$   &$8.926$\\
$-\ln\mL$&$60087$ &          &           &$42777$    &\\
\hline \hline
\end{tabular}
    \end{center}
    \end{table}

Оценки коэффициентов с соответствующими $t$-статистиками, логарифм правдоподобия и информационные критерии представлены в таблице \ref{tab:20.5}. Интерес представляют коэффициенты при переменных, отвечающих за страхование (LC, IDP, LPI и FMDE), поскольку они отражают чувствительность потребления к цене. Нас интересуют также коэффициенты при переменных, отвечающих за состояние здоровья (PHYSLIM, NDISEASE, HLTHG, HLTHF и HLTFP).

Рассмотрим коэффициент при ставке сострахования, измеренной в логарифмах, LC. Эта переменная представляет ключевой интерес, поскольку отражает эффект цены. % ALT: стоимостной эффект
Чем выше ставка сострахования, тем выше будет сумма (со)платежа пациентом за посещение, следовательно, тем ниже будет среднее число визитов к врачу. Оценка коэффициента в регрессии Пуассона (1-ый столбец в таблице \ref{tab:20.5}) отрицательна $(-0.42)$ с $t$-статистикой, равной $2.835$, что указывает на значимую отрицательную зависимость от цены, как и предсказывает классическая теория. Эластичность числа визитов к врачу по отношению к LC равна $-.042$. Однако при интерпретации следует быть аккуратным, поскольку ставка сострахования может принимать лишь четыре значения. С этой оговоркой, оценку можно интерпретировать как эластичность. Аналогично, оценка коэффициента при логарифме дохода (LINC) составляет $0.174$, что означает, что увеличение дохода приводит к увеличению среднего числа посещений.

Как узнать, насколько хорошо модель Пуассона соответствует данным? Один простой способ заключается в том, чтобы сравнить действительные и предсказанные значения вероятностей при различных количествах посещений врача. Такое сравнение представлено в таблице \ref{tab:20.6} для первых девяти визитов; мы не рассматриваем остальные значения, поскольку они составляют менее 10\% выборки. Для расчета предсказанной вероятности $\Pr[y_i|\x'_i\hat{\be}]$ для $y_i = 0, 1, \ldots , 9$, необходимо подставить оценку $\hat{\mu}_i$ в уравнение \ref{eq:20.1} и усреднить по наблюдениям. Заметим, что регрессия Пуассона значительно недооценивает вероятность нулевых значений и переоценивает долю ненулевых значений для количества посещений меньше семи. Следовательно, регрессия Пуассона плохо подходит для работы со счетными данными, что можно объяснить неучетом избыточной дисперсии в данных (Кэмерон и Триведи, 1998, глава 4).

    \begin{table}[!htbp]\caption{\textit{}}\label{tab:20.6}
    \begin{center}
\begin{tabular}{lcccccccccc}
\hline \hline
\textbf{Относительная частота} &$\mathbf{0}$&$\mathbf{1}$&$\mathbf{2}$&$\mathbf{3}$&$\mathbf{4}$&$\mathbf{5}$&$\mathbf{6}$&$\mathbf{7}$&$\mathbf{8}$&$\mathbf{9}$\\
\hline
Относительная частота           &31.2&18.9&13.8&9.3&6.7&4.8&3.4&2.6&2.0&1.4\\
Предсказанные значения          &10.6&19.2&20.9&17.6&12.6&7.99&4.69&2.64&1.46&0.8\\
по модели Пуассона              &&&&&&&&&&\\
Предсказанные значения          &30.9&19.6&13.6&9.67&6.97&5.07&3.70&2.72&2.0&1.47\\
по модели NB2                   &&&&&&&&&&\\
\hline \hline
\end{tabular}
    \end{center}
    \end{table}

Можно ожидать, что при неучете избыточной дисперсии $t$-статистики для ММП Пуассона будут завышены. Для сравнения в 3-ем столбце (PPML) таблицы \ref{tab:20.5} представлены робастные $t$-статистики. Например, при робастном оценивании $t$-статистика для LC возрастает с $-7.03$ до $-2.83$. В таблицах \ref{tab:20.5} и \ref{tab:20.6} также указаны оценки модели NB2, которая будет рассмотрена позже в разделе \ref{sec:20.7}. Из таблицы видно, что модель NB2 лучше соответствует данным, чем модель Пуассона.




\section{Параметрические модели регрессии для счетных данных}\label{sec:20.4}

\noindent
Модель регрессии Пуассона зачастую довольно ограничительна, поэтому в данном разделе мы представим несколько более гибких параметрических моделей.

Во-первых, избыточная дисперсия в счетных данных может являться следствием ненаблюдаемой гетерогенности. В таком случае предполагается, что наблюдения сгенерированы процессом Пуассона (с последовательно независимыми событиями), но параметр интенсивности этого процесса определить невозможно. Более того, параметр интенсивности сам является случайной величиной. Подход на основе смеси, представленный в разделах \ref{sec:20.4.1} и \ref{sec:20.4.2}, приводит к широко применяемой отрицательной биномиальной модели.

Во-вторых, избыточная и в некоторых случаях недостаточная дисперсия может возникать потому, что процесс, определяющий первое событие, может отличаться от процесса, определяющего остальные. Например, первичное обращение к врачу может быть обосновано исключительно решением пациента, однако последующие визиты могут быть назначены врачом. Следовательно, требуется модификация моделей счетных данных, что представлено в разделе \ref{sec:20.4.5}.

В-третьих, избыточная дисперсия в счетных данных может являться следствием нарушения предпосылки о независимости событий, что неявно предполагается в процессе Пуассона. Тогда зависимость можно определить таким образом, чтобы последующие визиты к врачу были более вероятны, чем первый. (Этот подход не нашел широкого применения в анализе счетных данных. В анализе выживаемости такая особенность данных называется истинной зависимостью от состояния.) Предпосылки о ненаблюдаемой гетерогенности или зависимости от состояния также приводят к отрицательной биномиальной модели; см. работу Винкелманна (1995). Модель дискретного выбора, моделирующая вероятности $\Pr[y = j|y\ge j - 1]$, представлена в разделе \ref{sec:20.4.6}.

В-четвертых, можно обратиться к литературе, посвященной одномерным независимо идентично распределенным счетным данным, например, логарифмическому ряду или гипергеометрическому распределению (Джонсон, Котц и Кемп, 1992). Новые модели регрессии можно получить с помощью моделирования одного или двух параметров распределения как функцию от объясняющих переменных. Этот подход не имеет тех же оснований, что и первые три, поэтому маловероятно, что такие модели могут оказаться лучше.

Помимо избыточной дисперсии может возникать и недостаточная, например, если выборочные данные содержат нулевые и единичные значения, а также небольшое количество двоек. Такое распределение будет близко к биномиальному с математическим ожиданием, превышающим дисперсию. Также могут быть использованы распределения Каца и другие распределения, основанные на методах разложения в ряд, которые представлены, например, в работе Кэмерона и Йоханссона (1997); см. также Кэмерона и Триведи (1998, глава 12).


\subsection{Отрицательная биномиальная модель}\label{sec:20.4.1}

\noindent
Несмотря на то, что отрицательная биномиальная модель является примером модели непрерывной смеси, она может быть получена различными способами. Представленный далее подход на основе смеси, однако, является одним из первых и наиболее популярных.

Предположим, что случайная величина $y$ следует распределению Пуассона при условии, что параметр $\la$ известен, $f(y|\la) = \exp(-\la)\la^y/y!$. Более того, пусть параметр $\la$ также содержит случайную компоненту так, что его нельзя точно определить с помощью регрессоров $\x$. То есть, $\la = \mu\nu$, где $\mu$ представляет собой детерминированную функцию от регрессоров $\x$, например, $\exp(\xib)$, а компоненты $\nu > 0$ независимо идентично распределены с плотностью $g(\nu|\al)$. Поскольку наблюдения могут иметь различные $\la$ (гетерогенность) со случайной (ненаблюдаемой) составляющей $\nu$, такая модель является примером модели с ненаблюдаемой гетерогенностью. Заметим, что $\E[\la|\mu] = \mu$, если $\E[\nu] = 1$, следовательно, интерпретация коэффициентов наклона аналогична интерпретации в модели Пуассона.

Маргинальную плотность $y$, при условии информации о параметрах $\mu$ и $\al$, но безусловно на $\nu$, можно получить за счет исключения $\nu$ при интегрировании. Таким образом,
    \begin{align}\label{eq:20.11}
    h(y|\mu,\al) = \int f(y|\mu, \nu) g(\nu|\al) d\nu,
    \end{align}
где $g(\nu|\al)$ называется смешиваемым распределением с неизвестным параметром $\al$. То есть, интегрирование позволяет найти ``усредненное'' распределение. В зависимости от выбора $f(\cdot)$ и $g(\cdot)$ интеграл может как иметь, так и не иметь аналитическую форму.

Пусть $f(y|\la)$ является плотностью распределений Пуассона, а $g(\nu) = \nu^{\de - 1}e^{-\nu\de}\de^\de / \Ga(\de)$, $\nu, \de > 0$ обозначает плотность распределения гамма с математическим ожиданием $\E[\nu] = 1$ и дисперсией $\V[\nu] = 1/\de$. Тогда можно записать \textbf{отрицательную биномиальную} модель в виде плотности смеси % ALT: смеси плотностей
    \begin{align}\label{eq:20.12}
    h[y|\mu, \de]   &= \int^{\infty}_{0} \frac{e^{-\mu\nu}(\mu\nu)^y}{y!} \frac{\nu^{\de - 1}e^{-\nu\de}\de^\de}{\Ga(\de)} d\nu \\
                    &= \int^{\infty}_{0} \frac{e^{-(\mu + \de)\nu} \mu^y}{y!} \frac{\nu^{y + \de - 1} \de^\de}{\Ga(\de)} d\nu \notag \\
                    &= \frac{\mu^y \de^\de}{\Ga(\de) y!} \int^{\infty}_{0} e^{-(\mu + \de)\nu} \nu^{y + \de - 1} d\nu \notag \\
                    &= \frac{\mu^y \de^\de \Ga(y + \de)}{\Ga(\de) y! (\mu + \de)^{y + \de}} \notag \\
                    &= \frac{\Ga(\al^{-1} + y)}{\Ga(\al^{-1}) \Ga(y + 1)} \left( \frac{\al^{-1}}{\al^{-1} +\mu } \right)^{\al^{-1}} \left( \frac{\mu}{\mu + \al^{-1}} \right)^{y}, \notag
    \end{align}
где $\al = 1/\de$. $\Ga(\cdot)$ обозначает гамма интеграл, который упрощается до факториала от целочисленного аргумента. Для преобразований в четвертой строке используется определение гамма функции. Модель Пуассона $(\al = 0)$, модель `с заменой параметров $\de$ на $\al$' (`\textit{the advantage of reparametrization from $\de$ to $\al$}') и геометрическая модель $(\al = 1)$ являются частными случаями отрицательной биномиальной модели.

Как и в случае с несколькими распределениями смеси, отрицательное биномиальное распределение имеет собственное обоснование; см. работу Кэмерона и Триведи (1998, глава 4). То есть, это распределение не всегда подразумевает распределение смеси, поскольку может быть получено несколькими различными способами.

Алгебраическое выражение отрицательной биномиальной модели как \textbf{смеси Пуассона--гамма} имеет Байесовскую интерпретацию. Априорное распределение $\mu$ является гамма распределением при данных $\al$ и сопряженных априорных распределениях  экспоненциального семейства из раздела 13.2.4. Предполагается, что апостериорное распределение может быть выражено в аналитической форме. Следовательно, оценка ММП и Байесовское апостериорное среднее будут совпадать при предпосылке о неопределенном априорном распределении (\textit{vague prior}) $\al$.

Первые два момента отрицательного биномиального распределения равны
    \begin{align}\label{eq:20.13}
    \E[y|\mu, \al] &= \mu \\
    \V[y|\mu, \al] &= \mu(1 + \al\mu). \notag
    \end{align}
Следовательно, дисперсия превышает математическое ожидание, поскольку $\al > 0$ и $\mu > 0$. Более того, можно легко показать, что избыточная дисперсия возникает всегда, когда $y|\la$ следует распределению Пуассона, а ненаблюдаемая гетерогенность имеет мультипликативную форму $\la = \mu\nu$ с математическим ожиданием $\E[\nu] = 1$. Избыточная дисперсия вида (\ref{eq:20.9}) также представлена в разделе \ref{sec:20.2.4}.

В регрессионном анализе применяются два стандартных типа отрицательной биномиальной модели с $\mu_i = \exp(\xib)$.
Наиболее распространенный из них предполагает оценивание параметра $\al$, и в таком случае условная функция дисперсии $\mu + \al \mu ^2$ из уравнения (\ref{eq:20.13}) имеет квадратичную зависимость от среднего.

Другой тип предполагает линейную функцию дисперсии $\V[y|\mu, \al] = (1 + \ga)\mu$, полученную с помощью замены $\al$ на $\ga/\mu$ в уравнении (\ref{eq:20.12}). Модель можно оценить непосредственно методом максимального правдоподобия, где логарифм правдоподобия легко записать с помощью уравнения (\ref{eq:20.12}). Иногда такой тип называют отрицательной биномиальной моделью 1 (NB1), а модель с квадратичной функцией дисперсии --- отрицательной биномиальной моделью 2 (NB2) (Кэмерон и Триведи, 1998). Детали относительно оценивания ММП можно найти, например, у Кэмерона и Триведи (1998). Интерпретация коэффициентов в обоих случаях одинакова, поскольку $\E[y|\x] = \exp(\xib)$. Модель NB2 рассматривается далее в разделе \ref{sec:20.7}.

Модель NB2 получила широкое распространение в прикладных исследованиях. В силу своей гибкости она обеспечивает высокое качество подгонки для различных типов счетных данных. Отчасти это объясняется тем, что квадратичная дисперсия является хорошей аппроксимацией во многих эмпирических примерах. К сожалению, сложилось негласное правило, что, если не получается подогнать модель Пуассона, то следует использовать отрицательную биномиальную модель. Такой механистический подход является неверным, поскольку низкое качество модели Пуассона может объясняться, например, неправильной спецификацией функции условного математического ожидания, а она в обеих моделях совпадает.

Отрицательная биномиальная модель менее робастна к мисспецификации распределения по сравнению с моделью Пуассона. Даже при верной спецификации условного математического ожидания оценки ММП в отрицательных биномиальных моделях несостоятельны, за исключением модели NB2, где оценки ММП для $\be$ (но не для $\al$) сохраняют состоятельность.

В качестве исходной плотности распределения $f(y|\mu, \nu)$ в моделях смеси для счетных данных логично выбрать плотность распределения Пуассона, поскольку процесс Пуассона является логичным и естественным способом моделирования счетных данных. В отличие от исходного распределения, выбор смешиваемого распределения $g(\nu)$ в (\ref{eq:20.12}) не имеет подобных оснований и остается за исследователем. Обсуждение данного вопроса представлено ранее в разделах 18.2--18.4. В роли смешиваемых распределений могут выступать также лог-нормальное и обратное гауссовское распределения. См. работу Виллмота (1987) и Гуо и Триведи (2002). В таком случае маргинальное распределение не имеет аналитической формы, и для оценивания требуется использовать такие методы, как имитационное максимальное правдоподобие. Современные вычислительные мощности легко позволяют оценивать такие модели. Оценка различных типов смешанных моделей Пуассона становится возможной при использовании методов на основе симуляций, представленных в главе 12.


\subsection{Имитационное максимальное правдоподобие}\label{sec:20.4.2}

\noindent
Исключительно в иллюстративных целях здесь мы покажем, как оценивать модель NB2 \textbf{методом симуляционного максимального правдоподобия}. Следует понимать, что на практике в этом нет необходимости, поскольку данная модель уже представлена в аналитическом виде. Предположим, что мы не этого не знаем и попробуем оценить ее с помощью симуляции.

Заметим, что $h(y|\al, \mu)$ в уравнении (\ref{eq:20.12}) может быть аппроксимировано выражением
    $$\frac{1}{S}\sum^{S}_{s = 1} \frac{e^{-\mu\nu_S}(\mu\nu_S)^y}{y!},$$
где $\nu_S(s = 1, \ldots , S)$ являются псевдо-случайными реализациями распределения $g(\nu|\al)$, а $S$ обозначает число симуляций. Случайные реализации гамма распределения с математическим ожиданием 1 и дисперсией $\al$ можно получить, выбрав значение из равномерного распределения и применив к нему соответствующее преобразование. Пусть $u_S$ обозначает равномерно распределенные случайные величины, а $\nu_S = -\ln u_s/\al$, тогда формула для симуляции будет выглядеть следующим образом
    $$\tilde{f}(y|\nu_S, \al, \mu)\frac{e^{-\mu(-\ln u_S / \al)}(\mu(-\ln u_S/\al))^y}{y!}.$$
Следовательно, оценка $\hat{\ttt}_{\mathrm{MSL}}$ максимизирует сумму
    \begin{align}\label{eq:20.14}
    Q_N(\bttt) = \sum^{N}_{i = 1}\ln \left( \frac{1}{S}\sum^{S}_{s = 1}\tilde{f}(y_i|\x_i, u^S_i, \bttt)\right),
    \end{align}
где $\mu_i = \exp(\xib)$ и $\bttt = (\al, \be)$.

Такой метод прост в применении, но требует компьютерных вычислений. Полное описание свойств метода симуляционного максимального правдоподобия представлено ранее в разделе 12.4. % Опечатка в Кэмероне (не глава, а раздел)
Следует также обратить внимание, что при $S, N \rightarrow \infty$, $S/\sqrt{N} \rightarrow 0$ оценки $\hat{\bttt}_{\mathrm{MSL}}$ и $\hat{\bttt}_{\mathrm{ML}}$ асимптотически эквивалентны.\footnote{MSL обозначает метод симуляционного максимального правдоподобия, а ML --- метод максимального правдоподобия.}


\subsection{Модели конечной смеси}\label{sec:20.4.3}

\noindent
Модель смеси, рассмотренная в предыдущем разделе, представляла собой модель непрерывной смеси, так как смешиваемая случайная величина $\nu$ имела непрерывное распределение. Альтернативный подход предполагает дискретный вид ненаблюдаемой гетерогенности, что порождает класс моделей, называемых \textbf{моделями конечной смеси} (\textit{finite mixture models}); см. раздел 18.5. Этот набор моделей  является частным случаем \textbf{моделей латентных классов}. 
Некоторые типы и частные случаи таких моделей также называются \textbf{дискретными факторными моделями}.

В прикладных исследованиях наиболее распространенной альтернативой непрерывным моделям являются модифицированные модели счетных данных, рассмотренные в следующем разделе. Однако, прежде чем перейти к анализу таких моделей, имеет смысл представить модели конечной смеси как логичное продолжение анализа счетных данных. Позже мы покажем, что подкласс модифицированных моделей является частным случаем конечных смесей.

Предположим, что плотность распределения $y$ является линейной комбинацией $m$ различных плотностей, где $j$-ая плотность равняется $f_j(y|\bttt_j)$, $j = 1, 2, \ldots , m$. Тогда конечная смесь с $m$ компонентами записывается как
    \begin{align}\label{eq:20.15}
    f(y|\bttt, \bpi) = \sum^{m}_{j = 1} \pi_j f_j (y|\bttt_j), \hspace{0.5cm} 0 \le \pi_j \le 1, \sum^{m}_{j = 1} \pi_j = 1.
    \end{align}

Данная формулировка представляет собой общий случай, когда все компоненты смеси различаются по параметрам. Менее общие модели допускают вариацию среди компонент только некоторых параметров (например, свободного члена), в то время как все остальные параметры одинаковы для каждой компоненты.

Для более простого понимания рассмотрим случай с количеством компонент $m = 2$. Предположим, что совокупность делится на два ``типа'' наблюдений, где исходы $y$ следуют распределениям $f_1 (y|\bttt_1)$ и $f_2 (y|\bttt_2)$ с различными моментами распределения. Предположим, что среднее для 1-го типа совокупности равно $\mu(\bttt_1)$, а для 2-го --- $\mu(\bttt_2)$, где $\mu(\bttt_2) < \mu(\bttt_1)$. В контексте исследования потребления медицинских услуг первая группа может относиться, например, к индивидам, привыкшим посещать врача регулярно, а вторая --- к индивидам, посещающим врача не так часто. Предположим, что доли обеих групп в совокупности равны, соответственно, $\pi_1$ и $\pi_2 = 1 - \pi_1$. Тогда случайная выборка будет содержать $\pi_1$ и $\pi_2$ процентов наблюдений первого и второго типов, хотя мы и не сможем наблюдать, к какой группе относится каждое наблюдение. Таким образом, ``типы'' являются \textbf{латентными классами}.

В данном контексте цель исследователя заключается в оценивании неизвестных параметров $\bttt_j$, $j = 1, \ldots , m$. Модели регрессии легко построить на основе уравнения (\ref{eq:20.15}). Например, для модели NB2 $f_j(y|\bttt_j)$ является плотностью распределения (\ref{eq:20.12}) с параметрами $\mu_j = \exp(\xb_j)$ и $\al_j$, где $\bttt_j = (\be_j, \al_j)$. Если число компонент $m$ известно, то можно оценить параметры $(\pi, \bttt_j)$, $j = 1, \ldots , m$ методом максимального правдоподобия при некоторых условиях регулярности.

Полное обсуждение преимуществ и недостатков моделей конечной смеси в контексте моделей времени жизни было представлено ранее в разделе 18.5. Здесь же мы упомянем их вкратце. Во-первых, модель конечной смеси содержит ограниченное число параметров, но при этом является довольно гибким методом моделирования данных, поскольку каждая компонента смеси представляет собой локальное приближение некоторой части истинного распределения. Во-вторых, подход на основе конечной смеси в некотором смысле \textbf{полупараметрический}, так как не требует каких-либо предпосылок о распределении смешиваемой переменной. Наконец, в большинстве случаев результаты легко интерпретируются. Модель особенно привлекательна, если исследователь заинтересован в поведении групп(ы) для анализа проводимой политики. Если забыть про латентные классы, что соответствует случаю $m = 1$, то оценки параметров будут равны взвешенным суммам по параметрам латентного класса.

Трудности могут заключаться в отсутствии теоретического обоснования для выбора числа компонент; более того, некоторые компоненты могут быть неразличимы друг с другом из-за отсутствия существенных различий в данных. В таком случае поступают следующим образом: обычно начинают с небольшого количества компонент, а затем добавляют дополнительные компоненты, если такое действие сопровождается значительным повышением качества модели. Иногда допускается вариация только свободного члена, в то время как коэффициенты наклона должны быть одинаковы между компонентами. Однако следует быть аккуратным, поскольку выборочные свойства оценок максимального правдоподобия полностью не изучены для случая с неизвестным числом параметров $m$.

Существует ряд работ, свидетельствующих о хорошем качестве моделей конечной смеси для счетных данных по медицинскому обслуживанию (Деб и Триведи, 1997; 2002). Это можно объяснить, например, тем, что совокупность индивидов делится на основе латентной переменной, отвечающей за состояние здоровья. В среднем, здоровые индивиды могут предъявлять низкий спрос на медицинские услуги, в то время как менее здоровые будут поддерживать высокий спрос. Другими словами, модели конечной смеси позволяют разделить совокупность на группы, когда состояние здоровья полностью ненаблюдаемо.


\subsection{Урезанные и цензурированные данные}\label{sec:20.4.4}

\noindent
В некоторых исследованиях в выборку включаются только те наблюдения, которые удовлетворяют интересующим характеристикам. Поскольку мы не наблюдаем объекты, которые им не удовлетворяют, и, как следствие, объекты с нулевыми значениями, счетные данные оказываются \textbf{урезаны} (усечены). Примером таких данных может быть число поездок, совершенных автобусом в течение недели, число походов в торговый центр индивидом на основе опроса, проводимого в торговом центре, или число периодов безработицы среди запаса (совокупности) безработных. Во всех этих случаях мы не наблюдаем нулевые значения, поэтому мы говорим, что такие данные содержат \textbf{урезанные нулевые значения}. В общем случае, такие данные называются урезанными слева. Урезание (усечение) справа возникает из-за потери наблюдений, значение которых превышает определенную величину.

Полное описание урезанных и цензурированных моделей, оцениваемых ММП, представлено в разделе 16.2. Здесь мы рассмотрим их применение по отношению к счетным данным.

Урезание приводит к несостоятельным оценкам параметров в случае, если отсутствует соответствующая поправка для функции правдоподобия. Рассмотрим пример с урезанными нулевыми значениями. Пусть $f(y|\bttt)$ обозначает функцию плотности, а $F(y|\bttt) = \Pr[Y \le y]$ --- кумулятивную функцию распределения дискретной случайной величины, где $\bttt$ --- вектор параметров. Если реализация $y$, не превышающая единицу, пропущена, то плотность с учетом урезанных нулей будет равна
    \begin{align}\label{eq:20.16}
    f(y|\bttt, y \ge 1) = \frac{f(y|\bttt)}{1 - F(0|\bttt)}, \hspace{0.5cm} y = 1, 2, \ldots  .
    \end{align}
На основе уравнения \ref{eq:20.16} можно построить модель Пуассона с \textbf{урезанными нулевыми значениями}, например, с функцией $f(y|\mu, y \ge 1) = e^{-\mu}\mu^y / [y!(1 - \exp(-\mu))]$. Тогда легко найти логарифм правдоподобия и получить оценки ММП.

\textbf{Цензурированные счетные данные} чаще всего возникают из-за агрегирования счетных данных, значение которых превышает определенную величину. Агрегирование характерно для обследований, где общая вероятность по агрегированным данным относительно невелика.
Важное различие между урезанными и цензурированными наблюдениями заключается в том, что при цензурировании ковариаты наблюдаемы, в то время как при урезании ни исходы, ни регрессоры ненаблюдаемы. Ценузирование также приводит к несостоятельным оценкам параметров, если применяется ошибочная функция правдоподобия для нецензурированных данных; см. раздел 16.2.

Например, число событий, превышающее некоторое пороговое значение $c$, может быть объединено в одну группу. Тогда некоторые значения $y$ будут наблюдаемы неполностью: для определенного наблюдения будет известно лишь то, что его значение равно или превышает некую константу $c$. Тогда плотность наблюдаемых данных будет равна
    \begin{align}\label{eq:20.17}
    g(y|\bttt) =\begin{cases}
                f(y|\bttt),              & \text{ если }y < c, \\
                1 - F(c - 1|\bttt),      & \text{ если }y \ge c
                \end{cases}
    \end{align}
при известном значении $c$.

Подобные трудности возникают при \textbf{отборе выборки} (Терца, 1998). Мы наблюдаем счетную переменную $y$ только тогда, когда другая случайная величина, возможно коррелированная с $y$, превышает определенное пороговое значение. Например, попасть к врачу-специалисту можно лишь предварительно записавшись к терапевту, который и выписывает соответствующие направления.


\subsection{Модифицированные модели счетных данных}\label{sec:20.4.5}

\noindent
Задача модифицированных моделей счетных данных, рассматриваемых в данном разделе, заключается в том, чтобы решить так называемую проблему \textbf{избыточных нулевых значений}, возникающую, когда в данных содержится больше нулей, чем предсказывает модель Пуассона или NB2.

        \begin{center}{Модель преодоления порогов}\end{center} % или модель, состоящая из двух частей
\noindent
\textbf{Модель преодоления порогов} (\textit{hurdle model}), или \textbf{двухчастная модель} (\textit{two-part model}) (см. раздел 16.4), снимает % убирает/отказывается/избавляется/ослабляет
предпосылку о том, что нулевые и ненулевые значения порождаются одним и тем же процессом.
В то время как нули распределены с плотностью $f(\cdot)$ так, что $\Pr[y = 0] = f_1(0)$, положительные значения имеют усеченную плотность распределения $f_2(y|y > 0) = f_2(y) / (1 - f_2(0))$, умноженную на $\Pr[y > 0] = 1 - f_1(0)$, что гарантирует, что сумма вероятностей равняется единице. Таким образом,
    \begin{align}\label{eq:20.18}
    g(y) =\begin{cases}
                f_1(0),                                     & \text{ если }y = 0, \\
                \frac{1 - f_1(0)}{1 - f_2(0)} f_2(y),       & \text{ если }y \ge 1.
                \end{cases}
    \end{align}
Следовательно, в модифицированной модели процессы, порождающие нулевые и положительные значения, различаются. Стандартная модель получается при $f_1(\cdot) = f_2(\cdot)$. Хотя модель и построена таким образом, чтобы решить проблему избыточных нулей, она также подходит для ситуаций, когда количество нулей слишком мало.

Оценивание модели преодоления порогов методом максимального правдоподобия подразумевает максимизацию двух частей правдоподобия по отдельности, одна из которых отвечает за нулевые значения, а другая за положительные.

Интерпретация модели преодоления порогов заключается в том, что она отражает двухэтапный процесс принятия решения. Например, решение о первичном посещении врача может быть принято самим пациентом, но второй и последующие визиты могут определяться другими механизмами (Полмейер и Улрих, 1995).

В регрессионном анализе применяются различные варианты модели преодоления порогов, построенные на основе спецификации распределений $f_1(\cdot)$ и $f_2(\cdot)$ в виде плотности распределений Пуассона или отрицательного биномиального, представленных ранее. Набор регрессоров в первой части модели не должен совпадать с набором во второй (усеченной), хотя на практике они часто одинаковы. В силу своей гибкости отрицательная биномиальная модель преодоления порогов получила широкое применение в анализе счетных данных. Вместе с этим она требует оценивания множества параметров, количество которых обычно удваивается по сравнению со стандартной моделью. При этом, их интерпретация также не всегда очевидна.

Выбор распределения в модели преодоления порогов играет важную роль. Использование более гибких распределений создает преимущества для отрицательной биномиальной модели перед моделью Пуассона. Условное математическое ожидание в модели преодоления препятствия является произведением вероятности положительных значений и условного математического ожидания для плотности с урезанными нулевыми значениями. Следовательно, применение регрессии Пуассона, в то время, когда модель преодоления порогов является истинной моделью, приводит к мисспецификации, и, как следствие, к несостоятельным оценкам. Более того, из-за формы условного среднего, расчет предельных эффектов оказывается сложнее, аналогично модели, состоящей из двух частей, из раздела 16.4.

        \begin{center}{Модель с нулевыми значениями}\end{center}
\noindent
Следующей модифицированной моделью является \textbf{модель с раздутым нулём} (\textit{with-zeros model} или \textit{zero-inflated model}). Идея заключается в том, что бинарный процесс с плотностью $f_1(\cdot)$ дополняет плотность распределения счетных данных $f_2(\cdot)$. То есть, если бинарный процесс принимает значение $0$ с вероятностью $f_1(0)$, то $y = 0$. Если же значение равно $1$ с вероятностью $f_1(1)$, то $y$ может принимать любые целые неотрицательные значения $0, 1, 2, \ldots $, в соответствии с плотностью распределения $f_2(\cdot)$. Таким образом, нули могут возникать двумя способами: как реализация бинарного процесса и как реализация счетного процесса при условии, что бинарный процесс принимает значение $1$. Следовательно, плотность равна
    \begin{align}\label{eq:20.19}
    g(y) =\begin{cases}
                f_1(0) + (1 - f_1(0)) f_2(0),              & \text{ если }y = 0, \\
                (1 - f_1(0))f_2(y),                        & \text{ если }y \ge 1.
                \end{cases}
    \end{align}
В регрессионных моделях плотность $f_1(\cdot)$ задается логит моделью, а $f_2(\cdot)$ --- плотностью распределения Пуассона или отрицательного биномиального. Данная модель менее популярна, чем модель преодоления порогов. При этом, она также подходит для моделирования ситуаций с недостаточным количеством нулей.

Заметим также, что в эконометрике счетные модели с нулевыми значениями используются гораздо реже, чем в других статистических дисциплинах.


\subsection{Модели дискретного выбора}\label{sec:20.4.6}

\noindent
Для моделирования счетных данных также подходят модели дискретного выбора. Предварительная группировка данных может потребоваться, чтобы ограничить число категорий. Например, если незначительное число наблюдений превышает четыре, то данные можно сгруппировать по пяти категориям как 0, 1, 2, 3 и 4 и более. Неупорядоченные модели, такие как логит модель множественного выбора из раздела 15.4, используют слишком много параметров и, что более важно, не подходят для работы со счетными данными. Следовательно, нужно использовать последовательные модели, то есть, те, которые учитывают порядок.

Одной из таких моделей является \textbf{модель упорядоченного множественного выбора} (\textit{ordered model}). % Термин должен совпадать!!!
Зависимая переменная принимает значения $y = 0, 1, 2, \ldots $, соответствующие пороговым значениям латентной переменной $y^* = \xb + u$, которые также подлежат оценке. Логит (или пробит) модель упорядоченного множественного выбора можно получить, предположив, что $u$ следует логистическому (или стандартному нормальному) распределению. Упорядоченные модели (см. раздел 15.9) особенно полезны в случае, если счетные данные могут принимать отрицательные значения, как, например, при моделировании изменений, таких как изменение числа фирм в отрасли.

Другую последовательную модель, хотя и с б\'{о}льшим числом оцениваемых параметров, можно построить на основе последовательности моделей бинарного выбора с вероятностями $\Pr[y = 1|y \ge 0]$, $\Pr[y = 2|y \ge 1]$ и так далее.

Наконец, в некоторых случаях помимо счетных данных могут быть доступны данные по длительностям. Например, если известны даты посещения врача, то можно смоделировать как число визитов в месяц, так и длительности между визитами. В общем случае, последний подход более эффективен, поскольку он использует более детальные данные. Однако регрессия для счетных данных также предоставляет полезную информацию о роли регрессоров (Дин и Балшоу, 1997).




\section{Частично параметрические модели}\label{sec:20.5}

\noindent
Частично параметрические модели подразумевают моделирование данных с помощью условных математического ожидания и дисперсии, однако даже их спецификация может быть неполностью определена. В разделе \ref{sec:20.5.1} мы рассмотрим модели, основанные на спецификации условных математического ожидания и дисперсии. В разделе \ref{sec:20.5.2} мы представим применение методов наименьших квадратов и их критику. В разделе \ref{sec:20.5.3} мы рассмотрим еще менее параметрические модели с неполной спецификацией условного математического ожидания.

Такой подход аналогичен нелинейному МНК, за исключением того, что мы допускаем наличие гетерогенности, заданной в виде функции от условного математического ожидания.


\subsection{Оценивание квази-ММП}\label{sec:20.5.1}

\noindent
Как было сказано в разделе \ref{sec:20.2.1}, при использовании псевдо- или квази-ММП итоговое распределение оценок не требует столь же строгих предпосылок о процессе, генерирующем данные, что и определенная функция правдоподобия.

Пересмотрим формулу (\ref{eq:20.6}). При определенной предпосылке о функциональной форме $\omega_i$ и состоятельной оценке $\hat{\omega_i}$ для $\omega_i$ можно получить состоятельную оценку матрицы ковариаций. В качестве предпосылки можно использовать распределение Пуассона с $\omega_i = \mu_i$, однако, как уже упоминалось, для счетных данных характерна избыточная дисперсия, что означает, что $\omega_i > \mu_i$. Другими распространенными функциями являются $\omega_i = (1 + \al \mu_i) \mu_i$ в модели NB2, представленной в разделе \ref{sec:20.4.2}, и $\omega_i = (1 + \al) \mu_i$ в модели NB1. Заметим, что в последнем случае выражение (\ref{eq:20.6}) упрощается до $\V_{\mathrm{PML}}[\hat{\be}_{\mathrm{P}}] = (1 + \al) (\sum_i \mu_i \x_i \x'_i)^{-1}$, следовательно, при наличии избыточной дисперсии $(\al > 0)$ матрица ковариаций (\ref{eq:20.7}) недооценивает истинную дисперсию.

Если спецификация функциональной формы $\omega_i = \E[(y_i - \xib)^2|\x_i]$ отсутствует, то состоятельную оценку $\V_{\mathrm{PML}}[\hat{\be}_{\mathrm{P}}]$ можно получить, применив формулу Эйкера–-Уайта (\textit{Eicker--White}). Сумма в середине уравнения (\ref{eq:20.6}) подлежит оцениванию. Если $\hat{\mu}_i \xrightarrow{\text{p}} \mu_i$, то $N^{-1} \sum_i (y_i - \hat{\mu}_i)^2 \x_i \x'_i \xrightarrow{\text{p}} \lim N^{-1} \sum_i \omega_i \x_i \x'_i$. Следовательно, произведя замену $\omega_i$ и $\mu_i$ на $(y_i - \hat{\mu}_i)^2$ и $\hat{\mu}_i$ в уравнении (\ref{eq:20.6}), найдем состоятельную оценку для $\V_{\mathrm{PML}}[\hat{\be}_{\mathrm{P}}]$.

В случае если существуют сомнения относительно функциональной формы дисперсии, рекомендуется использование псевдо-ММП. В вычислительном плане данный метод не отличается от ММП Пуассона, с той лишь оговоркой, что матрица дисперсии должна быть пересчитана. Стандартные статистические пакеты, как правило, предоставляют возможность расчета робастных оценок дисперсии.

Результаты оценивания псевдо-ММП Пуассона качественно похожи на оценивание линейной модели псевдо-ММП при условии нормальности. В общем случае, их можно обобщить до оценивания псевдо-ММП на основе плотностей экспоненциального семейства распределений. Так или иначе, для состоятельности требуется только верная спецификация условного математического ожидания (Нелдер и Веддербёрн, 1972; Гурьеру и др., 1984a), что позволяет осуществлять корректные выводы и строить различные модели --- непрерывную (нормальное распределение), счетных данных (распределение Пуассона), дискретную (биномиальное распределение), положительную (гамма распределение), что представлено в разделе 5.7.4. Это является основой для значительного объема литературы, посвященной обобщенным линейным моделям (МакКуллах и Нелдер, 1989). Многие методы, такие как временные ряды и панельные данные, представлены в контексте обобщенной линейной модели (\textit{generalized linear model, GLM}).

Некоторые эконометристы считают, что вместо обобщенной линейной модели логично использовать обобщенный метод моментов (\textit{generalized method of moments, GMM}), где исходной точкой для анализа является условный момент $\E[y_i - \exp(\xib)|\x_i] = \0$. Если данные независимы по наблюдениям $i$ и условная дисперсия является функцией от математического ожидания, то можно показать, что оптимальный набор инструментов равен $\x_i$, что в результате приводит к оцениваемым уравнениям (\ref{eq:20.5}); детали можно найти у Кэмерона и Триведи (1998, стр. 37--44). Обобщенный метод моментов хорошо зарекомендовал себя при работе с панельными счетными данными (см. раздел \ref{sec:20.5.3}) и \textbf{эндогенными регрессорами}. Анализ полностью параметрических моделей с одновременными уравнениями для счетных данных еще только начинает развиваться, поэтому методы инструментальных переменных выглядят довольно многообещающе. При данных инструментах $\bz_i$, $\dim (\bz) \ge \dim(\x)$, удовлетворяющих условию $\E[y_i - \exp(\xib)|\bz_i] = \0$, состоятельная оценка для $\be$ минимизирует выражение
    \begin{align}\label{eq:20.20}
    Q(\be) = \left[ \sum^{N}_{i = 1} (y_i - \exp(\xib))\bz_i\right]' \mathbf{W} \left[ \sum^{N}_{i = 1} (y_i - \exp(\xib))\bz_i\right],
    \end{align}
где $\mathbf{W}$ является симметричной взвешивающей матрицей.

Среди преимуществ такого подхода можно выделить небольшое число предпосылок о распределениях, что позволяет избежать возможной мисспецификации. Однако, данный подход не учитывает дискретность данных и свойственную им гетероскедастичность, что приводит к потере эффективности. Проблему можно отчасти решить с помощью соответствующей матрицы $\mathbf{W}$. Более того, акцентируя внимание на первом моменте распределения, можно упустить важную дополнительную информацию, которая содержится в моментах более высокого порядка, в результате чего инструментальная оценка будет чувствительна к наличию больших чисел. В таблице \ref{tab:20.2} представлены некоторые особенности типов данных, для работы с которыми обобщенный метод моментов не подходит.


\subsection{Оценивание МНК}\label{sec:20.5.2}

\noindent
При моделировании только условного математического ожидания методы наименьших квадратов уступают подходам, рассмотренным в предыдущем разделе.

Оценки параметров в \textbf{линейной регрессии МНК} $y$ на $\x$ состоятельны, если условное математическое ожидание линейно по $\x$. Однако, спецификация $\E[y|\x] = \xib$ неадекватна счетным данным, так как допускает отрицательные значения $\E[y|\x]$. По той же причине линейная модель вероятности не подходит для бинарных данных.

Для учета неотрицательности можно рассматривать функциональные преобразования $y$, в частности, логарифмическое, что соответствует регрессии $\ln y$ на $\x$. Однако, такое преобразование может быть проблематично, так как счетные данные обычно содержат нулевые значения. В качестве стандартного решения можно прибавить константу, например $0.5$, и работать уже с $\ln (y + .5)$. В свою очередь, такой специальный метод связан с проблемами обратного преобразования, если мы заинтересованы в анализе $\E[y|\x]$, а не $\E[\ln y|\x]$; (Маллахай, 1998).
Тем не менее, переход к линейной модели удобен, если в уравнении содержится эндогенная правая часть, которая требует применения инструментальных методов.

Вместо логарифмического преобразования можно использовать нелинейный МНК (НМНК) с экспоненциальным математическим ожиданием, что соответствует оцениванию нелинейной регрессии $y = \exp(\xib) + u$. Важно, чтобы статистические выводы были основаны на робастных стандартных ошибках Эйкера--Уайта, поскольку ошибка в данной регрессии гетероскедастична.

Как правило, нелинейный МНК на счетных данных менее эффективен, чем псевдо-ММП Пуассона. Условие первого порядка для НМНК выглядит как $\sum_i (y_i - \exp(\xib))\exp(\xib)\x = \0$. То есть, остатки при оценивании НМНК взвешены иначе, чем при оценивании псевдо-ММП Пуассона (см. (\ref{eq:20.5})). Веса НМНК оптимальны, если дисперсия $\V[y_i|\x_i]$ постоянна (гомоскедастична), веса псевдо-ММП оптимальны, если $\V[y_i|\x_i]$ является функцией от $\E[y_i|\x_i]$. Следовательно, последняя модель лучше учитывает свойственную счетным данным гетероскедастичность.


\subsection{Полупараметрические модели}\label{sec:20.5.3}

\noindent
Под \textbf{полупараметрическими моделями} подразумеваются частично параметрические модели, содержащие компоненту с бесконечной размерностью и представленные в разделе 9.7. Проклятие размерности служит основанием для того, чтобы определять функциональную форму математического ожидания.

Один из классов полупараметрических моделей, включающий одноиндексные и частично линейные модели, предполагает неполную спецификацию условного математического ожидания. В частности, одноиндексные модели определяют $\mu_i = g(\xib)$, где $g(\cdot)$ неизвестна. Частично линейные модели определяют $\mu_i = \exp(\xib + g(\bz_i))$, где функциональная форма $g(\cdot)$ остается неопределена. В обоих случаях $\sqrt{N}$-состоятельные асимптотически нормальные оценки $\be$ могут быть получены при неизвестной $g(\cdot)$

Другой класс моделей предполагает, что $\mu_i = \exp(\xib)$, в то время как $\V[y_i|\x] = \omega_i$ остается неизвестной. Бесконечная размерность имеет место, так как при $N \rightarrow \infty$ существует бесконечное множество параметров дисперсии $\omega_i$. Оптимальная оценка для $\be$, называемая адаптивной оценкой, обладает той же эффективностью, что и оценка при известном параметре $\omega_i$. Делгадо и Книснер (1997) рассматривают случай линейной модели регрессии для счетных данных с экспоненциальной функцией условного математического ожидания, используя методы ядерной регрессии для оценки весов, применяемых на втором шаге оценивания регрессии НМНК. Такая оценка обладает незначительным преимуществом по сравнению со спецификацией избыточной дисперсией в форме NB2, $\omega_i = \mu_i (1 + \al\mu_i)$.


\section{Многомерные счетные данные и эндогенные регрессоры}\label{sec:20.6}

\noindent
В данном разделе мы представим краткое описание моделей для пространственных данных, обобщенных на другие типы счетных данных (для дальнейшего анализа см. работу Кэмерона и Триведи, 1998). Несмотря на то, что существует значительное количество моделей для анализа многомерных счетных данных, какие-либо общепринятые методы отсутствуют. Относительно панельных данных в эконометрической литературе больше договоренности о том, какие методы использовать, хотя в статистической литературе рассматривается еще более широкий круг моделей; см. раздел 23.7.


\subsection{Многомерные данные}\label{sec:20.6.1}

\noindent
Иногда данные содержат информацию по нескольким переменным. Например, могут быть доступны данные по потреблению различных медицинских услуг, таких как число визитов к врачу и число дней, проведенных в госпитале. В случае если такие переменные коррелированы, совместное моделирование повышает эффективность оценок и позволяет строить более качественные модели. Данный раздел представляет собой краткий обзор \textbf{двумерных моделей счетных данных}, построенных на основе базовых моделей, представленных в этой главе. Читатель, знакомый с такими моделями, как \textbf{модель с внешне не связанными уравнениями} (\textit{SUR}) из раздела 6.9.3, может сопоставить их с моделями счетных данных, содержащими несколько уравнений с коррелированными ошибками. Предположим, что для определенного индивида мы наблюдаем несколько переменных (например, число посещений врача и число прописанных лекарств), причина коррелированности которых может заключаться в ненаблюдаемой гетерогенности. Совместное оценивание, учитывающее коррелированность ошибок, позволяет получить более эффективные оценки, однако для этого потребуются дополнительные компьютерные вычисления.

        \begin{center}{Полупараметрические методы}\end{center}
        \noindent
Частично параметрический подход можно рассматривать как задачу оценки внешне не связанных уравнений с применением методов для оценки линейных моделей регрессии к счетным данным, с учетом гетероскедастичности и нелинейности условных математических ожиданий; см. раздел 6.10.3.

Гурьеру, Монфорт и Троньон (1984b) предлагают основанный на моментах метод, который позволяет получить двумерную модель типа Пуассона. В частности, авторы определяют первые два момента распределения $y_1$ и $y_2$ и оценивают их квази-обобщенным псевдо-ММП. Такая модель является более общей, чем модель Пуассона, и допускает наличие избыточной дисперсии; однако она не учитывает целочисленность счетных данных.

Делгадо (1992) рассматривает многомерную модель счетных данных как многомерную нелинейную модель и предлагает полупараметрический обобщенный МНК. Матрица ковариаций остатков оценивается с помощью метода $k$-NN. Этот подход отличается от подхода Гурьеру, Монфорта и Троньона (1984b) типом оценки матрицы ковариаций.

Многие параметрические исследования используют \textbf{двумерное распределение Пуассона}. Распределение можно найти, предположив, например, что две счетных переменных $y_1$ и $y_2$ получены как $y_1 = \zeta_1 + \omega$ и $y_2 = \zeta_2 + \omega$, где все $\zeta_1$, $\zeta_2$ и $\omega$ независимы и распределены по Пуассону с положительными параметрами $\la_1$, $\la_2$ и $\la_{12}$, соответственно. Эти параметры, в свою очередь, задаются в виде функции от экзогенных ковариат. Такой метод называется \textbf{трехмерной редукцией} (\textit{trivariate reduction}).

Маргинальное распределение $y_j$ является распределением Пуассона с параметром $[\la_j + \la_{12}]$.
% $\sim \mathrm{Poisson}[\la_j + \la_{12}]$ ???
Следовательно, эта модель предполагает равенство условных математического ожидания и дисперсии для каждой счетной переменной, то есть
    \begin{align}\label{eq:20.21}
    \E[y_j|\x_j] = \V[y_j|\x_j]
    \end{align}
для $j = 1, 2$, где $\x_j$ обозначает вектор объясняющих переменных. Так как $\la_{12} > 0$, коэффициент корреляции положителен и равен
    \begin{align}\label{eq:20.22}
    \mathrm{Cor}[y_1, y_2] = \frac{\la_{12}}{\sqrt{(\la_1 + \la_{12})(\la_2 + \la_{12})}}.
    \end{align}


        \begin{center}{Полностью параметрические методы}\end{center}
        \noindent
Параметрические модели можно улучшить за счет включения коррелированной ненаблюдаемой гетерогенности для каждой счетной переменной. С этой идеей тесно связаны вопросы, рассмотренные в разделах 6.10.1 и 19.3.

Маршалл и Олкин (1990) рассматривают следующую модель с \textbf{мультипликативной ненаблюдаемой гетерогенностью} в маргинальных распределениях обеих счетных переменных. Пусть $y_j \sim \mathcal{P}[\la_j\nu]$, $j = 1, 2$, где $\mathcal{P}$ обозначает распределение Пуассона с математическим ожиданием $\la_j\nu$. Гетерогенность $\nu$ имеет гамма распределение с плотностью
        $$g(\nu) = \frac{\nu^{\al - 1}\exp(-\nu)}{\Ga(\al)}.$$
Случайную переменную $\nu$ можно интерпретировать как общую (распределенную) ненаблюдаемую гетерогенность. Итоговая модель является \textbf{однофакторной моделью}. \textbf{Двумерное отрицательное биномиальное распределение} (\textit{BVNB}) двух счетных переменных задается как
    \begin{align}\label{eq:20.23}
    f(y_1, y_2|\x_1, \x_2)  &= \int^{\infty}_{0} f_1(y_1|\x_1, \nu) f_2(y_2|\x_2, \nu) g(\nu) d\nu \\
                            &= \int \left[ \prod^2_{j = 1}\frac{\exp(-\la_j\nu)(\la_j\nu)^{y_j}}{y_j!}  \right] \frac{\nu^{\al - 1} \exp(-\nu)}{\Ga(\al)} d\nu \notag \\
                            &= \frac{\Ga(y_1 + y_2 + \al)}{y_1! y_2! \Ga(\al)} \left[ \frac{\la_1}{\la_1 + \la_2 + 1} \right]^{y_1} \left[ \frac{\la_2}{\la_1 + \la_2 + 1} \right]^{y_2} \notag \\
                            &\times \left[ \frac{1}{\la_1 + \la_2 + 1} \right]^{\al}. \notag
    \end{align}

Данная смесь имеет аналитическое решение, однако ненаблюдаемая гетерогенность должна быть идентична для обеих счетных переменных. Совместное правдоподобие состоит из элементов, аналогичных (\ref{eq:20.23}). Маргинальные распределения представляют собой одномерные отрицательные биномиальные распределения. Корреляция между переменными положительна и равна
    \begin{align}\label{eq:20.24}
    \mathrm{Cor}[y_1, y_2] = \frac{\la_1 \la_2}{\sqrt{(\la_1^2 + \al \la_1)(\la_2^2 + \al \la_2)}}.
    \end{align}

Другие модели с более \textbf{гибкой структурой корреляций}, но при этом требующие применения продвинутых вычислительных методов, были предложены в работах Кэмерона и Йоханссона (1998), Мункина и Триведи (1999) и Чиба и Винкелманна (2001).

Мункин и Триведи (1999) рассматривают обобщенную BVNB модель в виде
    \begin{align}\label{eq:20.25}
    f(y_1, y_2 | \x_1, \x_2) = \int^{\infty}_{0} \int^{\infty}_{0} f_1(y_1|\x_1, \nu_1) f_2(y_2|\x_2, \nu_2) g(\nu_1, \nu_2) d\nu_1 d\nu_2,
    \end{align}
где совместное распределение состоит из двух маргинальных моделей, каждая из которых определена при условии соответствующей переменной ненаблюдаемой гетерогенности, $\nu_1$ или $\nu_2$. Гетерогенность задается таким образом, чтобы получалось двумерное нормальное распределение. При условии $(\x_1, \x_2, \nu_1, \nu_2)$ каждое маргинальное распределение является распределением Пуассона с мультипликативной ненаблюдаемой нормальной гетерогенностью. Следовательно, такая модель называется \textbf{двумерной смесью Пуассона и лог-нормального распределений}. Функция правдоподобия представляет собой произведение элементов, аналогичных (\ref{eq:20.25}). В отличие от предыдущей, данная модель является ``\textbf{двухфакторной моделью}''. Такая спецификация является более гибкой, поскольку не накладывает ограничений на размер или знак коэффициента корреляции между двумя ненаблюдаемыми компонентами. Поскольку интеграл в уравнении (\ref{eq:20.25}) не имеет аналитического решения, такая дополнительная гибкость создает трудности, связанные с вычислением, которое подразумевает применение методов, основанных на симуляции (представленных в главе 12 и у Мункина и Триведи (1999)). % опечатка в Кэмероне
При увеличении числа переменных $y$, увеличивается и порядок численного интегрирования, что наряду с большим объемом выборки может создавать серьезные вычислительные проблемы. Чиб и Винкелманн (2001) предлагают альтернативный байесовский MCMC подход, сочетающий как гибкость вышеописанной спецификации, так и высокую размерность. Авторы демонстрируют доступность такого подхода с помощью шестимерной смеси Пуассона и лог-нормального распределений.

Еще одним альтернативным подходом к моделированию коррелированных счетных данных является \textbf{подход на основе копула-функций}, описанный ранее в разделе 19.3. Вначале определяется спецификация маргинальных распределений, а затем за счет их объединения с помощью копула-функции получается совместное распределение. Примеры с зависимыми длительностями представлены в разделе 19.3. См. также работу Кэмерона, Ли, Триведи и Циммера (2004).


\subsection{Модели счетных данных с эндогенными регрессорами}\label{sec:20.6.2}

\noindent
Модели одновременных уравнений для счетных данных подходят для описания различных ситуаций. Например, в работе Кэмерона и др. (1988) в качестве зависимой счетной переменной выступает потребление медицинских услуг, и, как следствие, регрессор, отражающий состояние здоровья объекта, является эндогенным. Маллахай (1997) и Крепон и Дюге (1997b) применяют обобщенный метод моментов к моделям счетных данных с эндогенными регрессорами на пространственных и панельных данных, соответственно. В известном примере по моделированию таких медицинских услуг, как визиты к врачу, один из регрессоров соответствует состоянию здоровья индивида. Поэтому предпосылка о независимости выбора программы страхования и ошибки нереалистична, и, как следствие, переменная страхования является эндогенной. Примеры и подробности для панельных моделей счетных данных с эндогенными регрессорами можно найти в главе 22.

На данный момент в эконометрике существует два подхода к оцениванию моделей с эндогенными регрессорами, один из которых основан на процедуре GMM/IV, а второй использует более строгие предпосылки о максимальном правдоподобии. Мы рассмотрим каждый из них по очереди.

Первый подход (Маллахай, 1997) вначале определяет условия на момент. Рассмотрим модель с экспоненциальным математическим ожиданием, где ошибка аддитивна и обладает нулевым средним
    \begin{align}\label{eq:20.26}
    y_i = \E[y_i|\x_i] + \nu_i = \exp(\xib) + \nu_i,
    \end{align}
    \begin{align}\label{eq:20.27}
    \E[\nu_i|\x_i] \ne 0.
    \end{align}
Предположим, что имеется набор инструментальных переменных $\bz_i$, которые удовлетворяют моментным тождествам
    \begin{align}\label{eq:20.28}
    \E[y_i|\bz_i]    &= 0, \\
    \E[y_i - \exp(\xib)\bz_i] &= 0. \notag
    \end{align}
Тогда процедура GMM/IV реализуема, при условии, что мы имеем достаточно моментных тождеств. Подробности по вопросам ее применения и другим возникающим вопросам можно найти в разделе 6.5.3. Заметим, что данный подход не учитывает счетный характер переменной, поэтому итоговая модель аналогична любой другой нелинейной модели с экспоненциальным математическим ожиданием. Также, в данных, скорее всего, присутствует гетероскедастичность, следовательно, модель GMM/IV должна учитывать и эту характерную черту данных.

Маллахай отметил, что мультипликативная спецификация ошибки имеет определенные преимущества, что, однако, соответствует другим условиям на момент. Пусть
    \begin{align}\label{eq:20.29}
    \E[y_i|\x_i, \nu_i] = \exp(\xib)\nu_i.
    \end{align}
Тогда условие на момент является частным случаем нелинейного моментного тождества $\E[r(y_i, \x_i, \be)|\bz_i] = 0$, рассмотренного в разделе 6.5, и равняется
    \begin{align}\label{eq:20.30}
    \E\left[ \frac{y_i}{\exp(\xib)} - 1| \bz_i \right] = 0,
    \end{align}
Метод GMM применим, если доступны подходящие и достаточные моментные тождества. Повторим, что для счетной переменной свойственна гетероскедастичность и потери эффективности возникают из-за того, что счетный характер переменной проигнорирован.

Альтернативные подходы, учитывающие одновременно и счетность зависимой переменной, и проблему эндогенных регрессоров, являются более \textbf{параметрическими} (Терца, 1998). Деб и Триведи (2004) предложили совместную модель подсчетов $(Y)$ с планом страхования $(D)$ в качестве объясняющей переменной и модель бинарного выбора плана страхования. Эндогенность в данном случае возникает из-за наличия коррелированной ненаблюдаемой гетерогенности в уравнении счетных данных и уравнении бинарного выбора.
Модель имеет следующую структуру:
    \begin{align}\label{eq:20.31}
    \Pr[Y_i = y_i|\x_i, D_i, l_i] = f(\xib + \ga_1 D_i + \la l_i),
    \end{align}

    \begin{align}\label{eq:20.32}
    \Pr[D_i = 1|\bz_i, l_i] = g(\bz'_i \bm{\al} + \de l_i),
    \end{align}
где $l_i$ являются \textbf{латентными факторами}, отражающими ненаблюдаемую гетерогенность, а $\de$ и $\la$ обозначают соответствующую факторную нагрузку. Предполагается, что $(Y, D)$ условно независимы, и совместное распределение зависимой (счетной) переменной и переменной отбора, условных на общие латентные факторы, может быть записано как
    \begin{align}\label{eq:20.33}
    \Pr[Y_i = y_i, D_i = 1|\x_i, \bz_i, l_i] = f(\xib + \ga_1 d_i + \la l_i) g(\bz'_i \bm{\al} + \de l_i).
    \end{align}

Так как $l_i$ неизвестны, при оценивании могут возникнуть проблемы. Однако, хотя мы и не знаем $l_i$, мы можем предположить, что известно их распределение $h$. Следовательно, его можно исключить из функции совместной плотности с помощью интегрирования
    \begin{align}\label{eq:20.34}
    \Pr[Y_i = y_i, D_i = 1|\x_i, \bz_i] = \int [f(\xib + \ga_1 D_i + \la l_i)g(\bz'_i\bm{\al} + \de l_i)]g(l_i)dl_i.
    \end{align}
В такой форме оценки неизвестных параметров модели можно получить методом максимального правдоподобия.

Для простоты предположим, что все параметры распределения $h(l_i)$ известны. Тогда оценки ММП соответствуют максимуму совместной функции правдоподобия $\mL(\bttt_1, \bttt_2|y_i, D_i, \x_i, \bz_i)$, где $\bttt_1 = (\be, \ga_1, \la)$ и $\bttt_2 = (\bm{\al}, \de)$ обозначают набор параметров в уравнении счетных данных и уравнении бинарного выбора, соответственно, а $\mL$ обозначает совместное правдоподобие, где $i$-ая компонента соответствует уравнению (\ref{eq:20.34}). Для идентификации, однако, могут потребоваться дополнительные условия нормализации.

С практической точки зрения основной проблемой является то, что в общем случае интеграл не может быть представлен в аналитическом виде при наличии подходящих распределений $f$, $g$ и $h$. Тогда применяется метод симуляционного максимального правдоподобия, который предполагает замену математического ожидания на симулированные выборочные моменты (средние), то есть,
    \begin{align}\label{eq:20.35}
    \tilde{\Pr}[Y_i = y_i, D_i = 1|\x_i, \bz_i] = \frac{1}{S} \sum^{S}_{s = 1}[f(\xib + \ga_1 D_i +\la \tilde{l}_{is})g(\bz'_i\bm{\al} + \de\tilde{l}_{is})],
    \end{align}
где $\tilde{l}_{is}$ обозначает $s$-ую (из возможных $S$) реализацию псевдо-случайной величины с плотностью $h$, а $\tilde{\Pr}$ --- симулированную вероятность. Оценки симуляционного ММП соответствуют максимуму функции симуляционного правдоподобия, построенной на основе \ref{eq:20.35}.

Хотя такой подход и был изначально предложен для анализа моделей счетных данных с эндогенной объясняющей дамми-переменной, он может быть также применен к многомерным дамми и счетным, дискретным или непрерывным, переменным. Ограничения на применение могут возникать в связи с объемом вычислений, довольно значительным по сравнению с оцениванием типа IV. Кроме того, как и в любой модели одновременных уравнений, под вопросом находится ее идентифицируемость. Также в прикладных работах вектор $\bz$ обычно включает некоторые нетривиальные регрессоры, исключенные из вектора переменных $\x$.




\section{Пример на счетных данных: дальнейший анализ}\label{sec:20.7}

\noindent
В данном разделе мы заново проведем анализ из раздела \ref{sec:20.3}, но с использованием более гибких параметрических моделей, начав с NB2 вместо Пуассона.

Расчеты по модели NB2, включая робастные стандартные ошибки и $t$-статистики, находятся в последних столбцах таблицы \ref{tab:20.5}, представленной в разделе \ref{sec:20.3}. Заметим, что коэффициент, соответствующий наличию избыточной дисперсии, $\al$, статистически значим.
Тестовая статистика Вальда составляет $8.926$, что отвергает нулевую гипотезу о равенстве математического ожидания и дисперсии $(\al = 0)$. С этим также согласуется значительный рост логарифма правдоподобия, с $-60,087$ до $-42,777$, что соответствует существенному повышению качества модели. Поскольку модели являются вложенными, нет необходимости указывать AIC и BIC.

Сравнив первую и третью строки таблицы \ref{tab:20.6}, можно увидеть, что рассчитанная по модели NB2 частота довольно точно предсказывает наблюдаемую, что подтверждает повышение качества модели как результат учета избыточной дисперсии.

Коэффициенты достаточно стабильны по сравнению с альтернативными методами оценивания, и все эффекты измерены с точностью, соответствующей большому размеру выборки. Такие результаты свидетельствуют о том, что применение модели NB2 обосновано. Как предсказывает классическая теория, потребление и коэффициент сострахования (LC) отрицательно коррелированы. При этом, сама оценка коэффициента слабо чувствительна к наличию избыточной дисперсии.

Возможно дальнейшее усовершенствование анализа. Например, Деб и Триведи (2002) сравнивают модель преодоления порогов с моделью двухкомпонентной конечной смеси и показывают, что последняя лучше соответствует данным. Однако, даже модель преодоления порогов показывает лучшие результаты, чем NB2. Но то, что эти модели предоставляют дополнительную информацию, не говорит о том, что результаты, представленные в данной главе, являются неверными в отношении главного вопроса о чувствительности потребления к цене.

Модель NB2 хорошо работает на данных о количестве визитов к врачу. Для других же счетных данных могут потребоваться более гибкие модели, чем NB2.




\section{Практические соображения}\label{sec:20.8}

\noindent
Большинство статистических и эконометрических пакетов позволяют оценивать модель Пуассона. При наличии опыта работы с нелинейным методом наименьших квадратов использование такого статистического программного обеспечения не должно представлять трудностей. Однако необходимо убедиться, что полученные стандартные ошибки являются робастными. Многие эконометрические пакеты также позволяют оценить отрицательную биномиальную регрессию и базовые модели панельных данных. Модели регрессии для счетных данных обычно включены в модули по оцениванию обобщенных линейных моделей. Стандартные пакеты также рассчитывают меры качества подгонки модели для регрессии Пуассона, такие как псевдо-$R^2$, см. раздел 8.7.1.

Применение более новых моделей, в частности, моделей конечной смеси, большинства моделей временных рядов и динамических моделей панельных данных, скорее всего, потребовало бы написания собственного программного модуля на основе матричного языка программирования и программного обеспечения, предназначенного для оценивания задаваемых пользователем целевых функций. Большинство программ умеют оценивать простые модели методом максимального правдоподобия и производить расчет робастной дисперсии для задаваемых пользователем функций.

Помимо оценок коэффициентов, полезно иметь представление о степени влияния объясняющих переменных на объясняемую, что обсуждалось ранее в разделе \ref{sec:20.2.3}. Также, как уже было сказано в разделе \ref{sec:20.2.4}, важно проследить, чтобы стандартные ошибки и $t$-статистики в модели регрессии Пуассона были основаны на оценках дисперсии, робастных к наличию избыточной дисперсии.

Кроме того, для оценки адекватности модели желательно проводить тесты на спецификацию. Так, тесты на избыточную дисперсию легко применить для регрессии Пуассона на пространственных данных. Для любой параметрической модели можно сопоставить действительные и предсказанные значения подсчетов, хотя в этом случае не всегда легко понять, в чем заключается недочет модели, если распределение наблюдаемых счетных данных характеризуется высокой дисперсией. Далее, можно провести формальные тесты на статистическую спецификацию и качество подгонки модели, также основанные на сравнении действительных и предсказанных значений.

Во многих практических ситуациях возникает проблема выбора модели. Для невложенных моделей на основе правдоподобия можно использовать такие критерии, как информационный критерий Акаике, построенный с помощью предсказанных значений логарифма правдоподобия, но со штрафом за большое число параметров.




\section{Библиографические заметки}\label{sec:20.9}

\noindent
\begin{itemize}
    \item[\textbf{20.2}]
Все вопросы, рассмотренные в данной главе, более подробно представлены в работе Кэмерона и Триведи (1998); там же можно найти подробный список литературы. Винкелманн (1997) представляет обзор эконометрической литературы по счетным данным. В статистической литературе анализ счетных данных представлен в контексте обобщенной линейной модели. Классической работой является работа МакКуллаха и Нелдера (1989). Эконометрическая литература уделяет относительно немного внимания обобщенной линейной модели. С точки зрения эконометрики эту модель рассматривают Фармейер и Тутц (1994). Материал из раздела \ref{sec:20.2} является базовым и применим во многих ситуациях.

    \item[\textbf{20.3}]
Деб и Триведи (2002) проводят детальный анализ данных RHIE.

    \item[\textbf{20.4}]
Кэмерон и Триведи (1986) является одной из первых работ, где предлагается обсуждение отрицательной биномиальной модели. Хаусман и др. (1984) рассматривает применение этой модели к панельным данным. Для справки по моделям конечной смеси из раздела \ref{sec:20.4.3} см. работу Деба и Триведи (1997). Модель преодоления порогов из раздела \ref{sec:20.4.5} впервые предложил Маллахай (1986), и также была рассмотрена в работах Полмейера и Улриха (1995) и Гурму и Триведи (1996).

    \item[\textbf{20.5}]
Квази-ММП, рассмотренный в разделе \ref{sec:20.5.1}, детально представлен в работах Гурьеру и др. (1984 a,b) и Кэмерона и Триведи (1986).

    \item[\textbf{20.6}]
Модели регрессии для типов данных, рассмотренных в главе \ref{sec:20.6}, еще не изучены полностью и только начинают развиваться. Исключение составляет (статическая) модель панельных счетных данных, которая достаточно хорошо представлена в классической работе Хаусмана и др. (1984). См. также работу Браннаса и Йоханссона (1996). Анализ и применение адекватных моделей регрессии для многомерных счетных данных и моделей с эндогенными регрессорами является довольно перспективной и активно развивающейся областью исследований; (Терца, 1998) и (Деб и Триведи, 2004).
\end{itemize}




\section{Упражнения}\label{sec:20.ex}

\noindent
\begin{itemize}
    \item[\textbf{20--1}]
Предположим, что $Y$ следует распределению Пуассона с математическим ожиданием $\mu$.
        \item[\textbf{(a)}]
Проверьте, что первые четыре момента распределения равны, соответственно, $\mu$, $\mu$, $\mu$ и $3\mu^2 + \mu$.
        \item[\textbf{(b)}]
Покажите, что существует линейное соотношение между $\Pr[Y = j]$ и $\Pr[Y = j - 1]$, $j = 1, 2, \ldots $.
        \item[\textbf{(c)}]
Рассмотрим оценку ММП Пуассона в регрессии с $\mu_i = \exp(\xib)$. Оценка дисперсии может быть равна $\hat{\V}[\hat{\be}] = [\sum_i \hat{\mu}_i \x_i \x'_i]^{-1}$ или $\tilde{\V}[\hat{\be}] = [\sum_i (y_i - \hat{\mu}_i)^2 \x_i \x'_i]^{-1}$.
Покажите, что эти оценки асимптотически эквивалентны (после умножения на $N$), если спецификация плотности распределения данных верна.

    \item[\textbf{20--2}]
Рассмотрим избыточную дисперсию в модели Пуассона.
        \item[\textbf{(a)}]
Пусть $Y|\mu \sim \mathcal{P}[\mu]$, где $\mu = \exp(\beta_0 + \beta_1\x)$, $\beta_0 = \ga_0 + \e$ и ошибка $\e$ является ненаблюдаемой случайной величиной с математическим ожиданием и дисперсией $\E[\e] = 0$ и $\V[\e] = \sis > 0$, соответственно. Покажите, что $\V[Y] > \E[Y]$.
        \item[\textbf{(b)}]
Рассмотрим модель NB2 с функцией дисперсии $\mu + \al \mu^2$ и функцией распределения вероятностей (\ref{eq:20.12}). Для четырех различных значений $\al \in [0, 3]$ графически опишите поведение вероятности для различных реализаций $Y$; особое внимание следует обратить на область начала координат и правый хвост распределения.
        \item[\textbf{(c)}]
Для плотности NB2 (\ref{eq:20.12}) из раздела \ref{sec:20.4.1} покажите, что при $\al \rightarrow 0$ плотность стремится к плотности распределения Пуассона [может быть непросто].

    \item[\textbf{20--3}]
Рассмотрим модель регрессии Пуассона с условным математическим ожиданием $\mu = \exp(\xb)$. Сформулируем задачу в виде нелинейного метода невзвешенных наименьших квадратов, где $y = \E[y|\x] + \e$, $\E[y|\x] = \exp(\xb)$ и $\e \sim \mathrm{iid}[0, \sis]$.
        \item[\textbf{(a)}]
Выведите уравнения (первого порядка) НМНК для $(\be, \sis)$. Сравните уравнения НМНК и ММП для $\be$ и объясните разницу между ними.
        \item[\textbf{(b)}]
Выведите уравнения \textit{взвешенным} НМНК для $\be$ и обоснуйте выбор весов. [Веса используются для учета гетероскедастичности].
        \item[\textbf{(c)}]
Сравните уравнения, полученные взвешенным НМНК и ММП, и объясните сходства, если такие присутствуют.

    \item[\textbf{20--4}]
Рассмотрим плотность конечной смеси $f(y|\ttt) = \sum^{C}_{j = 1} \pi_j f_j (y|\ttt_j)$, представляющую собой аддитивную смесь $C$ различных латентных классов, или групп, с неизвестными пропорциями $\pi_1, \ldots , \pi_C$, где $\sum^{C}_{j = 1} \pi_j = 1$, $\pi_j > 0$. Счетной переменной является $y$, и $j$-ая компонента плотности $f_j(y_i|\ttt_j)$ для $i$-го наблюдения записывается как
        $$f_j(y_j) = \frac{\Ga(y_i + \psi_{ji})}{\Ga(\psi_{ji})\Ga(y_i + 1)} \left( \frac{\psi_{ji}}{\la_{ji} + \psi_{ji}} \right)^{\psi_{ji}} \left( \frac{\la_{ji}}{\la_{ji} + \psi_{ji}} \right)^{y_{i}}$$
где $\la_{ji} = \exp(\xib_j)$, $\psi_{ji} = \la^k_{ji} / \al_j$, $\al_j > 0$ и $\bttt_j = (\be_j, \al_j)$. Параметр $k$ может принимать значения $0$ или $1$. Данная модель является отрицательной биномиальной конечной смесью с $C$ компонент, которая соответствует конечной смеси Пуассона при $\al_j = 0$.
        \item[\textbf{(a)}]
Покажите, что $\E[y_i|\x_i] = \tilde{\la}_i = \sum^{C}_{j = 1} \pi_j \la_{ji}$ и $\V(y_i|\x_i) = \sum^{C}_{j = 1} \pi_j \la^2_{ji} [1 + \al_j \la^{-k}_{ji}] + \tilde{\la}_i - \tilde{\la}_i^2$.
        \item[\textbf{(b)}]
Покажите, что любая модель смеси, основанная только на первом моменте, неидентифицируема.
        \item[\textbf{(c)}]
Покажите, что смесь Пуассона с $C$ компонентами, основанная на первых двух моментах, идентифицируема.

    \item[\textbf{20--5}]
(Бальтаджи и Ли, 1999) Простой тест на избыточную дисперсию в модели Пуассона, представленный в разделе \ref{sec:20.2.4}, проверяет нулевую гипотезу о равенстве коэффициента нулю в регрессии $[(y_i - \hat{\mu}_i)^2 - y_i] / \hat{\mu}_i$ на $\hat{\mu}_i$. Альтернативный тест, предложенный Бальтаджи и Ли (1999), проверяет ту же гипотезу в регрессии $(y_i - \hat{\mu}_i)^2$ на $\hat{\mu}_i$. Идея последнего теста аналогична тестам, основанным на регрессиях Гаусса--Ньютона (см. раздел 10.3.9). Проанализируйте различия между тестами и последствия таких различий в контексте применения второго теста.

    \item[\textbf{20--6}]
Для данного упражнения используйте 50\% выборки данных, представленных в данной главе.
        \item[\textbf{(a)}]
Оцените регрессию Пуассона и отрицательную биномиальную регрессию, взяв в качестве зависимой переменной MDU, а в качестве объясняющих факторов следующий набор переменных: LC, IDP, LINC, FEMALE, EDUDEC, XAGE, BLACK, HLTHG, HLTHF и HLTHP. Проведите тест отношения правдоподобия, чтобы проверить гипотезу о том, что переменные LC и IDP не оказывают влияния на MDU.
        \item[\textbf{(b)}]
Проведите тест на избыточную дисперсию в регрессии Пуассона, используя формулы дисперсии (\ref{eq:20.9}) с $g(\mu) = \mu$ и (\ref{eq:20.10}) с $g(\mu) = \mu^2$. % что?
Какой вариант формулы лучше соответствует данным? Какой вывод можно сделать на основе данного упражнения?
        \item[\textbf{(c)}]
Оцените отрицательную биномиальную модель (NB2). Сравните оценки параметра избыточной дисперсии в пункте (b). Объясните сходства и различия.
        \item[\textbf{(d)}]
Используя результаты по оцениванию отрицательной биномиальной модели, сравните оценку предельного эффекта от изменения LC для среднего индивида с отличным состоянием здоровья и среднего индивида с плохим состоянием здоровья (HLTHP = 1).
        \item[\textbf{(e)}]
Для спецификации Пуассона оцените модель ``преодоления порогов'', состоящую из части с нулевыми значениями (логит или пробит) и части с положительными (распределение Пуассона с урезанными нулями). Сравните эти результаты с обычной моделью Пуассона. Проанализируйте сходства и различия между выводами, следующими из обеих моделей. Какая модель лучше объясняет данные?
\end{itemize}


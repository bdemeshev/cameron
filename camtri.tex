\documentclass[pdftex,12pt,a4paper]{book}

\usepackage[X2,T2A]{fontenc}
\usepackage[utf8]{inputenc}
\usepackage[russian]{babel}

\usepackage[paper=a4paper,top=13.5mm, bottom=13.5mm,left=16.5mm,right=13.5mm,includefoot]{geometry}
\usepackage[pdftex,unicode,colorlinks=true,urlcolor=blue,hyperindex,breaklinks]{hyperref} 


\usepackage{amsmath}
\usepackage{amsfonts}
\usepackage{amssymb}
\usepackage{amsthm}
\usepackage{booktabs}
\usepackage{bm}
\usepackage{color}
\usepackage{dcolumn}
\usepackage{enumerate}		% small roman enumeration

\usepackage{floatflt}
\usepackage{fancyhdr}
\usepackage{graphicx}

\usepackage{indentfirst} 

\usepackage{longtable}

\usepackage{rotating}
\usepackage{setspace}
\usepackage{soulutf8}


\usepackage{tabularx}
\usepackage{tablefootnote}	% всё равно не работают. разобраться.
\usepackage{threeparttable}

\usepackage{xfrac}


\newcommand{\argmax}{\arg\max}
\newcommand{\AR}{\mathrm{AR}}
\newcommand{\cN}{\mathcal{N}}
\newcommand{\be}{\bm{\beta}}
\newcommand{\Cov}{\mathrm{Cov}}
\newcommand{\Cor}{\mathrm{Cor}}
\newcommand{\Corr}{\mathrm{Cor}}
\newcommand{\Diag}{\mathrm{Diag}}
\newcommand{\E}{\mathbb{E}}
\newcommand{\Expect}{\mathbb{E}}
\newcommand{\e}{\varepsilon}

\newcommand{\Loss}{\mathrm{L}}
\newcommand{\plim}{\mathrm{plim }}
\renewcommand{\Pr}{\mathrm{Pr}}
\renewcommand{\P}{\mathrm{Pr}}
\newcommand{\rank}{\mathrm{rank}}
\newcommand{\sign}{\mathrm{sign}}

\newcommand{\tr}{\mathrm{tr}}
\newcommand{\Vech}{\mathrm{Vech}}

\newcommand{\V}{\mathrm{V}}
\newcommand{\Var}{\mathrm{V}}


\newcommand{\ali}{\alpha_i}
\newcommand{\x}{\mathbf{x}}
\newcommand{\al}{\alpha}
\newcommand{\de}{\delta}
\newcommand{\eps}{\epsilon}
\newcommand{\ga}{\gamma}
\newcommand{\Ga}{\Gamma}
\newcommand{\ttt}{\theta}
\newcommand{\pa}{\partial}
\newcommand{\si}{\sigma}
\newcommand{\sis}{\sigma^2}
\newcommand{\bmu}{\bm{\mu}}
\newcommand{\bnu}{\bm{\nu}}
\newcommand{\bpi}{\bm{\pi}}
\newcommand{\bttt}{\bm{\theta}}
\newcommand{\bz}{\mathbf{z}}
\newcommand{\cI}{\mathcal{I}}
\newcommand{\cL}{\mathcal{L}}
\newcommand{\bs}{\mathbf{s}}
\newcommand{\bt}{\mathbf{t}}
\newcommand{\bd}{\mathbf{d}}
\newcommand{\0}{\mathbf{0}}
\newcommand{\xb}{\mathbf{x}'\bm{\beta}}
\newcommand{\xib}{\mathbf{x}_i'\bm{\beta}}
\newcommand{\om}{\omega}
\newcommand{\Om}{\Omega}
\newcommand{\mP}{\mathrm{P}}
\newcommand{\mL}{\mathrm{L}}
\newcommand{\la}{\lambda}
\newcommand{\La}{\Lambda}






\newcommand{\btheta}{\mathbf{\theta}}
\newcommand{\bbeta}{\mathbf{\beta}}
\newcommand{\hbtheta}{\hat{\btheta}}
\newcommand{\bx}{\mathbf{x}}
\newcommand{\mxx}{\mathbf{M}_{\mathbf{XX}}}
\newcommand{\bV}{\mathbf{\V}}
\newcommand{\bb}{\mathbf{b}}
%\newcommand{\bmu}{\mathbf{\mu}}
\newcommand{\bS}{\mathbf{S}}
\newcommand{\bB}{\mathbf{B}}
\newcommand{\bA}{\mathbf{A}}

\newcommand{\dto}{\overset{d}{\to}}
\newcommand{\pto}{\overset{p}{\to}}
\newcommand{\asto}{\overset{as}{\to}}
\newcommand{\sumton}{\sum_{i=1}^{N}}
\renewcommand{\b}{b} % иначе что-то в apx не варит




\includeonly{chapters/ch15}




\theoremstyle{definition}
\newtheorem{definition}{Определение}[chapter] % нумерация сбрасывается в новой chapter
\newtheorem{theorem}[definition]{Теорема} % сквозная нумерация с definition
\newtheorem{proposition}[definition]{Утверждение} % сквозная нумерация с definition




\numberwithin{equation}{chapter}


\begin{document}




\part{Предварительные сведения}

В части 1 обсуждаются основные компоненты микроэконометрического анализа --- экономическая спецификация, статистическая модель и набор данных. 

В главе 1 рассматриваются отличительные особенности микроэконометрики и описывается структура книги.  В главе подчеркивается, что дискретность данных, нелинейность и неоднородность поведенческих отношений являются ключевыми аспектами микроэконометрических моделей на индивидуальном уровне. В конце главы обсуждаются обозначения и соглашения, используемые далее в книге.
%%%
Главы 2 и 3 нацелены на то, чтобы познакомить читателя с ключевыми моделями и типами данными, которые анализируются в последующих главах.


Ключевое различие в эконометрике находится между  описательными по существу статистическими моделями или обобщением данных разных уровней сложности и моделями выходящими за рамки описания взаимосвязей и пытающимися оценить параметры причинно-следственных связей. 
Классическое определение причинности в эконометрике берет свое начало из систем одновременных уравнений Комиссии Коулса, которые проводят четкое различие между экзогенными и эндогенными переменными, а также между параметрами структурной и приведенной форм модели. 
Хотя приведенные формы моделей являются очень полезными для некоторых целей, знание структурных параметров имеет важное значение для анализа последствий проводимой политики. 
Идентификация структурных параметров в рамках одновременных уравнений связана с множеством теоретических и практических сложностей. Широко используемый альтернативный подход на основе модели потенциального результата также пытается выявить параметры причинно-следственных связей, но он делает это ставя более узкие вопросы в более удобных рамках. В главе 2 приведен обзор фундаментальных вопросов, которые возникают в рамках этих и других альтернативных концепций. Читатели, у которых сейчас сложилось впечатление, что материал данной главы сложный, могут вернуться к ней после знакомства с конкретными моделями рассмотренными далее в этой книге.
	
	
Способность исследователей правильно выявить причинно-следственную связь зависит не только от статистических инструментов и моделей, но и от типа доступных данных. Экспериментальные данные являются стандартом для установления причинно-следственных связей. Тем не менее, не экспериментальные, а описательные данные лежат в основе большинства эконометрических исследований. В Главе 3 рассмотрены  плюсы и минусы трёх основных типов данных: описательных данных, данных из социальных экспериментов и данных из естественных экспериментов; рассматриваются сильные и слабые стороны проведения причинного-следственного вывода в каждом случае.

\chapter{Обзор}

\section{Введение}

Эта книга содержит подробное рассмотрение микроэконометрического анализа, т.е. анализа  данных на индивидуальном уровне, например, экономического поведения отдельных лиц или фирм. Более широкое определение будет также включать сгруппированные данные. Обычно регрессионный метод применяются пространственным (cross-section) или панельным данным.
	
	
Анализ индивидуальных данных имеет долгую историю. Эрнст Энгель (1857) был одним из первых исследователей  бюджетов домашних хозяйств. Аллен и Боули (1935), Хаутаккер (1957), и Прайс и Хаутаккер (1955) внесли важный вклад  в том же направлении, используя те же подходы. Другие исследования также оказали значительное влияние на развитие микроэконометрики, однако они не всегда использовали информацию на индивидуальном уровне, например, Маршак и Эндрюс (1944) занимались  теорией производства и Вольд и Джюрин  (1953), Стоун (1953 ) и Тобин (1958) --- потребительским спросом.
	
	
Материал описанный в этой книге также связан с моделями дискретного выбора и моделями цензурированных и усеченных переменных. Серьёзное эконометрическое исследование этой темы было начато  в работе МакФаддена (1973 , 1984) и Хекмана (1974, 1979). Эти работы отходят от использования линейных моделей, которые использовались в более ранних работах . Впоследствии эти работы привели к значительным методологическим инновациям в эконометрике. Среди ранних учебников, в которых освещён этот материал (и более) являются работы Маддалы (1983) и Амэмии (1985). Как отмечали Хекман (2001), МакФадден (2001 ) и другие, многие из фундаментальных вопросов, которые фигурировали в ранних работах по рыночным данным, остаются важными и в настоящее время. Особенно это справедливо в отношении условий, необходимых для идентифицируемости причинно-следственных экономических отношений. Тем не менее, микроэконометрика имеет свой особый стиль, что оправдывает написание специального учебника.
	
%%%
Современная  микроэконометрика основанна на данных об индивидах, домохозяйствах и фирмах, и во многом  обязана широкой доступности различных данных от пространственных и панельных выборок до данных переписей. В последние два десятилетия с использованием электронного способа хранения данных  и сбора данных на индивидуальном уровне объем доступных данных возрос скачкообразно. 
Так же существенно возросли вычислительные мощности для анализа больших и сложных наборов данных. 
Во многих случаях доступны данные событийного уровня, например, маркетинговые исследования часто имеет дело с данными покупок, собранных при помощи электронных сканеров в супермаркетах, а в литературе по теории отраслевых рынков встречаются эконометрические исследования, основанные на данных авиаперевозок, собранных системами онлайн-бронирования билетов. В настоящее время появляются новые отрасли экономики, такие как социальное экспериментирование и  экспериментальная экономика, которые создают <<экспериментальные>> данные. Появляются новые возможности моделирования, которое отсутствует, в случае доступности только агрегированных рыночных данных. Между тем стремительный рост объемов и типов данных также породил множество методологических вопросов. Обработка и эконометрический анализ таких  микроданных большого объёма, с целью выявления моделей экономического поведения, составляет ядро микроэконометрики. Эконометрический анализ таких данных является главным предметом этой книги.
	
	
Основными предшественниками этой книги являются книги Маддалы (1983) и Амэмии (1985). Подобно им мы охватываем темы, которые представлены лишь кратко, или вообще не представлены, во вводных курсах или курсах первого года обучения. Особенно по сравнению с книгой Амэмии (1985) наша книга больше ориентирована на практическое применение. Уровень изложения тем не менее достаточно продвинутый, особенно для исследователей практиков из областей менее математизированных, чем экономика.
	
	
Относительно продвинутое математическое изложение  необходимо по нескольким причинам. Во-первых, данные часто являются дискретными или цензурированными, и в этом случае используются нелинейные методы, такие как логит, пробит, и тобит модели. Это приводит к статистическому анализу, основанному на более сложной асимптотической теории.
	
	
Во-вторых, предположения о распределении для анализа таких данных становятся крайне важными. Возможным решением является разработка сильно параметризованных моделей, которые являются достаточно гибкими, чтобы охватить всю сложность данных, но эти модели сложно оценить. Более распространенным решением является минимизация параметрических предположениях и предоставление статистических выводов на основе стандартных ошибок, которые являются робастными к таким проблемам, как гетероскедастичность и кластеризации. В таких случаях значительные знания могут быть необходимых для получения достоверных статистических выводов, даже если используется стандартный пакет регрессии.
	
	
В-третьих, экономические исследования часто ставят целью определение причинности, а не просто измерение корреляции, несмотря на доступ к данным наблюдений, а не экспериментальным данным. Это приводит к методам, которые способны выявить причинность, например, таким как инструментальные переменные, системы одновременных уравнений, коррекция ошибки измерения, коррекция смещения отбора, панельные данные с фиксированными эффектами, и разность разностей.
	
	
В-четвертых, микроэкономические данные, как правило, собираются с использованием пространственных или  панельных обследований, переписей или социальных экспериментов. Данные обследований, собранные с помощью этих методов, могут приводить к проблемам, связанным со сложной методологии обследования, отклонением от предположения случайности выборки, и проблеме самоотбора, ошибкам измерения, и неполным, и / или пропущенным данным. Для корректного оценивания таких эконометрических моделей необходимо использование продвинутых методов.
	
	
Наконец, довольно часто два или более осложнения происходят одновременно, например, эндогенность в логит модели с панельными данными. В этом случае очень трудно реализовать стандартные подходы. Вместо этого, необходимо значительное понимание теории, лежащей в основе методов, так как исследователю может потребоваться  чтение современных статей и адаптация под свои нужды существующего программного обеспечения.  


\section{Отличительные аспекты микроэконометрики}

Сейчас мы рассмотрим несколько преимуществ микроэконометрики, которые вытекают из её отличительных особенностей.

%%% --->
\subsection{Дискретность и нелинейность}

Первым и наиболее очевидным достоинством является то, что микроэконометрические данные, как правило, слабо агрегированы. Этот факт влияет на выбор функциональных форм, которые используются для анализа переменных, представляющих интерес. Во многих, если не в большинстве случаев, линейные функциональные формы не подходят. Более существенно то, что дезагрегирование выдвигает на первый план неоднородность индивидов, фирм и организаций, которая должны надлежащим образом учитываться, если необходимо сделать верные выводы о лежащих в основе данных отношениях. Мы обсудим эти вопросы более подробно в следующих разделах.
	
	
Конечно агрегирование не полностью отсутствует в микроданных, например, в данных по домохозяйствам или учреждениям, однако уровень агрегирования, как правило, на порядок ниже, чем это принято в макро анализе. 
В последнем случае процесс агрегирования приводит к сглаживанию, разнонаправленные изменения взаимно уничтожаются. 
Агрегированные переменные часто показывают более гладкое поведение, чем их компоненты, более того отношения между агрегированными переменными также часто показывают большую гладкость, чем отношение между компоненты. Например, связь между двумя переменными на микроуровне может быть кусочно-линейной со многими узлами. После агрегирования, связь будет хорошо аппроксимироваться гладкой функцией. Поэтому, следствием дезагрегирования данных является отсутствие непрерывности и гладкости как самих переменных, так и взаимосвязей между ними.
	
	
Обычно индивидуальные данные и данные фирм сильно изменчивы, как в одномоментной выборке так и во временных рядах. Например, среднее еженедельное потребление, например,  говядины, очень вероятно будет положительным и плавно меняющимся, тогда как потребление отдельного домохозяйства за конкретную неделю часто равно нулю и может становиться положительным значениям время от времени. 
Среднее число часов, отработанных работником женского пола вряд ли будет нулевым, но если рассматривать женщин по отдельности то многие из них имеют ноль часов работы (угловое решение), переходя на положительные значения в другое время в ходе своего трудового стажа. Средние расходы на отпуск, как правило, положительные, но многие домохозяйства могут иметь нулевое значение расходов на отпуск за конкретный год. Среднедушевое потребление табачных изделий, как правило, положительное, но многие индивиды из генеральной совокупности никогда не потребляли сигареты и никогда не будут, независимо от цены и дохода. Как отмечает Падни (1989), микро-данные чаще содержат  <<дыры, изломы и углы>>. Дыры соответствуют неучастию в исследуемой деятельности, изломы соответствуют смене поведения, а углы соответствуют непотреблению или неучастию в определенный момент времен. То есть, дискретность и нелинейность значений --- отличительная черта микроэконометрики.
	
	
Важный класс нелинейных моделей в микроэконометрике имеет дело с ограниченными зависимыми переменными (Маддала, 1983). Этот класс включает в себя множество моделей, которые обеспечивают подходящие рамки для анализа дискретных значений и значений с ограниченным диапазоном изменения. Такие инструменты анализа, конечно, также доступны для анализа макро данных, если это необходимо. Дело в том, что они незаменимы в микроэконометрике и являются её отличительной особенностью.

\subsection{Более высокая реалистичность}


Макроэкономика основана на сильных предположениях; таких, например, как предположение о репрезентативном агенте. Часто с помощью микроэкономических рассуждений обосновывают определенные спецификации моделей   или интерпретируют эмпирические результаты. 
Однако, чаще всего нельзя сказать, как влияет агрегирование по времени и по индивидам. Иногда при агрегировании делаются очень сильные предположения. Например, считается, что агрегирование отражают поведение гипотетического репрезентативного агента. Это предположение  однозначно не вызывают доверия.
	
%%% --->
С точки зрения микроэкономической теории, количественный анализ основанный на микроданных можно рассматривать как более реалистичный, чем  основанный на агрегированных данных. 
Есть три обоснования этого утверждения. Во-первых, измерение переменных, используемых в таких гипотезах, часто более просто (хотя и не обязательно свободно от ошибки измерения) и имеет большее соответствие с теорией, которая тестируется. 
Во-вторых, гипотезы об экономическом поведении, как правило, вытекают из теорий индивидуального поведения. Если эти гипотезы проверяются с использованием агрегированных данных, то необходимо вводить большое количество предположений и упрощений. Упрощающее предположение о репрезентативном агенте вызывает существенную потерю информации и серьезно ограничивает сферу эмпирического исследования. К счастью, таких предположений можно избежать в микроэконометрике, и, как правило,   микроданные обеспечивают более удобную основу для тестирования микроэкономических гипотез. Впрочем, это не означает, что потенциальная выгода от использования микроданных достигается на практике. 
Наконец, реалистичное изображение экономической деятельности должно обеспечивать широкий диапазон результатов, которые являются следствием индивидуальной неоднородности и, которые предсказываются соответствующей теорией. В этом смысле микроэкономические данные могут приводить к более реалистичным моделям.
	
	
Источником микроэконометрических данных часто являются опросы фирм или домохозяйств. В данных отражаются разнообразные формы поведение, результаты многих решений являются дискретными или качественными.  Такие наборы данных имеют много необычных  особенностей, которые требуют использования специальных инструментов при моделировании и анализе. Подобные особенности могут иметь место и в макроэконометрике, но там они встречаются гораздо реже.

%%% --->
\subsection{Более высокая насыщенность информацией}

Потенциальные преимущества микроданных могут быть реализованы, если эти данные являются информативными. Выборочные обследования часто содержат независимые наблюдения по тысячам единиц пространственной выборки, такие данные считаются  более информативными, чем типичные макроэкономические. Макроэкономические временные ряды часто сильно коррелированы и  обычно состоят всего из нескольких сотен наблюдений.
	
	
Как будет отмечено в следующей главе, на практике ситуация не такая идеальная потому, что микроданные могут быть сильно зашумлены. 
На индивидуальном уровне разнообразные уникальные факторы могут играть большую роль в определении поведения. Часто эти факторы ненаблюдаемы, поэтому исследователи трактуют их как случайные компоненты, и эти факторы могут объяснять существенную часть наблюдаемых изменений зависимой переменной. 
В этом смысле случайность играет большую роль в микроданных, чем в макроданных. Конечно же, эта случайность влияет на  показатели адекватности регрессии. Студенты, начинающие изучать эконометрику с анализа агрегированных временных рядов часто хотят увидеть большое значения $R^{2}$. Встретившись с пространственной выборкой в первый раз, они разочаровываются низкой объясняющей силой  регрессии. Тем не менее, есть серьезные основания полагать, что по крайней мере в определенном смысле, большие наборы микроданных несут много информации.
	
%%% --->	
Другая особенность состоит в том, что, когда мы имеем дело с пространственными данными (одномоментная выборка), очень мало  можно сказать о временных свойствах изучаемой зависимости. Временной  аспект поведения можно исследовать с помощью панельных и переходных данных.
	
	
Во многих случаях исследователь заинтересован в определении поведенческой реакции некоторой группы экономических агентов при некоторых заданных экономических условиях. Например, можно исследовать влияние страхования от безработицы на поведение молодых безработных при поиске работы. Другим примером является изменение предложения труда у лиц с низким доходом, которые получают материальную поддержку. Без использования микроданных ответить на эти вопросы напрямую в эмпирическом исследовании невозможно.
	
\subsection{Микроэкономические основания}

	Эконометрические модели различаются по роли, которую они отводят для экономической теории. С одной стороны, есть модели, в которых теоретические основания играют  главную роль в спецификации модели и в выборе процедуры оценивания. С другой стороны, есть эмпирические исследования, которые используют гораздо меньше экономической теории.
	
	%%% -->
	Цель анализа в первом случае заключается в выявлении и оценки фундаментальных параметров модели, которые также называют глубокими параметрами. Эти параметры  характеризуют индивидуальные вкусы и предпочтения и/или технологические отношения. Для краткости обозначения, мы называем это \textbf{структурным подходом}. Его отличительной чертой является сильная зависимость от экономической теории и упор на причинно-следственные выводы. 
		%%% -->
	Такие модели могут требовать большого количества предположений, например, точной спецификации функции затрат, производственной функции или случайной составляющей. Эмпирические выводы в таком случае могут быть неробастными в случае отклонения от сделанных предположений. В разделе 2.4.4 мы будем говорить подробнее об этом подходе. На данном этапе мы просто подчеркнем, что, если структурный подход реализуется по агрегированным данным, то оценки фундаментальных параметров можно получить  только при очень жестких (и вероятно нереалистичных) предпосылках. Микроданные более уместны для структурного подхода, поскольку они обеспечивают большую гибкость при построении модели.
	
		%%% -->
	
	Цель анализа во втором случае состоит в моделировании отношения между интересующей исследователя зависимой переменной и объясняющими переменными, которые рассматриваются как экзогенные. 
	Более формальные определения эндогенности и экзогенности приведены в главе 2. Для краткости, мы называем этот подход подходом приведенной формы (reduced form approach). 
	Важно отметить, что такой подход не всегда учитывает все причинно-следственные взаимозависимости. Регрессионная модель, в которой акцент делается на прогнозировании $y$ при заданных регрессорах $x$, а не на причинной интерпретации параметров регрессии, часто называется регрессией в приведенной форме. 
	Как объясняется в главе 2, параметры приведенной формы часто являются функциями структурных параметров, поэтому они не могут интерпретироваться без некоторой информации о структурных параметрах.



\subsection{Дезагрегирование и неоднородность}

Говорят, что многие проблемы в макроэконометрике возникают из-за автокорреляции макроэкономических временных рядов, а в микроэконометрике проблемы берут своё начало из-за гетероскедастичности на индивидуальном уровне. 
Хотя это утверждение в значительной степени относится к существенной части работ по микроэконометрике, оно нуждается в уточнении.
В ряде  микроэконометрических моделей моделирование динамической зависимости может быть важным вопросом.
	
	
Преимущества дезагрегирования, которые были выделены ранее в этом разделе, имеют свою издержки:  когда данные становятся более детализированными важность учёта неоднородности между индивидами увеличивается. Неоднородность, а точнее ненаблюдаемая неоднородность, играет очень важную роль в микроэконометрики. Очевидно, что многие переменные, которые отражают неоднородность, такие как пол, раса, образование и социально-демографические факторы, являются непосредственно наблюдаемыми и как следствие, могут быть учтены. Напротив, различия в индивидуальной мотивации, способностях, интеллекте и т.д. либо не наблюдаются, либо, в лучшем случае,  наблюдаются частично.
	
	
Самый простой выход из ситуации --- это игнорирование неоднородности, то есть включение её в случайную ошибку регрессии. В конце концов именно так поступают с кучей мелких ненаблюдаемых факторов. Этот шаг, конечно, увеличивает необъясняемую часть вариации зависимой переменной. Более важно, что игнорирование системных индивидуальных различий приводит к \textbf{смешиванию} (confoundign) с другими факторами, которые также являются источниками системных индивидуальных различий. 

Смешивание факторов происходит тогда, когда вклад различных регрессоры  не может быть статистически разделен. Предположим, например, что регрессор $x_{1}$ (образование), может быть источником изменения переменной $y$ (дохода), а другая переменная --- $x_{2}$ (способности), которая является еще одним источником изменения дохода, не включена в модель. Тогда та часть полной изменения зависимой переменной, которая приходится на второй регрессор, может быть неправильно соотнесена с первой переменной. Интуитивно, их относительный вклад в изменение дохода смешан. Основной причиной смешивания регрессоров является ошибочное невключение регрессоров в модель и включения других переменных, которые являются прокси-переменными для пропущенных регрессоров.
	

%%%	<----
Рассмотрим случай, в котором дамми-переменная участия в программе $D$ (0/1) входит в линейную регрессию вместе с вектором регрессоров $x$.
	
\begin{equation}
y=X'\beta+D\alpha+u
\end{equation}


где $u$ --- ошибки модели. Термин <<воздействие>> используется в биологической и экспериментальной литературе для обозначения режима назначенного определенной группе участников эксперимента. 
В эконометрике это обычно относится к участию в какой-либо деятельности, которая может повлиять на исследуемую переменную. 
Участие в этой деятельности может быть назначено  участникам случайным образом или может быть самостоятельно выбрано участником. 
Таким образом, хотя  люди сами выбирают сколько лет им обучаться в школе, считается, что продолжительность обучения можно рассматривать как <<воздействие>>. 
Предположим, что участие в программе является дискретным, тогда коэффициент $\alpha$ измеряет среднее воздействие участия в программе $(D = 1)$ при прочих равных. 
Если не учитывать ненаблюдаемую неоднородность, то интерпретация результатов может быть неоднозначной. 
Если коэффициент $\alpha$ значим, то возникает следующий вопрос: является ли $\alpha$ значимо отличным от нуля, потому что $D$ коррелирует с некоторой ненаблюдаемой переменной, которая влияет на $y$ или потому, что существует причинно-следственная связь между $D$ и $y$? 
Например, если модель рассматривает в качестве воздействия университетское образование, и в регрессоры не включаются способности индивида, причинно-следственная интерпретация становится сомнительной. Так как проблема ненаблюдаемой неоднородности является важной, большое внимание следует уделять тому, как её учитывать. 


В некоторых случаях, когда имеются динамические зависимости, тип доступных данных накладывает  ограничения на выбор способа учета неоднородности. 
Рассмотрим пример с двумя домашними хозяйствами, одинаковыми во всех отношениях за исключением того, что одно из них более склонно потреблять товар A. 
Можно учесть это различие, включив в  индивидуальные функции полезности параметр неоднородности, который отражает различия в предпочтениях. 
Предположим теперь, что существует теория потребительского поведения, которая утверждает, что потребители становятся зависимыми от товара А, в том смысле, что чем больше они потребляют его в первом периоде, тем выше вероятность того, что они будут потреблять его больше в будущем. Эта теория дает другое объяснение индивидуальных различий в потреблении товара А. Учитывая неоднородность предпочтений мы можем проверить, какая причина, неоднородность предпочтений или зависимость от потребления, вызывает разную структуру потребления. 
Этот тип проблемы возникает всякий раз, когда некоторые динамические особенности модели порождают постоянство  наблюдаемых результатов. Несколько примеров такого рода проблем разобраны в различных главах этой книги.
	
	
Различные подходы для моделирования неоднородности существуют в микроэконометрике. Краткое упоминание о некоторых из них следует далее, а подробное изложение отложено до следующих глав.
	
	
Крайним решением будет игнорировать все ненаблюдаемые индивидуальные различия. Если ненаблюдаемая неоднородность не коррелирует с наблюдаемой неоднородностью, и если зависимая переменная не имеет межвременной зависимости, то упомянутые проблемы не возникнут. Конечно, эти  предположения являются очень сильными, однако даже с их учетом  не все эконометрические трудности исчезают.
	
	
Один из способов учета неоднородности индивидов --- трактовать её как наличие фиксированных эффектов в модели, и оценить её с помощью коэффициентов при  индивидуальных дамми-переменных.
Например, в панельных данных, у каждого индивида может быть своя дамми-переменная. Это приводит к резкому увеличению числа параметров, т.к. при добавлении нового индивида добавляется новый параметр. Следовательно, данный подход не будет работать для пространственных данных. 
Бывает доступно несколько наблюдений для отдельного индивида, чаще всего в виде панельных данных, где для каждого из $N$ индивидов имеется временной ряд длины $T$.
Эти повторяющиеся наблюдения позволяют либо оценить, либо устранить фиксированный эффект, например, взятием первых разностей в случае, если модель является линейной с аддитивными  фиксированными эффектами. 
Если, как это часто бывает, модель является нелинейной, то фиксированный эффект обычно неаддитивный, и поэтому должны быть рассмотрены другие подходы.
	
	
Второй подход заключается в использовании моделей со случайными эффектами. Есть несколько способов специфицировать модель со случайными эффектами. Одна из популярных формулировок предполагает, что один или несколько параметров регрессии, чаще всего свободный член, изменяется случайным образом по индивидам. 
В другой формулировке ошибки регрессии состоят из нескольких компонент, включая случайную компоненту для каждого индивида.  Далее модель со случайными эффектами используется для оценивания  параметров распределения случайной составляющей. В некоторых случаях, таких как анализ спроса, случайная компонента может быть интерпретирована как случайное изменение предпочтений. Модели со случайными эффектами можно оценивать, используя как пространственные, так и панельные данные.
	

\subsection{Динамика}

Очень распространенным предположением в анализе пространственных данных является отсутствие межвременной зависимости, то есть, отсутствие динамики. Таким образом, неявно предполагается, что наблюдения соответствуют стохастическому равновесию. А отклонения от равновесия представлены  независимыми случайными ошибками. Даже в микроэконометрике для некоторых данных такое предположение может быть слишком сильным. Например, оно не согласуется с наличием коррелированной ненаблюдаемой неоднородности. Зависимость от лагированной зависимой переменной также нарушает это предположение.

Вышеизложенное иллюстрирует некоторые из потенциальных ограничений анализа пространственной выборки с одним наблюдением для каждого индивида. Некоторые ограничения могут быть преодолены, если имеются повторные пространственные выборки. Тем не менее, если есть динамические зависимости, наиболее простым подходом может оказаться использование панельных данных.

\section{Структура книги}
 
Книга разделена на шесть частей. В части 1 представлены вопросы, связанные с микроэконометрическим моделированием. В частях 2 и 3 мы познакомимся с общей теорией оценивания и статистических выводов для нелинейных регрессионных моделей. Части 4 и 5 посвящены основным моделям, используемым в прикладной микроэконометрике для анализа пространственных и панельных данных. Часть 6 охватывает более широкие темы, которые опираются на материал, представленный в предыдущих главах.

Структура книги представлена в таблице 1.1. Далее в этом разделе будет сказано про каждую часть более подробно.
	
\subsection{Часть 1. Предварительные сведения}

В главах 2 и 3 рассмотрены  особенности микроэконометрического подхода к моделированию и микроэкономические структуры данных в широких рамках  статистического регрессионного анализа. Многие из вопросов, поднятых в этих главах, обсуждаются на протяжении всей книги, по мере того, как развивается необходимый инструментарий.

\begin{table}[h]
\begin{center}
\caption{\label{tab:bookstructure}Структура книги}
\begin{tabular}{p{8cm}p{1cm}p{6cm}}
\hline
\hline
Часть и Глава  & Требуемые главы & Пример \\
\hline
\textbf{1. Предварительные сведения} & & \\
1. Обзор & - & \\
2. Причинно-следственные и статистические модели & - & Системы одновременных уравнений \\
3. Структуры микроэкономических данных & - & Данные наблюдений \\
\textbf{Часть 2. Основные методы} & & \\
4. Линейные модели & - & Метод наименьших квадратов \\
5. Оценивание с помощью метода максимального правдоподобия и
нелинейнего метода наименьших квадратов & - & М-оценки или экстремальные оценки \\
6. Обобщённый метод моментов и системы уравнений & 5 & Инструментальные переменные \\
7. Проверка гипотез & 5 & Тест Вальда, множителей Лагранжа и отношения правдоподобия \\
8. Тесты на спецификацию и выбор моделей & 5,7 & Тест на условный момент \\
9. Полупараметрические методы & - & Ядерная регрессия \\
10. Методы численной оптимизации & 5 & Итерационная процедура Ньютона-Рафсона \\
\textbf{Часть 3. Методы симуляционного моделирования} & & \\
11. Бутстрэп методы & 7 & Метод $t$-перцентилей \\
12. Методы симуляционного моделирования & 5 & Симуляционный метод максимального правдоподобия \\
13. Байесовские методы & - & Метод Монте-Карло по схеме марковской цепи \\
\textbf{Часть 4. Модели пространственных данных} & & \\
14. Модели бинарного выбора & 5 & Логит и пробит для $y=(0,1)$ \\
15. Мультиномиальные модели & 5, 14 & Мультиномиальный логит для $y=(1,2,\ldots, m)$ \\
16. Тобит-модели и модели выбора & 5, 14 & Тобит для $y=\max(y^*,0)$ \\
17. Транзитные данные: Анализ выживаемости & 5 & Модель Кокса пропорциональных рисков \\
18. Модели смеси и ненаблюдаемая гетерогенность & 5, 17 & Ненаблюдаемая гетерогенность \\
19. Модели множественных рисков & 5, 17 & Множественные риски \\
20. Модели счетных данных & 5 & Пуассоновская модель для $y=0, 1, 2, \ldots$ \\
\textbf{Часть 5. Модели анализа панельных данных} & & \\
21. Линейные модели панельных данных: основы & - & Фиксированные и случайные эффекты \\
22. Линейные модели анализа панельных данных: дополнения & 6, 21 & Динамические и эндогенные регрессоры \\
23. Нелинейные модели панельных данных & 5, 6, 21, 22 & Панельные логит, пробит и пуассоновская модель \\
\textbf{Часть 6. Дальнейшие темы} & & \\
24. Стратифицированные и кластеризованные выборки & 5 & Данные $(y_{ij},x_{ij})$ коррелированные по $j$ \\
25. Оценка эффектов воздействия & 5, 21 & Регрессор $d=1$ при участии \\
26. Модели ошибок измерения & 5 & Логит-модель с ошибками измерения\\
27. Пропущенные данные и восстановление данных & 5 & Регрессия с пропущенными наблюдениями 
\end{tabular}
\end{center}
\end{table}
В столбце <<Требуемые главы>> приведены главы необходимые помимо изложения метода наименьших квадратов и взвешенного метода наименьших квадратов в главе 4. 


\subsection{Часть 2. Основные методы}

Главы 4---10 подробно описывают общие методы, используемые при классическом оценивании моделей и статистических выводах. Результаты, приведенные в главе 5, в частности, широко используются на протяжении всей книги.
	
	
В Главе 4 изложены некоторые результаты для линейной регрессионной модели. Внимание  акцентируется  на тех вопросах и методах, которые наиболее актуальны для остальной части книги. Анализ относительно прост, поскольку существуют явные выражения для оценок линейных моделей, например, для метода наименьших квадратов.

В главах 5 и 6 обсуждаются способы оценивания, которые могут быть применены к нелинейным моделям, для которых обычно не существует явного решения. Асимптотическая теория используется для получения распределения оценок, с акцентом на получение робастных стандартных  ошибок коэффициентов, которые опираются на относительно слабые предположения о распределении ошибок. 
Общее изложение методов оценивания, в частности нелинейного метода наименьших квадратов и метода максимального правдоподобия, представлено в главе 5. Более сложные обобщенный метод моментов и метод инструментальных переменных приведены в главе 6.
	
	
Глава 7 описывает классические способы проверки гипотез, когда оценки являются нелинейными или, когда тестируемая гипотеза нелинейна по параметрам. Тесты на спецификацию  в дополнение к проверке гипотез обсуждаются в главе 8.
	
	
В главе 9 представлены полупараметрические методы оценки, такие как ядерная регрессия. Важным примером  является гибкое моделирование условного математического ожидания. В примере с патентами непараметрическая регрессионная модель задана в виде  $\E[y|x] = g(x)$, где функция $g(\cdot)$ не специфицирована и должна быть оценена. Требуется оценить бесконечномерный объект $g(\cdot)$, что приводит к использованию нестандартной асимптотической теории. При наличии дополнительных регрессоров необходимы  дополнительные предположения о структуре модели, такие  методы называются полупараметрическими.
	
	
В Главе 10 изложены способы вычисления оценок параметров, когда оценки определяются неявно, как правило, как решения некоторых условий первого порядка.

\subsection{Часть 3. Методы симуляционного моделирования}
	
В Главах 11---13 изложены методы оценивания и построения статистических выводов, использующие симуляционный подход. Эти методы обычно требуют большого количества вычислений и в настоящее время меньше используются, чем методы, представленные в части 2.
	
	
В главе 11 представлены бутстрэп методы статистического вывода. Бутстрэп позволяет получить  эмпирическое распределение оценки путем генерирования  выборки значений  с помощью симуляций. Например, могут использоваться новые выборки с повторениями из исходной выборки. Бутстрэп может обеспечить простой способ вычисления стандартных ошибок, если асимптотические являются сложными, например, в случае двухшаговых оценок. Кроме того, при корректной реализации, бутстрэп может привести к улучшению статистических свойств оценок в случае малого числа наблюдений.
	
	
В главе 12 рассматриваются методы оценивания, основанные на симуляциях, разработанные для моделей, где используется интеграл по закону распределения, для которого не существует явного аналитического решения. Оценивание возможно путём генерирования нескольких выборок из нужного распределения и последующего усреднения.
	
	
В главе 13 представлен байесовский подход, в котором  распределение наблюдаемых данных комбинируется с  априорным распределения параметров для получения  апостериорного распределения параметров, на базе которого осуществляется оценивание. Современные вычислительные методы  позволяют выполнить расчёты даже в случае при отсутствии аналитического представления для апостериорного распределения. Байесовский подход к оцениванию и статистическим выводам сильно отличается от классического подхода. Тем не менее, во многих случаях только байесовский подход позволяет решить проблемы, которые в противном случае оказываются неразрешимыми.

\subsection{Часть 4. Модели для пространственных данных}

В главах 14-20 представлены основные нелинейные модели для пространственных данных. Эта часть является основным элементом книги и содержит сложные темы, такие как модели для качественных зависимых переменных и самоотбор выборки. Классы моделей определяется исходя из диапазона значений, который принимает зависимая переменная.
	
	
Модели для случая бинарной зависимой переменной, то есть принимающей только два возможных значения, $y = 0$ или $y = 1$, представлены в главе 14. В Главе 15 приведено обобщение до мультиномиальных моделей, в которых зависимая переменная  принимает несколько дискретных значений. В качестве примеров можно привести статус занятости (занятые, безработные, и входящие в рабочую силу) или используемый для поездки на работу транспорт (на автомобиле, автобусе или поезде). Линейные модели могут быть полезными, но не адекватными, так как они могут предсказывать значение вероятности больше 1. Вместо линейных используется логит, пробит, и связанные с ними модели.
	
	
В главе 16 представлены модели с цензурированием, усечением и самоотбором выборки. Примерами могут служить: количество часов работы в год при решении работать, и медицинские расходы при госпитализации. В этих случаях данные не полностью наблюдаются, есть большое количество как $y = 0$, так и $y > 0$. Модель для наблюдаемых данных является нелинейной, даже если  процесс для скрытой переменной является линейным, а линейная регрессия по наблюдаемым данным  может сильно вводить в заблуждение. Возможна простая корректировка для цензурирования, усечения и смещения самоотбора, например, использование тобит моделей, но они сильно зависят от  предположений о законе распределения.
	
	
Модели длительности представлены в главах 17---19. Примером может служить продолжительность безработицы. Стандартные модели регрессии включают экспоненциальную регрессию, модель Вейбулла, и модель пропорциональных рисков Кокса. Кроме того, как и в главе 16, зависимая переменная часто наблюдается лишь частично. Например, может быть известна длительность текущего состояния, а не полная длительность, т.к. состояние не является законченным.
	
	
В Главе 20 представлены  модели для счётных данных. Например, количество визитов к врачу или длительность госпитализации в днях. Опять же модели являются нелинейными так как количества и, следовательно, условные средние неотрицательны. Известными параметрическими моделями являются модель Пуассона и отрицательная биномиальная.

\subsection{Часть 5. Модели анализа панельных данных}

В главах 21-23 рассматриваются способы оценки панельных данных. В этом случае данные наблюдаются в несколько периодов времени для каждого из индивидов в выборке, поэтому зависимая переменная и регрессоры проиндексированы как по времени, так и номеру индивида. Любой анализ должен учитывать возможность положительной корреляции ошибок в разные периоды времени для конкретного индивида. Кроме того, панельных данных обычно достаточно, чтобы учесть ненаблюдаемые постоянные во времени индивидуальные  эффекты, что позволяет делать причинно-следственные выводы при более слабых предположениях, чем  в пространственной выборке.


Основные линейные модели панельных данных представлены в главе 21, с акцентом на модели с фиксированным и случайным эффектом. Обобщения до моделей, допускающих лагированную зависимую переменную в качестве регрессора или эндогенные регрессоры, представлены в главе 22.  Нелинейные методы части 4 для панельных данных изложены в главе 23.


Методы для панельных данных представлены ближе к концу книги для единого и цельного изложения. Главу 21 можно было бы разместить сразу после главы 4. Она написана в доступной манере изложения и не требует дополнительных знаний кроме метода наименьших квадратов.

\subsection{Часть 6. Дальнейшие темы}

В этой части рассматриваются важные темы, которые могут в целом относится ко всем без исключения моделям рассмотренным в частях 4 и 5. Глава 24 посвящена моделированию кластерных данных в рамках различных моделей. 
В главе 25 обсуждается оценивание воздействия. Оценивание воздействия является общим термином, который может охватывать широкий спектр моделей, в которых основное внимание уделяется измерению влияния некоторого воздействия. Воздействие назначается либо экзогенно, либо случайно, при этом интерес состоит в измерении некоторой результирующей переменной. В главе 26 рассматриваются последствия ошибок измерения зависимой переменной или  регрессоров, с акцентом на некоторых распространённых нелинейные модели. В главе 27 рассматриваются некоторые методы работы с пропущенными наблюдениями в линейных и нелинейных моделях регрессии.

\section{Как пользоваться книгой}

Книга предполагает общее представление о модели линейной регрессии в матричной форме. Она написана на математическом уровне первого года Ph.D. по экономике и по уровню сопоставима с учебником Грина (2003).


Хотя часть материала этой книги изучается на первом году обучения по программе Ph.D., большая часть  появляется на втором курсе эконометрики Ph.D. или в курсах микроэкономики ориентированных на реальные данные, таких как экономика труда, государственная экономика, или теория отраслевых рынков. Книга может быть полезна как отдельный учебник эконометрики или как дополнительный учебник к указанным курсам. В целом, книга представляет собой справочник для прикладных исследователей в области экономики, в смежных социальных науках, таких как социология и политология и в эпидемиологии.
	
	
Для читателей, использующих эту книгу в качестве справочника, главы в которых представлены модели по возможности были сделаны максимально независимыми от остального содержания книги. В частях 4 и 5 для ознакомления с конкретными моделями  достаточно прочитать отдельно соответствующую главу. Может лишь потребоваться владение общими результатами теории оценивания из главы 5 и реже из главы 6. Большинство глав начинаются с обсуждения и примера понятного широкой аудитории.
	
\begin{table}[h]
\begin{center}
\caption{\label{tab:plan}План 20-ти лекций 10-ти недельного курса}
\begin{tabular}[t]{llcll|}
\hline
\bf{Лекции} & \bf{Главы} & \bf{Темы} \\
\hline
1-3   & 4, Приложение А & Повторение линейной модели и асимптотической теории \\
4-7   & 5               & Оценивание: М-оценивание, ММП, и НМНК\\
8     & 10              & Оценивание: численная оптимизация \\
9-11  & 14, 15          & Модели: бинарные и мультиномиальные \\
12-14 & 16              & Модели: цензурированные и усечённые \\
15    & 6               & Оценивание: ОММ \\
16    & 7               & Тестирование: проверка гипотез \\
17-19 & 21              & Модели: базовые линейные панели \\
20    & 9               & Оценивание: полупараметрическое \\
\hline
\end{tabular}
\end{center}
\end{table}
	
Преподавателям,  использующим эту книгу в качестве основного учебника курса, стоит ознакомить студентов как можно раньше с нелинейными пространственными моделями и линейными моделями для панельных данных, пропуская множество глав посвященных методам.
Основные нелинейные модели представлены в главах 14---16, они требуют знания метода максимального правдоподобия и метода наименьших квадратов, изложенных в главе 5. В главе 21 рассматриваются линейные модели панельных данных, для её чтения достаточно ознакомиться с главой 4.
	
	
В таблице 1.2 представлен план полусеместрового курса второго года магистратуры в Калифорнийском Университете в Дэвисе, который следует сразу же после курса статистики и эконометрики на первом году обучения. Половины семестра достаточно чтобы охватить основные результаты первой половины глав плана.
При наличии дополнительного времени можно углубиться в детали или покрыть частично материал глав 11-13 по методам требующим большого количества вычислений (симуляционное оценивание, бутстрэп, которые также кратко представлен в главе 7 и байесовские методы); дополнительный материал по пространственным моделям (модели длительности и счётные модели, представленные в главах 17-20); и дополнительный материал по панельным данным (обобщения линейных моделей и нелинейные модели, изложенные в главах 22 и 23).

	
	
В Университете штата Индиана, Блумингтон, 15-ти недельный курс микроэконометрики основан на материале, представленном в частях 4 и 5. Курсы, необходимые для начала  освоения 15-ти недельного курса, покрывают материал части 2.

Упражнения для самоподготовки представлены в конце каждой главы после первых трех вводных главах. Некоторые из них являются чисто методологическими, тогда как другие связаны с анализом фактических данных. Уровень сложности вопроса в основном связан с уровнем сложности темы.
	
\section{Программное обеспечение}

Существует огромное число пакетов программного обеспечения для анализа данных. Популярными пакетами с большими микроэконометрическими возможностями являются LIMDEP, SAS, и STATA, каждый из которых предлагает впечатляющий ассортимент встроенных процедур, а также допускает создание пользовательских процедур на матричном языке. Среди других широко используемых пакетов можно привести Eviews, PCGIVE и TSP,  они ориентированны на временные ряды, но несмотря на это  также могут использоваться для панельных данных и пространственных выборок. Пользователи, которые хотят писать собственные программы для анализа могут воспользоваться такими программами, как GAUSS, MATLAB, OX и SAS/IML. Свежую информацию об этих пакетах можно легко найти в Интернете.

\section{Обозначения и соглашения}

В данной книге весь анализ строится с применением векторов и матриц.
Векторы представлены в виде столбцов и обозначаются строчными буквами. К примеру, для линейной регрессии вектор регрессоров $ x $ --- это столбец размерности $ K \times 1 $, где на $j$-ом месте находитс $ x_{j} $, а вектор параметров $\beta$ --- это столбец размерности $ K \times 1 $, где на $j$-ом месте находится $ \beta_{j} $:

\[
\underset{(K\times1)}{x} = \begin{bmatrix} x_{1} \\ \vdots \\ x_{k} \end{bmatrix} \qquad 
\underset{(K\times1)}{\beta} = \begin{bmatrix} \beta_{1} \\ \vdots \\ \beta_{k} \end{bmatrix}
\]
	
Поэтому линейная регрессионная модель $y=\beta_{1}x_{1}+\beta_{2}x_{2}+\ldots +\beta_{k}x_{k}+u$ представляется в виде $y=x'\beta+u$. Иногда индекс $i$ добавляется для обозначения $i$-го наблюдения. Линейная регресси для $i$-го наблюдения записывается в виде:

\[
y_{i}=x_{i}'\beta+u_{i}
\]

В выборке обычно $N$ наблюдений, ${(y_{i}, x_{i}), i=1,\ldots ,N}$, а наблюдения обычно независимы по $i$.

Матрицы обозначаются заглавными буквами. В матричных обозначения выборка записывается в виде $(y,X)$, где $y$ вектор размера $ N \times 1 $ с $y_{i}$ на $i$-ом месте, а $X$ --- это матрица со строками $x_{i}'$:

\[
\underset{(N\times1)}{y} = \begin{bmatrix} y_{1} \\ \vdots \\ y_{N} \end{bmatrix} \qquad 
\underset{(N\times dim(x))}{X} = \begin{bmatrix} x_{1}' \\ \vdots \\ x_{N}' \end{bmatrix}
\]

В этом случае линейная регрессионная модель для всех наблюдений выглядит так:

\[
y=X\beta+u
\]
,
где $u$ вектор-столбец размера $ N \times 1 $ с $u_{i}$ на $i$-ом месте.

Матричная запись компактна, но иногда удобнее вместо произведения матриц записывать сумму произведений векторов. К примеру,  МНК-оценка может быть записана как:

\[
\hat{\beta}=(X'X)^{-1}X'y=\left(\sum_{i=1}^N x_i x'_i\right)^{-1}\sum_{i=1}^N x_i y_i.
\]
	
\begin{table}[h]
\caption{Часто используемые сокращения}
\label{tab:abbrev}
\begin{tabular}{@{}lll@{}}
\toprule
 \multirow{6}{*}{Линейные} & МНК & Метод наименьших квадратов  \\
 & ОМНК & Обобщенный метод наименьших квадратов \\
 & ДОМНК & Доступный обобщенный метод наименьших квадратов \\
 & - & Метод инструментальных переменных \\
 & 2МНК & Двухшаговый метод наименьших квадратов \\
 & 3МНК & Трехшаговый метод наименьших квадратов \\ \midrule
 \multirow{5}{*}{Нелинейные} & НМНК & Нелинейный метод наименьших квадратов \\
 & ДОНМНК & Доступный обобщенный нелинейный метод наименьших квадратов \\
 & - & Нелинейный метод инструментальных переменных \\
 & Н2МНК & Нелинейный двухшаговый метод наименьших квадратов \\
 & Н3МНК & Нелинейный трехшаговый метод наименьших квадратов \\ \midrule
 \multirow{4}{*}{Общие} & ММП & Метод максимального правдоподобия \\
 & - & Метод квази-максимального правдоподобия \\
 & ОММ & Обобщенный метод моментов \\
 & - & Обобщенный метод оценивающих уравнений \\ \bottomrule
\end{tabular}
\end{table}

Неизвестные параметры как правило обозначаются вектором $\theta$ размера $q\times 1$.  Параметры регрессии представлены в виде   вектора $ \beta $ размера $ K \times 1 $. Эти два вектора могут совпадать, или параметры регрессии могут быть частью вектора $\theta$ в зависимости от контекста.


В книге используются много сокращений и аббревиатур. В таблица 1.3 представлены все сокращения, используемые для некоторых распространенных методов оценки  линейных или нелинейных моделей регрессии. Мы часто используем следующие понятия: процесс порождающий данные (dgp, data generating process), одинаково распределенные и независимые (iid, independently и identically distributed), функция плотности (pdf, probability density function), функция распределения (cdf, cumulative distribution function), функция правдоподобия (L, likelihood), логарифмическая функция правдоподобия ($\ln L$, loglikelihood), фиксированные эффекты (FE, fixed effects), и случайные эффекты (RE, rиom effects)\footnote{В русском переводе количество используемых сокращений уменьшено}.











\chapter{Причинно-следственные и статистические модели}

\section{Введение}

	Микроэконометрика посвящена теориям и методам анализа микроданных, относящимся к отдельным лицам, домохозяйствам и компаниям. Более широкое определение может также включать данные на региональном или государственном уровне. Микроданные, как правило, представляют собой либо пространственную выборку, тогда они содержат характеристики объектов на  конкретную дату, либо панельные данные, тогда они содержат характеристики объектов за несколько периодов времени. Данные могут быть неэкспериментальными, например, данные из опросов и переписей или экспериментальными и квази-экспериментальными, например, данные социальных  экспериментов, проводимых властями с участием добровольцев. 
	
	
	Микроэконометрическая модель может полностью  специфицировать закон распределения для всех наблюдений; она также может частично специфицировать некоторые свойства распределения, например моменты, для части переменных. Математическое ожидание зависимой переменной при фиксированных регрессорах представляет особый интерес. 
	
	
	Целями микроэконометрики является как описание данных, так и нахождение причинно-следственных связей. 
	Первая цель может быть вольно определена как описание свойств моментов зависимой переменной и нахождение уравнений регрессий, описывающих статистическую, а не причинно-следственную связь. Вторая цель подразумевает исследование причинно-следственных связей и состоит в измерении силы, а также эмпирическом подтверждении или опровержении гипотез о микроэкономическом поведении. Тип и стиль эмпирических исследований, поэтому многообразен. 
	
	На одном конце спектра находятся высоко структурированные модели, полученные из подробной спецификации экономического поведения, с помощью которых анализируют причинные (поведенческие) или структурные отношения для взаимозависимых микроэкономических переменных. На другом конце находятся исследования приведенной формы, которые направлены на то, чтобы выявить корреляцию между переменными, не обязательно полагаясь на детальную спецификацию всех соответствующих взаимозависимостей. Оба подхода имеют общую цель в раскрытии важных и интересных отношений, которые могут быть полезны для понимания микроэкономического поведения, но они отличаются по степени, в которой они опираются на экономическую теорию. 
	

	Как отдельная дисциплина микроэконометрика моложе, чем макроэконометрика, которая занимается моделированием рыночных и агрегированных данных. Многие из ранних работ по прикладной эконометрике были основаны на агрегированных временных рядах, собранных государственными учреждениями. В большей части ранних работ по статистическому анализу спроса вплоть до приблизительно 1940 года использовались рыночные данные, а не индивидуальные или данные домашних хозяйств (Хендри и Морган, 1996). Книга Моргана (1990) по истории эконометрики не содержит ссылок на микроэконометрические работы до 1940 года, за одним важным исключением.  Это исключение --- работа по бюджетам домохозяйств, созданная при исследовании уровня жизни  менее обеспеченных слоев населения в разных странах. В результате были собраны данные по бюджетам домохозяйств,  которые явились основой для ряда ранних микроэконометрических исследований, например, работы Аллена и Боули (1935). Однако лишь в 1950 году микроэконометрика стала признанной и отдельной дисциплиной. Даже в 1960-х годах основная часть макроэконометрических работа была связана с анализом спроса на базе опросов домохозяйств. 
	
		
	После вручения Нобелевской премии по экономике в 2000 году Джеймсу Хекману и Даниэлу МакФаддену за их вклад в микроэконометрику, данная область добилась четкого признания в качестве отдельного раздела науки. Хекману вручили премию <<за развитие теории и методов анализа данных с самоотбором выборки>>, а МакФаддену <<за развитие теории и методов анализа моделей дискретного выбора>>. Примеры тем, с которыми сталкивается микроэконометрика, были упомянуты в цитате: <<\ldots какие факторы определяет, будет ли индивид работать, и если да, то сколько часов? Как экономические факторы влияют на решение индивида об образовании, роде деятельности и месте жительства? Как влияют различные образовательные программы и программы по трудоустройству на занятость и доход индивидов?>>.
	
		
	Применение микроэконометрических методов можно найти не только в  микроэкономике, но и в других родственных социальных науках, таких как политология, социология, и география. 
	
	
	Начиная с 1970-х и особенно в течение последних двух десятилетий произошли революционные изменения в анализе больших объемов данных и соответствующих вычислительных методах. Вместе с этим стало доступно огромное количество микроэкономических наборов данных, что позволило значительно расширить сферу микроэконометрики. 
	В результате, хотя эмпирический анализ спроса продолжает оставаться одной из наиболее важных областей применения микроэконометрических методов, его стиль и содержание были существенно изменены под влиянием новых методов и моделей. Сейчас микроэконометрические методы широко применяются в таких областях, как анализ экономического развития,  финансы, здравоохранение, теория отраслевых рынков, экономика труда, экономика государственного сектора, и эти приложения будут продемонстрированы в разных частях данной книги.
	
	
	Основное внимание в этой книге уделяется новым достижениям, которые появились в последние три десятилетия. Нашей целью является обзор концепций, моделей и методов, которые мы рассматриваем в качестве стандартных  инструментов современного исследователя. Конечно, понятие стандартных методов и моделей является субъективным, и зависит от предполагаемых читателей и уровня авторов книги. Также есть темы, которые мы считаем слишком сложными для такой вводной книги, как эта, но которые были бы по-другому оценены другими авторами.
	
		
	Микроэконометрика уделяет особое внимание нелинейным моделям и получению оценок, которые могут иметь структурную интерпретацию. Большая часть этой книги, особенно Части 2---4, рассматривает методы для нелинейных моделей. Эти нелинейные методы пересекаются со многими областями прикладной статистики, включая биостатистику. Отличительной особенностью эконометрики является именно моделирование причинно-следственных связей. В этой главе представлены основные понятия, связанные с причинно-следственным и описательным моделированием, концепции, которые относятся как к линейным, так и к нелинейным моделям. 
	
	
	В Разделах 2.2 и 2.3 вводятся ключевые понятия структурности и экзогенности. В Разделе 2.4 представлены линейные системы одновременных уравнений для иллюстрации структурной модели и указана их связь с моделями в приведенной форме. Идентифицируемость определяется в разделе 2.5.  Раздел 2.6 рассматривает структурные модели из одного уравнения. В Разделе 2.7 вводятся модели потенциального исхода и их причинно-следственная интерпретация сравнивается с моделями одновременных уравнений. Раздел 2.8 содержит краткое обсуждение стратегий моделирования и оценивания при наличии вычислительных трудностей.
	
	
\section{Структурные модели}

В описание структурной модели входят:
\begin{enumerate}
\item набор переменных $W$ разделённых для удобства на $[Y \, Z]$
\item закон распределения вероятностей для $W$, $F(W)$
\item априорное упорядочивание $W$ в соответствии с гипотетическими причинно-следственными связями и спецификация априорных ограничений на гипотетическую модель
\item параметрическая, полупараметрическая или непараметрическая спецификация функциональной формы и ограничений на параметры модели.
\end{enumerate}

Данное определение структурной модели соответствует определению, данному комиссией Коулса. Например, Сарган (1988, стр. 27) утверждает: \\


Модель --- это спецификация закона распределения  вероятностей для набора наблюдений. Структура это спецификация параметров данного распределения. Следовательно, структура --- это модель в которой параметрам присвоены  числовые значения. 


Рассмотрим случай, когда цель моделирования заключается в объяснении наблюдаемых значений векторной переменной $y$, $y'= (y_{1}, \ldots , y_{G})$. Каждый элемент $y$ является функцией некоторых других элементов $y$, объясняющих переменных $z$ и случайных ошибок $u$. Обратите внимание, что переменные $y$ считаются взаимозависимыми. Заметим, что  взаимозависимость между $z_{i}$ не моделируется. Наблюдение номер  $i$ удовлетворяет набору неявных уравнений:
	
\begin{equation}
g(y_{i},z_{i},u_{i}|\theta)=0,
\end{equation}

где $g$ --- известная функция. Мы называем такой тип уравнений структурной моделью, а $\theta$  --- структурными параметрами. Это соответствует свойству 4, которое приведено в начале раздела. 


Предположим, что существует единственное решение $y_{i}$ для каждой пары $(z_{i},u_{i})$. Тогда мы можем записать уравнения в явном виде, где  $y$ --- это функция от $(z, u)$:

\begin{equation}
y_{i}=f(z_{i},u_{i}|\pi).
\end{equation}


Это уравнение называется приведенной формой структурной модели, где $\pi$ --- вектор параметров приведенной формы, который является функцией от $\theta$. Приведенная форма получается путем решения уравнения структурной модели относительно эндогенных переменных $y_{i}$ при фиксированных $(z_{i},u_{i})$. 


Если целью моделирования является получение выводов об элементах $\theta$, то (2.1) обеспечивает прямой путь, который подразумевает оценку структурной формы. Однако, поскольку элементы $\pi$ являются функциями $\theta$, (2.2) также предоставляет косвенный путь для построения выводов относительно $\theta$. Если $f(z_{i},u_{i}|\pi)$ имеет известную функциональную форму, и если она аддитивно-сепарабельна по $z_{i}$ и $u_{i}$, то есть мы можем записать её в виде:
	
	
	
\begin{equation}
y_{i}=g(z_{i}|\pi)+u_{i}=E[y_{i}|z_{i}]+u_{i},
\end{equation}

тогда  регрессия $y$ на $z$ является естественной прогнозной функцией $y$ для заданного $z$. В этом смысле приведенная форма модели полезна для условных прогнозов $y_{i}$ при заданных $(z_{i},u_{i})$. Чтобы получить прогнозы объясняемой переменной левой части  для заданных значений объясняющих переменных правой части (2.2) требуется иметь оценки $\pi$, получение которых может быть вычислительно проще. 

Важным обобщением (2.3) является  модель с преобразованием, которая для скалярного $y$ принимает следующий вид:
	
\begin{equation}
\Lambda(y)=z'\pi+u,
\end{equation}

где $\Lambda(y) $ является преобразования (например,  $\Lambda(y)=ln(y)$ или $\Lambda(y)=y^{1/2}$). В некоторых случаях преобразование может зависеть от неизвестных параметров. Модель с преобразованием  отличается от регрессии, но она также может быть использована для оценки $E[y|x]$. Важным примером такой модели является модель ускоренная модель жизни (см. Главу 17). 


Одним из наиболее важных и спорных этапов в спецификации структурной модели является пункт 3, который априори располагает переменные  в порядке причинно-следственных связей.
В сущности мы проводим различия между теми переменными, чьё изменение объясняется моделью и теми, чьё изменение определяется внешними факторами и, следовательно, выходит за рамки исследования. В микроэконометрике, в качестве первых выступают количество лет обучения в школе и продолжительность рабочего времени; примерами переменных второго типа являются пол, этническая принадлежность, возраст, и аналогичные демографические показатели. Переменные первой группы, обозначаемые $y$, называются эндогенными, а переменные второй группы, обозначаемые $z$, называется экзогенными переменными. 
	
	
Экзогенность переменное является важным упрощением и по сути оправдывает решение трактовать данную переменную как вспомогательную, и отказаться от её моделирования, поскольку параметры модели для неё не оказывают прямого влияния на изучаемую переменную.  Экзогенность требует более формального определения, которое мы сейчас предоставим. 

	
\section{Экзогенность}

	
Мы начнем с рассмотрения общего случая конечномерного параметрического представления, в котором совместное распределение $W$, с параметрами $\theta$, разбитыми на $(\theta_{1},\theta_{2})$, раскладывается в произведение условной плотности распределения $Y$ при заданном $Z$, и в частной функции плотности $Z$:


\begin{equation}
f_{J}(W|\theta)=f_{C}(Y|Z,\theta)\times f_{M}(Z|\theta).
\end{equation}

Частным случаем является разложение вида:

\[
f_{J}(W|\theta)=f_{C}(Y|Z,\theta_{1})\times f_{M}(Z|\theta_{2}),
\]

где $ \theta_{1} $ и $ \theta_{2} $ функционально независимы. Тогда говорят, что $Z$ является экзогенной по отношению к  $ \theta_{1} $, а это означает, что знание $ f_{M}(Z|\theta_{2}) $  не требуется для статистических выводов о $ \theta_{1}$ и, следовательно, мы можем изучать распределение $Y$ при фиксированном  $Z$. 


Модели можно параметризовать по-разному. Поэтому далее рассмотрим случай, в котором модель описана с помощью параметров $\varphi$, которые находятся во  взаимно-однозначном соответствии с параметрами $\theta$, скажем $\varphi=h(\theta)$, где $\varphi$ разбивается на $(\varphi_{1},\varphi_{2})$. Такая параметризация может представлять интерес, если $\varphi_1$ инвариантны относительно некоторого класса реформ. Допустим, что цель состоит в оценивании $\varphi_1$. Следовательно нас интересует экзогенность $Z$ относительно $\varphi_1$. В таком случае условие экзогенности выглядит следующим образом:
	
\begin{equation}
f_{J}(W|\varphi)=f_{C}(Y|Z,\varphi_{1})\times f_{M}(Z|\varphi_{2}),
\end{equation}

где $\varphi_{1}$ и $\varphi_{2}$ независимы.

%%%%%
Наконец, рассмотрим случай, когда параметр $\lambda$ функцией от $\varphi$, скажем $h(\varphi)$. Тогда для экзогенности $Z$ относительно $\lambda$, нам нужны два условия: (i) $\lambda$ зависит только от $ \varphi_{1}) $, т.е. $\lambda=h(\varphi)$, (при этом интерес представляет только условное распределение) и (ii) $\varphi_{1}$ и $\varphi_{2}$ не имеют перекрестных ограничений, т.е. $(\varphi_{1}, \varphi_{2}) \in \Phi_{1}\times\Phi_{2} = \lbrace\varphi_{1} \in \Phi_{1}, \varphi_{2} \in \Phi_{2}\rbrace$. 

Разложение в (2.5) --- (2.6) играет важную роль в развитии концепции экзогенности. В этой книге акцентируется внимание на трех понятиях экзогенности: (1) слабая экзогенность; (2) отсутствие причинности по Грейнджеру; (3) сильная экзогенность.

{\bf Определение 2.1 (слабая экзогенность):}  $Z$ является слабо экзогенной по $\lambda$, если условия (i) и (ii) выполнены.\\

Если параметры модели неинформативны для статистических выводов о $\lambda$, то статистические выводы о $\lambda $  могут быть сделаны только на основе условного распределения $f(Y|Z,\varphi_{1})$. То есть на практике можно принять слабую экзогенность переменных как данность, если основной интерес состоит в получении выводов о $\lambda$ или $\varphi_1$. Это не означает, что нет никакой статистической модели для $Z$, однако параметры этой модели не играют никакой роли в статистических выводах относительно $\varphi$, и, следовательно, не имеют значения. 

%%% --->
\subsection{Условная независимость}


Первоначально концепция причинности по Грейнджеру была определена в контексте предсказания  временных рядов. В целом, она может быть интерпретирована как форма условной независимости (Холланд, 1986, с. 957). 


Разделяя вектор $z$ на два подмножества $z_{1}$ и $z_{2}$, обозначим с помощью $W=[y,z_{1},z_{2}]$ матрицу интересующих нас переменных. Переменные $z_{1}$ и $y$ условно независимы при заданном $z_{2}$ если

\begin{equation}
f(y|z_{1},z_{2})=f(y|z_{2})
\end{equation}

Это предположение сильнее, чем независимость условного среднего:
 
\begin{equation}
\E(y|z_{1},z_{2})=\E(y|z_{2})
\end{equation}
%%%
В этом случае $z_{1}$ не имеет ценности для прогнозирования $y$. То есть $z_{1}$ не является причиной $y$ по Грейнджеру.

В контексте временных рядов $z_1$ и $z_2$ --- это непересекающиеся подмножества лагированных значений переменной $y$.

{\bf Определение 2.2 (сильная экзогенность):}  $z_{1}$ сильно экзогенна для $\varphi$, если она слабо экзогенна для $\varphi$ и не является причиной по Грейнджеру для $y$, т.е. выполнено (2.8).

\subsection{Экзогенные переменные}

%%% --->
	Экзогенность является сильным предположением. Это свойство случайных величин по отношению к параметрам, представляющим интерес. Поэтому переменная может обоснованно считаться экзогенной в одной структурной модели, а в другой нет, ключевым вопросом является выбор целевых параметров. Необдуманное введение этого предположения будет иметь некоторые нежелательные последствия, которые будут обсуждаться в разделе 2.4. 
	
	
	Предположение об экзогенности может быть априори оправдано теорией, в этом случае оно является частью поддерживаемой гипотезы модели. В некоторых случаях экзогенность может быть хорошей аппроксимацией, в этом случае могут она может тестироваться способами, описанными в разделе 8.4.3. В анализе пространственных данных  данное предположение может быть оправдано как следствие естественного эксперимента или квази-эксперимента, в котором значение переменной определяется внешним вмешательством, например, правительство или регулирующий орган могут выбирать ставку налога или параметр вмешательства. Особый интерес представляет случай, когда внешнее вмешательство приводит к изменению значения важных переменных политики. Такой естественный эксперимент равносилен экзогенности некоторых переменных. Как мы увидим в главе 3, это создает квазиэкспериментальные возможности изучения влияния переменной при отсутствии других осложняющих факторов. 


\section{Линейная модель одновременных уравнений}


Важным частным случаем общей структурной модели описанной в (2.1) является система линейных одновременных уравнений, разработанная эконометристами Комиссии Коулса. Подробное изложение данной модели можно найти во многих учебниках (например, Сарган, 1988). В данном разделе предлагается краткое и выборочное изложение данной модели, см. также раздел 6.9.6. Цель состоит в том, чтобы обсудить нескольких ключевых идей и концепций, которые имеют более общий характер. Хотя анализ ограничен линейными моделями, много интересных идей применимо к нелинейным моделям.
  
\subsection{Система одновременный уравнений}

\textbf{Система линейных одновременных уравнений} (СЛОУ) представляется в виде
\[
\begin{aligned}
y_{1i}\beta_{11}+ &\dots + y_{Gi}\beta_{1G}+z_{1i}\gamma_{11} + \dots + &z_{Ki}\gamma_{1K}=  &u_{1i} \\
                  &\vdots                                               &\vdots           =  &\vdots \\
y_{1i}\beta_{G1}+ &\dots + y_{Gi}\beta_{GG}+z_{1i}\gamma_{G1} + \dots + &z_{Ki}\gamma_{GK}=  &u_{Gi},
\end{aligned}
\]

где $i$ --- обозначает номер наблюдения.

Четкое различие или упорядочивание сделано между эндогенными переменными, $y'_{i}=(y_{1i},\dots,y_{Gi})$, и экзогенными переменными, $z'_{i}=(z_{1i},\dots,z_{Ki})$. По определению экзогенные переменные не коррелируют со случайными ошибками $u_{1i},\dots,u_{Gi}$. В неограниченной форме каждая переменная входит в каждое уравнение. 

В матричной форме уравнение для $i$-го наблюдения системы из $G$ уравнений выглядит следующим образом:

\begin{equation}
y'_{i}B+z'_{i}\Gamma=u'_{i},
\end{equation}

где $y'_{i}$,$B$,$z'_{i}$,$\Gamma$ и $u'_{i}$ имеют размерности $G \times 1$ , $G \times G$ , $ K \times 1$, $K \times G$ и $G \times 1$, соответственно. Для конкретных значений $(B,\Gamma)$ и $(z_i,u_i)$ система из $G$ одновременных уравнений в принципе может быть решена относительно $y_i$.

Стандартные предположения для СЛОУ следующие:

\begin{enumerate}
\item $B$ невырожденна и имеет ранг $G$
\item $\rank Z = K$,   матрица $Z$ размера $N \times K$ составлена из $z'_{i}$, $i=1, \dots, N$, расположенных друг под другом.
\item $\plim N^{-1}Z'Z=\Sigma_{ZZ}$ --- симметричная положительно определенная матрица размера $K \times K$.
\item $u_{i}\sim N[0,\Sigma]$, то есть  $\E[u_{i}]=0$ и $\E[u_{i}u'_{i}]=\Sigma=[\sigma_{ij}]$, где $\Sigma$ --- симметричная положительно определенная матрица размера $G \times G$.
\item Ошибки в каждом уравнении не зависят от прошлых значений.
\end{enumerate}

В этой модели структурными параметрами являются  $(B,\Gamma,\Sigma)$. Введем обозначения

\[
Y=\begin{bmatrix} y'_{1} \\ \vdots \\ y'_{N} \end{bmatrix}, \quad  Z=\begin{bmatrix} z'_{1} \\ \vdots \\ z'_{N} \end{bmatrix}, \quad U=\begin{bmatrix} u'_{1} \\ \vdots \\ u'_{N} \end{bmatrix}
\]

Тогда можно представить структурную модель в более компактной записи:

\begin{equation}
YB+Z\Gamma=U,
\end{equation}

где $Y$,$B$, $Z$,$\Gamma$ и $U$ имеют размерности $N \times G$ , $G \times G$ , $ N \times K$, $K \times G$ и $N \times G$, соответственно. В таком случае, решение для всех эндогенных переменных в терминах экзогенных выглядит следующим образом:


\begin{equation}
Y+Z\Gamma B^{-1}=UB^{-1},
Y=Z\Pi+V,
\end{equation}

где $\Pi= -\Gamma B^{-1}$, с учётом предпосылки 4, $v_{i} \sim N[0,B^{-1^{'}} \Sigma B^{-1}]$. Данная форма записи СЛОУ называется \textbf{приведенной формой}.

%%% -->
	В рамках СЛОУ структурная модель более важна по нескольким причинам. Во-первых, сами уравнения имеют экономическую интерпретацию, например, описывают спрос, предложение или  производственные функции, и так далее, и на них распространяются ограничения, диктуемые экономической теорией. Следовательно, $B$ и $\Gamma$ являются параметрами, описывающими экономическое поведение. Следовательно, априорная теория может быть использована для формирования гипотез относительно знака и размера отдельных коэффициентов. С другой стороны, параметры неограниченной приведенной формы являются сложной функцией структурных параметров, и поэтому сложно делать выводы после получения оценок. Этот довод может иметь малый вес, если целью эконометрического моделирования являются прогнозы, а не выводы о параметрах, имеющих поведенческую интерпретацию.  

	
Рассмотрим, без ограничения общности, первое уравнение в модели (2.11), с зависимой переменной $y_{i}$. Кроме того, некоторые из оставшихся $G-1$ эндогенных переменных и $K-1$ экзогенных переменных могут не включаться в это уравнение. 
Из (2.12) мы видим, что в общем случае эндогенные переменные $Y$ зависят стохастически от $V$, которые в свою очередь являются функцией структурных ошибок $U$, поэтому, в общем  случае $\plim N^{-1}Y'U\neq 0$. Как правило, применение метода наименьших квадратов в системах линейных одновременных уравнений дает несостоятельные оценки. Этот хорошо известный и базовый результат в литературе по системам одновременных уравнений часто называют проблемой смещения из-за одновременности. Литература по одновременным уравнениям посвящена идентификации и состоятельному оцениванию в случае, когда  метод наименьших квадратов не работает, см. работы Саргана (1988) и Шмидта (1976), а также раздел 6.9.6. 
	
	
В приведенной форме СЛОУ  каждую эндогенная переменная представлена в виде линейной функции всех экзогенных переменных и всех структурных ошибок. Ошибки в приведенной форме являются линейными комбинациями ошибок структурной формы. Из приведенной формы для $i$-го наблюдения получаем:

\begin{equation}
\E[y_{i}|z_{i}]=z'_{i}\Pi,
\end{equation}

\begin{equation}
\V[y_{i}|z_{i}]=\Omega=B^{-1^{'}} \Sigma B^{-1}
\end{equation}
 
 
Параметры приведенной формы $\Pi$ являются производными параметрами, зависящими от структурных параметров. Если $\Pi$  можно состоятельно оценить, то приведенная форма может быть использована для прогнозирования изменений $Y$ под влиянием внешних изменений в $Z$. Это возможно даже если матрицы $B$ и $\Gamma$ не известны. Учитывая экзогенность $Z$, полный набор регрессий приведенной формы является многомерной регрессионной моделью, которая может быть состоятельно оценена методом наименьших квадратов. Приведенная форма позволяет строить прогнозы $Y$ при заданных $Z$. 


Ограниченная приведенная форма получается из неограниченной при добавлении ограничений. Если эти ограничения совпадают с ограничениями структуры, то структурная информация может быть извлечена из приведенной формы. 
	
	
В рамках СЛОУ, неизвестные структурные параметры, ненулевые элементы $B$, $\Gamma$, $\Sigma$, играют ключевую роль, потому что они описывают причинную структуру модели. Взаимозависимость между эндогенными переменными описывается матрицей $B$, а реакция эндогенных переменных на экзогенные шоки в $Z$ отражается в параметрах матрицы $\Gamma$. Поэтому нас интересуют те параметры, которые измеряют прямой предельный эффект изменения независимой переменной, $y_{j}$ или $z_{k}$ на зависимую $y_{l}$, где $l\neq j$, а также функции этих параметров и данных. Элементы $\Sigma$ описывают дисперсию и структуру зависимости случайных ошибок, следовательно, они измеряют некоторые свойства процесса, порождающего данные. 
	

\subsection{Причинная интерпретация в СЛОУ}


Простой пример может иллюстрировать причинную интерпретацию параметров в СЛОУ. Допустим, структурная модель имеет две непрерывные эндогенные переменные $y_{1}$ и $y_{2}$, одну непрерывную экзогенную переменную $z_{1}$, одно стохастическое соотношение, связывающее $y_{1}$ и $y_{2}$ и одно тождество, связывающее все три переменные в модели:


\[
y_{1}=\gamma_{1}+\beta_{1}y_{2}+u_{1}, \quad 0<\beta_{1}<1, \\
y_{2}=y_{1}+z_{1}.
\]

%%% --->
В этой модели $u_{1}$ является случайной ошибкой, независимой от $z_{1}$, с заданным  распределением. На параметр $\beta_{1}$ наложено ограничение в виде неравенства, которое также является частью спецификации модели. Переменная $z_{1}$ является экзогенной и, следовательно, её изменения вызваны внешними источниками, которые мы можем рассматривать как вмешательства. Эти вмешательства  оказывают непосредственное влияние на $y_{2}$ через тождество и косвенное влияние через первое уравнение. Это влияние измеряется при помощи приведенной формы модели, которая выглядит следующим образом: 

\[
y_{1}=\frac{\gamma_{1}}{1-\beta_{1}}+\frac{\beta_{1}}{1-\beta_{1}}z_{1}+\frac{1}{1-\beta_{1}}u_{1} \\
= \E[y_{1}|z_{1}]+v_{1}, \\
\]

\[
y_{2}=\frac{\gamma_{1}}{1-\beta_{1}}+\frac{1}{1-\beta_{1}}z_{1}+\frac{1}{1-\beta_{1}}u_{1} \\
= \E[y_{2}|z_{1}]+v_{1},
\]

где $v_{1}=\frac{u_{1}}{1-\beta_{1}}$. Коэффициенты приведённой формы $\frac{\beta_{1}}{1-\beta_{1}}$ и $\frac{1}{1-\beta_{1}}$  имеют причинную интерпретацию. Внешнее   изменение $z_{1}$ на единицу вызовет  изменение $y_{1}$ и $y_{2}$ на соответствующие коэффициенты. Следует отметить, что в этой модели $y_{1}$ и $y_{2}$ также реагируют на изменения $u_{1}$. Чтобы не смешивать влияние двух источников изменчивости, требуется, чтобы $z_{1}$ и $u_{1}$ являлись независимыми. 

Также отметим, что

\[
\frac{\partial y_{1}}{\partial y_{2}}=\beta_{1}=\frac{\beta_{1}}{1-\beta_{1}}\div\frac{1}{1-\beta_{1}} \\
=\frac{\partial y_{1}}{\partial z_{1}}\div\frac{\partial y_{2}}{\partial z_{1}}.
\]


В каком смысле $\beta_{1}$ измеряет причинно-следственное влияние $y_{2}$ на $y_{1}$? Чтобы увидеть возможные трудности, заметим, что $y_{1}$ и $y_{2}$ являются взаимозависимыми или, другими словами,  определяются одновременно, так что неясно, в каком смысле $y_{2}$ <<влияет на>> $y_{1}$. 
Хотя $z_{1}$ (и $u_{1}$) являются конечной причиной изменений в смысле  приведённой формы, $y_{2}$ является промежуточной причиной изменения $y_{1}$. То есть, первое структурное уравнение представляет собой мгновенное влияние  $y_{2}$ на  $y_{1}$, в то время как приведённая форма показывает равновесное воздействие после учета всех взаимодействий  между эндогенными переменными. В рамках СЛОУ даже эндогенные переменные рассматриваются как причинные, а их коэффициенты --- как причинные параметры. Такой подход может вызвать недоумение тех, кто рассматривает причинность в контексте экспериментальных условий, когда независимые источники изменения являются причинными переменными. Подход СЛОУ имеет смысл, если $y_{2}$ имеет независимый и экзогенный источник изменений, в модели данный источник представлен как $z_{1}$. Поэтому предельный коэффициент $\beta_{1}$ является функцией того, как $y_{1}$ и $y_{2}$ реагируют на изменение $z_{1}$, что и показывает уравнение выше.


Конечно, данная модель является только частным случаем. В общем случае мы можем задаться вопросом: при каких условиях параметры СЛОУ имеют осмысленную причинную интерпретацию? Мы вернемся к этому вопросу при обсуждении понятия идентификации в разделе 2.5.


\subsection{Нелинейный модели и модели скрытых переменных}


Если система одновременных уравнений нелинейна только по параметрам, то структурную модель можно записать в виде:



\begin{equation}
YB(\theta)+Z\Gamma(\theta)=U,
\end{equation}

где $B(\theta)$ и $\Gamma(\theta)$ матрицы, элементами которых зависят от структурных параметров $\theta$. Приведенная форма в явном виде может быть получена как раньше. 

Если модель нелинейна по переменным, то получение приведенной формы в явном (аналитическом) виде может быть невозможно. Впрочем, обычно зависимые переменные можно выразить численно или воспользоваться линейной аппроксимацией.


Во многих микроэконометрических моделях присутствуют скрытые или ненаблюдаемые переменные, помимо наблюдаемых эндогенных переменных. Например, в моделях поиска и моделях аукционов используется понятие минимального уровня оплаты труда или резервной цены, в моделях выбора --- косвенной функции полезности, и так далее. В случае таких моделей структурная модель (2.1), может быть заменена на

\begin{equation}
g(y^{\ast}_{i},z_{i},u_{i}|\theta)=0,
\end{equation}

где скрытая переменная $y^{\ast}_{i}$ заменяет наблюдаемую переменную $y_{i}$. Соответствующая приведенная форма выражает $y^{\ast}_{i}$ через $(z_{i},u_{i})$ в виде

\begin{equation}
y^{\ast}_{i}=f(z_{i},u_{i}|\pi).
\end{equation}


Эта приведенная форма имеет ограниченную пользу, так как $y^{\ast}_{i}$ не наблюдаются полностью. Тем не менее, если у нас есть функция $y_{i}=h(y^{\ast}_{i})$, которая связывает  наблюдаемые  переменные $y_{i}$ с соответствующими скрытыми, то приведенная форма может быть записана с помощью  наблюдаемых переменных  так:

\begin{equation}
y_{i}=h(f(z_{i},u_{i}|\pi)).
\end{equation}
Подробнее см. в разделе 16.8.2


Когда структурная форма модели  нелинейна по  переменным или, когда существуют скрытые переменные, приведённую форму может быть  трудно получить. В таких случаях на практике используется аппроксимация. Из соображений удобства аналитических выкладок или численных подсчетов, специальная функциональная может использоваться для того, чтобы выразить эндогенную переменную через  экзогенные. Данный результат называют приведенной формой записи модели.  



\subsection{Интерпретация структурных отношений}


Маршак (1953, стр. 26) в своей статье, оказавшей существенное влияние, дает следующее определение структуры: 


Структура определяется как набор условий, которые не меняются в то время, пока делаются наблюдения, но которые могут измениться в будущем. Если некоторое изменение структуры ожидаемо, то для прогнозирования переменных при осуществлении некоторой политики необходимо  знание прошлой структуры\ldots 
В экономике, условиями, определяющими структуру являются (1) соотношения, описывающие поведение человека или институциональную среду, как правило включающие случайные возмущения или случайные ошибки измерения, (2) совместное распределение этих случайных величин. 


Маршак утверждал, что структура имеет большое значение для количественной оценки или тестирования экономической теории, а выбор наилучшей политики требует знания структуры. 


В литературе про системам одновременных уравнений  структурная модель относится к <<автономным>> (не выведенным) взаимосвязям. Есть и другие близкие понятия структуры. Одна из таких концепций формулируется с помощью <<глубоких параметров>>, под которыми подразумеваются параметры технологии или предпочтений, не зависящие от внешних воздействий. 


В последние годы распространилось альтернативное использование термина структуры. Под структуро подразумевают эконометрические модели, основанные на гипотезе динамической стохастической оптимизации рациональными агентами. При таком подходе отправной точкой для любой структурной модели  является множество необходимых условий первого порядка, которое определяют оптимальное поведение агента. 
Например, в стандартном случае  решения задачи максимизации полезности поведенческими условиями являются будут детерминистические условия первого порядка на предельную полезность. Если соответствующие функциональные формы заданы в явной форме, и присутствуют стохастические ошибки оптимизации, то условия первого порядка определяют поведенческую модель, параметры которой характеризуют функцию полезности. Эти параметры называют  <<глубокими>> или инвариантными по отношению к политике. Примеры рассмотрены в разделах 6.2.7 и 16.8.1. 


Укажем две особенности этого сильно структурированного подхода. 
Во-первых, он весьма серьёзным образом априорно полагается на экономическую теорию. Экономическая теория  используется не просто для создания списка релевантных переменных, которые можно использовать в более или менее произвольной  функциональной форме. Скорее,  экономическая теория играет большую (но не исключительную) роль в спецификации, оценивании и статистических выводах. Второй особенностью является то, что идентификация, спецификация и оценивание полученной модели могут быть очень сложными, потому что сама задача оптимизации агента является потенциально сложной, особенно если это динамическая оптимизация в условиях неопределенности, с наличием дискретности и разрывов в данных, см. работу Раста (1994) .


\section{Идентификация}


Цель подхода систем одновременных уравнений заключается в получении состоятельной оценки $(B,\Gamma,\Sigma)$ и в проведении статистических выводов. Важной предпосылкой для состоятельности оценок является идентифицируемость модели. Мы кратко обсуждаем две важные  взаимосвязанные концепции  эквивалентности и идентифицируемости в контексте параметрических моделей.


Идентификация состоит в определении значений параметров при достаточном количестве наблюдений. В этом смысле она является асимптотической концепцией. Статистическая неопределенность неизбежно затрагивает любые выводы на основе конечного числа наблюдений. 
Предположив, что имеется достаточное количество наблюдений, мы встаем перед вопросом, возможно ли определить значение интересующего нас параметра. Речь идет как об определении точечного значения, так и об определении множества возможных  значений. 
Идентификация является одним из ключевых свойств, логически происходит до и отдельно от  статистического оценивания. Большая часть эконометрической литературы под идентификацией подразумевает точечную идентификацию. В этом разделе мы также работает с понятием точечной идентификации. Тем не менее, идентификация множества или идентификация границ, также являются важными подходами, которые будут использоваться в отдельных местах этой книги (например, главы 25 и 27, см. Мански, 1995).


{\bf Определение 2.3 (Эквивалентность):} Две структуры модели, определяемые с помощью закона распределения  $\Pr[x|\theta]$,$x\in W$, $\theta \in \Theta$,  эквивалентны, если $\Pr[x|\theta^{1}]=\Pr[x|\theta^{2}] \forall x \in W.$ 


Менее формально,  если две структурных  модели приводят к идентичным  совместным функциям   распределения, то они эквивалентны. Существование множества  эквивалентных структур равносильно отсутствию идентификации.

{\bf Определение 2.4 (Идентификация):} Структура $\theta^{0}$ идентифицируема, если не существует других эквивалентных ей структур в множестве $\Theta$.


Простой пример неидентифицируемости  возникает в случае жесткой мультиколлинеарности между регрессорами в линейной модели $y=X\beta+u$. Тогда мы можем идентифицировать линейную комбинацию $C\beta$, где $\rank[C] < \rank[\beta]$, но мы не можем идентифицировать $\beta$.


Это определение относится к уникальности структуры. В контексте систем одновременных уравнений  это определение означает, что существует единственная тройка $(B,\Gamma,\Sigma)$ соответствующая данным. 
В системах одновременных уравнений, как и в других случаях, идентификация подразумевает возможность получить уникальные оценки структурных параметров при известных выборочных моментах. Например, в случае приведенной формы (2.12), при указанных предположениях, метод наименьших квадратов позволяет получить единственные оценки матрицы $\Pi$, то есть $\hat{\Pi}=[Z'Z]^{-1}Z'Y$, а для идентифицируемости  $(B,\Gamma)$ необходимо, чтобы существовало решение относительно $B$ и $\Gamma$  уравнения $\Pi+\Gamma B^{-1}=0$, при некоторых  априорных ограничениях  модели. 
Единственность решения означает точную  идентифицируемость модели.

Модель в целом называется идентифицируемой,  если все параметры модели идентифицируемы. Вполне возможно, что в некоторых моделях только подмножество параметров идентифицируемо. В некоторых случаях важно иметь возможность идентифицировать заданную функцию параметров, и не обязательно все  параметры по отдельности. Идентифицируемость функции от параметров означает, что функция может быть однозначно восстановлена по $F(W|\Theta)$.


Как можно убедиться в том, что структуры альтернативных спецификаций модели могут быть исключены? В системах одновременных уравнений решение этой проблемы связано с добавлением к имеющимся наблюдениям априорных ограничений на $(B,\Gamma,\Sigma)$. Эти априорные ограничения должны внести в модель достаточное количество дополнительной информации, чтобы исключить существование других эквивалентных структур.
	
	
О необходимости априорных ограничений свидетельствует следующее рассуждение. Прежде всего заметим, что при предположениях раздела 2.4.1, приведенная форма, определяемая формулой $(\Pi,\Omega)$, всегда единственна. Изначально предположим, что нет никаких ограничений на $(B,\Gamma,\Sigma)$. Далее предположим, что существует два эквивалентных структуры $(B_{1},\Gamma_{1},\Sigma_{1})$ и $(B_{2},\Gamma_{2},\Sigma_{2})$. Тогда
	
\begin{equation}
\Pi=-\Gamma_{1} B^{-1}_{1}=-\Gamma_{2} B^{-1}_{2}.
\end{equation}


Пусть $H$ --- невырожденная  матрица размера $G \times G$, тогда
$\Gamma_{1} B^{-1}_{1}=\Gamma_{1} H H^{-1} B^{-1}_{1}=\Gamma_{2} B^{-1}_{2}$, что означает, что $\Gamma_{2}=\Gamma_{1}H$, $B_{2}=B_{1}H$. Таким образом, вторая структура является линейным преобразованием первой.


В системах одновременных уравнений решение данной проблемы заключается во введение ограничений на $(B,\Gamma,\Sigma)$ таких, что мы можем исключить существование линейных преобразований, которые приводят к  эквивалентным структурам. Другими словами, ограничения на $(B,\Gamma,\Sigma)$, должны быть такими, чтобы не существовало матрицы $H$, которая даёт другую структуру с такой же приведённой формой; при заданных $(\Pi,\Omega)$ будет существовать единственные решения уравнений $\Pi=-\Gamma B^{-1}$ и $\Omega=(B^{-1})'\Sigma B^{-1}$.
	
	
На практике могут быть наложены различные ограничения, в том числе (1) нормализация, например, можно приравнять диагональные элементы матрицы $B$ к единице, (2) исключение (зануление) отдельных коэффициентов, и линейные однородные и неоднородные ограничения и (3) ковариационные ограничения и ограничения в виде неравенств.  Подробную информация о необходимых и достаточных условиях  идентификации  линейных и нелинейных моделей можно найти во многих текстах, включая книгу Саргана (1988).
	
	
Исключающие ограничения по существу означают, что модель содержит некоторые переменные, которые имеют нулевое влияние на некоторые эндогенные переменные. То есть, определенные направления причинно-следственных связей априори невозможны. Это дает возможность идентифицировать другие направления причинно-следственных связей. Например, в простом случае с двумя переменными, приведенном выше, $z_{1}$ не входит в уравнение $y_{1}$, что позволяет оценить  прямое воздействие $y_{2}$ на $y_{1}$. Хотя исключающие ограничения являются самыми простыми для реализации, в параметрических моделях идентификация может быть достигнута за счет ковариационных ограничений или ограничений типа неравенств.
	
	
Если нет никаких ограничений на $\Sigma$, а диагональные элементы $B$ нормированы и равны 1, то необходимым условием для идентификации является условие порядка, согласно которому, количество исключенных из уравнения экзогенных переменных должно быть не меньше, чем  количество включенных эндогенных переменных. Часто приводимым в учебниках достаточным условием является условие ранга, которое гарантирует, что  параметры $j$-го уравнения $\Pi \Gamma_{j}=-B_{j}$ соответствуют единственному решению для $\Gamma_{j},B_{j}$ при заданном $\Pi$.

Если модель идентифицируема, то используют термин точно идентифицируема для случая, когда условие порядка выполнено как равенство, и  термин сверх-идентифицируема для случая, когда количество ограничений на систему уравнений превосходит количество, необходимое для точной идентификации.

Идентификация в нелинейных системах уравнений обсуждает Сарган (1988), он также приводит ссылки на более ранние работы.
	
\section{Модель с одним уравнением}


Без ограничения общности рассмотрим первое уравнение системы одновременных уравнений при нормализации $\beta_{11}=1$. Пусть $y=y_{1}$, а $\by_{1}$ обозначает эндогенные компоненты $\by$ отличные от $y_{1}$, и пусть $z_{1}$ обозначает экзогенные компоненты $z$, тогда


\begin{equation}
y=\by'_{1}\alpha+z'_{1}\gamma+u.
\end{equation}

Во многих работах пропускаются формальные шаги при переходе от системы уравнений к одному уравнению и начинают с написания уравнения регрессии

\[
y=x'\beta+u,
\]

где некоторые компоненты $x$ являются эндогенными (неявно $\by_{1}$), а другие экзогенными (неявно $z_{1}$).
Основная задача состоит в оценке влияния изменений  ключевых регрессоров, которые могут быть эндогенными или экзогенными, в зависимости от предположений. Инструментальные переменные или двухшаговый метод наименьших квадратов являются естественными способами оценивания (см. разделы 4.6, 6.4 и 6.5).


В системах одновременных уравнений естественно специфицировать по крайней мере некоторые из оставшихся уравнений в модели, даже если они не являются предметом исследования. Пусть $y_{1}$ имеет размерность 1. Тогда одна из возможностей состоит в том, что можно указать структурное уравнение для $y_{1}$ и других эндогенных переменных, которые могут появиться в  структурном  уравнении для $y_{1}$. Вторая возможность состоит в использовании уравнения для  $y_{1}$ в приведенной форме. В этой форме будут видны экзогенные переменные, которые влияют на $y_{1}$, но не влияют напрямую на $y$. Преимуществом является то, что при таком подходе инструментальные переменные возникают естественным путем. Тем не менее, в последних эмпирических исследованиях с использованием инструментальных переменных для одного уравнения формальный шаг с записью уравнения в приведенной форме для эндогенной переменной в правой части нередко пропускают.
	
	
\section{Модели потенциального результата}


Необходимость причинно-следственных рассуждений в эконометрических моделях особенно сильна, когда акцент делается на влияние государственной политики или индивидуальных решений на некоторый результат. Примерами могут служить  влияние социальных трансфертов на предложение труда, влияние количества учеников в классе на эффективность обучения, а также влияние наличия медицинской страховки на интенсивность пользования услугами здравоохранения. 
Во многих случаях причинные переменные сами отражают индивидуальные решения и, следовательно, потенциально эндогенны. Когда, как это обычно бывает, эконометрические оценки и выводы основаны на данных наблюдений, идентификация и выводы о причинных параметрах ставят перед исследователем немало проблем. 
Эти трудности можно быть преодолены, если для причинно-следственного моделирования  использовать данные, полученные в результате контролируемого социального эксперимента с продуманным планом. 
Хотя такие эксперименты существуют (см. примеры и подробности в разделе 3.4.), как правило, они дороги и их трудно организовать.
Поэтому гораздо удобнее оценивать причинно-следственные модели на данных полученных в результате  естественного эксперимента или в квази-экспериментальных условиях. 
В разделе 3.4 обсуждаются плюсы и минусы таких  данных. Для текущих задач о естественном эксперименте или о квази-экспериментальных условиях, можно думать как о ситуации, в которой некоторые причинные переменные изменяются экзогенно и независимо от других объясняющих переменных, что позволяет относительно легче  оценить причинно-следственные параметры.
	
	
Одним из основных препятствий для моделирования причинности является фундаментальная проблема причинно-следственных статистических выводов (Холланд , 1986). Пусть $X$ --- предполагаемая причина, а $Y$ --- результат. Изменяя значение $X$, мы можем изменить значение $Y$. Предположим, что значение $X$ изменяется с $x_{1}$ до $x_{2}$. Мера причинного  изменения $Y$ определяется путем сравнения двух значений $Y$: $y_{2}$, которое является результатом изменений, и $y_{1}$, которое было бы, если бы не было никаких изменений в $X$. Однако, если $X$ изменился, то значение $Y$, которое было бы в отсутствие изменений, не наблюдается. Следовательно, ничего больше нельзя сказать о причинном влиянии без некоторых гипотез о том, какое значение имел бы $Y$ в отсутствие изменений $X$. Это гипотетическое ненаблюдаемое значение называют контрфактическим. Иными словами, все причинно-следственные  выводы должны быть сделаны исходя из сравнения фактического и контрфактического результата. В обычной эконометрической модели (например системах одновременных уравнений) нет необходимости явно формулировать контрфактический результат.
	
	
Относительно новый сюжет в микроэконометрической литературе --- оценивание эффективности программы мер или оценивание воздействия --- задает статистическую парадигму для оценивания   причинно-следственных параметров. В статистической литературе этот подход также известен как причинно-следственная модель Рубина (Rubin causal model, RCM), названной в знак признания существенного вклада Рубина ( 1974, 1978 ), который  в свою очередь, ссылается на Р.А. Фишера, как создателя подхода. Согласно действующей традиции, мы называем данный подход причинно-следственной моделью Рубина, однако Нейман (Шплава-Нейман) также предложил похожую модель в статье опубликованной на польском языке в 1923 году, см. работу Неймана (1990). Модели с использованием контрфактических значений в эконометрике разрабатывались независимо после ключевой работы Роя (1951).  В оставшейся части этого раздела будут проанализированы характерные особенности модели Рубина. 


Оценивание причинно-следственных параметров на основе контрфактических ситуаций дает  статистически осмысленное и рабочее  определение причинности, которое в некоторых отношениях отличается от традиционного определения комиссии Коулса. Во-первых, в идеальных условиях этот подход приводит к использованию простых эконометрических методов. Во-вторых, при этом подходе внимание сосредоточено на небольшом количестве параметров, которые считаются релевантными для анализируемой политики. Здесь видно отличие с традиционным эконометрическим подходом, где внимание уделяется оценке всех структурных параметров. В-третьих, данный подход дает дополнительные сведения о свойствах причинно-следственных параметров, оцениваемых структурными методами.
	
\subsection{Причинно-следственная модель Рубина}



Мы используем термины <<воздействие>> и <<причина>> наравне. В медицинских исследованиях нового лекарства участвуют две группы людей: те, кто принимает лекарство и те, кто не принимает. Реакцию на препарат у тех, кто принимал, сравнивают с показателями у тех, кто не принимал лекарство. Мерой причинно-следственного воздействия является средняя разница показателей этих групп. В экономике термин воздействие используется очень широко. По существу он охватывает переменные, влияние которых на некоторый результат является объектом исследования. Примерами причинно-следственных связей являются: связь количества лет обучения с заработной платой, размер класса и успеваемость, обучение на рабочем месте и заработок. Следует отметить, что воздействие не обязательно должно быть экзогенным, а во многих случаях оно является эндогенной переменной.


В рамках  модели потенциального результата (potential outcome model, POM),  предполагается, что каждый индивид целевой группы населения потенциально может быть подвержен воздействию. Оценивание строится на базе тройки $(y_{1i},y_{0i},D_{i})$, $i=1,\dots,N$. Переменная $D$ принимает значения 1 и 0 соответственно, когда воздействие получено или не получено; $y_{1i}$ измеряет зависимую переменную для индивида $i$, получающего воздействие, а $y_{0i}$ --- для не получающего. То есть


\begin{equation}
y_{i}=
\begin{cases}
y_{1i}, & \text{ если }D_{i}=1 \\
y_{0i}, &  \text{ если }D_{i}=0.
\end{cases}
\end{equation}

Получение и неполучение воздействия являются взаимоисключающими для $i$-го индивида, только одна из двух зависимых переменных доступна для $i$-го индивида, недоступная величина является контрфактической. Эффект от воздействия $D$ измеряется как $(y_{1i}-y_{0i})$. Средний причинно-следственный эффект $D_{i}=1$, относительно $D_{i}=0$, измеряется с помощью среднего эффекта воздействия (average treatment effect, АТЕ):

\begin{equation}
ATE=\E[y|D=1]-\E[y|D=0],
\end{equation}

где ожидания берутся относительно вероятностного распределения по целевой группе населения. В отличие от обычной структурной модели, в которой важны  предельные эффекты, в модели потенциального результата подчеркивается роль среднего эффекта  воздействия и параметров, связанных с ним.


Экспериментальный подход к оценке параметров ATE состоит в случайном назначении воздействия с последующим сравнением результатов с индивидами, не получившими воздействия, которые служат в качестве контрольной группы. Этот план выборки подробнее обсуждается в главе 3. 
Случайное назначение воздействия  предполагает, что лица, подвергающиеся воздействию выбираются случайным образом, и, следовательно, назначение воздействия не зависит от результата и не коррелирует с характеристиками индивидов. 
В этом случае имеют место два существенных упрощения.
Индикатор воздействия может рассматриваться как экзогенная переменная. Коэффициент при нем в линейной регрессии не будет страдать от смещения вызванного пропуском переменных, даже при неизбежном пропуске  некоторых существенных переменных в регрессии. При определенных условиях, обсуждаемых более подробно в главах 3 и 25, средняя разница между результатами экспериментальной и контрольной групп даст оценку ATE. Выигрыш от хорошо продуманных экспериментов заключается в относительной простоте, с которой могут быть сделаны выводы о причинно-следственных связях. Конечно, чтобы достичь высокой точности оценки эффекта воздействия, необходимо учитывать в модели факторы, которые помимо самого воздействия могли повлиять на результат. 
	
	
Из-за того, что случайное назначение воздействия, как правило, невозможно в экономике, оценивание ATE параметров должно быть основано на данных наблюдений, полученных при неслучайном назначении воздействия. 
В такой ситуации состоятельное оценивание ATE будет сопряжено с трудностями, которые включают, например, возможную корреляцию между гипотетическим результатом и назначением воздействия, пропущенные переменные, и эндогенность индикатора воздействия. Некоторые эконометристы считают, что отсутствие рандомизации является главным препятствием для убедительного статистического вывода о причинно-следственных связях.
	
	
Модель потенциального результата можно использовать для заключений о причинно-следственных связях, если контрфактические величины можно четко определить и работать с ними.
Ясное определение контрфактических величин с объяснением того, какие показатели нужно сравнивать, является важной составляющей этой модели. Если четко не определена разница  между наблюдаемыми и контрфактическими величинами,  как это может быть в случае с данными наблюдений, то ответ на вопрос о том, на кого происходит воздействие, остается неясным. 
Величина ATE взвешивает и комбинирует предельные эффекты разных групп генеральной совокупности. Для того, чтобы с контрфактическими величинами можно было работать необходимы определенные предположения. Для того, чтобы оценить ATE необходимо учесть результаты и группы подвергавшейся воздействию, и группы неподвергавшейся воздействию.
Например, необходимо определить группу индивидов, неподвергавшуюся воздействию, которая наиболее похожа по характеристикам на экспериментальную группу, если бы к ней воздействие не применялось. Не всегда данный шаг может быть реализован.
Способ отбора индивидов, получающих воздействие, связан с особенностями плана выборки, обсуждаемыми в главах 3 и 25.

	
	
Вторая полезная особенность моделей потенциального результата состоит в том, что они позволяют увидеть  возможности причинно-следственного моделирования, появившиеся в результате естественного или квази-эксперимента. Когда данные порождаются в таких условиях, и при наличии некоторых предположений,  моделирование причинно-следственных связей  может происходить и без всех сложностей систем одновременных уравнений. Эта тема развивается далее в главах 3 и 25.

%%%% --->
В-третьих, в отличие от структурной формы систем одновременных уравнений, где все переменные кроме объясняемых могут считаться <<причинами>>, в модели потенциального результата не все объясняющие переменные можно рассматривать как причинные. Многие из них просто являются характеристиками индивида, которые необходимо включать в регрессионный анализ, а характеристики не являются причинами (Холланд, 1986). Причинные параметры должны быть связаны с переменными, на которые фактически или потенциально, прямо или косвенно, можно воздействовать.
	
	
Наконец, идентифицируемость  параметров ATE может быть более легкой исследовательской целью  и, следовательно, возможной в ситуациях, когда  система одновременных уравнений в целом не идентифицируема (Ангрист, 2001). Вопрос о верности данного утверждения надо решать отдельно в каждом конкретном случае. Заметим, что многие из имеющихся приложений модели потенциального результата обычно используют ограниченную, а не полную информацию. Тем не менее, даже в рамках систем одновременных уравнений использование подхода ограниченной информации также возможно, как было описано ранее.


%%% --->
\section{Причинное моделирование и стратегии оценивания}


В этом разделе мы кратко расскажем о некоторых подходах, которые используют эконометристы для моделирования причинно-следственных связей. Эти подходы могут быть использованы и в системах одновременных уравнений, и в моделях потенциального результата, но, их как правило, связывают с системами одновременных уравнений.

\subsection{Подходы к идентификации}
\begin{center}
Структурные модели с полной информацией
\end{center}

%%%
Один из вариантов данного подхода основан на параметрической спецификации совместного распределения эндогенных переменных при фиксированных экзогенных. Взаимосвязи не обязательно являются следствием из  модели оптимального поведения. Параметрические ограничения накладываются для обеспечения идентификации параметров модели, которые являются целью статистического анализа. Вся модель оценивается одновременно с использованием метода максимального правдоподобия или метода моментов. Мы называем этот подход структурным подходом с полной информацией. Для хорошо специфицированных моделей этот подход является привлекательным, но в целом его потенциальный недостаток заключается в том, что модель может содержать плохо специфицированные  уравнения. При одновременном оценивании ошибка спецификации в одном месте может затронуть оценки остальных параметров. 


Этот подход статистически можно интерпретировать, как подход, в котором совместное распределение  эндогенных переменных, при фиксированных экзогенных переменных, лежит в основе вывода о причинности. Совместное распределение может быть следствием одновременной или динамической взаимозависимости между эндогенными переменными и/или случайными ошибками разных уравнений.
	
	
\begin{center}
Структурные модели с неполной информацией
\end{center}

%%% 
Напротив, когда главной задачей статистического анализа является оценка одного или двух ключевых параметров, может быть использован подход с ограниченной информацией. 
Особенностью данного подхода является то, что, хотя  в центре внимания находится одно уравнение, учитывается зависимость между ним и другими эндогенными переменными. Это требует явных предположения о некоторых особенностях модели, которые не являются основным объектом исследования. Метод инструментальных переменных, последовательные многошаговые методы, и метод  максимального правдоподобия с ограниченной информацией являются  примерами такого подхода. Для реализации данного подхода обычно работают с одним (или более) структурным уравнением и несколькими явно или неявно заданными уравнениями в приведенной форме. Здесь видно отличие от подхода полной информации, где все уравнения являются структурными. Подход с ограниченной информацией часто требует меньшего объёма вычислений, чем подход с полной информацией.

%%%
Статистически можно считать, что в подходе с ограниченной информацией   совместное распределение раскладывается в произведение условной модели для эндогенной переменной, представляющей интерес, например $y_{1}$, и частной модели для других эндогенных переменных, скажем, $y_{2}$, которые входят во множество условных переменных, а именно

\begin{equation}
f(y|x,\theta)=g(y_{1}|x,y_{2},\theta_{1})h(y_{2}|x,\theta_{2}), \quad \theta \in \Theta.
\end{equation}


Моделирование может быть основано на компоненте $g(y_{1}|x,y_{2},\theta_{1})$, при этом минимальное  внимание уделяется компоненте $h(y_{2}|x,\theta_{2})$, если $\theta_{2}$ рассматриваются как мешающие параметры. Конечно, такое разложение не является единственным, поэтому может существовать несколько вариантов подход с ограниченной информацией.


\begin{center}
Идентифицируемая приведённая форма
\end{center}


Третий вариант подхода СЛОУ работает с идентифицируемой приведённой  формой. Здесь тоже мы заинтересованы в структурных параметрах. Тем не менее, может быть удобно оценивать параметры восстановленной формы с учетом ограничений. В качестве примера можно привести идентифицируемые  векторные авторегрессии.


\subsection{Стратегии идентификации}

%%% ---->
Есть множество потенциальных проблем, из-за  которых идентификация ключевых параметров модели может оказаться под угрозой. Пропущенные переменные, неправильная спецификация функциональной формы, ошибки  измерения объясняющих переменных, использование данных нерепрезентативных для генеральной совокупности, и игнорирование эндогенности объясняющих переменными являются основными примерами. Микроэконометрика содержит много конкретных примеров того, как эти проблемы могут быть решены. Ангрист и Крюгер (2000) приводят подробный обзор популярных стратегий идентификации в экономике труда, с акцентом на модели потенциального результата. Большинство вопросов рассматриваются в других частях книги и кратко упоминаются здесь.


\begin{center}
Экзогенизация
\end{center}
%%% --->

Данные иногда порождаются в результате эксперимента или квази-эксперимента. Идея заключается в том, что переменная, описывающая воздействие, может экзогенно измениться для некоторой части выборки и остаться неизменной для другой части выборки. Например,  минимальная заработная плата в одном штате может измениться, а в соседнем --- нет. Такие события естественным образом создают тестовую и контрольную группы. Если естественный эксперимент похож на случайное назначение воздействия, то использование таких данных для оценки структурных параметров может быть проще, чем оценка большой системы одновременных уравнений с эндогенными переменными воздействия. Возможно также, что переменную воздействия в естественном эксперименте можно рассматривать как экзогенную, но воздействие само по себе не назначается случайным образом.


\begin{center}
Устранение мешающих параметров
\end{center}

%%% -->
Идентификация может быть затруднена при наличии большого количества мешающих параметров. Например, в пространственной регрессионной модели функция условного математического ожидания $\E[y_{i}|x_{i}]$ может включать в себя индивидуальные фиксированные эффекты $\alpha_{i}$, которые коррелируют с ошибками регрессии. 
Эти эффекты не могут быть идентифицированы без большого количества наблюдений для каждого индивида (т.е. без панельных данных). Тем не менее, с небольшой панелью от этих эффектов можно  избавиться путём простого преобразования модели. Другим примером является наличие не меняющейся во времени экзогенной переменной, которая может быть общей для групп индивидов. Примером преобразования, устраняющего фиксированные эффекты, является взятие разностей и работа с моделью в форме разности разностей.


\begin{center}
Учитывание смешивающих факторов
\end{center}


Когда невключенные в модель регрессоры коррелируют с включенными переменными, оценки становятся смещёнными. Например, в регрессии, где заработная плата является зависимой переменной, а длительность обучения --- объясняющей переменной, индивидуальные способности можно рассматривать как не включённую переменную, поскольку обычно доступны только её несовершенные прокси. Это означает, что потенциально коэффициент при длительности обучения не может быть идентифицирован. Одной из возможных стратегий является введение контрольных переменных в модель; это называется подходом контрольных функций. Эти переменные являются попыткой примерного учёта не включённых переменных. Например, различные учебные достижения могут служить в качестве контрольных переменных для индивидуальных способностей.



\begin{center}
Создание искусственных выборок
\end{center}


В рамках модели потенциального результата причинный параметр  может быть неидентифицируемым из-за отсутствия подходящей для сравнения контрольной группы. Возможным решением является создание искусственной выборки, включающей сравнительную группу, которая была бы прокси для контрольной группы. Такая выборка создается путем сопоставления (см. главу 25). Если выборка, получившая воздействие, может быть дополнена подходящей контрольной выборкой, то идентификация причинных параметров может быть достигнута, а именно, параметр, связанный с ATE может быть оценён.


\begin{center}
Инструментальные переменные
\end{center}


Если идентификация невозможна, потому что переменные воздействия являются эндогенными, то стандартное решение заключается в использовании инструментальных переменных. Впрочем, это легко сказать, а трудно сделать. Выбор инструментальных переменных, а также интерпретация полученных результатов должны быть сделаны аккуратно, потому что результаты могут быть чувствительны к выбору инструментов. Этот подход анализируется в разделах 4.8, 4.9, 6.4, 6.5 и 25.7, а также в других местах книги по мере необходимости. Опять же естественный эксперимент может предоставить годную инструментальную переменную.
\\
\\
\begin{center}
Перевзвешивание выборок
\end{center}


Выводы о генеральной совокупности на основе выборки действительны только, если выборочные данные являются репрезентативными. Проблема самоотбора выборки возникает тогда, когда выборочные данные не являются репрезентативными, и в этом случае параметры генеральной совокупности неидентифицируемы. Эта проблема может быть решена как введением поправки на самоотбор выборки (Глава 16) так и корректировкой выборочных весов (Глава 24).



\subsection{Библиографические заметки}

\begin{enumerate}
\item Нобелевские лекции 2001 года Хекмана и МакФаддена являются прекрасным источником информации об истории и современном состоянии микроэконометрики. Лекция Хекмана замечательна своей широтой и содержит интересные факты, касающиеся разных аспектов микроэконометрики. Его рассуждения о неоднородности соприкасаются с разными сюжетами данной книги.
\item Работа Маршака (1953) --- это классический пример, показывающий превосходство структурного моделирования для оценки политики. Это одно из ранних упоминаний идеи инвариантности параметров. 
\item Энгл, Хендри и Ричард (1983) приводят определения сильной и слабой экзогенности в рамках распределения наблюдаемых переменных. Также в их работе можно найти отсылки к более ранней литературе, посвященной концепции экзогенности. 
\item Термин <<индентификация>> использовался Купмансом (1949). Точечная идентификация в линейных параметрических моделях описывается в большинстве учебников, например, у Саргана (1988) можно найти полное и подробное изложение, также стоит упомянуть книги Дэвидсона и МакКиннона (2004) и Грина (2003). Гурьеру и Монфорт (1989, глава 3.4) обсуждает идентификацию с точки зрения расстояния Кульбака-Лейблера. Интервальная идентификация на нескольких важных примерах исследуется в работе Мански (1995).
\item Хекман (2000) описывает историю и современную интерпретацию причинности в традиционных эконометрических моделях. Вопросы причинности в контексте модели потенциального результата аккуратно анализирует в своей работе Холланд (1986), он также исследует связь причинности с другими понятиями. В работе Фридмана (1999) можно найти историческую подборку взглядов статистиков на тему причинности. Пирл (2000) поучительно схематично излагает идею <<трактовки причинности как поведения в результате вмешательства>>. Также в работе Пирла описано множество проблем с причинной трактовкой в неэкспериментальных ситуациях. 
\item Ангрист и Крюгер (1999) приводят обзор решений проблем связанных с идентификацией с примерами из экономики труда. 


\end{enumerate}












\chapter{Структуры микроэкономических данных}

\section{Введение}

В данной главе дается обзор потенциальной полезности и ограничений различных типов микроэкономических данных. Наиболее распространенной структурой данных, используемой в микроэконометрики являются данные обследования или переписи. Эти данные, как правило, называются данными наблюдений, чтобы отличить их от экспериментальных данных.


В этой главе обсуждается потенциальное ограничение вышеупомянутых структур данных. Ограничения, присущие данным наблюдений могут усиливаться способом сбора данных, то есть основой выборки (sample frame) (способ создания выборки), планом выборки (sample design) (простая случайная выборка или стратифицированная случайная выборка) и объемом выборки (sample scope) (пространственные или панельные данные). Поэтому мы также обсуждаем вопросы построения выборки в связи с использованием данных наблюдений. Все новые термины будут определены позже в данной главе.
	
	
Микроэконометрика выходит за рамки анализа данных обследования в предположении простой случайной выборки. В этой главе представлены различные обобщения. Раздел 3.2 описывает структуру многоэтапного выборочного обследования и некоторые распространенные отклонения от случайной выборки; более детальный анализ их статистических последствий обсуждается в следующих главах. Здесь также рассматриваются некоторые часто встречающиеся проблемы, в результате которых выборка не является репрезентативной для генеральной совокупности. Учитывая недостатки данных наблюдений при оценке причинно-следственных параметров, исследователи чаще используют экспериментальные и квази-экспериментальные данные. Раздел 3.3 рассматривает преимущество данных из социальных экспериментов. В Разделе 3.4 рассматриваются возможности моделирования с использованием особого типа данных наблюдения, получающихся в квази-экспериментальных условиях, когда естественным образом возникают индивиды получившие и неполучившие воздействие. Такие ситуации получили название естественного эксперимента. Раздел 3.5 охватывает практические вопросы использования микроданных.
	
\section{Данные наблюдений}

	
Основным источником микроэкономических данных наблюдения являются обследования домашних хозяйств, фирм и государственных административных учреждений. Данные переписи могут также использоваться для создания выборки. Также  источником данных могут являться маркетинговые опросы, интернет-аукционы и т.д. 

Существует огромная литература по выборочным обследованиям с точки зрения как статистиков, так и пользователей данных. В литературе для статистиков обсуждается как правильно получить выборку из генеральной совокупности, а также последствия использования различных планов выборки, а литература для пользователей занимается вопросами оценки параметров и статистических выводов, которые возникают, когда данные обследования собираются с использованием различных планов выборки. Ключевым вопросом является то, насколько репрезентативной является выборка. Эта глава фактически является введением в данную теорию с двух точек зрения. Дополнительные детали  приведены в главе 24.
	

\subsection{Природа данных обследования}


Термин <<данные наблюдений>> обычно относится к данным обследования, собранных путем выборки  индивидов из  генеральной совокупности без попыток контролировать характеристики выборки данных.
Обозначим через $t$ время, а через $w$  набор переменных, представляющих интерес. При этом  $t$ может быть моментом времени или периодом времени. Пусть $S_{t}$ обозначает выборку из генеральной совокупности с  функции распределения  $F(w_{t}|\theta_{t}$; $S_{t}$ --- это выборка из $F(w_{t}|\theta_{t}$, где $\theta$ --- вектор параметров. Генеральная совокупность должна рассматриваться как множество точек с интересующими нас характеристиками, а для простоты будем считать, что форма функции $F$ известна. Случайная выборка предполагает, что каждый элемент генеральной совокупности имеет равные шансы попасть в выборку. Более сложные схемы составления выборки будут рассмотрены позже.


Абстрактное понятие стационарности генеральной совокупности является очень полезным. Если моменты характеристик генеральной совокупности постоянны, то мы можем записать, что $\theta_{t}=\theta$, для всех $t$.  Это предположение является сильным, поскольку оно предполагает неизменность моментов характеристик генеральной совокупности во времени. Например, возрастное распределение по полу и возрасту должно быть постоянным. 
Более реалистичным было бы предположение, что некоторые характеристики генеральной совокупности могу меняться во времени. 
Чтобы допустить возможность изменений, можно предположить, что параметры каждой генеральной совокупности --- это случайная выборка из суперсовокупности (superpopulation) с постоянными характеристиками. В частности, каждое $\theta_{t}$ рассматривается как случайная выборка из  распределения  с постоянными гиперпараметрами $\theta$. 
Термины суперсовокупность и гиперпараметры часто встречаются в литературе по иерархическим моделям, которые обсуждаются в главе 24. 
Дополнительные трудности возникают, если в $\theta_t$ есть эволюционная составляющая, например, если параметры зависят от $t$, или если зависимы соседние значения.
Использование иерархических моделей, обсуждаемых в главах 13 и 26, --- это один из подходов к моделированию связи между гиперпараметрами и  характеристиками генеральной совокупности.


\subsection{Простая случайная выборка}

В качестве ориентира для последующего обсуждения, рассмотрим простую случайную выборку, в которой вероятность выбора объекта $i$ из генеральной совокупности большого размера $N$, составляет $1/N$ для любого объекта. Представим набор переменных $w$ в виде $[y:x]$. Предположим, наша цель заключается в моделировании $y$, вектора возможных исходов, обусловленного  экзогенными объясняющими переменными $x$, чье совместное распределение обозначается $f_{J}(y,x)$. Совместная плотность всегда может быть представлена в виде произведения  условного распределения $f_{C}(y|x,\theta)$ и частного распределения $f_{M}(x)$:
\begin{equation}
f_{J}(y,x)=f_{C}(y|x,\theta)f_{M}(x)
\end{equation}

Суть простой случайно выборки заключается в равномерном отборе $(Y,X)$ из всей совокупности.


\subsection{Многоэтапные опросы}

Одной из альтернатив является стратифицированная многоэтапная  кластерная выборка, также называемая сложной выборкой. Такой подход используется в крупномасштабных обследованиях, таких как Текущее обследование населения (Current population survey, CPS), Панельное исследование динамики доходов (Panel Study of Income Dynamics, PSID). В главе 24.2 мы приводим дополнительное описание структуры CPS.
	
	
Сложный план выборки имеет свои преимущества. Он является более эффективным с точки зрения затрат, поскольку сокращает географическую разбросанность, становится возможным сделать более интенсивной выборку из определенной подсовокупности.  Например, можно осуществить избыточную выборку из малых подсовокупностей, обладающих интересующими характеристиками, в то время как случайная выборка из генеральной совокупности будет давать слишком мало наблюдений для получения надежных результатов. Недостатки данного метода заключаются в том, что стратифицированная выборка уменьшает изменчивость характеристик между индивидами, что имеет важное значение для большей точности.


Большое количество литературы про выборочные обследования  посвящено многоэтапным опросам, которые последовательно разбивают генеральную совокупность на следующие категории:

\begin{enumerate}
\item Страта: Непересекающиеся подгруппы, которые исчерпывают всю генеральную совокупность.
\item Первичная единица выборки (ПЕВ, Primary sampling units, PSU): Непересекающиеся подмножества страты.
\item Вторичная единица выборки (ВЕВ, Secondary sampling units, SSU): часть ПЕВ, в свою очередь может быть разделена дальше.
\item Конечная единица выборки (КЕВ): Финальная единица, выбранная для опроса, может быть как домохозяйство, так и группа домохозяйств.
\end{enumerate}


В качестве примера, страта может быть штатом или провинцией в стране, ПЕВ может быть регионом в штате или провинции, а КЕВ может быть небольшой группой домохозяйств в том же районе.
	
	
Обычно опрашиваются все страты, так что, например, все штаты будут включены в выборку. Но не все их ПЕВ и ВЕВ включаются в выборку, также они могут выбираться с разной интенсивностью. В двухэтапной выборке обследуемые ПЕВ взяты случайным образом, а затем внутри отобранных ПЕВ случайным образом выбираются КЕВ. В многоэтапной выборке появляются промежуточные единицы выборки.


Следствием этих методов отбора является то, что разные домохозяйства будут иметь различные вероятности попадания в выборку. Поэтому выборка не является репрезентативной для генеральной совокупности. Многие обследования приводят веса, которые должны быть обратно пропорциональны вероятности попадания объекта в выборку, в этом случае эти веса могут быть использованы для получения несмещенной оценки характеристик генеральной совокупности.
	
	
Данные исследования могут быть кластеризованы в связи с выбором большого количества домохозяйств в том же самом районе. Наблюдения в одном кластере могут быть зависимы или коррелируемы, поскольку они могут зависеть от некоторых наблюдаемых или ненаблюдаемых факторов, которые могут влиять на все наблюдения в кластере. 
Например, в пригороде могут преобладать домохозяйства с высоким доходом или домохозяйства, которые являются относительно однородными по некоторым своим предпочтениям. Данные по этим  домохозяйствам скорее всего будут коррелированы, по крайней мере безусловно. Хотя вполне возможно, что корреляция будет пренебрежимо мала после учета наблюдаемых характеристик домохозяйств. 
Статистические выводы при  игнорировании корреляции между выборочными наблюдениями приводят к ошибочным оценкам дисперсии, которые меньше, чем в случае правильной формулы. Эти вопросы рассматриваются более подробно в разделе 24.5. Двухэтапная и многоэтапная выборка усложняют вычисление стандартных ошибок.
	
	
Таким образом, (1) стратификация с различными частотами внутри страт означает, что выборка не репрезентативная; (2) веса, обратно пропорциональны вероятности попадания в выборку, могут быть использованы для получения несмещенной оценки характеристик генеральной совокупности, и (3) кластеризация может привести к корреляции наблюдений и занижению истинной стандартной ошибки, если не внесена соответствующая поправка.
	
	
\subsection{Смещенные выборки}

Если выборка случайна, то закон распределения любой характеристики данных такой же, как закон  распределения в генеральной совокупности. Некоторые отклонения от случайной выборки вызывают расхождения между этими двумя распределениями, это и называют смещением выборки. Распределение данных отличается от распределения генеральной совокупности, и отличие  зависит от характера отклонения от случайной выборки. Отклонение от случайной выборки происходит потому, что иногда более удобно или экономически эффективно делать выборку из подсовокупности, даже если она не является репрезентативной. Рассмотрим теперь несколько примеров таких отклонений, начиная со случая, в котором нет никаких отклонений от случай выборки.

\begin{center}
Экзогенная выборка
\end{center}


Экзогенная выборка на основе данных обследований возникает тогда, когда аналитик разделяет имеющуюся выборку на подвыборки основанные лишь на  экзогенных переменных $x$, но не на зависимой переменной. Например, в исследовании госпитализаций в Германии Гейл и соавт. (1997) разделили данные на две категории: имеющие хронические заболевания и не имеющие. 
Также распространена классификация по категориям доходов. 
Более точно было бы называть этот тип выборки экзогенной подвыборкой, так как она осуществляется на базе уже сделанной выборки.
Сегментирование существующей выборки по полу, здоровью, социально-экономическому статусу также очень популярно. 
При экзогенной выборке предполагается, что распределение  экзогенных переменных не зависит от $y$ и не содержит информации об интересующих параметрах генеральной совокупности $\theta$. Таким образом, можно не учитывать частные распределения экзогенных переменных и строить оценивание на базе условного распределения $f(y|x,\theta)$. Конечно, предположение может быть ошибочным и наблюдаемое распределение объясняемой переменной может зависеть от конкретной сегментирующей перменной. 


\begin{center}
Выборка на основе ответов
\end{center}

Выборка на основе ответов имеет место, если вероятность включения отдельного индивида в выборку зависит от ответа,  сделанного данным индивидом. В этом случае самоотбор происходит по правилам, определенным с помощью изучаемой эндогенной переменной.


Можно привести три примера: (1) В исследовании влияния отрицательного подоходного налога или Помощи семьям с детьми-иждивенцами (Aid to Families with Dependent Children, AFDC) на предложение труда опрашивались только индивиды за чертой бедности. (2) В исследовании факторов  определяющих выбор общественного транспорта опрашивались только пользователи  общественного транспорта (субсовокупность). (3) В исследовании факторов влияющих на количество посещений места отдыха опрашивались только индивиды хотя бы с одним посещением.


Снижение затрат на исследования --- это важная причина использования выборки на основе ответов, а не  простой случайной выборки. Для того чтобы получить достаточное количество наблюдений (информации) по относительно редкому выбору,  необходима очень большая случайная выборка. Следовательно, дешевле собрать выборку из тех, кто сделал данный выбор.


Практический вывод состоит в том, что состоятельные оценки параметров генеральной совокупности $\theta$ больше не могут быть получены  с использованием только условной плотностью для генеральной совокупности $f(y|x)$. План выборки также должен быть принят во внимание. Эта тема обсуждается далее в разделе 24.4.


\begin{center}
Выборка основанная на длительности
\end{center}

Выборка основанная на длительности является примером того, как смещение может возникнуть в результате использования выборки из одной генеральной совокупности, чтобы сделать выводы о другой генеральной совокупности. Строго говоря, это не столько пример отклонения от простой случайной выборки, а скорее выборка  из <<неправильной>> генеральной совокупности.


Эконометрические  модели переходов моделируют время, проведенное в состоянии $j$ индивидом $i$ до перехода в другую состояние $s$. Например, при моделировании безработицы $j$ соответствует безработице, а $s$ занятости. Данные, используемые в таких исследованиях могут происходить из одного или нескольких возможных источников. 
Одним из возможных источников является выборка лиц, которые являются безработными на определенную дату, другим --- выборка лиц, находящихся в составе рабочей силы, независимо от их текущего состояния, а третьим --- выборка лиц, которые либо входят, либо выходят из стадии безработицы в течение определенного периода времени. Каждой схеме выборки соответствует своя генеральная совокупность. 
В первом случае генеральной совокупностью будут все безработные, во втором --- рабочая сили, а в третьем --- индивиды, меняющие статус занятости. Эта тема обсуждается далее в разделе 18.6.


Предположим, что целью исследования является вычислить показатель средней продолжительности безработицы. Речь идет о средней продолжительности безработицы, с которой столкнется случайно выбранный индивид, если когда-либо станет безработным. Ответ на этот казалось бы простой вопрос  может меняться в зависимости от того, каким способом получены данные. Распределение  завершенной длительности безработицы довольно сильно отличается, если измерять её у тех, кто находится в состоянии безработицы (запас безработных индивидов) и у тех, кто покидает состояние безработицы (исходящий поток). Когда мы строим выборку из запаса, вероятность нахождения в выборке выше у лиц с более длительной продолжительностью. Когда мы выбираем из потока, вероятность не зависит от времени, проведенного в данным состоянии. Это хорошо известный пример выборки на основе длительности, в которой оценка, полученная на основе выборки запасов является смещенной оценкой средней продолжительности безработицы.

Следующая простая схема может прояснить идею:

\vspace{3cm}

\[ \bullet \: \circ \: \longmapsto \: \begin{aligned} \bullet &\: \bullet \\ \bullet &\: \circ \end{aligned}  \: \longmapsto \: \circ  \: \bullet \]

Entry flow --- Входящий поток

Stock --- Запас

Exit flow --- Исходящий поток

Здесь мы используем символ $\bullet$ для обозначения тех, кто медленно переходит из одного состояния в другое, и символ $\circ$ для обозначения тех, кто быстро переходит. Предположим, что два типа в равной степени представлены во входящем потоке, но $\bullet$ остаётся в запасе дольше, чем $\circ$. Тогда среди индивидов в запасе более высока доля тех, кто переходит медленно. Наконец, в исходящем потоке из состояния снова будет равная доля тех, кто быстро и медленно переходит. Этот аргумент обобщается и на другие виды неоднородности.


Суть этого примера не в том, что выборка из потока  лучше, чем выборка из запаса. Скорее суть в том, что в зависимости от вопроса, выборка из запаса  может не быть случайной выборкой из нужной генеральной совокупности.



\subsection{Смещение самоотбора}

Рассмотрим следующую задачу. Исследователь заинтересован в измерении эффекта обучения (воздействия), обозначаемого $z$, на заработную плату, обозначаемую $y$, при заданных характеристиках работника, обозначаемых $x$. Переменная $z$ принимает значение 1, если работник прошел обучение и 0 в противном случае. Наблюдения по $(x, D)$ имеются для всех работников, а по $y$ только для тех, кто прошел подготовку $(D = 1)$. Необходимо оценить  среднее влияние обучения на заработную плату случайно выбранного работник с известными характеристиками, который в настоящее время не обучался $(D = 0)$. Проблема самоотбора выборки связана с  трудностью  нахождения данной оценки.


Мански (1995), который рассматривает это как проблему идентификации, определяет проблему самоотбора формально следующим образом:


Это проблема идентификации условного распределения вероятностей по случайной выборке, в которой  значения объясняющих переменных всегда наблюдаемы, а значения зависимой переменной подвергаются цензурированию.


Пусть $y$ --- объясняемая переменная, а объясняющие переменные обозначаются через $x$. Переменная $z$ является цензурирующей переменной, то есть принимает значение 1, если значение $y$ наблюдается и 0 в противном случае. Переменные $(D, x)$ всегда наблюдается, а $y$ наблюдается только при $D = 1$, поэтому Мански называет эту ситуацию цензурированной выборкой. Этот процесс получения выборки  не позволяет идентифицировать $\Pr[y|x]$, как видно из равенства

\begin{equation}
\Pr[y|x]=\Pr[y|x,D=1]\Pr[D=1|x]+\Pr[y|x,D=0]\Pr[D=0|x].
\end{equation}

Процесс построения выборки позволяет идентифицировать три из четырех слагаемых в правой части, но не дает никакой информации о  
$\Pr[y|x, D = 0]$. Потому что

\[
\E[y|x]=\E[y|x,D=1]\Pr[D=1|x]+\E[y|x,D=0]\Pr[D=0|x],
\]
всякий раз, когда вероятность цензурирования $\Pr[D=0|x]$ является положительной, имеющиеся эмпирические данные не накладывают никаких ограничений на $\E[y|x]$. Следовательно, цензурированный процесс построения выборки позволяет идентифицировать $\Pr[y|x]$ с точностью до неизвестного значения $\Pr[y|x,D=0]$. Чтобы узнать что-нибудь о $\E[y|x]$, ограничения должны быть наложены на $\Pr[y|x]$.


Альтернативные подходы к решению этой проблемы обсуждаются в разделе 16.5.


\subsection{Качество данных опросов}

Качество данных зависит не только от плана выборки и способа опроса, но и от ответов индивидов. Этот факт особенно относится к данным наблюдений. Мы рассмотрим несколько ситуаций, в которых качество выборочных данных может быть существенно ухудшено. Некоторые из этих проблем (например, истощение) также могут возникнуть при использовании других типов данных. Эта тема пересекается со смещением выборки.


\begin{center}
Проблема с отсутствием ответа
\end{center}


Опросы, как правило, добровольны, и стимул отвечать может варьироваться в зависимости от  характеристик домохозяйства и типа поставленного вопроса. Индивиды могут отказаться отвечать на некоторые вопросы. Если есть связь между  отказом отвечать на вопрос и характеристиками индивида, то возникает вопрос о репрезентативности опроса. Если неполучение ответов игнорируется, а  анализ проводится только с использованием данных от респондентов, как это будет влиять на оценку интересующих исследователя параметров?


Неполучения ответов является частным случаем проблемы самоотбора, упомянутой в предыдущем разделе. Оба случая связаны со смещением выборки. Чтобы проиллюстрировать, как это приводит к искажению статистических выводов рассмотрим следующую модель:

\begin{equation}
\begin{bmatrix} y_{1} \\ y_{2} \end{bmatrix} \Biggm| x,z \sim N \Biggm( \begin{bmatrix} x'\beta \\ z'\gamma \end{bmatrix}, \begin{bmatrix} \sigma^2_{1} & \sigma_{12} \\ \sigma_{12} &  \sigma^2_{2}\end{bmatrix}\Biggm),
\end{equation}
где $y_{1}$ является непрерывной случайной величиной (например, расходы), которые зависят от $x$, а $y_{2}$ является скрытой переменной, которая измеряет <<склонность к участию>> в опросе и зависит от $z$. Домохозяйство участвует в опросе, если $y_{2}>0$, а в противном случае не участвует. Переменные $x$ и $z$ предполагаются экзогенными. Данная спецификация допускает, что $y_{1}$ и $y_{2}$  коррелированы. 


Предположим, мы оцениваем $\beta$ по данным, предоставленным участниками опроса, методом наименьших квадратов. Является ли эта оценка несмещенной в случае, когда есть отказавшиеся участвовать? Ответ в том, что если неучастие является случайным и независимым от $y_{1}$, то нет никакой смещённости, а иначе будет смещение появляется:

Действительно:

\[
\begin{aligned}
&{\beta}=[X'X]^{-1}X'y_{1}, \\
\E[\hat{\beta}-\beta]&=E\Bigm[ [X'X]^{-1}X'\E[y_{1}-X\beta|X,Z,y_{2}>0] \Bigm],
\end{aligned}
\]

где первая строка содержит формулу для оценки $\beta$, а вторая ---  смещение. Если $y_{1}$ и $y_{2}$ независимы при фиксированных $X$ и $Z$, $\sigma_{12}=0$, то

\[
\E[y_{1}-X\beta|X,Z,y_{2}>0]=\E[y_{1}-X\beta|X,Z]=0,
\]
и здесь нет смещения.

\begin{center}
Пропущенные данные и ошибки измерения
\end{center}
%%%%
Респонденты, заполняющие обширную анкету, не обязательно отвечают на все вопросы, и даже если они отвечают, ответы могут быть умышленно или случайно ложными. Предположим, что выборочное обследование пытается получить вектор ответов обозначенный как $x_{i}=(x_{i1},\dots,x_{iK})$ от $N$ участников, $i= 1,\dots,N$. Предположим теперь, что если человек не в состоянии предоставить информацию об одном или более элементов $x_{i}$, то весь вектор отбрасывается. Первая проблема, в результате отсутствия данных является то, что размер выборки уменьшается. Второй потенциально более серьезной проблемой является то, что недостающие данные могут привести к смещение как и в случае неправильной выборки. Если отсутствуют данные на систематической основе, то оставшаяся выборка не может быть репрезентативной. Форма смещения отбора может быть вызвана любым систематическим характером пропущенных данных. Например, респонденты с высоким доходом систематически не отвечают на вопросы о доходах. Глава 27 обсуждает проблему пропущенных данных и её решение.


Измерение ошибки в ответах являются широко распространенной проблемой. Они могут возникнуть в результате различных причин, в том числе неправильные ответы, связанные с небрежностью, намеренный неправильный ответ, ошибочная интерпретация данных. Более важным источником погрешности измерений является несовершенное прокси для соответствующей теоретической концепции. Последствия таких ошибок измерения является основной темой главы 26.


\begin{center}
Sample Attrition
\end{center}

В случае панельных данных обследование включает в себя повторные наблюдения на множество индивидов. В этом случае мы можем имеем:

\begin{itemize}
\item полные ответы во все периоды (полное участие),
\item нет ответов в первом периоде и во всех последующих периодах (полное неучастие), или
\item частичный ответ в смысле ответа на начальных периодов, но отсутствие ответа в более поздние периоды (неполное участие) - ситуация Sample Attrition
\end{itemize}


Sample Attrition приводит к недостающим данным, наличие любой формы неслучайной <<missingness>> приведет к проблеме выборки, которая уже упоминалась. Это можно интерпретировать как особый случай задачи отбора данных. Пример этого кратко обсуждается в разделах 21.8.5 и 23.5.2.

\subsection{Типы данных наблюдения}

Кросс-секционные данные получены при наблюдении $w$, для выборки $S_{t}$ для некоторых $t$. Хотя это обычно нецелесообразно собирать все домохозяйства в одну выборку в один момент времени, кросс-секционные данные предоставляют характеристики каждого элемента подмножества генеральной совокупности, которая будет использоваться, чтобы сделать вывод о генеральной совокупности. Если генеральная совокупность стационарная, то выводы сделанные о $\theta_{t}$ с использованием $S_{t}$ могут быть справедливыми и для $t'\neq t$. Если существует значительная зависимость между прошлым и текущим поведением, то лонгитюдные данные, необходимые для идентификации отношений, представляющих интерес. Например, ранее принятые решения могут повлиять на текущие результаты; инерции или привычка могут стимулировать покупки, но такая зависимость не может быть смоделирована, если история покупок недоступна. Это одно из ограничений, налагаемых кросс-секционными данными.


Повторные кросс-секционные данные получены последовательностью независимых выборок $S_{t}$, взятых из $F(w_{t},\theta_{t}), t=1,\dots,T$. Поскольку выборка не пытается сохранить ту же единицу в себе, то сведения о динамических зависимостях теряется. Если генеральная совокупность стационарная, тогда повторные кросс-секционные данные получаются путем обработки выборок похожей на схему с возвращением от постоянной генеральной совокупности. Если же она нестационарная, повторные кросс-секции связаны таким образом, что зависят от того, как население меняется с течением времени. В таком случае целью является, сделать выводы о лежащих в основе постоянных (гипер)параметров. Анализ повторяющихся данных обсуждается в разделе 22.7.


Панель или лонгитюдные данные получают сначала выбрав образец $S$, а затем собрав наблюдения для последовательности периодов времени, $t=1,\dots,T$. Это может быть достигнуто путем опроса и сбора как нынешних, так и прошлых данных в то же время, или путем отслеживания предметов, когда они были введены в опросе. Это создает последовательность данных векторов ${w_{1},\dots,w_{T}}$, которые используются, чтобы сделать вывод о любом поведении населения или конкретной выборки лиц. Соответствующие методологии в каждом случае могут быть не одинаковыми. Если данные взяты из нестационарного населения, соответствующей целью должен быть вывод о (гипер)параметрах суперпопуляции.


Некоторые из недостатков этих типов данных очевидны. Кросс-секционные и повторные кросс-секционные данные не обеспечивают в целом подходящие данные для моделирования межвременной зависимости результатов. Такие данные пригодны только для моделирования статических отношений. В отличие от лонгитюдных данных, особенно если они охватывают достаточно длительный период времени, являются подходящими для моделирования статических и динамических отношений.


Лонгитюдные данные также имеют проблемы. Первая проблема в  не репрезентативности панели. Проблемы в выводах о генеральной совокупности на основе лонгитюдных данных становятся всё более сложными, если совокупность не является стационарной. Для анализа динамики поведения, сохраняя оригинальные домашние хозяйства в панели как можно дольше является привлекательным вариантом. На практике лонгитюдные наборы данных страдают от проблемы <<sample attrition>>, возможно, из-за <<sample fatigue>>. Это просто означает, что респонденты не продолжают предоставлять ответы на вопросы. Это создает две проблемы: (1) панели становятся несбалансированными и (2) существует опасность того, что домохозяйство не может быть <<типичной>> и, что образец становится не репрезентативным для генеральной совокупности. Когда имеющиеся данные выборки не являются случайной выборкой из совокупности, результаты, основанные на данных различных типов будут восприимчивы к смещению в разной степени. Проблема <<sample fatigue>> возникает потому, что с течением времени становится все более трудно сохранить индивидов внутри панели или они могут быть <<потеряны>> по некоторым другим причинам, таким как изменение местоположения. Эти вопросы рассматриваются далее в книге. Анализ лонгитюдных данных тем не менее может предоставить информацию о некоторых аспектах поведения единиц выборки, хотя экстраполяция поведения генеральной совокупности не может быть простой.


\section{Данные социальных экспериментов}

Наблюдаемые и экспериментальные данные различны, так как экспериментальная среда в принципе может быть тщательно контролируема и управляема. В отличие от этого, наблюдаемые данные создаются в неконтролируемой среде, оставляя открытой возможность наличие сопутствующих факторов, которые усложнять определение причинно-следственных связей. Например, при попытке изучения заработка  от образованности по данным наблюдений, надо признать, что количество лет обучения отдельного индивида является результатом принятия решения, и, следовательно, никто не может рассматривать уровень школьного образования, если в экспериментальных условиях.


В социальных науках, данные аналогичные экспериментальным данным получают либо от социальных экспериментов, определены и описаны более подробно ниже, или в  <<лабораторных>> экспериментах на небольших группах добровольцев , которые имитируют поведение экономических агентов в реальной жизни. Социальные эксперименты относительно редкое явление, и все же экспериментальные концепции, методы и данные служат основой для оценки эконометрических исследований, основанных на данных наблюдений.


В этом разделе представлен краткий обзор методологии социальных экспериментов, характер данных, вытекающих из них, и некоторые проблемы и вопросы эконометрической методологии, которые они производят.


Главной особенностью экспериментальной методологии является сравнение между результатами случайно выбранной экспериментальной группы, которая подвергается <<воздействию>> с теми, кто в контрольной группе. В хорошем эксперименте уделяется много времени для сравнения контрольных и экспериментальных групп, чтобы избежать возможных смещений в результатах. Такие условия не могут быть реализованы в наблюдательной среды, что ведет к возможному отсутствию идентификации причинных параметров, представляющих интерес. Иногда, экспериментальные условия приближенно можно повторить в данных наблюдений. Рассмотрим, например, два соседних региона или государства, в которых разная политика установления минимальной зарплаты, создавая условия естественного эксперимента, в котором наблюдения в состоянии <<воздействия>> можно сравнить с теми, которые в состоянии <<контроля>>. Структура данных естественного эксперимента также привлекает внимание эконометристов.


Социальный эксперимент включает экзогенные изменения в экономической среде перед набором испытуемых, которая разбита на одно подмножество, которое получает экспериментальное воздействие и другое, которое служит в качестве контрольной группы. В отличие от наблюдательных исследований, в которых изменения в экзогенных и эндогенных факторов часто смятение, хорошо продуманный социальный эксперимент направлен на выделение воздействованных переменных. В некоторых экспериментальных конструкциях может не быть никакой контрольной группы, но применяются разные уровни воздействия, в этом случае становится возможным в принципе оценить всю поверхность экспериментальных результатов.


Основной задачей социального эксперимента является оценка влияния фактических или потенциальных социальных программ. Потенциальная модель результата раздел 2.7 обеспечивает соответствующую подготовку для моделирования влияния социальных экспериментов. Несколько альтернативных мер воздействия были предложены и будут обсуждаться в главе, посвященной оценке этих программ (глава 25).


Burtless (1995) обобщает случай социальных экспериментов, отмечая при этом некоторые потенциальные ограничения. В сопутствующей статье Хекмана и Смит (1995) сосредотачиваются на ограниченности реальных социальных экспериментов, которые были реализованы. 


\subsection{Характеристики социального эксперимента}


Социальные эксперименты мотивированы политическим вопросам о том, как предметы будут реагировать на тип политики, который никогда не был опробован и, следовательно, для которого нет данных наблюдений. Идея социального эксперимента заключается в привлечении группы участников, некоторые из которых случайным образом распределены в группу воздействия, а остальные в контрольную группу. Разница между ответами тех, кто в группе воздействия, и тех, кто в контрольной группе, представляет собой оценку влияния политики. Схема стандартной экспериментальной конструкции,  показано на рисунке 3.1.


Термин <<экспериментальные данные>> относится к группе, получавшей воздействие, <<контрольный>> к группе не получавшее воздействие, а <<рандомизация>> к процессу деления лиц на две группы.


Рандомизированные исследования были введены в статистике Р. Фишером (1928) и его сотрудниками. Типичным сельскохозяйственным экспериментом является испытание, в котором новые методы воздействия, такие как удобрения будут применяться для растений, растущих на случайно-выбранных частях земли, а затем результаты будут сравниваться с теми растениями, которые входили в контрольную группу. Если эффект всех других различий между экспериментальной и контрольной группой может быть устранен, по оценкам, разница между этими двумя наборами результатов может быть отнесены к воздействию. В простейшей ситуации можно сконцентрировать внимание на сравнение средних результатов, воздействованной и контрольной группы.


Хотя в сельском хозяйстве и биомедицинских науках, методологии рандомизированных экспериментов давно установлены, в области экономики и социальных наук это явление новое. Более того, оно привлекательно для изучения результатов политические изменения, для которых нет наблюдательных данных. Рандомизированные эксперименты также приводят к более сильному изменению в политике переменных и параметров, чем их наличие в данных наблюдений, тем самым облегчая для выявления и изучения ответов на политические изменения.


Социальные эксперименты все еще довольно редки за пределами Соединенных Штатов, отчасти потому, что они дорогие. В США число таких экспериментов росло с начала 1970-х. Таблица 3.1 суммирует особенности некоторых относительно известных примеров; для более широкого охвата см. Burtless (1995).


Эксперимент может производиться либо на кросс-секционных или на лонгитюдных данных, из соображения стоимости, не используются временные ряды. В случае, когда эксперимент длится несколько лет и имеет несколько этапов и / или географических мест, промежуточный анализ на основе <<неполны>> встречается очень часто (Ньюхаусом соавт., 1993).



\subsection{Плюсы социального эксперимента}


Burtless (1995) исследовал преимуществ социальных экспериментов с большой ясностью. Главное преимущество проистекает из рандомизированных исследований, которые устраняют любые корреляции между наблюдаемыми и не наблюдаемыми характеристиками участников эксперимента.

\begin{table}[h]
\begin{center}
\caption{\label{tab:pred}Черты некоторых экспериментов}
\begin{tabular}[t]{llcll|}
\hline
\bf{Эксперимент} & \bf{Тестирумые переменные} & \bf{Целевая аудитория} \\
\hline
Rand Health Insurance Experiment (RHIE), 1974–1982 & Планы медицинского страхования с различной ставкой и с разным уровнем максимальных расходы за свой счет  & Индивиды и домохозяйства со средним и низким уровнем дохода \\
Negative Income Tax (NIT), 1968–1978 & Планы NIT с альтернативными гарантиями дохода и налоговой ставкой & Индивиды и домохозяйства со средним и низким уровнем дохода \\
Job Training Partnership Act (JTPA), (1986–1994)& Ассистенты по поиску работы, тренинги, финансируемый JTPA & Абитуриенты и безработные люди\\
\hline
\end{tabular}
\end{center}
\end{table}

Следовательно, вклад воздействия по итогам разницы между обработанной и контрольной группой может быть оценена без вмешивающихся факторов, даже если человек не может контролировать вмешивающиеся переменные. Наличие корреляции между воздействованными и вмешивающимися переменными часто страдает от наблюдательных исследований и усложняет выводы причинно-следственных результатов. С другой стороны, экспериментальные исследования, проведенные в идеальных условиях могут производить последовательную оценку средней разницы в результатах обработанной и необработанной группы без особой вычислительной сложности.


Если, однако, результат зависит от воздействия также, как и от других наблюдаемых факторов, то контролирование будет улучшать точность влияния оценок.


Даже если данные наблюдений доступны, создание и использование экспериментальных данных имеет большую привлекательность, потому что это дает возможность экзогеннизации политической переменной, а рандомизация воздействованных переменных может привести к большому упрощению статистического анализа. Выводы основанные на данных наблюдений часто не хватает общности, поскольку они основаны на неслучайной выборки из генеральной совокупности --- проблема смещения отбора. Примером может служить вышеупомянутое исследование RHIE, в котором основной акцент делается на чувствительность к ценам спроса на медицинские услуги. Наличие медицинской страховки влияет на цену медицинских услуг и тем самым на её использование. Можно, конечно, использовать данные наблюдений для моделирования отношения между спросом на медицинские услуги и уровнем страхования. Тем не менее, такой анализ подлежит критике, что уровень медицинского страхования не должен рассматриваться как экзогенный. Теоретический анализ показывает, что спрос на страхование здоровья и здравоохранения определяется совместно, поэтому причинно-следственная связь разно направлена. Этот факт может потенциально сделать затруднительным определение роли медицинского страхования. Медицинское страхование как экзогенная переменная смещает оценку чувствительности к ценам. Однако в экспериментальной установке участвующие домохозяйства могут иметь страховой полис, что делает его экзогенной переменной. Как только ключевые переменные становятся экзогенными, направление причинно-следственной становится очевидными, и влияние воздействия может быть изучено однозначно. Кроме того, если эксперимент не содержит некоторые из проблем, о которых мы говорим далее, это значительно упрощает статистический анализ относительно того, что часто возникает необходимость в данных обследования.




\subsection{Ограничения социального эксперимента}

Применение негуманной методологии, вызывает оживленную дискуссию в литературе. См. особенно Хекмана и Смит (1995), которые утверждают, что многие социальные эксперименты могут страдать от ограничений, которые применяются к наблюдательных исследований. Эти вопросы касаются общих моментов, таких как достоинства экспериментальных наблюдений по сравнению с наблюдательной методологией, а также конкретные вопросы, касающиеся предубеждения и проблем, связанных с использованием человека в экспериментах. Некоторые вопросы рассматриваются более подробно в последующих главах.


Социальные эксперименты являются очень дорогостоящими для запуска. Иногда, они не соответствуют <<чистым>> случайным  исследованиям. Таким образом, результаты таких экспериментов не всегда однозначны и легко интерпретируемы или свободны от предубеждений. Если воздействуемая переменная имеет много альтернативных интересов, или если экстраполяция является важной задачей, то должна быть собрана очень большая выборка, чтобы обеспечить достаточное изменение данных и точно измерить эффект воздействия вариации. В этом случае стоимость эксперимента также будет увеличиваться. Если фактор стоимости не позволяет провести достаточно большой эксперимент, его полезность по сравнению с наблюдательным исследованием может быть сомнительной; см. работы Розена и Стаффорд в Хаусман и Вайса (1985).


К сожалению, проектирование некоторых социальных экспериментов является неправильным. Хаусман и Вайс (1985) утверждают, что данные из Нью-Джерси эксперимент отрицательного подоходного налога был подвержены эндогенным изменениям, которые они описывают следующим образом:

$\dots$ Причиной эксперимента, по рандомизации, является устранение корреляции между переменными воздействия и другими детерминантами реакции переменной, которая находится в стадии изучения. В каждой эксперименте, в котором исследуется доход, выборка производится отчасти на основе зависимой переменной. В общем, группы, имеющие право на выбор - на основе семейного положения, расы, возраста главы семьи, и т.д. - была стратифицирована на основе дохода (и других переменных), а индивиды, были отобраны из этих слоев. (Хаусман и Вайс, 1985, стр. 190-191)


Авторы приходят к выводу, что, в присутствии эндогенного стратификации, объективной оценки результатов лечения не совсем просто получить. К сожалению, полностью рандомизированное исследование, в котором назначение лечения в случайно выбранный из экспериментальной группы населения не зависит от дохода будет гораздо более дорогостоящим и может оказаться невозможным.


Есть несколько других вопросов, которые вытекают тз идеальной простоты рандомизированных экспериментов. Во-первых, если экспериментальные участки выбираются случайно, сотрудничество администраторов и потенциальных участников на этом этапе не потребуется. Если этого не последует, то альтернативные места воздействия, где такое сотрудничество может быть получено будут заменены, тем самым ставя под угрозу принцип случайного назначения, см. Хотц (1992).


Второй проблемой является проблема отбора, потому что участие в эксперименте является добровольным. По этическим причинам есть много экспериментов, которые просто не могут быть совершены (например, случайное распределение студентов в годы обучения). В отличие от медицинских экспериментов, которые могут использовать официальные проток, в социальных экспериментов экспериментуемые знают, являются ли они группой воздействия или контрольной группой. Если решение об участии не коррелирует с $x$ или $\epsilon$, анализ экспериментальных данных упрощается.


Третья проблема заключается sample attrition вызванное тем, что испытыемый выбывает из эксперимента после ее начала. Даже если первоначальная выборка была случайной на истощение неслучайных вполне может привести к проблеме похожей на истощение смещения в панелях. Наконец, существует проблема эффект Hawthorne. Термин происходит в социальной психологии исследования, проведенного совместно Гарвардской высшей школы делового администрирования и управления Западной Электрической компании в Хоторне; Hawthorne, который работал в Чикаго с 1926 по 1932 год. Человек, в отличие от неодушевленных предметов, может изменять или адаптировать своё поведение во время участия в эксперименте. В этом случае изменение ответа, наблюдаемого в экспериментальных условиях не может быть отнесено исключительно к воздействию.


Хекман и Смит (1995) упоминают ряд других трудностей в осуществлении рандомизированного воздействия. Потому, что администрация социального эксперимента включает в себя бюрократию, есть возможность для предубеждений. Рандомизированное смещение происходит, если задание представляет систематическое различие между экспериментальными участника и участника в процессе ее эксплуатации. Хекман и Смит документировали возможности такого смещения в реальных экспериментах. Другой тип смещения, называется погрешностью замещения, когда вводимое управления может получать некоторые формы лечения, которое заменяет экспериментальное воздействие. Наконец, анализ социальных экспериментов неизбежно несёт в себе характер частичного равновесия. 


В частности, ключевым вопросом является возможность экстраполяции результатов эксперимента, на всю генеральную совокупность. Если эксперимент проводится в качестве экспериментальной программы в небольших масштабах, но есть намерение предсказать влияние политики, более широкое применение, то очевидным ограничением является то, что пилотная программа не может включать более широкого воздействия на переменные. Широко применимое воздействие может изменить экономическую среду, достаточно признать недействительным прогнозы от частичного равновесия. 


Таким образом, социальные эксперименты, в принципе, являются данными, которые проще анализировать и понимать с точки зрения причинно-следственных связей, чем данные наблюдений. Однако это зависит от дизайна эксперимента. Плохо сконструированный эксперимент создаёт свои статистические сложности, которые влияют на точность выводов. Социальные эксперименты принципиально отличаются от тех, которые в биологии и сельском хозяйстве, потому что человеку как правило, активный и и вперед смотрящий агент с личными предпочтениями, что усложняет процесс эксперимента.



\section{Данные естественного эксперимента}


Иногда, исследователь может иметь в наличии данные <<естественного эксперимента>>. Данный эксперимент происходит, когда подмножество генеральной совокупности подвергается экзогенным изменения в переменной, возможно, в результате изменения политики, которое обычно является эндогенным изменением. В идеале, источник вариации известен.


В микроэконометрики широко используются два способа применения идеи естественного эксперимента. Для конкретности рассмотрим простую регрессионную модель:

\begin{equation}
y=\beta_{1}+\beta_{2}x+u,
\end{equation}

где $x$ эндогенный объясняющие переменные коррелируемые с $u$.


Предположим, что существует экзогенные последствия, которые измененяют $x$. Примерами такого внешнего вмешательства могут быть административные правила, непредвиденное законодательство, природные явления (см. таблицу 3.2). Экзогенные вмешательство создает возможность для оценки ее воздействия путем сравнения поведения влияние группы как до, так и после вмешательства. То есть, <<естественное>> сравнение генерируется по событию, которое облегчает оценку $\beta_{2}$. Оценка упрощается, так как $x$ можно рассматривать как экзогенный.


Второй способ, в котором естественный эксперимент может помочь с выводом является создания естественных инструментальных переменных. Пусть $z$ это переменная, которая коррелирует с $x$, или, возможно, причинно связана с $x$, и не коррелирует с $u$. Тогда инструментальные переменные оценки $\beta_{2}$ выражаются через ковариацию

\begin{equation}
\hat{\beta_{2}}=\frac{Cov[z,y]}{Cov[z,x]}
\end{equation}
(см. раздел 4.8.5). В наблюдаемых данных сложно найти инструментальные переменные, но они легко возникают в случае естественного эксперимента. Мы рассмотрим первый случай в следующем разделе; тема естественной генерации инструментов будет рассмотрена в главе 25.

\subsection{Естественное экзогенное воздействие}


Такие данные являются менее дорогими для сбора и они также позволяют исследователю оценить роль некоторых специфических факторов, как в контролируемом эксперименте, потому что <<природа>> содержит постоянные изменения, связанные с другими факторами, которые не имеют непосредственного интереса. Такие естественные эксперименты привлекательны тем, что они создают группу воздействия и контрольную группу без каких-либо издержек и в реальных условиях. Способность естественного эксперимента поддерживать устойчивые выводы зависит, в частности, от того, предполагается ли экзогенное вмешательство, или его влияние достаточно велико, чтобы быть измеримыми.


Исследования основанные на естественных экспериментах имеют несколько потенциальных ограничений, важность которых в той или иной степени можно оценить только путем тщательного рассмотрения соответствующей теории, фактов и институциональных установок. Следуя Кэмпбеллу (1969) и Мейеру (1995), эти ограничения делатся на группы, влияющие на внутреннюю валидность исследования (т.е. выводы о политике воздействия взяты из исследования), и  на те, которые влияют на внешнюю валидность исследования (т.е. обобщение выводов на других членов генеральной совокупности).


Рассмотрим исследование изменения в политике, в котором делаются выводы из сравнения до и после вмешательств с использованием метода регрессии кратко описанной ниже и более подробно в главе 25. В любом исследовании будут опущены переменные, которые могут также изменяться в интервале времени между изменением политики и ее воздействием. Характеристики выборки лиц, таких как возраст, состояние здоровья, и их фактических или ожидаемых экономических условий также может меняться. Эти опущенные факторы будут непосредственно влиять на измеренное воздействие изменения политики. Можно ли обобщить  результаты на других членов совокупности будет зависеть от отсутствие предвзятости из-за неслучайной выборки, наличия существенных эффектов взаимодействия между изменением политики и ее установками, отсутствием исторических факторов, которые могут также на результат.


\subsection{Разница в разнице}


Одним из простых регрессионных методов является сравнение результатов в одной группе до и после вмешательства. К примеру, рассмотрим

\[
y_{it}=\alpha+\beta D_{t} +\epsilon_{it}, i=1, \dots, N, t=0,1,
\]

где $D_{t}=1$ в первом периоде (после вмешательства), $D_{t}=0$ в периоде 0 (до вмешательства), и $y_{it}$ измеряет результат. Регрессии оценивается по обобщенным данным, и даст оценку $\beta$. Легко показать, что она будет равна  средней разнице результата до и после вмешательства,


\[
\hat{\beta}=N^{-1}\sum_{i}(y_{i1}-y_{i0})=\bar{y_{1}}-\bar{y_{0}}.
\]

Сильным предположением является сопоставимость группы с течением времени. Это необходимо для идентифицируемости $\beta$. Если, например, мы допускаем изменение $\alpha$ между этими двумя периодами, $\beta$ больше не может быть идентифицирована. Изменения в $\alpha$ смешиваются с политикой воздействия.

Одним из способов улучшения предыдущей модели является включение дополнительной необработанной группы сравнения, то есть ту, которая не влияет на политику, и для которых имеются данные в обоих периодах. Используя обозначение Майера (1995), соответствующая регрессия выглядит так


\[
y^{j}_{it}=\alpha+\alpha_{1}D_{t}+\alpha^{1}D^{j}+\beta D^{j}_{t} +\epsilon^{j}_{it}, i=1, \dots, N, t=0,1,
\]
где $j$ является группой индексов, $D^{j}=1$, если $j=1$ и $D^{j}=0$, если $j=0$.  $D^{j}_{t}=1$, если $j$ и $t$ равняется 1 и$D^{j}_{t}=0$ в противном случае, а $\epsilon$ случайное возмущение с нулевым математическим ожиданием и постоянной дисперсией. Уравнение не включает ковариацию, но она может быть добавлена, и то, что не изменяется можно отнести к $\alpha$. Это означает, что для группы воздействия, модель будет выглядеть до вмешательства

\[
y^{1}_{i0}=\alpha+\alpha^{1}D^{1}+\epsilon^{1}_{i0},
\]

и после вмешательства

\[
y^{1}_{i1}=\alpha+\alpha_{1}+\alpha^{1}D^{1}+\beta +\epsilon^{1}_{i1}.
\]

Результат, поэтому такой

\begin{equation}
y^{1}_{i1}-y^{1}_{i0}=\alpha_{1}+\beta+\epsilon^{1}_{i1}-\epsilon^{1}_{i0}.
\end{equation}

Соответствующее уравнение для контрольной группы такого

\[
y^{0}_{i0}=\alpha+\epsilon^{0}_{i0}, \qquad
y^{0}_{i1}=\alpha+\alpha_{1}+\epsilon^{0}_{i1},
\]
а разница
\begin{equation}
y^{1}_{i1}-y^{1}_{i0}=\alpha_{1}+\epsilon^{0}_{i1}-\epsilon^{1}_{i0}.
\end{equation}

Эти уравнения первой разности включают первый период влияние $\alpha_{1}$, которое может быть устранено путём вычитания из уравнения (3.6) и (3.7):

\begin{equation}
(y^{1}_{i1}-y^{1}_{i0})-(y^{1}_{i1}-y^{1}_{i0})=\beta+(\epsilon^{1}_{i1}-\epsilon^{1}_{i0})-(\epsilon^{0}_{i1}-\epsilon^{1}_{i0}).
\end{equation}
Предполагая что $\E[(\epsilon^{1}_{i1}-\epsilon^{1}_{i0})-(\epsilon^{0}_{i1}-\epsilon^{1}_{i0})]=0$, мы можем получить несмещённую оценку $\beta$ как среднее $(y^{1}_{i1}-y^{1}_{i0})-(y^{1}_{i1}-y^{1}_{i0})$. Этот метод использует разницу в разнице. Если присутствуют меняющиеся во времени регрессоры, то они могут быть включены в соответствующие уравнения и их различия появятся в уравнение регрессии (3.8).


Для простоты, наш анализ игнорирует возможность того, что остаются наблюдаемые различия в распределении характеристик между группами воздействия и контроля. Если так, то такие различия, необходимо контролировать. Стандартным решением является включение таких контрольных переменных в регрессию.


Примером исследования, основанного на естественном эксперименте является то, что исследовали Эшенфельтер и Крюгер (1994). Они оценивали влияния образования на уровень заработной платы у идентичных близнецов с разным уровнем образования. В этом случае обычный эксперимент, в котором людям заданы различные уровни образования просто не возможен. Тем не менее, некоторые экспериментальные типы контроля необходимы. Как объясняют авторы:


Наша цель заключается в обеспечении,того что корреляция наблюдаемая между образованием и ставок заработной платы не появляется из-за корреляции между образованием и способностью работника или другим признакам. Мы делаем это, пользуясь тем, что однояйцевые близнецы генетически идентичны и имеют схожие характеристики.


Данные о близнецах послужили основой для ряда других эконометрических исследований (Розенцвейг и Wolpin, 1980; Bronars и Grogger, 1994). Поскольку вероятность двойников в совокупности не является высокой, важным вопросом  является репрезентативность данной выборки. Одним из источников таких данных является перепись. Другой источник <<фестиваль близнецов>>, которые проводятся в Соединенных Штатах. Эшенфельтер и Крюгер (1994, с. 1158) сообщают, что их данные были получены из интервью, проведенном на 16-ом ежегодном фестивале близнецов.


Привлекательность использования данных близнецов заключается в том, что наличие общих эффектов от наблюдаемых и не наблюдаемых факторов может быть устранено путем моделирования различий между результатами близнецов. Например, Эшенфельтер и Крюгер оценили регрессионную модель разницы в ставках заработной платы между близнецами. Первая разность исключает эффекты от возраста, пола, этнической принадлежности, и так далее. Остальные объясняющие переменные различаются между школьным уровнем, который является переменной интереса, и переменной, такие как различия в годы пребывания в должности и семейном положении.


\subsection{Идентификация естественного эксперимента}

Естественный эксперимент образования имел бы полезное влияние на практике. Поощряя оппортунистической эксплуатации квази-экспериментальных данных, и с помощью моделирования структур, таких как POM главы 2, эконометрическая практика устраняет разрыв между наблюдений и экспериментальными данными. Понятия идентификации параметров, заключённые в рамки SEM будут расширены, чтобы включить определение меры, которое интересна с точки зрения политики. Основное преимущество использования данных из естественного эксперимента является то, что политика переменной может быть обоснованно рассматриваться как экзогенная. Однако при использовании данных естественного эксперимента, как и в случае социального эксперимента, выбор контрольной группы играет важную роль в определении достоверности выводов. Несколько потенциальных проблем, которые влияют на социальный эксперимент, такие как селективность и истощение смещённости, также останутся в случае естественных экспериментов. Эксперимент может применяться только к небольшой части генеральной совокупности, а также условия, при которых он происходит, не распространяются легко. Пример, приведенный в разделе 22.6 иллюстрирует этот момент в контексте разности в расностях.

\section{Практические соображения}


Хотя существует огромный спрос на микроданные, количество имеющихся баз данных можно пересчитать по пальцам. Мы предоставляем очень неполный список некоторых из очень известных американских баз. Для получения дополнительной информации, см. соответствующие веб-сайты для этих наборов данных. Многие из них позволяют загружать данные напрямую.


\subsection{Некоторые источники микроданных}

{\bf Панель изучения динамики доходов в (PSID)}: Основываясь на Центр по обзору исследований в Университете штата Мичиган, PSID является национальным опросом, который проводится с 1968 года. Сегодня она охватывает более 40 000 индивидов и собирает экономические и демографические данные. Эти данные были использованы для поддержки широкого спектра микроэконометрических анализов. Браун, Дункан и Стаффорд (1996) подводят последние разработки в PSID данных.


{\bf Текущее обследование населения (CPS)}: Это ежемесячная национальное обследования около 50000 домохозяйств, которая предоставляет информацию о характеристиках рабочей силы. Исследование проводилось в течение более 50 лет. Основные дополнения в выборке следовали каждому из десятилетних переписей. Для получения дополнительной информации об этом исследовании см. раздел 24.2. Оно является важным источником микроданных, которые поддержали многочисленные исследования особенностей рынка труда. Опрос был переработан в 1994 году (Поливка, 1996).


{\bf Национальное лонгитюдное обследование (NLS)}: NLS имеет четыре оригинальных когорты: NLS пожилых мужчин, NLS молодых мужчин, NLS пожилых женщины, и NLS молодых женщин. Каждый из первоначальных когорт является национальным ежегодным опросом более 5000 лиц, которые были неоднократно проинтервьюированы, начиная с середины 1960---х годов. Опросы содержат информацию об опыте работы каждого респондента, образование, обучение, доход семьи, состав семьи, семейное положение, и здоровье. Дополнительные данные возраста, пола и т.д. также имеются.


{\bf Национальное лонгитюдное обследование молодежи (NLSY)}: NLSY является национальным ежегодного опросом 12686 молодых мужчин и женщин, в возрасте от 14 до 22 лет, когда они были впервые обследованных в 1979 году; содержит три подвыборки. Данные обеспечивают уникальную возможность для изучения всего жизненного цикла большой выборки молодых людей, которые являются представителями американских мужчин и женщин, родившихся в конце 1950---х и начале 1960---х. Вторая NLSY началась в 1997 году.


{\bf Обследование доходов и участия в программах (SIPP)}: SIPP --- лонгитюдное обследование около 8000 единиц домохозяйств в месяц. Оно охватывает источники доходов, участие в правовых программах, корреляция между этими предметами, и отдельных вложений на рынке труда с течением времени. Это многопанельное обследования, при котором новая панелью внедряется в начале каждого календарного года. Первая панель SIPP была начата в октябре 1983 года. По сравнению с CPS, SIPP имеет меньше занятых и больше безработных лиц.


{\bf Изучение здравоохранения и пенсионеров (HRS)}: HRS -- лонгитюдное национального обследование, которое состоит из интервью с членами 7600 домохозяйств в 1992 году (респондентов в возрасте от 51 до 61) с последующим интервью каждые два года в течение 12 лет. Данные содержат огромное количество экономической, демографической и медико---санитарной информации.


{\bf Изучение уровня жизни Всемирным Банком (LSMS)}: Данные Всемирного Банка содержат обследование домашних хозяйств  <<по многим аспектам благосостояния домохозяйств, которые могут быть использованы для оценки благосостояния домохозяйств, для понимания поведения домохозяйств, и оценивания влияния различной политики правительства на условий жизни населения>> во многих развивающихся странах. Многие примеры использования этих данных можно найти в Deaton (1997) и в экономической литературе. Грош и Glewwe (1998) подчёркивают характер данных и предоставляют ссылки на исследования, в которых использовались эти данные.


{\bf Данные расчетные палаты}: Межвузовский консорциум политических и социальных исследований (ICPSR) обеспечивает доступ ко многим базам данных, в том числе PSID, CPS, NLS, SIPP, Национальным обследованиям медицинских расходов (NMES), и многим другим. Американское Бюро статистики труда обрабатывает данные CPS и NLS. Американское Бюро переписи населения обрабатывает данные SIPP. Американский Национальный центр статистики здравоохранения обеспечивает доступ ко многим наборам данных по здоровью. Полезный канал к европейским данным является архив Совет Европейских Социальных Наук (CESSDA), который содержит ссылки на несколько европейских национальных архивов данных.


{\bf Данные из архива журналов}: Для некоторых целей, таких как копирование опубликованных результатов для работы в классе, вы можете получить данные из архивов журнала. Два архива можно загрузить посредством интернет-браузера. The Journal of Business and Economic Statistics архив данных, используе большинство, но не все статьи, опубликованные в этом журнале. The Journal of Applied Econometrics содержит данные, относящиеся к большинству статей, опубликованных с 1994 года.



\subsection{Обработка микроданных}


Микроэкономический наборы данных, как правило, очень большой. Выборка из нескольких сотен или тысяч наблюдений являются типичным явлением, и даже десятки тысяч наблюдений не так удивительно. Распределения результатов чаще всего не поддаётся нормальному распределению, потому что эти данные являются дискретными. Обработка больших наборов данных создает некоторые проблемы обобщения и описания важных особенностей данных. Часто бывает полезно использовать одну вычислительную среду (программу) для извлечения, восстановление и подготовки данных, а другую для  оценки моделей.


\subsection{Подготовка данных}

Самая основная особенность микроэконометрического анализа является то, что процесс получения выборки, которая должна будет использоваться в исследовании, скорее всего, будет долгим. Важно точно документировать решения и выбор, на основе которого делалась <<очистка>> данных. Рассмотрим несколько конкретных примеров.


Одним из наиболее общих черт данных выборочного обследования является неполученные или частичные ответы. Проблемы неполучение уже обсуждались. Частичный ответ обычно означает, что некоторые части опросных анкет остались без ответа. Если при этом, некоторая часть из необходимой информации недоступна, описываемые наблюдения, будут удалены. Это называется listwise удаления. Если эта проблема возникает в значительном числе случаев, то она должна быть тщательным образом проанализирована и сообщена, потому что это может привести к не репрезентативности выборки и погрешности в оценке; этот вопрос анализируется в главе 27. 


Вторая проблема заключается в погрешности измерения отчетных данных. Микроэкономические данные, как правило, noisy. Степень, тип и серьезность ошибки измерения зависит от типа обследования кросс---секции или панели, человека, который отвечает на опрос, и переменных, о котором запрашивается информация. Deaton (1997) исследовал некоторые из источников погрешностей измерений с особым упором на данные Всемирного Банка, хотя некоторые из поднятых вопросов имеют более широкое значение. Отклонения от измерения ошибки зависит от того, что делается на данных в терминах преобразований (например, первых разностей). Следовательно, чтобы сделать информативное заявления о серьезности смещения, возникающего в результате погрешности, нужно проанализировать четко определенные модели. В последующих главах будут приведены примеры влияния ошибки измерения в конкретных условиях.


\subsection{Проверка данных}

В больших наборов данных легко могут встретиться ошибочные данные, полученные от неправильного ввода с клавиатуры или кодирование ошибок. Поэтому следует применять некоторые элементарные способы проверки, позволяющие судить об существовании проблем. Прежде, чем исследовать некоторые описательной статистики, необходимо проверить данные. Во-первых, использовать сводные статистические данные (минимальное, максимальное, среднее и медианное), чтобы убедиться, что данные находятся в надлежащем отрезка и соответствующем масштабе. Например, категориальные переменные должно быть между нулем и единицей, численные должны быть больше или равны нулю. Иногда отсутствующие данные кодируются как -999, или каким---либо другим целым числом, поэтому позаботьтесь о том, чтобы этих чисел не было. Во-вторых, надо знать, какие изменения являются дробными, а какие процентными. В-третьих, можно использовать гистограмму для выявления проблемных наблюдений.  Проверка наблюдений до оценивания, может также подразумевать нормализацию и / или предположение о функции распределения, для моделирования определенного набора данных. Наконец, может быть важно, проверить шкалы измерения переменных. Для некоторых целей, таких как использование нелинейных оценок, желательно масштабировать переменные, так чтобы они имели приблизительно одинаковые масштабы. Описательные статистики могут быть использованы для проверки того, что средние, дисперсии и ковариации переменных имеют корректное масштабирование.

\subsection{Представление описательных статистик}

Так как микроданные, как правило, большие, очень важно, предоставить в начальной таблице описательной статистики, как правило, среднее, стандартное отклонение, минимум и максимум для каждой переменной. В некоторых случаях неожиданно большие или малые значения могут выявить наличие грубой ошибки записи или ошибочное включение неправильных значений  данных. Для дискретных переменных могут быть полезны гистограммы, а для непрерывных переменных информативными являются графики плотности распределения.




\part{Основные методы}

В части два представлены основные методы --- метод наименьших квадратов, метод максимального правдоподобия и метод моментов, а также соответствующие методы статистических выводов для нелинейных моделей, которые наиболее важны в микроэконометрике. Материал также покрывает такие современные сюжеты, как квантильная регрессия, последовательное оценивание, эмпирическая функция правдоподобия, полупараметрическая и непараметрическая регрессия, статистические выводы основанные на бутстрепе. В целом уровень изложения материала рассчитан на то, чтобы дать практику уровень подготовки, достаточный для чтения и понимания статей в ведущих эконометрических журналах, и, где это необходимо, материала следующих глав. Мы предполагаем предварительное знакомство читателя с линейной регрессией.

Основа теории оценивания представлена в трех следующих главах. Глава 4 начинает изложение с линейной модели регрессии. Она также на вводном уровне покрывает квантильную регрессию, которая моделирует не условное среднее, а другие особенности распределения зависимой переменной. Подробно излагаются инструментальные переменные --- основной метод исследования причинно-следственных связей. В главе 5 представлены основные методы оценивания нелинейных моделей. Она начинается с М-оценок, затем излагается метод максимального правдоподобия и нелинейным метод наименьших квадратов. Глава 6 содержит подробное изложение обобщенного метода моментов, который является общим методом, применимым и к линейным, и к нелинейным моделям, состоящим как из одного уравнения, так и из нескольких. В главе также приведен частный случай использования инструментальных переменных.

Затем мы обращаемся к тестированию моделей. В главе 7 представлены классический подход к тестированию гипотез и бутстрэп. А в главе 8 представлены более современные методы выбора моделей и анализа спецификации. Из-за своей важности бутстрэп методы, требующие большого объема вычислений, занимают отдельную 11 главу в части 3. Особенностью данной книги является то, что по возможности тесты излагаются в единообразной манере в этих трех главах. Далее алгоритмы иллюстрируются примерами на протяжении всей книги. 

Глава 9 стоит отдельно от остальных, в ней излагаются непараметрические и полупараметрические методы, налагающие слабые предпосылки на структуру эконометрической модели.

В главе 10 изложены численные методы, используемые при вычислении нелинейных оценок из глав 5 и 6. Этот материал становится важным для практика, если необходимые оценки не реализованы в статистическом пакете, или если появляются существенные численные трудности.




\chapter{Линейные модели}

\section{Введение}

Внушительное число эпирических микроэконометрических исследований использует линейную регрессию и её различные расширения. Прежде чем перейти к линейным моделям, на которых фокусируется эта книга, мы предоставляем обзор некоторых важных результатов для модели регрессии с одной зависимой переменной с пространственными (cross-section) данными. Представлены несколько различных оценок модели линейной регрессии.

Метод наименьших квадратов (МНК, или ordinary least squares --- OLS) особенно популярен. Для типичных микроэконометрических пространственных данных ошибки в модели часто могут быть гетероскедастичными. Статистический анализ должен быть устойчивым к гетероскедастичности, и можно получить выигрыш в эффективности, используя взвешенный метод наименьших квадратов.

Оценка МНК минимизирует сумму квадратов остатков. Одна из альтернатив, минимизация суммы модулей остатков, порождает оценку метода наименьших модулей. Эта оценка также представлена вместе с её расширением до квантильной регрессии.

Различные ошибки спецификации модели могут приводить к несостоятельности оценки МНК. В таких случаях вынесение суждений о параметрах, интересующих экономиста, может потребовать более сложных процедур, которые описаны с достаточным объёмом и глубиной в других частях этой книги. Метод инструментальных переменных --- одна из наиболее часто используемых процедур. Текущая глава содержит вводное описание этого важного метода и дополнительно указывает сложности, связанные со слабыми инструментами.

Раздел 4.2 предоставляет определение регрессии и различные функции потерь, которые приводят к разным оценкам функции регрессии. Пример представлен в разделе 4.3. Некоторые популярные процедуры оценки, а именно метод наименьших квадратов, взвешенный метод наименьших квадратов и квантильная регрессия представлены в разделах 4.4, 4.5 и 4.6 соответственно. Ошибки спецификации модели рассмотрены в разделе 4.7. Метод	инструментальных переменных представлен в разделах 4.8 и 4.9. Разделы 4.3-4.5, 4.7 и 4.8 охватывают стандартный материал вводных курсов, тогда как разделы 4.2, 4.6 и 4.9 вводят более продвинутый материал.

\section{Регрессии и функции потерь}
В современной микроэконометрике термин \textbf{регрессия} относится к огромному множеству процедур, изучающих отношение между зависимой переменной $y$ и набором независимых переменных $x$. Поэтому будет полезным начать с мотивации и обоснования некоторых распространённых типов регрессий.

Для начала будет правильно думать о цели регрессии как об \textbf{условном прогнозе} $y$ при заданном $x$. На практике модели регрессии также используют для других целей, в особенности для изучения причинно-следственной связи. Но даже тогда функция прогноза содержит полезное описание данных и представляет интерес. См. Раздел 4.2.3, где даётся различие между линейным прогнозом и изучением причинно-следственной связи, основанном на линейном причинном среднем.

\subsection{Функции потерь}
Обозначим символом $\hat{y}$  прогноз, т.е. оценку $y$, определённую как функция от $x$. Пусть $e \equiv y - \hat{y} $ обозначает \textbf{ошибку прогноза}, и пусть
\begin{equation}
\Loss(e)=\Loss(y - \hat{y}).
\end{equation}
обозначает \textbf{потери}, связанные с ошибкой $e$. Мы предполагаем, что прогноз формирует основу некоторого решения, и ошибка прогноза приводит к уменьшению полезности для лица, принимающего решения. Функция потерь обозначается как $\Loss(e)$, а её функциональная форма определяется лицом, принимающим решения. Одним из свойств функции потерь является возрастание по $|e|$.

Рассматривая $(y, \hat{y})$ как случайный вектор, лицо, принимающее решения, минимизирует ожидаемое значение функции потерь $\E{[\Loss(e)]}$. Если прогноз зависит от $x$, $K$-мерного вектора, то \textbf{ожидаемые потери} выражаются как
\begin{equation}
\E{[\Loss((y - \hat{y})|x)]}
\end{equation}
Выбор функции потерь должен зависеть от действительных потерь, связанных с ошибкой прогноза. В некоторых ситуациях, таких, как прогнозирование погоды, могут существовать серьёзные основания предпочесть одну функцию потерь другой.

В эконометрике часто не существует ясных указаний, но общепринятым считается специфицировать функцию потерь как квадратичную. Тогда (4.1) уточняется до $\Loss(e)=e^2$, и согласно (4.2) оптимальный прогноз минимизирует $\E{[\Loss(e|x)]}=\E{[e^2|x]}$. Следовательно, в этом случае критерий наименьшей среднеквадратической ошибки предсказания используется для сравнения качества прогнозов.

\subsection{Оптимальный прогноз}
Подход теории принятия решений к выбору \textbf{оптимального прогноза} описывается в терминах \textbf{минимизации ожидаемых потерь},

\[
\min_{\hat{y}} \E{[\Loss((y - \hat{y})|x)]}.
\]

Следовательно, свойство оптимальности определено относительно функции потерь лица, принимающего решения.

\begin{table}[h]
\caption{\label{tab:loss}Функции потерь и соответствующие оптимальные прогнозы}
\begin{tabular}[t]{lll}
\hline
\hline
\bf{Тип функции потерь} & \bf{Определение} & \bf{Оптимальный прогноз}  \\
\hline
Квадрат ошибки &  $\Loss(e)=e^2$  & $\E{[y|x]}$ \\
Модуль ошибки &  $\Loss(e)=|e|$  & $\mathrm{med}[y|x]$ \\
Асимметричный модуль ошибки &  $\Loss(e)= \left\{
\begin{array}{c l}      
    (1-\alpha) |e| &\text{  если  } e < 0 \\
    \alpha |e| & \text{ если } e \geqq 0
\end{array}
\right. $
 & $q_{\alpha}[y|x]$ \\
Дискретные потери &  $\Loss(e)=  \left\{
\begin{array}{c l}      
    0 & \text{ если } e < 0\\
    1  & \text{ если } e \geqq 0
\end{array}
\right. $ & $\mathrm{mod}[y|x]$ \\
\hline
\hline
\end{tabular}
\end{table}

Четыре распространённых примера функций потерь и связанные с ними функции оптимальных прогнозов приведены в таблице 4.1.  Мы предоставляем краткий обзор их всех по очереди. Подробный анализ можно найти в работе Мански (1988a). 

Наиболее известной функцией потерь является \textbf{квадратичная функция потерь}  (или среднеквадратичная функция потерь). Тогда оптимальным прогнозом является функция \textbf{условного математического ожидания} $\E{[y|x]}$. В самом общем случае на функцию $\E{[y|x]}$ не накладывается никаких ограничений, и оценкой является непараметрическая регрессия (см. Главу 9). Но чаще модель для $\E{[y|x]}$ специфицирована как $\E{[y|x]} = g(x, \beta)$, где $g(\cdot)$ --- специфицированная функция, а $\beta$ --- конечномерный вектор параметров, которые должны быть оценены. Тогда оптимален прогноз $\hat{y} = g(x, \hat{\beta})$, где $\hat{\beta}$ выбрана так, чтобы минимизировать потери в выборке 
\[
\sum_{i=1}^{N}\Loss(e_i) = \sum_{i=1}^{N}e_i^2 = \sum_{i=1}^{N}(y_i - g(x_i, \beta))^2
\].
Функция потерь задана как сумма квадратов остатков, и потому для оценки используется нелинейный МНК (см. Раздел 5.8). Если функция условного математического ожидания ограничена до линейной по $\beta$, так что $\E{[y|x]} = x' \beta $, то оптимальным прогнозом является $\hat{y}=x'\hat{\beta}$, где $\hat{\beta}$ --- оценка МНК, подробно описанная в Разделе 4.4. 

Если критерием потерь является \textbf{модуль ошибки}, то оптимальным прогнозом является \textbf{условная медиана}, обозначенная как $\mathrm{med}[y|x]$. Если функция условной медианы линейная, так, что $\mathrm{med}[y|x] = x'\beta$, то оптимальный прогноз --- $\hat{y}=x'\hat{\beta}$, где  $\hat{\beta}$ --- оценка методом наименьших модулей, минимизирующая $\sum_i |y_i - x_i ' \beta|$. Эта оценка представлена в Разделе 4.6.

Функции потерь, равные квадрату ошибки и модулю ошибки, симметричные, т.е. одинаковый штраф  налагается на ошибки одинаковые по модулю, независимо от их знака.  При использовании \textbf{асимметричного модуля ошибки}  функция потерь  налагает штраф  $(1-\alpha) |e|$ за слишком высокие прогнозы и штраф $\alpha |e|$ за слишком низкие. Параметр асимметрии $\alpha$ специфцирован. Он лежит в интервале $(0, 1)$, с симметрией при $\alpha = 0.5$ и увеличением асимметрии при приближении $\alpha$ к 0 или 1. Можно показать, что оптимальным прогнозом является \textbf{условная квантиль}, обозначаемая как $q_{\alpha}[y|x]$; условная медиана является её частным случаем при $\alpha = 0.5$. Условные квантили определены в Разделе 4.6, который представляет квантильную регрессию (Коэнкер и Бассетт, 1978).

Последняя функция, приведённая в таблице 4.1, это \textbf{дискретные потери}, которые основываются только на знаке ошибки предсказания независимо от её величины. Оптимальным предиктором является \textbf{условная мода}, обозначенная как mod[$y$|$x$]. Это создаёт обоснование для использования модальной регрессии (Ли, 1989).

Метод максимального правдоподобия не так легко умещается в рамки текущего раздела, связанные с предсказанием. Однако ему можно дать интерпретацию, связанную с ожидаемыми потерями, как предсказывающему плотность распределения и минимизирующему информацию Куллбэка-Либлера (см. Раздел 5.7).

Только что обозначенные результаты означают, что эконометрист, заинтересованный в оценке предсказывающей функции по данным ($y$, $x$) должен выбрать метод оценки, соответствующий функции потерь. Использование популярной линейной регрессии означает, по крайней мере неявно, что лицо, принимающее решения, имеет квадратичную функцию потерь и верит, что функция условного математического ожидания линейна. Однако если специфицирована одна из трёх других функций потерь, то оптимальный предиктор будет основан на одном из трёх других типов регрессий. На практике однозначной причины предпочесть определённую функцию потерь может не существовать.

Регрессии часто используют для описания данных, а не для предсказания как такового. В этом случае может быть полезным рассмотреть целый ряд оценок, так как альтернативные оценки могут предоставлять полезную информацию о чуствительности оцениваемых параметров. Мански (1988a, 1991) заметил, что обе функции, квадрата ошибки и модуля ошибки, являются выпуклыми. Если условное распределение $y|x$ симметрично, то обе оценки, условного математического ожидания и условной медианы, состоятельны, и можно ожидать, что они будут близки друг к другу. Более того, если избегать предположений о распределении $y|x$, то различия между альтернативными оценками могут предоставить хороший способ получения информации о распределении данных.

\subsection{Линейное предсказание}
Оптимальным предиктором при квадратичной функции потерь является условное математическое ожидание $E{[y|x]}$. Если оно линейно по $x$, так что $\E{[y|x]} = x' \beta $, то параметр $\beta$ имеет структурную или причинно-следственную интерпретацию, и состоятельная оценка $\beta$ означает состоятельную оценку $\E{[y|x]} = x' \beta $. Это позволяет 	содержательно анализировать эффект, оказываемый изменениями в регрессорах на условное среднее.

Если, однако, условное математическое ожидание нелинейно по $x$, так что $\E{[y|x]} \neq x, \beta)$, структурная интерпретация МНК исчезает. Однако всё ещё можно интерпретировать $\beta$ как наилучший линейный предиктор при квадратичной функции потерь. Дифференцирование ожидаемых потерь $\E{[(y - x' \beta)^2]}$ по $\beta$ даёт условия первого порядка $-2 \E{[y - x' \beta]} =0 $, так что оптимальным линейным предиктором является $\beta = (\E[xx'])^{-1}\E[xy]$ c оценкой МНК как аналогом для выборки.

Обычно мы специфицируем модели со свободным членом. Изменив обозначения, мы определим $x$ как  множество регрессоров кроме свободного члена, и заменим $x' \beta$ на $\alpha + x' \gamma$. Условия первого порядка относительно $\alpha$ и $\gamma$ выражаются как $-2\E{[u]}=0$ и $-2\E{[x u]}=0$, где $u = y - (\alpha + x' \gamma)$. Отсюда следует, что $\E{[u]}=0$ и $\Cov[x,u]=0$. Решением системы является
\begin{equation}
\begin{matrix}
\gamma = (\V[x])^{-1}\Cov[x,y], \\
\alpha = \E[y] - \E[x'] \gamma;
\end{matrix}
\end{equation} 
см, например, Goldberger (1991, стр. 52).

Из выведения (4.3) должно быть ясно, что для данных $(y, x)$ мы всегда можем написать модель линейной регрессии 
\begin{equation}
 y = \alpha + x' \gamma + u ,
\end{equation}
где параметры  $\alpha$ и $\gamma$ определены в (4.3) и ошибка $u$ удовлетворяет  $\E[u]=0$ и $\Cov[x,u]=0$.

Таким образом, модели линейной регрессии всегда можно придать неструктурную интерпретацию, или интерпретацию приведённой формы, как \textbf{наилучшему линейному предсказанию} (или линейной проекции) при квадратичной функции потерь. Однако, чтобы условное математическое ожидание было линейным, так, что $\E{[y|x]} = \alpha + x' \gamma$, требуется дополнительное предположение, что $\E{[u|x]} = 0$, вдобавок к уже имеющимся условиям $\E{[u]}=0$ и $\Cov[x,u]=0$. 

Это различие важно с практической точки зрения. Например, если $\E{[u|x]} = 0$, так, что  $\E{[y|x]} = \alpha + x' \gamma$, то предел по вероятности оценки наименьших квадратов $\hat{\gamma}$  равен $\gamma$ независимо от того, взвешенный или обыкновенный МНК используется, и от того, получена ли выборка простым случайным отбором или экзогенным стратифицированным. Если, однако, $\E{[y|x]} \neq \alpha + x' \gamma$, то эти различные оценки наименьших квадратов могут иметь разные пределы по вероятности. Этот пример обсуждается дальше в Разделе 24.3.

Структурная интерпретация МНК требует, чтобы условное ожидание случайного члена при любых заданных регрессорах равнялось нулю.


\section{Пример: отдача от образования}
Яркий пример применения линейной регрессии из эконометрики труда ставит задачу измерения влияния образования на заработную плату или доходы.

Типичная модель отдачи от образования специфицирует
\begin{equation}
\mathrm{ln} w_i = \alpha s_i + x_{2i}' \beta + u_i, i=1, ..., N,
\end{equation}
где $w$ обозначает почасовую заработную плату или годовой доход, $s$ обозначает число лет законченного образования, а $x_{2}$ обозначает контрольные переменные, такие, как опыт работы, пол и семейное происхождение. Индекс $i$ обозначает $i$-того индивида в выборке. Поскольку зависимая переменная --- логарифм заработной платы, модель является лог-линейной, и коэффициент $\alpha$  измеряет пропорциональные изменения заработной платы, связанные с увеличением образования на один год.

Для оценки этой модели наиболее часто используется метод наименьших квадратов. Трансформация ln$w$ на практике делает ошибки приблизительно гомоскедастичными, но, тем не менее, лучше использовать оценки стандартных ошибок, устойчивые к гетероскедастичности, как это описано в Разделе 4.4. Оценить модель также можно с помощью квантильной регрессии (см. Раздел 4.6), если интерес представляют вопросы распределения, такие, как поведение нижней квартили. 

Регрессия (4.5) может быть немедленно использована в описательных целях. Например, если $\hat{\alpha}=0.10$, то увеличение образования на один год связано с на 10\% более высоким заработком при контроле на все переменные, включённые в $x_{2}$. Последнее уточнение важно добавить, поскольку в этом примере оценка  $\hat{\alpha}$ обычно уменьшается по мере того, как $x_{2}$ расширяется за счёт новых контрольных переменных, которые могут повлиять на заработок.

Интерес с точки зрения политики заключается в определении влияния \textit{экзогенного изменения} в образовании на заработную плату. Однако образование не является присвоенным индивиду случайно, напротив, это исход, который зависит от выбора, сделанного индивидом. Теория человеческого капитала рассматривает образование как инвестицию самих индивидов, и $\alpha$ интерпретируется как мера отдачи на человеческий капитал. Регрессия (4.5) в этом случае является регрессией одной эндогенной переменной, $y$, на другую эндогенную переменную, $s$, и потому не отражает причинно-следственное воздействие экзогенного изменения $s$. Функция условного математического ожидания в этом случае бессмысленна с точки зрения причинности, потому что она <<условная>> относительно фактора --- образования --- являющегося \textit{эндогенным}. Действительно, пока мы не докажем, что $s$ сама является функцией от переменных, хотя бы одна из которых варьируется независимо от $u$, неясно, что может значить интерпретация $\alpha$ как причинно-следственного параметра.

Такая обеспокоенность по поводу эндогенных регрессоров с данными, являющимися наблюдениями относительно индивидов, пронизывает микроэконометрический анализ. Стандартные предположения модели линейной регрессии, приведённые в Разделе 4.4, требуют, чтобы регрессоры были экзогенными. Последствия эндогенности регрессоров рассмотрены в Разделе 4.7. Один из методов учёта эндогенных регрессоров, инструментальные переменные, описан в Разделе 4.8. Недавний широкий обзор способов учёта эндогенности в этом примере с образованием и заработной платой приведён в Angrist и Krueger (1999). Эти методы описаны в Разделе 2.8 и представлены на протяжении всей этой книги.


\section{Метод наименьших квадратов}

Простейшим примером регрессии является МНК в линейной регрессионной модели.

После определения модели и оценки дано весьма детальное представление асимптотического распределения оценки МНК. Такое рассмотрение предполагает, что читатель уже знаком с более вводным представлением материала. Предположения, на которых базируется модель здесь, допускают стохастические регрессоры и гетероскедастичные ошибки и предназначены для данных, полученных методом экзогенной стратифицированной выборки. 

Ключевой результат по получению устойчивых к гетероскедостичности оценок стандартных ошибок оценки МНК дан в Разделе 4.4.5.

\subsection{Линейная регрессионная модель}

В стандартной пространственной модели регрессии с $N$ наблюдениями скалярной зависимой переменной и нескольких регрессоров данные специфицированы как $(y,X)$, где $y$ обозначает вектор наблюдений независимой переменной, а $X$ обозначает матрицу объясняющих переменных.

В общем виде регрессионная модель с аддитивными ошибками в вектороном виде записывается как 
\begin{equation}
y = \E{[y|X]}+u,
\end{equation}
где $\E{[y|X]}$ обозначает условное математическое ожидание случайной величины $y$ при данном $X$, а $u$ обозначает вектор ненаблюдайных случайных ошибок или возмущений. Правая сторона уравнения раскладывает $y$ на две компоненты, одна из которых является фиксированной при данных регрессорах, а другая приписывается случайной вариации или шуму. Мы рассматриваем $\E{[y|X]}$ как функцию условного математического ожидания, которая определяет среднее или, говоря более точно, ожидаемое значение $y$ при данном $X$. 

\textbf{Линейная регрессионная модель} имеет место, если $\E{[y|X]}$ специфицирована до линейной функции от X. Обозначения для этой модели были подробно представлены в разделе 1.6. В векторной форме $i$-тое наблюдение равняется
\begin{equation}
y_i = x'_i \beta + u_i ,
\end{equation}
где $x_i$ --- \textbf{вектор регрессоров} размером $K\times 1$, а $\beta$ --- \textbf{вектор параметров} размером $K\times 1$. Иногда проще опустить индекс $i$ и описывать модель для типичного наблюдения как $y = x' \beta + u$. В матричном виде $N$ наблюдений соединены по строкам до  
\begin{equation}
y = X' \beta + u,
\end{equation}
где $y$ - \textbf{вектор зависимых переменных} размером  $N\times 1$, $X$ - \textbf{регрессионная матрица} размером $N\times K$, а $u$ - \textbf{вектор ошибок} размером  $N\times 1$. 

Уравнения (4.7) и (4.8) --- эквивалентные выражения для линейной регрессии и будут использоваться взаимозаменяемо. Второе из них более лаконичное и обычно более удобно для представления модели.

В такой постановке об $y$ говорят, как о \textbf{зависимой переменной}, \textbf{эндогенной переменной} или \textbf{регрессанде}, вариацию которой мы хотим представить через вариацию $X$ и $u$; об $u$ говорят, как об \textbf{случайной ошибке} или \textbf{возмущении}; об $X$ говорят, как о \textbf{регрессорах} или \textbf{предикторах}. Если к тому же выполняется Предположение 4 из Раздела 4.4.6, то все компоненты $x$ являются \textbf{экзогенными переменными} или \textbf{независимыми переменными}.

\subsection{Оценка МНК}

Оценка МНК определена как оценка, минимизирующая сумму квадратов ошибок
\begin{equation}
\sum_{i=1}^N u_i^2 = u'u = (y-X\beta)'(y-X\beta)
\end{equation}
Приравнивая производную по $\beta$ к нулю и выражая $\beta$, можно получить оценку МНК,
\begin{equation}
\hat{\beta}_{\text{МНК}} = (X'X)^{-1}X'y.
\end{equation}
Более общий результат представлен в Упражнении 4.5, где предполагается, что существует матрица, обратная к $(X'X)^{-1}$. Если ранг $(X'X)^{-1}$ не полный, вместо обратной матрицы можно использовать псевдообратную. В этом случае оценка МНК всё так же является наилучшим линейным предиктором $y$ при данном x, если используется квадратичная функция потерь, но множество других линейных комбинаций x также выдадут эту оптимальную оценку. 

\subsection{Идентификация}

Оценку МНК всегда можно вычислить при условии, что $X'X$ невырождена. Более интересным является вопрос, что $\hat{\beta}_{\text{МНК}}$ может нам сказать о данных.

Мы фокусируемся на способности оценки МНК реализовать идентификацию (см. Раздел 2.5) условного математического ожидания $\E[y|X]$. Для линейной модели параметр $\beta$ идентифицирован, если:
\begin{enumerate}
\item $\E{[y|X]}=X\beta$ и
\item $X\beta^{(1)}=X\beta^{(2)}$, если и только если $\beta^{(1)}=\beta^{(2)}$.
\end{enumerate}
Первое условие, что условное математическое ожидание правильно специфицировано, гарантирует, что $\beta$ сама по себе представляет интерес; второе условие эквивалентно тому, что матрица $X'X$ невырождена, то есть тому самому условию, которое требовалось для вычислимости единственной оценки МНК (4.10).

\subsection{Распределение оценки МНК}
Мы фокусируемся на асимптотических свойствах оценки МНК. Масштабируя оценку МНК, можно установить состоятельность и получить предельное распределение оценки. В этом случае статистические выводы можно делать, если состоятельно оценена ковариационная матрица оценки. Данный анализ широко использует асимптотичекую теорию, обзор которой дан в Приложении A.

\begin{center}
Состоятельность
\end{center}

Свойства оценки зависят от процесса, результатом которого стали имеющиеся данные --- \textbf{процесса, порождающего данные}, или \textbf{ППД} (data generating process, или dgp). Мы предполагаем, что ППД есть $Y = X'\beta + u$, так что модель (4.8) специфицирована корректно. В некоторых местах, особенно в Главах 5 и 6 и Приложении A к $\beta$ часто добавляется индекс $0$, так что ППД записывается как  $Y = X'\beta_0 + u$. См. обсуждение в Разделе 5.2.3.

Тогда
\[
\begin{array}{rcl}
\hat{\beta}_{\text{ МНК }}&=&(X'X)^{-1}X'y \\ 
&=& (X'X)^{-1}X'(X'\beta+u) \\
&=&(X'X)^{-1}X'X'\beta+(X'X)^{-1}X'u, \\
\end{array}
\]
и оценка МНК может быть выражена как
\begin{equation}
\hat{\beta}_{\text{МНК}} = \beta + (X'X)^{-1}X'u.
\end{equation}
Чтобы доказать состоятельность, мы перепишем (4.11) как
\begin{equation}
\hat{\beta}_{\text{МНК}} = \beta + (N^{-1}X'X)^{-1}N^{-1}X'u.
\end{equation}
Причина ренормализации в правой части состоит в том, что $N^{-1}X'X = N^{-1}\sum_i x_i x_i'$ является средним значением, которое сходится по вероятности к конечной ненулевой матрице, если $x_i$ удовлетоворяет предпосылкам, которые позволяют применить закон больших чисел к $x_i x_i'$ (см. подробности в Разделе 4.4.8). Тогда

\begin{center}
$\plim \hat{\beta}_{\text{МНК}} = \beta + (\plim N^{-1} X'X)^{-1}(\plim N^{-1}X'u)$, 
\end{center} 

согласно теореме Слуцкого (Теорема A.3). Оценка МНК \textbf{состоятельна} для $\beta$ (т.е. $\plim \hat{\beta}_{\text{МНК}}$), если 
\begin{equation}
\plim N^{-1}X'u = 0.
\end{equation}

Если к среднему $N^{-1}X'u = N^{-1} \sum_i x_i u_i$ можно применить закон больших чисел, то необходимым условием для (4.13) является $\E{x_i u_i}=0$.

\begin{center}
 Предельное распределение
 \end{center} 

При условии состоятельности предельное распределение $\hat{\beta}_{\text{МНК}}$ вырождено со всей массой, состредоточенной в $\beta$. Чтобы получить предельное распределение, $\hat{\beta}_{\text{МНК}}$ нужно умножить на $\sqrt{N}$, поскольку это масштабирование приводит к случайной величине, которая при стандартных предположениях для пространственной регрессии асимтотически имеет ненулевую, но конечную дисперсию. Тогда (4.11) становится
\begin{equation}
\sqrt{N} (\hat{\beta}_{\text{МНК}} - \beta) = (N^{-1}X'X)^{-1}N^{-1/2}X'u.
\end{equation}
Доказательство состоятельности основывалось на предположении, что $\plim N^{-1}X'X$ существует, конечный и ненулевой. Мы предполагаем, что можно применить центральную предельную теорему к $N^{-1/2}X'u$, чтобы получить многомерное нормальное предельное распределение с конечной невырожденной ковариационной матрицей. Применение правила произведения для предельных нормальных распределений  (Теорема A.17) означает, что произведение в правой части (4.14) имеет предельное нормальное распределение. Подробности предоставлены в разделе 4.4.8. 

Это приводит к следующему утверждению, которое допускает стохастические регрессоры и не ограничивает ошибки модели до гомоскедастичных и некоррелированных.

\textbf{Утверждение 4.1 (Распределение оценки МНК)}. Сделаем следующие предположения:
\begin{enumerate}[(i)]
 \item ППД описывается моделью (4.8), т.е. $y = X \beta +u$.
 \item Данные независимы по $i$ с $\E{[u|X]}=0$ и $\E{[uu'|X]}=\Omega = Diag[\sigma_i^2]$.
 \item Матрица $X$ имеет полный ранг, так что $X\beta^{(1)}=X\beta^{(2)}$ если, и только если $\beta^{(1)} = \beta^{(2)}$.
 \item Матрица размером $K \times K $
 \begin{equation}
 	M_{XX}= \plim N^{-1}X'X = \plim \frac{1}{N}\sum_{i=1}^N x_ix_i' = \lim \frac{1}{N} \sum_{i=1}^N \E{[x_i x_i']}
 \end{equation}
 существует, конечная и невырожденная.
 \item Вектор размером $K \times 1 $, $N^{-1/2}X'u=N^{-1/2}\sum_{i=1}^N x_ix_i' \xrightarrow{d} \mathcal{N}[0,M_{x \Omega x}] $, где
 \begin{equation}
 M_{x \Omega x} = \plim N^{-1} X'uu'X = \plim \frac{1}{N} \sum_{i=1}^N u_i^2 x_i x_i' = \lim \frac{1}{N} \sum_{i=1}^N \E [u_i^2 x_i x_i'].
 \end{equation}
\end{enumerate} 

Тогда оценка МНК $\hat{\beta}_{\text{МНК}}$, определённая в (4.10), состоятельна по отношению к $\beta$, и 
\begin{equation}
\sqrt{N}(\hat{\beta}_{\text{МНК}}-\beta) \xrightarrow{d} \mathcal{N} [0, M_{xx}^{-1}M_{x \Omega x} M_{xx}^{-1}].
\end{equation}


Предположение (i) используется, чтобы получить (4.11). Предположение (ii) обеспечивает $\E [y|X] = X\beta$  и допускает гетероскедастичные ошибки с дисперсией $\sigma_i^2$, более общие, чем гомоскедастичные некоррелированные ошибки, которые накладывают ограничение  $\Omega = \sigma^2 I$. Предположение (iii) исключает совершенную мультиколлинеарность регрессоров. Предположение (iv) приводит к масштабированию $X'X$ на $N^{-1}$  в (4.12) и (4.14). Заметим, что, согласно закону больших чисел, $\plim = \lim \E$ (см. Приложение, Раздел A.3).

Ключевым условием для состоятельности является (4.13). Вместо того, чтобы предположить это напрямую, мы использовали более сильное предположение (v), которое нужно, чтобы получить результат (4.17). При условии, что $N^{-1}X'u$ имеет предельное распределение с нулевым математическим ожиданием и конечной дисперсией, умножение на $N^{-1/2}$ порождает случайную величину, которая сходится по вероятности к нулю, и (4.13) выполняется, чего и требовалось. Предположение (v) требуется вместе с предположением (iv), чтобы получить предельное нормальное распределение (4.17), которое по теореме A.17 незамедлительно следует из (4.14). Более простые предположения на $u_i$ и $x_i$, нужные, чтобы обеспечить (iv) и (v), даны в Разделе 4.4.6, с формальным доказательством в разделе 4.4.8.

\begin{center}
 Асимптотическое распределение
 \end{center} 
Утверждение (4.1) даёт \textbf{предельное распределение} $\sqrt{N}(\hat{\beta}_{\text{МНК}}-\beta)$, масштабированной $\hat{\beta}_{\text{МНК}}$. Многие практики предпочитают видеть асимптотические результаты, записанные напрямую в терминах распределения $\hat{\beta}_{\text{МНК}}$, в случае чего распределение называется \textit{асимптотическим распределением}. Асимптотическое распределение следует понимать как применимое \textbf{в больших выборках}, то есть выборках, настолько больших, чтобы предельное распределение было достаточно хорошим приближением, но не настолько больших, что $\hat{\beta}_{\text{МНК}} \xrightarrow{d}\beta$, так как в последнем случае асимптотическое распределение было бы вырожденным. Обсуждение этого вопроса отражено в Приложении A.4.6.

Асимптотическое распределение можно получить из (4.17) делением на $\sqrt{N}$ и прибавлением $\beta$. Таким образом \textbf{асимптотическое распределение} 
\begin{equation}
\hat{\beta}_{\text{МНК}} \stackrel{a}{\sim} \mathcal{N} [\beta,N^{-1} M_{xx}^{-1}M_{x \Omega x} M_{xx}^{-1}], 
\end{equation}
где $\stackrel{a}{\sim}$ означает \textit{<<асимптотически распределено как>>}. Ковариационная матрица в (4.18) называется \textbf{асимптотической ковариационной матрицей} для $\hat{\beta}_{\text{МНК}}$ и обозначается как $\mathrm{V}[\hat{\beta}_{\text{МНК}}]$. Ещё более простой вариант обозначения опускает пределы и математические ожидания в определениях $M_{xx}$ и $M_{x \Omega x}$, и асимптотическое распределение обозначается как
\begin{equation}
\hat{\beta}_{\text{МНК}} \stackrel{a}{\sim} \mathcal{N} [\beta, (X'X)^{-1} X\Omega X (X'X)^{-1}],
\end{equation}
и $\mathrm{V}[\hat{\beta}_{\text{МНК}}]$ определяется как ковариационная матрица в (4.19).

Мы используем и (4.18), и (4.19) для обозначения асимптотического распределения в дальнейших главах. Их использование обусловлено удобством представления. Формальные асимптотические результаты, использующиеся для статиcтических заключений, основаны на предельном, а не на асимптотическом распределении.

При практическом использовании матрицы $M_{xx}$ и $M_{x \Omega x}$ в (4.17) и (4.18) заменяются на состоятельные оценки $\hat{M}_{xx}$ и $\hat{M}_{x \Omega x}$. Тогда \textbf{оценкой асимптотической ковариационной матрицы}  $\hat{\beta}_{\text{МНК}}$ является
\begin{equation}
\mathrm{\hat{V}}[\hat{\beta}_{\text{МНК}}] = N^{-1} \hat{M}_{xx}^{-1} \hat{M}_{x \Omega x} \hat{M}_{xx}^{-1}.
\end{equation}
Эта оценка называется \textbf{сэндвич-оценкой}, так как матрица $ \hat{M}_{x \Omega x}$ вложена между $\hat{M}_{xx}^{-1}$ и ещё одной $\hat{M}_{xx}^{-1}$.


\subsection{Стандартные ошибки для МНК, устойчивые к гетероскедастичности}

Очевидным выбором для $\hat{M}_{xx}$ в (4.20) является $N^{-1}X'X$. Оценка $ \hat{M}_{x \Omega x}$, определённая в (4.16), зависит от допущений о распределении случайного члена. 

В микроэкономических приложениях ошибки модели часто задаются условно гетероскедастичными, где $\mathrm{V}[u_i|x_i] = \E[u_i^2|x_i]= \sigma_i^2$ различается для различных $i$. White (1984a) предложил использовать  $ \hat{M}_{x \Omega x} = N^{-1}\sum_i \hat{u}_i^2 x_i x_i'$. Эта оценка требует дополнительных допущений, данных в Разделе 4.4.8.

Объединяя эти оценки $\hat{M}_{xx}$ и $\hat{M}_{x \Omega x}$ и упрощая выражение, можно получить оценку асимптотической ковариационной матрицы оценки МНК:
\begin{equation}
\mathrm{\hat{V}}[\hat{\beta}_{\text{МНК}}] = (X'X)^{-1} X\Omega X (X'X)^{-1} = (\sum_{i=1}^N x_i x_i')^{-1} \sum_i \hat{u}_i^2 x_i x_i (\sum_{i=1}^N x_i x_i')^{-1},
\end{equation}
где $\hat{\Omega} = \mathrm{Diag}[\hat{u}_i^2]$, а $\hat{u}_i = y_i - x_i'\hat{\beta}$ --- остаток МНК. Эта оценка, известная благодаря White (1980a), называется \textbf{устойчивой к гетероскедастичности} оценкой асимптотической ковариационной матрицы оценки МНК, и приводит к оценкам стандартных ошибок, которые называются \textbf{устойчивыми к гетероскедастичности стандартными ошибками}, или \textbf{робастными стандартными ошибками}. Они предоставляют состоятельную оценку для $\mathrm{\hat{V}}[\hat{\beta}_{\text{МНК}}]$, несмотря на то, что $\hat{u}_i$ не состоятельны по отношению к $\sigma_i^2$.

Во вводных курсах на ошибки накладывается допущение о \textbf{гомоскедастичности}. Тогда $\Omega = \sigma^2 \mathrm{I}$, так что $X'\Omega X = \sigma^2 X'X$, и потому $M_{x \Omega x} = \sigma^2 M_{xx}$. Ковариационная матрица предельного распределения в этом случае упрощается до $\sigma^2 M_{xx}^{-1}$, и многие компьютерные пакеты на её месте используют оценку, которая называется \textbf{оценкой по умолчанию} для ковариации оценок МНК:
\begin{equation}
 \mathrm{\tilde{V}}[\hat{\beta}_{\text{МНК}}] = s^2 (X'X)^{-1},
 \end{equation} 
где $s^2 = (N-K)^{-1} \sum_i \hat{u}_i^2$.

Выводы, сделаннные на основе (4.22), а не на (4.21) недействительны, если только ошибки не являтся гомоскедастичными и некоррелированными. В общем случае ошибочное использование (4.22), когда ошибки гетероскедастичные, как это часто бывает в пространственных данных, может приводить как к недооценке, так и к переоценке стандартных ошибок.

На практике $\hat{M}_{x \Omega x}$ вычисляется с делением на $(N-K)$, а не на $N$, для соответствия со сходным делением при расчёте $s^2$ в гомоскедастичном случае. Тогда $\mathrm{\hat{V}}[\hat{\beta}_{\text{МНК}}]$ в (4.21) умножается на $N/(N-K)$. В случае гетероскедастичных ошибок теоретического обоснования для этой коррекции степеней свободы не существует, но некоторые работы, использовавшие симуляцию, поддерживают её (см. MacKinnon и White, 1985, и Long и Ervin, 2000).

В микроэконометрическом анализе устойчивые стандартные ошибки используются всегда, когда это возможно. В данном случае ошибки устойчивы к гетеорскедастичности. Защита от других ошибок специфиации также может быть обеспечена. Например, если данные кластеризованы, стандартные ошибки также должны быть устойчивы к кластерам; см. Разделы 21.2.3 и 24.5.

\subsection{Предположения модели пространственной регрессии}

Утверждение 4.1 является весьма общей теоремой, которая основывается на допущениях о $N^{-1}X'X$ и $N^{-1/2}X'u$. На практике эти допущения удовлетворяют, применяя закон больших чисел и центральные предельные теоремы к средним значениям $x_i'x_i$ и $x_i u_i$. Это, в свою очередь, требует предположений о том, как наблюдения $x_i$ и ошибки $u_i$ были порождены, а, следовательно, как был порождён $y_i$. Об этих предположениях, взятых вместе, говорят, как о предположениях о \textbf{процессе, порождающем данные} (ППД). Простой обучающий пример приведён в Задаче 4.4.

На данном этапе нашей целью является сделать допущения, которые подходят для большого числа прикладных задач, использующих пространственные данные. Это допущения, описанные в White (1980a), и они включают в себя три важных расхождения с допущениями во вводных курсах. Во-первых, регрессоры могут быть стохастическими (Предположения 1 и 3 ниже), так что предположения об ошибках являются условными по отношению к регрессорам. Во-вторых, условная дисперсия ошибок может быть различна для различных наблюдений (Предположение 5). В-третьих, от ошибок не требуется быть нормально распределёнными.

Вот эти предположения:
\begin{small}
\begin{enumerate}
\item Данные $(y_i, x_i)$ независимые, но не распределены независимо от $i$ (inid).
\item Модель правильно специфицирована, так что
$$y_i = x_i' \beta + u_i .$$
\item Вектор регрессоров может быть стохастическим с конечной дисперсией, и $\E[|x_{ij}x_{ik}|^{1+\delta}]\leq \infty$ для всех $j, k = 1, ..., K$ для некоторой $\delta >0$, а матрица $M_{xx}$, определённая в (4.15) существует и является конечной положительно определённой матрицей ранга $K$. К тому же, в анализируемой выборке $X$ тоже имеет ранг $K$. 
\item Ошибки имеют нулевое математематическое ожидание при условии на регрессоры:
$$\E [u_i | x_i] = 0$$
\item Ошибки условно гетероскедастичны при условии на регрессоры с
\begin{equation}
\sigma_i^2 = \E[u_i^2 | x_i], \\
\Omega = \E [uu'|X] = \mathrm{Diag} [\sigma_i^2],
\end{equation}
где $\Omega$ является положительно определённой матрицей размера $N \times N$. К тому же, для некоторой $\delta>0$ верно $\E [|u_i^2|^{1+\delta}]< \infty$
\item Матрица $M_{x \Omega x}$, определённая в (4.16), существует и является конечной положительно определённой матрицей ранга $K$, где $M_{x \Omega x} = \plim N^{-1} \sum_{i} u_i^2 x_i x_i'$ при независимости от $i$. К тому же, для некоторой $\delta>0$ верно $\E [|u_i^2 x_{ij} x_{ik}|^{1+\delta}]< \infty$ для всех $j, k = 1, ..., K$. 
\end{enumerate}
\end{small}

\subsection{Примечания к предположениям}
Для полноты мы предоставляем подробное обсуждение каждого предположения, прежде чем перейти к ключевым результатам в следующем разделе.

\begin{center}
Стратифицированная случайная выборка
\end{center}

Предположение 1 --- одно из тех, которые часто делаются неявно при работе с пространственными данными. Здесь мы делаем его явным. Оно требует, чтобы $(y_i, x_i)$ были независимыми при различных $i$, но позволяет распределению различаться в зависимости от $i$. Многие микроэконометрические наборы данных основываются на \textbf{стратифицированной случайной выборке} (см. Раздел 3.2). В этом случае генеральная совокупность разбивается на страты, и из страт выбираются случайные наблюдения. Однако из одних страт наблюдения берутся с большей вероятностью, чем из других, вследствие чего выбранные $(y_i, x_i)$ $inid$ (независимые, но необязательно одинаково распределённые), а не $iid$ (независимые и одинаково распределённые, independent and identically distributed). Если же вместо этого данные происходили бы из \textbf{простой случайной выборки}, то $(y_i, x_i)$ $inid$ были бы независимыми и одинаково распределёнными, что является более строгим предположением, чем $inid$ и, по сути, его частным случаем. Во многих вводных курсах предполагается, что регрессоры являются \textbf{фиксированными в повторяющихся выборках}. В этом случае  $(y_i, x_i)$ $inid$, поскольку $y_i$ --- случайная величина, значение которой зависит от $x_i$. Предположение о фиксированных регрессорах редко выполняется для микроэконометрических данных, которые обычно основаны на наблюдении. С другой стороны, оно используется для экспериментальных данных, где $x$ - уровень воздействия.

Эти различные предположения о распределении $(y_i, x_i)$ затрагиваются отдельными законами больших чисел и центральными предельными теоремами, которые используются для получения асимптотических свойств оценки МНК. Заметим, что, даже если $(y_i, x_i)$ $iid$, $y_i$ при условии на $x_i$ не $iid$, поскольку, например, $\E[y_i|x_i] = x_i'\beta$ различается в зависимости от $x_i$.

Предположение 1 отбрасывает большую часть временных рядов, поскольку в них наблюдения обычно являются зависимыми. Оно также наршается, если устройство выборки предполагает кластеризацию наблюдений. Оценка МНК в этих случаях по-прежнему может быть состоятельной, если выполняются Предположения 2---4, но обычно её ковариационная матрица отличается от матрицы, представленной в этом разделе.

\begin{center}
Правильно специфицированная модель
\end{center}

Предположение 2 выглядит весьма очевидным, поскольку является важным ингредиентом при выводе оценки МНК. Однако его всё же стоит сделать явно, поскольку $\hat{\beta} = (X'X)^{-1} X'y$ --- функция от $y$, и её свойства зависят от $y$.

Если Предположение 2 выполняется, то предполагается, что модель регрессии линейна по $x$, что в регрессии нет \textit{пропущенных переменных}, и что в регрессорах нет \textbf{ошибки измерения}, так как регрессоры $x$, использующиеся для вычисления $\hat{\beta}$, являются теми же самыми, что были включены в процесс, порождающий данные. Вдобавок параметры $\beta$ одинаковы для всех индивидов, что исключает модели с переменными параметрами. 

Если Предположение 2 не выполняется, оценку МНК можно интерпретировать лишь как наилучшую из линейных оценок (см. Раздел 4.2.3).

\begin{center}
 Стохастические регрессоры
 \end{center} 
 
Предположение 3 допускает использование \textbf{стохастических регрессоров}, какими они обычно бывают, когда используются данные опросов, а не экспериментов. Предполагается, что в пределе выборочная ковариационная матрица постоянна и невырождена.

Если регрессоры распределены одинаково и независимо, как это предполагается при простой случайной выборке, то $M_{xx}=\E[xx']$, и Предположение 3 может быть упрощено до допущения, что вторые моменты существуют. Если регрессоры стохастические и независимые, но не обязательно одинаково распределённые, как это предполагается в стратифицированной выборке, необходимо более строгое Предположение 3, которое позволяет применить закон больших чисел Маркова, чтобы получить $\plim N^{-1}X'X$. Если регрессоры являются фиксированными в повторяющихся выборках --- обычное менее удовлетворительное предположение, которое часто делают во вводных курсах, --- то $M_{xx} = \lim N^{-1}X'X$, и Предположение 3 сводится к предположению, что этот предел существует.
 
 
\begin{center}
 Слабо экзогенные регрессоры
 \end{center} 
 
Предположение 4 о \textbf{нулевом условном математическом ожидании ошибок} необходимо, поскольку при использовании Предположения 2 из него следует, что $\E[y|X] = X\beta$, то есть условное математематическое ожидание действительно такое, каким моделируется.

Из предположения, что $\E[u|X]=0$, следует, что $\Cov[x,u]=0$, то есть ошибка не коррелирует с регрессорами. Это следует из того, что $\Cov[x,u] = \E[xu]-\E[x]\E[u]$, а $\E[u|x]$ означает, что $\E[u]=0$ и $\E[xu]=0$ по закону о вложенных математических ожиданиях. Более слабого предположения $\Cov[x,u]=0$ достаточно, чтобы МНК был состоятельным, тогда как более сильное предположение  $\E[u|X]=0$ требуется для несмещённости МНК.

Экономический смысл Предположения 4 состоит в том, что случайный член отображает все исключённые из модели факторы, которые, как предполагается, не коррелируют с $X$, и они, в среднем, не оказывают никакого влияния на $y$. Это ключевое предположение, которое в Разделе 2.3 было названо \textit{предположением о слабой эндогенности}. По существу это означает, что знание процесса, порождающего $X$, не содержит полезной информации для оценки $\beta$. Если данная предпосылка не выполняется, один или более из $K$ регрессоров называются \textit{совместно зависимыми} c $y$ или просто \textit{эндогенными}. Общий термин для корреляции регрессоров с ненаблюдаемыми ошибками --- \textbf{эндогенность}, или \textbf{эндогенные регрессоры}.  Термин <<эндогенный>> здесь означает вызванный факторами внутри системы. Как будет показано в Разделе 4.7, нарушение слабой экзогенности может привести к несостоятельности оценки. Существует множество способов нарушения экзогенности, но в одном из наиболее часто встречающихся одна из переменных $x$ является выбором или решением индивида, связанным с $y$ в большей модели. Если игнорировать эти взаимосвязи, относясь к $x_i$, как будто его значение было случайно присвоено наблюдению $i$ и, таким образом, было некоррелировано с $u_i$, последствия будут нетривиальными. \textbf{Эндогенный отбор} исключается Предположением 4. Вместо этого, если данные собраны из стратифицированной выборки, это должна быть \textbf{экзогенная стратифицированная выборка}.

\begin{center}
  Условно гетероскедастичные ошибки
\end{center}  

Предполагаются \textbf{независимые ошибки регрессии}, не коррелирующие с регрессорами, как следствие Предположений 1, 2 и 4. Вводные курсы обычно сосредоточивают внимание на гомоскедастичных ошибках с однородной или постоянной дисперсией, где $\sigma_i^2 = \sigma^2$ для всех $i$. Тогда эти ошибки называются $\mathrm{iid} (0,\sigma^2)$ и \textbf{сферическими ошибками}, поскольку $\Omega = \sigma^2 \times \mathrm{I}$.

Предположение 5, напротив, вводит \textbf{условно гетероскедастичные ошибки регрессии} где \textit{гетероскедастичные} означает неоднородные или различные дисперсии. Это предположение сформулировано через $\E[u^2|x]$, но это эквивалентно дисперсии $\mathrm{V}[u|x]$, поскольку $\E[u|x]=0$ по Предположению 4. Это более общее предположение о гетероскедастчиных ошибках сделано, потому что для пространственной регрессии оно часто оказывается верным. Более того, ослабление предположения о гомоскедастичности ничего не стоит, так как корректные стандартные ошибки для МНК можно вычислить, даже если функциональная форма гетероскедастичности неизвестна.

Термин \textit{условная} гетероскедастичность используется по следующей причине. Даже если $(y_i, x_i)$ независимо и одинаково распределены, как в случае простой случайной выборки, условное математическое ожидание и условная диспресия $y_i$ могут зависеть от $x_i$. Точно так же, ошибки $u_i = y_i - x_i'\beta$ idd в простой случайной выборке, и потому безусловно гомоскедастичны. Но как только мы поставим условие на $x_i$ и рассмотрим распределение $u_i$ \textit{при условии} $x_i$, дисперсия этого условного распределения может зависеть от $x_i$.

\begin{center}
Конечная предельная ковариационная матрица $N^{-1}X'u$ 
\end{center}
Предположение 6 нужно, чтобы получить предельную ковариационную матрицу $N^{-1}X'u$. Если регрессоры независимы от ошибок (более сильное допущение, чем Предположение 4), то из Предположения 5 ($\E [|u_i^2|^{1+\delta}]< \infty$) и Предположения 3 ($\E[|x_{ij}x_{ik}|^{1+\delta}]\leq \infty$) следует, что $\E[|u_i^2 x_{ij}x_{ik}|^{1+\delta}]\leq \infty$

Мы намеренно не сделали седьмого предположения о том, что ошибка $u$ нормально распределена при условии $X$. Предположение о нормальности нужно, чтобы получить точное распределение оценки МНК для конечных выборок. Однако на протяжении этой книги мы фокусируемся на асимптотических методах, поскольку точные распределения для малых выборок редко доступны для оценок, использующихся в микроэконометрике, и предположение о нормальности больше не нужно.

\subsection{Вывод оценки МНК}
Ниже мы представляем распределение оценки МНК как для малых выборок, так и в пределе, и обосновываем применение оценки Вальда ковариационной матрицы оценки МНК при выполнении Предположений 1---6.

\begin{center}
 Распределение в малых выборках
 \end{center} 

Параметр $\beta$ можно идентифицировать при выполнении предположений 1-4, поскольку в этом случае $\E	[y|X] = X\beta$ и $X$ имеет ранг $K$.

В маленьких выборках оценка МНК является несмещённой при выполнении предположений 1---4,  а её ковариационную матрицу легко получить, если истинно Предположение 5. Эти результаты можно получить, используя закон вложенных математических ожиданий, то есть сначало взять ожидание $u$ при условии $X$,  а затем взять безусловное математическое ожидание. Тогда из (4.11) следует
\begin{equation}
\begin{array}{rcl}
\E[\hat{\beta}_{\text{МНК}}]  & = & \beta+\E_{X,u}[(X'X)^{-1}X'u] \\
& = &  \beta+\E_{X}[\E_{u|X}[(X'X)^{-1}X'u|X]] \\
& = & \beta+\E_{X}[(X'X)^{-1}X'\E_{u|X}[u|X]] \\
& = & \beta, \\
\end{array}
\end{equation}
используя закон вложенны ожиданий (Теорема A.23) и при выполнении предположений 1 и 4, из которых следует, что $\E[u|X]=0$. Аналогично, из (4.11) следует
\begin{equation}
\mathrm{V}[\hat{\beta}_{\text{МНК}}] = \E_{X}[(X'X)^{-1}X'\Omega X (X'X)^{-1}],
\end{equation}
при Предположении 5, что $\E[uu'|X]=\Omega$. Мы использовали теорему A.23, говорящую, что в общем случае 
$$\mathrm{V}_{X,u}[g(X,u)] = \E_X[\mathrm{V}_{u|X}[g(X,u)]] + \mathrm{V}_{X}[\E_{u|X}[g(X,u)]]$$.
Это выражение можно упростить, обнулив второе слагаемое, поскольку $\E_{u|X}[(X'X)^{-1}X'u] = 0$.

Таким образом, оценка МНК является \textbf{несмещённой}, если $\E[u|X]=0$. Это ценное свойство, как правило, не обобщается на нелинейные оценки. Многие нелинейные оценки, такие, как нелинейный МНК, смещены, и даже линейные оценки, такие, как оценка метода инструментальных переменных, могут быть смещёнными. Оценка МНК \textbf{неэффективна}, так как её ковариационная матрица не является наименьшей ковариационной матрицей среди линейных несмещённых оценок, кроме случая, когда $\Omega = \sigma^2 \times \mathrm{I}$. Неэффективность МНК обосновывает использование других оценок, таких, как обобщённый МНК, хотя потеря эффективности МНК не всегда высока. При дополнительном допущении о нормальности ошибок при условии $X$, которое обычно не делается в микроэконометрических приложениях, оценка МНК нормально распределена при условии $X$.

\begin{center}
Состоятельность
\end{center}
Из Предположения 3 следует, что $\plim (N^{-1} X'X)^{-1}=M_{xx}^{-1}$. В этом случае состоятельность требует, чтобы выполнялось (4.13). Это доказывается, применяя закон больших чисел к среднему $N^{-1}X'u=N^{-1}\sum_i x_i u_i$, который сходится по вероятности к нулю, если $\E[x_i u_i]=0$. При выполнении предположений 1 и 2, $x_i u_i$ распределены независимо, и предположения 1---5 допускают использование закона больших чисел Маркова (Теорема A.9). Если Предположение 1 упрощено до независимости и одинакого распределения $(y_i,x_i)$, то $x_i u_i$ тоже распределены независимо и одинаково, и предположения 1---4 позволяют использовать более простой закон больших чисел Колмогорова (Теорема A.8).

\begin{center}
Предельное распределение
\end{center}

Согласно Предположению 3,  $\plim (N^{-1} X'X)^{-1}=M_{xx}^{-1}$. Требуется получить предельное распределение $N^{-1}X'u=N^{-1}\sum_i x_i u_i$, применив центральную предельную теорему. При выполнении предположений 1 и 2, $x_i u_i$ распределены независимо, и предположения 1---6 допускают использование центральной предельной теоремы Ляпунова (Теорема A.15). Если Предположение 1 упрощено до независимости и одинакого распределения $(y_i,x_i)$, то $x_i u_i$ тоже распределены независимо и одинаково, и предположения 1---5 позволяют использовать более простую центральную предельную теорему Линдеберга-Леви (Теорема A.14).

Отсюда следует, что 
\begin{equation}
\frac{1}{\sqrt{N}}X'u \xrightarrow{d} \mathcal{N} [0,M_{x \Omega x}],
\end{equation}
где $M_{x \Omega x} =\plim N^{-1} X'uu'X = \plim \sum_i u_i^2 x_i x_i'$, если наблюдения независимы по $i$. Применяя закон больших чисел, получаем $M_{x \Omega x} = \lim N^{-1} \sum_i \E_{x_i}[\sigma_i^2 x_i x_i']$, используя то, что $\E_{u_i,x_i}[u_i^2 x_i x_i'] = \E_{x_i}[\E[u_i^2|x_i]x_i x_i']$ и $\sigma_i^2 = \E[u_i^2|x_i]$. Следовательно, $,M_{x \Omega x} = \lim N^{-1} \E[X'\Omega X]$, где $\Omega = \mathrm{Diag}[\sigma_i^2]$, и математическое ожидание берётся только по $X$, а не по $X$ и $u$.

Данное представление материала предполагает независимость по $i$. В более общем случае мы можем допустить корреляцию между наблюдениями. Тогда $M_{x\Omega x} = \plim N^{-1} \sum_i \sum_j u_i u_j x_i x_j'$, а в ячейках $ij$ матрицы $\Omega$ содержится $\sigma_{ij} = \Cov[u_i, u_j]$. Это усложнение отложено до изучения оценивания методом нелинейного МНК в Разделе 5.8.

\begin{center}
Стандартные ошибки, устойчивые к гетероскедастичности
\end{center}

Рассмотрим ключевой шаг состоятельного оценивания $M_{x \Omega x}$. Начав с первоначального определения $M_{x \Omega x} = \plim \sum_i u_i^2 x_i x_i'$, заменим $u_i$ на $\hat{u}_i = y_i-x_i'\hat{\beta}$, где асимптотически  $\hat{u}_i \xrightarrow{p} u_i$, поскольку $\hat{\beta} \xrightarrow{p} \beta$. Отсюда выводится состоятельная оценка
\begin{equation}
\hat{M}_{x \Omega x} = \frac{1}{N} \sum_{i=1}^{N}\hat{u}_i^2 x_i x_i' = N^{-1} X' \hat{\Omega} X,
\end{equation}
где $\hat{\Omega} = \mathrm{Diag}[\hat{u}_i^2]$. Необходимо дополнительное предположение о том, что $\E [|x_{ij}^2 x_{ik} x_{il}|^{1+\delta}]< \Delta$ для положительных констант $\delta$  и $\Delta$ и $j,k,l = 1, ..., K$, так как $\hat{u}_i^2 x_i x_i' = (u_i - x_i'(\hat{\beta} - \beta))^2 x_i x_i'$ использует четвёртую степень $x_i$ (см. White (1980a)).

Заметим, что $\hat{\Omega}$ не сходится к матрице $\Omega$ размером $N \times N$. Без внесения дополнительной структуры это очевидно невозможная задача, поскольку требуется оценить $N$ дисперсий $\sigma_i^2$. Но всё, что нам нужно, это сходимость $N^{-1} X' \hat{\Omega} X$ к матрице $\plim N^{-1} X' \Omega X =  N^{-1} \plim \sum_i \sigma_i^2 x_i x_i'$  размером $K \times K$. Этого достичь проще, поскольку число регрессоров $K$ фиксировано. Чтобы понять оценку Уайта, рассмотрим оценку МНК для модели с одним только свободным членом $y_i = \beta + u_i$ и с гетероскедастичной ошибкой. Тогда в нашей системе обозначений мы можем показать, что $\hat{\beta} = \bar{y}$, $M_{xx} = \lim N^{-1} \sum_{i} 1 = 1$, а $M_{x \Omega x} = \lim N^{-1} \sum_{i} \E[u_i^2]$. Очевидно, $M_{x \Omega x}$ можно оценить с помощью $\hat{M}_{x \Omega x} = N^{-1} \sum_{i} \hat{u}_i^2$, где $ \hat{u}_i = y_i-\hat{\beta}$. Чтобы получить предел по вероятности этой оценки, достаточно рассмотреть $N^{-1}\sum_i u_i^2$, поскольку $\hat{u}_i - u_i \xrightarrow{p} 0$, если $\hat{\beta}\xrightarrow{p}\beta$. Если можно применить закон больших чисел, это среднее значение сходится по вероятности к своему математическому ожиданию, так что $\plim N^{-1} \sum_i u_i^2 = \lim N^{-1} \E[u_i^2] = M_{x \Omega x}$, как и требовалось. Eicker (1967) приводит формальные условия для этого примера.

\section{Взвешенный метод наименьших квадратов}

В ситуациях, когда требуется использовать робастные стандартные ошибки, обычно возможно увеличить эффективность оценки. Например, в присутствии гетероскедастичности оценка доступного обобщённого МНК (ОМНК) эффективнее, чем оценка МНК.

В этом разделе мы представляем оценку доступного ОМНК, которая использует более сильные предположения о дисперсии случайного члена. Тем не менее, можно получить стандартные ошибки доступного ОМНК, которые были бы устойчивы к неправильной спецификации дисперсии ошибки, как и в случае МНК.

Многие микроэконометрические работы не используют возможное преимущество в эффективности, которое даёт ОМНК, по причинам удобства и потому, что ожидают относительно небольшой выигрыш в эффективности. Вместо этого общепринятым является использование менее эффективных взвешенных оценок, особенно МНК, с робастными оценками стандартных ошибок.

\subsection{ОМНК и доступный ОМНК}

Согласно теореме Гаусса-Маркова, приводимой во вводных текстах, оценка МНК является эффективной в классе линейных несмещнных оценок, если ошибки в линейной регрессионной модели независимы и гомоскедастичны. 

Вместо этого, мы предполагаем, что ковариационная матрица ошибок равна $\Omega \neq \sigma^2\mathrm{I}$.  Если $\Omega$ известа и невырождена, мы можем домножить линейную модель регрессии (4.8) на $\Omega^{-1/2}$, где $\Omega^{-1/2}\Omega^{-1/2}=\Omega$, чтобы получить 
$$
\Omega^{-1/2} y  = \Omega^{-1/2} X\beta + \Omega^{-1/2} u
$$
Применив алгебру, получим $\V [\Omega^{-1/2} u] = \E [(\Omega^{-1/2} u)(\Omega^{-1/2} u)'|X] = \mathrm{I}$. Таким образом, ошибки в этой трансформированной модели обладают нулевым математическим ожиданием, одинаковой --- единичной --- дисперсией и некоррелированы друг с другом. Поэтому $\beta$ можно эффективно оценить с помощью МНК-регрессии $\Omega^{-1/2} y$ на $\Omega^{-1/2} X$. 

Этот аргумент создаёт \textbf{оценку обобщённого метода наименьших квадратов} (generalized least-squares):
\begin{equation}
\hat{\beta}_{\text{ОМНК}} = (X'\Omega^{-1}X)^{-1}X'\Omega^{-1}y
\end{equation}
Оценку ОМНК нельзя использовать непосредственно, поскольку на практике $\Omega$ неизвестна. Вместо этого мы специфицируем $\Omega = \Omega(\gamma)$, где $\gamma$ - конечномерный вектор параметров, получаем его состоятельную оценку $\hat{\gamma}$, и формируем $\hat{\Omega} =  \Omega(\hat{\gamma})$. Например, если ошибки гетероскедастичны, можно положить $\V[u|x] = exp (z'\gamma)$, где $z$ - подмножество $x$, а экспоненциальная функция используется, чтобы обеспечить положительную дисперсию. Тогда $\hat{\gamma}$ можно оценить состоятельно с помощью регрессии методом нелинейного МНК квадрата ошибки $\hat{u}^2_i = (y-x\hat{\beta}_{\text{МНК}})^2$ на $ exp (z'\gamma)$. Эту оценку $\hat{\Omega}$ можно затем использовать вместо $\Omega$ в (4.28). Заметим, что нельзя  заменить $\Omega$ в (4.28) на $\hat{\Omega} =  \mathrm{Diag}[\hat{u}^2_i]$, поскольку такая оценка будет несостоятельной (см. Раздел 5.6.8).

Оценкой \textbf{доступного обобщённого МНК} (feasible GLS) называют
\begin{equation}
\hat{\beta}_{\text{ДОМНК}} = (X'\hat{\Omega}^{-1}X)^{-1}X'\hat{\Omega}^{-1}y.
\end{equation}
Если Предположения 1---6 выполняются и $\Omega(\gamma)$ правильно специфицирована --- сильное предположение, которое в дальнейшем будет ослаблено, --- и $\hat{\gamma}$ состоятельно оценивает $\gamma$, то можно показать, что
\begin{equation}
\sqrt{N}(\hat{\beta}_{\text{ДОМНК}} - \beta) \xrightarrow{d} \mathcal{N} [0, (\plim N^{-1}X'\Omega X)^{-1}].
\end{equation}

Оценка доступного ОМНК имеет ту же предельную ковариационную матрицу, что и оценка ОМНК, и поэтому является эффективной. На практике в (4.30)  $\Omega$ заменяют на $\hat{\Omega}$. 

Можно показать, что оценка ОМНК минимизирует $u' \Omega^{-1} u$ (см. упражнение 4.5), что упрощается до $\sum_i u_i^2/\sigma_i^2$, если ошибки гетероскедастичные, но некоррелированы. Мотивацией для ОМНК было эффективное оценивание $\beta$. С точки зрения обсуждаемых в Разделе 4.2 функции потерь и оптимального предсказания, для гетероскедастичных ошибок можно указать функцию потерь $\Loss(e) = e^2/\sigma^2$. В сравнении с функцией потерь для МНК $\Loss(e) = e^2$, функция потерь ОМНК облагает относительно меньшим наказанием ошибки предсказания для наблюдений с большей условной дисперсией случайного члена. 

\subsection{Взвешенный МНК}

Результат (4.30) предполагает верную спецификацию ковариационной матрицы ошибок $\Omega(\gamma)$. Если это не так, оценка доступного ОМНК по-прежнему состоятельна, но (4.30) даёт неверную дисперсию. К счастью, можно найти робастную оценку дисперсии оценки ОМНК, даже если $\Omega(\gamma)$ специфицирована неверно.

Определим $\Sigma = \Sigma(\gamma)$ как \textbf{рабочую ковариационную матрицу}, которая не обязательно совпадает с истинной ковариационной матрицей $\Omega  = \E [uu'|X]$. Сформируем оценку $\hat{\Sigma} = \Sigma(\hat{\gamma})$, где $\hat{\gamma}$ --- оценка $\gamma$. Тогда можно использовать аналог ОМНК с матрицей весов $\hat{\Sigma}^{-1}$.

Так получается оценка \textbf{взвешенного МНК} (weighted least-squares, WLS):
\begin{equation}
\hat{\beta}_{\text{ВМНК}} = (X'\hat{\Sigma}^{-1}X)^{-1}X'\hat{\Sigma}^{-1}y.
\end{equation}
Статистические выводы делаются без использования предпосылки $\Sigma = \Omega$.  В литературе этот подход называют подходом рабочей матрицы весов. Мы называем его взвешенным МНК, но сознаём, что другие под этим названием подразумевают ОМНК или доступный ОМНК в особом случае, когда матрица $\Omega ^{-1}$ диагональная. Здесь мы не делаем таких допущений о матрице весов $\Sigma^{-1} = \Omega^{-1}$.

Используя алгебру, как при получении оценки МНК в Разделе 4.4.5, можно получить оценку асимптотической ковариационной матрицы. 
\begin{equation}
\hat{\V}[\hat{\beta}_{\text{ВМНК}}] = (X'\hat{\Sigma}^{-1}X)^{-1}X'\hat{\Sigma}^{-1}\hat{\Omega}\hat{\Sigma}^{-1}X(X'\hat{\Sigma}^{-1}X)^{-1},
\end{equation}
где $\Omega$ --- такая матрица, что
$$
\plim N^{-1} X'\hat{\Sigma}^{-1}\hat{\Omega}\hat{\Sigma}^{-1}X = \plim N^{-1} X'\Sigma^{-1}\Omega\Sigma^{-1}X.
$$
В случае гетероскедастичности $\hat{\Omega} = \mathrm{Diag}[\hat{u}_i^{*2}]$, где $\hat{u}_i^{*2} = y_i - x_i'\hat{\beta}_{\text{ВМНК}}$.

В случае гетероскедастичных ошибок базовым подходом является выбор простой модели гетероскедастичности, такой, где дисперсия ошибок зависит только от одного или двух ключевых регрессоров. Например, в линейной регрессионной модели уровня зарплат как функции от образования и других переменных гетероскедостичность можно моделировать как функцию одного только уровня образования. Пусть из этой модели можно получить $\hat{\Sigma} = \mathrm{Diag}[\hat{\sigma}_i^2]$. Тогда МНК-регрессия $y_i/\sigma_i$ на $x_i/\sigma_i$ (без константы) даёт оценку $\hat{\beta}_{\text{ВМНК}}$, а стандартные ошибки Уайта для этой регрессии, устойчивые к гетероскедастичности, как можно показать, описываются (4.32).

Подход взвешенного МНК или рабочей матрицы особенно удобен, когда есть более чем одна проблема. Например, в модели случайных эффектов, основанной на панельных данных (Глава 21), ошибки можно рассматривать одновременно как коррелированные во времени для данного индивида и гетероскедастичные. Можно использовать оценку случайных эффектов, которая устраняет только первую проблему, но потом рассчитать стандартные ошибки для этой оценки, устойчивые к гетероскедастичности.

Различные оценки метода наименьших квадратов приведены в таблице 4.2.
\begin{table}[h]
\begin{center}
\caption{\label{tab:gls}Оценки МНК и их асимптотические ковариационные матрицы}
\begin{tabular}[t]{lll}
\hline
\hline
\bf{Оценка\footnote{Оценки для линейной модели с условной ковариационной матрицей ошибок $\Omega$. Для ОМНК предполагается, что $\hat{\Omega}$ состоятельна. Для МНК и ВМНК устойчивая к гетероскедостичности ковариационная матрица $\beta$  использует $\hat{\Omega}$, равну диагональной матрице с квадратами остатков на диагонали.}} & \bf{Определение} & \bf{Оценка предельной дисперсии}  \\
\hline
МНК &  $\hat{\beta} = (X'X)^{-1}X'y$  & $(X'X)^{-1}X'\hat{\Omega}X(X'X)^{-1}$ \\
Доступный ОМНК &  $\hat{\beta} = (X'\hat{\Omega}^{-1}X)^{-1}X'\hat{\Omega}^{-1}y$   & $(X'\hat{\Omega}^{-2}X)^{-1}$ \\
Взвешенный МНК & $\hat{\beta} = (X'\hat{\Sigma}^{-1}X)^{-1}X'\hat{\Sigma}^{-1}y$   &
$(X'\hat{\Sigma}X)^{-1}X'\hat{\Sigma}^{-1}\hat{\Omega}\hat{\Sigma}^{-1}X(X'\hat{\Sigma}X)^{-1}$ \\
\hline
\hline
\end{tabular}
\end{center}
\end{table}

\subsection{Пример робастных стандартных ошибок}

В качестве примера оценки робастных стандартных ошибок рассмотрим таковую для стандартных ошибок коэффициента наклона в модели с мультипликативной гетероскедастичностью
$$
y = 1 + 1\times x +u,
$$
$$
u = x\epsilon,
$$
где скалярный регрессор $x \sim \mathcal{N} [0, 25]$, а $\epsilon\sim \mathcal{N} [0, 4]$.

Эти ошибки обладают условной гетероскедастичностью, так как $\V [u|x] = \V [x\epsilon | x] = x^2\V[\epsilon|x] = 4x^2$ --- функция от регрессора $x$. Это не так для безусловной дисперсии, где $\V[u] = \V[x\epsilon] = \E[(x\epsilon)^2]-(\E[x\epsilon])^2 = \E[x^2] \E[\epsilon^2] = \V[x] \V[\epsilon] = 100$, при условии, что $x$ независимы $\epsilon$.

Стандартные ошибки оценки МНК следует рассчитывать с помощью оценки, устойчивой к гетероскедастичности (4.21). Поскольку МНК не вполне эффективен, ВМНК может помочь увеличить эффективность. ОМНК даст выигрыш в эффективности в любом случае, и в примере с искусственно сгенерированными данными мы пользуемся преимуществом знания того, что $\V [u|x] = 4x^2$. Все способы оценивания дают несмещённые оценки свободного члена и коэффициента наклона.

Различные оценки МНК и связанные с ними стандартные ошибки из сгенерированного набора данных размером 100 приведены в Таблице 4.3. Нас интересует в первую очередь коэффициент наклона.

\begin{table}[h]
\caption{\label{tab:wls} ВМНК: пример с условно гетероскедастичными ошибками}
\begin{minipage}{\textwidth}
\begin{tabular}[t]{llll}
\hline
\hline
 & \bf{МНК}  \footnote{Сгенерированные данные для выборки размером 100. МНК, ОМНК и ВМНК все состоятельны, но МНК и ВМНК неэффективны. Даны два различных типа стандартных ошибок: в круглых скобках стандартные ошибки по умолчанию, предполагающие гомоскедастичные ошибки, а в квадратных скобках стандартные ошибки, устойчивые к гетероскедастичности. Процесс, порождающий данные, описан в тексте.} & \bf{ВМНК} & \bf{ОМНК}  \\
\hline
    Константа & 2.213 & 1.060  & 0.996 \\
          & (0.823) & (0.150) & (0.007) \\
          & [0.820] & [0.051] & [0.006] \\
    $x$ & 0.979 & 0.957 & 0.952 \\
          & (0.178) & (0.190) & (0.209) \\
          & [0.275] & [0.232] & [0.208] \\
    $R^2$ & 0.236 & 0.205 & 0.174 \\

\hline
\hline
\end{tabular}
\end{minipage}
\end{table}

Оценка МНК коэффициента наклона равна 0.979. Показаны две стандартные ошибки, и правильная, устойчивая к гетероскедастичности (рассчитанная согласно (4.21)), равна 0.275, что намного больше, чем ошибка 0.177, использующая $s^2(X'X)^(-1)$. Такая существенная разница в оценках стандартных ошибок может привести к принципиальным различиям в статистических выводах. В этом примере можно показать что теоретически, в пределе, робастные стандартные ошибки в $\sqrt{3}$ раз больше неправильных. Для конкретно этого примера с размером выборки $N$ правильные и неправильные оценки стандартных ошибок коэффициента наклона сходятся, соответственно, к $\sqrt{12 / N}$ и $\sqrt{4 / N}$. 

В качестве примера оценки ВМНК, предположим, что $u = \sqrt{|x|}\epsilon$, а не $u = x\epsilon$, то есть что $\V [u|x] = \sigma^2 |x|$. Оценку ВМНК тогда можно получить с помощью регрессии МНК, предварительно поделив $y$, свободный член и $x$ на $\sqrt{|x|}$. Так как это неверная модель гетероскедастичности, правильная (робастная) оценка стандартной ошибки для коэффициента наклона рассчитывается по формуле (4.32) и равна 0.232.

Оценку ОМНК для этой модели можно получить с помощью регрессии МНК, предварительно поделив $y$, свободный член и $x$ на $|x|$, так как в этом случае ошибки станут гомоскедастичными. Обычная и робастная оценки стандартной ошибки коэффициента наклона теперь очень похожи (0.209 и 0.208). Это ожидаемо, так как обе асимптотически несмещённые, поскольку оценка ОМНК в данном случае использует верную модель гетероскедастичности. Можно показать теоретически, что в этом случае стандартная ошибка оценки ОМНК коэффициента наклона сходится к $\sqrt{4 / N}$. 

И МНК, и ВМНК менее эффективны, чем ОМНК, как и ожидалось, со стандартными ошибками коэффициента наклона $0.275 >  0.232 > 0.208$.

Модель в данном примере похожа на стандартные модели, использующиеся при оценивании пространственных данных. И $y$ и $x$ --- случайные величины.  Пары $(y_i, x_i)$ независимы по $i$ и одинаково распределены, как при случайной выборке. Однако условное распределение $y_i|x_i$ различается для разных $i$, поскольку условные математическое ожидание и дисперсия $y_i$ зависят от $x_i$.

\section{Медианная и квантильная регрессия}

В модели с одним только свободным членом описательные статистики для выборочного распределения кроме выборочного среднего включают квантили, такие, как медиана, верхний и нижний квартили и процентили.

В контексте регрессии нас могут подобным образом интересовать условные квантили. Например, интерес может представлять то, насколько проценитили распределения доходов низкообразованных работников более сжаты, чем у работников с высоким уровнем образования. В этом простом примере можно просто разделить вычисления для низко- и высокообразованных работников. Однако этот подход становится неприменимым, когда есть несколько регрессоров, каждый из которых может принимать нескко различных значений. Вместо этого требуется использовать методы квантильной регрессии, чтобы оценить квантили условного распределения $y$ при условии $x$.

Согласно Таблице 4.1, квантильной регрессии соостветсвует функция потерь с асимметричным модулем ошибки, а в специальном случае медианной регрессии --- с модулем ошибки. Эти методы являются альтернативой МНК, пользующимся квадратичной функцией потерь.

Методы квантильной регрессии имеют другие преимущества помимо предоставления более богатой характеристики данных. Медианная регрессия более устойчива к выбросам, чем регрессия МНК. Более того, оценки квантильной регрессии могут быть состоятельными при более слабых предположениях, чем это возможно для оценок МНК. Ведущими примерами являются оценка максимальных maximum score Мански (1975) для моделей с бинарной зависимой переменной (см. Раздел 14.6) и цензурированная оценка наименьших модулей Powell (1984) для цензурированных моделей (см. Раздел 16.6).

Мы начнём с краткого объяснения квантилей в генеральной совокупности, а после перейдём к оценке выборочных квантилей.

\subsection{Квантили в генеральной совокупности}

Для непрерывной случайной величины $y$ $q$-той квантилью в генеральной совокупности называется число $\mu_q$, такое, что $y$, меньший или равен $\mu_q$ с вероятностью $q$. То есть
$$
q = \mathrm{Pr}[y \leq \mu_q] = F_y(\mu_q),
$$
где $F_y$ - кумулятивная функция распределения (cumulative distribution function, cdf) $y$. Например, если $\mu_{0.75}=3$, то вероятность, что $y\leq 3$ равна $0.75$. Следовательно,
$$
\mu_q = F_y^{-1}(q).
$$
Популярными примерами являются медиана, $q=0.5$, верхняя квартиль, $q=0.75$, и нижняя квартиль, $q=0.25$. Для стандартного нормального распределения $\mu_{0.5}=0$, $\mu_{0.95}=1.645$, а $\mu_{0.975}=1.960$.  $100q$-той \textbf{процентилью} называется $q$-тая квантиль.

В регрессионной модели \textbf{$q$-той квантилью в генеральной совокупности} для $y$ при условии $x$ называют функцию $\mu_q(x)$, такую, что при условии $x$ вероятность, что $y$ не превышает  $\mu_q$, равна $q$, где эта вероятность вычислена с помощью условного распределения $y$ при данном $x$. Тогда
\begin{equation}
\mu_q(x) = F^{-1}_{y|x}(q),
\end{equation}
где $F_{y|x}$ --- условная функция распределения $y$ при $x$, и мы опустили роль параметров в этом распределении.

Познавательным будет вывести квантильную функцию $\mu_q(x)$ при предположении, что $y$ порождён линейной моделью с мультипликативной гетероскедастичностью
\[
\begin{array}{c} 
y = x'\beta + u \\ 
u = x'\alpha \times \epsilon \\
\epsilon \sim\mathrm{iid} [0, \sigma^2],
\end{array}
\]
где предполагается, что $x'\alpha > 0$. Тогда $q$-тая квантиль в генеральной совокупности $y$  при условии $x$ равна такой функции $\mu_q(x,\beta,\alpha)$, что
\[
\begin{array}{rcl} 
q & = & \mathrm{Pr} [y\leq \mu_q(x,\beta,\alpha)] \\
& = & \mathrm{Pr} [u \leq \mu_q(x,\beta,\alpha) - x\beta] \\
& = & \mathrm{Pr} [\epsilon \leq [\mu_q(x,\beta,\alpha) - x\beta]/x'\alpha] \\
& = & F_{\epsilon} ([\mu_q(x,\beta,\alpha) - x\beta]/x'\alpha),
\end{array}
\]
где мы используем $u = y-x'\beta$ и $\epsilon = u/x'\alpha$, а $F_{\epsilon}$ --- функция распределения $\epsilon$. Следовательно, $[\mu_q(x,\beta,\alpha) - x\beta]/x'\alpha =F^{-1}_q$, и 

\[
\begin{array}{rcl}
\mu_q(x,\beta,\alpha) & = & x'\beta +x'\alpha \times F^{-1}_q \\
& = & x' (\beta + \alpha \times F^{-1}_q).
\end{array}
\]

Таким образом, в линейной модели с мультипликативной гетероскедастичностью в форме $u = x'\alpha \times \epsilon$ условные квантили также линейны по $x$. В особом случае гомоскедастичности $x'\alpha$ константа и все условные квантили имеют один и тот же наклон и различаются только свободным членом, который увеличивается при увеличении $q$.

В более общем случае квантильная функция может быть нелинейной по $x$ благодаря другим формам гетероскедастичности, таким, как $u=h(x,\alpha) \times \epsilon$, где $h(\cdot)$ нелинейна по $x$, или потому что сама регрессионная функция имеет нелинейную форму $g(x, \beta)$. Однако принято оценивать линейные квантильные функции и интерпретировать их как наилучшие линейные предикторы при функции ошибок квантильной регрессии, приведённой в (4.34) в следующем разделе.  

\subsection{Квантили в выборке}

Для одномерной случайной величины $y$ обычном способом вычисления квантилей является сортировка выборки по возрастанию. Тогда $\hat{\mu_q}$ равняется $[Nq]$-тому наименьшему значению, где $N$ --- размер выборки, а  $[Nq]$ обозначает $Nq$, округлённое вверх до ближайшего целого. Например, если $N=97$, нижним квартилем считается 25-тое наблюдение, поскольку $[97 \times 0.25] =[24.25] = 25$.

Коэнкер и Basset (1978) заметили, что $q$-тую \textbf{выборочную квантиль} $\hat{\mu}_q$ можно выразить эквивалентно как решение проблемы минимизации по $\beta$
$$\sum_{i:y_i \geq x_i'\beta}^{N} q|y_i-\beta| + \sum_{i:y_i < x_i'\beta}^{N} (1-q)|y_i-\beta|. $$
Результат не очевиден. Чтобы осознать его, рассмотрим медиану, для которой $q=0.5$. Тогда медиану можно найти как минимум $\sum_i |y_i -\beta|$. Представим, что в выборке из 99 наблюдений 50-е по возрастанию наблюдение, медиана, равно 10, а 51-е наблюдение равно 12. Если мы установим $\beta$ 12, а не 10, то $\sum_i |y_i -\beta|$ увеличится на 2 для меньших 50 наблюдений и уменьшится на 2 для оставшихся 49 наблюдений, то есть в целом увеличится на $50 \times 2 - 49 \times 2 = 2$. Таким образом, 51-е наблюдение --- худший выбор, чем 50-е. Аналагично можно показать, что 49-е наблюдение также является худшим выбором, чем 50-е.

Эту целевую функцию легко обобщить на случай линейной регрессии, так, что $q$-тая \textbf{оценка квантильной регрессии} $\hat{\beta}_q$ минимизирует
\begin{equation}
Q_n(\beta_q) = \sum_{i:y_i \geq x_i'\beta_q}^{N} q|y_i-x_i'\beta_q| + \sum_{i:y_i < x_i'\beta_q}^{N} (1-q)|y_i-x_i'\beta_q|,
\end{equation}
где мы используем $\beta_q$ вместо $\beta$, чтобы показать явно, что различный выбор $q$ приводит к различным оценкам $\beta$. Заметим, что это функция потерь с асимметричным модулем, приведнная в Таблице 4.1, где $\hat{y}$ ограничен до линейного по $x$, так, что $e = y -x'\beta$.  Особый случай $q=0.5$ назвается \textbf{оценкой медианной регрессии} или \textbf{оценкой наименьших модулей отклонений}. 

\subsection{Свойстав оценок квантильной регрессии}

Целевая функция (4.34) не является дифференцируемой, и потому градиентные методы оптимизации, описанные в Главе 10, не применимы к ней. К счастью, можно использовать методы линейного программирования, и они делают возможным относительно быстрое вычисление $\hat{\beta}_q$.

Так как явного выражения для $\hat{\beta}_q$ не существует, определить асимптотическое распределение $\hat{\beta}_q$, пользуясь подходом, применённым в Разделе 4.4 для МНК, невозможно. Методы Главы 5 также требуют адаптации, поскольку целевая функция недифференцируема. Можно показать, что 
\begin{equation}
\sqrt{N}(\hat{\beta}_q-\beta_q)  \xrightarrow{d} \mathcal{N} [0, A^{-1}BA^{-1}],
\end{equation}
(см. например Buchinsky, 1998, стр.85), где 
\begin{equation}
A = \plim \frac{1}{N} \sum_{i=1}^{N} f_{u_{q}}(0|x_i)x_i'x_i, \\
B = \plim \frac{1}{N} \sum_{i=1}^{N}q(1-q)x_i'x_i, \\
\end{equation}
а $f_{u_{q}}(0|x_i)$ --- условная плотность распределения случайного члена $u_q = y -x'\beta_q$, оцененная в $u_q = 0$. Оценка дисперсии $\hat{\beta}_q$ затруднена необходимостью оценки  $f_{u_{q}}(0|x_i)$. Проще оценивать стандартные ошибки $\hat{\beta}_q$, используя методы парного бутстрэпа, описанные в Главе 11.

\subsection{Пример квантильной регрессии}

В этом разделе мы производим оценку условных квантилей и сравниваем её с обычной оценкой условного математического ожидания, использующей регрессию МНК. В этом приложении оценивается кривая Энгеля для годовых расходов домохозяйств на медицину. Более конкретно, мы рассматриваем регрессионное соотношение между логарифмом медицинских расходов и логарифмом совокупных расходов домохозяйств. Таким образом, эта регрессия оценивает (постоянную) эластичность медицинских расходов по совокупным расходам.

Данные взяты из вьетнамского исследования стандартов жизни, проведённого Мировым банком в 1997 году. Выборка состоит из 5006 домохозяйств, понесших положительные расходы на медицину, после исключения 16.6\% выборки, которые имели нулевые расходы, чтобы было возможно взятие логарифма. Нулевые значения можно обрабатывать с помощью методов цензурированной квантильной регрессии, разработанного Powell (1986a), представленных в Разделе 16.9.2. Для простоты мы не использовали эти наблюдения в расчётах. Наибольшая доля медицинских расходов, особенно при низких уровнях дохода, состоит из лекарств, купленных в аптеках. Хотя доступны различные характеристики домохозяйств, для простоты мы рассмотрим только один регрессор, логарифм совокупных расходов домохозяйства, в качестве прокси для его совокупного дохода. 

Линейная регрессия МНК даёт оценку эластичности, равную 0.57. Эту оценку обычно интерпретировали бы в том смысле, что лекарства являются <<товаром первой необходимости>>, а потому спрос на них не эластичен по доходу. Эта оценка не является неожиданной, но прежде чем принять её на веру, мы должны допустить, что между различными группами по доходу эластичность может быть неоднородна.

Квантильная регрессия является полезным инструментом для изучения такой неоднородности, как подчёркивают Коэнкер и Hallock (2001). Мы минимизируем функцию (4.34), где $y$ --- логарифм медицинских расходов, а $x'\beta = \beta_1 +\beta_2 x$, где $x$ --- логарифм совокупных расходов домохозяйства. Это было сделано для 19 квантилей $q = {0.05, 0.10,..., 0.95}$, где $q = 0.5$ --- медиана. Результаты этого упражнения помещены на Рис. 4.1 и Рис. 4.2. 
\begin{figure}[t]
 \caption{Оценки квантильной регрессии для коэффициента наклона и $q = 0.05,0.10,...0.95$ и соответствующие 95\% доверительные интервалы, нарисованные в зависимости от $q$, из регрессии логарифма медицинских расходов на логарифм совокупных расходов.}
\end{figure}
Рисунок 4.1 изображает коэффициент наклона $\hat{\beta}_2$ для различных значений $q$ и соответствующий ему доверительный интервал. Это показывает, как квантильная оценка элатстичности изменяется в зависимости от $q$. Оценка эластичности систематически увеличивается с ростом дохода домохозяйста, поднимаясь от 0.15 при $q=0.05$ до максимума в $0.80$ при $q=0.85$. МНК-оценка наклона, равная 0.57, также представлена в виде горизонтальной линии, не зависящей от квантили. Оценки эластичности на верхних и нижних квантилях определённо статистически отличаются друг от друга и от оценки МНК, которая имеет стандартную ошибку 0.032. Кажется вероятным, что агрегированная оценка эластичности будет меняться соответственно изменениям в распределении дохода. Этот график поддерживает наблюдения Mosteller и Tukey (1977, стр.236), цитируемые Koeker и Hallock (2001), о том, что, фокусируясь только на функции условного среднего, регрессия МНК предоставляет неполное описание совместного распределения зависимой и объясняющей переменных.

\begin{figure}[t]
 \caption{Графики оценки квантильной регрессии для $q=0.1, q=0.5$ и $q=0.9$ из регрессии логарифма медицинских расходов на логарифм совокупных расходов. Данные по 5006 вьетнамским домохозяйствам с положительными медицинскими расходами в 1997 году.}
\end{figure}

Рисунок 4.2 совмещает три оцененные линии квантильной регрессии $\hat{y}_q = \hat{\beta}_{1,q}+\hat{\beta}_{2,q}x$ для $q=0.1, 0.5, 0.9$ и линию регрессии МНК. Линия регрессии МНК полностью совпадает с медианной ($q=0.5$) линией регрессии. На рисунке 4.2 заметно расхождение линий квантильных регрессий при росте $q$. Это не удивительно, если оценки наклона увеличиваются, как это видно по Рис. 4.1. Коэнкер и Basset (1982) разработали квантильную регрессию как средство тестирования ошибок на гетероскедастичность, когда процесс, порождающий данные, является линейной моделью. В этом случае расхождение линий квантильных регрессий рассматривается как свидетельство гетероскедастичности. Другая интерпретация состоит в том, что условное математическое ожидание нелинейно по $x$, а, напротив, наклон возрастает, и это приводит к возрастанию квантильных коэффициентов наклона по $q$.

Более подробная информация о квантильных регрессиях дана в Buchincky (1994) и Коэнкер и Hallock (2001).

\section{Ошибки спецификации модели}

Термин <<ошибки спецификации>> в самом широком смысле обозначает, что как минимум одно предположение о процессе, порождающем данные, некорректно. Ошибки спецификации могут происходить индивидуально или в комбинации, но анализ упрощается, если рассматривать последствия только одной ошибки.

В нижеследующем обсуждении мы подчёркиваем ошибки спецификации, которые приводят к несостоятельности оценок МНК и потере идентифицируемости параметров, представляющих интерес. Оценка МНК, тем не менее, может по-прежнему обладать содержательной интерпретацией, однако отличной от той, которая предполагалась для корректно специфицированной модели. В частности, оценка может асимптотически сходиться к значению, отличному от истинного значения в генеральной совокупности, --- понятие, определённое в Разделе 4.7.5 как псевдоистинное значение.

Проблемы, поднятые здесь относительно состоятельности МНК, имеют отношение и к оценкам других моделей. Для последних состоятельность может требовать более сильных допущений, чем требуются для состоятельности МНК, и, следовательно, несостоятельность этих моделей более вероятна. 

\subsection{Несостоятельность МНК}

Наиболее серьёзным последствием ошибок спецификации является несостоятельная оценка параметров регрессии $\beta$. Из Раздела 4.4 мы знаем, что двумя ключевыми условиями, требующимися для состоятельности оценки МНК, являются (1) данные порождены процессом $y = X\beta+u$ и (2) процесс, порождающий данные, такой, что $\plim N^{-1}X'u=0$. Тогда
\begin{equation}
\hat{\beta}_{\text{МНК}}=\beta + (N^{-1}X'X)^{-1}N^{-1}X'u \xrightarrow{p} \beta,
\end{equation}
где первое равенство следует из  $y = X\beta+u$ (см. (4.12)), а второе (сходимость) использует $\plim N^{-1}X'u=0$.

Оценка МНК, скорее всего, будет \textbf{несостоятельна}, если ошибки спецификации модели приводят либо к неправильной спецификации модели для $y$, нарушая условие (1), или ошибки коррелируют с регрессорами, так, что нарушается условие (2).

\subsection{Ошибки спецификации функциональной формы}

Линейная спецификация функции условного математического ожидания является не более чем приближением в $\mathcal{R}^K$ истинной неизвестной функции условного математического ожидания в пространстве параметров неопределённой размерности. Даже если выбраны корректные регрессоры, это ещё не означает, что условное математическое ожидание специфицировано корректно.

Предположим, что процесс, порождающий данные --- нелинейная регрессионная функция 
$$ y = g(x) + \upsilon, $$
где зависимость $g(x)$ от неизвестных параметров скрыта, и предположим, что $\E[u|X] = 0$. Модель линейной регрессии
$$ y = x'\beta +u$$
специфицирована некорректно. Вопрос состоит в том, можно ли дать оценке МНК какую-либо содержательную интерпертацию, даже если модель данных в действительности нелинейна.

Обычным способом интерпретации коэффициентов регрессии является истинная \textit{микро взаимосвязь}, в данном случае ---
$$\E[y_i|x_i] = g(x_i).$$
Тогда $\hat{\beta}_{\text{МНК}}$ не измеряет влияния на микроуровне изменений в $x_i$ на $\E[y_i|x_i]$, так как она не сходится к $\partial g(x_i)/\partial x_i$. Поэтому обычная интерпретация $\hat{\beta}_{\text{МНК}}$ невозможна.

White (1980b) показал, что оценка МНК сходится к такому значению $\beta$, которое минимизирует среднеквадратическую ошибку предсказания
$$\E_x[(g(x)-x'\beta)^2].$$
Следовательно, оценка МНК является наилучшим линейным предиктором для нелинейной функции регрессии, если использовать среднеквадратичную функцию потерь. Это полезное свойство уже было отмечено в Разделе 4.2.3, но оно мало добавляет к интерпретации $\hat{\beta}_{\text{МНК}}$.

В целом, если истинная функция регрессии нелинейна, МНК бесполезна для индивидуальных предсказаний. МНК по-прежнему можно использовать для предсказания агрегированных изменений, так как он даёт среднее по выборке изменений $\E [y|x]$ вследствие изменения $x$ (см. Stoker, 1982). Однако микроэконометрический анализ обычно стремится найти модели, содержательные на микроуровне.

Значительная часть этой книги представляет альтернативы линейной модели, которые с большей вероятностью будут специфицированы корректно. Например, Глава 14 о бинарных зависимых переменных представляет спецификации модели, которые гарантируют, что предсказанная вероятность будет лежать в пределах между 0 и 1. Также предпочтительными могут быть модели, которые опираются на минимальные допущения о распределении, так как они оставляют меньше места для ошибок спецификации.

\subsection{Эндогенность}
Эндогенность формально определена в Разделе 2.3. В широком смысле говорят, что регрессор эндогенный, если он коррелирует со случайным членом. Если какой-либо из регрессоров эндогенный, то в общем случае МНК-оценки всех параметров регрессии несостоятельны (кроме случев, когда экзогенный регрессор некоррелирован с эндогенным регрессором). 

Яркиме примеры эндогенности, которые в этой книге подробно рассмотрены в контексте как линейных, так и нелинейных моделей, включают смещение в одновременных уравнениях (Раздел 2.4), ошибку пропущенных переменных (Раздел 4.7.4), ошибку неслучайного отбора (Раздел 16.5), и ошибки измерения (Глава 26). Эндогенность часто встречается при использовании пространственных данных, и экономисты весьма обеспокоены этим затруднением.

Довольнм общим подходом к учёту эндогенности является метод инструментальных переменных, представленный в Разделах 4.8 и 4.9 и в Разделах 6.4 и 6.5.  Этот метод, однако, не всегда можно применить, так как необходимые инструменты не всегда доступны.

Другие методы учёта эндогенности, перечисленные в Разделе 2.8, включают контроль смешаных параметров, модель <<разность разностей>> (если доступны повторные пространственные или панельные данные --- см. Главу 21), фиксированные эффекты (если доступны панельные данные, а причинной эндогенности является пропущенная переменная, не меняющаяся во времени --- см. Раздел 21.6), и разрывную модель регрессии (см. Раздел 25.6).

\subsection{Пропущенные переменные}
\textit{Пропуск переменной} в модели линейной регрессии часто приводят в качестве первого примера несостоятельности оценки МНК, приводимого во вводных курсах. Такой пропуск может быть следствием ошибочного исключения переменной, данные по которой доступны, или исключения переменной, которая напрямую не наблюдаема. Например, пропуск способостей в регрессии заработной платы (или, что более типично, её логарифма) на образование обычно объясняется недоступностью полноценной меры способностей.

Пусть процесс, порождающие данные, будет 
\begin{equation}
y = x'\beta + z\alpha + \upsilon,
\end{equation}
где $x$ и $z$ --- регрессоры, $z$, для простоты, скаляр, а $\upsilon$ --- случайный член, который мы полагаем некоррелированным с $x$ и $z$. Регрессия МНК $y$ на $x$ и $z$ оценит параметры $\beta$ и $\alpha$ состоятельно. 

Предположим, однако, что $y$ регрессируется на один только $x$, а $z$ опущен вследствие недоступности. Тогда слагаемое $z\alpha$ становится частью <<случайного>> члена. Оценивается модель
\begin{equation}
y = x'\beta + (z\alpha + \upsilon),
\end{equation}
где ошибкой является $(z\alpha + \upsilon)$. Как и раньше,$\upsilon$  некоррелирован с $x$, но если $z$ коррелирует с $x$, то ошибка $(z\alpha + \upsilon)$  также коррелирует с регрессором $x$. Таким образом, оценка $\beta$ с помощью МНК будет несостоятельна, если $z$ коррелирует с $x$.

В этом примере достаточно структуры, чтобы определить направление смещения. Собрав все наблюдения в одну матрицу, получим процесс, порождающий данные $y = X'\beta +z\alpha + v$. Подставив это в $\hat{\beta}_{\text{МНК}} = (X'X)^{-1}X'y$, получим
$$
\hat{\beta}_{\text{МНК}} = \beta + (N^{-1}X'X)^{-1} (N^{-1}X'z)\alpha + (N^{-1}X'X)^{-1} (N^{-1}X'v).
$$
При обычной предпосылке о некоррелированности $X$ и $v$, последнее слагаемое сходится по вероятности к нулю. Однако $X$ коррелирован с $z$, и 
\begin{equation}
\plim \hat{\beta}_{\text{МНК}} = \beta +\delta \alpha,
\end{equation}
где 
$$
\delta = \plim [(N^{-1}X'X)^{-1} (N^{-1}X'z)]
$$
является пределом по вероятности оценки МНК регрессии пропущенного регрессора ($z$) на оставшиеся регрессоры ($X$).

Эта несостоятельность называется \textbf{смещением пропущенной переменной}, где общепринятая терминология называет последствия различных ошибок спецификации смещением, тогда как формально они приводят к несостоятельности. Несостоятельность существует, пока $\delta \neq 0$, то есть пока пропущенная переменная коррелирует со включёнными регрессорами. В общем случае несостоятельность может быть как положительной, так и отрицательной, и может даже приводить к смене знака коэффициента МНК.

Схожей ошибкой спецификации является \textbf{включение неуместных регрессоров}. Например, можно построить регрессию. $y$  на $x$ и $z$, тогда как истинная модель данных просто $y = x'\beta +u$. В этом случае несложно показать, что оценка МНК состоятельна, но имеет место потеря эффективности. 

Учёт смещения пропущенных переменных необходим, если оценкам параметров приписывается причинно-следственная интерпретация. Поскольку лишние регрессоры причиняют относительно мало вреда, но исключение нужных регрессоров может приводить к несостоятельности, микроэконометрические модели, оцениваемые на больших выборках, обычно включают много регрессоров. Если некоторые переменные по-прежнему опущены, необходимо использовать один из методов, приведённых в конце Раздела 4.7.3.

\subsection{Псевдоистинное значение}
В примере с пропущенными значениями оценка МНК подвергается \textit{смешиванию} в том смысле, что она оценивает не $\beta$, а вместо этого некую функцию $\beta$, $\delta$ и $\alpha$. 

Оценку МНК нельзя использовать в качестве оценки $\beta$, которая, например, измеряет эффект от экзогенного изменения регрессора $x$, такого, как образование, при неизменных остальных переменных, включая ненаблюдаемые способности.

Из (4.40), однако, следует, что $\hat{\beta}_{\text{МНК}}$ --- состоятельная оценка функции $(\beta +\delta \alpha)$ и имеет содержательную интерпретацию. Предел по вероятности $\hat{\beta}_{\text{МНК}}$, равный $\beta{*} = (\beta +\delta \alpha)$ называют \textbf{псевдоистинным значением} (см. формальное определение в Разделе 5.7.1), соответствующим $\hat{\beta}_{\text{МНК}}$

Более того, можно найти распределение $\hat{\beta}_{\text{МНК}}$, несиотря на то, что эта оценка несостоятельна относительно $\beta$. Оценка асимптотической ковариацонной матрицы для $\hat{\beta}_{\text{МНК}}$ измеряет дисперсию вокруг $(\beta +\delta \alpha)$ и рассчитывается с помощью обычной оценки, например, $s^2 (X'X)^{-1}$, если ошибка в (4.38) гомоскедастична.

\subsection{Неоднородность параметров}
До сих пор наш текст допускал различие регрессоров и случайных ошибок для разных индивидов, но требовал, чтобы параметры регрессии $\beta$ были одинаковыми для всех индивидов. 

Предположим вместо этого, что процесс, порождающий данные
\begin{equation}
y_i = x_i'\beta_i + u_i, 
\end{equation}
с индексом $i$ при параметрах. Это пример \textbf{неоднородности параметров}, где предельный эффект $\E[\partial y_i/\partial x_i| x_i]=\beta_i$ теперь может различаться у разных индивидов.

\textbf{Модель со случайными коэффициентами}, или \textbf{модель с рандомными параметрами}, специфицирует $\beta_i$ как независимые и одинаково распределённые случайные величины, распределение которых не зависит от $x_i$. Обозначим за $\beta$ общее математическое ожидание $\beta_i$. Тогда модель данных можно переписать как 
$$
y_i = x_i'\beta + (u_i +x'i(\beta_i -\beta)),
$$
и сделанных предположений достаточно, чтобы регрессоры $x$ были некоррелированы с ошибкой $(u_i +x'i(\beta_i -\beta))$. Регрессия МНК $y$ на $x$, таким образом, сможет состоятельно оценить $\beta$, но заметим, что в этой модели ошибки гетероскедастичны, хотя $u_i$ гомоскедастичны.

Для панельных данных стандартом является модель случайных эффектов (см. Раздел 21.7), где свободный член может быть различным для разных индивидов, а коэффициенты наклона не случайны.

Для нелинейных моделей подобного результата не существует, и можно предпочесть модели со случайными параметрами, так как они делают возможной более богатую параметризацию. Модели с рандомными параметрами согласуются со существованием неоднородной реакции индивидов на изменения $x$. Ярким примером является логит со случайными параметрами в Разделе 15.7.

Более серьёзные затруднения могут возникнуть, когда параметры регрессии $\beta_i$ связаны с наблюдаемыми характеристиками индивидов. Примером является модель с фиксированными эффектами для панельных данных (см. Раздел 21.6), для которых оценка регрессии МНК $y$ на $x$ несостоятельна. В этом примере, но не во всех таких примерах, доступны альтернативные оценки для подмножества параметров регрессии, являющиеся состоятельными.

\section{Инструментальные переменные}

Основнм затруднением, подчёркиваемым в микроэконометрике, явялется возмножность несостоятельной оценки параметров вследствие эндогенности регресоров. В этом случае регрессия измеряет только степень ассоциации между переменными, а не силу и направление причинно-следственной связи, что требуется для анализа политик.

Оценка методом инструментальных переменных, тем не менее, предоставляет способ получения состоятельных оценок параметров. Этот метод, широко используемый в микроэконометрике и редко используемый где-либо ещё, концепциально сложен, и им часто злоупотребляют.

Мы представляем a lengthy expository treatment, где определяется инструментальная переменная и объясняется, как метод инструментальных переменных работает в простой задаче.

\subsection{Несостоятельность МНК}

Рассмотрим скалярную модель регрессии с зависимой переменной $y$ и единственным регрессором $x$. Целью регрессионного анализа является оценка функции условного математичского ожидания $\E[y|x]$. Линейная модель условного среднего (без свободного члена ради удобстваобозначений) специфицирует:
\begin{equation}
\E[y|x] = \beta x.
\end{equation}

Эта модель без свободного члена может заменить модель со свободным членом, если зависимая и объясняющая переменные выражены в отклонениях от своих средних значений. Интерес представляет получение состоятельной оценки $\beta$, которая отражает изменение в условном математическом ожидании вследствие \textit{экзогенного} изменения $x$. Например, интерес может представлять изменение доходов, вызванное увелиением образования по экзогенным причинам, таким, как увеличение минимального возраста, в котором ученики могут закончить школу, которые не являются выбором индивида.

Модель регрессии МНК специфицирует 
\begin{equation}
y = \beta x +u,
\end{equation}
где $u$ является случайной ошибкой. Регрессия $y$ на $x$ даёт $\hat{\beta}$, МНК оценку $\beta$. 

В стандартных результатах о регрессии делается предположение, что регрессоры некоррелированы со случайным членом в модели (4.4.3). Тогда всё воздействие $x$ на $y$ является прямым эффектом через слагаемое  $\beta x$. У нас есть следующая диаграмма path analysis:
\begin{figure}
\end{figure}
где между $x$ и $y$ нет взаимосвязи. Таким образом, $x$ и $u$ являются независимыми факторами, влияющими на $y$.

Однако в нектороых ситуациях может существовать ассоциация между регрессором и случайным членом. Например,  рассмотрим регрессию логарифма заработной платы $y$ на годы образования $x$. Ошибка $u$ включает все факторы, кроме образования, которые определяют заработную плату, такие, как способности индивида. Рассмотрим индивида, имеющего высокий уровень $u$ в результате выдающихся (ненаблюдаемых) способностей. Это приведёт к увеличению заработной платы, поскольку $y = \beta x +u$, но это также может привести к увеличению уровня $x$, так как образование у тех, кто обладает высокими способностями, может длиться дольше. Более подходящей диаграммой причинно-следственных связей  тогда будет следующая:
\begin{figure}
\end{figure} 
где теперь есть ассоциация между $x$ и $u$.

Каковы последствия этой корреляции между  $x$ и $u$? Теперь высокий уровень $x$ связан c $y$ двумя путями. Из (4.4.3), есть как прямой эффект через $\beta x$, так и косвенный эффект через $u$, влияющий на $x$, который, в свою очередь, влияет на $y$. Целью регрессии является оценка только первого эффекта, чтобы получить оценку $\beta$. Вместо этого оценка МНК совместит оба эффекта, оценив $\hat{\beta}>\beta$ в данном конкретном примере, где оба эффекта положительны. Используя математический анализ, для выражения $y = \beta x +u(x)$ можно получить полый дифференциал:
\begin{equation}
\frac{dy}{dx} = \beta + \frac{du}{dx}.
\end{equation}
Из данных можно извлечь информацию о $\frac{dy}{dx}$, и потому МНК измеряет совокупный эффект $\beta + \frac{du}{dx}$, а не $\beta$ саму по себе. Таким образом, оценка МНК является смещённой и несостоятельной оценкой $\beta$, если между $x$ и $u$ существует взаимосвязь.

Более формальное рассмотрение модели линейной регрессии с $K$ регрессорами приводит к тем же выводам. Согласно Разделу 4.7.1, неоходимым условием состоятельности МНК является $\plim N^{-1}X'u=0$. Для состоятельности требуется, чтобы регрессоры были асимптотически некоррелиорованы с ошибками. Из (4.37) величина несостоятельности равна $(X'X)^{-1}X'u$, коэффициент МНК из регрессии $u$ на $X$. Это просто оценка МНК величины $\frac{du}{dx}$, что подтверждает наш интуитивный результат в (4.44).

\subsection{Инструментальные переменные}

Несостоятельность МНК обусловлена эндогенностью $x$, в том смысле, что изменения $x$ связаны не только с изменениями $y$, но и с изменениями в ошибке $u$. То, что нам необходимо, это метод генерации только экзогенных изменений в $x$. Очевидным способом является рандомизированный эксперимент, однако в большей части экономических приложений такие эксперименты чересчур дороги или вообще невозможны.
\begin{center}
Определение инструмента
\end{center}
Чистый экспериментальный подход вс же возможен при использовании данных, полученных из наблюдений, если существует \textbf{инструмент} $z$, обладающий свойством, что изменения $z$ связаны с изменениями $x$, но не приводят напрямую к изменениям $y$ (а только косвенно, через изменение $x$). Это приводит к следующей диаграмме:
\begin{figure}
\end{figure}
которая вводит переменную $z$, причинно связанную с $x$, но не с $u$. Всё ещё вероятно, что $z$ и $y$  коррелируют, однако единственным источником этой корреляции является непрямой путь через корреляцию $z$ с $x$, который, в свою очередь, коррелирует с $y$. Более прямой путь, где $z$ является регрессором в модели для $y$, исключён.

Более формально, $z$ называют \textbf{инструментом} или \textbf{инструментальной переменной}  (instrumental variable) для регрессора $x$ в скалярной регрессионной модели $y = \beta x +u $, если (1) $z$ не коррелирует с ошибкой $u$ и (2) $z$ коррелирует с регрессором $x$.

Первое предположение исключает возможность того, что $z$ сам является регрессором в модели для $y$, так как если бы $y$ зависел и от $x$, и от $z$, но регрессия была бы построена только на $x$, то $z$ был бы включён в ошибку и потому коррелировал бы с ней.  Второе предположение требует, чтобы между инструментом и переменной, к которой он применяется, существовала связь
\begin{center}
Примеры инструментов
\end{center}
Во многих микроэконометрических приложениях сложно найти обоснованные инструменты. Здесь мы приводим два примера таковых.

Представим, что мы хотим измерить реакцию рыночного спроса на экзогенные изменения в рыночной цене. Величина спроса, очевидно, зависит от цены, однако цены не заданы экзогенно, так как они частично определяются рыночным спросом. Приемлемый инструмент для цены должен быть переменной, которая коррелирует с ценой, но напрямую не влияет на величину спроса. Очевидными кандидатами являются переменные, влияющие на рыночное предложение, так как они также влияют на цены, но не являются прямыми факторами спроса. Примером может служить мера благоприятности погодных условий, если моделируется сельскохозяйственный продукт. Такой выбор инструмента безукоризнен, если погодные условия не влияют напрямую на спрос, и пользуется поддержкой формальной экономической модели спроса и предложения.

Далее представим, что мы хотим измерить отдачу от экзогенных изменений в количестве образования. В большей части данных, полученных в результате наблюдения, отсутствует мера индивидуальных способностей. Поэтому регрессия зарплаты на образование имеет ошибку, в которую включены ненаблюдаемые способности, и потому коррелирующую с регрессором --- образованием. Нам нужен инструмент $z$, который коррелирует с образованием, не коррелирует со способностями и с ошибкой вообще, а значит, не может напрямую влиять на зарплату.

Одним из популярных кандидитов на роль $z$ является близость колледжа или университета (Card, 1995). Она, очевидно, удовлетворяет условию 2, поскольку люди, которые живут на большом расстоянии от муниципального колледжа или государственного университета, с меньшей вероятностью будут их посещать. Скорее всего, она удовлетворяет и условию 1, хотя можно возразить, что люди, живущие далеко от образовательных учреждений, вероятно, находятся на рынках труда с низкими зарплатами. Поэтому необходимо оценивать множественную регрессию $y$ включающую дополнительные переменные, такие, как индикатор столичного региона.

Вторым кандидатом на роль инструмента является месяц рождения (Angrist и Krueger, 1991). Он явно удовлетворяет условию 1, так как нет причин полагать, что заработная плата напрямую может зависеть от месяца рождения, если в регрессию включён возраст в годах. Удивительно, но условие 2 также может быть удовлетворено, так как в США месяц рождения определяет возраст поступления в школу, который, в свою очередь, может влиять на годы обучения, так как законы часто определяют минимальный возраст окончания школы. Bound, Jaeger и Baker (1995) предоставляют критический анализ этого инструмента.

Последствия выбора плохих инструментов подробно рассмотрены в Разделе 4.9.

\subsection{Оценка методом инструментальных переменных}

Для регрессии со скалярным регрессором $x$ и скалярным инструментом $z$ \textbf{оценка метода инструментальных переменных} (instrumental variable, IV) определена как
\begin{equation}
\hat{\beta}_{IV} = (z'x)^{-1}z'y,
\end{equation}
где, в случае скалярных регрессоров, $x$, $y$ и $z$ --- векторы размером $N \times 1$. Этот метод предоставляет состоятельную оценку коэффициента наклона $\beta$ в линейной модели $y = \beta x +u$, если $z$ коррелирует с $x$ и не коррелирует с ошибкой регрессии.

Есть несколько способов вывода (4.45). Мы предоставляем интуитивный вывод, отличающийся от тех, которые обычно предоставляют, таких, как в Разделе 6.2.5.

Вернёмся к примеру с образованием и доходом. Предположим, что увеличение инструмента $z$ на единицу соответствует  увеличению длительности образования в среднем на 0.2 года и с возрастанию годового дохода на \$500. Это увеличение дохода является следствием косвенного воздействия $z$ через увеличение образование, которое привело к росту дохода. Тогда можно считать, что увеличение образования на 0.2 года связано с возрастанием дохода на \$500, то есть увеличение образования на один год соответствует росту дохода на $\$500/0.2 = \$2500$. Говоря языком математики, мы оценили изменения $\frac{dx}{dz}$ и $\frac{dy}{dz}$ и вычислили причинно-следственную оценку как
\begin{equation}
\hat{\beta}_{IV} = \frac{dy/dz}{dx/dz}.
\end{equation}
Этот подход к идентификации причинно-следственного параметра $\beta$ описан в Heckman (2000, стр.58), см. также пример в Разделе 2.4.2.

Остаётся только состоятельно оценить $\frac{dx}{dz}$ и $\frac{dy}{dz}$. Очевидным способом оценки $\frac{dy}{dz}$ является регрессия МНК $y$ на $z$ с оценкой коэффициента наклона $(z'z)^{-1}z'y$. Аналогично, $\frac{dx}{dz}$ можно оценить с помощью регрессии МНК $x$ на $z$, получив оценку $(z'z)^{-1}z'x$. Тогда
\begin{equation}
\hat{\beta}_{IV} =\frac{(z'z)^{-1}z'y}{(z'z)^{-1}z'x} = (z'x)^{-1}z'y,
\end{equation}

\subsection{Оценка Вальда}
Ярким примером метода инструментальных переменных является такая оценка, когда переменная $z$ является \textbf{бинарным инструментом}. Обозначим средние значения $x$ и $y$ как, соответственно, $\bar{x}_1$ и $\bar{y}_1$ для подвыборки, где $z=1$ и как $\bar{x}_0$ и $\bar{y}_0$, для подвыборки, где $z=0$. Тогда $\Delta y /\Delta z = (\bar{y}_1-\bar{y}_0)$, а $\Delta x /\Delta z = (\bar{x}_1-\bar{x}_0)$, и из (4.46) следует 
\begin{equation}
\hat{\beta}_{Wald} = \frac{(\bar{y}_1-\bar{y}_0)}{(\bar{x}_1-\bar{x}_0)}.
\end{equation}
Эта оценка называется \textbf{оценкой Вальда}, по работе Вальда (1940), или \textbf{группировочной оценкой}.

Оценку Вальда также можно вывести из формулы (4.45). В модели без свободного члена переменные выражены в отклонениях от своих средних значений, а потому $z'y = \sum_i (z_i - \bar{z})(y_i - \bar{y})$. Для бинарного $z$ отсюда следует, что $z'y = N_1 (\bar{y}_1-\bar{y}) = N_1 N_0 (\bar{y}_1-\bar{y}_0)/N$, где $N_0$ и $N_1$ --- число наблюдений, для которых, соответственно, $z=0$ и $z=1$.  Этот результат использует $\bar{y}_1 - \bar{y} = (N_0 \bar{y}_1 + N_1 \bar{y}_1)/N - (N_0 \bar{y}_0 + N_1 \bar{y}_1)/N = N_0 (\bar{y}_1-\bar{y}_0)/N$. Аналогично, $z'x = N_0 N_1 (\bar{x}_1-\bar{x}_0)/N$. Объединив эти результаты с (4.45), получаем (4.48).

Для примера с образованием и доходом предполагается, что мы определили две группы, принадлежность к которым напрямую не влияет на доходы, но оказывает влияние на образование и потому косвенно оказывает влияние на доходы. Тогда оценка метода инструментальных переменных равна отношению разностей средних доходов х и среднего уровня образования в двух группах.

\subsection{Анализ выборочных ковариации и корреляции}
Оценку метода инструментальных переменных также можно интерпретировать через ковариации или корреляции. 

Для выборочной ковариации из (4.45) напрямую следует, что 
\begin{equation}
\hat{\beta}_{IV} = \frac{\Cov[z,y]}{\Cov[z,x]},
\end{equation}
где под $\Cov[\cdot]$ мы понимаем выборочную ковариацию.

Что касается корреляций, заметим, что оценку МНК в модели (4.43) можно выразить как $\hat{\beta}_{\text{МНК}} = r_{xy}\sqrt{y'y}/\sqrt{x'x}$, где $r_{xy} = x'y/\sqrt{(x'x)(y'y)}$ --- \textbf{выборочная корреляция} между $x$ и $y$.  Отсюда следует интерпретация коэффициента МНК как показывающего, что изменение $x$ на одно стандартное отклонение соответствует изменению $y$ на $r_{xy}$ стандартных отклонений. Проблема заключатся в том, что $r_{xy}$ <<загрязнена>> корреляцией между $x$ и $u$. Альтернативный подход измеряет корреляцию между $y$ и $x$ косвенно, через корреляцию между $z$ и $y$, делённую на корреляцию между $z$ и $x$.  Тогда
\begin{equation}
\hat{\beta}_{IV} = \frac{r_{zy}\sqrt{(y'y)}}{r_{zx}\sqrt{(x'x)}},
\end{equation}
и можно показать, что эта оценка совпадает с $\hat{\beta}_{IV}$ в (4.45).

\subsection{Оценка множественной регрессии методом инструментальных переменных}

Теперь рассмотрим модель множественной регрессии с типичным наблюдением 
$$ y =x'\beta +u,$$
с $K$ регрессорами, так, что $x$ и $\beta$ --- векторы $K \times 1$.

Допустим, что существует вектор \textbf{инструментов} $z$ размером $r \times 1$, где $r\geq K$, удовлетворяющий условиям:
\begin{small}
\begin{enumerate}
\item $z$ не коррелирует с ошибкой $u$.
\item $z$ коррелирует с вектором регрессоров $x$.
\item $z$ скорее сильно коррелирует, чем слабо коррелирует с вектором регрессоров $x$.
\end{enumerate}
\end{small}
Первые два условия являются необходимыми для состоятельности и были рассмотрены ранее для скалярного случая. Третье свойство, определённое в Разделе 4.9.1, является усилением второго, призванным обеспечить хорошее поведение оценки IV в конечных выборках.

В случае множественной регрессии у $x$ и $z$ могут быть общие компоненты. Некоторые компоненты $x$, называемые \textbf{экзогенными регрессорами}, могут не коррелировать с $u$. Эти компоненты, очевидно, годятся на роль инструментов, так как они удовлетворяют условиям 1 и 2. Другие компоненты $x$, называемые \textbf{эндогенными регрессорами}, могут коррелировать с $u$. Эти компоненты делают МНК несостоятельным, и, очевидно, не могут быть инструментами, так как не удовлетворяют условию 1. Разделим $x$ на $x = [x_1' x_2']'$, где $x_1$ содержит эндогенные регрессоры, а $x_2$ --- экзогенные. Тогда валидным инструментом будет $z = [z_1' x_2']'$, где $x_2$ может быть инструментом для самого себя, но нам требуется хотя бы столько инструментов $z_1$, сколько эндогенных переменных содержится в $x_1$. 
\begin{center}
Идентификация
\end{center}
Идентификация в модели с одновременными уравнениями была представлена в Разделе 2.5. Здесь у нас есть только одно уравнение. \textbf{Условие порядка} требует, чтобы число инструментов было не меньше числа эндогенных регрессоров, так, что $r \geq K$. Модель называется \textbf{точно идентифицированной}, если $r=K$, и \textbf{сверхидентифицированной}, если $r>K$.

Во многих приложениях множественной регрессии эндогенный регрессор только один. Например, регрессия дохода на образование  может включать множество других регрессоров, таких, как возраст, географическое расположение и семейный статус. Интерес предствляет коэффициент при образовании, но это эндогенная переменная, которая, скорее всего, коррелирует с ошибкой, поскольку способности ненаблюдаемы. Возможные кандидаты на роль требуемого единственного инструмента для образования были приведены в Разделе 4.8.2.

Если инструмент не удовлетворяет первому условию, это \textbf{невалидный (invalid) инструмент}. Если инструмент не удовлетворяет втоому условию, его называют \textbf{нерелевантным (irrelevant) инструментом}, и модель может быть \textbf{не идентифицирована}, если релевантных инструментов слишком мало.  Третье условие не выполняется, если корреляция инструмента с переменной, к которой он применяется, очень низкая. В этом случае модель называется \textbf{слабо идентифицируемой}, а инструмент называется \textbf{слабым инструментом}.
\begin{center}
Оценка методом инструментальных переменных
\end{center}
Если модель точно определена, то есть $r=K$, \textbf{оценка методом инструментальных переменных} является очевидным матричным обобщением (4.45):
\begin{equation}
\hat{\beta}_{IV} = (Z'X)^{-1}Z'y, 
\end{equation}
где $Z$ --- матрица размером $N\times K$ c $i$-тым рядом, равным $z_i'$. Подставив модель $y = X\beta+u$ в (4.51), получим

\[
\begin{array}{rcl}
\hat{\beta}_{IV} & = & (Z'X)^{-1} Z'[X\beta +u] \\
& = & \beta + (Z'X)^{-1} Z'u \\
& = & \beta + (N^{-1}Z'X)^{-1} N^{-1}Z'u.
\end{array}
\]

Из этого незамедлительно следует, что оценка методом инструментальных переменных состоятельна, если 
$$
\plim N^{-1}Z'u = 0
$$
и
$$
\plim N^{-1}Z'X \neq 0.
$$
По существу, это условия 1 и 2, говорящие, что $z$ не должен коррелировать с $u$ и должен коррелировать с $x$. Чтобы гарантировать существование матрицы, обратной к 	$N^{-1}Z'X$, предполагается, что $Z'X$ обладает полным рангом $K$ --- более сильное условие, чем условие порядка $r=K$.

C гетероскедастичными ошибками оценка IV асимптотически нормальна с математическим ожиданием $\beta$ и ковариационной матрицей, которую можно оценить как
\begin{equation}
\hat{\V}[\hat{\beta}_{IV}] = (Z'X)^{-1} Z'\hat{\Omega}Z(Z'X)^{-1},
\end{equation}
где $\hat{\Omega} = \mathrm{Diag}[\hat{u}_i^2]$. Этот результат получен способом, похожим на тот, который использовался для МНК в Разделе 4.4.4.

Оценка методом инструментальных переменных, хоть и состоятельная, приводит к потере эффективности, которая на практике может быть весьма значительной. Интуитивно, IV не будет хорошо работать, если корреляция инструмента $z$ с регрессором $x$ низкая (см. Раздел 4.9.3).

\subsection{Двухстадийный МНК}

Оценка методом инструментальных переменных в (4.51) требует, чтобы число инструментов было равно числу регрессоров. Для сверхидентифицированных моделей оценку IV можно использовать, если исключить часть инструментов так, чтобы модель стала точно идентифицированной. Однако такое исключение инструментов может привести к потере асимптотической эффективности.

Вместо этого общепринятой процедурой является \textbf{оценка методом двухстадйиного МНК} (two-stage least squares, 2SLS):
\begin{equation}
\hat{\beta}_{2SLS} = [X'Z(Z'Z)^{-1}Z'X]^{-1}[X'Z(Z'Z)^{-1}Z'y],
\end{equation}
представленная и обоснованная в Разделе 6.4.

Оценка 2SLS является оценкой методом инструментальных переменных. В точно идентифицированной модели она упрощается до оценки IV, приведённой в (4.51), с инструментами $Z$. В сверхидентифицированной модели оценка 2SLS равняется оценке IV, если последняя в качестве инструментов использует $\hat{X}=Z(Z'Z)^{-1}Z'X$, т.е. значения $x$, предсказанные из регрессии $x$ на $z$.

Оценка методом двустадийного МНК берёт своё название из того результата, что она может быть получена с помощью двух последовательных регрессий МНК: МНК-регрессии $x$ на $z$, чтобы получить $\hat{x}$, а затем МНК-регрессии $y$ на $\hat{x}$, чтобы получить $\hat{\beta}_{2SLS}$. Эта интерпретеция не всегда обобщается на нелинейные модели; см. Раздел 6.5.6

Оценку 2SLS можно записать более компактно как 
\begin{equation}
\hat{\beta}_{2SLS} = [X'P_ZX]^{-1}[X'P_Zy],
\end{equation}
где
$$
P_Z = Z(Z'Z)^{-1}Z',
$$
идемпотентная матрица, удовлетворяющая $P_Z = P_Z'$,$P_Z P_Z' = P_Z$ и $P_Z Z = Z$. Можно показать, что оценка 2SLS асимптотически нормально распределена с оценкой асимптотической дисперсии
\begin{equation}
\hat{\mathrm{V}}[\hat{\beta}_{2SLS}] =N[X'P_ZX]^{-1}[X'Z(Z'Z)^{-1}\hat{S}(Z'Z)^{-1}Z'X][X'P_ZX]^{-1},
\end{equation}
где в обычном случае с гетероскедастичными ошибками $\hat{S} = N^{-1}\sum_i\hat{u}_i^2 z_i z_i'$, а $\hat{u}_i = y_i - x_i'\hat{\beta}_{2SLS}$. Общепринято использовать поправку для малых выборок, деля на $N-K$ вместо $N$ в формуле для $\hat{S}$.

В особом случае, когда ошибки гомоскедастичны, происходит упрощение до $\hat{\mathrm{V}}[\hat{\beta}_{2SLS}] = s^2 [X'P_ZX]^{-1}$. Последний результат даётся во многих вводных курсах, однако более общая формула (4.55) предпочтительна, так как современным подходом является относиться к ошибкам как к потенциально гетерскедастичным.

Для сверхидентифицированных моделей с гетероскедастичными ошибками оценка, которую White (1982) называет \textbf{двухстадийной оценкой методом инструментальных переменных} более эффективна, чем 2SLS. Более того, некоторые широко используемые тесты на спецификацию модели требуют применения именно этой оценки, а не 2SLS. Подробнее см. в Разделе 6.4.2.

\subsection{Пример оценки методом инструментальных переменных}

В качестве примера оценки IV рассмотрим оценивание коэффициента наклона при $x$ в модели	
$$ y = 0+0.5x_u $$,
$$x = 0+z+v $$,
где $z \sim \mathcal{N} [2,1]$, а $(u,v)$ распределены совместно нормально со средним 0, дисперсией 1 и ковариацией 0.8. 

Регрессия МНК $y$ на $x$ даёт несостоятельную оценку, так как $x$ коррелирует с $u$, поскольку по построению $x$ коррелирует с $z$, который, в свою очередь, коррелирует с  $u$. Оценка IV даёт состоятельные результаты. Переменная $z$ является валидным инструментом, так как по построению она не коррелирует с $u$, но коррелирует с $x$. Трансформации $z$, такие, как $z^3$, также являются валидными инструментами.

Различные оценки и соответствующие им стандартные ошибки из сгенерированной выборки размером 10000 даны в Таблице 4.4. Нас интересует коэффициент наклона. 

\begin{table}[h]
\caption{\label{tab:iv}Пример метода инструментальных переменных}
\begin{minipage}{\textwidth}
\begin{tabular}[t]{lcccc}
\hline
\hline
& \bf{МНК}\footnote{Сгенерированные данные для выборки размером 10000. МНК несостоятелен, но другие оценки состоятельны. Указаны робастные стандартные ошибки, хотя в них нет необходимости, т.к. ошибки гомоскедастичны. Стандартные ошибки 2SLS некорректны. Процесс, порождающий данные, описан в тексте.} & \bf{IV} & \bf{2SLS} & \bf{IV ($z^3$)}  \\
\hline
Константа & -0.804 & -0.017 & -0.017 & -0.014 \\
 & (-0.014) & (-0.022) & (-0.032) & (-0.025) \\
$x$ & 0.902 & 0.510 & 0.510 & 0.509 \\
 & (-0.006) & (-0.010) & (-0.014) & (-0.012) \\
$R^2$ & 0.709 & 0.576 & 0.576 & 0.574 \\
\hline
\hline
\end{tabular}
\end{minipage}
\end{table}

Оценка МНК несостоятельна, и её коэффициент наклона, равный 0.902, более, чем на 50 стандартных ошибок, удалён от истинного значения 0.5. Остальные оценки состоятельны и все находятся в пределах двух стандартных ошибок от 0.5.

Существует несколько способов вычисления оценки МНК. Коэффициент наклона из МНК-регрессии $y$ на $z$ равен 0.5168, а из МНК-регрессии $x$ на $z$ --- 1.0124, и оценка IV равна 0.5168/1.0124 = 0.510, по (4.47). На практике оценка IV вычисляется напрямую, с использованием (4.45) или (4.51), где $z$ используется как инструмент для $x$, а стандартные ошибки рассчитываются по (4.52). Оценка 2SLS (см.(4.54)) может быть вычислена с помощью регрессии $y$ на $\hat{x}$, где $\hat{x}$ --- предсказание из регрессии $x$ на $z$. Оценки 2SLS в точности равны оценкам МНК в этой точно идентифицированной модели, хотя стандартные ошибки из регрессии $y$ на $x$ некорректны, как будет объяснено в Разделе 6.4.5.

Последний столбец использует $z^3$, а не $z$ в качестве инструмента для $x$. Эта альтернативная оценка IV состоятельна, так как $z^3$ не коррелирует с $u$ и коррелирует с $x$. Однако она менее эффективна для конкретно этой модели данных, и стандартная ошибка коэффициента наклона увеличивается с 0.010 до 0.012. 

Оценивание методом инструментальных переменных приводит к потере эффективности по сравнению с МНК, см (4.61) с общим результатом в случае одного регрессора и одного инструмента. Здесь $r^2_{x,z}=0.510$ (не указан в Таблице 4.4) высок, так что потеря невелика и 	стандартная ошибка коэффициента наклона увеличивается примерно с 0.006 до 0.010. На практике потеря эффективности может быть гораздо больше, чем здесь.


\section{Инструментальные переменные на практике}

Важные практические вопросы включают определение, необходимо ли использование инструментальных переменных, и если необходимо, какие инструменты валидны. Подходящие тесты на спецификацию представлены в Разделе 8.4. К сожалению, валидность этих тестов ограничена. Они требуют, чтобы в точно идентифицированной модели инструменты были валидны, и тестируют только сверхидентифицирующие ограничения.

Хотя с валидными инструментами оценки IV состоятельны, как подробно описано ниже, оценки IV могут быть намного менее эффективны, чем оценки МНК и могут иметь распределение в конечных выборках, которое для стандартных размеров выборки существенно отличается от асимптотического. Эти проблемы только ухудшаются,  если инструменты слабо коррелируют с переменными, к которым они применяются. Одна из причин, по которой слабые инструменты могут возникнуть, это применение большего числа инструментов, чем необходимо. Эту проблему легко исправить, отказавшись от части инструментов (см. также Donald и Newey, 2001). Более фундаментальные проблемы возникают, когда даже при минимальном количестве инструментов один или более из них слабые. 

Этот раздел посвящён проблемам, связанным со слабыми инструментами.

\subsection{Слабые инструменты}

Не существует единого определения слабого инструмента. Многие авторы используют следующие признаки \textbf{слабого инструмента}, представленные здесь в порядке возрастания сложности:
\begin{small}
\begin{itemize}
\item Скалярный регрессор $x$ и скалярный инструмент $z$: Слабый инструмент --- такой, что $r^2{x,z}$ мал.
\item Скалярный регрессор $x$ и вектор инструментов $z$: Инструменты слабые, если $R^2$ из регрессии $x$ на $z$, обозначаемый как $R^2_{x,z}$, мал, или $F$-статистика в тесте на значимость регрессии в целом мала.
\item Несколько регрессоров $x$, из которых эндогенный только один: слабый инструмент --- тот, для которого частный $R^2$ низок или частная $F$-статистика низка, где эти частные статистики определены в конце раздела 4.9.1. 
\item Несколько регрессоров $x$ и несколько инструментов $z$: Существуют различные меры.
\end{itemize}
\end{small}

\begin{center}
Меры $R^2$
\end{center}
Рассмотрим модель с одним уравнением
\begin{equation}
y = \beta_1 x_1 + x_2'\beta_2 +u,
\end{equation}
где только один регрессор $x_1$ эндогенный, а остальные регрессоры, собранные в вектор $x_2$, экзогенны. Предположим, что вектор инструментов $z$ включает экзогенные инструменты $x_2$ и хотя бы один инструмент, не входящий в $x_2$.

Одной из возможных мер $R^2$ является обычный $R^2$ из регрессии $x1$ на $z$. Однако он может быть высоким из-за того, что $x_1$ коррелирует с $x_2$, тогда как интуитивно нам нужна высокая корреляция $x_1$ с инструментами, отличными от $x_2$.

По этой причине Bound, Jaeger и Baker (1995) предложили использовать \textbf{частный $R^2$}, обозначаемый как $R^2_p$, который очищен от влияния $x_2$. $R^2_p$ получается как $R^2$ из регрессии  
\begin{equation}
(x_1-\tilde{x}_1)= (z-\widetilde{z})'\gamma + \upsilon,
\end{equation}
где $\tilde{x}_1$и  $\tilde{z}$ --- предсказанные значения из регрессий $x_1$ и $z$ на $x_2$. В точно идентифицированном случае $z-\tilde{z}$ упростится до $z_1-\tilde{z}_1$, где $z_1$ --- единственный инструмент, кроме $x_2$, а $\tilde{z}_1$ --- предсказанное значение из регрессии $z_1$ на $x_2$.

Нередко $R^2_p$ получается значительно ниже, чем $R^2_{x,z}$. Формула для $R^2_p$ упрощается до $r^2_{x,z}$, где есть только один регрессор и он эндогенный. Она далее упрощается до $\mathrm{Cor}[x,z]$, где и инструмент только один.

Если число эндогенных переменных больше, чем одна, анализ менее прямолинеен, и было предложено несколько различных обобщений $R^2_p$.

Рассмотрим модель с более чем одной эндогенной переменной и сосредоточимся на оценке коэффициента при первой эндогенной переменной. Тогда в (4.56) $x_1$ эндогенный и вдобавок часть переменных в $x_2$ также эндогенны. Различные альтернативные меры заменяют правую часть (4.57) на остаток, учитывающий пристутствие других эндогенных регрессоров. Shea (1997) предложил частичный $R^2$, скажем, $R_p^{*2}$, который вычисляется как квадрат выборочной корреляции между $(x_1-\tilde{x}_1)$ и $(\hat{x}_1-\tilde{\hat{x}}_1)$. Здесь $(x_1-\tilde{x}_1)$ снова обозначает остатки из регерссии $x_1$ на $x_2$, тогда как  $(\hat{x}_1-\tilde{\hat{x}}_1)$ --- остатки из регрессии $\hat{x}_1$ (предсказанного значения из регрессии $x_1$ на $z$) на $\hat{x}_2$ (предсказанное значение из регрессии $x_2$ на $z$). Poskitt и Skeels (2002) предложили альтернативный частный $R^2$, который, как и $R_p^{*2}$ Shea, упрощается до $R^2_p$, если эндогенный регрессор только один. Hall, Rudebusch и Wilcox (1996) вместо этого предлагают канонические корреляции.

Эти меры для коэффициента при первой эндогенной переменной можно повторно применить для всех остальных эндогенных переменных.  Poskitt и Skeels (2002) вдобавок рассматривают меру $R^2$, которая применяется одновременно ко всем эндогенным переменным и их инструментам.

Проблемы несостоятельности оценок и потери точности увеличиваются по мере того, как меры $R^2$ падают, как это показано в Разделах 4.9.2 и 4.9.3, в особенности в (4.60) и (4.62).
\begin{center}
Частные $F$-статистики
\end{center}
В случае с плохим поведением в конечных выборках, рассмотренном в Разделе 4.9.4, часто используют другую меру, $F$-статистику, тестирующую, что все коэффициенты в регрессии эндогенного регрессора на инструменты равны нулю. 

В случае с единственным регрессором, который является эндогенным, мы используем обычную общую $F$-статистику для теста, что $\pi=0$ в регрессии $x=z'\pi+\upsilon$ эндогенного регрессора на инструменты. Такая $F$-статистика является функцией от $R^2_{x,z}$.

Обычно в модели также появляются экзогенные регрессоры, и в модели (4.56) c одним эндогенным регрессором $x_1$ мы используем $F$-статистику, тестирующую, что $\pi_1=0$ в регрессии 
\begin{equation}
x = z_1'\pi_1 + x_2'\pi_2+\upsilon,
\end{equation}
где $z_1$ --- инструменты, не являющиеся экзогенными регрессорами, а $x_2$ --- экзогенные регрессоры. Это регрессия первой стадии в интерпретации метода инструментальных переменных двухстадийного МНК.

Эту статистику используют как сигнал о потенциальном смещении конечной выборки в оценке IV.  В Разделе 4.9.4 мы объясняем результаты Staiger и Stock (1997), которые предлагают понимать её значения, меньшие 10, как проблемные, а значения меньше 5 как признак сильного смещения конечной выборки, и мы рассматриваем его расширение на случай с более чем одним эндогенным регрессором.

\subsection{Несостоятельность оценок метода инструментальных переменных}

Необходимым условием для состоятельности IV является условие 1 в Разделе 4.8.6 о том, что инструмент должен не коррелировать с ошибкой. В случае с точно идентифицированной моделью протестировать это невозможно. В сверхидентифицированном случае можно провести тест на сверхидентифицирующие допущения (см. Раздел 6.4.3). Отклонение нулевой гипотезы может свидетельствовать либо об эндогенности инструментов, либо о неверно специфицированной модели. Следовательно, условие 1 сложно проверить напрямую, и определение, является ли инструмент экзогенным, зачастую остаётся субъективным решением, хоть и опирающимся на экономическую теорию.

Всегда можно создать экзогенные инструменты с помощью \textbf{ограничений функциональной формы}. Для примера рассмотрим модель с двумя регрессорами $y = \beta_1 x_1 +beta_2 x_2 +u$, в которой $x_1$ не коррелирует с $u$, а $x_2$ коррелирует с $u$. Заметим, что во всём этом разделе переменные рассматриваются как отклонения от своих средних значений, и свободный член можно исключить без потери общности. Тогда МНК несостоятелен, т.к. $x_2$ является эндогенным. Очевидно хорошим инструментом для $x_2$ является $x_1^2$, так как $x_1^2$, скорее всего, не коррелирует с $u$, потому что с ним не коррелирует $x_1$. Однако валидность этого инструмента требует ограничения на функциональную форму модели --- что $x_1$ включён в неё только линейно, а не квадратично. Но на практике линейную модель следует рассматривать только как приближение, и получение таких искусственных инструментов можно легко подвергнуть критике.

Более подходящим способом создания валидных инструментов является \textbf{исключающие ограничения}, которые не зависят в такой значительной степени от выбора функциональной формы. Некоторые примеры из практики были даны в Разделе 4.8.2.

Структурные модели, такие, как классическая модель линейных одновременных уравнений (см. Разделы 2.4 и 6.10.6), делают такие исключающие ограничения в явном виде. Даже если эти ограничения часто можно подвергнуть критике, как взятые с потолка, если только их не поддерживают доводы экономической теории.

В приложениях с панельными данными может быть разумным предположение о том, что в интересующее нас уравнение могут входить только текущие значения переменных. Такое искючающее ограничение позволяет использовать в качестве инструментов лаговые значения переменных при допущении, что ошибки не испытывают серийной корреляции (см. Раздел 22.2.4). Аналогично, в моделях принятия решений при неопределённости (см. Раздел 6.2.7), лагированные переменные можно использовать в качестве инструментов, т.к. они являются частью информационного множества.

Не сущестует формального теста на экзогенность инструментов, который одновременно не тестировал бы гипотезу, является ли уравнение корректно специфицированным. Экзогенность инструментов неизбежно опирается на априорную информацию, такую, как экономическая или статистическая теория. Оценка Bond et.al. (1995, стр. 446-447) валидности инструментов, использованных Angrist и Crueger (1991) предоставляет поучительный пример тонкостей, используемых при определении экзогенности инструментов.

Особенно важно, чтобы инструмент был экзогенным, если это слабый инструмент, так как в случае слабых инструментов даже вполне умеренная эндогенность может сделать оценки IV гораздо более несостоятельными, чем и так несостоятельные оценки МНК.

Для простоты рассмотрим линейное уравнение с одним регрессором и одним инструментом, т.е. $y = \beta x +u$. Тогда с помощью алгебры, которую мы оставим как упражнение, можно получить 
\begin{equation}
\frac{\plim \hat{\beta}_{IV}-\beta}{\plim \hat{\beta}_{IV}-\beta}
=\frac{\Cor[z,u]}{\Cor[x,u]}\times \frac{1}{\Cor[z,x]}.
\end{equation}
Таким образом, в случае невалидного инструмента и низкой корреляции между инструментом и регрессором оценка IV может быть даже более несостоятельной, чем оценка МНК. К примеру, положим корреляцию между $z$ и $x$ равной 0.1, что не является необычным для пространственных данных. Тогда оценка метода инструментальных переменных становится более несостоятельной, чем оценка МНК, как только коэффициент корреляции между $z$ и $u$ превосходит всего лишь 0.1 корреляции между $x$ и $u$. 

Результат (4.59) можно расширить на модель (4.56) с одним эндогенным и несколькими экзогенными регрессорами, iid ошибками и инструментами, включающими экзогенные регрессоры. Тогда
\begin{equation}
\frac{\plim \hat{\beta}_{1, 2SLS}-\beta_1}{\plim \hat{\beta}_{1, OLS}-\beta_1}
=\frac{\Cor[\hat{x},u]}{\Cor[x,u]}\times \frac{1}{R_p^2},
\end{equation}
где $R_p^2$ определён после (4.56). Обобщение до случая с более чем одним эндогенным регрессором см. у Shea (1997).

Эти результаты, подчёркнутые в Bound и др. (1995), имеют серьёзные последствия для использования инструментальных переменных. Если инструменты слабые, то даже мягкая эндогенность регрессоров может приводить к тому, что оценки IV даже более несостоятельны, чем оценки МНК. Возможно, именно потому что этот вывод так печален, литература в основном умалчивала об этом аспекте слабых инструментов. Важным недавним исключением является работа Hahn и Hausman (2003a).

В большей части работ предполагается, что условие 1 выполнено, и оценка метода инструментальных переменных состоятельна, и делается фокус на других затруднениях, связанных со слабыми инструментами.

\subsection{Низкая точность}

Хотя оценки методом инструментальных переменных могут быть состоятельными, когда несостоятелен МНК, они также приводят к потере точности. Интуитивно, из Раздела 4.8.2 следует, что инструмент $z$ является экзогенным воздействием, которое приводит к экзогенному сдвигу $x$, но делает это при значительным уровне шума. 

Потеря точности увеличивается и стандартные ошибки растут, если инструменты делаются более слабыми. Это легко видеть в простейшем случае единственного регрессора и единственного инсрумента с iid ошибками. Тогда асимптотическая дисперсия равна
\begin{equation}
\begin{array}{rcl}
\V[\hat{\beta}_{IV}]& = & \sigma^2 (x'x)^{-1}z'z(x'x)^{-1} \\
 & = & [ \sigma^2 / (x'x)]/[(z'x)^2/(z'z)(x'x)] \\
 & = & \V [\hat{\beta}_{\text{МНК}}]/r^2_{xz}. \\
\end{array}
\end{equation}
Например, если квадрат выборочной корреляции между $x$ и $z$ равен 0.1, стандартные ошибки оценок IV в 10 раз больше, чем для МНК. Более того, дисперсия оценки метода инструментальных переменных больше дисперсии оценки МНК, кроме случая, когда $\Cor[x,z]=1$.

Результат (4.61) можно обобщить до модели (4.56) с одним эндогенным регрессором и несколькими экзогенными регрессорами, iid ошибками и инструментами, включающими экзогенные регрессоры. Тогда
\begin{equation}
\mathrm{se}[\hat{\beta}_{1,2SLS}] = \mathrm{se}[\hat{\beta}_{1,OLS}]/R^2_p,
\end{equation}
где $\mathrm{se}[\cdot]$ обозначает асимптотические стандартные ошибки, а $R^2_p$ определён после (4.56). При обобщении до модели с более чем одним эндогенным регрессором этот $R^2_p$ заменяется на $R^{*2}_p$, предложенны Shea (1997). Это предоставляет обоснование для статистики Shea.

Низкая точность сконцентрирована в коэффициентах при эндогенных переменных. Для экзогенных переменных стандартные ошибки коэффициентов для оценки 2SLS схожи с такими же для МНК. Интуитивно, экзогенные переменные являются инструментами для самих себя, а значит, очень сильными инструментами.

Для коэффициентов при эндогенных переменных низкий \textbf{частный} $R^2$, а не $R^2$ приводит к потере точности оценки. Это объясняет, почему стандартные ошибки 2SLS могут быть гораздо больше, чем для МНК, несмотря на высокую <<сырую>> корреляцию между эндогенной переменной и инструментами. С другой стороны, если стандартные ошибки 2SLS для коэффициентов при эндогенных переменных заметно выше, чем стандартные ошибки МНК, это является чётким сигналом, что инструменты слабые.

Статистики, используемые для распознавания низкой точности оценок методом инструментальных переменных, называются мерами \textbf{релевантности инструментов}. В некоторой степени они не являются необходимыми, так как проблему легко заметить, если стандартные ошибки IV намного выше, чем стандартные ошибки МНК.

\subsection{Смещённость вследствие конечности выборки}

Этот раздел резюмирует относительно многообещающую и до сих пор не завершённую литературу о <<слабых инструментах>>, сосредоточенную на той практической проблеме, что даже в <<больших>> выборках асимптотическая теория может предоставлять плохое приближение распределения оценок методом инструментальных переменных. В частности, в конечных выборках оценка IV смещена, даже если асимптотически она состоятельна. Это смещение может быть особенно заметным, если инструменты слабые.

Это смещение IV, направленное в сторону несостоятельной оценки МНК, может быть удивительно большим, как это было продемонстрировано в простом эксперименте Монте Карло в работе Nelson и Startz (1990), и в приложении с реальными данными, включающими несколько сотен тысяч наблюдений, но очень слабые инструменты в Bound и др. (1995). Более того, стандартные ошибки также могут быть серьёзно смещёнными, что также было продемонстрировано  Nelson и Startz (1990).

Теоретическая литература использует весьма специализированную и продвинутую эконометрическую теорию, так как получить выборочное среднее оценки IV действительно сложно. Чтобы увидеть это, рассмотрим адаптацию к оценке IV обычное доказательство несмещённости оценки МНК, приведённое в Разделе 4.4.8. Для $\hat{\beta}_{IV}$, определённой в (4.51), в точно идентифицированном случае следует
\[
\begin{array}{rcl}
\E[\hat{\beta}_{IV}] & = & \beta + \E_{Z,X,u}[(Z'X)^{-1}Z'u] \\
& = &  \beta + \E_{Z,X}[(Z'X)^{-1}Z' \times \E[u|Z,X]],
\end{array}
\]
где безусловное математическое ожидание по всем случайным переменным $Z, X$ и $u$ мы получили, сперва взяв ожидание по $u$ при условии $Z$ и $X$, используя закон вложенных математических ожиданий (см. раздел А.8.). Очевидным достаточным условием того, что математическое ожидание оценки IV равно $\beta$, является равенство $\E[u|Z,X]=0$. Однако это слишком сильное предположение, так как из него следует, что $\E[u|X]=0$, а в этом случае нет необходимости вообще использовать инструменты. Поэтому простого способа получить $\E[\hat{\beta}_{IV}]$ не существует. При доказательстве состоятельности подобной проблемы не возникает: $hat{\beta}_{IV} = \beta +(N^{-1}Z'X)^{-1}N^{-1}Z'u$, где множитель $N^{-1}Z'u$ можно рассмотреть отдельно от $X$, и из предположения $\E[u|Z]=0$ следует $\plim N^{-1}Z'u=0$.

Таким образом, необходимо использовать альтернативные методы для получения математического ожидания оценки методом инструментальных переменных. Здесь мы просто перечисляем основные результаты.

В первоначальных исследованиях делалось сильное предположение о совместной нормальности переменных и гомоскедастичных ошибках. Тогда оценка IV имеет распределение Висхарта (определённое в Главе 13). Удивительно, но в случае точно определённой модели математического ожидания оценки IV вообще не существует --- сигнал, что может существовать проблема конечных выборок. Математическое ожидание существует, если есть хотя бы одно сверхидентифицирующее ограничение, а дисперсия существует, если сверхидентифицирующих ограничений хотя бы два. Но даже если ожидание существует, оценка IV смещена, причём в сторону оценки МНК. С увеличением числа сверхидентифицирующих ограничений смещение увеличивается, в конце концов достигая смещения оценки МНК. Подробное обсуждение дано в Davidson и MacKinnon (1993, стр. 221-224). Также использовались приближения, основанные на разложении в степенные ряды.

Что определяет размер смещения в конечных выборках? Для регрессии с единственным регрессором $x$, являющимся эндогенным и связанным с инструментами $z$ моделью в приведённой форме $x = z\pi +\upsilon$, \textbf{параметр концентрации} $\tau^2$ определяется как $\tau^2 = \pi'Z'Z\pi/\sigma^2_\upsilon$. Можно показать, что смещение оценки IV является возрастающей функцией $\tau^2$. Величина $\tau^2/K$, где $K$ --- число инструментов, является аналогом $F$-статистики для генеральной совокупности, тестирущей ограничение $\pi=0$. Можно показать, что статистика $F-1$, где $F$ --- действительная $F$-статистика из регрессии первой стадии модели в приведённой форме, является приблизительно несмещённой оценкой $\tau^2/K$. Это делает возможным тесты на смещение в конечной выборке, основанные на $F$-статистике, данной в Разделе 4.9.2.

Staiger и Stock (1997) получили результаты при более слабых предположениях о распределении. В частности, нормальность больше не требуется. Их подход использует асимптотику слабых инструментов, находящую предельное распределение оценок IV  для последовательности моделей, где $\tau^2/K$ зафиксировано на постоянном уровне, а $N\longrightarrow \infty$. В простейшей модели $1/F$ является приблизительной оценкой для смещения в конечной выборке оценки IV по сравнению с оценкой МНК. В более общем случае величина смещения при данной $F$ зависит от числа эндогенных регрессоров и числа инструментов. Симуляции показывают, что, чтобы обеспечить максимальное смещение оценки IV, не превышающее 10\% смещения МНК, необходима $F>10$. Этот порог широко цитируется, но падает, например, до приблизительно 6.5, если смещение в 20\% смещения МНК является удовлетворительным. Таким образом менее строгим правилом является $F>5$. Shea (1997) продемонстрировал, что низкий частный $R^2$ также связан со смещением в конечных выборках, но не существует похожего простого правила, позволящего использовать частный $R^2$ для диагностики смещения в конечных выборках.

В моделях с более чем одним эндогенным регрессором $F$-статистики можно рассчитать для всех эндогенных регрессоров. В качестве совместной статистики Stock, Wright и Yogo (2002) предлагают использовать наименьшее собственное значение матричного аналога $F$-статистики для регрессии первой стадии. Stock и Yogo (2003) представляют релевантные критические значения для этого собственного значения при различных уровнях удовлетворительного смещения, числа эндогенных регрессоров и числа сверхидентифицирущих ограничений. Эти таблицы включают случай с единственным эндогенным регрессором как особый, но предполагают как минимум два сверхидентифицирующих ограничения, так что они не применимы к точно идентифицированным моделям. 

Смещение в конечных выборках возникает не только для оценок коэффициентов методом инструментальных переменных, но также для оценок стандартных ошибок и тестовых статистик. Stock и др. (2002) представляет подход, аналогичный тестам Вальда, согласно которому тест $\beta = \beta_0$ на номинальном уровне 5\% имеет действительный размер не более 15\%. Stock и Yogo (2003) также представляют подробные таблицы, описывающие изменение размера теста, которые включают и точно идентифицированные модели. 

\subsection{Реакция на слабые инструменты}

Что может сделать практик, столкнувшись со слабыми инструментами?

Как уже было замечено, одним из подходов является ограничение числа используемых инструментов. Это можно осуществить, удалив часть инструментов из модели или объединив некоторые инструменты.

Если проблемой является смещение в конечной выборке, можно использовать альтернативные оценки, обладающие лучшими свойствами в конечных выборках, чем двухстадийный МНК. Несколько альтернатив, различных вариантов IV, представлены в Разделе 6.4.4.

Несмотря на акцент на смещении в малых выборках, другие проблемы, связанные со слабыми инструментами, могут иметь большую важность в приложениях. В достаточно большой выборке возможна $F$-статистика для регрессии первой стадии в приведённой форме достаточно большая, чтобы смещение в конечной выборке не представляло проблемы. В то же время частичный $R^2$ может быть очень низким, приводя к чувствительности даже к очень слабой корреляции между инструментом и ошибкой модели. Эту трудность сложно проверить и преодолеть.

Также может иметь место значительное снижение точности оценок, как описано в Разделах 4.9.3 и 4.9.4. В таких случаях либо требуются большие выборки, либо необходимо использовать альтернативные подходы к оценке причинно-следственных предельных эффектов. Эти методы кратко описаны в Разделе 2.8 и представлены в различных местах этой книги.

\subsection{Применение инструментальных переменных}

Kling (2001) подробно проанализировал использование близости колледжа как инструмент для образования. Здесь мы используем те же данные из NLS по когорте из 3010 мужчин в возрасте от 24 до 34 лет в 1976 году, какие были использованы для получения Таблицы 1 в Kling (2001), и первоначально использованной в Card (1995). Оценивается модель
$$
\mathrm{ln} w_i = \alpha + \beta_1 s_i + \beta_2 e_i + \beta_3 e_i^2 + x_{2i}'\gamma+u_i,
$$
где $s$ обозначает образование в годах, $e$ обозначает опыт работы в годах, $e^2$ обозначает квадрат опыта, а $x_2$ --- вектор из 26 контрольных переменных, состоящий в основном из географических индикаторов и меры образования родителей.

Переменная образования рассматривается как эндогенная вследствие отсутствия данных по способностям. Кроме того, две переменные, отвечающие за опыт работы также эндогенны, поскольку опыт работы вычислен как возраст минус длительность образования минус шесть, как принято в таких работах, и образование эндогенно. Таким образом, требуется как минимум три инмтрумента.

Здесь используется ровно три инструмента, то есть модель точно идентифицирована. Первым инструментом является $col4$, индикатор наличия поблизости четырёхлетнего колледжа. Этот инструмент уже был раассмотрен в Разделе 4.8.2. Двумя другими инструментами являются возраст и возраст в квадрате. Они высоко коррелируют с опытом работы и квадратом опыта, но предполагается, что их можно исключить из модели заработной платы, поскольку для работодателя важен именно опыт работы. Оставшийся вектор регрессоров $x_2$ используется как инструмент для самого себя. 

Хотя возраст очевидно экзогенен, некоторые ненаблюдаемые характеристики, такие, как социальные навыки, могут коррелировать одновременно и с возрастом, и с заработной платой. В таком случае использование возраста и квадрата возраста в качестве инструментов можно подвергнуть сомнению. Это иллюстрирует общую мысль о том, что относительно предположений о валидность инструментов могут существовать разногласия.

\begin{table}[h]
\begin{minipage}{\textwidth}
\caption{  \label{tab:ivapp}Отдача от образования: оценки метода инструментальных переменных} 

    \begin{tabular}{lcc}
    \hline
	\hline
	& \bf{МНК}\footnote{Выборка из 3010 молодых людей. Зависимая переменная --- логарифм почасовой зарплаты. Даны коэффициенты и их стандартные ошибки для образования; оценки для опыта и опыта в квадрате, 26 контрольных переменных и свободного члена не указаны. Для трёх эндогенных регрессоров --- образования ($s$), опыта ($e$) и квадрата опыта ($e^2$) --- тремя инструментами являются индикатор близости четырёхлетнего колледжа, возраст и возраст в квадрате. Частный $R^2$ и $F$-статистика из регрессии первой стадии используются для диагностики слабых инструментов, объяснённой в тексте.} & \bf{IV} \\
	\hline
    Образование (s) & 0.073 & 0.132 \\
    	& (0.004) & (0.049) \\
    $R^2$	& 0.304 & 0.207 \\
    Частный $R^2$ Shea & --- & 0.006 \\
    $F$-статистика для s & --- & 8.07 \\
    \hline
	\hline
\end{tabular}
\end{minipage}
\end{table} 


Результаты приведены в Таблице 4.5. Оценка МНК $\beta_1$ равна 0.073, то есть зарплаты растут в среднем на 7.6\% ( $= 100 \times (e^{0.073}-1)$) с каждым дополнительным годом образования. Эта оценка является несостоятельной оценкой $\beta_1$, если способности не включены в регрессию. Оценка IV или эквивалентная ей оценка 2SLS (так как модель точно идентифицирована) равны 0.132. Дополнительный год образования приводит к увеличению зарплаты на 14.1\% ( $= 100 \times (e^{0.132}-1)$). 

Оценка IV намного гораздо менее эффективна, чем оценка МНК. Формальный тест не отвергает гипотезу о гомоскедастичности, и мы, следуя Kling (2001), используем обычные стандартные ошибки, которые здесь очень близки к станадртным ошибкам, устойчивым к гетероскедастичности. Стандартная ошибка $\hat{\beta}_{1, OLS}$ равна 0.004, тогда как $\hat{\beta}_{1, IV}$ превышает её более чем в 10 раз. Стандартные ошибки при других эндогенных регрессорах примерно в 4 раза больше, а стандартные ошибки при экзогенных регрессорах примерно в 1.2 раза больше. $R^2$ падает с 0.304 до 0.207. 

Меры $R^2$ подтверждают, что инструменты не вполне релевантны для образования. Простейшим тестом  является заметить, что регрессия (4.58) образования на все инструменты даёт  $R^2=0.297$, который падает лишь немного до  $R^2=0.291$, если исключить три дополнительные инструмента. Более формально, частный  $R^2$ Shea здесь равен $0.0064 = 0.08^2$, откуда по (4.62) следует, что стандартная ошибка $\hat{\beta}_{1, IV}$ будет увеличена на множитель $12.5 = 1/0.08$, очень близко к увеличению, наблюдаемому здесь. Это сокращает $t$-статистику при образовании с 19.64 до 2.68. Во многих приложениях такое сокращение привело бы к статистической незначимости. Вдобавок, согласно Разделу 4.9.2, даже лёгкая корреляция между инструментом $col4_i$ и ошибкой $u_i$ может привести к несостоятельности метода инструментальных переменных. 

Чтобы увидеть, что смещение в конечной выборке тоже является проблемой, мы строим регрессию (4.58) образования на все остальные инструменты. Тестирование совместной значимости трёх дополнительных инструментов даёт $F$-статистику 8.07, говорящую, что смещение оценки IV может составлять от 20\% до 10\% смещения оценки МНК. Аналогичная регрессия для двух других экзогенных переменных даёт гораздо более высокие  $F$-статистики, так как например, возраст является хорошим инструментом для опыта работы. С учётом того, что эндогенных регрессора 3, в действительности лучше использовать метод Stock и др. (2002), описанный в Разделе 4.9.4, хотя здесь проблема ограничена до образования, поскольку для опыта и опыта в квадрате, соответственно, частный $R^2$ Shea равен 0.0876 и 0.0138, тогда как $F$-статистики для первой стадии равны 1.772 и 1.542.

Если доступны дополнительные инструменты, модель становится сверхидентифицированной, и стандартной процедурой является дополнительный тест на сверхидентифицирующие ограничения (см. Раздел 8.4.4).
 
\section{Практические соображения}

Процедуры оценки, описанные в этой главе, присутствуют во всех стандартных статистических паетах, работающих с пространственными данными, за исключением того, что не во всех пакетах реализована квантильная регрессия. Большая часть пакетов предоставляет робастные стандартные ошибки как опцию, а не по умолчанию.

Самой сложной в применении является оценка методом инструментальных переменных, так как во многих потенциальных приложениях может быть сложно найти инструменты, не коррелирующие с ошибкой, но существенно коррелирующие с регрессором или регрессорами, к которым они применяются. Такие инструменты можно получить через спецификацию полной структурной модели, такой, как модель одновременных уравнений. Современные прикладные исследования подчёркивают альтернативные подходы, такие, как натуральные эксперименты.

\section{Библиографические примечания}

Результаты из этой главы представлены во многих текстах для первого курса магистратуры, таких, как тексты Davidson и MacKinnon (2004), Greene (2003), Hayashi (2000), Johnston и diNardo
(1997), Mittelhammer, Judge, and Miller (2000) и (2000). Мы сделали акцент на регрессии со стохастическими регрессорами, робастных стандартных ошибках, квантильной регрессии, эндогенности и инструментальных переменных.

\begin{itemize}
\item [$4.2$] Мански (1991) предоставляет отличное обсуждение общей постановки задачи, включающее обсуждение функций потерь, данных в Разделе 4.2.
\item [$4.3$] Пример с отдачей от образования хорошо изучен. Angrist и Krueger (1999) и Card
(1999) предоставляют свежие обзоры этой темы.
\item [$4.4$] Историю метода наименьших квадратов см. у Stigler (1986). Этот метод был введён Лежандром в 1805 году. Гаусс в 1810 применил метод наименьших квадратов к модели с нормально распределённой ошибкой и предложил метод исключения для вычисления, а в дальнейшей работе он предложил теорему, называемую теоремой Гаусса-Маркова. Гальтон предложил понятие регрессии, имея в виду возвращение к среднему в контексте наследования семейных черт, в 1887 году. 

Раннее ''современое'' описание с приложением к пауперизму и доступности благосостояния см. у Yule (1897). Статистические выводы, основанные на оценках МНК линейной регрессионной модели, были разработаны в особенной степени Фишером. Устойчивые к гетероскедастичности оценки ковариационной матрицы оценки МНК, благодаря White (1980a), основывавшемуся на более ранней работе Eicker (1963), оказали глубокое воздействие на статистические заключения в микроэконометрике и были расширены для множества задач.
\item [$4.6$] Boscovich в 1757 году предложил оценку метода наименьших модулей ошибок, опередившую метод наименьших квадратов, см. Stigler (1986). Обзор квантильной регрессии, введённой Коэнкер и
Бассетт (1978), дан в Buchinsky (1994). Более элементарное рассмотрение дано в Коэнкер and Hallock (2001).
\item [$4.7$] Наиболее раннее из известных использование инструментальных переменных для обеспечения идентификации в модели одновременных уравнений относится к  Wright (1928). Другим часто цитируемым источником является Reiersol (1941), который использовал метод инструментальных переменных для поправки ошибки измерения в регрессорах. Sargan (1958) даёт классическое раннее описание оценки IV.  Stock and
Trebbi (2003) представляют дополнительные ранние работы.
\item [$4.8$] Оценка методои инструментальных переменных представлена в эконометрических текстах с акцентом на алгебре, но не всегда на интуиции. Этот метод широко используется в эконометриуе благодаря желательности получения оценок, обладающих причинно-следственной интерпретацией.
\item [$4.9$] К проблеме слабых инструментов внимание прикладных исследователей привлекли Nelson и Startz (1990) и Bound и др. (1995). Существует некоторое число их теоретических предшественников, особенно стоит упомянуть работу Nagar (1959). Эта проблема убавила энтузиазм по отношению к оцениванию IV, а смещение в конечных выборках вследствие слабых инструментов является сейчас очень активной темой исследований. Результаты часто предполагают нормальные iid ошибки и ограничивают анализ до случая с одним эндогенным регерссором. Исследование Stock и др.(2002)  предоставляет много ссылок с акцентом на асимптотику слабых инструментов. Исследование Hahn и Hausman (2003b) предоставляет дополнительные методы и результаты, которые мы не привели здесь. Недавние работы по смещению в стандартных ошибках см. у Bond и Windmeijer (2002). Пример аккуратного применения см. C.-I. Ли (2001).
\end{itemize}

\begin{center}
Упражнения
\end{center} 
\begin{small}
\begin{enumerate}
\item [$4-1$] Рассмотрим модель линейной регрессии $y_i = x_i'\beta +u_i$ с нестохастическими регрессорами $x_i$ и ошибкой $u_i$, которая имеет нулевое среднее, но коррелирует следующим образом: $\E[u_i u_j] = \sigma^2$, если $i=j$, $\E[u_i u_j] = \rho \sigma^2$, если $|i-j|=1$, и $\E[u_i u_j] = 0$, если $|i-j|>1$. Таким образом, ошибки для непосредственно соседствущих наблюдений корелируют, тогда как остальные ошибки не коррелируют. В матричных обрзначениях $y = x'\beta +u$, где $\Omega = \E [u'u]$. Ответьте на вопросы для этой модели, используя результаты из Раздела 4.4.
\begin{enumerate}
\item Проверьте, что  $\Omega$ является <<перевязанной>> матрицей с ненулевыми элементами только на диагонали и в напрямую смежных с ней ячейках и приведите эти ненулевые элементы. 
\item Получите асимптотическое распределение $\hat{\beta}_{\text{МНК}}$, используя (4.19).
\item Объясните, как получить состоятельную оценку $\V[\hat{\beta}_{\text{МНК}}]$, не зависящую от неизвестных параметров.
\item Является ли обычная оценка $s^2 (X'X)^{-1}$, выводимая с МНК, состоятельной оценкой $\V[\hat{\beta}_{\text{МНК}}]$?
\item Является ли оценка Уайта $\V[\hat{\beta}_{\text{МНК}}]$, устойчивая к гетероскедастичности, состоятельной здесь?
\end{enumerate}
\item [$4-2$] Предположим, что мы оцениваем модель $y_i = \mu + u_i$, где $u_i \sim \mathcal{N}[0,\sigma^2_i]$.
\begin{enumerate}
\item Покажите, что оценка МНК для $\mu$ упрощается до $\bar{y}$.
\item Поскольку этот так, получите дисперсию $\bar{y}$ напрямую. Покажите, что она равна оценке дисперсии Уайта, устойчивой к гетероскедастичности, данной в (4.21). 
\end{enumerate}
\item [$4-3$] Предположим, что процесс, порождающий данные $y_i = \beta_0 x_i + u_i$, $x_i \sim \mathcal{N}[0,1]$, $u_i = x_i \epsilon_i$, и $\epsilon_i \sim \mathcal{N}[0,1]$. Предположим, что данные независимы по $i$, а $x_i$ не зависит от $\epsilon_i$. Заметим, что первые четыре центральных момента $\mathcal{N}[0,1]$ равны $0$, $\sigma^2$, $0$ и $3 \sigma^4$.
\begin{enumerate}
\item Покажите, что ошибки $u_i$ условно гетероскедастичны.
\item Получите $\plim N^{-1}X'X$. [Подсказка: получите $\E[x_i^2]$ и примените закон больших чисел].
\item Получите $\sigma_0^2 = \V[u_i]$, где математическое ожидание берётся по всем стохастическим перменным в модели.
\item Получите $\plim N^{-1}X'\Omega_0 X = \lim N^{-1}\E[X'\Omega_0 X]$, где $\Omega_0 = \mathrm{Diag}[\V[u_i|x_i]]$
\item Используя ответы на предыдущие вопросы, рассчитайте оценку по умолчанию (4.22), рассчитываемую для МНК, ковариационной матрицы в предельном распределении $\sqrt{N}(\hat{\beta}_{\text{МНК}}-\beta_0)$, игнорируя потенциальную гетероскедастичность. Ваш конечный ответ должен быть выражен числом. 
\item Теперь рассчитайте дисперсию  $\sqrt{N}(\hat{\beta}_{\text{МНК}}-\beta_0)$, учитывая гетероскедастичность. Ваш конечный ответ должен быть выражен числом.
\item Согласуется ли расхождение между ответами на (e) и (f) с вашими априорными предположениями?
\end{enumerate}
\item [$4-4$] Рассмотрим модель линейной регрессии со скалярным регрессором $y_i=\beta x_i +u_i$, где данные $(y_i, x_i)$ распределены одинаково и независимо от $i$, хотя ошибка может быть условно гетероскедастичной.
\begin{enumerate}
\item Покажите, что $(\hat{\beta}_{\text{МНК}}-\beta) = (N^{-1}\sum_i x_i^2)^{-1}N^{-1}\sum_i x_i u_i$.
\item Примените закон больших чисел Колмогорова (Теорема A.8) к средним значениям $x_i^2$ и $x_i u_i$, чтобы показать, что $\hat{\beta}_{\text{МНК}} \xrightarrow{p} \beta$. Сформулируйте дополнительные допущения о процессе, порождающем данные для $x_i$ и $u_i$.
\item Примените центральную теорему Линдеберга-Леви (Теорема A.14) к средним значениям $x_i u_i$, чтобы показать, что $N^{-1}\sum_i x_i u_i /N^2 \sum_i \E [u_i^2 x_i^2] \xrightarrow{p} \mathcal{N}[0,1]$. Сформулируйте дополнительные допущения о процессе, порождающем данные для $x_i$ и $u_i$.
\item Используйте правило произведения для предельных нормальных распределений (Теорема A.17), чтобы показать, что из части (с) следует $N^{-1/2}\sum_i x_i u_i \xrightarrow{p} \mathcal{N}[0,\lim  \sum_i \E [u_i^2 x_i^2]]$ Сформулируйте дополнительные допущения о процессе, порождающем данные для $x_i$ и $u_i$.
\item Объедините результаты, используя (2.14) и правило произведения для предельных нормальных распределений (Теорема A.17), чтобы получить предельное распределение $\beta$. 
\end{enumerate}
\item [$4-5$] Рассмотрим линейную модель регрессии $y = X\beta +u$.
\begin{enumerate}
\item Получите формулу для $\hat{\beta}$, минимизирующую $Q(\beta) = u'Wu$, где $W$ --- матрица полного ранга. [Подсказка: цепное правило для дифференцирования матриц для столбцов $x$ $z$ выглядит как $\partial f(x)/\partial x = (\partial z')/\partial x) \times (\partial f(z)/\partial z)$, если $f(x) = f(g(x)) = f(z)$, где $z = g(x)$].
\item Покажите, что эта формула упрощается до оценки МНК, если $W=I$.
\item Покажите, что эта формула даёт оценку ОМНК, если $W= \Omega^{-1}$.
\item Покажите, что эта формула даёт оценку 2SLS, если $W= Z(Z'Z)^{-1}Z'$.
\end{enumerate}
\item [$4-6$] Рассмотрим оценивание IV (Раздел 4.8) модели $y = x'\beta +u$ с использованием инструмента $z$ в точно идентифицированном случае, когда $Z$ является матрицей $N \times K$ полного ранга.
\begin{enumerate}
\item Каким существенным предположениям должна удовлетворять $z$, чтобы оценка IV была состоятельной для $\beta$? Объясните.
\item Покажите, что в случае точной идентификации оценка 2SLS, определённая в (4.53), упрощается до оценки IV, данной в (4.51).
\item Приведите пример из реальной жизненной ситуации, когда оценка IV необходима из-за несостоятельности МНК, и специфицируйте подходящие инструменты. 
\end{enumerate}
\item [$4-7$] (Адаптировано из Nelson и Startz, 1990.) Рассмотрим модель с тремя уравнениями $y = \beta x +u$; $x = \lambda u + \epsilon$; $z = \gamma \epsilon + v$, где взаимно независимые ошибки $u$, $\epsilon$ и $v$ распределены одинаково, независимо и нормально со средним $0$ и дисперсиями, соответственно, $\sigma^2_{u}$,$\sigma^2_{\epsilon}$ и $\sigma^2_{v}$.
\begin{enumerate}
\item Покажите, что $\plim (\hat{\beta}_{\text{МНК}}-\beta) = \lambda \sigma^2_{u}/(\lambda^2 \sigma^2_{u} +\sigma^2_{\epsilon})$.
\item Покажите, что $\rho_{XZ}^2 = \gamma \sigma^2_{\epsilon} /(\lambda^2 \sigma^2_{u} +\sigma^2_{\epsilon})\lambda^2 \sigma^2_{\epsilon} +\sigma^2_{v}$.
\item Покажите, что $\hat{\beta}_{IV} = m_{ZY}/m_{ZX} = \beta + m_{ZY}/(\lambda m_{ZU}+m_{ZX})$, где, например, $m_{ZU} = \sum_i z_i y_i$.
\item Покажите, что $\hat{\beta}_{IV}\rightarrow 1/\lambda$ по мере того, как $\gamma$ (или $\rho_{XZ}$) $\rightarrow 0$.
\item  Покажите, что $\hat{\beta}_{IV}\rightarrow \infty$ по мере того, как $m_{ZU} \rightarrow -\gamma \sigma^2_{\epsilon}/\lambda$.
\item К чему приводят последние два результата в контексте смещения в конечной выборке и моментов $\hat{\beta}_{IV}-\beta$, когда инструменты плохие?
\end{enumerate}
\item [$4-8$] Выберите 50\% случайную подвыборку из данных Раздела 4.6.4 с логарифмом расходов на здравоохранение ($y$) и логарифмом совокупных расходов $x$.
\begin{enumerate} 
\item Получите оценки МНК и сопоставьте обычную оценку стандартных ошибок коэффициента наклона и ошибку по Уайту.
\item Получите оценки медианной регрессии и сравните их с оценками МНК. 
\item Получите оценки квантильной регрессии для $q=0.25$ и $q=0.75$.
\item Воспроизведите Рис.4.2, используя свои ответы из частей (a)---(c).
\end{enumerate}
\item [$4-9$] Выберите 50\% случайную подвыборку из данных Раздела 4.9.6 с доходами и образованием, воспроизведите как можно большую часть Таблицы 4.5  и дайте подходящую интерпретацию.
\end{enumerate}

\end{small}



\chapter{Оценивание с помощью метода максимального правдоподобия и
нелинейнего метода наименьших квадратов}

\section{Вступление}

Нелинейная оценка --- это нелинейная функция зависимой переменной. Большинство оценок, используемых в микроэконометрике, кроме оценок метода наименьших квадратов и метода инструментальных переменных в линейной регрессионной модели, представленной в Главе 4, нелинейны. Нелинейность может возникнуть несколькими путями. Условное математическое ожидание может быть нелинейно по параметрам. Функция потерь может привести к нелинейной оценке даже в том случае, когда условное математическое ожидание линейно по параметрам. Усечение и цензурирование также могут привести к нелинейной оценке даже, если условное математическое в первоначальной модели линейно по параметрам.

Здесь мы представляем наиболее важные статистические результаты для нелинейного оценивания. Для нелинейных оценок на конечных выборках доступны очень ограниченные результаты. Большинство статистических методов основаны на асимптотической теории, которая может быть применена для больших выборок. Оценки, которые используются в микроэконометрике, состоятельны и асимптотически нормальны. 

Асимптотическая теория влечёт за собой два главных отличия от работы с линейной регрессионной моделью, которые изложены во вводном курсе. Во-первых, альтернативные методы доказательства необходимы, поскольку нет явной формулы для большинства оценок. Во-вторых, асимптотическое распределение обычно получают при самых слабых предпосылках о распределении. Это отличие было введено в Разделе 4.4 для получения оценок, устойчивых к гетероскедастичности, методом наименьших квадратов. При таких более слабых предпосылках стандартные ошибки, полученные в обычной регрессии по умолчанию, неверны. Однако необходимо быть осторожным, поскольку эти более слабые предпосылки могут привести к несостоятельности оценки самих коэффициентов, что представляет собой более существенную проблему.

По возможности в этой главе мы приводим пояснения. Определения сходимости по вероятности и по распределению, закон больших чисел (ЗБЧ), центральная предельная теорема (ЦПТ) представлены в большом количестве литературы, в этой книге данные темы помещены в Приложение А. В прикладных исследованиях редко ставится цель формально доказать состоятельность и асимптотическую нормальность. Однако можно нередко столкнуться с прикладными исследованиями, в которых есть новые или сложные проблемы с оцениванием, что ведёт к необходимости ознакомления с передовыми статьями в эконометрических журналах. Тогда знание доказательства состоятельности и асимптотической нормальности очень полезно, особенно для того, чтобы заранее хорошо представлять вероятную форму ковариационной матрицы оценки.

В Разделе 5.2 представлен обзор ключевых результатов. Более формальное рассмотрение оценок экстремума, которые максимизируют или минимизируют любую целевую функцию, приведено в Разделе 5.3. Определения оценок, основанных на оценивании уравнений, и сами оценки приведены в Разделе 5.4. Статистические результаты, основанные на стандартных ошибках с поправкой, кратко представлены в Разделе 5.5 с последующей полной интерпретацией в Главе 7. Оценивание методом максимального правдоподобия и методом квази-максимального правдоподобия представлены в Разделах 5.6 и 5.7. Оценивание нелинейным методом наименьших квадратов приведено в Разделе 5.8. В Разделе 5.9 приведён подробный пример. 

Другие важные процедуры параметрического оценивания --- обобщённый метод моментов и нелинейный метод инструментальных переменных --- представлены в Главе 6.

\section{Обзор нелинейных оценок}

В данном разделе кратко изложены асимптотические свойства нелинейных оценок, которые более тщательно разобраны в Разделе 5.3. Также здесь представлены пути интерпретации регрессионных коэффициентов в нелинейных моделях. Материал необходим для понимания моделей для работы с пространственными и панельными данными, которые приведены в последующих главах.

\subsection{Пример пуассоновской регрессии}

Полезно ввести конкретный пример нелинейного оценивания. Здесь мы рассмотрим регрессию Пуассона, которая более подробно проанализирована в Главе 20.

Пуассоновское распределение подходит для зависимой переменной $y$, которая принимает только неотрицательные целые значения $0,1, \dots$. Оно может применяться для моделирования количества произошедших событий, например числа патентных заявок, поданных фирмой, или числа визитов человека к врачу.

Функция вероятности для распределения Пуассона с параметром $\lambda$, выглядит следующим образом:
\[
f(y|\lambda)= e^{-\lambda}\lambda^{y} / y! , y = 0,1,2, \dots,
\]
где может быть показано, что $\E[y]=\lambda$ и $\Var[y]=\lambda$.

Регрессионная модель предполагает, что параметр $\lambda$ меняется между индивидами соответственно определённой функции от вектора регрессоров $x$ и вектора параметров $\beta$. Обычная спецификация Пуассона выглядит так:
\[
\lambda = \exp(x'\beta).
\]
Такая спецификация автоматически обеспечивает $\lambda > 0$. Таким образом в регрессии Пуассона вероятность для отдельного наблюдения равна:
\begin{equation}
f(y|x,\beta)= e^{-\exp(x'\beta)} \exp(x'\beta)^{y}/y!.
\end{equation}

Рассмотрим оценку методом максимального правдоподобия, основанную на выборке $\{(y_i,x_i),i=1, \dots, N \}$. Оценка максимального правдоподобия максимизирует логарифмическую функцию правдоподобия (См. Раздел 5.6). Функция правдоподобия --- это совместная функция плотности или вероятности, которая при условии независимости наблюдений является произведением $\prod_i f(y_i|x_i,\beta)$ индивидуальных функций плотности. Здесь мы говорим об условной вероятности при фиксированных регрессорах. Логарифмическая функция правдоподобия --- это логарифм от произведения, который равен сумме логарифмов, или $\sum_i{\ln f(y_i|x_i,\beta)}$.

Для регрессии Пуассона (5.1) логарифм вероятности для $i$-ого наблюдения следующий:
\[
\ln f(y_i|x_i,\beta)= -\exp(x'_i\beta)+ y_i x'_i \beta - \ln y_i!.
\]

Таким образом, оценка ММП $\hat{\beta}$ в регрессии Пуассона максимизирует
\begin{equation}
\mathcal{Q}_N(\beta)= \frac{1}{N} \sum_{i=1}^{N}\{-\exp(x'_i\beta)+ y_i x'_i \beta - \ln y_i!\},
\end{equation}
где включается нормирование $1/N$ для того, чтобы $\mathcal{Q}_N(\beta)$ оставалось конечным, поскольку $N \rightarrow \infty $. Оценка ММП регрессии Пуассона является решением условия первого порядка $ \left. \partial \mathcal{Q}_N(\beta) / \partial \beta \right|_{\hat{\beta}} = 0$ или
\begin{equation}
\frac{1}{N} \sum_{i=1}^{N}(y_i-\exp(x'_i\beta))x_i |_{\hat{\beta}}=0.
\end{equation}

Не существует явного решения для $\hat{\beta}$ в (5.3). Численные методы для расчёта $\hat{\beta}$ представлены в Главе 10. В данной Главе мы концентрируемся на статистических свойствах полученной оценки $\hat{\beta}$.

\subsection{М-оценки}

В более общем случае мы называем М-оценкой $\hat{\theta}$ вектора параметров $\theta$ размера $q \times 1$ оценку, которая максимизирует целевую функцию, которая представляет собой сумму или среднее $N$ дополнительных функций
\begin{equation}
\mathcal{Q}_N(\theta)= \frac{1}{N} \sum_{i=1}^{N} q(y_i,x_i,\theta),
\end{equation}
где $q(\cdot)$ --- скалярная функция, $y_i$ --- зависимая переменная, $x_i$ --- вектор регрессоров. К тому же во всех результатах, полученных в данном Разделе, предполагается независимость по $i$.

Для простоты $y_i$ записывается в скалярном виде, однако результаты могут быть распространены на случай векторного $y_i$, и соответственно покрывают многомерные и панельные данные, а также системы уравнений. У целевой функции нижний индекс $N$ введен для того, чтобы показать что она зависит от выборочных данных. В данной книге $q$ используется для обозначения размерности $\theta$. Обратите внимание на то, что в данном случае $q$ используется для обозначения дополнительной функции $q(\cdot)$ в (5.4).

Многие эконометрические оценки и модели --- это М-оценки для конкретных функциональных форм $q(y,x,\theta)$. Главными примерами могут послужить метод максимального правдоподобия (см. (5.39) далее) и нелинейный метод наименьших квадратов (НМНК) (см. (5.67) далее). Оценка максимального правдоподобия регрессии Пуассона, которая максимизирует (5.2), является примером (5.4) при $\theta=\beta$ и $q(y,x,\beta)=-\exp(x'\beta)+y x' \beta - \ln y!$.

Мы сконцентрируемся на оценке $\hat{\theta}$, которая является решением условия первого порядка  
$ \partial \mathcal{Q}_N(\theta) / \partial \theta |_{\hat{\theta}} = 0$ или
\begin{equation}
\left. \frac{1}{N} \sum_{i=1}^{N} \frac{\partial q(y_i,x_i,\theta)}{\partial \theta} \right|_{\hat{\theta}}=0.
\end{equation}

Это система из $q$ уравнений с $q$ неизвестными, в которой нет явного решения для $\hat{\theta}$.

Термин М-оценка, который был введён Губером (1967), интерпретируется как сокращение для оценки, похожей на оценку максимального правдоподобия. Многие эконометристы, включая Амэмия (1985, p. 105), Грина (2003, p. 461), и Вулдриджа (2002, p. 344), определяют М-оценку как точку максимума суммы слагаемых, то есть как (5.4). Другие авторы, включая Серфлинга (1980), определяют М-оценку как решение уравнений, как (5.5). Губер (1967) рассматривал оба случая и он (1981, p. 43) явно определил М-оценку для обоих случаев. В данной книге мы берём первое определение для М-оценки, а под вторым мы подразумеваем оценку метода оценочных уравнений (это будет рассмотрено отдельно в Разделе 5.4).

\subsection{Асимптотические свойства М-оценок}

Ключевые асимптотические свойства оценки заключаются в том, что она может быть состоятельна и иметь асимптотическое распределение. Это необходимо для того, чтобы делать статистические  выводы хотя бы для больших выборок.

\begin{center}
Состоятельность
\end{center}

Первый шаг в определении свойств $\hat{\theta}$ заключается в необходимости точно определить, что $\hat{\theta}$ должна оценивать. Мы предполагаем, что существует единственное значение $\theta$, которое обозначается как $\theta_0$ и называется реальным значением параметра, описывающим данные. Данное условие идентификации (см. Раздел 2.5) требует правильной спецификации интересующего процесса, порождающего данные, а также единственности данного представления. Таким образом, для примера с распределением Пуассона можно предположить, что процесс порождающий данные --- это пуассоновское распределение с параметром $\exp(x'\beta_0)$ и регрессор $x$ таков, что $x'\beta^{(1)}=x'\beta^{(2)}$, если и только если $\beta^{(1)}=\beta^{(2)}$. 

Формальная запись с нижним индексом 0 для реального значения параметра широко применяется в Главах с 5 по 8. Это обусловлено тем, что $\theta$ может принимать много разных значений, но интерес представляют два особых значения --- настоящее значение $\theta_0$ и оценочное значение $\hat{\theta}$.

Оцениваемое $\hat{\theta}$ никогда не будет равно $\theta_0$ даже для очень больших выборок из-за случайности выборки. Вместо этого нам необходимо, чтобы $\hat{\theta}$ была состоятельной оценкой для $\theta_0$ (см. определение А.2 в Приложении А), подразумевая под этим, что $\hat{\theta}$ должна сходиться по вероятности к $\theta_0$, что обозначается следующим образом:  $\hat{\theta} \xrightarrow{p} \theta_0$. 

Строго доказать состоятельность М-оценок бывает достаточно затруднительно. Формальные результаты приведены в Разделе 5.3.2, а полезное неформальное условие приведено в 5.3.7. Более конкретно оценки метода максимального правдоподобия и нелинейного метода наименьших квадратов рассматриваются в последующих Разделах.

\begin{center}
Предельное нормальное распределение
\end{center}

При состоятельности, при $N \rightarrow \infty$, оценка $\hat{\theta}$ имеет распределение, сосредоточенное около $\theta_0$. Для метода наименьших квадратов мы увеличиваем или масштабируем $\hat{\theta}$, умножая на $\sqrt{N}$, для того, чтобы получить случайную величину, которая имеет невырожденное распределение при $N \rightarrow \infty$. Затем делаются статистические выводы при предположении о том, что $N$ достаточно большое для того, чтобы асимптотическая теория давала хорошую аппроксимацию, но не такое большое, чтобы $\hat{\theta}$ вырождалось в точку $\theta_0$.

В связи с этим мы анализируем поведение $\sqrt{N}(\hat{\theta} - \theta_0)$. Для большинства оценок распределение в конечной выборке у такой конструкции слишком сложное, чтобы  можно было делать выводы. Вместо этого используется асимптотическая теория, чтобы получить предельное распределение при $N \rightarrow \infty$. Для большинства оценок в микроэконометрике этим пределом является многомерное нормальное распределение. Более формально 
$\sqrt{N}(\hat{\theta} - \theta_0)$ сходится по распределению к многомерному нормальному распределению (определение сходимости по распределению можно найти в Приложении А).

Вспомним из Раздела 4.4, что оценка МНК может быть представлена следующим образом:
\[
N(\hat{\beta} - \beta_0)= \left( \frac{1}{N} \sum_{i=1}^{N} x_i x'_i \right)^{-1} \frac{1}{\sqrt N} \sum_{i=1}^{N} x_i u_i.
\]

Предел распределения был выведен с помощью получения предела по вероятности первого члена справа и предельного нормального распределения второго члена. Предел распределения М-оценки может быть получен аналогичным образом. В Разделе 5.3.3 мы показываем, что для оценки, которая является решением (5.5), всегда верно представление
\begin{equation}
\sqrt{N}(\hat{\theta} - \theta_0)=- \left. \left(\frac{1}{N} \sum_{i=1}^{N} \frac{\partial^2 q_i(\theta)}{\partial \theta \partial \theta'} \right|_{\theta^+} \right)^{-1} \left. \frac{1}{\sqrt N} \sum_{i=1}^{N} \frac{\partial q_i(\theta)}{\partial \theta} \right|_{\theta_0},
\end{equation}
где $q_i(\theta)=q(y_i,x_i,\theta)$ для какого-то $\theta^+$ между $\hat{\theta}$ и $\theta_0$ при условии, что существуют вторые производные и обратная матрица. Этот результат может быть получен разложением в ряд Тейлора.

При подходящих предпосылках это приводит к следующему предельному распределению М-оценки
\begin{equation}
\sqrt{N}(\hat{\theta} - \theta_0) \xrightarrow{d} \mathcal{N}[0,A_0^{-1}B_0 A_0^{-1}],
\end{equation}
где $A_0^{-1}$ --- предел по вероятности первого члена справа в (5.6), и предполагается, что второй член сходится к нормальному распределению $\mathcal{N}[0,B_0]$. Выражения $A_0$ и $B_0$ приведены в Таблице 5.1.

\begin{center}
Асимптотическая нормальность
\end{center}

Чтобы получить распределение $\hat{\theta}$ из результата предельного распределения (5.7), разделим левую часть (5.7) на $\sqrt{N}$, следовательно, разделим дисперсию на $N$. Тогда
\begin{equation}
\hat{\theta} \stackrel{a}{\sim}\mathcal{N}[\theta_0,\Var[\hat{\theta}]],
\end{equation}
где $\stackrel{a}{\sim}$ означает <<асимптотически распределена>>, $\Var[\hat{\theta}]$ --- это асимптотическая дисперсия $\hat{\theta}$ такая, что 
\begin{equation}
\Var[\hat{\theta}]=N^{-1}A_0^{-1}B_0 A_0^{-1}.
\end{equation}
Более полное обсуждение термина <<асимптотическое распределение>> уже было дано в Разделе 4.4.4, а также оно приведено в Разделе А.6.4. 

\begin{table}[h]
\caption{\label{tab:as}Асимптотические свойства М-оценок}
\begin{minipage}{\textwidth}
\begin{tabular}[t]{ll}
\hline
\hline
\bf{Свойство}\footnote{Дисперсия предельного распределения и асимптотическая оценка дисперсии являются робастными сэндвич формами, которые предполагают независимость по $i$. См. Раздел 5.5.2 для других оценок дисперсии.} & \bf{Алгебраическая формула} \\
\hline
Целевая функция &   $\mathcal{Q}_N(\theta)=N^{-1}\sum_i q(y_i,x_i,\theta)$ максимизируется по $\theta$ \\
Примеры &  ММП: $q_i=\ln f(y_i|x_i,\theta)$ --- логарифмическая плотность \\
& НМНК: $q_i= - (y_i-g(x_i,\theta))^2$ --- минус квадрат ошибки \\
Условие первого порядка & 
$ \partial \mathcal{Q}_N(\theta) / \partial \theta =
N^{-1} \sum_{i=1}^{N} \partial q(y_i,x_i,\theta) / \partial \theta |_{\hat{\theta}}=0 $\\
Состоятельность & Максимален ли $\plim \mathcal{Q}_N(\theta)$ при $\theta=\theta_0$? \\
Состоятельность (неформально) & Выполняется ли $\E[\partial q(y_i,x_i,\theta) / \partial \theta |_{\theta_0}]=0$?\\
Предельное распределение & $\sqrt{N}(\hat{\theta} - \theta_0) \xrightarrow{d} \mathcal{N}[0,A_0^{-1}B_0 A_0^{-1}]$ \\
& $A_0=\plim N^{-1} \sum_{i=1}^{N} \partial^2 q_i(\theta) / \partial \theta \partial \theta'|_{\theta_0}$\\
& $B_0=\plim N^{-1} \sum_{i=1}^{N} \partial q_i/ \partial \theta \times \partial q_i/ \partial \theta'|_{\theta_0}$ \\
Асимптотическое распределение &  $\hat{\theta} \stackrel{a}{\sim}\mathcal{N}[\theta_0,N^{-1} \hat{A}^{-1} \hat{B} \hat{A}^{-1}]$\\
& $\hat{A}=N^{-1} \sum_{i=1}^{N} \partial^2 q_i(\theta) / \partial \theta \partial \theta'|_{\hat{\theta}}$\\
& $\hat{B}=N^{-1} \sum_{i=1}^{N} \partial q_i/ \partial \theta \times \partial q_i/ \partial \theta'|_{\hat{\theta}}$ \\
\hline
\hline
\end{tabular}
\end{minipage}
\end{table}

Результат (5.9) зависит от неизвестного истинного значения параметра $\theta_0$. На практике вычисляют оценку асимптотической дисперсии
\begin{equation}
\widehat{\Var}[\hat{\theta}]=N^{-1} \hat{A}^{-1} \hat{B} \hat{A}^{-1},
\end{equation}
где $\hat{A}$ и $\hat{B}$ --- это состоятельные оценки для $A_0$ и $B_0$.

Вместо этого многие статистические пакеты по умолчанию часто используют более простую оценку $\widehat{\Var}[\hat{\theta}]=-N^{-1}\hat{A}^{-1}$, которую правомерно использовать только в некоторых особых случаях. См. раздел 5.5 для дальнейшего обсуждения различных способов оценки $A_0$ и $B_0$ и проверки гипотез.

Два важных примера М-оценок --- это оценки ММП и НМНК. Формальные результаты для этих оценок представлены, соответственно, утверждениями 5.5 и 5.6. Более простые представления асимптотических распределений этих оценок даны в (5.48) и (5.77).

\begin{center}
Пример ММП для распределения Пуассона 
\end{center}

Как и другие оценки ММП, оценка ММП для распределения Пуассона состоятельна, если плотность правильно специфицирована. Однако, подставляя (5.25) из Раздела 5.3.7 в (5.3), видно, что необходимым условием состоятельности является на самом деле более слабое условие  $\E[y|x]=\exp(x'\beta_0)$, что есть правильная спецификация математического ожидания. Похожая робастность к частично неправильной спецификации оценки ММП справедлива и для некоторых других особых случаев, они подробно описаны в Разделе 5.7.

Оценка ММП для распределения Пуассона $\partial q(\beta) / \partial \beta= (y-\exp(x'\beta_0))x$ приводит к 
\[
A_0=-\plim N^{-1} \sum_i \exp(x'_i\beta_0)x_i x'_i
\]
и
\[
B_0=\plim N^{-1} \sum_i \Var[y_i|x_i] x_i x'_i.
\]
 
Тогда $\hat{\beta} \stackrel{a}{\sim}\mathcal{N}[\theta_0,N^{-1} \hat{A}^{-1} \hat{B} \hat{A}^{-1}]$, где $\hat{A}=- N^{-1} \sum_i \exp(x'_i \hat{\beta})x_i x'_i$ и $\hat{B}=N^{-1} \sum_i (y_i - \exp(x'_i \hat{\beta}))^2 x_i x'_i$.

\begin{table}[h]
\begin{center}
\caption{\label{tab:pred3est}Предельный эффект: три разные оценки}
\begin{tabular}[t]{ll}
\hline
\hline
\bf{Формула} & \bf{Описание} \\
\hline
$N^{-1} \sum_i \partial \E[y_i|x_i] / \partial x_i$ & Средний эффект по всем индивидам \\
$\partial \E[y|x] / \partial x |_{\bar{x}}$ & Предельный эффект для среднего человека \\
$\partial \E[y|x] / \partial x |_{x^*}$ & Предельный эффект для человека с $x=x^*$\\
\hline
\hline
\end{tabular}
\end{center}
\end{table}

Если данные на самом деле распределены по Пуассоновскому закону, тогда $\Var[y|x]=\E[y|x]= \exp(x'\beta_0)$, что ведёт к возможному упрощению, поскольку $A_0$=$-B_0$, то есть $A_0^{-1}B_0 A_0^{-1}=-A_0^{-1}$. Однако в большинстве приложений с численными данными $\Var[y|x]>\E[y|x]$, поэтому лучше не вводить это ограничение.

\subsection{Интерпретация коэффициентов нелинейной регрессии}

Важной целью оценивания часто являются предсказания, а не проверка статистической значимости регрессоров.

\begin{center}
Предельные эффекты
\end{center}

Часто интерес заключается в измерении предельных эффектов, то есть в изменении условного математического ожидания $y$, когда регрессор $x$ изменяется на единицу.

Для линейной регрессионной модели, $\E[y|x]= x'\beta$ подразумевает $\partial \E[y|x] / \partial x = \beta$ для того, чтобы коэффициент имел прямую интерпретацию как предельный эффект. Для нелинейных регрессионных моделей эта интерпретация не применима. Например, если $\E[y|x]= \exp(x'\beta)$, то $\partial \E[y|x] / \partial x = \exp(x'\beta)\beta$ является функцией двух параметров и регрессоров, и размер предельного эффекта зависит и от $\beta$, и от $x$.

\begin{center}
Общее уравнение регрессии
\end{center}

Для общего уравнения регрессии
\[
\E[y|x]= g(x, \beta),
\]
предельный эффект меняется со значением $x$.

Общепринято представлять одну из оценок предельного эффекта, приведенных в Таблице 5.2. Первая оценка усредняет предельные эффекты для всех индивидов. Вторая оценка вычисляет предельный эффект при $x=\bar{x}$. Третья оценка вычисляет предельный эффект при определённых характеристиках $x=x^*$. Например, $x^*$ может представлять человека, который является женщиной с 12-ью годами обучения в школе и так далее. Может быть рассмотрен более, чем один репрезентативный индивид.

Эти три показателя отличаются в нелинейных моделях, в то время как в линейной модели все они равны $\beta$. Даже знак эффекта может быть не связан со знаком параметра. Значение $\partial \E[y|x] / \partial x_j$ может быть положительным для некоторых значений $x$ и отрицательным для других значений $x$. Значительное внимание должно быть уделено интерпретации коэффициентов в нелинейных моделях.

Компьютерные программы и прикладные исследования часто приводят второй из этих предельных эффектов. Это может быть полезно для того, чтобы определить величину предельного эффекта, но при принятии решений важен общий эффект, то есть первый показатель, или эффект на репрезентативного индивида или группу, то есть третий показатель. Первый показатель имеет тенденцию изменяться относительно мало для различных функциональных форм $g(\cdot)$, в то время как два другие показателя могут изменяться существенно. Можно также представить полное распределение предельных эффектов, используя гистограмму или непараметрическую оценку плотности.

\begin{center}
Одноиндексные модели
\end{center}

Прямая интерпретация коэффициентов регрессии возможна для одноиндексных моделей, которые определяют
\begin{equation}
\E[y|x]= g(x'\beta),
\end{equation}
то есть данные и параметры входят в нелинейную функцию среднего $g(\cdot)$ в виде одного индекса $x'\beta$. Здесь мы видим нелинейность в легкой форме --- ожидание является нелинейной функцией линейной комбинации регрессоров и параметров. Для одноиндексных моделей эффект от изменения в $j$-ом регрессоре получается взятием производной:
\[
\frac{\partial \E[y|x]}{\partial x_j}=g'(x'\beta)\beta_j,
\]
где $g'(z)= \partial g(z)/\partial z$. Из этого следует, что относительный эффект изменений в регрессорах задаётся отношением коэффициентов, поскольку
\[
\frac{\partial \E[y|x] / \partial x_j}{\partial \E[y|x] / \partial x_k}=\frac{\beta_j}{\beta_k},
\]
потому что общий член $g'(x'\beta)$ сокращается. Таким образом, если $\beta_j$ равен удвоенному $\beta_k$, тогда изменение на одну единицу $x_j$ имеет в два раза больший эффект, чем эффект при изменении на одну единицу $x_k$. Если $g(\cdot)$ ещё и монотонна, то из этого следует, что знаки коэффициентов дают знаки эффектов, для всех возможных $x$.

Одноиндексные модели полезны из-за их простой интерпретации. Многие стандартные нелинейные модели, таких как логит, пробит, и тобит имеют одноиндексную форму. Кроме того, некоторые функции $g(\cdot)$ позволяют дать дополнительную интерпретацию, особенно экспоненциальная функция, которая будет рассмотрена далее в этом Разделе, и логистическая функция распределения, проанализированная в Разделе 14.3.4.

\begin{center}
Метод конечных разностей
\end{center}

Мы подчеркнули использование методов математического анализа. Вместо взятия производной метод конечных разностей вычисляет предельный эффект путём сравнения условного среднего, когда $x_j$ увеличивается ровно на одну единицу, со значением до увеличения. Таким образом,
\[
\frac{\Delta \E[y|x]}{ \Delta x_j}=g(x+e_j,\beta)-g(x,\beta),
\]
где $e_j$ --- это вектор, где на $j$-ом месте стоит единица, а на остальных --- нули.

Для линейной модели конечно-разностный метод и взятие производной приводят к тем же оцениваемым эффектам, поскольку $\Delta \E[y|x] / \Delta x_j = (x'\beta + \beta_j)-x'\beta = \beta_j$. Однако для нелинейных моделей эти два подхода дают одинаковые оценки предельного эффекта, только если изменение $x_j$ очень малое.

Дифференциирование часто применяется для непрерывных регрессоров, а конечно-разностный метод используется для целочисленных регрессоров таких, как (0, 1) переменной-индикатора.

\begin{center}
Экспоненциальное условное математическое ожидание
\end{center}

В качестве примера рассмотрим интерпретацию коэффициента функции экспоненциального условного среднего, т.е. $\E[y|x]= \exp(x'\beta)$. Многие счетные модели и модели длительности используют экспоненциальную форму.

Небольшое алгебраическое преобразование приводит к тому, что $\partial \E[y|x] / \partial x_j=\E[y|x]\times \beta_j$. Таким образом, параметры могут быть интерпретированы как полу-эластичности, с изменением $x_j$ на единицу условное математическое ожидание увеличивается в $\beta_j$ раз. Например, если $\beta_j$=0.2, то изменение $x_j$ на единицу, по прогнозам, приведёт к пропорциональному увеличению $\E[y|x]$ на 0.2 раза, или увеличению на 20$\%$.

Если же наоборот используется конечно-разностный метод, то предельный эффект высчитывается как $\Delta \E[y|x] / \Delta x_j = \exp(x'\beta + \beta_j)-\exp(x'\beta) = \exp(x'\beta)(e^{\beta_j}-1)$. Это не отличается от результата, полученного с помощью производной, если только $\beta_j$ настолько малое, что $e^{\beta_j} \backsimeq 1+\beta_j$. Например,  $\beta_j$=0.2, то увеличение составляет 22.14$\%$, а не 20$\%$.

\section{Экстремальные оценки}

Данный раздел предназначен для продвинутого курса по микроэконометрике. В нём представлены основные результаты по состоятельности и асимптотической нормальности экстремальных оценок, очень общего класса оценок, которые минимизируют или максимизируют целевую функцию. Представленные результаты очень сжаты. Более полное понимание требует большего внимания как, например, у Амэмия (1985), где представлены основы, или у Ньюи и МакФаддена (1994).

\subsection{Экстремальные оценки}

Для анализа пространственных данных по одной зависимой переменной, выборка состоит из $N$ наблюдений, 
$\{(y_i,x_i),i=1,\cdots,N\}$, зависимой переменной $y_i$ и вектора-столбца регрессоров $x_i$. 
В матричном виде --- $(y,X)$, где $y$ представляет собой вектор размера $N \times 1$ с $i$-ым значением $y_i$, и $X$ представляет собой матрицу с $i$-ой строкой $x_i$, как это определено более подробно в Разделе 1.6.

Интерес заключается в оценке вектора параметров $\theta=[\theta_1 \cdots \theta_q]'$ размера $q \times 1$. Значением $\theta_0$, называемого истинным значением параметра, является конкретное значение $\theta$ в процессе, порождающем данные.

Мы считаем оценки $\hat{\theta}$, которые максимизируют по $\theta_0 \in \Theta$ стохастическую целевую функцию $\mathcal{Q}_N(\theta)=\mathcal{Q}_N(y,X,\theta)$, где для простоты обозначения зависимость $\mathcal{Q}_N(\theta)$ от данных выражена за счёт нижнего индекса $N$. Такие оценки называются экстремальными оценками, так как они решают максимизационную или минимизационную задачу.

Экстремальная оценка может быть глобальным максимумом, так
\begin{equation}
\hat{\theta}= \argmax _{\theta \in \Theta} \mathcal{Q}_N(\theta)
\end{equation}

Обычно  экстремальная оценка --- локальный максимум, который вычисляется как решение связанных условий первого порядка
\begin{equation}
\left. \frac{\partial \mathcal{Q}_N(\theta)}{\partial \theta} \right|_{\hat{\theta}}=0,
\end{equation}
где $\partial \mathcal{Q}_N(\theta)/ \partial \theta$ --- вектор-столбец размера $q \times 1$, в котором $k$-ый член --- это $\partial \mathcal{Q}_N(\theta)/ \partial \theta_k$. Обратим внимание на локальный максимум, потому что именно локальный максимум может быть распределён асимптотически нормально. Локальный и глобальный максимумы совпадают, если $\mathcal{Q}_N(\theta)$ глобально вогнута.

Есть два основных примера экстремальных оценок. Для М-оценок, рассмотренных в этой главе, в частности, для ММП и НМНК оценок, $\mathcal{Q}_N(\theta)$ является средним арифметическим, например, средним квадратов остатков. В обобщенном методе моментов оценка (см. Раздел 6.3) $\mathcal{Q}_N(\theta)$ является квадратичной формой выборочных средних.

Для определенности речь пойдёт об одном уравнении для пространственной регрессии. Но результаты носят довольно общий характер и могут быть применимы к любым оценкам, основанным на оптимизации со свойствами, приведенными в данном разделе. В частности, нет никаких ограничений на скалярную зависимую переменную и некоторые авторы используют обозначение $z_i$ вместо $(y_i,x_i)$. Тогда $\mathcal{Q}_N(\theta)$ равна $\mathcal{Q}_N(Z,\theta)$, а не $\mathcal{Q}_N(y,X,\theta)$.

\subsection{Формальные теоремы о состоятельности}

Рассмотрим сначала спецификацию параметров, введенную в Разделе 2.5. Интуитивно параметр $\theta$ идентифицируем, если распределение данных или особенность распределения, определяется с помощью $\theta_0$, в то время как любые другие значения $\theta$ приводят к другому распределению. Например, в линейной регрессии мы потребовали, чтобы $\E[y|X]=X\beta_0$ и $X\beta^{(1)}=X\beta^{(2)}$ тогда и только тогда, когда $\beta^{(1)}=\beta^{(2)}$.

Процедура оценки может не идентифицировать $\theta_0$. Например, это имеет место, если процедура оценки пропускает значимые регрессоры. Будем говорить, что метод оценки идентифицирует $\theta_0$, если предел по вероятности целевой функции, взятый согласно процессу порождающему данные с параметром $\theta=\theta_0$, достигает максимума только при $\theta=\theta_0$. Это условие идентификации является асимптотическим. Практические проблемы оценки, которые могут возникнуть в конечной выборке, рассматриваются в Главе 10.

Состоятельность определяется следующим образом. При $ N \rightarrow \infty $ стохастическая целевая функция $\mathcal{Q}_N(\theta)$, средняя в случае М-оценки, может сходиться по вероятности к предельной функции, обозначаемой $\mathcal{Q}_0(\theta)$, которая в простейшем случае не является стохастической. Соответствующие максимумы (глобальной или локальной) из $\mathcal{Q}_N(\theta)$ и $\mathcal{Q}_0(\theta)$ тогда должны существовать для значений $\theta$, близких друг к другу. Поскольку максимум $\mathcal{Q}_N(\theta)$ --- это $\hat{\theta}$ по определению, то отсюда следует, что $\hat{\theta}$ сходится по вероятности к $\theta_0$ при условии, что $\theta_0$ максимизирует $\mathcal{Q}_0(\theta)$.

Очевидно, что состоятельность и идентифицируемость тесно связаны, и у Амэмия (1985, с. 230) говорится, что наиболее простой подход заключается в том, чтобы под идентифицируемостью понимать существование состоятельной оценки. Для дальнейшего обсуждения см. Ньюи и МакФаддена (1994, с. 2124) и Дейстлера и Зейферта (1978).

Основные области применения этого подхода можно найти в работах Дженриха (1969) и Амэмия (1973). Амэмия (1985) и Ньюи и МакФадден (1994) представляют достаточно общие теоремы. Эти теоремы используют разные предпосылки, в том числе о гладкости (непрерывности) и существовании необходимых производных целевой функции, предпосылки о процессе порождающем данные для обеспечения сходимости $\mathcal{Q}_N(\theta)$  к $\mathcal{Q}_0(\theta)$, и максимальном значении $\mathcal{Q}_0(\theta)$  при $\theta=\theta_0$. Разные теоремы о состоятельности используют различные предпосылки.

Приведём две теоремы о состоятельности из Амэмия (1985), одну для глобального максимума и одну для локального максимума. Обозначения в теоремах Амэмия были изменены, поскольку он (1985) определяет целевую функцию без нормирования $1/N$, которая присутствует, например, в (5.4).

\begin{theorem}[Состоятельность глобального максимума] (Амэмия, 1985, теорема 4.1.1): 
Сделаем следующие предположения:
\begin{enumerate}
\item Множество возможных значений параметра $\Theta$ --- компактное подмножество в $R^q$.
\item Целевая функция $\mathcal{Q}_N(\theta)$ является измеримой функцией от данных для всех $\theta \in \Theta$, и $\mathcal{Q}_N(\theta)$ непрерывна при $\theta \in \Theta$.
\item $\mathcal{Q}_N(\theta)$ равномерно сходится по вероятности к нестохастический функции $\mathcal{Q}_0(\theta)$, и $\mathcal{Q}_0(\theta)$ достигает единственного глобального максимума в $\theta_0$.
\end{enumerate}

Тогда оценка $\hat{\theta}= \arg\max _{\theta \in \Theta} \mathcal{Q}_N(\theta)$ состоятельна для $\theta_0$, то есть $ \hat{\theta} \xrightarrow{p} \theta_0$.
\end{theorem}


Равномерная сходимость по вероятности $\mathcal{Q}_N(\theta)$ к 
\begin{equation}
\mathcal{Q}_0(\theta)=\plim \mathcal{Q}_N(\theta)
\end{equation}
в третьем условии означает, что $sup _{\theta \in \Theta} |\mathcal{Q}_N(\theta)-\mathcal{Q}_0(\theta)| \xrightarrow{p} 0$.

Для локального максимума должны существовать первые производные, но затем нужно рассматривать поведение $\mathcal{Q}_N(\theta)$ и её производной в окрестности $\theta_0$.

\begin{theorem}[Состоятельность локального максимума] (Амэмия, 1985, теорема 4.1.2): 
Сделаем следующие предположения:
\begin{enumerate}
\item Множество возможных значений параметра $\Theta$ --- открытое подмножество $R^q$.
\item $\mathcal{Q}_N(\theta)$ является измеримой функцией от данных для всех $\theta \in \Theta$, и  $\partial \mathcal{Q}_N(\theta) / \partial \theta $ существует и непрерывна в открытой окрестности $\theta_0$.
\item  Целевая функция $\mathcal{Q}_N(\theta)$ равномерно сходится по вероятности к $\mathcal{Q}_0(\theta)$ в открытой окрестности $\theta_0$, и  $\mathcal{Q}_0(\theta)$  достигает единственного локального максимума в $\theta_0$.
\end{enumerate}

Тогда одно из решений $\partial \mathcal{Q}_N(\theta) / \partial \theta =0 $ состоятельно для $\theta_0$. 
\end{theorem}

Пример применения теоремы 5.2 приведён далее в Разделе 5.3.4.

Условие 1 в теореме 5.1 позволяет глобальному максимуму быть на границе множества значений параметров, в то время как в теореме 5.2 локальный максимум должен быть строго внутренней точка множества возможных значений параметров. Условие 2 в теореме 5.2 также подразумевает непрерывность $\mathcal{Q}_N(\theta)$ в открытой окрестности $\theta_0$. Окрестность $N(\theta_0)$ $\theta_0$ открыта тогда и только тогда, когда существует шар с центром $\theta_0$, которая целиком содержится в $N(\theta_0)$ $\theta_0$. В обеих теоремах условие 3 является необходимым. Максимум, глобальный или локальный,  $\mathcal{Q}_0(\theta)$ должен достигаться при $\theta=\theta_0$. Вторая часть условия 3 --- это условие идентифицируемости $\theta_0$: это значение должно иметь  содержательную интерпретацию и быть единственным.

Для локального максимума анализ прост, если существует только один локальный максимум. Тогда 
$\hat{\theta}$ определяется единственным образом как $\partial \mathcal{Q}_N(\theta) / \partial \theta|_{\hat{\theta}}=0$. Когда существует больше одного локального максимума, теорема говорит, что один из локальных максимумов состоятельный, но не даёт указаний, какой конкретно является состоятельным. Лучше всего в таких случаях рассматривать глобальный максимум и применять теорему 5.1. См. Ньюи и МакФаддена (1994, с. 2117) для обсуждения.

Нужно обратить внимание на разницу между спецификацией модели, отражающееся в выборе целевой функции $\mathcal{Q}_N(\theta)$ и фактическим процессом порождающим данные $(y,X)$, используемом при получении $\mathcal{Q}_0(\theta)$ в (5.14). Для некоторых процессов порождающих данные оценка может быть состоятельной, тогда как для других --- нет. В некоторых случаях таких, как для оценок ММП Пуассона и МНК, состоятельность возникает в широком классе процессов порождающих данные при условии, что правильно специфицировано условное математическое ожидание. В других случаях состоятельность требует более сильных предположений о процессе порождающем данные, например, правильную спецификацию плотности.

\subsection{Асимптотическая нормальность}

Результаты по асимптотической нормальности, как правило, ограничиваются локальным максимумом $\mathcal{Q}_N(\theta)$. Оценка $\hat{\theta}$ является решением (5.13), которое в общем случае нелинейно по $\hat{\theta}$ и не имеет явного решение для $\hat{\theta}$. Вместо этого, мы заменим левую часть этого уравнения линейной функцией от $\hat{\theta}$ путём использования разложения в ряд Тейлора, а затем решим его для $\hat{\theta}$.

Наиболее часто используемый вариант теоремы Тейлора --- это приближение с остаточным членом. Здесь мы вместо этого рассмотрим точное разложение Тейлора первого порядка. Для дифференцируемой функции $f(\cdot)$ всегда существует точка $x^+$ между $x$ и $x_0$ такая, что
\[
f(x)=f(x_0)+f'(x^+)(x-x_0),
\]
где $f'(x)=\partial f(x) / \partial x$ --- производная $f(x)$. Это результат также известен как теорема о среднем значении.

Применение к текущей задаче требует ряда изменений. Скалярная функция $f(\cdot)$ заменяется на векторную функции $\mathfrak{f}(x)$, и скалярные аргументы $x$, $x_0$ и $x^+$ заменяются векторами $\hat{\theta}$, $\theta_0$ и $\theta^+$. Тогда 
\begin{equation}
\left. \mathfrak{f}(\hat{\theta})=\mathfrak{f}(\theta_0)+\frac{\partial \mathfrak{f}(\theta)}{\partial \theta} \right|_{\theta^+} (\hat{\theta}-\theta_0),
\end{equation}
где $ \partial \mathfrak{f}(\theta)/ \partial \theta$ является матрицей для некоторого $\theta^+$ между $\hat{\theta}$ и $\theta_0$, и формально $\theta^+$ отличается для каждой строки этой матрицы (см. Ньюи и МакФадден, 1994, с. 2141). Для оценки локального экстремума функция $\mathfrak{f}(\theta)= \partial \mathcal{Q}_N(\theta) / \partial \theta$ --- это уже первая производная. Тогда точное разложение в ряд Тейлора первого порядка в окрестности $\theta_0$:
\begin{equation}
\left. \frac{\partial \mathcal{Q}_N(\theta)}{\partial \theta} \right|_{\hat{\theta}} = \left. \frac{\partial \mathcal{Q}_N(\theta)}{\partial \theta} \right|_{\theta_0} + \left. \frac{\partial^2 \mathcal{Q}_N(\theta)}{\partial \theta \partial \theta'} \right|_{\theta^+} (\hat{\theta}-\theta_0),
\end{equation}
где $\partial^2 \mathcal{Q}_N(\theta) / \partial \theta \partial \theta'$ --- это матрица размера $q \times q$ c $(j,k)$-ым элементом равным $\partial^2 \mathcal{Q}_N(\theta) / \partial \theta_j \partial \theta_k$, а $\theta^+$ лежит между $\hat{\theta}$ и $\theta_0$.

Условие первого порядка подразумевает, что левая часть (5.16) приравнивается к 0, тогда правая часть тоже приравнивается к 0 и решение для $(\hat{\theta}-\theta_0)$ следующее:
\begin{equation}
\left. \sqrt{N}(\hat{\theta} - \theta_0)=- \left( \frac{\partial^2 \mathcal{Q}_N(\theta)}{\partial \theta \partial \theta'} \right|_{\theta^+} \right) ^{-1} \sqrt{N} \left. \frac{\partial \mathcal{Q}_N(\theta)} {\partial \theta} \right|_{\theta_0},
\end{equation}
где мы нормируем на $\sqrt{N}$, чтобы гарантировать невырожденное предельное распределение (что обсуждается далее).

Результат (5.17) содержит решение для $\hat{\theta}$. Это не нужно для численного нахождения $\hat{\theta}$, поскольку решение зависит от $\theta_0$ и $\theta^+$, оба из которых неизвестны, но это полезно для теоретического анализа. В частности, если было установлено, что $\hat{\theta}$ состоятельна для $\theta_0$, тогда неизвестная $\theta^+$ сходится по вероятности к $\theta_0$, потому что лежит между $\hat{\theta}$ и $\theta_0$, а из-за состоятельности $\hat{\theta}$ сходится по вероятности к $\theta_0$. 

Из результата (5.17) $\sqrt{N}(\hat{\theta} - \theta_0)$ выражается в виде, аналогичном тому, который используется для получения предельного распределения МНК-оценки (см. Раздел 5.2.3). Всё, что нам нужно сделать, --- это предположить, что существуют предел по вероятности для первого члена в правой части (5.17) и предельное нормальное распределение для второго члена.

Это приводит к теореме из Амэмия (1985) об экстремальной оценке, удовлетворяющей локальному максимуму. Снова отметим, что Амэмия (1985) определяет целевую функцию без нормирования $1/N$. Кроме того, Амэмия определяет $A_0$ и $B_0$ как $\lim E$, а не $\plim$.

\begin{theorem}[Предельное распределение локального максимума] (Амэмия, 1985, теорема 4.1.3): В дополнение к условиям предыдущей теоремы о состоятельности локального максимума сделаем следующие предположения:
\begin{enumerate}
\item $\partial^2 \mathcal{Q}_N(\theta) / \partial \theta \partial \theta'$ существует и непрерывна в открытой выпуклой окрестности $\theta_0$.
\item $\partial^2 \mathcal{Q}_N(\theta) / \partial \theta \partial \theta'|_{\theta^+}$ сходится по вероятности к конечной невырожденной матрице
\begin{equation}
A_0=\plim \partial^2 \mathcal{Q}_N(\theta) / \partial \theta \partial \theta'|_{\theta_0}
\end{equation}
для любой последовательности $\theta^+$ такой, как $\theta^+ \xrightarrow{p} \theta_0$.
\item $\sqrt{N} \partial \mathcal{Q}_N(\theta) / \partial \theta |_{\theta_0} \xrightarrow{d} \mathcal{N}[0,B_0]$, где
\begin{equation}
B_0=\plim[N \partial \mathcal{Q}_N(\theta) / \partial \theta \times \partial \mathcal{Q}_N(\theta) / \partial \theta'|_{\theta_0}].
\end{equation}
\end{enumerate}
Тогда предельное распределение экстремальной оценки
\begin{equation}
\sqrt{N}(\hat{\theta}-\theta_0) \xrightarrow{d} \mathcal{N}[0,A_0^{-1}B_0A_0^{-1}],
\end{equation}
где оценка $\hat{\theta}$ --- состоятельное решение $\partial \mathcal{Q}_N(\theta) / \partial \theta=0$.
\end{theorem}

Доказательство непосредственно вытекает из теоремы о нормальности предела произведения (теорема А.17), применяемого к (5.17). Отметим, что доказательство предполагает, что состоятельность $\hat{\theta}$ уже была установлена. Выражения для $A_0$ и $B_0$, приведённых в таблице 5.1, --- уточнения для случая $\mathcal{Q}_N(\theta)=N^{-1} \sum_i q_i(\theta)$ при условии независимости по $i$.

Пределы по вероятности в (5.18) и (5.19) получены с учетом процесса порождающего данные для $(y,X)$. В некоторых приложениях предполагается, что регрессоры нестохастические и математическое ожидание берется только по $y$. В других случаях рассматриваются стохастические регрессоры и математическое ожидание берется и по $y$, и по $X$.

\subsection{Пример с асимптотическими свойствами ММП-оценки Пуассона}

Мы формально доказываем состоятельность и асимптотическую нормальность  ММП-оценки Пуассона при экзогенной стратифицированной выборке со стохастическими регрессорами, где $(y_i,x_i)$ независимы но неодинаково распределены. При этом необязательно предполагать, что $y_i$ распределены по Пуассону.

Ключевыми моментами для доказательства состоятельности являются получение $\mathcal{Q}_0(\beta)=\plim \mathcal{Q}_N(\beta)$ и проверка того, что $\mathcal{Q}_0(\beta)$ достигает максимума в точке $\beta=\beta_0$. Для $\mathcal{Q}_N(\beta)$, которая была определена в (5.1), получается
\[
\mathcal{Q}_0(\beta)= \plim N^{-1} \sum_i \{-e^{x'_i\beta}+y_i x'_i \beta - \ln y_i! \}
\]
\[
= \lim N^{-1} \sum_i \{-\E[e^{x'_i\beta}]+\E[y_i x'_i \beta] - \E[\ln y_i!] \} \\
\]
\[
= \lim N^{-1} \sum_i \{-\E[e^{x'_i\beta}]+\E[e^{x'_i \beta_0} x'_i \beta] - \E[\ln y_i!] \}
\]

Второе равенство предполагает, что закон больших чисел может быть применен к каждому члену. Так как $(y_i,x_i)$ независимы и неодинаково распределены, закон больших чисел Маркова (теорема A.8) может быть применён, если каждая из ожидаемых величин, приведённых во второй строке, дополнительно существует и соответствующий $(1+\delta)$-й абсолютный момент существует для некоторого $\delta>0$ и граничное условие в теореме А.8 выполнено. Например, допустим, что $\delta=1$, так чтобы были применены моменты второго порядка. Для третьей строки требуется предположение, что процесс порождающий данные является таким, что $\E[y|x]=\exp(x'\beta_0)$. Первые два математических ожидания в третьей строке зависят от $x$, который является стохастическим. Обратите внимание, что $\mathcal{Q}_0(\beta)$ зависит и от $\beta$, и от $\beta_0$. Дифференцируя по $\beta$, и предполагая, что пределы, производные и математические ожидания могут быть поменены местами, мы получаем
\[
\frac{\partial \mathcal{Q}_0(\beta)}{\partial \beta}= -\lim N^{-1} \sum_i \E[e^{x'_i \beta} x_i] + \lim N^{-1} \sum_i \E[e^{x'_i \beta_0} x_i],
\]
где производная $\E[\ln y!]$ по $\beta$ равна 0, поскольку $\E[\ln y!]$ зависит от $\beta_0$, истинного значения параметра в процессе порождающем данные, но не от $\beta$. Очевидно, что $\partial \mathcal{Q}_0(\beta)/ \partial \beta=0$ при $\beta=\beta_0$ и $\partial^2 \mathcal{Q}_0(\beta) / \partial \beta \partial \beta' = -\lim N^{-1} \sum_i \E[e^{x'_i \beta} x_i x'_i]$ отрицательно определена, поэтому $\mathcal{Q}_0(\beta)$ достигает локального максимума при $\beta=\beta_0$, и ММП-оценка Пуассона состоятельна по теореме 5.2. Так как здесь $\mathcal{Q}_N(\beta)$ глобально вогнута, то локальный максимум совпадает с глобальным максимумом, и состоятельность также может быть показана с использованием теоремы 5.1.

Перейдем к асимптотической нормальности ММП-оценки Пуассона. Точное  разложение Тейлора до первого члена условия первого порядка ММП-оценки Пуассона (5.3) даёт:

\begin{equation}
\sqrt{N}(\hat{\beta}-\beta_0)=-\left[-N^{-1} \sum_i e^{x'_i \beta^+} x_i x'_i \right]^{-1} N^{-1/2} \sum_i (y_i-e^{x'_i \beta_0}) x_i,
\end{equation}
для некоторой неизвестной $\beta^+$, лежащей в промежутке между $\hat{\beta}$ и $\beta_0$.

Сделав достаточные предположения о регрессорах $x$, чтобы было можно применить закон больших Маркова к первому члену, и, используя $\beta^+ \xrightarrow{p} \beta_0$, поскольку $\hat{\beta} \xrightarrow{p} \beta_0$, получаем
\begin{equation}
-N^{-1} \sum_i e^{x'_i \beta^+} x_i x'_i \xrightarrow{p} A_0 = -\lim N^{-1} \sum_i \E[e^{x'_i \beta_0} x_i x'_i].
\end{equation}
Для второго члена в (5.21) начнём с предположения о скалярном регрессоре $x$. Тогда у $X=(y-\exp(x \beta_0))x$ математическое ожидание $\E[X]=0$, поскольку для состоятельности было уже введено предположение о $\E[y|x]=\exp(x \beta_0)$, и дисперсия $\Var[X]=\E[\Var[y|x]x^2]$. Центральная предельная теорема Ляпунова (теорема A.15) может быть применена, если граничное условие, включающее абсолютный момент $y-\exp(x \beta_0)x$ порядка $(2+\delta)$, выполнено. В данном примере с $y>0$ достаточно предположить, что момент третьего порядка для $y$ существует, то есть $\delta=1$ и $x$ ограничен. Применение ЦПТ даёт следующее:
\[
Z_N=\frac{\sum_i (y_i-e^{\beta_0 x_i})x_i}{\sqrt{\sum_i \E[\Var[y_i|x_i]x_i^2]}} \xrightarrow{d} \mathcal{N}[0,1],
\]
поэтому 
\[
N^{-1/2} \sum_i (y_i-e^{\beta_0 x_i})x_i \xrightarrow{d} \mathcal{N}\left[0,\lim N^{-1}\sum_i \E[\Var[y_i|x_i]x_i^2]\right],
\]
предполагая, что в выражении предел асимптотической дисперсии существует. Этот результат может быть распространен на случай вектора регрессоров, используя теорему Крамера-Вольда (см. теорему A.16). Тогда
\begin{equation}
N^{-1/2} \sum_i (y_i-e^{x'_i \beta_0})x_i \xrightarrow{d} \mathcal{N} \left[0,B_0= \lim N^{-1} \E[\Var[y_i|x_i]x_i x'_i]\right].
\end{equation}

Таким образом из (5.21) следует, что $\sqrt{N}(\hat{\beta}-\beta_0) \xrightarrow{d} \mathcal{N}[0,A_0^{-1}B _0 A_0^{-1}]$, где $A_0$ было определено в (5.22), а $B_0$ в (5.23).

Отметим, что для этого конкретного примера нет необходимости в том, чтобы $y|x$ была распределена по Пуассону для того, чтобы ММП-оценка Пуассона была состоятельна и асимптотически нормальна. Существенным предположением для состоятельности ММП-оценки Пуассона является то, что процесс порождающий данные таков, что $\E[y|x]=\exp(x' \beta_0)$.

Для асимптотической нормальности существенным предположением является то, что $\Var[y|x]$ существует, хотя дополнительные предположения о существовании моментов высших порядков необходимы, чтобы было можно применять ЗБЧ и ЦПТ. Если на самом деле $\Var[y|x]=\exp(x' \beta_0)$, то $A_0=-B_0$ и проще $\sqrt{N}(\hat{\beta}-\beta_0) \xrightarrow{d} \mathcal{N}[0,-A_0^{-1}]$. Результаты этого примера ММП распространяются на экспоненциальное семейство распределений, которое определены в Разделе 5.7.3.

\subsection{Доказательства состоятельности и асимптотической нормальности}

Предположения, сделанные в теоремах 5.1-5.3, носят довольно общий характер и не обязательно должны присутствовать при каждом применении. Эти предположения должны быть проверены в каждом конкретном случае на индивидуальной основе, по аналогии с предыдущим примером ММП-оценки Пуассона. Здесь мы подробно остановимся на М-оценках.

Для состоятельности ключевым шагом является получение предела по вероятности $\mathcal{Q}_N(\theta)$. Это делается путем применения ЗБЧ, потому что для М-оценки $\mathcal{Q}_N(\theta)$ является средним $N^{-1} \sum_i q_i(\theta)$. Различные предположения о процессе порождающем данные ведут к использованию различных вариантов ЗБЧ и к существенным различиям в выражении $\mathcal{Q}_0(\theta)$.

Асимптотическая нормальность требует предположений о процессе порождающем данные в дополнение к тем, которые необходимы для состоятельности. В частности, нам необходимы предположения о процессе порождающем данные, чтобы обеспечить применение ЗБЧ для получения $A_0$ и применение ЦПТ для получения $B_0$.

Для М-оценки ЗБЧ скорее всего удовлетворяет условию 2 теоремы 5.3, поскольку каждый элемент матрицы $\partial^2 \mathcal{Q}_N(\theta) / \partial \theta \partial \theta'$ является средним, так как $\mathcal{Q}_N(\theta)$ является средним. Из ЦПТ может следовать условие 3 теоремы 5.3, так как у $\sqrt{N} \partial \mathcal{Q}_N(\theta) / \partial \theta|_{\theta_0}$ математическое ожидание равно нулю из неформального условия состоятельности (5.24) в Разделе 5.3.7 и конечная дисперсия $\E[N \partial \mathcal{Q}_N(\theta) / \partial \theta \times \partial \mathcal{Q}_N(\theta) / \partial \theta'|_{\theta_0}]$.

Частные случаи ЦПТ и ЗБЧ, которые используются для получения предельного распределения оценки, меняются в зависимости от предположений о процессе порождающем данные для $(y|X)$. Во всех случаях зависимая переменная является стохастической. Тем не менее, регрессоры могут быть детерминированными или стохастическими, и в последнем случае они могут иметь зависимость во времени. Эти вопросы уже рассматривались для МНК в Разделе 4.4.7.

Частое предположение в микроэконометрике заключается в том, что регрессоры стохастические и  наблюдения независимы, что разумно для пространственных данных из обследований населения. Для простой случайной выборки данные $(y_i,x_i)$ независимы и одинаково распределены, и ЗБЧ Колмогорова и ЦПТ Линдеберга-Леви (теоремы А.8 и А.14) могут быть применены. Более того, согласно простой случайной выборке (5.18) и (5.19) упростим 
\[
A_0=\E \left[  \left. \frac{\partial^2 q(y,x,\theta)}{\partial \theta \partial \theta'} \right|_{\theta_0} \right]
\]
и
\[
B_0=\E \left[ \left. \frac{\partial q(y,x,\theta)}{\partial \theta} \frac{\partial q(y,x,\theta)}{\partial \theta'} \right|_{\theta_0} \right],
\]
где $(y,x)$ обозначает одно наблюдение и математические ожидания зависят от совместного распределения $(y,x)$. Это простое обозначение используется в нескольких текстах.

Для стратифицированной случайной выборки и для фиксированных регрессоров данные $(y_i,x_i)$  являются независимыми и неодинаково распределенными, поэтому необходимо использовать ЗБЧ Маркова и ЦПТ Ляпунова (теоремы A.9 и A.15). Они требуют дополнительных предположений о моментах к тем, которые применяются в случае независимо и одинаково распределённых данных. В случае стохастических регрессоров математические ожидания зависят от совместного распределения $(y,x)$, тогда как в случае детерминированных регрессоров, например, в контролируемом эксперименте, где уровень $x$ может быть задан, математические ожидания в (5.18) и (5.19) зависят только от $y$.

Для временных рядов предполагается, что регрессоры стохастические, но также предполагается зависимость разных наблюдений, что является необходимой основой для использования лаговых зависимых переменных. Гамильтон (1994) фокусируется на этом случае, который также был изучен Уайтом (2001a). Самая простая процедура предусматривает, что случайные величины $(y,x)$ имеют стационарное распределение. Если вместо этого данные нестационарные и есть единичный корень, тогда скорость сходимости больше не может быть $\sqrt{N}$ и предельное распределение может быть ненормальным.

Однако, несмотря на эти важные концептуальные и теоретические разногласия по поводу стохастического характера $(y,x)$, для пространственной регрессии вариация предельной теоремы, как правило, приводится в общей форме, описанной в теореме 5.3.

\subsection{Обсуждение}

Форма ковариационной матрицы (5.20) называется сэндвич-формой с $B_0$, помещённым между $A_0^{-1}$ и $A_0^{-1}$. Сэндвич-форма, введённая в Разделе 4.4.4, будет рассмотрена более подробно в Разделе 5.5.2.

Асимптотические результаты можно распространить на несостоятельные оценки. Тогда $\theta_0$ будет заменён на псевдо-истинное значение $\theta^*$, которое определяется как такое значение $\theta_0$, которое даёт локальный максимум $\mathcal{Q}_0(\theta)$. Это рассматривается более подробно для оценки квази-ММП в Разделе 5.7.1. В большинстве случаев, однако, оценка состоятельна, и в последующих главах индекс 0 часто опускается для более простого обозначения.

В предыдущих результатах целевая функция $\mathcal{Q}_N(\theta)$ первоначально определяется с нормированием на $1/N$, тогда первая производная $\mathcal{Q}_N(\theta)$ нормируется на $\sqrt{N}$, а вторая производная не нормируется, что приводит к $\sqrt{N}$-состоятельной оценке. В некоторых случаях могут быть необходимы альтернативные способы нормирования, в первую очередь, для временных рядов с нестационарным трендом.

Результаты предполагают, что $\mathcal{Q}_N(\theta)$ является непрерывной дифференцируемой функцией. Это исключает некоторые оценки, такие, как оценку методом наименьших абсолютных значений, для которой  $\mathcal{Q}_N(\theta)=N^{-1} \sum_i |y_i-x'_i \beta|$. Один из способов продолжить в этом случае --- получение дифференцируемой апроксимирующей функции  $\mathcal{Q^*}_N(\theta)$ такой, что  $\mathcal{Q^*}_N(\theta)-\mathcal{Q}_N(\theta) \xrightarrow{p} 0$, и тогда применить предыдущую теорему к  $\mathcal{Q^*}_N(\theta)$.

Ключевым компонентом для получения предельного распределения является линеаризация с использованием разложения в ряд Тейлора. Разложение в ряд Тейлора может быть не очень хорошей глобальной аппроксимацией функции. Оно хорошо работает в нашем статистическом приложении, поскольку аппроксимация асимптотически локальна, так как состоятельность предполагает, что при больших выборках $\hat{\theta}$ близка к разложению в точке $\theta_0$. Более точную асимптотическую теорию можно построить при использовании разложение Эджуорта (см. Раздел 11.4.3 ). Бутстреп (см. Главу 11) является методом для реализации разложения Эджуорта на практике.

\subsection{Неформальный подход к состоятельности М-оценки}

На практике предельную нормальность из теоремы 5.3 гораздо легче доказать, чем состоятельность, используя теорему 5.1 или 5.2. Здесь мы представляем неформальный подход к определению характера и силы предположений о распределении, необходимых для того, чтобы М-оценка была состоятельной.

Для М-оценки, которая является локальным максимумом, из условия первого порядка (5.4) следует, что $\hat{\theta}$ выбирается таким образом, чтобы средняя $\partial q_i(\theta) / \partial \theta |_{\hat{\theta}}$ равнялась нулю. Интуитивно, необходимым условием для этого, чтобы получить состоятельную оценку для $\theta_0$ является то, что в пределе математическое ожидание $\partial q(\theta) / \partial \theta |_{\theta_0}$ стремится к 0, или 
\begin{equation}
\left. \plim \frac{\partial \mathcal{Q}_N(\theta)} {\partial \theta} \right|_{\theta_0}= \left. \lim \frac{1}{N} \sum_{i=1}^{N} \E \left[ \frac{\partial q_i(\theta)}{\partial \theta} \right|_{\theta_0} \right] =0,
\end{equation}
где первое равенство требует предположения, что закон больших чисел может быть применён и математическое ожидание в (5.24) зависит от процесса порождающего данные для $(y,X)$. Предел используется, поскольку равенство не обязательно должно быть точным при условии, что любое отклонение от нуля исчезает при $N \rightarrow \infty$. Например, состоятельность должна иметь место, если математическое ожидание равно $1/N$.

Условие (5.24) представляет собой весьма полезную проверку для практика. Неформальный подход к состоятельности заключается в том, чтобы посмотреть на условие первого порядка для оценки $\hat{\theta}$ и определить, равно ли математическое ожидание в пределе нулю, когда $\theta=\theta_0$.

Еще менее формально, если мы рассмотрим компоненты в сумме, существенным условием для состоятельности является следующее равенство для типичного наблюдения
\begin{equation}
\E[\partial q(\theta)/ \partial \theta|_{\theta_0}]=0.
\end{equation}

Это условие может оказаться очень полезным для практика. Тем не менее, оно не является ни необходимым, ни достаточным условием. Если математическое ожидание в (5.25) равно $1/N$, то всё ещё возможно, что предел по вероятности в (5.24) равен нулю, поэтому условие (5.25) не является необходимым. Чтобы увидеть, что этого недостаточно, предположим, что $y$ независимы и одинаковы распределены с математическим ожиданием $\mu_0$ и оценены с использованием только одного наблюдения, допустим, первого наблюдения $y_1$. Тогда $\hat{\mu}$ --- решение $y_1-\mu=0$ и (5.25) выполнено. Но ясно, что $y_1 \not \xrightarrow{p} \mu_0$, поскольку у одного наблюдения $y_1$ дисперсия не стремится к нулю. Проблема в том, что здесь $\plim$ в (5.24) не равен $\lim E$. При формальном доказательстве несостоятельности необходимо применять такие теоремы, как теоремы 5.1 или 5.2.

Для регрессии Пуассона применение (5.25) показывает, что необходимым условием для состоятельности является правильная спецификации условного среднего $ (y|x) $ (см. Раздел 5.2.3). Похожим образом, оценка МНК является решением $N^{-1} \sum_i x_i (y_i-x'_i \beta)=0$, поэтому согласно (5.25) условие состоятельности требует, чтобы $\E[x(y-x'\beta_0)]=0$. Это условие не выполняется, если $\E[y|x] \not= x'\beta_0$, что может произойти по многим причинам, как указано в Разделе 4.7. В других примерах использование (5.25) может указать на то, что состоятельность будет требовать значительно большего числа параметрических предположений, чем правильная спецификации условного математического ожидания.
 
Чтобы связать использование (5.24) с условием 3 в теореме 5.2, обратим внимание на следующее:
\[
\begin{matrix}
\partial \mathcal{Q}_0(\theta)/ \partial \theta =0 & \text{(условие 3 в теореме 5.2)} \\
\Longrightarrow \partial(\plim (\mathcal{Q}_N(\theta))/ \partial \theta =0 & \text{(из определения} \mathcal{Q}_0(\theta)) \\
\Longrightarrow \partial(\lim \E[\mathcal{Q}_N(\theta)])/ \partial \theta =0 & \text{ЗБЧ} \Longrightarrow \mathcal{Q}_0=\plim \mathcal{Q}_N = \lim \E[\mathcal{Q}_N]) \\
\Longrightarrow \lim  \partial \E[\mathcal{Q}_N(\theta)]/ \partial \theta =0 & \text{(поменяем местами предел и дифференциал) и } \\
\Longrightarrow \lim \E[ \partial \mathcal{Q}_N(\theta)/ \partial \theta ]=0 & \text{(поменяем местами дифференциал и математическое ожидание).}
\end{matrix}
\]

Последняя строка является неформальным условием (5.24). Однако получение этого результата требует дополнительных предположений, в том числе ограничения на локальный максимум, применения закона больших чисел, права менять местами предел и производную, права менять местами производную и математическое ожидание (например, интегрирование). В скалярном случае достаточным условием для того, чтобы поменять производную и предел местами, является то, чтобы $\lim_{h \rightarrow 0} (\E[\mathcal{Q}_N (\theta+h)]-\E[\mathcal{Q}_N(\theta)])/h=d\E[\mathcal{Q}_N(\theta)]/d\theta$ равномерно в $\theta$.

\section{Оценочные уравнения}

Вывод предельного распределения, который приведён в Разделе 5.3.3, может быть расширен от локальной экстремальной оценки на оценки, которые определяются как решение оценочного уравнения, в котором среднее приравнивается к нулю. Несколько примеров приведено в Главе 6.

\subsection{Оценка методом оценочных уравнений}

Пусть $\hat{\theta}$ --- это решение системы из $q$ оценочных уравнений
\begin{equation}
h_N(\hat{\theta})=\frac{1}{N} \sum_{i=1}^{N} h(y_i,x_i,\hat{\theta})=0,
\end{equation}
где $h(\cdot)$ --- вектор размера $q \times 1$, к тому же предполагается независимость по $i$. Примеры $h(\cdot)$ приведены далее в Разделе 5.4.2.

Поскольку $\hat{\theta}$ выбрано так, чтобы выборочное среднее $h(y,x,\hat{\theta})$ было равно нулю, мы ожидаем, что $\hat{\theta} \xrightarrow{p} \theta_0$, если в пределе среднее $\hat{\theta} \xrightarrow{p} \theta_0$ стремится к нулю, то есть, если $\plim h_N(\theta_0)=0$. Если, ЗБЧ может быть применён, необходимо, чтобы $\lim \E[h_N(\theta_0)]=0$, или, проще говоря, что для $i$-ого наблюдения
\begin{equation}
\E[h(y_i,x_i,\theta_0)]=0.
\end{equation}
Самый простой способ, чтобы формально установить состоятельность, заключается в выводе (5.26) в качестве условия первого порядка для М-оценки.

Предполагая состоятельность, предельное распределение оценки оценочных уравнений может быть получено таким же образом, как и в Разделе 5.3.3 для экстремальной оценки. Возьмём точное разложение в ряд Тейлора первого порядка $h_N(\theta_0)$ в окрестности $\theta_0$, как и в (5.15) с $\mathfrak{f}(\theta)=h_N(\theta_0)$, и приравняем правую сторону к 0 и решим. Тогда
\begin{equation}
\left. \sqrt{N}(\hat{\theta}-\theta_0)=- \left( \frac{\partial h_N(\theta_0)}{\partial \theta'} \right|_{\theta^{+}} \right) ^{-1} \sqrt{N} h_N(\theta_0).
\end{equation}
Это приводит к следующей теореме.

\begin{theorem}[Предельное распределение оценок оценочных уравнений]:
Предположим, что оценка оценочных уравнений, которая является решением (5.26), состоятельная для $\theta_0$, и сделаем следующие предположения:
\begin{enumerate}
\item $\partial h_N(\theta)/ \partial \theta'$ существует и непрерывна в открытой выпуклой окрестности $\theta_0$. 
\item $\partial h_N(\theta)/ \partial \theta'|_{\theta^+}$ сходится по вероятности к конечной невырожденной матрице
\begin{equation}
A_0= \left. \plim \frac{\partial h_N(\theta)}{\partial \theta'} \right|_{\theta_0}=\plim \frac{1}{N} \sum_{i=1}^{N} \left. \frac{\partial h_i(\theta)}{\partial \theta'} \right|_{\theta_0},
\end{equation}
для любой последовательности $\theta^+$ такой, что $\theta^+ \xrightarrow{p} \theta_0$.
\item $\sqrt{N} h_N(\theta_0) \xrightarrow{d} \mathcal{N}[0,B_0]$, где
\begin{equation}
B_0= \plim N h_N(\theta_0) h_N(\theta_0)'=\plim \frac{1}{N} \sum_{i=1}^{N} \sum_{j=1}^{N} h_i(\theta_0) h_j(\theta_0)'.
\end{equation}
\end{enumerate}

Тогда предельное распределение оценки оценочных уравнений это
\begin{equation}
\sqrt{N}(\hat{\theta}-\theta_0) \xrightarrow{d} \mathcal{N}[0,{A_0}^{-1} B_0 {A'_0}^{-1}], 
\end{equation}
где, в отличие от экстремальной оценки, матрица $A_0$ может быть несимметричной, поскольку это уже необязательно матрица Гессе.
\end{theorem}

Эта теорема может быть доказана путём адаптации доказательства Амэмия теоремы 5.3. Отметим, что теорема 5.4 предполагает, что состоятельность уже была установлена.

Годамбе (1960) показал, что для анализа, зависящего от регрессоров, наиболее эффективная оценка оценочных уравнений подразумевает, что $h_i(\theta) = \partial \ln f(y_i|x_i, \theta)/ \partial \theta$. Тогда (5.26) является условием первого порядка для ММП-оценки.

\subsection{Принцип аналогии}


Принцип аналогии использует условия для генеральной совокупности, чтобы сформулировать оценки. Книга Мански (1988a) подчёркивает важность принципа аналогии в качестве объединяющей темы для оценивания. Мански (1988a, стр. xi) приводит следующую цитату из Голдбергера (1968, стр. 4.):

Согласно принципу аналогии в оценивании $\dots$, параметры генеральной совокупности следует оценивать такими статистиками, которые обладают тем же свойством в выборке, что и параметры в генеральной совокупности.

Условия моментов для генеральной совокупности предлагают в качестве оценки решение соответствующего условия для момента выборки.

Примеры применения принципа аналогии для экстремальных оценок были приведены в Разделе 4.2. Например, если целью предсказания является минимизация ожидаемых потерь в генеральной совокупности и используются потери в квадратах ошибок, то параметры регрессии $\beta$ оцениваются путём минимизации выборочной суммы квадратов ошибок.

Оценки метода моментов также могут послужить примером. Например, в случае одинаково и независимо распределённых величин, если $\E[y_i-\mu]=0$ для генеральной совокупности, то мы используем в качестве оценки $\hat{\mu}$, которая является решением соответствующих условий выборочных моментов $N^{-1} \sum_i (y_i-\mu)=0$, что приводит к равенству с выборочным средним $\hat{\mu}=\bar{y}$.


Оценка оценочных уравнений может быть определена как оценка метода аналогий. Если (5.27) верно для генеральной совокупности, тогда можно оценить $\theta$, решая соответствующее условие выборочного момента (5.26).

Оценки оценочных уравнений широко используются в микроэконометрике. Соответствующая теория может быть рассмотрена как частный случай обобщённого метода моментов. Обобщенный метод моментов представлен в следующей главе, и допускает, что условий моментов больше, чем параметров. В прикладной статистике подход используется в контексте обобщённых оценочных уравнений.

\section{Cтатистические выводы} 

Подробное изложение проверки гипотез и построение доверительных интервалов приведены в Главе 7. Здесь мы опишем, как проверить линейные ограничения, в том числе исключающие ограничения. Мы используем самый распространенный метод, тест Вальда для оценок, которые у нас могут быть нелинейными.  Используется асимптотическая теории, поэтому формальные результаты приводят к распределению хи-квадрат и нормальному распределению, а на небольших выборках --- к $F$- и $t$-распределениям в линейной регрессии в условиях нормальности. Кроме того, существует несколько способов, чтобы получить состоятельную оценку ковариационной матрицы экстремальной оценки, что приводит к альтернативным оценкам стандартных ошибок и связанным с ними тестовых статистик и Р-значений.

\subsection{Тест Вальда для проверки гипотезы о линейных ограничениях}

Рассмотрим тестирование $h$ линейных независимых ограничений, проверим гипотезу $H_0$ против альтернативной $H_a$, где
\[
H_0: R \theta_0 - r = 0,
\]
\[
H_a: R \theta_0 - r \not= 0,
\]
c матрицей констант $R$ размера $h \times q$ и вектором констант $r$ размера $h \times 1$. Например, если $\theta=[\theta_1, \theta_2, \theta_3]$, то, чтобы протестировать верно ли, что $\theta_{10}-\theta_{20}=2$ можно взять $R=[1,-1,0]$ и $r=-2$.

Тест Вальда отвергает $H_0$, если $R\hat{\theta} - r$, выборочная оценка $R \theta_0 - r$, сильно отличается от $0$. Это определяет необходимость знать распределение $R\hat{\theta} - r$. Предположим, что $\sqrt{N}(\hat{\theta}-\theta_0) \xrightarrow{d} \mathcal{N}[0,C_0]$, где $C_0={A_0}^{-1} B_0 {A'_0}^{-1}$ из (5.20). Тогда 
\[
\hat{\theta} \stackrel{a}{\sim} \mathcal{N}[\theta_0, N^{-1} C_0],
\]
поэтому при верной $H_0$ линейная комбинация 
\[
R\hat{\theta}-r \stackrel{a}{\sim} \mathcal{N}[0,R(N^{-1}C_0)R'],
\]
где математическое ожидание равно 0, поскольку $R \theta_0 - r = 0$ в $H_0$.

\begin{center}
Тесты хи-квадрат
\end{center}

Удобно перейти от многомерного нормального распределения к распределению хи-квадрат, применив квадратичную форму. Это даёт статистику Вальда:
\begin{equation}
W=(R\hat{\theta}-r)'(R(N^{-1}\hat{C})R')^{-1}(R\hat{\theta}-r) \xrightarrow{d} \chi^{2}(h)
\end{equation}
при верной $H_0$, где $R(N^{-1}C_0)R'$ имеет полный ранг $h$ в предположении линейно независимых ограничений, а $\hat{C}$ является состоятельной оценкой $C_0$. Большие значения $W$ приводят к отвержению $H_0$, и $H_0$ отвергается на уровне $\alpha$, если $W>\chi_{\alpha}^{2}(h)$ и не отвергается в противном случае.

Практики часто вместо этого используют $F$-статистику $F=W/h$. Вывод тогда основывается на распределении $F(h,N-q)$ в надежде, что это может обеспечить лучшую аппроксимацию для конечной выборки. Обратите внимание, что $h$ умноженное на $F(h,N)$ сходится к распределению $\chi^{2}(h)$ при $N \rightarrow \infty$.

Замена $C_0$ на $\hat{C}$ при получении (5.32) не оказывает никакого влияния в асимптотическом случае, но в конечных выборках различные $\hat{C}$ приведут к различным значениям $W$. В случае классической линейной регрессии этот шаг соответствует замене $\sigma^2$ на $s^2$. Тогда $W/h$ имеет $F$-распределение, если ошибки распределены нормально (см. Раздел 7.2.1).

\begin{center}
Тесты на один коэффициент
\end{center}

Часто внимание уделяется тестированию отличия от нуля одного коэффициента, например, $j$-ого. Тогда $R \theta - r = \theta_j$ и $W=\hat{\theta_j}^2/(N^{-1}\hat{c_{jj}})$, где $\hat{c}_{jj}$ --- диагональный элемент $\hat{C}$. Если извлечём корень из $W$, то получим
\begin{equation}
t=\frac{\hat{\theta_j}}{se[\hat{\theta_j}]} \xrightarrow{d} \mathcal{N}[0,1]
\end{equation}
при верной $H_0$, где $se[\hat{\theta_j}]=\sqrt{N^{-1}\hat{c}_{jj}}$ --- асимптотическая стандартная ошибка $\hat{\theta_j}$. Большие значения $t$ приводят к отвержению $H_0$, и в отличие от $W$, $t$-cтатистика может быть применена для односторонних тестов.

Формально $\sqrt{W}$ является асимптотической $z$-статистикой, но мы используем обозначение $t$, поскольку это обозначает обычную <<$t$-статистику>>, получаемой делением оценки на её стандартную ошибку. В конечных выборках некоторые статистические пакеты используют стандартное нормальное распределение, в то время как другие используют $t$-распределения для расчёта критических значений, $P$-значений и доверительных интервалов. Ни то, ни другое не является правильным для конечных выборок, за исключением весьма частного случая линейной регрессии с ошибками, которые нормально распределены, и в этом случае $t$-распределение является точным. Оба способа ведут к одним и тем же результатам на бесконечно больших выборках, так как $t$-распределение сходится к стандартному нормальному распределению.

\begin{center}
Оценка ковариационной матрицы
\end{center}

Есть много возможных способов оценки ${A_0}^{-1} B_0 {A'_0}^{-1}$, потому что есть много способов состоятельно оценить $A_0$ и $B_0$. Таким образом, различные эконометрические программы  должны давать одинаковые оценки коэффициентов, но, вполне резонно, что они могут давать стандартные ошибки, $t$-статистики, и $P$-значения, отличающиеся в конечных выборках. Именно эконометрист определяет метод и мощность соответствующих предположений о распределении процесса порождающего данные.

\begin{center}
Сэндвич-оценка ковариационной матрицы
\end{center}

Предельное распределение $\sqrt{N}(\hat{\theta}-\theta_0)$ имеет ковариационную матрицу ${A_0}^{-1} B_0 {A'_0}^{-1}$. Отсюда следует, что у $\hat{\theta}$ асимптотическая ковариационная матрица $N^{-1}{A_0}^{-1} B_0 {A'_0}^{-1}$, где появляется деление на $N$, потому что мы рассматриваем $\hat{\theta}$, а не $\sqrt{N}(\hat{\theta}-\theta_0)$.

Сэндвич-оценка асимптотической ковариационной матрицы $\hat{\theta}$ --- это оценка вида
\begin{equation}
\hat{\Var}[\hat{\theta}]= N^{-1} \hat{A}^{-1} \hat{B} \hat{A}'^{-1},
\end{equation}
где $\hat{A}$ --- состоятельная оценка $A_0$ и $\hat{B}$ --- состоятельная оценка $B_0$. Это называется сэндвич-формой, поскольку $\hat{B}$ находится между $\hat{A}^{-1}$ и $\hat{A}'^{-1}$. Для многих оценок $A$ --- это матрица Гессе, поэтому $\hat{A}^{-1}$ симметрична, но это не всегда верно.

Скорректированной сэндвич-оценкой является сэндвич-оценка, где оценка $\hat{B}$ --- состоятельная оценка $B_0$ при относительно слабых предположениях. Это приводит к тому, что называют скорректированными стандартными ошибками. Ярким примером является устойчивая к гетероскедастичности оценка Уайта ковариационной матрицы МНК-оценки (см. Раздел 4.4.5). В различных конкретных контекстах, подробно описанных в последующих Разделах, скорректированными сэндвич-оценками называются оценки Губера, в честь Губера (1967); оценки Эйкера и Уайта, в честь Эйкера (1967) и Уайта (1980a,б, 1982), а также в применении к стационарным временным рядам оценки Ньюи-Веста, в честь Ньюи и Веста (1987b).

\begin{center}
Оценки $A$ и $B$
\end{center}

Здесь мы представляем различные оценки для $A_0$ и $B_0$ как для оценок методом оценочных уравнений, которые являются решением $h_{N}(\hat{\theta})=0$ так и для локальных экстремальных оценок, которые являются решением $\partial \mathcal{Q}_N(\theta) / \partial \theta |_{\hat{\theta}}=0 $

Две стандартные оценки $A_0$ в (5.29) и (5.18) --- это матрица Гессе
\begin{equation}
\hat{A}_H=\left. \frac{\partial h_{N}(\theta)}{\partial \theta'} \right|_{\hat{\theta}}= \left. \frac{\partial^2 \mathcal{Q}_N(\theta)}{\partial \theta \partial \theta'} \right|_{\hat{\theta}},
\end{equation}
где второе равенство объясняет использование термина матрица Гессе, и ожидаемая матрица Гессе
\begin{equation}
\hat{A}_{EH}=\E \left. \left[ \frac{\partial h_{N}(\theta)}{\partial \theta'}\right] \right|_{\hat{\theta}}= \E \left. \left[ \frac{\partial^2 \mathcal{Q}_N(\theta)}{\partial \theta \partial \theta'} \right] \right|_{\hat{\theta}}.
\end{equation}

Первая матрица аналитически проще и потенциально полагается на меньшее число предположений о распределении; последняя более вероятно, будет отрицательно определена и обратима.

Для $B_0$ в (5.30) или (5.19) нет возможности использовать очевидную оценку $Nh_{N}(\hat{\theta})h_{N}(\hat{\theta})'$, поскольку выражение равно нулю, так как $\hat{\theta}$ задаётся таким образом, чтобы удовлетворять условию $h_{N}(\hat{\theta})=0$. При потенциально сильных предположениях о распределении получаем оценку
\begin{equation}
\hat{B}_{E}=\E \left. \left[ Nh_{N}(\theta)h_{N}(\theta)' \right] \right|_{\hat{\theta}}= \E \left. \left[ N \frac{\partial \mathcal{Q}_N(\theta)}{\partial \theta} \frac{\partial \mathcal{Q}_N(\theta)}{\partial \theta'} \right] \right|_{\hat{\theta}}.
\end{equation}

Более слабые предположения возможны для M-оценок и оценок методом оценочных уравнений с данными, которые независимы по $i$. Тогда (5.30) упрощается до следующего вида:
\[
B_0=\E \left[ \frac{1}{N} \sum_{i=1}^{N} h_i(\theta) h_i(\theta)' \right],
\]
поскольку независимость означает, что для $i \not= j$, $\E[h_i h_j']=\E[h_i] \E[h'_j]$, которая, в свою очередь, равна нулю при условии, что $\E[h_i(\theta)]=0$. Мы получаем оценки с использованием внешнего произведения или оценкам BHHH (в честь Берндт, Холла, Холла, и Хаусмана, 1974)

\begin{equation}
\hat{B}_{OP}=  \frac{1}{N} \left. \sum_{i=1}^{N} h_i(\hat{\theta}) h_i(\hat{\theta})'= \frac{1}{N} \sum_{i=1}^{N} \frac{\partial q_i(\theta)}{\partial \theta} \right|_{\hat{\theta}} \left. \frac{\partial q_i(\theta)}{\partial \theta'} \right|_{\hat{\theta}}.
\end{equation}
$\hat{B}_{OP}$ требует меньшего числа предположений, чем $\hat{B}_{E}$.

На практике корректировка степеней свободы часто используется при оценке $B_0$, при
делении в (5.38) $\hat{B}_{OP}$ на $(N-q)$, а не $N$, и аналогичном умножении $\hat{B}_{E}$ в (5.37) на $N/(N-q)$. Нет теоретического обоснования для этой корректировки в нелинейных моделях, но в некоторых модельных исследованиях эта корректировка приводит к улучшению результатов в конечных выборках, и оно совпадает с корректировкой степеней свободы для МНК с гетероскедастичными ошибками. Никаких подобных корректировок не производится для $\hat{A}_H$ или $\hat{A}_{EH}$.

Упрощение происходит в некоторых особых случаях с $A_0=-B_0$. Наиболее яркими примерами являются МНК или НМНК с гетероскедастичными ошибкам (см. Раздел 5.8.3) и ММП с правильно специфицированным распределением (см. Раздел 5.6.4). Тогда либо $-\hat{A}^{-1}$, либо
$\hat{B}^{-1}$ может быть использованы для оценки дисперсии $\sqrt{N}(\hat{\theta} - \theta_0)$. Эти оценки являются менее устойчивыми к неправильной спецификации процесса порождающего данные, чем те, которые применяют сэндвич-форму. Неправильная спецификация процесса порождающего данные, тем не менее, может дополнительно привести к несостоятельности $\hat{\theta}$, в этом случае даже выводы, основанные на скорректированной сэндвич-оценке, будут недействительными.

Для примера с регрессией Пуассона в Разделе 5.2 $\hat{A}_H = \hat{A}_{EH}=-N^{-1} \sum_i \exp(x'_i \hat{\beta})x_i x'_i$ и  $\hat{B}_{OP}=(n-q)^{-1} \sum_i (y_i-\exp(x'_i \hat{\beta}))^{2} x_i x'_i$. Если $\Var[y|x]=\exp(x' \beta_0)$, что будет верно, если $y|x$ распределён действительно по Пуассону, то $\hat{B}_{E}=-[N/(N-q)] \hat{A}_{EH}$ и имеет место упрощение.

\section{Метод максимального правдоподобия}

Оценки ММП занимают особое место среди оценок. Это самые эффективные оценки среди состоятельных асимптотически нормальных оценок. Они также важны в учебных целях, так как многие методы нелинейной регрессии такие, как M-оценки, можно рассматривать как расширения и адаптацию результатов, впервые полученных для оценок ММП.

\subsection{Функция правдоподобия}

\begin{center}
Принцип максимального правдоподобия
\end{center}

Принцип максимального правдоподобия, введённый Р.А. Фишером (1922), заключается в выборе в качестве оценки вектора параметров $\theta_0$ того значения $\theta$, которое максимизирует вероятность получения фактической выборки. В дискретном случае это вероятность, полученная из функции вероятности, в непрерывном случае --- это плотность. Рассмотрим дискретный случай. Если значение $\theta$ означает, что вероятность наблюдаемых данных --- 0.0012, в то время как второе значение $\theta$ даёт более высокую вероятность 0.0014, то второе значение $\theta$ является лучшей оценкой.

Совместная функция вероятности $f(y,X|\theta)$ рассматривается здесь как функция от $\theta$ при фиксированных данных $(y,X)$. Это называется функцией правдоподобия и обозначается как $L_{N}(\theta|y,X)$. Максимизация $L_{N}(\theta)$ эквивалентна максимизации логарифмической функции правдоподобия.
\[
\mathcal{L}_{N}(\theta)=\ln L_{N}(\theta).
\]

Возьмём натуральный логарифм, потому что после логарифмирования мы получим сумму, а не произведение $N$ членов.

\begin{center}
Условная функция правдоподобия
\end{center}

Функция правдоподобия $L_{N}(\theta)=f(y,X|\theta)=f(y|X,\theta)f(X|\theta)$ требует задания как условной плотности $y$ при заданном $x$, так и предельной плотности $X$.

Вместо этого, оценки, как правило, основываются на условной функции правдоподобия $L_{N}(\theta)=f(y|X,\theta)$, так как цель регрессии заключается в моделировании поведения $y$ при заданном $X$. Это не является ограничением, если $f(y|X)$ и $f(X)$ зависят от взаимоисключающих наборов параметров. Когда это так, часто опускают слово условный. Для редких исключений таких, как эндогенные выборки (см. Главы 3 и 24), состоятельное оценивание требует, чтобы оценка основывалась на полной совместной плотности $f(y,X|\theta)$, а не условной плотности $f(y|X,\theta)$.

\begin{table}[h]
\begin{center}
\caption{\label{tab:max} Максимальное правдоподобие: часто используемые плотности}
\begin{tabular}[t]{llll}
\hline
\hline
\bf{Распределение} & \bf{Диапозон $y$} & \bf{Плотность $f(y)$} & \bf{Наиболее частая} \\
 & & & \bf{параметризация} \\
\hline
Нормальное & $(-\infty,\infty)$ & $[2\pi\sigma^{2}]^{-1/2}e^{-(y-\mu)^{2}/2\sigma^{2}}$ & $\mu=x'\beta, \sigma^{2}=\sigma^{2}$\\
Бернулли & $0$ или $1$ & $p^{y}(1-p)^{1-y}$ & $Logit$ $p=e^{x'\beta}/(1+e^{x'\beta})$\\
Экспоненциальное & $(0,\infty)$ & $\lambda e^{-\lambda y}$ & $\lambda=e^{x'\beta}$ или $1/\lambda=e^{x'\beta}$ \\
Пуассона & $0,1,2,\cdots$ & $e^{-\lambda}\lambda^{y}/y!$ & $\lambda=e^{x'\beta}$ \\
\hline
\hline
\end{tabular}
\end{center}
\end{table}

\vspace{3cm}

Для пространственных данных наблюдения $(y_i,x_i)$ независимы по $i$ с функцией условной плотности $f(y_i|x_i,\theta)$. Тогда в силу независимости совместной условной плотности $f(y|X,\theta)=\prod_{i=1}^{N}f(y_i|x_i,\theta)$, что приводит к 
\begin{equation}
\mathcal{Q}_{N}(\theta)=N^{-1}\mathcal{L}_{N}(\theta)=\frac{1}{N} \sum_{i=1}^N \ln f(y_i|x_i,\theta),
\end{equation}
где мы делим на $N$ для того, чтобы целевая функция была средним арифметическим.

Результаты распространяются на многомерные данные, системы уравнений и панельные данные путём замены скаляра $y_i$ на вектор $y_i$, в этом случае $f(y_i|x_i,\theta)$ --- совместная плотность $y_i$
при фиксированных $x_i$. См. также Раздел 5.7.5.

\begin{center}
Примеры
\end{center}

Для разных типов данных следующий метод используется для построения полностью параметрической пространственной модели регрессии. Сначала выберем распределение с одним параметром или с двумя (или в некоторых редких случаях с тремя), данное распределение будет использоваться для зависимой переменной $y$ в случае одинаково и независимо распределённых величин, как в базовом курсе статистики. Один или два основных параметра распределения сделаем функциями от  регрессоров $x$ и параметров $\theta$.

Некоторые часто используемые распределения и параметризации приведены в таблице 5.3. Дополнительные распределения приведены в Приложении B, в котором также представлены методы для генерирования псевдо-случайных чисел.

Для непрерывных на $(-\infty,\infty)$ данных, нормальное распределение является распространенным распределением. В классической модели линейной регрессии $\mu=x'\beta$ и $\sigma^2$ постоянна.

Для дискретных бинарных данных, принимающих значения $0$ или $1$, всегда используется распределение Бернулли, частный случай биномиального распределения при одной попытке. Обычные параметризации для вероятности Бернулли приводят к логит-модели, представленной в таблице 5.3, и пробит-модели с  $p=\Phi(x'\beta)$, где $\Phi(\cdot)$ является  функцией стандартного нормального распределения. Эти модели анализируются в главе 14.

Для положительных непрерывных данных на $(0,\infty)$, в частности, данных по длительности, которые рассматриваются в Главах 17-19, более сложные модели такие, как модель Вейбулла, гамма модель и лог-нормальная модель часто используются в дополнение к экспоненциальному распределению, приведённому в таблице 5.3.

Для целочисленных данных, принимающих значения $0,1,2, \dots$, (см. главу 20), более сложное отрицательное биномиальное распределение часто используется в дополнение к распределению Пуассона, представленному в Разделе 5.2.1. Равенство $\lambda=\exp(x'\beta)$ обеспечивает положительное условное математическое ожидание.

Для неполностью наблюдаемых данных могут быть использованы цензурированные или усечённые варианты этих распределений. Наиболее распространённым примером является цензурированное нормальное распределение, которое называют моделью тобит, и оно представлено в Разделе 16.3.

Модели, основанные на стандартном методе максимального правдоподобия, редко специфицируются с помощью введения предположений о распределении ошибок. Напротив, они определяются непосредственно в терминах распределения зависимой переменной. В частном случае при $y \sim \mathcal{N}[x'\beta,\sigma^2]$ можно эквивалентно определить $y=x'\beta+u$, где остаточный член $u \sim \mathcal{N}[0,\sigma^2]$. Тем не менее, здесь используется аддитивность нормального распределения, которая также есть у некоторых других распределений. Например, если $y$ распределён по Пуассону с математическим ожиданием $\exp(x'\beta)$, мы всегда можем написать $y=\exp(x'\beta)+u$, но ошибка уже имеет нестандартное распределение.

\subsection{Оценка максимального правдоподобия}

Оценка метода максимального правдоподобия (ММП) --- оценка, которая максимизирует (условную) логарифмическую функцию правдоподобия и, очевидно, является экстремальной оценкой. Обычно ММП-оценка --- локальный максимум, который является решением условия первого порядка
\begin{equation}
\frac{1}{N} \frac{\partial \mathcal{L}_{N}(\theta)}{\partial \theta} = \frac{1}{N} \sum_{i=1}^{N} \frac{\partial \ln f(y_i|x_i,\theta)}{\partial \theta}=0.
\end{equation}
Более формально эта оценка является условной ММП-оценкой, так как она основана на условной плотности $y$ при заданном $x$, но часто используют просто термин ММП-оценка.

Вектор градиентов $\partial \mathcal{L}_{N}(\theta) / \partial \theta$ называется скор-вектором, так как он является суммой первых производных логарифмической функции плотности. Когда он оценивается в точке $\theta_0$, он называется эффективным значением (efficient score).

\subsection{Равенство информационных матриц}

Результаты Раздела 5.3 можно упростить для ММП при условии, что плотность правильно специфицирована и для неё диапазон значений $y$ не зависит от $\theta$.

\begin{center}
Условия регулярности
\end{center}

Условия регулярности ММП: 
\begin{equation}
\E_{f} \left[ \frac{\partial \ln f(y|x,\theta)}{\partial \theta} \right] = \int \frac{\partial \ln f(y|x,\theta)}{\partial \theta} f(y|x,\theta)=0
\end{equation}
и
\begin{equation}
-\E_{f} \left[ \frac{\partial^2 \ln f(y|x,\theta)}{\partial \theta \partial \theta'} \right] = \E_{f} \left[ \frac{\partial \ln f(y|x,\theta)}{\partial \theta} \frac{\partial \ln f(y|x,\theta)}{\partial \theta'}\right],
\end{equation}
где введено обозначение $\E_{f}[\cdot]$, чтобы сделать явным, что математическое ожидание зависит от указанной плотности $f(y|x,\theta)$. Из результата (5.41) следует, что ожидаемая оценка скор-вектора равна нулю, и (5.42) приводит к (5.44).

Выкладки из Раздела 5.6.7 требуют, чтобы диапазон значений $y$ не зависел от $\theta$ для того, чтобы интегрирование и дифференцирование можно было поменять местами.

\begin{center}
Равенство информационных матриц
\end{center}

Информационная матрица является математическим ожиданием внешнего произведения градиентов,
\begin{equation}
\mathcal{I}=\E \left[ \frac{\partial \mathcal{L}_{N}(\theta)}{\partial \theta} \frac{\partial \mathcal{L}_{N}(\theta)}{\partial \theta'}\right].
\end{equation}
Применяется термин информационная матрица, так как $\mathcal{I}$ --- это дисперсия $\partial \mathcal{L}_{N}(\theta) / \partial \theta$, поскольку в силу (5.41) математическое ожидание $\partial \mathcal{L}_{N}(\theta) / \partial \theta$ равно 0. Тогда большие значения $\mathcal{I}$ говорят о том, что небольшие изменения в $\theta$ приведут к большим изменениям в логарифме правдоподобия, который, соответственно, содержит значительную информацию о состоянии $\theta$. Величина $\mathcal{I}$ более точно называется информацией Фишера, так как есть альтернативные измерители информации.

Для логарифмической функции правдоподобия (5.39) из условия регулярности (5.42) следует, что
\begin{equation}
-\E_{f} \left[ \left. \frac{\partial^2 \mathcal{L}_{N}(\theta)}{\partial \theta \partial \theta'} \right|_{\theta_0} \right] = \E_{f} \left[ \left. \frac{\partial \mathcal{L}_{N}(\theta)}{\partial \theta} \frac{\partial \mathcal{L}_{N}(\theta)}{\partial \theta'} \right|_{\theta_0} \right],
\end{equation}
если математическое ожидание зависит от $f(y|x,\theta_0)$. Соотношение (5.44) называется равенством информационных матриц и означает, что информационная матрица также равна $-\E [\partial^2 \mathcal{L}_{N}(\theta) / \partial \theta \partial \theta']$. Из равенства информационных матриц (5.44) следует, что $-A_0=B_0$, где $A_0$ и $B_0$ определены в (5.18) и (5.19). Теорема 5.3 в таком случае упрощается, поскольку ${A_0}^{-1}B_0{A_0}^{-1}=-A_0^{-1}=B_0^{-1}$. 

Равенство (5.42), в свою очередь, является частным случаем равенства обобщённых информационных матриц
\begin{equation}
\E_{f} \left[ \frac{\partial m(y,\theta)}{\partial \theta'} \right] = -\E_{f} \left[ m(y,\theta) \frac{\partial \ln  f(y|\theta)}{ \partial \theta'} \right],
\end{equation}
где $m(\cdot)$ является векторной функцией моментов с $\E_{f}[m(y,\theta)]=0$, и математическое ожидание зависит от плотности  $f(y|\theta)$. Этот результат также был получен в Разделе 5.6.7 и используется в главах 7 и 8, чтобы получить более простые формы некоторых тестовых статистик.

\subsection{Распределение ММП-оценок}

Условия регулярности (5.41) и (5.42) приводят к упрощению общих результатов Раздела 5.3.

Необходимое условие состоятельности (5.25) заключается в том, что $\E[\partial \ln f(y|x,\theta) / \partial \theta|_{\theta_0}] = 0$. Это имеет место в силу условия регулярности (5.41) при условии, что математическое ожидание зависит от $f(y|x,\theta_0)$. Таким образом, если процесс порождающий данные --- это $f(y|x,\theta_0)$, то есть плотность была правильно специфицирована, ММП-оценка состоятельная для $\theta_0$.

Для асимптотического распределения упрощение происходит, поскольку $-A_0=B_0$ в силу равенства информационных матриц, которое опять же предполагает, что плотность правильно определена. Эти результаты могут быть собраны в следующем утверждении.

\begin{proposition}[Распределение  ММП-оценки]: Если выполнены допущения:
\begin{enumerate}
\item Процесс порождающий данные описывается условной плотностью $f(y_i|x_i,\theta_0)$, которая используется для того, чтобы определить функцию правдоподобия.
\item Функция плотности $f(\cdot)$ удовлетворяет следующему условию $f(y,\theta^{(1)})=f(y,\theta^{(2)})$ тогда и только тогда, когда $\theta^{(1)}=\theta^{(2)}$.
\item Матрица 
\begin{equation}
A_0= \left. \plim \frac{1}{N} \frac{\partial^2 \mathcal{L}_{N}(\theta)}{\partial \theta \partial \theta'} \right|_{\theta_0}
\end{equation}
существует и невырождена.
\item Порядок дифференцирования и интегрирования логарифмической функции правдоподобия может быть изменен несколько раз.
\end{enumerate}
Тогда ММП-оценка $\hat{\theta}_{ML}$, которая определяется как решение условий первого порядка $ \partial N^{-1} \mathcal{L}_{N}(\theta) / \partial \theta=0$, состоятельна для $\theta_0$ и
\begin{equation}
\sqrt{N} (\hat{\theta}_{ML} - \theta_0) \xrightarrow{d} \mathcal{N}[0,{-A_0}^{-1}].
\end{equation}
\end{proposition}

Условие 1 говорит о том, что условная плотность правильно специфицирована, условия 1 и 2 гарантируют, что можно найти $\theta_0$; условие 3 аналогично предположению о $\plim N^{-1}X'X$ в случае оценки МНК; и условие 4 является необходимым для регулярности. Как и в общем случае предел по вероятности и математическое ожидание зависят от процесса порождающего данные для $(y,X)$, или только от $y$, если регрессоры предполагаются нестохастическими или  производится условный анализ, то есть при заданном $X$.

Ослабление условия 1 подробно рассматривается в Разделе 5.7. Большинство примеров ММП удовлетворяют условию 4, но оно исключает некоторые модели такие, как равномерно распределённый $y$ на интервале $[0,\theta]$, так как в этом случае диапазон $y$ зависит от $\theta$. Тогда не только $A_0 \not= -B_0$, но и глобальная ММП-оценка сходится со скоростью отличной от $\sqrt{N}$ и имеет предельное распределение, которое не является нормальным. См., например, Хирано и Портер (2003).

Учитывая утверждение 5.5, полученное асимптотическое распределение ММП-оценки  часто выражается следующим образом:
\begin{equation}
\hat{\theta}_{ML} \stackrel{a}{\sim} \mathcal{N} \left[ \theta, - \left( E \left[ \frac{\partial^2 \mathcal{L}_{N}(\theta)}{\partial \theta \partial \theta'} \right] \right) ^{-1} \right],
\end{equation}
где для простоты записи указание точки $\theta_0$ опускается, и мы предполагаем, что ЗБЧ применим, поэтому оператор $\plim$ в определении $A_0$ заменяется на $\lim E$, а затем убирается предел. Это обозначение часто используется в последующих главах.

Правая часть (5.48) является нижней границей Рао-Крамера, которая, из базового курса статистики, является нижней границей дисперсии несмещённых оценок в малых выборках. В больших выборках, рассматриваемых здесь, это нижняя граница для ковариационной матрицы состоятельных и асимптотически нормальных оценок с равномерной сходимостью к нормальному распределению $\sqrt{N}(\hat{\theta}-\theta_0)$  на компактных интервалах $\theta_0$ (см. Рао, 1973 , стр. 344-351). Грубо говоря ММП-оценка привлекательна тем, что имеет наименьшую асимптотическую дисперсию среди $\sqrt{N}$-состоятельных оценок. Этот результат требует сильного предположения о правильной спецификации условной плотности.

\subsection{Пример регрессии Вейбулла}

В качестве примера рассмотрим регрессию, основанную на распределении Вейбулла, которое используется для моделирования данных о длительности, таких как продолжительность периода безработицы (см. Главу 17).

Плотность распределения Вейбулла $f(y)=\gamma \alpha y^{\alpha-1}\exp(-\gamma y^{\alpha})$, где $y>0$ и параметры $\alpha>0$ и $\gamma>0$. Можно показать, что $\E[y] = \gamma^{-1/\alpha} \Gamma( \alpha^{-1}+1)$, где $\Gamma(\cdot)$ --- гамма-функция. Стандартная модель регрессии Вейбулл получается, если  $\gamma=\exp(x'\beta)$, в этом случае $\E[y|x]= \exp(-x' \beta/ \alpha) \Gamma(\alpha^{-1}+1))$. Учитывая независимость по $i$, логарифмическая функция правдоподобия выглядит следующим образом:
\[
N^{-1} \mathcal{L}_{N}(\theta)= N^{-1} \sum_i \{x'_i \beta + \ln\alpha + (\alpha-1)\ln y_i - \exp(x'_i\beta){y_i}^{\alpha} \}.
\]
Дифференцирование по $\beta$ и $\alpha$ приводит к условиям первого порядка:
\[
N^{-1} \sum_i \{1- \exp(x'_i\beta){y_i}^{\alpha} \} x_i=0, 
\]
\[
N^{-1} \sum_i \{\frac{1}{\alpha}+\ln y_i- \exp(x'_i\beta){y_i}^{\alpha}\ln y_i \}=0.
\]

В отличие от примера Пуассона, состоятельность по существу требует правильной спецификации распределения. Чтобы убедиться в этом, рассмотрим условия первого порядка для $\beta$. Неформальное условие (5.25), $\E[\{1-\exp(x'\beta){y}^{\alpha}\}x]=0$, требует, чтобы $\E[y^{\alpha}|x]=\exp(-x'\beta)$, где степень $\alpha$ может быть не только целым числом. Условия первого порядка для $\alpha$ приводят к ещё более необычным условиям на моменты $y$.

Итак, мы должны исходить из того, что это, в действительности, плотность Вейбулла с $\gamma=\exp(x'\beta_0)$ и $\alpha=\alpha_0$. Теорема 5.5 может быть применена, поскольку диапазон $y$ не зависит от параметров. Тогда из (5.48), ММП-оценка Вейбулла асимптотически нормальна с асимптотической дисперсией
\begin{equation}
\Var \begin{bmatrix} \hat{\beta} \\ \hat{\alpha} \end{bmatrix}  = \begin{pmatrix} -\E \begin{bmatrix} \sum_i -e^{x'_i\beta_0} {y_i}^{\alpha_0} x_i x'_i & \sum_i -e^{x'_i\beta_0}{y_i}^{\alpha_0}\ln(y_i)x_i \\ \sum_i -e^{x'_i\beta_0}{y_i}^{\alpha_0}\ln(y_i)x_i & \sum_i d_i \end{bmatrix} \end{pmatrix} ^{-1}.
\end{equation}
где $d_i=-(1/{\alpha_0}^{2})-e^{x'_i\beta_0}{y_i}^{\alpha_0} (\ln y_i)^2$, обратная матрица (5.49) должна быть получена обращением по блокам, потому что математическое ожидание недиагонального элемента $\partial^2 \mathcal{L}_{N}(\beta,\alpha)/ \partial \beta \partial \alpha$ не равно нулю. Упрощение имеет место в моделях с нулевым математическим ожиданием смешанной производной $\E[\partial^2 \mathcal{L}_{N}(\beta,\alpha)/ \partial \beta \partial \alpha']=0$ таких, как регрессия с нормально распределёнными ошибками, и в этом случае информационная матрица называется блочно-диагональной по $\beta$ и $\alpha$.

\subsection{Оценка ковариационной матрицы ММП-оценок} 

Есть несколько способов, чтобы состоятельно оценить ковариационную матрицу
экстремальной оценки, как уже отмечалось в Разделе 5.5.2 . Для ММП-оценки возникают дополнительные возможности, если равенство информационных матриц считается выполненным. Тогда ${A_0}^{-1}B_0{A_0}^{-1}$, ${-A_0}^{-1}$ и ${B_0}^{-1}$, все асимптотически эквивалентны, как и соответствующие состоятельные оценки этих величин. Детальное обсуждение ММП-оценок приведено у Дэвидсона и МакКиннона (1993, глава 18).

Сэндвич-оценка $\hat{A}^{-1}\hat{B}\hat{A}^{-1}$ называется оценкой Губера, в честь Губера (1967), или оценкой Уайта, в честь Уайта (1982), который работал над распределением ММП-оценки, не применяя равенство информационных матриц. Сэндвич-оценка в теории считается
более устойчивой, чем $-\hat{A}^{-1}$ или $\hat{B}^{-1}$. Важно отметить, однако, что причина неравенства информационных матриц может дополнительно привести к более фундаментальному осложнению в виде несостоятельности $\hat{\theta}_{ML}$. Это является предметом рассмотрения в Разделе 5.7.

\subsection{Вывод ММП условий регулярности}

Теперь мы формально получим  условия регулярности, изложенные в Разделе 5.6.3. Для простоты обозначений индекс $i$ и вектор регрессоров опускаются.

Начнём с получения первого условия (5.41). Интеграл плотности равен одному, то есть
\[
\int f(y|\theta)dy=1.
\]
Взяв производную по $\theta$ от обеих частей, получим $\frac{\partial}{\partial \theta} \int f(y|\theta)dy=0$. Если границы интегрирования (диапазон $y$) не зависят от $\theta$, то
\begin{equation}
\int \frac{\partial f(y|\theta)}{\partial \theta}dy=0.
\end{equation} 
Тогда из  $\partial \ln f(y|\theta)/\partial \theta= [\partial f(y|\theta) / \partial \theta]/ [f(y|\theta)]$ следует
\begin{equation}
\frac{\partial f(y|\theta)}{\partial \theta}=\frac{\partial \ln f(y|\theta)}{\partial \theta}f(y|\theta).
\end{equation} 
Подставляя (5.51) в (5.50), получим
\begin{equation}
\int \frac{\partial \ln f(y|\theta)}{\partial \theta}f(y|\theta)dy=0,
\end{equation} 
что то же самое, что (5.41) при условии, что математическое ожидание берется по плотности $f(y|\theta)$.

Теперь рассмотрим второе условие (5.42), чтобы получить более общий результат. Предположим
\[
\E[m(y,\theta)]=0,
\]
для некоторой (возможно, векторной) функции $m(\cdot)$. Тогда, когда ожидание берётся
по функции плотности $f(y|\theta)$
\begin{equation}
\int m(y,\theta)f(y|\theta)dy=0.
\end{equation} 
Продифференцировав обе части по $\theta'$ и предположив, что дифференцирование и интегрирование можно поменять местами, получим
\begin{equation}
\int \left( \frac{\partial m(y,\theta)}{\partial \theta'}f(y|\theta)+m(y,\theta)\frac{\partial f(y|\theta)}{\partial \theta'}\right)dy=0.
\end{equation} 
Подставляя (5.51) в (5.54), получим
\begin{equation}
\int \left(\frac{ \partial m(y,\theta)}{\partial \theta'}f(y|\theta)+m(y,\theta)\frac{\partial \ln  f(y|\theta)}{\partial \theta'}f(y|\theta) \right)dy=0,
\end{equation} 
или
\begin{equation}
\E \left[ \frac{\partial m(y,\theta)}{\partial \theta'} \right] = - \E \left[ m(y,\theta)\frac{\partial \ln  f(y|\theta)}{\partial \theta'} \right],
\end{equation} 
когда математическое ожидание берётся по плотности  $f(y|\theta)$. Условие регулярности (5.42) является частным случаем $m(y,\theta)= \partial \ln f(y|\theta)/\partial \theta$ и приводит к равенству информационных матриц (5.44). Более общий результат (5.56) приводит к обобщённому равенству информационных матриц (5.45).

Что происходит, когда интегрирование и дифференцирование нельзя поменять местами? Стартовая формула (5.50) уже не является верной, поскольку на основании фундаментальной теоремы анализа производная по $\theta$ от $\int f(y|\theta)dy$ включает дополнительный член, отражающий наличие функции $\theta$ в пределах интеграла. Тогда $\E[\partial \ln f(y|\theta)/\partial \theta] \not=0$.

Что происходит, когда плотность специфицирована неправильно? Тогда (5.52) остаётся в силе, но это не обязательно означает, что и (5.41) тоже, так как в (5.41) математическое ожидание больше не будет считаться с использованием плотности $f(y|\theta)$.

\section{Метод квази-максимального правдоподобия}

Оценка квази-ММП $\hat{\theta}_{QML}$ определяется как оценка, которая максимизирует логарифмическую функцию правдоподобия, которая неправильно специфицирована в результате неправильный спецификации плотности. Обычно такая неправильная спецификация приводит к несостоятельным оценкам.

В этом разделе сначала представлены общие свойства оценки квази-ММП, а затем некоторые особые случаи, когда оценка квази-ММП сохраняет состоятельность.

\subsection{Псевдо-истинные значения}

В принципе любая неправильная спецификация плотности может привести к несостоятельности, поскольку тогда математическое ожидание оценки $\E[\partial \ln f(y|x,\theta)/\partial \theta|_{\theta_0}]$ (см. раздел 5.6.4) считается с использованием функции отличной от $f(y|x,\theta_0)$.

Путём модификации общего доказательства состоятельности в Разделе 5.3.2 оценка квази-ММП $\hat{\theta}_{QML}$ сходится по вероятности к псевдо-истинному значению $\theta^*$, которое определяется следующим образом:
\begin{equation}
\theta^*= \argmax _{\theta \in \Theta} (\plim N^{-1} \mathcal{L}_{N}(\theta)).
\end{equation}

Предел по вероятности берётся по отношению к истинному процессу порождающему данные. Если истинный процесс порождающий данные отличается от предполагаемой плотности $f(y|x,\theta)$, используемой для формирования  $\mathcal{L}_{N}(\theta)$, то обычно $\theta^* \not = \theta_0$ и оценка квази-ММП несостоятельна.

Губер (1967) и Уайт (1982) показали, что асимптотическое распределение оценки квази-ММП похоже на асимптотическое распределение для оценки ММП за исключением того, что оно сосредоточено вокруг $\theta^*$ и больше нет равенства информационных матриц. Тогда

\begin{equation}
\sqrt{N}(\hat{\theta}_{QML} - \theta^*) \xrightarrow{d} \mathcal{N}[0,A^{*-1}B^{*}A^{*-1}],
\end{equation}
где $A^*$ и $B^*$ определены в (5.18) и (5.19) за исключением  того, что предел по вероятности берётся по неизвестному истинному процессу порождающему данные и оценивается  в точке $\theta^*$. Состоятельные оценки $\hat{A}^*$ и $\hat{B}^*$ могут быть получены как в Разделе 5.5.2 с помощью оценки $\hat{\theta}_{QML}$.

Этот результат используется для статистических выводов, если оценка квази-ММП сохраняет состоятельность. Если оценка квази-ММП не состоятельна, тогда обычно $\theta^*$ не имеет простой интерпретации кроме той, которая дана в следующем разделе. Тем не менее, (5.58) всё ещё может быть полезен, если, тем не менее, важна точность оценивания. Результат (5.58) также является мотивацией для теста Уайта для информационных матриц (см. Раздел 8.2.8), а также для теста Вуонга на различие параметрических моделей (см. Раздел 8.5.3).

\subsection{Расстояние Кульбака-Лейблера}

Вспомним из Раздела 4.2.3, что если $\E[y|x] \not = x' \beta_0$, то МНК-оценка ещё может быть интерпретирована как наилучший линейный предиктор $\E[y|x]$ при потере квадратных ошибок. Уайт (1982) предложил в сущности аналогичную интерпретацию для оценки квази-ММП.

Пусть $f(y|\theta)$ обозначает предполагаемую совместную плотность $y_1, \dots, y_n$ и пусть $h(y)$ обозначает истинную плотность, которая неизвестна, где для простоты зависимость от регрессоров не обозначена. Определим информационный критерий Кульбака-Лейблера (Kullback–Leibler information criterion):
\begin{equation}
KLIC = \E \left[ \ln \left( \frac{h(y)}{f(y|\theta)} \right) \right],
\end{equation}
где математическое ожидание берется по $h(y)$. Информационный критерий Кульбака-Лейблера принимает минимальное значение 0, когда существует $\theta_0$ такое, что $h(y)=f(y|\theta_0)$, то есть плотность правильно специфицирована, а большие значения информационного критерия Кульбака-Лейблера указывают на большую степень неуверенности относительно истинной плотности.

Тогда оценка квази-ММП $\hat{\theta}_{QML}$ минимизирует расстояние между $f(y|\theta)$ и $h(y)$, где расстояние измеряется с помощью информационного критерия Кульбака-Лейблера. Для получения этого результата обратим внимание, что при подходящих предположениях $\plim N^{-1} \mathcal{L}_{N}(\theta)=\E[\ln f(y|\theta)]$, поэтому $\hat{\theta}_{QML}$ сходится к $\theta^*$, которая максимизирует $\E[\ln f(y|\theta)]$. Однако это эквивалентно минимизации информационного критерия Кульбака-Лейблера, поскольку он равен $KLIC=\E[\ln h(y)]-\E[\ln f(y|\theta)]$ и первое слагаемое не зависит от $\theta$, поскольку математическое ожидание считается с использованием функции $h(y)$.

\subsection{Экспоненциальное семейство распределений}

В некоторых особых случаях оценки квази-ММП состоятельны даже тогда, когда плотность частично неправильно специфицирована. Одним хорошо известным примером является то, что оценка квази-ММП для линейной регрессионной модели при нормальности является состоятельной, даже если ошибки ненормальные при условии, что $\E[y|x]=x' \beta_0$. ММП-оценка Пуассона может послужить вторым примером (см. Раздел 5.3.4).

Похожая устойчивость к неправильной спецификации есть и в других моделях, основанных на плотности из экспоненциального семейства распределений. Для данного семейства плотность или вероятность может быть выражена как
\begin{equation}
f(y|\mu)=\exp \{ a(\mu)+b(y)+c(\mu)y \},
\end{equation}
для параметризации семейства с помощью математического ожидания, $\mu=\E[y]$. Можно показать, что для этой плотности $\E[y]=-[c'(\mu)]^{-1}a'(\mu)$ и $\Var[y]=[c'(\mu)]^{-1}$, где $c'(\mu)=\partial c(\mu) / \partial \mu$ и $a'(\mu)=\partial a(\mu) / \partial \mu$. Различные функции $a(\cdot)$ и $c(\cdot)$ приводят к различным плотностям. Величина $b(y)$ в (5.60) является нормализующей константой, которая обеспечивает, чтобы сумма вероятностей или интеграл были равны одному. Остальная часть плотности $\exp \{ a(\mu)+c(\mu)y \}$ является экспоненциальной функцией, линейной по $y$, что объясняет название семейства.

Большинство плотностей не может быть выражено в этой форме. Некоторые важные плотности --- это плотности из экспоненциального семейства, в том числе и приведенные в таблице 5.4. Эти плотности, уже представленные в таблице 5.3, выражены в таблице 5.4 в виде (5.60). Другие примеры экспоненциального семейства --- это биномиальное распределения с известным числом испытаний (частный случай --- Бернулли), некоторые негативные биномиальные модели (частные случаи --- геометрическое распределение и распределение Пуассона), а также однопараметрическое гамма-распределение (частный случай --- экспоненциальное).

\begin{table}[h]
\begin{center}
\caption{\label{tab:LEF} Плотности и вероятности экспоненциального семейства:  распространенные примеры}
\begin{tabular}[t]{llll}
\hline
\hline
\bf{Распределение} & $f(y)=\exp \{ a(\cdot)+b(y)+c(\cdot)y \}$ & $\E(y)$ &  $\Var[y]=[c'(\mu)]^{-1}$ \\
\hline
Нормальное ($\sigma^2$ известна) & $\exp \{ \frac{-\mu^2}{2 \sigma^2} - \frac{1}{2} \ln(2 \pi \sigma^2)- \frac{y^2}{2 \sigma^2} + \frac{\mu}{\sigma^2} y \}$ & $\mu$ & $\sigma^2$ \\
Бернулли & $\exp \{\ln(1-p) + \ln[p/(1-p)]y \}$ & $\mu=p$ & $\mu(1-\mu)$ \\
Экспоненциальное & $\ln \{ \ln \lambda - \lambda y \}$ & $\mu=1/\lambda$ & $\mu^2$  \\
Пуассона & $\ln \{ -\lambda - \ln y! + y \ln \lambda \}$ & $\mu=\lambda$ & $\mu$ \\
\hline
\hline
\end{tabular}
\end{center}
\end{table}

Для регрессии параметр $\mu=\E[y|x]$ моделируется как 
\begin{equation}
\mu=g(x,\beta),
\end{equation}
для заданной функции  $g(\cdot)$, которая меняется в зависимости от модели (см. Раздел 5.7.4), частично в зависимости от ограничений на диапазон $y$ и, следовательно, $\mu$. Логарифм правдоподобия в таком случае:
\begin{equation}
\mathcal{L}_{N}(\beta)=\sum_{i=1}^{N} \{ a(g(x_i,\beta))+b(y_i)+c(g(x_i,\beta))y_i \},
\end{equation}
с условием первого порядка, которое может быть представлено иначе, с использованием вышеупомянутой информации о первых двух моментах $y$ в следующем виде:
\begin{equation}
\frac{\partial \mathcal{L}_{N}(\beta)}{\partial \beta}=\sum_{i=1}^{N} \frac{y_i-g(x_i,\beta)}{{\sigma_i}^2} \times \frac{\partial g(x_i,\beta)}{\partial \beta}=0,
\end{equation}
где ${\sigma_i}^2=[c'(g(x_i,\beta))]^{-1}$ --- предполагаемая функция дисперсии, соответствующая конкретному распределению из семейства. Например, для распределения Бернулли, экспоненциального и Пуассона ${\sigma_i}^2$ равны, соответственно, $g_i(1-g_i)$,$1/{g_i}^2$,$g_i$, где $g_i=g(x_i,\beta)$.

Оценка квази-ММП является решением этих уравнений, но уже не предполагается, что распределение правильно специфицирована. Гурьеру, Монфор, и Трогнон (1984а) доказали, что оценка квази-ММП $\hat{\beta}_{QML}$ состоятельна при условии, что $\E[y|x]=g(x,\beta_0)$. Это ясно из ожидаемого значения условия первого порядка (5.63), которое оценивается при $\beta=\beta_0$ и является взвешенной суммой ошибок $y-g(x,\beta_0)$ с ожидаемым значением, равным нулю, если $\E[y|x]=g(x,\beta_0)$.

Таким образом, оценка квази-ММП, основанная на распределении экспоненциального семейства, состоятельна, если только условное математическое ожидание $y$ при заданном $x$ правильно специфицировано. Следует отметить, что фактический процесс порождающий данные для $y$ не обязательно должен быть из экспоненциального семейства. Семейству принадлежит плотность спецификации, и она может быть потенциально неправильна указана.

Однако даже при правильном условном математическом ожидании корректировка базового результата ММП для дисперсии, стандартных ошибок и $t$-статистики, основанной на $-{A_0}^{-1}$, является оправданной. В общем случае нужно применять сэндвич-форму ${A_0}^{-1}B_0{A_0}^{-1}$, кроме ситуации, когда условная дисперсия $y$ при заданном $x$ тоже определена правильно, в этом случае $A_0=-B_0$. Однако для моделей Бернулли $A_0=-B_0$ верно всегда. Состоятельные стандартные ошибки можно получить с помощью (5.36) и (5.38).

Экспоненциальное семейство является особым случаем. Обычно неправильная спецификация любого аспекта плотности приводит к несостоятельности ММП-оценок. Даже в случае экспоненциального семейства оценки квази-ММП могут использоваться только для прогнозирования условного математического ожидания, тогда как с правильно специфицированной плотностью можно предсказать условное распределение.

\subsection{Обобщённые линейные модели}

Модели, основанные на экспоненциально семействе, называются обобщёнными линейными моделями (ОЛМ) в литературе статистики (см. книгу с таким названием Маккаллоу и Нельдера, 1989). Класс обобщённых линейных моделей наиболее широко используется в прикладной статистике для нелинейных пространственных регрессий, как следует из таблицы 5.3, они включают следующие регрессионные модели: нелинейный метод наименьших квадратов, модель Пуассона, геометрическую модель, пробит, логит, биномиальную модель с известным числом испытаний, гамма и экспоненциальная модель. Мы предлагаем краткий обзор с введением стандартной терминологии ОЛМ.

Стандартные ОЛМ определяют условное математическое ожидание $g(x,\beta)$ в (5.61) в более простой одноиндексной форме такой, чтобы $\mu=g(x'\beta)$. Тогда  $g^{-1}(\mu)=x'\beta$, а функция $g^{-1}(\cdot)$ называется функцией связи. Например, обычная спецификация для модели Пуассона соответствует логарифмической  функции связи, так как если $\mu=\exp(x'\beta)$, то $\ln \mu =x'\beta$.

Условия первого порядка (5.63) можно представить в виде $\sum_i [(y_i-g_i)/c'(g_i)]g'_i x_i=0$, где $g_i=g(x'_i\beta)$ и $g'_i=g'(x'_i\beta)$. Есть вычислительные преимущества в выборе функции связи, для которой $c'(g(\mu))=g'(\mu)$, поскольку тогда условия первого порядка сводятся к
$\sum_i (y_i-g_i)x_i=0$, и ошибка $(y_i-g_i)$ ортогональна регрессорам. Каноническая функция связи --- такая функция $g^{-1}(\cdot)$, что приводит к $c'(g(\mu))=g'(\mu)$ и меняется при изменении $c(\mu)$ и, следовательно, ОЛМ. Каноническая функция связи приводит к $\mu=x'\beta$ для нормально распределённных данных, к $\mu=\exp(x'\beta)$ для данных, распределённых по Пуассону и к $\mu=\exp(x'\beta)/[1+\exp(x'\beta)]$ для бинарных данных. Последняя спецификация --- это логит модель, приведённая ранее в таблице 5.3.

Отклонением называется удвоенная разница между максимально достижимой логарифмической функцией правдоподобия и оценённой логарифмической функцией правдоподобия. Это мера, которая обобщает сумму квадратов остатков в линейной регрессии для других регрессионных моделей экспоненциального семейства.

Модели, основанные на экспоненциальном семействе, являются довольно ограниченными, поскольку все моменты зависят только от одного параметра $\mu=g(x'\beta)$. В литературе рассматривают дополнения к моделям, в которых сделано  удобное предположение, что дисперсия распределения из экспоненциального семейства потенциально неправильно специфицирована на множитель $\alpha$, так что $\Var[y|x]=\alpha \times [c'(g(x,\beta))]^{-1}$, где обязательно $\alpha \not = 1$.

Например, для модели Пуассона пусть $\Var[y|x]=\alpha g(x,\beta)$, а не $g(x,\beta)$. При данной неправильной спецификации дисперсии можно показать, что $B_0=-\alpha A_0$, поэтому ковариационная матрица оценки квази-ММП это $-\alpha {A_0}^{-1}$, что требует только изменения масштаба обычной (не сэндвич) оценки ММП ковариационной матрицы $-{A_0}^{-1}$ с помощью умножения на $\alpha$. Часто используемая состоятельная оценка для $\alpha$ --- это $\hat{\alpha}=(N-K)^{-1} \sum_i(y_i-\hat{g_i})^{2}/ {\hat{\sigma_i}}^2$, где $\hat{g_i}=g(x_i,\hat{\beta}_{QML})$, ${\hat{\sigma_i}}^2=[c'(\hat{g_i})]^{-1}$ и берётся деление на $ N-K$, а не $N$, чтобы обеспечить лучшую оценку в малых выборках. См. предыдущие ссылки и Кэмерона и Триведи (1986, 1998) для дальнейших подробностей.

Многие статистические пакеты включают ОЛМ модуль, по умолчанию дающий стандартные ошибки, которые являются правильными при условии $\Var[y|x]=\alpha [c'(g(x,\beta))]^{-1}$. Кроме того, можно получить оценки с использованием ММП со стандартными ошибками, полученными с помощью скорректированной сэндвич формулы ${A_0}^{-1}B_0{A_0}^{-1}$. На практике сэндвич стандартные ошибки аналогичны тем, которые получены с применением простой коррекции ОЛМ. Ещё одним способом оценки ОЛМ является метод взвешенных нелинейных наименьших квадратов, описанный в конце Раздела 5.8.6.

\subsection{Оценки квази-ММП для многомерных зависимых переменных}

Эта глава посвящена скалярным зависимым переменным, но теория может применяться также к многомерному случаю. Предположим, что зависимая переменная $y$ --- вектор размера $m \times 1$ и данные $(y_i,x_i)$, $i=1,\dots, N$ независимы по $i$. Примеры, приведённые в последующих главах, включают внешне несвязанные уравнения, панельные данные с $m$ наблюдениями для $i$-го индивида для той же зависимой переменной и кластерных данных, где данные для $i j$-ого наблюдения коррелированы по $m$ возможных значений индекса $j$.

При заданной спецификации $f(y|x,\theta)$ совместной функции плотности $y=(y_1, \dots, y_m)$ при условии $x$, полностью эффективная оценка ММП максимизирует $N^{-1} \sum_i \ln f(y_i|x_i,\theta)$, как отмечалось после (5.39). Тем не менее, в многомерном случае совместная плотность $y$ может быть сложной. Возможна более простая оценка при знании только $m$ одномерных плотностей $f_j(y_j|x,\theta)$, $j=1,\dots,m$, где $y_j$ --- $j$-ый элемент $y$. Например, для многомерных данных можно было бы работать с $m$ независимыми одномерными отрицательными биномиальными вероятностями для каждого значения, а не с более богатой многофакторной моделью, в которой возможна корреляция.

Рассмотрим  оценку квази-ММП $\hat{\theta}_{QML}$, основанную на произведении одномерных плотностей $\Pi_j f_j(y_j,|x,\theta)$, которая максимизирует
\begin{equation}
\mathcal{Q}_{N}(\theta)=\frac{1}{N} \sum_{i=1}^{N} \sum_{j=1}^{m} \ln f(y_{ij}|x_i,\theta).
\end{equation}
Вулдридж (2002) называет эту оценку частичной оценкой ММП, так как плотность была лишь частично специфицирована. 

Частичная оценка ММП --- это М-оценка с $q_i=\sum_j \ln f(y_{ij}|x_i,\theta)$. Необходимое условие состоятельности (5.25) требует, чтобы $\E[\sum_j \partial f(y_{ij}|x_i,\theta) / \partial \theta|_{\theta_0}]=0$. Это условие выполняется, если предельные плотности $f(y_{ij}|x_i,\theta)$ правильно специфицированы, поскольку тогда в силу условия регулярности (5.41) $\E[ \partial f(y_{ij}|x_i,\theta)/ \partial \theta|_{\theta_0}]=0$ .

Таким образом, частичная оценка ММП состоятельна при условии, что одномерные плотности $f_j(y_j|x,\theta)$ являются правильно специфицированными. Состоятельность не требует, чтобы  $f(y|x,\theta)= \Pi_j f_j(y_j,|x,\theta)$. Однако зависимость $y_1, \dots, y_m$  приведёт к невыполнению равенства информационных матриц, поэтому стандартные ошибки должны быть вычислены с использованием сэндвич-формы для ковариационной матрицы с
\begin{equation}
A_0= \left. \frac{1}{N} \sum_{i=1}^{N} \sum_{j=1}^{m} \frac{\partial^2 \ln  f_{ij}}{ \partial \theta \partial \theta'} \right|_{\theta_0}, 
\end{equation}
\[
B_0= \left. \frac{1}{N} \sum_{i=1}^{N} \sum_{j=1}^{m} \sum_{k=1}^{m} \frac{\partial \ln  f_{ij}}{\partial \theta} \right|_{\theta_0} \left. \frac{\partial \ln  f_{ik}}{\partial \theta'} \right|_{\theta_0} 
\]

где $f_{ij}=f(y_{ij}|x_i,\theta)$. Кроме того, частичная оценка ММП является неэффективной по сравнению с оценкой ММП, основанной на совместной плотности. Дальнейшее обсуждение приведено в Разделах 6.9 и 6.10.

\section{Нелинейный метод наименьших квадратов}

Оценка нелинейного метода наименьших квадратов является естественным продолжением оценки линейного метода наименьших квадратов для нелинейной модели с $\E[y|x]=g(x,\beta)$, где $g(\cdot)$ является нелинейной по $\beta$. Анализ и результаты являются по существу такими же, как для метода линейных наименьших квадратов с одним изменением, что в формуле для ковариационной матрицы вектор регрессоров $x$ заменяется на $\partial g(x,\beta)/ \partial \beta|_{\hat{\beta}}$, производную функции условного математического ожидания при $\beta=\hat{\beta}$.

Для микроэконометрического анализа контроль за гетероскедастичностью ошибок может оказаться необходимым так же, как и в линейном случае. Оценки НМНК и обобщения, которые моделируют гетероскедастичные ошибки, как правило, менее эффективны, чем оценки ММП, но они широко используются в микроэконометрике, потому что они полагаются на более слабые предположения о распределении.

\begin{table}[h]
\begin{center}
\caption{\label{tab:NLS} Нелинейный метод наименьших квадратов: распространённые примеры}
\begin{tabular}[t]{ll}
\hline
\hline
\bf{Модель} & \bf{Функция регрессии} $g(x,\beta)$ \\
\hline
Экспоненциальная & $\exp(\beta_1 x_1 + \beta_2 x_2 + \beta_3 x_3)$ \\
Регрессор в степени & $\beta_1 x_1 + \beta_2 x_2^{\beta_3}$ \\
Производственная & $\beta_1 {x_1}^{\beta_2} {x_2}^{\beta_3}$\\ 
функция Кобба-Дугласа & \\
Производственная функция & $[\beta_1 {x_1}^{\beta_3} + \beta_2 x_2^{\beta_3}]^{1/\beta_3}$\\
с постоянной эластичностью замещения & \\ 
Нелинейные ограничения & $\beta_1 x_1 + \beta_2 x_2 + \beta_3 x_3$, where $\beta_3=-\beta_2 \beta_1$ \\
\hline
\hline
\end{tabular}
\end{center}
\end{table}

\subsection{Нелинейная модель регрессии}

Нелинейная модель регрессии определяет скалярную зависимую переменную $y$ так, чтобы её условное математическое ожидание равнялось
\begin{equation}
\E[y_i|x_i]=g(x_i,\beta),
\end{equation}
где $g(\cdot)$ является заданной функцией, $x$ --- вектор объясняющих переменных, а $\beta$ --- вектор параметров размера $k \times 1$. Линейная модель регрессии в Главе 4 является частным случаем $g(x,\beta)=x'\beta$.

Наиболее распространённые причины спецификации нелинейной функции для $\E[y|x]$ включают ограничение множества значений (например, для обеспечения того, $\E[y|x]>0$) и спецификацию моделей спроса или предложения, стоимости или расходов, удовлетворяющих ограничениям из теории производства или потребления. Некоторые часто используемые нелинейные модели регрессий приведены в таблице 5.5.

\subsection{Оценки нелинейного метода наименьших квадратов}

Остатки определяются как разница между зависимой переменной и её условным математическим ожиданием, $y_i-g(x_i,\beta)$. Оценка нелинейного метода наименьших квадратов $\hat{\beta}_{NLS}$ минимизирует сумму квадратов остатков, $\sum_i (y_i-g(x_i,\beta))^{2}$, или, что эквивалентно, максимизирует
\begin{equation}
\mathcal{Q}_{N}(\beta)= - \frac{1}{2N} \sum_{i=1}^{N} (y_i-g(x_i,\beta))^{2},
\end{equation}
где коэффициент $1/2$ введён для упрощения последующего анализа.

Дифференциирование даёт условия первого порядка для НМНК:
\begin{equation}
\frac {\partial \mathcal{Q}_{N}(\beta)} {\partial \beta}= \frac{1}{N} \sum_{i=1}^{N} \frac{\partial g_i}{\partial \beta}(y_i-g_i)=0,
\end{equation}
где $g_i=g(x_i,\beta)$. Эти условия требуют, чтобы остатки $(y-g)$ были ортогональны $\partial g/ \partial \beta$, а не $x$, как и в линейном случае. Здесь нет явного решения для $\hat{\beta}_{NLS}$, которая вместо этого вычисляется с использованием итерационных методов (которые есть в Главе 10).

Нелинейная модель регрессии может быть более компактно представлена в матричном виде. Записав наблюдения одно под другим получим следующее:
\begin{equation}
\begin{bmatrix} y_1 \\ \cdot \\ \cdot \\ \cdot \\ y_N \end{bmatrix} = \begin{bmatrix} g_1 \\ \cdot \\ \cdot \\ \cdot \\ g_N \end{bmatrix} + \begin{bmatrix} u_1 \\ \cdot \\ \cdot \\ \cdot \\ u_N \end{bmatrix},
\end{equation}
где $g_i=g(x_i,\beta)$ или, что, эквивалентно,
\begin{equation}
y=g+u,
\end{equation}
где $y$, $g$ и $u$ --- векторы размера $N \times 1$ с $i$-ыми элементами $y_i$, $g_i$ и $u_i$ соответственно.
Тогда
\[
\mathcal{Q}_{N}(\beta)=- \frac{1}{2N} (y-g)'(y-g)
\]
и
\begin{equation}
\frac {\partial \mathcal{Q}_{N}(\beta)} {\partial \beta}= \frac{1}{N} \frac{\partial g'}{\partial \beta}(y-g),
\end{equation}
где
\begin{equation}
\frac{\partial g'}{\partial \beta}= \begin{matrix} \begin{bmatrix} 
\frac{\partial g_1}{\partial \beta_1} & \cdots & \frac{\partial g_N}{\partial \beta_1}\\
\cdot &  & \cdot \\
\cdot &  & \cdot \\
\cdot &  & \cdot \\
\frac{\partial g_1}{\partial \beta_k} & \cdots & \frac{\partial g_N}{\partial \beta_k}\\
\end{bmatrix} \end{matrix}
\end{equation}
это матрица размера $K \times N$ частных производных $g(x,\beta)'$ по $\beta$.

\subsection{Распределение оценок нелинейного метода наименьших квадратов}

Распределение оценки НМНК будет меняться в зависимости от процесса порождающего данные. Процесс порождающий данные всегда может быть записан в следующем виде:
\begin{equation}
y_i=g(x_i,\beta_0)+u_i,
\end{equation}
в виде нелинейной регрессионной модели с добавочным членом --- ошибкой $u$. Условное математическое ожидание правильно специфицировано, если $\E[y|x]=g(x,\beta_0)$ для процесса порождающего данные. Тогда ошибка должна удовлетворять условию: $\E[u|x]=0$.

Учитывая условия первого порядка НМНК (5.68), необходимое условие состоятельности (5.25) становится
\[
\E[\partial g(x,\beta) / \partial \beta|_{\beta_0} \times (y-g(x_i,\beta_0))]=0.
\]

Учитывая (5.73), это равносильно условию $\E[\partial g(x,\beta) / \partial \beta|_{\beta_0} \times u]=0$. Это верно, если $\E[u|x]=0$, таким образом, состоятельность требует правильной спецификации условного математического ожидания, как и в линейном случае. Если вместо этого $\E[u|x] \not = 0$, то для состоятельной оценки требуется применение нелинейных инструментальных методов (которые представлены в Разделе 6.5).

Предельное распределение $\sqrt{N}(\hat{\beta}_{NLS}-\beta_0)$ получают с использованием точного разложения первого порядка в ряд Тейлора условий первого порядка (5.68). Получаем
\[
\sqrt{N}(\hat{\beta}_{NLS}-\beta_0)=-\left( -\frac{1}{N}  \sum_{i=1}^{N} \frac{\partial g_i}{\partial \beta} \frac{\partial g_i}{\partial \beta'} + \left. \frac{1}{N} \frac{\partial^2 g_i}{\partial \beta \partial \beta'} (y_i-g_i) \right|_{\beta^+} \right) ^{-1} \left. \frac{1}{\sqrt{N}} \sum_{i=1}^{N} \frac{\partial g_i}{\partial \beta} u_i \right|_{\beta_0},
\]
для некоторого $\beta^+$  между $\hat{\beta}_{NLS}$ и $\beta_0$. Для $A_0$ в (5.18) возможно упрощение, так как член, включающий $(\partial^2 g/ \partial \beta \partial \beta')$ исчезает, поскольку $\E[u|x]=0$. Поэтому нам нужно рассматривать асимптотически только
\[
\sqrt{N}(\hat{\beta}_{NLS}-\beta_0)= \left( \left.\frac{1}{N} \sum_{i=1}^{N} \frac{\partial g_i}{\partial \beta} \frac{\partial g_i}{\partial \beta'} \right|_{\beta_0}  \right) ^{-1} \left. \frac{1}{\sqrt{N}} \sum_{i=1}^{N} \frac{\partial g_i}{\partial \beta} u_i \right|_{\beta_0},
\]
Эта формула совпадает с формулой для МНК, см. Раздел 4.4.4, за исключением того, что $x_i$ заменяется на $\partial g_i / \partial \beta'|_{\beta_0}$.

Получаем утверждение, аналогичное утверждению 4.1 для МНК-оценки.

\begin{proposition}[Распределение оценок НМНК]: Сделаем следующие предположения:
\begin{enumerate}
\item Модель задана уравнением (5.73), то есть $y_i=g(x_i,\beta_0)+u_i$.
\item В процессе порождающем данные $\E[u_i|x_i]=0$ и $\E[uu'|X]=\Omega_0$, где $\Omega_{0,ij}=\sigma_{ij}$.
\item Функция математического ожидания  $g(\cdot)$ удовлетворяет $g(x,\beta^{(1)})=g(x,\beta^{(2)})$, если и только если $\beta^{(1)}=\beta^{(2)}$.
\item Матрица
\begin{equation}
A_0= \left. \plim \frac{1}{N} \sum_{i=1}^{N} \frac{\partial g_i}{\partial \beta} \frac{\partial g_i}{\partial \beta'} \right|_{\beta_0} =\plim \left. \frac{1}{N} \frac{\partial g'}{\partial \beta} \frac{\partial g'}{\partial \beta'} \right|_{\beta_0}
\end{equation}
существует и конечная невырожденная.
\item $N^{-1/2} \sum_{i=1}^{N} \partial g_i/ \partial \beta \times u_i|_{\beta_0} \xrightarrow{d} \mathcal{N}[0,B_0]$, где
\begin{equation}
B_0= \left. \plim \frac{1}{N} \sum_{i=1}^{N} \sum_{j=1}^{N} \sigma_{ij} \frac{\partial g_i}{\partial \beta} \frac{\partial g_j}{\partial \beta'} \right|_{\beta_0} = \left. \plim \frac{1}{N} \frac{\partial g'}{\partial \beta} \Omega_0 \frac{\partial g'}{\partial \beta'} \right|_{\beta_0}.
\end{equation}
\end{enumerate}

Тогда оценка НМНК $\hat{\beta}_{NLS}$, которая определяется как решение условий первого порядка $\partial N^{-1} \mathcal{Q}_{N}(\beta)/ \partial \beta=0$, состоятельна для $\beta_0$ и

\begin{equation}
\sqrt{N}(\hat{\beta}_{NLS}-\beta_0) \xrightarrow{d} \mathcal{N}[0,{A_0}^{-1}B_0{A_0}^{-1}].
\end{equation}
\end{proposition}

Из условий с 1 по 3 следует, что функция регрессии правильно специфицирована и регрессоры не коррелируют с ошибками и что определён $\beta_0$. Ошибки могут быть гетероскедастичны и коррелированы по $i$. Условия 4 и 5 подразумевают релевантные предельные результаты, необходимые для применения теоремы 5.3. Для выполнения условия 5 необходимо, чтобы были наложены некоторые ограничения на корреляцию ошибок по $i$. Пределы по вероятности в (5.74) и (5.75) для $X$ становятся обычными пределами, если $X$ нестохастические.

Матрицы $A_0$ и $B_0$ в предложении 5.6 такие же, как и матрицы $M_{XX}$ и $M_{X \Omega X}$ в Разделе 4.4.4 для оценок МНК при $x_i$ заменённом на $\partial g_i/\partial \beta|_{\beta_0}$. Асимптотическая теория для НМНК такая же, как и для МНК, не считая этого единственного изменения.

В частном случае сферических ошибок, $\Omega_0={\sigma_0}^{2}I$, поэтому $B_0={\sigma_0}^{2}A_0$ и $\Var[\hat{\beta}_{NLS}]={\sigma_0}^{2}{A_0}^{-1}$. Тогда оценки нелинейного метода наименьших квадратов асимптотически эффективны среди оценок наименьших квадратов. Тем не менее, для пространственных данных ошибки не обязательно гетероскедастичны.

С учётом утверждения 5.6 получающееся асимптотическое распределение оценки НМНК может быть выражено следующим образом:
\begin{equation}
\hat{\beta}_{NLS} \stackrel{a}{\sim} \mathcal{N}[\beta,(D'D)^{-1} D' \Omega_0 D (D'D)^{-1}],
\end{equation}
где производная матрицы $D=\partial g / \partial \beta'|_{\beta_0}$ имеет $i$-ую строку $\partial g_i/ \partial \beta'|_{\beta_0}$ (см. (5.72)), для простоты обозначений указание точки $\beta_0$ опускается, и мы предполагаем, что ЗБЧ может быть применён, чтобы оператор $\plim$ в определениях $A_0$ и $B_0$ был заменён на $\lim E$, а затем убираем знак предела. Этот приём часто используется в последующих главах.

\subsection{Ковариационная матрица оценок НМНК}

Мы рассматриваем статистические выводы для обычной ситуации в микроэконометрике с независимыми ошибками с гетероскедастичностью неизвестной функциональной формы. Это требует состоятельной оценки ${A_0}^{-1}B_0{A_0}^{-1}$, определённой в утверждении 5.6. 

Для $A_0$, определённой в (5.74), можно непосредственно использовать очевидную оценку
\begin{equation}
\hat{A}= \left. \frac{1}{N} \sum_{i=1}^{N} \frac{\partial g_i}{\partial \beta} \right|_{\hat{\beta}} \left. \frac{\partial g_i}{\partial \beta'} \right|_{\hat{\beta}},
\end{equation}
поскольку в $A_0$ не используются моменты ошибок.

Учитывая независимость по $i$ двойная сумма в $B_0$, определённая в (5.75), упрощается до одной суммы:
\[
B_0= \left. \plim \frac{1}{N} \sum_{i=1}^{N} {\sigma_i}^2 \frac{\partial g_i}{\partial \beta} \frac{\partial g_i}{\partial \beta'} \right|_{\beta_0}.
\]
Что касается МНК-оценки (см. Раздел 4.4.5) необходимо только состоятельно оценить сумму матриц $B_0$ размера $K \times K$. Это не требует состоятельности оценки ${\sigma_i}^2$, $N$ отдельных компонентов в сумме.

Уайт (1980b) определил условия, при которых
\begin{equation}
\hat{B}= \left. \frac{1}{N} \sum_{i=1}^{N} {\hat{u}_i}^{2} \frac{\partial g_i}{\partial \beta} \frac{\partial g_i}{\partial \beta'}\right|_{\hat{\beta}}= \left. \frac{1}{N}  \frac{\partial g'}{\partial \beta}|_{\hat{\beta}} \hat{\Omega} \frac{\partial g}{\partial \beta'}\right|_{\hat{\beta}} 
\end{equation}
состоятельна для $B_0$, где $\hat{u}_{i}=y_i-g(x_i,\hat{\beta})$, $\hat{\beta}$ состоятельна для $\beta_0$ и
\begin{equation}
\hat{\Omega}=\Diag[{\hat{u}_i}^{2}].
\end{equation}
Мы получаем следующую устойчивую к гетероскедастичности оценку асимптотической ковариационной матрицы оценки НМНК:
\begin{equation}
\widehat{\Var}[\hat{\beta}_{NLS}]=(\hat{D'}\hat{D})^{-1} \hat{D'} \hat{\Omega} \hat{D} (\hat{D'}\hat{D})^{-1},
\end{equation}
где $\hat{D}=\partial g/ \partial \beta'|_{\hat{\beta}}$. Это уравнение такое же, как в случае МНК в Разделе 4.4.5 при замене матрицы регрессоров $X$ на $\hat{D}$. На практике может быть использована коррекция степеней свободы, чтобы $\hat{B}$ в формуле (5.79) вычислялось с помощью деления на $(N-K)$, а не $N$. Тогда правую часть в (5.81) следует умножить на $N/(N-K)$.

Обобщение для ошибок, коррелированных по $i$, приведено в Разделе 5.8.7.

\subsection{Пример экспоненциальной регрессии}

В качестве примера предположим, что $y$ при заданном $x$ имеет экспоненциальное условное математическое ожидание, чтобы $\E[y|x]=\exp(x'\beta)$. Модель может быть выраженa в виде нелинейной регрессии с
\[
y=\exp(x'\beta)+u,
\]
где ошибка $u$ имеет $\E[u|x]=0$ и ошибки потенциально гетероскедастичны.

Оценки НМНК удовлетворяют следующим условиям первого порядка:
\begin{equation}
N^{-1} \sum_i (y_i-\exp(x'_i\beta))\exp(x'_i\beta)x_i=0,
\end{equation}
таким образом, состоятельность $\hat{\beta}_{NLS}$ требует только, чтобы условное математическое ожидание было правильно специфицировано с $\E[y|x]=\exp(x'\beta_0)$. Здесь $\partial g/ \partial \beta = \exp(x'\beta)x$, поэтому результат обобщённого НМНК (5.81) даёт оценки, устойчивые к гетероскедастичности:
\begin{equation}
\widehat{\Var}[\hat{\beta}_{NLS}]=\left( \sum_i e^{2 x'_i \hat{\beta}} x_i x'_i \right) ^{-1} \sum_i {\hat{u}_i}^{2} e^{2 x'_i \hat{\beta}} x_i x'_i \left( \sum_i e^{2 x'_i \hat{\beta}} x_i x'_i \right) ^{-1},
\end{equation}
где $\hat{u}_i=y_i-\exp(x'_i \hat{\beta}_{NLS})$.

\subsection{Взвешенный НМНК и допустимый обобщенный НМНК}

Для пространственных данных ошибки часто гетероскедастичны. Тогда допустимый обобщённый НМНК, который делает поправку на гетероскедастичность, более эффективен, чем НМНК.

Допустимый обобщённый нелинейный метод наименьших квадратов (ДОНМНК) всё же менее эффективен, чем ММП. Заметным исключением является то, что оценки ДОНМНК асимптотически эквивалентны оценкам ММП при условной плотности $y$ принадлежащей экспоненциальному семейству. Частным случаем этого исключения является тот факт, что оценка ДОМНК асимптотически эквивалентна оценке ММП в линейной регрессии при нормальности ошибок.

\begin{table}[h]
\begin{center}
\caption{\label{tab:NLSdisp} Оценки нелинейного метода наименьших квадратов и оценки их асимптотических ковариационных матриц}
\begin{minipage}{16cm}
\begin{tabular}[t]{llc}
\hline
\hline
\bf{Оценка}\footnote{Берутся функции для нелинейной регрессионной модели с ошибкой $u=y-g$, определённой в (5.70), и условной ковариационной матрицей остатков $\Omega$. $\hat{D}$ является производной от вектора условного математического ожидания в зависимости от $\beta'$ в точке $\hat{\beta}$. Для ДОНМНК предполагается, что $\hat{\Omega}$ состоятельна для $\Omega$. Для НМНК и взвешенного НМНК используются ковариационные матрицы, устойчивые к гетероскедастичности, $\hat{\Omega}$ равна диагональной матрице с квадратами остатков на диагонали, оценка, которая не обязана быть состоятельной для $\Omega$.} & \bf{Целевая функция} & \bf{Оценочная асимптотическая} \\
& & \bf{ковариационная матрица}\\ 
\hline
НМНК & $\mathcal{Q}_{N}(\beta)=\frac{-1}{2N}u'u$ & $(\hat{D'}\hat{D})^{-1} \hat{D'} \hat{\Omega} \hat{D} (\hat{D'}\hat{D})^{-1}$\\
ДОНМНК & $\mathcal{Q}_{N}(\beta)=\frac{-1}{2N}u' \Omega(\hat{\gamma})^{-1} u$ & $(\hat{D'} {\hat{\Omega}}^{-1} \hat{D})^{-1}$ \\
взвешенный НМНК & $\mathcal{Q}_{N}(\beta)=\frac{-1}{2N}u' \hat{\Sigma}^{-1} u$ & $(\hat{D'} \hat{\Sigma}^{-1} \hat{D})^{-1} \hat{D'} \hat{\Sigma}^{-1} \hat{\Omega} \hat{\Sigma}^{-1} \hat{D} (\hat{D'} \hat{\Sigma}^{-1} \hat{D})^{-1}$ \\ 
\hline
\hline
\end{tabular}
\end{minipage}
\end{center}
\end{table}

Если гетероскедастичность неправильно смоделирована, тогда оценка ДОНМНК остаётся состоятельной, но в таком случае нужно получить стандартные ошибки, которые являются устойчивыми к неправильной спецификации гетероскедастичности. Анализ очень похож на анализ для линейной модели, приведённый в Разделе 4.5.

\begin{center}
Допустимый обобщённый нелинейный метод наименьших квадратов
\end{center}

Оценка допустимого обобщённого нелинейного метода наименьших квадратов $\hat{\beta}_{FGNLS}$ максимизирует
\begin{equation}
\mathcal{Q}_{N}(\beta)=-\frac{1}{2N}(y-g)' \Omega(\hat{\gamma})^{-1} (y-g),
\end{equation}
где предполагается, что $\E[uu'|X]=\Omega(\gamma_0)$ и $\hat{\gamma}$ является состоятельной оценкой для $\gamma_0$.

Если наши предположения, сделанные для оценки НМНК, удовлетворены и фактически $\Omega_0=\Omega(\gamma_0)$, то оценка ДОНМНК является состоятельной и асимптотически нормальной с оценочной асимптотической ковариационной матрицей, приведённой в таблице 5.6. Оценки ковариационной матрицы аналогичны оценкам линейного ДОМНК, $[X'\Omega(\hat{\gamma})^{-1}X]^{-1}$, только $X$ заменяется на $\hat{D}=\partial g / \partial \beta'|_{\hat{\beta}}$. 

Оценка ДОНМНК является наиболее эффективной состоятельной оценкой, которая минимизирует
квадратичную функцию потерь вида $(y-g)'V(y-g)$, где $V$ --- матрица весов. 

В целом реализация ДОНМНК требует обращения матрицы $\Omega(\hat{\gamma})$ размера $N \times N$. Это может быть вычислительно невозможно для больших $N$, но на практике $\Omega(\hat{\gamma})$ обычно имеет структуру такую, как диагональность, что приводит к возможности найти обратную матрицу в явном виде.

\begin{center}
Взвешенный НМНК
\end{center}

Подход ДОНМНК полностью эффективен, но приводит к неправильным стандартным оценкам ошибок, если модель для $\Omega_0$ будет неправильно специфицирована. Здесь мы рассмотрим подход для НМНК и ДОНМНК, который определяет модель для ковариационной матрицы ошибок и затем получает скорректированные стандартные ошибки. Обсуждение похоже на то, которое представлено в Разделе 4.5.2.

Оценка взвешенного нелинейного метода наименьших квадратов (ВНМНК) $\hat{\beta}_{WGNLS}$ максимизирует
\begin{equation}
\mathcal{Q}_{N}(\beta)=-\frac{1}{2N}(y-g)' \hat{\Sigma}^{-1} (y-g),
\end{equation}
где $\Sigma = \Sigma (\gamma)$  является рабочей ковариационной матрицей ошибок $\hat{\Sigma}=\Sigma(\hat{\gamma})$, где $\hat{\gamma}$ является оценкой $\gamma$, и в отличие от ДОНМНК $\Sigma \not = \Omega_0$.

При предположениях, аналогичных тем, что были введены для оценок НМНК, и предполагая, что $\Sigma_0=\plim \hat{\Sigma}$, оценки ВНМНК являются состоятельными и асимптотически нормальными с оценочными асимптотическими ковариационными матрицами, приведёнными в таблице 5.6.

Эта оценка называется ВНМНК, чтобы отличить её от оценки ДОНМНК, которая предполагает, что $\Sigma=\Omega_0$. Оценки ВНМНК находятся между НМНК и ДОНМНК с точки зрения эффективности, хотя они могут быть менее эффективными, чем оценки НМНК, если  выбрана неудачная модель ковариационной матрицы ошибки. НМНК и МНК оценки являются частными случаями  ВНМНК с $\Sigma=\sigma^{2}I$.

\begin{center}
Гетероскедастичные ошибки
\end{center}

Очевидной рабочей моделью для гетероскедастичности является ${\sigma_i}^{2}=\E[{u_i}^2|x_i]=\exp(z'_i\gamma_0)$, где вектор $z$ является заданной функцией от $x$ (например, выбранные подкомпоненты $x$), и использование экспоненты обеспечивает положительную дисперсию.

Тогда $\Sigma=\Diag[\exp(z'_i\gamma)]$ и $\hat{\Sigma}=\Diag[\exp(z'_i\hat{\gamma})]$ , где $\hat{\gamma}$ может быть получено из нелинейной регрессии квадратов остатков НМНК $(y_i-g(x_i,\hat{\beta}_{NLS}))^2$ на $\exp(z'_i\gamma)$. Поскольку $\Sigma$ диагональная, $\Sigma^{-1}=\Diag[1/{\sigma_i}^{2}]$. Тогда (5.84) упрощается и оценки ВНМНК максимизирует
\begin{equation}
\mathcal{Q}_{N}(\beta)=-\frac{1}{2N} \sum_{i=1}^{N} \frac{(y_i-g(x_i,\beta))^2}{\hat{\sigma_i}^{2}}.
\end{equation}

Ковариационные матрицы оценок ВНМНК, приведённые в таблице 5.6, дают следующее:
\begin{equation}
\widehat{\Var}[\hat{\beta}_{WNLS}]= \left( \sum_{i=1}^{N} \frac{1}{\hat{\sigma_i}^{2}} \hat{d}_i \hat{d'}_i \right) ^{-1} \left( \sum_{i=1}^{N} {\hat{u}_i}^2 \frac{1}{\hat{\sigma_i}^{4}} \hat{d}_i \hat{d'}_i \right) \left( \sum_{i=1}^{N} \frac{1}{\hat{\sigma_i}^{2}} \hat{d}_i \hat{d'}_i \right) ^{-1},
\end{equation}
где $\hat{d}_i=\partial g(x_i,\beta) / \partial \beta|_{\hat{\beta}}$ и $\hat{u}_i=y_i-g(x_i,\hat{\beta}_{WNLS})$ --- остатки. На практике поправка на степени свободы может быть использована, чтобы правая часть (5.87) умножалась на $N/(N-K)$. Если сделано сильное предположение, что $\Sigma=\Omega_0$, то ВНМНК становится ДОНМНК и 
\begin{equation}
\widehat{\Var}[\hat{\beta}_{FGNLS}]=\left( \sum_{i=1}^{N} \frac{1}{\hat{\sigma_i}^{2}} \hat{d}_i \hat{d'}_i \right) ^{-1}.
\end{equation}

Оценки ВНМНК и ДОНМНК могут быть реализованы с использованием НМНК программ. Во-первых, сделаем НМНК регрессию $y_i$ на $g(x_i,\beta)$. Во-вторых, получим $\hat{\gamma}$, например, с помощью НМНК регрессии $(y_i-g(x_i,\hat{\beta}_{NLS}))^{2}$ на $\exp(z'_i\gamma)$, если ${\sigma_i}^{2}=\exp(z'_i\gamma)$. В-третьих, выполним НМНК регрессию $y_i/\hat{\sigma}_i$ на $g(x_i,\beta)/\hat{\sigma}_i$, где $\hat{\sigma_i}^{2}=\exp(z'_i\hat{\gamma})$. Это эквивалентно максимизации (5.86). Скорректированные сэндвич стандартные ошибки Уайта из этой преобразованной регрессии дают скорректированные стандартные ошибки ВНМНК, основанные на (5.87). Обычные нескорректированные стандартные ошибки из этой преобразованной регрессии дают стандартные ошибки ДОНМНК, основанные на (5.88).

С гетероскедастичными ошибками очень заманчиво сделать ещё один шаг вперёд и попробовать сделать ДОНМНК с использования $\hat{\Omega}=\Diag[{\hat{u}_i}^2]$. Однако это даст несостоятельную оценку параметра $\beta_0$, поскольку ДОНМНК регрессия $y_i$ на $g(x_i,\beta)$ сводится к регрессии НМНК $y_i/|\hat{u}_i|$ на $g(x_i,\beta)/|\hat{u}_i|$. Метод страдает от фундаментальной проблемы корреляции регрессоров и ошибок. Альтернативные полупараметрическое методы, которые дают такие же эффективные оценки, как оценки МНК, без уточнения функциональной формы $\Omega_0$, представлены в разделе 9.7.6.

\begin{center}
Обобщённые линейные модели
\end{center}

Реализация взвешенного НМНК требует разумной спецификации для рабочей матрицы. Подход ad-hoc, который уже был представлен, заключается в том, чтобы ${\sigma_i}^{2}=\exp(z'_i\gamma)$, где $z$ часто является подмножеством $x$. Например, в регрессии заработка на образование и на другие контрольные переменные мы можем моделировать гетероскедастичность в виде функции нескольких регрессоров, прежде всего школьного образования.

Некоторые типы пространственных данных приводят к очень экономной естественной модели гетероскедастичности. Например, для счетных данных при использовании распределения Пуассона дисперсия равна математическому ожиданию, то есть ${\sigma_i}^{2}=g(x_i,\beta)$. Это даёт рабочую модель гетероскедастичности, которая не вносит никаких дополнительных параметров, чем те, которые уже используются в моделировании условного математического ожидания.

Этот подход, подразумевающий, что рабочая модель для дисперсии --- функция математического ожидания, естественно возникает для обобщённой линейной модели, введённой в Разделах 5.7.3 и 5.7.4. Из (5.63) условия первого порядка для оценки квази-ММП на основе распределения экспоненциального семейства имеют следующий вид:
\[
\sum_{i=1}^{N} \frac{y_i-g(x_i,\beta)}{{\sigma_i}^{2}} \times \frac{\partial g(x_i,\beta)}{\partial \beta}=0,
\]
где ${\sigma_i}^2=[c'(g(x_i,\beta))]^{-1}$ --- предполагаемая функция дисперсии, соответствующая конкретной ОЛМ (см. (5.60)). Например, для Пуассона, Бернулли и экспоненциального распределений ${\sigma_i}^2$ равны, соответственно, $g_i$, $g_i(1-g_i)$ и $1/(g_i)^2$, где $g_i=g(x_i,\beta)$.

Эти условия первого порядка могут быть решены для $\beta$ в один шаг, который допускает зависимость ${\sigma_i}^2$ от $\beta$. В более простом двухшаговом методе вычисляют $ \hat{\sigma_i}^2=c'(g(x_i,\hat{\beta}))$ при исходной оценке НМНК $\beta$, а затем выполняют взвешенную регрессию НМНК $y_i/\hat{\sigma}_i$ на $g(x_i,\beta)/\hat{\sigma}_i$. Полученная оценка $\beta$ асимптотически эквивалентна оценке квази-ММП, которая является непосредственно решением (5.63) (см. Гурьеру, Монфор, и Трогнан 1984a, или Кэмерон и Триведи, 1986). Таким образом ДОНМНК асимптотически эквивалентна оценке ММП, когда распределение принадлежит экспоненциальному семейству. Чтобы предотвратить неправильную спецификацию ${\sigma_i}^2$, следует использовать скорректированные сэндвич стандартные ошибки, или взять $\hat{\sigma_i}^2=\hat{\alpha}[c'(g(x_i,\beta))]^{-1}$, где оценка $\hat{\alpha}$ приведена в Разделе 5.7.4.

\subsection{Временные ряды}

Общий результат НМНК в предложении 5.6 распространяется на все типы данных, в том числе на  временные ряды. Дальнейшие результаты по оценкам ковариационных матриц сосредоточены на пространственном случае гетероскедастичных ошибок, но они легко адаптируются на случай временных рядов с автокорреляцией ошибок. Действительно, результаты о робастном оценивании ковариационной матрицы с использованием спектральных  методов для случая временных рядов предшествовали тем, которые использовались для пространственного случая.

Нелинейная регрессионная модель временных рядов:
\[
y_t=g(x_t,\beta)+u_t, t=1, \dots, T.
\]
Если ошибки $u_t$ автокоррелированны, часто используют авторегрессию скользящего среднего или модель $ARMA(p,q)$:
\[
u_t=\rho_{1} u_{t-1}+\cdots+\rho_{p} u_{t-p} + \e_t + \alpha_{1} \e_{t-1}+\cdots+\alpha_{q} \e_{t-q},
\]
где $\e_t$ является одинаково и независимо распределённым с математическим ожиданием 0 и дисперсией $\sigma^2$, и могут быть введены ограничения на параметры $ARMA$ модели для обеспечения стационарности и обратимости. Модель ошибок $ARMA$ означает конкретную структуру ковариационной матрицы ошибок $\Omega_0=\Omega(\rho,\alpha)$.

$ARMA$ модель представляет собой хорошую модель для $\Omega_0$ для случая временных рядов. В отличие от временных рядов, в пространственном случае, гораздо труднее правильно смоделировать гетероскедастичность, поэтому в нём большее внимание уделяется робастным методам, которые не требуют спецификации модели для $\Omega_0$.

Что делать, если ошибки и гетероскедастичны, и коррелированы? Оценка НМНК состоятельна  хотя и неэффективна, если ошибки коррелированы при условии, что $x_t$ не включает лаговые зависимые переменные, ведь в этом случае она становится несостоятельной. Уайт и Домовитс (1984) обобщили (5.79), чтобы получить скорректированную оценку ковариационной матрицы оценки НМНК при гетероскедастичности и автокорреляции в неизвестной функциональной форме, предполагая, что корреляция отлична от нуля не более, чем, скажем, на $l$ лагов. На практике используются незначительные уточнения Ньюи и Веста (1987b). Это уточнение --- изменение масштаба, что гарантирует, что оценка ковариационной матрицы неотрицательно определена. Были предложены и несколько других уточнений, и предположение о фиксированной длине лага было смягчено. Например, допускается, что  $l \rightarrow \infty$ со скоростью существенно меньше, чем $N \rightarrow \infty$. Это условие допускает возможность $AR$ составляющей в ошибках.

\section{Пример: оценивание с помощью ММП и НМНК}

Оценки максимального правдоподобия и МНМК оценки, расчёт стандартных ошибок и интерпретация коэффициентов проиллюстрированы с помощью искусственных данных.

\subsection{Модель и оценки}
 
Экспоненциальное распределение используется для непрерывных положительных данных, особенно для данных по длительности, изучаемых в Главе 17. Экспоненциальная плотность задается функцией
\[
f(y)=\lambda e^{-\lambda y}, y>0, \lambda>0,
\]
с математическим ожиданием $1/\lambda$ и дисперсией $1/\lambda^2$. Введём регрессоры в эту модель, положив
\[
\lambda=\exp(x'\beta),
\]
при этом $\lambda>0$. Следует отметить, что
\[
\E[y|x]=\exp(-x'\beta)
\]
Вместо этого возможна альтернативная параметризация, $\E[y|x]=\exp(x'\beta)$, так что $\lambda=\exp(-x'\beta)$. Обратите внимание, что экспонента используется двумя различными способами: для плотности и для условного математического ожидания.

МНК-оценка регрессии $y$ на $x$ является несостоятельной, поскольку она соответствует прямой, когда функция регрессии на самом деле является экспоненциальной кривой.

Оценка ММП легко получается. Логистическая плотность --- $\ln f(y|x)=x'\beta-y\exp(x'\beta)$, что ведёт к следующим условиям первого порядка $N^{-1}\sum_i (1-y_{i}\exp(x'_i\beta))x_i=0$ или
\[
N^{-1}\sum_i \frac {y_{i}-\exp(-x'\beta)}{\exp(-x'\beta)}x_i=0.
\]

Чтобы оценить регрессию НМНК, отметим, что модель может также быть выражена как нелинейная регрессия с
\[
y=\exp(-x'\beta)+u,
\]
где ошибки $u$ имеют $\E[u|x]=0$, хотя они гетероскедастичны. Условия первого порядка для экспоненциального условного математического ожидания для этой модели, уже были приведены в (5.82) и приводят явно к оценке, отличной от оценки ММП, помимо изменения знака.

В качестве примера взвешенного НМНК предположим, что дисперсия ошибки пропорциональна  математическому ожиданию. Тогда рабочая дисперсия $\Var[y]=\E[y]$ и взвешенный метод наименьших квадратов может быть реализован с помощью оценки НМНК регрессии $y_{i}/\hat{\sigma}_i$ на $\exp(-x'_{i}\beta)/\hat{\sigma}_i$, где ${\hat{\sigma}_i}^2=\exp(-x'_{i}\hat{\beta}_{NLS})$. Эта оценка является менее эффективной, чем оценка ММП и может  быть более или менее эффективной, чем оценка НМНК.

Допустимый обобщённый НМНК может быть применен в данном случае, так как мы знаем процесс порождающий данные. Так как $\Var[y]=1/\lambda^2$ для экспоненциальной плотности, дисперсия равна квадрату математического ожидания, из этого следует, что $\Var[u|x]=[\exp(-x'\beta)]^2$. Оценка ДОНМНК оценивает ${\sigma_i}^2$ с помощью ${\hat{\sigma}_i}^2=[\exp(-x'_{i}\hat{\beta}_{NLS})]^2$, и может быть реализована с помощью НМНК регрессии $y_{i}/\hat{\sigma}_i$ на $\exp(-x'_{i}\beta)/\hat{\sigma}_i$. В общем оценки ДОНМНК менее эффективны, чем оценки ММП. В данном примере оценка на самом деле эффективна, так как экспоненциальная плотность лежит в экспоненциальном семействе (см. обсуждение в конце Раздела 5.8.6).

\subsection{Симуляции и результаты}

Для простоты рассмотрим регрессию на константу и регрессор. Данные были сгенерированы с помощью следующего процесса:
\[
y|x \sim exponential[\lambda],
\]
\[
\lambda=\exp(\beta_1 + \beta_2 x),
\]
где $x \sim \mathcal{N}[1,1^{2}]$ и $(\beta_1,\beta_2)=(2,-1)$. Была взята большая выборка размера 10 000, чтобы минимизировать различия в оценках, в частности, стандартных ошибках, возникающие в связи с изменчивостью выборки. Для конкретной выборки из 10 000, сделанной здесь, выборочное среднее $y$ составляет $0.62$ и выборочное стандартное отклонение составляет $1.29$.

\begin{table}[h]
\begin{center}
\caption{\label{tab:Expex} Экспоненциальный пример: оценки метода наименьших квадратов и оценки ММП}
\begin{minipage}{14cm}
\begin{tabular}[t]{cccccc}
\hline
\hline
& & & \bf{Оценка}\footnote{Все оценки соcтоятельны, кроме МНК. Приведены до трёх альтернативных стандартных оценок: нескорректированные в скобках, скорректированные с помощью внешнего произведения в квадратных скобках и альтернативные скорректированные оценки для НМНК в фигурных скобках. Процесс порождающий данные --- экспоненциальное распределение с константой $2$ и параметром наклона $-1$. Объём выборки N = 10 000.} & & \\
\hline
\bf{Переменная} & \bf{МНК} & \bf{ММП} & \bf{НМНК} & \bf{ВНМНК} & \bf{ДОНМНК} \\ 
\hline
Константа & $-0.0093$ & $1.9829$ & $1.8876$ & $1.9906$ & $1.9840$ \\
& $(0.0161)$ & $(0.0141)$ & $(0.0307)$ & $(0.0225)$ & $(0.0148)$ \\
& $[0.0172]$ & $[0.0144]$ & $[0.1421]$ & $[0.0359]$ & $[0.0146]$ \\
& & & $\{0.2110\}$ & & \\
$x$ & $0.6198$ & $-0.9896$ & $-0.9575$ & $-0.9961$ & $-0.9907$ \\
& $(0.0113)$ & $(0.0099)$ & $(0.0097)$  & $(0.0098)$ &  $(0.0100)$ \\
& $[0.0254]$ &  $[0.0099]$ & $[0.0612]$ & $[0.0224]$ & $[0.0101]$ \\
& & & $\{0.0880\}$ & & \\
$\ln L$ & --- & $-208.71$ & $-232.98$ & $-208.93$ & $-208.72$ \\
$R^2$ & $0.2326$ & $0.3906$ & $0.3913$ & $0.3902$ & $0.3906$ \\
\hline
\hline
\end{tabular}
\end{minipage}
\end{center}
\end{table}

Таблица 5.7 представляет оценки МНК, ММП, НМНК, ВНМНК и ДОНМНК. Также приведены до трёх разных стандартных ошибок. По умолчанию результаты регрессии дают нескорректированные стандартные ошибки, приведенные в скобках. Для оценок МНК и НМНК ошибки предполагаются одинаково и независимо распределёнными, что является ошибочным предположением здесь.  Для оценки ММП одинаковая распределенность ошибок приводит к равенству информационных матриц, выполненному в данном случае, так как процесс порождающий данные правильно специфицирован. Скорректированные стандартные ошибки, которые даются в квадратных скобках, используют устойчивую оценку дисперсии $N^{-1}{\hat{A}_H}^{-1}\hat{B}_{OP}{\hat{A}_H}^{-1}$, где $\hat{B}_{OP}$ является внешним произведением, оценённом при (5.38). Эти оценки устойчивы к гетероскедастичности. Для стандартных ошибок НМНК оценок приведена в фигурных скобках более корректная альтернативных оценка (объясняется в следующем Разделе). Оценки стандартных ошибок, представленные здесь, используют численные, а не аналитические производные при подсчёте $\hat{A}$ и $\hat{B}$.

\subsection{Сравнение оценок и стандартные ошибки}

Оценка МНК не состоятельна, и никак не связана с $(\beta_1,\beta_2)$  экспоненциального процесса, порождающего данные.

Остальные оценки являются состоятельными, и оценки ММП, НМНК, ВНМНК и ДОНМНК находятся в пределах двух стандартных отклонений от истинных значений параметров $(2,-1)$, где скорректированные стандартные ошибки должны быть использованы для НМНК. Оценки ДОНМНК весьма близки к оценкам ММП, что является следствием использования распределения из экспоненциального семейства в процессе, порождающем данные.

Для оценки ММП нескорректированные и скорректированные стандартные ошибки ММП очень похожи. Ожидается, что они асимптотически эквивалентны (поскольку имеет место равенство информационных матриц, если оценки ММП основаны на истинной плотности) и размер выборки здесь большой.

Для НМНК нескорректированные стандартные ошибки являются недействительными, так как у процесса порождающего данные гетероскедастичные ошибки, и значительно преувеличивают точность оценок НМНК. Формула для скорректированной оценки ковариационнной матрицы НМНК дана в (5.81), где $\hat{\Omega}=\Diag[{\hat{u}_i}^2]$. Альтернативная оценка, которая использует $\hat{\Omega}=\Diag[\hat{\E}[{u}_i^2]]$, где $\hat{\E}[{u}_i^2]=[\exp(-x'_i\hat{\beta})]^2$ приведена в скобках. 
Две оценки отличаются: $0.0612$ по сравнению с $0.0880$ для коэффициента наклона. Разница связана с тем, что ${\hat{u}_i}^2=(y_i-\exp(x'_i \hat{\beta}))^2$ отличается от $[\exp(-x'_i \hat{\beta})]^2$. В более общем случае стандартная ошибка, оценённая с использованием внешнего произведения (см. Раздел 5.5.2), может быть смещена даже в достаточно больших выборках. Оценки НМНК значительно менее эффективны, чем оценки ММП, и их стандартные ошибки (рекомендуемые --- в фигурных скобках) превышают во много раз стандартные ошибки оценок ММП.

Оценки ВНМНК не использует правильную модель гетероскедастичности, поэтому нескорректированные и скорректированные стандартные ошибки снова отличаются. При использовании робастных стандартных ошибок  ВНМНК даёт более эффективные оценки, чем оценки НМНК и менее эффективны, чем оценки ММП.

В этом примере оценки ДОНМНК столь же эффективны, как и оценки ММП, что является следствием использования распределения из экспоненциального семейства в процессе, порождающем данные. В результатах видно, что коэффициенты и стандартные ошибки очень близки к соответствующим оценкам ММП. Скорректированные и нескорректированные  стандартные ошибки для оценки ДОНМНК по существу такие же, как и ожидалось, поскольку здесь модель гетероскедастичности правильно специфицирована.

Таблица 5.7 также содержит оценки логарифма функции правдоподобия, $\ln L=\sum_i [x'_i\hat{\beta}-\exp(-x'_i\hat{\beta})y_i]$, и $R^2$, $R^2=1-\sum_i (y_i-\hat{y_i})^2/\sum_i (y_i-\bar{y})^2$, где $\hat{y_i}=\exp(-x'_i\hat{\beta})$ оценивается с помощью ММП, НМНК, ВНМНК и ДОНМНК. $R^2$ мало отличается в различных моделях и является самым низким для оценки МНМК, как и ожидалось, так как НМНК минимизирует $\sum_i (y_i-\hat{y_i})^2$. Логарифмическая функция правдоподобия достигает максимума в оценке ММП, как и ожидалось, и значительно выше, чем для оценки НМНК.

\subsection{Интерпретация коэффициентов}

Интерес заключается в подсчете изменения $\E[y|x]$ при изменении $x$. Рассмотрим оценки ММП $\hat{\beta}_2=-0.99$ приведенные в таблице 5.7.

Условное математическое ожидание $\exp(-\beta_1-\beta_2x)$ имеет одноиндексную форму, поэтому, если дополнительный регрессор $z$ с коэффициентом $\beta_3$ будет включён, то предельный эффект изменения на одну единицу $z$ будет в $\hat{\beta}_3/\hat{\beta}_2$ раза больше, чем изменение на одну единицу $x$ (см. п. 5.2.4).

Условное математическое ожидание монотонно убывает по $x$, поэтому знак $\hat{\beta}_2$ противоположный знаку предельного эффекта (см. раздел 5.2.4). Здесь предельный эффект увеличения $x$ является увеличением условного математического ожидания, поскольку $\hat{\beta}_2$ отрицательная.

Рассмотрим теперь величину предельного эффекта изменений в $x$ с использованием численных методов. Здесь $\partial \E[y|x]/\partial x=-\beta_2 \exp(-x'\beta)$ варьируется в зависимости от точки $x$ и принимает значения от $0.01$ до $19.09$ в выборке. Выборочное среднее предельного эффекта равно $0.99 N^{-1} \sum_i \exp(x'_i\hat{\beta})=0.61$. Предельный эффект для среднего  $x$ равен $0.99 \exp(\bar{x'}\hat{\beta})=0.37$, значительно меньше. Поскольку $\partial \E[y|x]/\partial x=-\beta_2 \E[y|x]$, получаем ещё одну оценку предельного эффекта, $0.99\bar{y}=0.61$.

Взятие разности вместо производной приводит к другим оценкам предельного эффекта. Для $\Delta x=1$ мы получаем $\Delta \E[y|x]=(e^{\beta_2}-1)\exp(-x'\beta)$ (см. Раздел 5.2.4) Получаем средний эффект для выборки $1.04$, а не $0.61$. Взятие разности и производной дают одинаковый предельный эффект, если $x$ мало.

Предыдущие предельные эффекты можно складывать. Для экспоненциального условного математического ожидания можно рассмотреть и мультипликативный или пропорциональной предельный эффект (см. п.5.2.4). Например, изменение $x$ на $0.1$, по прогнозам, приведёт к пропорциональному росту $\E[y|x]$ на $0.1 \times 0.99$ или к росту $9.9\%$. Опять взятие разности даст другую оценку.

Какая из этих мер наиболее полезна? Ограничение на одноиндексную форму очень полезно, поскольку относительное влияние регрессоров можно вычислить сразу. Что касается величины ответной реакции, наиболее точным является вычисление средней реакции по выборке, используя подсчет разности при изменении регрессора на $c$ единиц, где величина $c$ --- это осмысленное число, например,  одно стандартное отклонение $x$.

Аналогичные расчёты можно сделать для  оценок НМНК, ВНМНК и ДОНМНК с аналогичными результатами. Для МНК-оценки обратим внимание, что коэффициент при $x$ может быть интерпретирован как выборочный среднее предельного эффекта изменения в $x$ (см. Раздел 4.7.2). Здесь оценка МНК $\hat{\beta}_2=0.61$ равна до двух десятичных знаков среднему предельному эффекту по выборке, вычисленному ранее с использованием экспоненциальной оценки ММП. Здесь МНК обеспечивает хорошую оценку выборочного среднего предельной реакции, хотя он может обеспечить очень плохую оценку предельной реакции для любого конкретного значения $x$.

\subsection{Практические соображения}

Большинство эконометрических пакетов предоставляют простые команды для получения оценок максимального правдоподобия для стандартных моделей, введённых в Разделе 5.6.1. Для других плотностей многие пакеты обеспечивают ММП оценивание, для которого пользователь записывает уравнение плотности и, возможно, первые производные или даже вторые производные. Аналогично, для НМНК записывается уравнение условного математического ожидания. Для некоторых нелинейных моделей и наборов данных ММП и НМНК оценивание, реализованное в пакетах, может  быть связано c вычислительными трудностями при получении оценок. В таких случаях может быть необходимо использовать более надёжные оптимизационные процедуры, которые предоставляются в качестве дополнительных модулей к Gauss, Matlab и OX. Gauss, Matlab и OX хорошо подходят для нелинейного моделирования, но требуют более высокой подготовки.

Для пространственных данных становится распространенным использование стандартных ошибок, основанных на сэндвич-форме ковариационной матрицы. Возможность использовать сэндвич-форму часто предоставляется в виде опции при оценивании. Для оценок МНК это даёт устойчивые к гетероскедастичности стандартные ошибки. Для максимального правдоподобия необходимо знать, что неправильная спецификация плотности может привести к несостоятельности помимо  необходимости использовать сэндвич ошибки.

Параметры нелинейных моделей, как правило, не могут быть напрямую интерпретируемы, и хорошо дополнительно вычислить подразумеваемые предельные эффекты, вызванные изменениями в регрессорах (см. Раздел 5.2.4). Некоторые пакеты делают это автоматически, для других может быть необходимо несколько строчек кода после оценки, используя сохранённые коэффициенты регрессии.

\subsection{Библиографические примечания}

Краткая история развития теории асимптотических результатов экстремальных оценок приведена в Ньюи и МакФаддене (1994, с. 2115). Значительный прорыв вперёд в эконометрике был сделан Амэмия (1973), который разработал довольно общие теоремы, которые были применены к тобит модели оценки ММП. Полезные книги --- это книги Галанта (1987), Галанта и Уайта (1987), Байеренса (1993), Уайта (1994, 2001а). Основы по теории статистики приведены во многих книгах, в том числе в книгах Амэмия (1985, глава 3), Дэвидсона и МакКиннона (1993, глава 4), Грина (2003, Приложение D), Дэвидсона (1994), и Замана (1996).

\begin{enumerate}
\item [$5.3$] Презентация общих результатов для экстремальной оценки во многом основывается на книге Амэмия (1985, глава 4), и в меньшей степени на книге Ньюи и МакФаддена (1994). Последний источник имеет очень широкий охват материала.
\item [$5.4$] Подход оценочных уравнений используется в литературе об обобщённых линейных моделях (см. Маккалоу и Нельдер, 1989). Эконометристы обобщили его до обобщённого метода моментов (см. Главу 6).
\item [$5.5$] Статистические выводы подробно изложены в главе 7.
\item [$5.6$] См. новаторскую статью Фишера (1922) для общих результатов по оценке ММП, в том числе для результатов по эффективности, а также для сравнения подхода правдоподобия с обратными вероятностями или Байесовским подходом и с методом моментов.
\item [$5.7$] Современные приложения часто используют квази-ММП и сэндвич-оценки ковариационной матрицы (см. Уайт, 1982, 1994). В статистике этот подход называется обобщёнными линейными моделями,  см. Маккалоу и Нельдер (1989).
\item [$5.8$] Аналогично и в случае НМНК используются сэндвич-оценки ковариационной матрицы, при этом накладываются относительно слабые ограничения на ошибки. Работа Уайта (1980a, c) оказала большое влияние на статистические выводы в эконометрике. Обобщение и подробный обзор асимптотической теории представлены у Уайта и Домовитца (1984). У Амэмия (1983) содержится широко используемы методы нелинейных регрессий.
\end{enumerate}

\begin{center}
Упражнения
\end{center}

\begin{enumerate}
\item [$5-1$] Предположим, мы получили оценки модели, которые дали следующее условное математическое ожидание: $\hat{\E}[y|x]=\exp(1+0.01x)/[1+\exp(1+0.01x)]$. Предположим, что выборка имеет размер 100 и $x$ принимает целые значения $1,2,\dots,100$. Найдите следующие оценки оцениваемого предельного эффекта $\partial \hat{\E}[y|x]/\partial x$.
\begin{enumerate}
\item Средний предельный эффект по всем наблюдениям.
\item Предельный эффект для среднего наблюдения.
\item Предельный эффект при $x=90$.
\item Предельный эффект  изменения на одну единицу, когда $x=90$, вычислив его посчитав разницу.
\end{enumerate}

\item [$5-2$] Рассмотрим следующий частный однопараметрический случай гамма-распределения, 

$f(y/\lambda^2)\exp(-y/\lambda),y>0,\lambda>0$. Для этого распределения можно показать, что $\E[y]=2\lambda$ и $\Var[y]=2\lambda^2$. Здесь мы вводим регрессоры и предположим, что в истинной модели параметр $\lambda$ зависит от регрессоров в соответствии с $\lambda_i=\exp(x'_i\beta)/2$. Таким образом, $\E[y_i|x_i]=\exp(x'_i\beta)$ и $\Var[y_i|x_i]=[\exp(x'_i\beta)]^{2}/2$. Предположим, что данные являются независимыми по $i$ и $x_i$ являются нестохастическими и $\beta=\beta_0$ в процессе порождающем данные.
\begin{enumerate}
\item Покажите, что логарифмическая функция правдоподобия (домножаемая на $N^{-1}$) для этой гамма модели равна $\mathcal{Q}_{N}(\beta)=N^{-1} \sum_i\{\ln y_i -2 x_i'\beta + 2\ln2 -2 y_i \exp(-x'_i \beta) \}$.
\item Найдите $\plim \mathcal{Q}_{N}(\beta)$. Можно предполагать, что условия любого ЗБЧ удовлетворены. [Подсказка: $\E[\ln y_i]$ зависит от $\beta_0$, а не от $\beta$.]
\item Докажите, что оценка $\hat{\beta}$, являющаяся локальным максимумом $\mathcal{Q}_{N}(\beta)$, состоятельна для $\beta_0$. Запишите сделанные предположения.
\item Укажите, какой ЗБЧ нужно использовать для проверки пункта (b) и какая необходима дополнительная информация, если таковая необходима, чтобы применить этот закон. Подойдёт краткий ответ. Нет необходимости формального доказательства.
\end{enumerate}

\item [$5-3$] Продолжим работу с гамма моделью из упражнения 5-2.
\begin{enumerate}
\item Покажите, что $\partial \mathcal{Q}_{N}(\beta) / \partial \beta= N^{-1} \sum_i 2[(y_i-\exp(x'_i\beta))/\exp(x'_i\beta)]x_i$.
\item Какое из условий первого порядка, является существенным, чтобы $\hat{\beta}$ была состоятельной? 
\item Примените центральную предельную теорему для получения предельного распределения $\sqrt{N}\partial \mathcal{Q}_{N}/\partial \beta|_{\beta_0}$. Здесь вы можете предположить, что условия ЦПТ, удовлетворены.
\item Напишите, какую ЦПТ используете, чтобы использовать для проверки пункта $c$ и какая необходима дополнительная информация, если таковая необходима, чтобы применить этот закон. Подойдёт краткий ответ. Нет необходимости формального доказательства.
\item Найдите предел по вероятности от $\partial^2 \mathcal{Q}_{N}/\partial \beta \partial \beta'|_{\beta_0}$.
\item Объедините предыдущие результаты, чтобы получить предельное распределение $\sqrt{N}(\hat{\beta}-\beta_0)$.
\item С учётом пункта (f), укажите, как проверить $H_0: \beta_{0j} \geq {\beta_j}^*$ против $H_a: \beta_{0j} < {\beta_j}^*$ на уровне значимости 0.05, где $\beta_j$ --- $j$-ая компонента $\beta$.
\end{enumerate}

\item [$5-4$] Неотрицательная целая переменная $y$, которая распределена по геометрическому закону, описывается функцией вероятности $f(y)=(y+1)(2\lambda)^{y}(1+2\lambda)^{-(y+0.5)},y=0,1,2,\dots,\lambda>0$. Тогда $\E[y]=\lambda$ и $\Var[y]=\lambda(1+2\lambda)$. Введите регрессоры, и пусть $\gamma_i=\exp(x'_i\beta)$. Предположите, что данные независимы по $i$, $x_i$ является нестохастическим и $\beta=\beta_0$ в процессе порождающем данные.
\begin{enumerate}
\item Повторите упражнение 5-2 для этой модели. 
\item Повторите упражнение 5-3 для этой модели.
\end{enumerate}

\item [$5-5$] Предположим, выборка даёт оценки $\hat{\theta}_1=5,\hat{\theta}_2=3,se[\hat{\theta}_1]=2$, а $se[\hat{\theta}_2]=1$, а коэффициент корреляции между $\hat{\theta}_1$ и $\hat{\theta}_2$ равен 0.5. Проведите следующие тесты на уровне значимости 0.05, предполагая асимптотическую нормальность оценок параметров. 
\begin{enumerate}
\item Протестируйте $H_0: \theta_1=0$ против $H_a: \theta_1 \not= 0$.
\item Протестируйте $H_0: \theta_1=2\theta_2$ против $H_a: \theta_1 \not =2\theta_2$.
\item Протестируйте $H_0: \theta_1=0, \theta_2=0$ против $H_a:$ по меньшей мере один из $\theta_1,\theta_2 \not= 0$.
\end{enumerate}

\item [$5-6$] Рассмотрите нелинейную модель регрессии $y=\exp(x'\beta)/[1+\exp(x'\beta)]+u$, где ошибки, возможно, гетероскедастичны.
\begin{enumerate}
\item При таком ограничении в каком диапазоне должно лежать $\E[y|x]$?
\item Запишите условия первого порядка для оценки НМНК.
\item Получите асимптотическое распределение НМНК оценки, используя результат (5.77).
\end{enumerate}

\item [$5-7$] Этот вопрос предполагает доступ к программному обеспечению, которое позволяет получить оценки НМНК и ММП. Рассмотрим гамма регрессионную модель из упражнения 5-2. Соответствующую переменную гамма можно сгенерировать с помощью $y=-\lambda \ln r_1 - \lambda \ln r_2$, где $\lambda=\exp(x'\beta)/2$, $r_1$ и $r_2$ являются случайными выборками из равномерного распределения $[0,1]$. Пусть $x'\beta=\beta_1+\beta_2 x$. Сгенерируйте выборку размером 1 000, где $\beta_1=-1.0$ и $\beta_2=1$ и $x \sim \mathcal{N}[0,1]$.
\begin{enumerate}
\item  Получите оценки $\beta_1$ и $\beta_2$ из регрессии НМНК $y$ на $\exp(\beta_1+\beta_2 x)$. 
\item Нужно ли здесь использовать сэндвич стандартные ошибки?
\item Получите ММП оценки $\beta_1$ и $\beta_2$ из  НМНК регрессии $y$ на $\exp(\beta_1+\beta_2 x)$. 
\item Нужно ли здесь использовать сэндвич стандартные ошибки?
\end{enumerate}
\end{enumerate}


\chapter{Обобщённый метод моментов и системы уравнений}

\section{Вступление}

Предыдущая глава была посвящена М-оценкам, в том числе оценкам ММП и НМНК. Теперь рассмотрим более широкий класс экстремальных оценок, основанных на методе моментов (ММ) и обобщённом методе моментов (ОММ).

Основой ММ и ОММ является спецификация набора теоретических условных моментов,  которая включает в себя данные и неизвестные параметры. Оценки ММ являются решением условий для выборочных моментов, которые соответствуют условиям для теоретических моментов. Например, среднее по выборке --- ММ оценка математического ожидания в генеральной совокупности. В некоторых случаях возможно, что нет явного аналитического решения для ММ оценки, но есть численное. Тогда оценка является примером оценки методом оценочных уравнений, кратко введённой в Разделе 5.4.

В некоторых ситуациях, однако, невозможно найти оценку ММ, поскольку есть больше условий моментов и, следовательно, больше уравнений для решения, чем параметров. Ярким примером является оценка инструментальных переменных в сверх-идентифицированной модели. ОММ оценки, введённые Хансеном (1982), являются расширением ММ подхода для работы с этим случаем.

ОММ определяет класс оценок. Различные оценки ОММ получаются при выборе различных теоретических моментов так же, как различные спецификации плотности или вероятности приводят к различным оценкам ММП. Мы подчёркиваем подход к оцениванию, основанный на моментах, даже в тех случаях, когда возможны альтернативные представления, так как он обеспечивает единство подход и может предоставить очевидный способ расширить методы от линейной к нелинейной модели.

Основы ОММ оценивания определены в Разделах 6.2 и 6.3, в которых представлены, соответственно, разъяснительные примеры и асимптотические результаты для статистических выводов. Остальная часть главы концентрируется на более специализированных оценках. Оценки инструментальных переменных представлены в Разделах 6.4 и 6.5. Для линейных моделей то, что представлено в Разделах 4.8 и 4.9, может быть достаточно, но для нелинейных моделей нужно использовать ОММ подход. Раздел 6.6 охватывает методы для вычисления стандартных ошибок последовательных двуступенчатых М-оценок. Разделы 6.7 и 6.8 представляют оценку минимального расстояния, вариант ОММ, и эмпирические оценки правдоподобия, то есть альтернативные оценки ОММ. Оценивание систем уравнений, используемое в относительно небольшой доле микроэконометрических исследований, обсуждается в Разделах 6.9 и 6.10.

В этой главе рассматриваются многие методы оценивания с точки зрения ОММ. Примеры использования этих методов на реальных данных включают применение линейного метода инструментальных переменных в Разделе 4.9.6 и применение линейного ОММ для панелей в Разделе 22.3.

\section{Примеры}

ОММ оценки основаны на принципе аналогии (см. Раздел 5.4.2). Условия для теоретических моментов приводят к условиям для выборочных моментов, которые могут быть использованы для оценки параметров. В этом разделе приводится несколько основных применений этого принципа, при этом свойства полученной оценки будут рассматриваться только в Разделе 6.3.

\subsection{Линейная регрессия}

Классическим примером метода моментов является оценка теоретического момента, когда
$y$ является одинаково и независимо распределённым с математическим ожиданием $\mu$. Для генеральной совокупности:
\[
\E[y-\mu]=0.
\]
Замена оператора математического ожидания $\E[\cdot]$ для генеральной совокупности на средний оператор $N^{-1} \sum_{i=1}^{N} (\cdot)$ для выборки даёт соответствующий выборочной момент:
\[
\frac{1}{N} \sum_{i=1}^{N} (y_i-\mu)=0.
\]
Решение для $\mu$ приводит к оценке $\hat{\mu}_{MM}=N^{-1} \sum_i y_i=\bar{y}$. Оценка ММ
математического ожидания генеральной совокупности --- выборочное среднее.

Этот подход может быть обобщён на линейную регрессионную модель $y=x'\beta+u$, где
$x$ и $\beta$ --- векторы размера $K \times 1$. Предположим, что ошибки $u$ имеет нулевое математическое ожидание при фиксированных регрессорах. Одно ограничение на условный момент $\E[u|x]=0$ приводит к $K$ уравнениям на безусловное математическое ожидание $\E[xu]=0$, так как
\begin{equation}
\E[xu]=\E_{x}[\E[xu|x]]=\E_{x}[x\E[u|x]]=\E_{x}[x \cdot 0].
\end{equation}
Здесь мы использовали закон повторного  математического ожидания (см. Раздел А.8) и предполагали, что $\E[u|x]=0$. Таким образом,
\[
\E[x(y-x'\beta)]=0,
\]
если условное математическое ожидание ошибки нулевое. Оценки ММП являются решением  соответствующих условий для выборочных моментов:
\[
\frac{1}{N} \sum_{i=1}^{N} x_i(y_i-x'_i\beta)=0.
\]
Мы получаем $\hat{\beta}_{MM}=(\sum_i x_i x'_i)^{-1} \sum_i x_i y_i$.

Поэтому МНК-оценка --- частный случай оценки ММ. Вывод формулы для ММ-оценки, несмотря на её совпадение с МНК оценкой, существенно отличаются от обычной минимизации суммы квадратов остатков.

\subsection{Нелинейная регрессия}

Для нелинейной регрессии подход метода моментов сводится к НМНК, если ошибки регрессии являются аддитивными. Для более общей нелинейной регрессии с неаддитивными ошибками (определённой далее) метод моментов даёт состоятельную оценку в то время, как НМНК даёт несостоятельную оценку.

Из Раздела 5.8.3 нелинейная регрессионная модель с аддитивной ошибкой является моделью, которая определяет $y$ в виде:
\[
y=g(x,\beta)+u.
\]
Подход моментов, аналогичный подходу для линейной модели, в которой мы получили, что $\E[u|x]=0$, здесь приводит к условию $\E[h(x)(y-x'\beta)]=0$, где $h(x)$ --- любая функция от $x$. Конкретный выбор $h(x)=\partial g(x,\beta)/ \partial \beta$, мотивированный в Разделе 6.3.7, приводит к условию на выборочные моменты, совпадающие с условиями первого порядка для оценки НМНК, приведённому в Разделе 5.8.2.

Более общая нелинейная регрессионная модель с неаддитивными ошибками специфицирует
\[
u=r(y,x,\beta),
\]
где снова $\E[u|x]=0$, но теперь $y$ уже необязательно должен быть аддитивной функцией от $u$. Например, в регрессии Пуассона можно определить стандартизированные ошибки как $u=[y-\exp(x'\beta)]/[\exp(x'\beta)]^{1/2}$, который имеет $\E[u|x]=0$ и $\Var[u|x]=1$, так как условное математическое ожидание и дисперсия $y$ равны $\exp(x'\beta)$.

Оценка НМНК несостоятельна при неаддитивных ошибках. Минимизация $N^{-1} \sum_i u^2_i=$

$N^{-1} \sum_i r(y_i,x_i,\beta)^2$ приводит к условиям первого порядка:
\[
\frac{1}{N} \sum_{i=1}^{N} \frac{\partial r(y_i,x_i,\beta)}{\partial \beta} r(y_i,x_i,\beta)=0.
\]

Здесь появляется $y_i$ в обоих сомножителях, и нет никакой гарантии, что математическое ожидание этого произведения равно нулю, даже если $\E[r(\cdot)|x]=0$. Эта несостоятельность не возникала при аддитивных ошибках $r(\cdot)=y-g(x,\beta)$, поскольку тогда $\partial r(\cdot) / \partial \beta=-\partial g(x,\beta) / \partial \beta$, то есть только второй сомножитель зависит от $y$.

Подход метода моментов даёт состоятельную оценку. Из предположения $\E[u|x]=0$ следует, что
\[
\E[h(x)r(y,x,\beta)]=0,
\]
где $h(x)$ --- функция от $x$. Если $\dim[h(x)]=K$, то соответствующий выборочный момент: 
\[
\frac{1}{N} \sum_{i=1}^{N} h(x_i)r(y_i,x_i,\beta)=0
\]
даёт состоятельную оценку $\beta$, где решение получается численными методами.

\subsection{Метод максимального правдоподобия}

Информационный критерий Кульбака - Лейблера  был определён в Разделе 5.7.2. Из этого определения локальный максимум критерия достигается, если $\E[s(\theta)]=0$, где $s(\theta)= \partial \ln f(y|x,\theta) / \partial \theta$ и $f(y|x,\theta)$ --- условная плотность.

Замена теоретических моментов на выборочные моменты даёт оценку $\hat{\theta}$, которая является решением $N^{-1} \sum_i s_i(\theta)=0$. Это условия первого порядка ММП, так что оценка ММП может быть рассмотрена как оценка ММ.

\subsection{Дополнительные ограничения моментов}

Использование дополнительных моментов может повысить эффективность оценки, но требует адаптирования обычного метода моментов, если есть больше условий моментов, чем параметров для оценивания.

Простой пример неэффективной оценки ---  математическое ожидание выборки. Это неэффективная оценка математического ожидания генеральной совокупности, за исключением случая, когда данные являются случайной выборкой нормального распределения или другого распределения экспоненциального семейства. Одним из способов повышения эффективности является использование альтернативных оценок. Медиана выборки, состоятельная для $\mu$, если распределение симметрично, может быть более эффективной. Очевидно, что может быть использована оценка ММП, если распределение задано полностью, но вместо этого здесь повышается эффективность с помощью дополнительных условий моментов.

Рассмотрим оценку $\beta$ в линейной регрессионной модели. МНК-оценка неэффективна, даже при гомоскедастичных ошибках, крому случая нормально распределенных ошибок. Из Раздела 6.2.1, МНК-оценка --- оценка ММ, основанная на $\E[xu]=0$. Теперь введём предположение о дополнительных моментах такое, что ошибки условно симметричны, $\E[u^3|x]=0$ и, следовательно, $\E[x u^3]=0$. Тогда оценка $\beta$ может быть основана на $2K$ условных моментах 
\[
\begin{bmatrix} \E[x(y-x'\beta)] \\ \E[x(y-x'\beta)^3] \end{bmatrix} = \begin{bmatrix} 0 \\ 0 \end{bmatrix}.
\]
ММ подход мог бы оценивать $\beta$ как решение соответствующих уравнений для выборочных моментов $N^{-1} \sum_i x_i(y_i-x'_i\beta)=0$ и $N^{-1} \sum_i x_i(y_i-x'_i\beta)^3=0$. Тем не менее, с $2K$ уравнениями и только $K$ неизвестными параметрами $\beta$, невозможно, чтобы были удовлетворены все эти условия для выборочных моментов.

Подход ОММ вместо этого устанавливает выборочные моменты как можно ближе к нулю с использованием функции квадратичных потерь. Другими словами $\hat{\beta}_{MM}$ минимизирует
\begin{equation}
\mathcal{Q}_{N}(\beta)= \begin{bmatrix}\frac{1}{N} \sum_i x_i u_i \\ \frac{1}{N} \sum_i x_i u^3_i \end{bmatrix}' W_N \begin{bmatrix}\frac{1}{N} \sum_i x_i u_i \\ \frac{1}{N} \sum_i x_i u^3_i \end{bmatrix},
\end{equation}
где $u_i=y_i-x'_i\beta$ и $W_N$ --- матрица весов размера $2K \times 2K$. Для некоторых $W_N$ эта оценка является более эффективной, чем оценка МНК. Этот пример анализируется в Разделе 6.3.6.

\subsection{Регрессия инструментальных переменных}

Оценивание методом инструментальных переменных является одним из главных примеров обобщённого метода моментов.

Рассмотрим линейную регрессионную модель $y=x'\beta+u$ с усложнением, что некоторые компоненты $x$ коррелируют с ошибками, поэтому МНК даёт несостоятельную оценку для $\beta$. Предположим существование инструментов $z$ (введенных в Разделе 4.8), которые коррелируют с $x$, но удовлетворяют условию $\E[u|z]=0$. Тогда $\E[y-x'\beta|z]=0$. Используя алгебраические преобразования, аналогичные тем, которые используются для получения (6.1) для МНК, мы умножаем на $z$, чтобы получить $K$ условий для безусловных теоретических моментов
\begin{equation}
\E[z(y-x'\beta)]=0.
\end{equation}
Оценки метода моментов являются решением соответствующих условий для выборочных моментов:
\[
\frac{1}{N} \sum_{i=1}^{N} z_i(y_i-x'_i\beta)=0.
\]
Если $\dim(z)=K$, то $\hat{\beta}_{MM}= (\sum_i z_i x'_i)^{-1} \sum_i z_i y_i$, что является оценкой линейного метода инструментальных переменных, введённой в Разделе 4.8.6.

Единственного решения не существует, если есть больше потенциальных инструментов, чем регрессоров, поскольку $\dim(z)>K$ и уравнений больше, чем неизвестных. Одна из возможностей
заключается в использовании только $K$ инструментов, но тогда происходит потеря эффективности. Оценки ОММ вместо этого выбирают $\hat{\beta}$ так, чтобы вектор $N^{-1} \sum_i z_i(y_i-x'_i\beta)$ был как можно меньшим, используя квадратичные потери, чтобы $\hat{\beta}_{GMM}$ минимизировала
\begin{equation}
\mathcal{Q}_{N}(\beta)= \left[ \frac{1}{N} \sum_{i=1}^{N} z_i(y_i-x'_i\beta) \right]' W_N \left[ \frac{1}{N} \sum_{i=1}^{N} z_i(y_i-x'_i\beta) \right],
\end{equation}
где $W_N$ --- матрица весов размера $\dim(z) \times \dim(z)$. Оценка двухшагового МНК (см. Раздел 4.8.6) соответствует определённой матрице $W_N$.

Метод инструментальных переменных для линейных моделей представлен в деталях в Разделе 6.4. Преимуществом подхода ОММ является то, что он предоставляет возможность указать оптимальный выбор матрицы весов $W_N$, что приводит к более эффективной оценке, чем оценка двухшагового МНК.

Раздел 6.5 охватывает метод инструментальных переменных для нелинейных моделей. Одним из преимуществ подхода ОММ является то, что обобщение на случай нелинейной регрессии является прямым. Тогда мы просто заменяем $y-x'\beta$ в предыдущем выражении для $\mathcal{Q}_{N}(\beta)$ на нелинейную модель ошибки $u=y-g(x'\beta)$ или $u=r(y,x,\beta)$.

\subsection{Панельные данные}

Другое важное применение ОММ и связанных с ним методов оценивания --- регрессии панельных данных.

В качестве примера предположим, что $y_{it}=x'_{it}\beta+u_{it}$, где $i$ обозначает индивидов и $t$ обозначает время. Из Раздела 6.2.1 сквозная МНК регрессия $y_{it}$ на $x_{it}$ является оценкой ММ, основанной на условии $\E[x_{it}u_{it}]=0$. Предположим дополнительно, что ошибка $u_{it}$ не коррелирует с регрессорами в период времени, выходящий за рамки текущего периода. Тогда условие $\E[x_{is}u_{it}] = 0$ для $s \not= t$ создаёт дополнительные условия моментов, которые могут быть использованы для получения более эффективных оценок.

Главы 22 и 23 содержат множество применений ОММ для панельных данных.

\subsection{Условия моментов из экономической теории}

Экономическая теория может генерировать условия на моменты, которые могут быть использованы в качестве основы для оценивания.

Начнём с модели:
\[
y_t=\E[y_t|x_t,\beta]+u_t,
\]
где первое слагаемое в правой части измеряет <<ожидаемую>> компоненту $y$, зависимого от $x$, а второе слагаемое измеряет <<непредвиденную>> компоненту. Например, $y$ может означать доходность актива или уровень инфляции. При выполнении предположений о рациональности ожиданий и о равновесии на рынке или о рыночной эффективности, непредвиденная компонента непредсказуема с использованием любой информации, которая была доступна в момент времени $t$ для определения $\E[y|x]$. Тогда
\[
\E[(y_t-\E[y_t|x_t,\beta])|\mathcal{I}_{t}]=0,
\]
где $\mathcal{I}_{t}$ обозначает информацию, доступную на момент $t$.

На основе закона о повторном математическом ожидании, $\E[z_t(y_t-\E[y_t|x_t,\beta])]=0$, где $z_t$ формируется на базе информации из $\mathcal{I}_{t}$. Поскольку любая часть информации может быть использована в качестве инструмента, это обеспечивает много условий моментов, которые могут быть основой оценки. Если временные ряды недоступны, тогда ОММ минимизирует квадратичную форму:
\[
\mathcal{Q}_{T}(\beta)= \left[ \frac{1}{T} \sum_{t=1}^{T} z_t u_t \right]' W_T \left[ \frac{1}{T} \sum_{t=1}^{T} z_t u_t \right],
\]
где $u_t=y_t-\E[y_t|x_t,\beta]$. Если пространственные данные доступны для единственного момента времени $t$, то ОММ минимизирует квадратичную форму:
\[
\mathcal{Q}_{N}(\beta)= \left[ \frac{1}{N} \sum_{i=1}^{N} z_i u_i \right]' W_T \left[ \frac{1}{N} \sum_{i=1}^{N} z_i u_i \right],
\]
$u_i=y_i-\E[y_i|x_i,\beta]$, а индекс $t$ может быть удалён, поскольку анализируется только один период времени.

Этот подход не ограничивается только аддитивной структурой, используемой в мотивации. Нужно только чтобы для ошибки $u_t$ выполнялось свойство $\E[u_t|\mathcal{I}_{t}]=0$. Такие условия
вытекают из условий Эйлера в межвременной модели принятия решений в условиях неопределенности. Например, Хансен и Синглтон (1982) предлагают модель максимизации ожидаемой полезности в течение жизни, в которой выполнено условие Эйлера $\E[u_t|\mathcal{I}_{t}]=0$, где $u_t=\beta g^{\alpha}_{t+1} r_{t+1}-1,g_{t+1}=c_{t+1}/c_t$ представляет собой отношение потребления в двух периодах и $r_{t+1}$ --- доходность актива.
Параметры $\alpha$ и $\beta$, межвременная ставка дисконтирования и коэффициент относительного неприятия риска, соответственно, могут быть оценены с использованием ОММ на временных рядах или на пространственных данных, как это делалось раньше, с этим определением $u_t$. Хансен (1982) и Хансен и Синглтон (1982) рассматривают временные ряды; МаКарди (1983) моделировал как потребление, так и предложение рабочей силы с использованием панельных данных.

\begin{table}[h]
\begin{center}
\caption{\label{tab:GMM} Обобщённый метод моментов: примеры}
\begin{tabular}[t]{ll}
\hline
\hline
\bf{Функция моментов $h(\cdot)$} & \bf{Метод оценки} \\
\hline
$y-\mu$ & Метод моментов для  генеральной совокупности \\
& математического ожидания \\
$x(y-x'\beta)$ & Регрессия методом наименьших квадратов \\
$z(y-x'\beta)$ & Регрессия с инструментальными переменными \\
$\partial ln f(y|x,\theta) / \partial \theta$ & Метод максимального правдоподобия \\
\hline
\hline
\end{tabular}
\end{center}
\end{table}

\section{Обобщённый метод моментов}

В этом разделе представлена общая теория оценивания ОММ. Обобщённый метод моментов определяет класс оценок. Другой выбор условия моментов и матрицы весов приводит к различным оценкам ОММ так же, как различный выбор распределения приводит к различным оценками ММП. Мы рассмотрим эти проблемы, в дополнение к рассмотрению обычных свойств состоятельности и асимптотической нормальности и методов оценки ковариационной матрицы оценки ОММ.

\subsection{Оценка метода моментов}

Отправной точкой является наличие условий для $r$ моментов для $q$ параметров, где \begin{equation}
\E[h(w_i,\theta_0)]=0,
\end{equation}
где $\theta$ --- вектор размера $q \times 1$, $h(\cdot)$ --- векторная функция размера $r \times 1$ с $r \ge q$, $\theta_0$ указывает значение $\theta$ в процессе, порождающем данные. Вектор $w$ включает в себя все наблюдаемые величины. В том числе, в соответствующих случаях, зависимую переменную $y$, потенциально эндогенные регрессоры $x$, а также $z$ инструментальных переменных. Зависимая переменная $y$ может быть вектором, чтобы приложения с системами уравнений или с панельными данными были частными случаями этого подхода. Математическое ожидание зависит от всех стохастических компонент $w$ и, следовательно, $y$, $x$ и $z$.

Выбор функциональной формы $h(\cdot)$ качественно похож на выбор модели, и потому будет зависеть от области приложения. Таблица 6.1 суммирует некоторые примеры с одним уравнением $h(w)=h(y,x,z,\theta)$, уже представленные в Разделе 6.2.

Если $r=q$, то может быть применён метод моментов. Равенство нулю теоретического момента заменяется на равенство нулю соответствующего выборочного момента, и оценка метода моментов $\hat{\theta}_{MM}$ определяется как решение:
\begin{equation}
\frac{1}{N} \sum_{i=1}^{N} h(w_i,\hat{\theta})=0.
\end{equation}

Это оценка метода оценочных уравнений, которая эквивалентно минимизирует 
\[
\mathcal{Q}_{N}(\theta)= \left[ \frac{1}{N} \sum_{i=1}^{N} h(w_i,\theta) \right]' \left[ \frac{1}{N} \sum_{i=1}^{N} h(w_i,\theta) \right],
\]
с асимптотическим распределением, представленным в Разделе 5.4 и воспроизведённым в (6.13) в Разделе 6.3.3 .

\subsection{Оценка ОММ}

Оценки ОММ основаны на $r$ независимых условиях моментов (6.5) при оценивании $q$ параметров.

Если $r=q$, модель называют точно идентифицированной и могут быть использованы ММ оценки в (6.6). Более формально $r=q$ является лишь необходимым условием для достаточной идентификации, и мы дополнительно требуем, чтобы $G_0$ в предложении 5.1 имела ранг $q$. Идентификация рассматривается в Разделе 6.3.9.

Если $r>q$, модель называется сверх-идентифицированной и (6.6) не имеет решения для $\hat{\theta}$, так как уравнений $(r)$ больше, чем неизвестных $(q)$. Вместо этого $\hat{\theta}$ выбирается так, чтобы квадратичная форма $N^{-1} \sum_i h(w_i,\hat{\theta})$ была как можно ближе к нулю. В частности, оценка обобщённого
метода моментов $\hat{\theta}_{GMM}$ минимизирует целевую функцию:
\begin{equation}
\mathcal{Q}_{N}(\theta)= \left[ \frac{1}{N} \sum_{i=1}^{N} h(w_i,\theta) \right]' W_N \left[ \frac{1}{N} \sum_{i=1}^{N} h(w_i,\theta) \right],
\end{equation}
где матрица весов $W_N$ размера $r \times r$ является симметричной положительно определённой, возможно стохастической с конечным пределом по вероятности и не зависит от $\theta$. Индекс $N$ в $W_N$ используется для указания, что её значение может зависеть от выборки. Однако размерность $r$ $W_N$ фиксирована, так как $N \rightarrow \infty$. Целевая функция также может быть выражена в матричном виде как $\mathcal{Q}_{N}(\theta)= N^{-1}I'H(\theta) \times W_N \times N^{-1}H(\theta)'I$, где $I$ представляет собой вектор  из единиц размера $N \times 1$ и $H(\theta)$ --- матрица размера $N \times r$ с $i$-ой строкой равной $h(y_i,x_i,\theta)'$.

Различные варианты матрицы весов $W_N$ приводят к различным оценками, которые, хотя и состоятельные, имеют различные дисперсии, если $r>q$. Простой выбор, хотя часто и не очень хороший, взять в качестве $W_N$ единичную матрицу. Тогда $\mathcal{Q}_{N}(\theta)={\bar{h}_1}^2+{\bar{h}_2}^2+\cdots+{\bar{h}_r}^2$ --- сумма $r$ квадратов средних арифметических, где $\bar{h_j}=N^{-1} \sum_i h_j(w_i,\theta)$ и $h_j(\cdot)$ --- $i$-ый элемент $h(\cdot)$. Оптимальный выбор $W_N$ приведён в Разделе 6.3.5.

Дифференцирование $\mathcal{Q}_{N}(\theta)$ в (6.7) по $\theta$ даёт условия первого порядка для ОММ:
\begin{equation}
\left[ \left. \frac{1}{N} \sum_{i=1}^{N} \frac{\partial h_i(\hat{\theta})'}{\partial \theta} \right|_{\hat{\theta}} \right] \times W_N \times \left[ \frac{1}{N} \sum_{i=1}^{N} h_i(\hat{\theta}) \right]=0,
\end{equation} 
где $h_i(\theta)=h_i(w_i,\theta)$ и для масштабирования мы умножили на коэффициент $1/2$. Эти уравнения обычно будут нелинейными по $\hat{\theta}$ и могут быть достаточно сложными для решения, так как $\hat{\theta}$  может появиться как в первом, так и в третьем слагаемом. Численные методы решения представлены в Главе 10.

\subsection{Распределение оценок ОММ}

Асимптотическое распределение оценки ОММ даётся в следующих утверждениях, доказываемых в Разделе 6.3.9.

\begin{proposition}[Распределение оценок ОММ] Сделаем следующие допущения:
\begin{enumerate}
\item Процесс, порождающий данные, удовлетворяет условиям на моменты (6.5), то есть $\E[h(w,\theta_0)]=0$.
\item Вектор-функция $h(\cdot)$ размера $r \times 1$ удовлетворяет $h(w,\theta^{(1)})=h(w,\theta^{(2)})$, если и только если $\theta^{(1)}=\theta^{(2)}$.
\item  Следующая матрица размера $r \times q$ существует и конечна с рангом $q$: 
\begin{equation}
G_0=\plim \frac{1}{N} \sum_{i=1}^{N} \left[ \left. \frac{\partial h_i}{\partial \theta'} \right|_{\theta_0} \right].
\end{equation}
\item  $W_N \xrightarrow{p} W_0$, где $W_0$ конечная симметричная положительно определённая матрица. 
\item $N^{-1/2} \sum_{i=1}^{N} h_i|_{\theta_0} \xrightarrow{d} \mathcal{N}[0,S(\theta_0)]$, где 
\begin{equation}
S_0=\plim N^{-1} \sum_{i=1}^{N} \sum_{j=1}^{N} \left[ h_i h'_j|_{\theta_0} \right].
\end{equation}
\end{enumerate}
Тогда оценка ОММ $\hat{\theta}_{GMM}$, которая определяется как решение условий первого порядка $\partial \mathcal{Q}_{N}(\theta) / \partial \theta=0$, приведенных в (6.8), состоятельна $\theta_0$ и
\begin{equation}
\sqrt{N} (\hat{\theta}_{GMM}-\theta_0) \xrightarrow{d} \mathcal{N}[0,(G'_0 W_0 G_0)^{-1} (G'_0 W_0 S_0 W_0 G_0) (G'_0 W_0 G_0)^{-1}].
\end{equation}
\end{proposition}

Некоторые важные специализации представлены ниже.

Во-первых, в микроэконометрическом анализе данные, как правило, предполагаются независимыми по $i$, поэтому (6.10) упрощается до следующего вида:
\begin{equation}
S_0= \plim N^{-1} \sum_{i=1}^{N} \left[ h_i h'_i|_{\theta_0} \right].
\end{equation}

Если ещё предполагается, что данные одинаково распределены, тогда (6.9) и (6.10) упрощаются до $G_0=\E[\partial h/ \partial \theta'|_{\theta_0}]$ и $S_0=\E[hh'|_{\theta_0}]$. Такие обозначения используются многими авторами.

Во-вторых, в случае точной идентификации, $r=q$, случая верного для многих оценок, в том числе ММП и МНК, результаты упрощаются до тех, которые уже представлены в Разделе 5.4 для оценки оценочных уравнений. Чтобы убедиться в этом обратим внимание, что при $r=q$ матрицы $G_0$, $W_0$ и $S_0$ --- квадратные матрицы, которые являются обратимыми, поэтому $(G'_0 W_0 G_0)^{-1} = G^{-1}_0 W^{-1}_0 {(G'_0)}^{-1}$ и ковариационная матрица (6.11) упрощается. Отсюда следует, что для ММ оценки в (6.6), 
\begin{equation}
\sqrt{N} (\hat{\theta}_{MM}-\theta_0) \xrightarrow{d} \mathcal{N}[0,{G}^{-1}_0 S_0 (G'_0)^{-1}].
\end{equation}

Оценки ММ всегда могут быть вычислены как оценки ОММ и будет инвариантны к выбору матрицы весов полного ранга.

В-третьих, лучший выбор матрицы $W_N$ --- такой, что $W_0=S^{-1}_0$. Тогда ковариационная
матрица в (6.11) упрощается до $(G'_0 S^{-1}_0 G_0)^{-1}$. Это рассматривается подробно в Разделе 6.3.5.

\subsection{Оценка ковариационной матрицы}

Статистические выводы для оценки ОММ возможны при состоятельных оценок $\hat{G}$ для $G_0$,$\hat{W}$ для $W_0$, $\hat{S}$ для $S_0$ в (6.11). Состоятельные оценки легко получаются при относительно слабых предположениях о распределении.

Для $G_0$ очевидной оценкой является следующая:
\begin{equation}
\hat{G}=\frac{1}{N} \left. \sum_{i=1}^{N} \frac{\partial h_i}{\partial \theta'} \right|_{\hat{\theta}}.
\end{equation}

Для матрицы $W_0$ используется матрица весов $W_N$ для выборки. Оценка матрицы $S_0$  размера $r \times r$ меняется в зависимости от стохастических предположений, сделанных относительно процесса, генерирующего данные. Микроэконометрический анализ обычно предполагает независимость по $i$, поэтому $S_0$ имеет более простой вид (6.12). Очевидной оценкой будет:
\begin{equation}
\hat{S}=\frac{1}{N} \sum_{i=1}^{N} h_i(\hat{\theta}) h_i(\hat{\theta})'.
\end{equation}
Поскольку $h(\cdot)$ имеет размер $r \times 1$, количество уникальных значений в $S_0$, подлежащих оцениванию, конечно и не превосходит $r(r+1)/2$. Поэтому $\hat{S}$ --- состоятельная оценка при $N \rightarrow  \infty$. Необходимости параметризации дисперсии $\E[h_i h'_i]$ меньшим числом параметров не появляется, хотя требуется её существование дисперсии. Всё, что необходимо, это некоторые дополнительные мягкие предположения, чтобы $\plim N^{-1} \sum_i \hat{h}_i \hat{h'}_i= \plim N^{-1} \sum_i h_i h'_i$. Например, если $\hat{h}_i=x_i \hat{u}_i$, где $\hat{u}_i$ --- остатки МНК, то, как мы знаем из Раздела 4.4, необходимо предполагать существование четвёртых моментов регрессоров.

Объединяя эти результаты, мы получаем, что оценки ОММ асимптотически нормально распределены с математическим ожиданием $\theta_0$ и оценочной асимптотической ковариационной матрицей:
\begin{equation}
\widehat{\Var}[\hat{\theta}_{MM}]=\frac{1}{N} (\hat{G'} W_N \hat{G})^{-1} \hat{G'} W_N \hat{S}  W_N  \hat{G}(\hat{G'} W_N \hat{G})^{-1}.
\end{equation}
Эта оценка ковариационной матрицы является робастной оценкой и является расширением устойчивой к гетероскедастичности оценке Эйкера-Уайта для случая наименьших квадратов.

Можно также взять математическое ожидание и использовать $\hat{G}_{E}=N^{-1} \sum_i \E[\partial h_i / \partial \theta']|_{\hat{\theta}}$ для $G_0$ и $\hat{S}_{E}= N^{-1} \sum_i \E[h_i h'_i]|_{\hat{\theta}}$ для $S_0$. Тем не менее, обычно для того, чтобы взять математические ожидания, требуются дополнительные предположения о распределении, и оценка ковариационной матрицы не будет столь же устойчивой к неправильной спецификации распределения.

В случае временных рядов $h_t$ имеет индекс $t$, и асимптотическая теория основана на количестве периодов времени $T \rightarrow \infty$. Для временных рядов, где  $h_t$ --- векторный $MA(q)$ процесс, обычный оценкой $\Var[\hat{\theta}_{GMM}]$ является предложенная Ньюи и Вестом (1987b), которая использует (6.16) при $\hat{S}=\hat{\Omega_0} + \sum_{j=1}^{q}(1-\frac{j}{q+1})(\hat{\Omega}_j+\hat{\Omega'}_j)$, $\hat{\Omega}_j=T^{-1} \sum_{i=j+1}^{T} \hat{h}_t \hat{h'}_{t-j}$. Это допускает корреляцию рядов во времени в $h_t$ помимо одновременной корреляции. Более подробная информация об оценке ковариационной матрицы, в том числе об изменениях для случая временных рядов, приведена у Дэвидсона и МакКиннона (1993, Разделе 17.5), Гамильтона (1994), и Хаана и Левина (1997).

\subsection{Оптимальная матрица весов}

Применение ОММ требует спецификации функции моментов $h(\cdot)$ и матрицы весов $W_N$ в (6.7).

Легкая часть заключается в выборе $W_N$, чтобы получить оценки ОММ с наименьшей асимптотической дисперсией при заданной функции $h(\cdot)$. Это часто называют оптимальным ОММ, хотя это ограниченная форма оптимальности, поскольку плохой выбор $h(\cdot)$ может всё же привести к очень неэффективным оценкам.

Для точно идентифицированных моделей одинаковые оценки (ММ оценки) получаются для любой матрицы весов полного ранга, то есть можно просто приравнять $W_N=I_q$.

Для сверх-идентифицированных моделей с $r>q$ и при известной $S_0$ наиболее эффективные оценки  ОММ получают путём выбора матрицы весов $W_N=S^{-1}_0$. Тогда ковариационная матрица, заданная в утверждении, упрощается и
\begin{equation}
\sqrt{N} (\hat{\theta}_{GMM}-\theta_0) \xrightarrow{d} \mathcal{N}[0,(G'_0 S^{-1}_0 G_0)^{-1}], 
\end{equation}
результат Хансена (1982).

Этот результат можно получить, используя аргументы, аналогичные тем, с помощью которых доказывалось, что оценка ОМНК является наиболее эффективной оценкой ВНМК в линейной модели. Ещё проще, можно работать непосредственно с целевой функцией. Для оценок МНК, которые минимизируют квадратичную форму $u'Wu$, наиболее эффективной оценка является оценка ОМНК, для которой $W=\Sigma^{-1}=\Var[u]^{-1}$. Целевая функция ОММ в (6.7) является такой же квадратичной формой с $u=N^{-1} \sum_i h_i(\theta)$ и поэтому оптимальная матрица $W$ $(\Var[N^{-1} \sum_i h_i(\theta)])^{-1}=S^{-1}_0$. Веса оптимальных ОММ оценок равны обратной ковариационной матрице условий выборочных моментов.

\begin{center}
Оптимальный ОММ
\end{center}

На практике $S_0$ неизвестна, и мы предполагаем $W_N={\hat{S}}^{-1}$, где $\hat{S}$ состоятельная оценка для $S_0$. Оптимальные оценки ОММ можно получить с помощью двухшаговой процедуры. На первом шаге оценки ОММ получают с использованием субоптимального выбора $W_N$, например, $W_N=I_r$ для простоты. С этого первого шага формируют оценку $\hat{S}$, используя (6.15). На втором этапе получают оптимальную оценку ОММ при применении оптимальной матрицы весов $W_N={\hat{S}}^{-1}$.

В этом случае оптимальные оценки ОММ или двухступенчатые оценки ОММ $\hat{\theta}_{OGMM}$, основанные на $h_i(\theta)$, минимизирует
\begin{equation}
\mathcal{Q}_{N}(\theta)=\left[ \frac{1}{N} \sum_{i=1}^{N} h_i(\theta) \right]' {\hat{S}}^{-1}  \left[ \frac{1}{N} \sum_{i=1}^{N} h_i(\theta) \right].
\end{equation}
Предельное распределение дано в (6.17). Оптимальная оценка ОММ является асимптотически нормально распределённой с математическим ожиданием $\theta_0$ и оценочной асимптотической ковариационной матрицей с относительно простой формулой:
\begin{equation}
\Var[\hat{\theta}_{OGMM}]=N^{-1}(\hat{G'} {\tilde{S}}^{-1} \hat{G})^{-1}.
\end{equation}

Обычно матрицы $\hat{G}$ и $\tilde{S}$ оцениваются в точке $\hat{\theta}_{OGMM}$, поэтому $\tilde{S}$ использует ту же формулу, что и $\hat{S}$, за исключением точки оценивания $\hat{\theta}_{OGMM}$. В качестве альтернативы можно считать (6.19) в точке, полученной на первом шаге оценки, поскольку любая состоятельная оценка для $\theta_0$ может быть использована.

Примечательно, что оптимальная оценка ОММ в (6.18) не требует никаких дополнительных стохастических предположений помимо тех, которые необходимы, чтобы использовать (6.16) для оценки ковариационной матрицы субоптимального ОММ. В обоих случаях $\hat{S}$ должна быть состоятельной оценкой $S_0$, и из обсуждения после (6.15) ясно, что для этого требуется несколько дополнительных предположений. Здесь видно значительное отличие от дополнительных предположений, необходимых для ОМНК, чтобы оценки были более эффективными, чем оценки МНК при гетероскедастичности. Однако гетероскедастичность в ошибках повлияет на оптимальный выбор $h_i(\theta)$(см. Раздел 6.3.7).

\begin{center}
Смещение двухступенчатого ОММ в небольших выборках 
\end{center}

Теория предполагает, что для сверх-идентифицированных моделей лучше всего использовать оптимальный ОММ. Однако при применении теоретически оптимальная матрица весов $W_N={S_0}^{-1}$ должна быть заменена на состоятельную оценку ${\hat{S}}^{-1}$. Эта замена не меняет асимптотики, но будет играть существенную роль в конечных выборках. В частности, отдельные наблюдения, которые увеличивают $h_i(\theta)$ в (6.18), вероятно, увеличивают $\hat{S}=N^{-1} \sum_i \hat{h}_i \hat{h'}_i$, что ведёт к корреляции между $N^{-1} \sum_i h_i(\theta)$ и $\hat{S}$. Следует отметить, что аналогичного эффекта на $S_0=\plim N^{-1} \sum_i h_i h'_i$ нет, потому что берётся предел по вероятности.

Алтони и Сегал (1996) продемонстрировали эту проблему для оценки моделей ковариационной
структуры с использованием панельных данных (см. Раздел 22.5). Они использовали соответствующие оценки минимального расстояния (см. раздел 6.7), но в литературе их результаты интерпретируются как относящиеся к ОММ с пространственными данными или короткими панелями. В моделировании оптимальная оценка была более эффективна, чем оценка одношагового метода, как и ожидалось. Тем не менее, у оптимальной оценки было смещение в конечной выборке настолько большое, что корень из среднеквадратической ошибки был гораздо больше, чем для оценки одного шага.

Алтони и Сегал (1996) также предложили вариант независимо взвешенной оптимальной оценки, которая формирует матрицу весов, используя наблюдения, отличные от тех, которые используются для построения выборочных моментов. Они разделили выборку на $G$ групп, например, подойдет простой выбор $G=2$, и минимизировали:
\begin{equation}
\mathcal{Q}_{N}(\theta)= \frac{1}{G} \sum_g h_g(\theta) {\hat{S}_{(-g)}}^{-1} h_g(\theta),
\end{equation}
где $h_g(\theta)$ вычисляется для $g$-ой группы и $\hat{S}_{(-g)}$ вычисляется с использованием всех групп, кроме $g$-ой. Эта оценка является менее смещённой, так как матрица весов ${\hat{S}_{(-g)}}^{-1}$ по построению независима от $h_g(\theta)$. Тем не менее, деление выборки приводит к потере эффективности. Горовиц (1998a) вместо этого использовал бутстрэп (см. Раздел 11.6.4). 

У Алтони и Сегала (1996) в примере в $h_i$ использовались вторые моменты, поэтому в $\hat{S}$ задействованы четвёртые моменты. Проблемы для оптимальной оценки в конечной выборке могут быть не столь значительными в других примерах, когда $h_i$ включает в себя только первые моменты. Тем не менее, результаты Алтони и Сегала действительно призывают к осторожности при использовании оптимального ОММ и что различия между оценкой одношагового ОММ и оптимальной оценки ОММ может указывать на проблемы смещения в оптимальном ОММ в конечной выборке.

\begin{center}
Количество ограничений на моменты
\end{center}

Как правило добавление дополнительных ограничений на моменты улучшает асимптотическую эффективность, так как это снижает предел дисперсии $(G'_0 S^{-1}_0 G_0)^{-1}$ оптимальной оценки ОММ или в худшем случае не меняет её.

Преимущества добавления дополнительных условий моментов меняются в зависимости от применения. Если, например, речь идёт об оценке ММП, то нет никакой выгоды, так как оценка ММП уже полностью эффективна. Много литературы посвящено оценке метода инструментальных переменных, где выгода может быть значительной, потому что переменная, которая является инструментальной, может быть гораздо более коррелированной с комбинацией многих инструментов, чем с одним инструментом.

Однако существует предел, так как количество моментных ограничений не может превышать число наблюдений. Кроме того, добавление нескольких условий моментов увеличивает вероятность смещения в конечной выборке и связанных с этим проблем, аналогичных таким, как слабые инструменты в линейных моделях (см. Раздел 4.9). Сток и другие (2002) кратко рассматривают слабые инструменты в нелинейных моделях.

\subsection{Пример регрессии с симметричными ошибками}

Чтобы продемонстрировать асимптотические результаты ОММ вернёмся к примеру дополнительных моментных ограничений, введённому в Разделе 6.2.4. Для этого примера целевая функция $\hat{\beta}_{GMM}$ была уже дана в (6.2). Всё, что требуется, это спецификация $W_N$ такая, как, например, $W_N=I$.

Чтобы получить распределение этой оценки, мы используем общие обозначения Раздела 6.3. Функция $h(\cdot)$ в (6.5) может быть представлена таким образом:
\[
h(x,y,\beta)= \begin{bmatrix} x(y-x'\beta)  \\ x(y-x'\beta)^3 \end{bmatrix} \Rightarrow \frac{\partial h(x,y,\beta)}{\partial \beta'}= \begin{bmatrix} -xx' \\ -3xx'(y-x'\beta)^2 \end{bmatrix}.
\]
Эти выражения ведут непосредственно к выражениям для $G_0$ и $S_0$, используя (6.9) и (6.12), и формулы (6.14) и (6.15) дают состоятельные оценки:
\begin{equation}
\hat{G}= \begin{bmatrix} -\frac{1}{N} \sum_i x_i x'_i \\ -\frac{1}{N} \sum_i 3 {\hat{u}_i}^2 x_i x'_i \end{bmatrix}
\end{equation}
и
\begin{equation}
\hat{S}= \begin{bmatrix} \frac{1}{N} \sum_i {\hat{u}_i}^2 x_i x'_i & \frac{1}{N} \sum_i {\hat{u}_i}^4 x_i x'_i \\ \frac{1}{N} \sum_i {\hat{u}_i}^4 x_i x'_i & \frac{1}{N} \sum_i {\hat{u}_i}^6 x_i x'_i \end{bmatrix},
\end{equation}
где $\hat{u}_i=y-x'_i \hat{\beta}$. Альтернативные оценки могут быть получены с помощью вычисления первых математических ожиданий в $G_0$ и $S_0$, но это потребует предположений о $\E[u^2|x]$, $\E[u^4|x]$ и $\E[u^6|x]$. Подстановка $\hat{G}$ и $\hat{S}$ и $W_N$ в (6.16) даёт оценку асимптотической ковариационной матрицы для $\hat{\beta}_{GMM}$.

Теперь рассмотрим ОММ с оптимальной матрицей весов. Снова минимизируется выражение (6.2), но из (6.18) теперь $W_N={\hat{S}}^{-1}$, где $\hat{S}$ определена в (6.22). Вычисление $\hat{S}$ требует состоятельной оценки $\hat{\beta}$ на первом шаге. Естественный выбор --- ОММ с $W_N=I$. 
В этом примере МНК-оценка также состоятельна и также могла бы быть использована. Использование (6.19) даёт двухступенчатую оценку с оценкой асимптотической ковариационной матрицы $\widehat{\Var}[\hat{\beta}_{OGMM}]$, равной
\[
\left( \begin{bmatrix} \sum_i \tilde{u}_i x_i x'_i \\ \sum_i {\tilde{u}_i}^3 x_i x'_i \end{bmatrix}' \begin{bmatrix} \sum_i {\tilde{u}_i}^2 x_i x'_i & \sum_i {\tilde{u}_i}^4 x_i x'_i \\  \sum_i {\tilde{u}_i}^4 x_i x'_i & \sum_i {\tilde{u}_i}^6 x_i x'_i \end{bmatrix}^{-1} \begin{bmatrix} \sum_i \tilde{u}_i x_i x'_i \\ \sum_i {\tilde{u}_i}^3 x_i x'_i \end{bmatrix} \right)^{-1},
\]
где $\tilde{u}_i=y-x'_i \hat{\beta}_{OGMM}$ и деления на $N$ сокращаются. Выигрыш эффективности оптимального ОММ в этом примере можно легко посчитать в нерегрессионном случае, где $y$ независимы и одинаково распределены с математическим ожиданием $\mu$. Предположим, что $y$ распределён по Лапласу с параметром масштаба, равным единице, в этом случае плотность $f(y)=(1/2) \times \exp \{-|y-\mu| \}$ c $\E[y]=\mu$, $\Var[y]=2$, и центральные моменты высших порядков $\E[(y-\mu)^{r}]$ равны нулю для нечётных $r$ и равны $r!$ для чётных $r$. Выборочная медиана является полностью эффективной, так как это оценка ММП, и может быть показано, что у неё асимптотическая дисперсия равна $1/N$. Выборочное среднее $\bar{y}$ неэффективно с дисперсией $\Var[\bar{y}]=\Var[y]/N=2/N$. Оптимальная оценка ОММ $\hat{\mu}^{opt}$, основанная на условиях двух моментов $\E[(y-\mu)]=0$ и $\E[(y-\mu)^3]=0$, имеет матрицу весов, которая даёт намного меньший вес условию второго момента, потому что оно имеет относительно высокую дисперсию. У матрицы весов будут отрицательные недиагональные элементы. Может быть показано, что у оптимальной оценки ОММ $\hat{\mu}_{OGMM}$ асимптотическая дисперсия равна $1.7143/N$ (см. упражнение 6.3). Поэтому она более эффективна, чем выборочное среднее (дисперсия $2/N$), хотя она всё ещё значительно менее эффективна, чем выборочная медиана.

Для этого примера единичная матрица является исключительно плохим выбором матрицы весов. Она даёт слишком большой вес условию второго момента, давая субоптимальную ОММ оценку для $\mu$  с асимптотической дисперсией $19.14/N$, что во много раз больше, чем даже $\Var[\bar{y}]=2/N$. Для получения дополнительной информации см. упражнение 6.3.

\subsection{Оптимальные моментные условия}

Раздел 6.3.5 даёт удивительный результат, что оптимальный ОММ требует по существу не больше предположений, чем в ОММ без оптимальный матрицы весов. Однако эта оптимальность очень ограничена, так как она зависит от выбора функции моментов $h(\cdot)$ в (6.5) или (6.18).

ОММ определяет класс оценок.  Различный выбор $h(\cdot)$, соответствует разным представителям этого класса. Некоторые функции $h(\cdot)$ лучше, чем другие, и зависит это от предположений о распределениях. Например, $h_i=x_i u_i$ даёт МНК-оценки, в то время как $h_i=x_i u_i / \Var[u_i|x_i]$ даёт оценку ОМНК, когда ошибки гетероскедастичны. Из-за множества потенциальных вариантов для $h(\cdot)$ может казаться, что любая ОММ оценка выбирается ad hoc. Тем не менее, качественно аналогичные решения должны быть сделаны и при выборе М-оценки, например, минимизировать сумму квадратов ошибок, или взвешенную сумму квадратов ошибок или сумму абсолютных отклонений ошибок.

Если сделаны полные допущения о распределении, наиболее эффективная оценка --- оценка ММП. Таким образом, оптимальный выбор $h(\cdot)$ в ( 6.5) --- это
\[
h(w,\theta)=\frac{ \partial \ln f(w,\theta)}{\partial \theta},
\]
где $f(w,\theta)$ является совместной плотностью $w$. Для регрессии с зависимой(ыми) переменной(ми) $y$ и регрессорами $x$ это оценка безусловного ММП на основе безусловной совместной плотности $f(y,x,\theta)$ $y$ и $x$. Во многих случаях $f(y,x,\theta)=f(y|x,\theta)g(x)$, где опущенные параметры частной функции плотности $x$ не зависят от параметров $\theta$, которые нас интересуют. Тогда так же эффективно использовать оценку условного ММП, основанную на условной плотности $f(y|x,\theta)$. На основе этой идеи можно построить ММ оценку или оценку ОММ с матрицей весов $W_N=I_q$, хотя любая матрица полного ранга также даст оценку ММП. Однако этот результат имеет ограниченное практическое применение, так как цель оценки ОММ заключается в том, чтобы избежать полного набора предположений о распределении.

При неполной спецификации распределении стандартной отправной точкой является спецификация уравнений для условных моментов, экзогенные переменные при этом полагаются фиксированными. Как правило, это условие на моменты низких порядков для ошибок модели, например, $\E[u|x]=0$ или $\E[u|z]=0$. Эти уравнения для условного момента могут привести к большому количеству уравнений для безусловных моментов, которые могут стать основой для оценки ОММ, например, $\E[zu]=0$. Ньюи (1990a, 1993) получил результаты касающиеся оптимального выбора уравнений для безусловного момента на данных независимых по $i$.

В частности, начнём с $s$ уравнений для условных моментов:
\begin{equation}
\E[r(y,x,\theta_0)|z]=0,
\end{equation}
где $r(\cdot)$ является векторной функцией, например, ошибок модели, размера $s \times 1$, которая была введена в Разделе 6.2.2. Скалярный пример --- $\E[y-x'\theta_0|z] = 0$. Мы используем обозначения принятые для инструментальных переменных, здесь $x$ --- это регрессоры, некоторые могут быть  эндогенными, а $z$ --- инструменты, включающие экзогенные компоненты $x$. В более простых моделях без эндогенности $z=x$.

Оценки ОММ $q$ параметров $\theta$, основанные на (6.23), невозможны, так как обычно существует только несколько ограничений условных моментов и часто только одно такое, что $s \le q$. Вместо этого, мы вводим матричную функцию инструментов $D(z)$ размера $r \times s$, где $r \ge q$, и необходимо обратить внимание, что по закону повторного математического ожидания $\E[D(z)r(y,x,\theta_0)]=0$. Эту идею можно использовать в качестве основы для оценки ОММ. Может быть показано, что оптимальные инструменты или оптимальный выбор матричной функции $D(z)$ --- это матрица размера $q \times s$:
\begin{equation}
D^*(z,\theta_0)=\E \left[ \frac{\partial r(y,x,\theta_0)'}{\partial \theta}|z \right] \{ \Var[r(y,x,\theta_0)|z]\}^{-1}.
\end{equation}
Вывод приведён, например, у Дэвидсона и МакКиннона (1993, с.604). Матрица оптимальных инструментов $D^*(z)$ --- это матрица размера $q \times s$, поэтому условие для безусловного момента $\E[D^*(z) r(x,y,\theta_0)]=0$ даёт ровно столько условий моментов, сколько параметров. Оптимальная оценка ОММ является решением соответствующих условий выборочных моментов:
\begin{equation}
\frac{1}{N} \sum_{i=1}^{N} D^*(z_i,\theta) r(y_i,x_i,\theta)=0.
\end{equation}

Оптимальная оценка требует дополнительных предположений, а именно математические ожидания, используемые при образовании $D^*(z,\theta_0)$ в ( 6.24), и реализация требуют замены неизвестных параметров на известные параметры, чтобы были использованы сгенерированные регрессоры $\hat{D}$.

Например, если $r(y,x,\theta)=y-\exp(x'\theta)$, то $\partial r / \partial \theta = -\exp(x'\theta)x$ и (6.24) требует спецификации $\E[\exp(x'\theta_0)x|z]$ и $\Var[y-\exp(x'\theta)|z]$. Одна из возможностей заключается в предположении, что $\E[\exp(x'\theta_0)x|z]$ является многочленом меньшего порядка по $z$, в этом случае будет больше условий моментов, чем параметров, и поэтому оценки являются оценками ОММ, а не просто результатом решения (6.25), и необходимо предположить гомоскедастичность ошибок. Если эти дополнительные предположения неверны, тогда оценка по-прежнему остаётся состоятельной при условии выполнения (6.23) и состоятельные стандартные ошибки можно получить с помощью скорректированной формы ковариационной матрицы в (6.16). Часто для простоты применяют $z$, а не $D^*(z,\theta)$ в качестве инструмента.

\begin{center}
Пример оптимальных моментных условий для нелинейной регрессии 
\end{center}

Результат (6.24) полезен в некоторых случаях, особенно в тех, где $z=x$. Здесь мы подтверждаем,
что оценка ОМНК является наиболее эффективной оценкой ОММ, основанной на $\E[u|x]=0$.

Рассмотрим нелинейную регрессионную модель $y=g(x,\beta)+u$. Если отправной точкой является ограничение условного момента $\E[u|x]=0$ или $\E[y-g(x,\beta)|x]=0$, то $z=x$ в (6.23) и (6.24) даёт
\[
D^*(x,\beta)=\E \left[ \frac{\partial}{ \partial \beta} (y- g(x,\beta_0))|x \right] \{ \Var[y-g(x,\beta_0)|x] \}^{-1}=-\frac{\partial g(x,\beta_0)}{\partial \beta} \times \frac{1}{\Var[u|x]},
\]
что требует лишь спецификации $\Var[u|x]$. Из (6.25) оптимальная оценка ОММ является решением соответствующего условия для выборочного момента
\[
\frac{1}{N} \sum_{i=1}^{N} - \frac{\partial g(x_i,\beta)}{\partial \beta} \times \frac{(y_i-g(x_i,\beta))}{{\sigma_i}^2}=0,
\]
где ${\sigma_i}^2=\Var[u_i|x_i]$ функционально независимы от $\beta$. Это условия первого порядка обобщённого НМНК, когда ошибки гетероскедастичны. Реализация возможна с использованием состоятельной оценки ${\hat{\sigma_i}}^2$ для ${\sigma_i}^2$, в таком случае оценка ОММ такая же, как оценка ДОНМНК. Можно получить стандартные ошибки, устойчивые к неправильной спецификации ${\hat{\sigma_i}}^2$, как описано в Разделе 5.8.

Для линейной модели, $g(x,\beta)=x'\beta$, оптимальная оценка ОММ, основанная на $\E[u|x]=0$, --- оценка ОМНК, при этом для случая гомоскедастичных ошибок, оптимальная оценка ОММ, основанная на $\E[u|x]=0$, --- оценка МНК. Как уже видно из примера в Разделе 6.3.6, более эффективные оценки могут быть получены, если  используются дополнительные условия для условных моментов.

\subsection{Тесты на сверх-идентифицирующие ограничения}

Тестирование гипотез для $\theta$ можно выполнить с помощью теста Вальда (см. Раздел 5.5) или с помощью других методов, приведённых в Разделе 7.5.

Кроме того, есть достаточно общий тест на спецификацию модели, который может быть использован для сверх-идентифицированных моделей с количеством моментных условий $(r)$ большим, чем параметров $q$. Тест является одним из тестов близости $N^{-1} \sum_i \hat{h}_i$ к 0, где $\hat{h}_i=h(w_i,\hat{\theta})$. Это естественный тест для гипотезы $H_0$: $\E[h(w,\theta_0)]=0$ об исходных условиях для теоретических моментов. Для точно идентифицированных моделей всегда выполнено условие $N^{-1} \sum_i \hat{h}_i=0$, поэтому невозможно провести тест. Для сверх-идентифицированных моделей, однако, условия первого порядка (6.8) устанавливают, что матрица размера $q \times r$, умноженная на $N^{-1} \sum_i \hat{h}_i$, равна нулю при $q < r$, поэтому $N^{-1} \sum_i \hat{h}_i \not= 0$.

В специальном случае, когда $\theta$ оценивается с помощью $\hat{\theta}_{OGMM}$, определённой в (6.18), Хансен (1982) показал, что статистика сверх-идентифицирующих ограничений (over identifying restrictions, OIR)
\begin{equation}
\text{OIR}= \left( N^{-1} \sum_i \hat{h}_i \right)' {\hat{S}}^{-1} \left( N^{-1} \sum_i \hat{h}_i \right)
\end{equation}
асимптотически распределена по $\chi^2(r-q)$ при $H_0: \E[h(w,\theta_0)]=0$. Обратите внимание, что OIR равна целевой функции ОММ (6.18), оценённой в $\hat{\theta}_{OGMM}$. Если OIR велика, то условия теоретических моментов отвергаются и оценки ОММ несостоятельны для $\theta$.

Априори не очевидно, что квадратичная форма $N^{-1} \sum_i \hat{h}_i$, приведённая в (6.26), распределена по $\chi^2(r-q)$ при $H_0$. Формальное доказательство приведено в следующем разделе, и интуитивное объяснение в случае оценки линейным методом инструментальных переменных приведено в Разделе 8.4.4.

Классическое применение --- модели жизненного цикла для потребления (см. Раздел 6.2.7), в этом случае условия ортогональности являются уравнениями Эйлера. Большое значение тестовой статистики хи-квадрат часто трактуют в пользу того, что гипотеза о жизненном цикле отвергается. Следует, однако, вместо этого более узко интерпретировать данный факт, как отвержение конкретной спецификации функции полезности и набора стохастических допущений, используемых в исследовании.

\subsection{Вывод оценки ОММ}

Выкладки можно упростить путём введения более компактных обозначений. Оценка ОММ минимизирует 
\begin{equation}
\mathcal{Q}_{N}(\theta)=g_N(\theta)'W_N g_N(\theta),
\end{equation}
где $g_N(\theta)=N^{-1} \sum_i h_i(\theta)$. Тогда условие первого порядка для ОММ из (6.8):
\begin{equation}
G_N(\hat{\theta})'W_N g_N(\hat{\theta})=0,
\end{equation}
где $G_N(\theta)= \partial g_N(\theta)/ \partial \theta'= N^{-1} \sum_i \partial h_i(\theta) / \partial \theta'$.

Для состоятельности мы рассматриваем неформальное условие, что предел по вероятности $\partial \mathcal{Q}_{N}(\theta) / \partial \theta|_{\theta_0}$ равен нулю. Из (6.28) оно будет следовать, так как $G_N(\theta_0)$ и $W_N$ имеют конечные пределы по вероятности по предположениям 3 и 4 в утверждении 6.1 и $\plim g_N(\theta_0)=0$ как следствие предположения
5. Более интуитивно $g_N(\theta_0)=N^{-1} \sum_i h_i(\theta)$ имеет предел по вероятности, равный 0, если закон больших чисел может быть применён и $\E[h_i(\theta_0)]=0$, что предполагалось в (6.5).

Параметр $\theta_0$ идентифицируется с помощью ключевого предположения 2 и дополнительно с помощью предположений 3 и 4, согласно которым пределы по вероятности $G_N(\theta_0)$ и $W_N$ должны быть матрицами полного ранга. Предположение, что $G_0=\plim G_N(\theta_0)$ является матрицей полного ранга, называется условием ранга для идентификации. Более слабое необходимое условие для идентификации --- условие порядка, что $r \ge q$.

Для асимптотической нормальности требуется более общая теория, чем для М-оценки, основанной на целевой функции $\mathcal{Q}_{N}(\beta)=N^{-1} \sum_i q(w_i,\theta)$, которая включает в себя только одну сумму. Мы меняем масштаб в (6.28) умножением на $\sqrt{N}$, чтобы
\begin{equation}
G_N(\hat{\theta})'W_N \sqrt{N} g_N(\hat{\theta})=0.
\end{equation}

Подход общей теоремы 5.3 --- взять разложение в ряд Тейлора в окрестности $\theta_0$ всей левой части (6.28). Поскольку $\hat{\theta}$ есть и в первом, и третьем члене, это является сложным и требует наличие первых производных $G_N(\theta)$ и, следовательно, вторых производных $g_N(\theta)$. Так как $G_N(\hat{\theta})$ и $W_N$  имеют конечные пределы по вероятности достаточно просто взять более точное разложение в ряд Тейлора только для $\sqrt{N} g_N(\hat{\theta})$. Мы получаем выражение, аналогичное тому, которое дано в Главе 5 при обсуждении М-оценки, с
\begin{equation}
\sqrt{N} g_N(\hat{\theta})= \sqrt{N} g_N(\theta_0)+G_N(\theta^+) \sqrt{N} (\hat{\theta}-\theta_0),
\end{equation}
где $G_N(\theta)=\partial g_N(\theta) / \partial \theta'$, а $\theta^+$ является точкой между $\theta_0$ и $\hat{\theta}$. Подстановка (6.30) обратно в (6.29) даёт следующее:
\[
G_N(\hat{\theta})' W_N \left[ \sqrt{N} g_N(\theta_0)+G_N(\theta^+) \sqrt{N} (\hat{\theta}-\theta_0) \right]=0.
\]
Выражаем $\sqrt{N} (\hat{\theta}-\theta_0)$:
\begin{equation}
\sqrt{N} (\hat{\theta}-\theta_0)=- \left[ G_N(\hat{\theta})' W_N G_N(\theta^+) \right] ^{-1} G_N(\hat{\theta})' W_N \sqrt{N} g_N(\theta_0).
\end{equation}

Уравнение (6.31) является ключевым результатом для получения предельного распределения оценки ОММ. Мы получаем пределы по вероятности каждого из первых пяти членов, используя $\hat{\theta} \xrightarrow{p} \theta_0$ при состоятельности, в этом случае $\theta^+ \xrightarrow{p} \theta_0$. Последний член в правой части (6.31) имеет предельное нормальное распределение по предположению 5. Таким образом,
\[
\sqrt{N} (\hat{\theta}-\theta_0) \xrightarrow{d} -(G'_0 W_0 G_0)^{-1} G'_0 W_0 \times \mathcal{N}[0,S_0],
\]
где $G_0$, $W_0$ и $S_0$ были определены в утверждении 6.1. Применение теоремы о нормальности предела  произведения (теорема А.17) даёт (6.11).

Для этих преобразований мы  рассматриваем условия первого порядка ОММ как $q$ линейных комбинаций $r$ выборочных моментов $g_N(\hat{\theta})$, так как $G_N(\hat{\theta})' W_N$ --- матрица размера $q \times r$. Оценка ММ является частным случаем при $q=r$, поскольку $G_N(\hat{\theta})' W_N$ является квадратной матрицей полного ранга, поэтому $G_N(\hat{\theta})' W_N g_N(\hat{\theta})=0$ требует, чтобы $g_N(\hat{\theta})=0$.

Для вывода распределения тестовой статистики OIR в (6.26) начнём с разложения первого порядка в ряд Тейлора $\sqrt{N} g_N(\hat{\theta})$ в окрестности $\theta_0$, чтобы получить
\[
\sqrt{N} g_N(\hat{\theta}_{OGMM}) = \sqrt{N} g_N(\theta_0)+G_N(\theta^+) \sqrt{N} (\hat{\theta}_{OGMM}-\theta_0)
\]
\[
= \sqrt{N} g_N(\theta_0)-G_0(G'_0 S^{-1}_0 G_0)^{-1} G'_0 S^{-1}_0 \sqrt{N} g_N(\theta_0) +o_{p}(1)
\]
\[
=[I-M_0 S^{-1}_0] \sqrt{N} g_N(\theta_0) +o_{p}(1),
\]
где второе равенство использует (6.31) с  $W_N$ состоятельной для $S^{-1}_0$, $M_0=G_0(G'_0 S^{-1}_0 G_0)^{-1} G'_0$ и $o_{p}(1)$ определяется в определении А.22. Отсюда следует, что 
\begin{equation}
\begin{matrix}
S^{-1/2}_0 \sqrt{N} g_N(\hat{\theta}_{OGMM})= S^{-1/2}_0 [I-M_0 S^{-1}_0] \sqrt{N} g_N(\theta_0) + o_{p}(1) \\
= [I-S^{-1/2}_0 M_0 S^{-1/2}_0 ] S^{-1/2}_0  \sqrt{N} g_N(\theta_0) + o_{p}(1).
\end{matrix}
\end{equation}
Теперь $[I-S^{-1/2}_0 M_0 S^{-1/2}_0 ] = [I-S^{-1/2}_0  G_0 (G'_0 S^{-1}_0 G_0)^{-1} G'_0 S^{-1/2}_0]$ является идемпотентной матрицей ранга $(r-q)$ и $S^{-1/2}_0 \sqrt{N} g_N(\theta_0) \xrightarrow{d} \mathcal{N} [0,I]$, если $\sqrt{N} g_N(\theta_0) \xrightarrow{d} \mathcal{N} [0,S_0]$. Из стандартных результатов для квадратичных форм нормальных переменных следует, что скалярное произведение
\[
\tau_N = (S^{-1/2}_0 \sqrt{N} g_N(\hat{\theta}_{OGMM}))'(S^{-1/2}_0 \sqrt{N} g_N(\hat{\theta}_{OGMM}))
\]
сходится к распределению $\chi^2(r-q)$.

\section{Линейный метод инструментальных переменных}

Корреляция регрессоров с ошибкой приводит к несостоятельности метода наименьших квадратов. Примерами появления такой корреляции являются пропущенные переменные, одновременность, погрешность измерения регрессоров и смещение выборки. Методы инструментальных переменных обеспечивают общий подход, который может работать с любой из этих проблем при условии существования подходящих инструментов.

Методы инструментальных переменных могут быть применены в рамках ОММ, так как избыток инструментов приводит к избытку моментных условий, которые могут быть использованы для оценивания. Многие результаты метода инструментальных переменных могут быть легче получены в рамках ОММ.

Линейный метод инструментальных переменных является достаточно важным, поэтому к нему мы будем обращаться во многих местах этой книги. Введение было дано в Разделах 4.8 и 4.9. В этом разделе представлен линейный метод инструментальных переменных с одним уравнением как пример конкретного применение ОММ. Для полноты раздел также представляет ранее описанный в литературе особый случай, оценку двухступенчатого метода наименьших квадратов. Линейный метод инструментальных переменных для систем уравнений приведен в Разделе 6.9.5. Тесты на эндогенность и сверх-идентифицирующие ограничения для линейных моделей подробно описаны в Разделе 8.4. Глава 22 представляет оценки линейного метода инструментальных переменных в панельных данных.

\subsection{Линейный ОММ с инструментами}

Рассмотрим модель линейной регрессии
\begin{equation}
y_i = x'_i \beta+ u_i, 
\end{equation}
где каждый элемент $x$ рассматривается как экзогенный регрессор, если он не коррелирует с ошибкой в модели (6.33), или как эндогенный регрессор, если он коррелирует. Если все регрессоры экзогенны, то могут быть использованы оценки МНК, но если хотя бы один регрессор $x$ является эндогенным, то оценки МНК несостоятельны для $\beta$.

Как сказано в Разделе 4.8, состоятельные оценки могут быть получены с помощью метода инструментальных переменных. Ключевым предположением является наличие вектора инструментов $z$ размера $r \times 1$, который удовлетворяет
\begin{equation}
\E[u_i|z_i]=0.
\end{equation}
Экзогенные регрессоры могут быть инструментами для самих себя. Поскольку должно быть по крайней мере столько же инструментов сколько регрессоров, задача состоит в том, чтобы найти дополнительные инструменты, количество которых по меньшей мере равны числу эндогенных переменных в модели. Некоторые примеры таких инструментов были приведены в Разделе 4.8.2.

\begin{center}
Оценки линейного ОММ
\end{center}

Согласно разделу 6.2.5, из ограничения на условный момент (6.34) и модели (6.33) следует ограничение для безусловного момента
\begin{equation}
\E[z_i(y_i-x'_i\beta)]=0,
\end{equation}
где для простоты обозначений мы далее используем обозначение $\beta$, а не более формальное обозначение $\beta_0$ для обозначения истинного значения параметра. Квадратичная форма для соответствующих выборочных моментах приводит к целевой функции ОММ $\mathcal{Q}_{N}(\beta)$, приведённой в (6.4).

В матричном виде определим $y=X \beta+u$, как обычно, и обозначим $Z$ матрицу инструментов размера $N \times R$ с $i$-ой строкой $z'_i$. Тогда $\sum_i z_i(y_i-x'_i\beta)=Z'u$, и (6.4) преобразуется до
\begin{equation}
\mathcal{Q}_{N}(\beta)= \left[ \frac{1}{N} (y-X\beta)'Z \right] W_N \left[ \frac{1}{N} Z'(y-X\beta) \right],
\end{equation}
где $W_N$ --- симметричная матрица весов полного ранга и размера $r \times r$. Основные примеры приведены в конце этого раздела. Условия первого порядка:
\[
\frac{\partial \mathcal{Q}_{N}(\beta)}{\partial \beta}=-2 \left[ \frac{1}{N} X'Z \right] W_N \left[ \frac{1}{N} Z'(y-X \beta) \right]=0
\] 
действительно могут быть решены относительно $\beta$ в этом частном случае ОММ, что приведёт к оценкам ОММ линейной модели инструментальных переменнных
\begin{equation}
\hat{\beta}_{GMM}=[X' Z W_N Z' X]^{-1} X' Z W_N Z' y,
\end{equation}
где деление на $N$ сократилось.

\begin{center}
Распределение линейной оценки ОММ
\end{center}

Общие результаты Раздела 6.3 могут быть использованы для получения асимптотического распределения. Кроме того, поскольку существует явное решение для $\hat{\beta}_{GMM}$, может быть адаптирован анализ для МНК, приведённый в Разделе 4.4. Подстановка $y=X\beta+u$ в ( 6.37 ) даёт
\begin{equation}
\hat{\beta}_{GMM}=\beta+[(N^{-1}X'Z) W_N (N^{-1}Z'X)]^{-1} (N^{-1}X'Z) W_N (N^{-1}Z'u).
\end{equation}

Из последнего члена состоятельность оценки ОММ обязательно требует, чтобы $\plim N^{-1}Z'u=0$. При полностью случайной выборке для этого необходимо условие (6.35), тогда как при других общих схемах проведения выборки (см. Раздел 24.3) требуется более сильное предположение (6.34).

Кроме того, условие ранга для идентифицируемости $\beta$, что $\plim N^{-1}Z'X$ имеет ранг $K$, гарантирует, что правая часть обратима при условии, что $W_N$ имеет полный ранг. Более слабым условием порядка является то, что $r \ge K$.

Предельное распределение основано на выражении для $\sqrt{N} (\hat{\beta}_{GMM}-\beta)$, полученном с помощью простого преобразования (6.38). Получаем асимптотическое нормальное распределение для $\hat{\beta}_{GMM}$ с математическим ожиданием $\beta$ и асимптотической ковариационной матрицей:
\begin{equation}
\widehat{\Var}[\hat{\beta}_{GMM}]=N[X' Z W_N Z'X]^{-1}[X' Z W_N \hat{S} W_N Z'X] [X' Z W_N Z'X]^{-1},
\end{equation}
где $\hat{S}$ является состоятельной оценкой:
\[
S= \lim \frac{1}{N} \sum_{i=1}^{N} \E[u^2_i z_i z'_i],
\]
при обычном в пространственных данных предположении о независимости по $i$. Необходимое дополнительное условие, нужное для (6.39), заключается в том, что $N^{-1/2}Z'u \xrightarrow{d} \mathcal{N}[0,S]$. Результат (6.39) также следует из утверждения 6.1 с $h(\cdot)=z(y-x'\beta)$ и, следовательно, $\partial h / \partial \beta'=-zx'$.

Для пространственных данных с гетероскедастичными ошибками, $S$ может быть состоятельно оценена с помощью:
\begin{equation}
\hat{S}=\frac{1}{N} \sum_{i=1}^{N} {\hat{u}_i}^2 z_i z'_i =Z'DZ/N,
\end{equation}
где $\hat{u}_i=y_i-x'_i \hat{\beta}_{GMM}$ --- остатки ОММ и $D$ --- диагональная матрица размера $N \times N$ с ${\hat{u}_i}^2$ по диагонали. Обычно используется поправка для небольших выборок --- деление на $N-K$, а не на $N$ в формуле $\hat{S}$.

В более простом случае гомоскедастичных ошибок, $\E[{\hat{u}_i}^2|z_i]={\sigma}^2$ и поэтому $S=\lim N^{-1} \sum_i {\sigma}^2 \E[z_i z'_i]$, что приводит к оценке
\begin{equation}
\hat{S}=s^2 Z'Z/N,
\end{equation}
где $s^2=(N-K)^{-1} \sum_{i=1}^{N} {\hat{u}_i}^2$ состоятельна для ${\sigma}^2$. Эти результаты очень похожи на результаты для МНК, представленные в Разделе 4.4.5.

\subsection{Различные оценки линейного ОММ}

Применение результатов Раздела 6.4.1 требует спецификации матрицы весов $W_N$. Для точно идентифицированных моделей любой выбор $W_N$ приводит к одной и той же оценке. Для сверх-идентифицированных моделей есть два распространённых выбора $W_N$, которые приведены в таблице 6.2.

Таблица 6.2 приводит эти оценки и даёт соответствующую специализацию формулы оценки ковариационной матрицы, приведённой в (6.39), считая ошибки независимыми и гетероскедастичными.

\begin{table}[h]
\begin{center}
\caption{\label{tab:GMMest} Оценки ОММ в линейной модели инструментальных переменных и оценки их асимптотических ковариационных матриц}
\begin{minipage}{\textwidth}
\begin{tabular}[t]{ll}
\hline
\hline
\bf{Оценка}\footnote{Уравнения основаны на линейной регрессионной модели с зависимой переменной $y$, регрессорами $X$ и инструментами $Z$. $\hat{S}$ определена в (6.40) и $s^2$ определяется в (6.41). Все оценки ковариационных матриц предполагают независимые и гетероскедастичные ошибки, без упрощения о гомоскедастичных ошибках, приведённого для двухшаговой МНК-оценки. Оптимальный ОММ использует оптимальную матрицу весов.} & \bf{Определение и оценка} \\
& \bf{асимптотической ковариационной матрицы} \\
\hline
ОММ & $\hat{\beta}_{GMM}= [X' Z W_N Z' X]^{-1} X' Z W_N Z' y$ \\
(общая $W_N$) & $\widehat{\Var}[\hat{\beta}]= N [X' Z W_N Z' X]^{-1} [X' Z W_N \hat{S} W_N Z' X] [X' Z W_N Z' X]^{-1}$ \\
Оптимальный ОММ & $\hat{\beta}_{OGMM}= [X' Z {\hat{S}}^{-1} Z' X]^{-1} X' Z {\hat{S}}^{-1} Z' y$ \\
$(W_N={\hat{S}}^{-1})$ & $\widehat{\Var}[\hat{\beta}]=N[X' Z {\hat{S}}^{-1} Z' X]^{-1}$ \\
Двухшаговый МНК & $\hat{\beta}_{2SLS}=[X' Z (Z' Z)^{-1} Z' X ]^{-1} X' Z (Z' Z)^{-1} Z' y$ \\
$(W_N=[N^{-1} Z' Z]^{-1})$ & $\widehat{\Var}[\hat{\beta}]=N[X' Z (Z' Z)^{-1} Z' X] ^{-1} [X' Z (Z' Z)^{-1} \hat{S} (Z' Z)^{-1} Z' X]$ \\ 
& $\times [X' Z (Z' Z)^{-1} Z' X]^{-1} $ \\
& $\widehat{\Var}[\hat{\beta}]=s^2  [X' Z (Z' Z)^{-1} Z' X]^{-1}$ \\
& \text{если ошибки гомоскедастичны}\\
Метод инструментальных & $\hat{\beta}_{IV}=[Z' X]^{-1}Z'y$\\ 
переменных & \\
(точно идентифицирована) & $\widehat{\Var}[\hat{\beta}]=N(Z'X)^{-1} \hat{S} (X' Z)^{-1}$\\
\hline
\hline
\end{tabular}
\end{minipage}
\end{center}
\end{table}

\begin{center}
Оценки метода инструментальных переменных 
\end{center}

В случае точной идентификации $r=K$ и $X'Z$ представляет собой квадратную матрицу, которая обратима. Тогда $[X' Z W_N Z' X]^{-1}=(Z'X)^{-1} {W_N}^{-1} (X'Z)^{-1}$, и (6.37) упрощается до оценки метода инструментальных переменных:
\begin{equation}
\hat{\beta}_{IV}=(Z'X)^{-1} Z'y,
\end{equation}
введённой в Разделе 4.8.6. Для таточно идентифицированных моделей оценки ОММ для любого
выбора $W_N$ равны оценкам метода инструментальных переменных.

Оценка простого метода инструментальных переменных также может быть использована в сверх-идентифицированной модели путём отбрасывания некоторых инструментов, модель при этом становится точно идентифицированной, но это приводит к потере эффективности по сравнению с использованием всех инструментов.

\begin{center}
Оптимальный взвешенный ОММ
\end{center}

Из раздела 6.3.5 для точно идентифицированных моделей наиболее эффективной будет оценка ОММ,  то есть ОММ с оптимальным выбором матрицы весов, $W_N={\hat{S}}^{-1}$ в (6.37).

Оценки оптимального ОММ или двухшагового ОММ в линейной модели метода инструментальных переменных:
\begin{equation}
\hat{\beta}_{OGMM}=[(X'Z){\hat{S}}^{-1} (Z'X)]^{-1} (X'Z){\hat{S}}^{-1} (Z'y).
\end{equation}

Для гетероскедастичных ошибок $\hat{S}$ вычисляется с использованием (6.40), с использованием состоятельной оценки первого шага $\hat{\beta}$ такой, как двухшаговый МНК, определённый в (6.44). Уайт (1982) назвал эту оценку двухшаговой оценкой метода инструментальных переменных, так как оба шага влекут за собой оценки метода инструментальных переменных.

Оценки асимптотический ковариационной матрицы для оптимального ОММ, приведённые в таблице 6.2, имеют относительно простую форму, поскольку (6.39) упрощается, когда $W_N={\hat{S}}^{-1}$. При расчёте оценки ковариационной матрицы можно использовать $\hat{S}$, представленную в таблице 6.2, но более распространённым является использование вместо этого оценки $\tilde{S}$, которая также вычисляется, используя (6.40), но оценивает остатки при оптимальной оценке ОММ, а не оценке первого шага, используемой для формирования $\hat{S}$ в (6.43).

\begin{center}
Двухшаговый метод наименьших квадратов
\end{center}

Если ошибки гомоскедастичны, а не гетероскедастичны, ${\hat{S}}^{-1}=[s^2 N^{-1} Z' Z]^{-1}$ из (6.41). Тогда $W_N=(N^{-1}Z'Z)^{-1}$ в ( 6.37), что приводит к оценке двухшагового метода наименьших квадратов, введённой в разделе 4.8.7, которая может быть выражена более компактно как:
\begin{equation}
\hat{\beta}_{2SLS}=[X' P_{Z} X]^{-1} [X' P_{Z} y],
\end{equation}
где $P_{Z}=Z(ZZ')^{-1}Z'$. Причина названия двухшаговый метод наименьших квадратов представлена в следующем разделе. Оценки двухшагового МНК также называются оценками обобщённого метода инструментальных переменных, поскольку они обобщают оценки метода инструментальных переменных на случай сверх-идентификации при большем числе инструментов, чем регрессоров. Его также называют одношаговым ОММ, потому что (6.44) может быть посчитано в один шаг, тогда как оптимальный ОММ требует двух шагов.

Оценка двухшагового МНК распределена асимптотически нормально с оценкой асимптотической ковариационной матрицы, приведённой в таблице 6.2. Общий вид следует использовать, если мы хотим защититься от гетероскедастичных ошибок, в то время как более простой вид, представленный во многих вводных учебниках, состоятелен, только если ошибки действительно гомоскедастичны.

\begin{center}
Оптимальный ОММ и двухшаговый МНК: сравнение
\end{center}

И оценки оптимального ОММ, и оценки двухшагового МНК приводят к повышению эффективности в сверх-идентифицированных моделях. Оптимальный ОММ имеет то преимущество, что он более эффективен, чем двухшаговый МНК, если ошибки гетероскедастичны, хотя повышение эффективности не должно быть большим. Некоторые из процедур тестирования ОММ приведены в Разделе 7.5, и Глава 8 предполагает оценки c использованием оптимальной матрицы весов. Недостаток оптимального ОММ состоит в том, что он требует дополнительных вычислений по сравнению с двухшаговым МНК. Более того, как обсуждалось в Разделе 6.3.5, асимптотический подход может привести к более плохой аппроксимации для малых выборок оптимальной оценки ОММ.

В пространственных данных обычно используют менее эффективные оценки двухшагового МНК, хотя и с использованием устойчивых к гетероскедастичности стандартных ошибок.

\begin{center}
Ещё более эффективная оценка ОММ
\end{center}

Оценка $\hat{\beta}_{OGMM}$ является наиболее эффективной оценкой, основанной на безусловном моментном уравнении $\E[z_i u_i]=0$, где $u_i=y_i-x'_i\beta$. Однако это не лучшее моментное условие, если отправной точкой является уравнение для условного момента $\E[u_i|z_i]=0$ и ошибки гетероскедастичны, то есть $\Var[u_i|z_i]$ меняется с $z_i$.

Применяя общие результаты Раздела 6.3.7, мы можем написать оптимальное моментное условие для оценки ОММ, основанное на $\E[u_i|z_i]=0$, поскольку
\begin{equation}
\E[\E[x_i|z_i]u_i/ \Var[u_i|z_i]]=0.
\end{equation}
Как и в примере регрессии МНК в Разделе 6.3.7, следует разделить на дисперсию ошибки $\Var[u|z]$. Однако реализация гораздо сложнее, чем в случае МНК, поскольку модель для $\E[x|z]$ необходимо указать в дополнение к модели для $\Var[u|z]$. Для этого требуется указание новой структуры. В частности линейная система одновременных уравнений $\E[x_i|z_i]$ линейна по $z$, поэтому оценивание основывается на $\E[x_i u_i/\Var[u_i|z_i]]=0$.

Для линейных моделей оценки ОММ, как правило, основаны на более простом условии $\E[z_i u_i]=0$. При этом условии оптимальные оценки ОММ, определённые в (6.43), являются наиболее эффективными оценками ОММ.

\subsection{Альтернативные подходы к двухшаговому МНК}

Оценка двухшагового МНК, стандартная оценка метода инструментальных переменных для сверх-идентифицированных моделей, была получена в Разделе 6.4.2 как оценка ОММ.

Здесь мы представляем три других способа получить оценки двухшагового МНК. Один из этих способов, придуманный Тейлом, содержит исходную мотивацию  двухшагового МНК, предшествующую ОММ. Подход Тейла обычно излагается во вводных курсах. Тем не менее, он не обобщается на нелинейные модели, в то время как ОММ подход обобщается.

Рассмотрим линейную модель:
\begin{equation}
y=X \beta+u,
\end{equation}
с $\E[u|Z]=0$ и, кроме того $\Var[u|Z]=\sigma^2 I$.

\begin{center}
ОМНК в преобразованной модели
\end{center}

Умножение (6.46) на инструменты $Z'$ даёт преобразованную модель
\begin{equation}
Z'y=Z'X \beta+Z'u.
\end{equation}
Эта преобразованная модель часто используется в качестве мотивации для оценки метода инструментальных переменных при $r=K$, поскольку опускается $Z'u$, так как $N^{-1} Z'u \rightarrow 0$ и решение даёт $\hat{\beta}=(Z'X)^{-1}Z'y$.

Здесь вместо этого мы рассмотрим сверх-идентифицированный случай. Зависимые от $Z$ ошибки $Z'u$ имеют нулевое математическое ожидание и дисперсию $\sigma^2 Z'Z$ при предположении (6.46). Эффективная оценка ОМНК $\beta$ в модели (6.46) тогда имеет вид:
\begin{equation}
\hat{\beta}=[X' Z (\sigma^2 Z'Z)^{-1} Z' X]^{-1} X' Z (\sigma^2 Z'Z)^{-1} Z' y,
\end{equation}
которая равна оценке двухшагового МНК в (6.44), так как множители $\sigma^2$ сокращаются. В целом, отметим, что если преобразованная модель (6.47) вместо этого оценивается с помощью ВМНК с матрицей весов $W_N$, то более общая оценка (6.37) может быть получена.

\begin{center}
Интерпретация Тейла 
\end{center}

Тейл (1953) предложил получать оценку с помощью регрессии МНК в первоначальной модели (6.46) за исключением того, что регрессоры $X$ заменяются на предсказанные $\hat{X}$, которое асимптотически не коррелирует с остаточным членом.

Предположим, что в приведенной форме модели регрессоры $X$ являются линейной комбинацией инструментов плюс некоторые ошибки, 
\begin{equation}
X = Z \Pi + v,
\end{equation}
где $\Pi$  --- матрица размера $K \times r$. Многомерная регрессия МНК $X$ на $Z$ даёт оценки  $\hat{\Pi}=(Z' Z)^{-1} Z' X$ и прогноз МНК $\hat{X}=Z\hat{\Pi}$ или
\[
\hat{X}= P_{Z} X,
\]
где $P_{Z}=Z(Z' Z)^{-1} Z'$. МНК регрессия $y$ на $\hat{X}$, а не $y$ на $X$ даёт оценку:
\begin{equation}
\hat{\beta}_{Theil}=(\hat{X'}\hat{X})^{-1} \hat{X'}y.
\end{equation}
Интерпретации Тейла даёт возможность строить две регрессии обычным МНК. На первом этапе МНК даёт $\hat{X}$ и на втором шаге МНК даёт $\hat{\beta}$, что и приводит к термину оценка двухшагового метода наименьших квадратов.

Для проверки состоятельности этой оценки преобразуем линейную модель (6.46) как
\[
y= \hat{X} \beta+(X-\hat{X}) \beta + u,
\]

Второй шаг МНК регрессии $y$ на $X$ даёт состоятельную оценку $\beta$, если регрессоры $\hat{X}$ асимптотически не коррелируют с составными ошибками $(X-\hat{X}) \beta + u$. Если $\hat{X}$ --- любые прокси-переменные, то нет никакой причины для отсутствия корреляции, однако здесь $\hat{X}$ не коррелирует с $(X-\hat{X})$, поскольку МНК прогнозы ортогональны остаткам МНК. Таким образом, $\plim N^{-1} \hat{X'}(X-\hat{X})\beta=0$. Кроме того,
\[
N^{-1} \hat{X'} u= N^{-1} X' P_{Z} u = N^{-1} X' Z (N^{-1} Z' Z)^{-1} N^{-1} Z' u.
\]
Тогда $\hat{X}$ асимптотически не коррелирует с $u$ при условии, что $Z$ является хорошим инструментом, поэтому $N^{-1}Z'u=0$. Этот результат состоятельности для $\hat{\beta}_{Theil}$ сильно зависит от линейности модели и не может быть обобщён на нелинейные модели.

Оценки Тейла в (6.50) равны оценкам двухшагового МНК, определённым ранее в (6.44). Мы получаем
\[
\hat{\beta}_{Theil}=(\hat{X'}\hat{X})^{-1} \hat{X'}y
\]
\[
=(X' P'_{Z} P_{Z} X)^{-1} X' P_{Z} y 
\]
\[
=(X' P_{Z} X)^{-1} X' P_{Z} y, 
\]
оценку двухшагового МНК, используя $P'_{Z} P_{Z}=P_{Z}$ в конечном равенстве.

Необходимо проявлять осторожность в реализации двухшагового МНК методом Тейла. На втором этапе МНК ошибки будут иметь неправильные стандартные ошибки, даже если ошибки гомоскедастичны, поскольку будет оцениваться $\sigma^2$ с помощью использования остатков из регрессии МНК на втором шаге $(y-\hat{X} \hat{\beta})$, а не фактических остатков $(y- X \hat{\beta})$. На практике можно также сделать поправку на гетероскедастичность. Гораздо проще использовать программу, которая предлагает двухшаговый МНК в качестве опции и непосредственно вычисляет (6.44) и связанные с ними ковариационные матрицы, приведённые в таблице 6.2.

Интерпретация двухшагового МНК не всегда может быть перенесена на нелинейные модели, как описано в Разделе 6.5.4. Интерпретация ОММ позволяет это сделать, и поэтому на неё здесь делается больший акцент, чем на первоначальный вывод Тейла линейного двухшагового МНК.

Тейл на самом деле рассматривал модель, в которой только некоторые регрессоры $X$ являются эндогенными и все оставшиеся экзогенны. Предшествующий анализ всё ещё может быть применён, если все экзогенные компоненты $X$ включены в инструменты $Z$. Тогда МНК регрессия первого шага экзогенных регрессоров на инструменты даёт нулевые ошибки, и предсказания экзогенных регрессоров равняются их фактическим значениям. Таким образом, на практике на первом шаге только эндогенные переменные регрессируются на инструменты, а на втором шаге строится регрессия $y$ на экзогенные регрессоры и предсказания эндогенных регрессоров из первого шага.

\begin{center}
Интерпретация Басманна
\end{center}

Басманн (1957) предложил использовать в качестве инструментов прогнозы из модели МНК  $\hat{X}=P_{Z} X$  в  точно идентифицированном случае, поскольку тогда есть ровно столько инструментов $\hat{X}$, сколько регрессоров $X$. Мы получаем
\begin{equation}
\hat{\beta}_{Basmann}=(\hat{X'} X)^{-1} \hat{X'} y.
\end{equation}
Эта оценка состоятельна, так как $\plim N^{-1} \hat{X'} u =0$, как уже было показано для оценки Тейла. 

Оценка (6.51) фактически равна оценке двухшагового МНК, определённого в (6.44), так как $\hat{X'}=X' P_{Z}$.

Этот подход инструментальных переменных приведёт к правильным стандартным ошибкам и может быть расширен на нелинейные модели.

\subsection{Альтернативы стандартным оценкам метода инструментальных переменных}

Оценки оптимального ОММ, основанного на методе инструментальных переменных, и оценки двухшагового МНК, представленные в Разделе 6.4.2, --- это стандартные оценки, используемые при эндогенных регрессорах. Черножуков и Хансен (2005) представили оценки метода инструментальных переменных для квантильной регрессии.

Здесь мы кратко обсудим основные альтернативные оценки, которые вновь привлекли внимание, учитывая плохие свойства в конечной выборке двухшагового МНК со слабыми инструментами, подробно описанного в Разделе 4.9. Мы концентрируемся на линейных моделях с одним уравнением. На данном этапе не существует метода, который является относительно эффективным, и имеет небольшое смещение в малых выборках.

\begin{center}
Метод максимального правдоподобия с ограниченной информацией
\end{center}

Оценки метода максимального правдоподобия с ограниченной информацией (Limited Information ML, LIML) получают путём совместной оценки ММП одного уравнения (6.46), и редуцированной формы для эндогенных регрессоров в правой части (6.46) в предположении гомоскедастичных нормально распределённых ошибок. Деталльное обсуждение приводят Грин (2003, с.402) или Дэвидсон и МакКиннон (1993, с.644-651). В более общем случае $k$ класс оценок  (см., например, Грин, 2003, с.403) включает в себя оценки LIML, двухшагового МНК и МНК.

Оценки LIML предложенные Андерсону и Рубину (1949) появились раньше двухшагового МНК. В отличие от оценок двухшагового МНК оценки LIML инвариантны к нормализации, используемой в системе одновременных уравнений. Более того, оценки LIML и двухшагового МНК асимптотически эквивалентны при гомоскедастичных ошибках. Тем не менее, оценки LIML используются редко, поскольку этот метод более трудно осуществить и труднее объяснить, чем двухшаговый МНК. Беккер (1994) представляет результаты LIML для небольших выборок и обобщение LIML. См. также Ган и Хаусман (2002).

\begin{center}
Метод инструментальных переменных с делением выборки
\end{center}

Начнём с интерпретации Басманна оценки двухшагового МНК как оценки метода инструментальных переменных, представленной в (6.51). Подставив $y$ из (6.46), получим
\[
\hat{\beta}=\beta+(\hat{X'} X)^{-1} \hat{X'} u.
\]
По предположению $\plim N^{-1} Z' u=0$, поэтому $\plim  N^{-1} \hat{X'} u=0$ и $\hat{\beta}$ состоятельна. Однако корреляция между $X$ и $u$, причина выбора метода инструментальных переменных, означает, что $\hat{X}=P_{Z} X$ коррелирует с $u$. Таким образом, $\E[\hat{X'}u] \not= 0$, что приводит к смещению инструментальной оценки. Это смещение
возникает из-за использования $\hat{X}=Z \hat{\Pi}$, а не $\hat{X}=Z \Pi$ в качестве инструмента.

Вместо этого  можно использовать в качестве инструментов прогнозы $\tilde{X}$, у которых есть свойство, что $\E[\tilde{X'}u] = 0$ в дополнение к $\plim  N^{-1} \hat{X'} u=0$ и использовать оценку
\[
\tilde{\beta}=(\tilde{X'}X)^{-1} \tilde{X'} y.
\]
Поскольку $\E[\tilde{X'}u]=0$ не означает, что $\E[(\tilde{X'}X)^{-1} \tilde{X'} u]=0$, эта оценка всё равно будет смещённой, но смещение можно уменьшить.

Ангрист и Крюгер (1995) предложили получение таких инструментов путём деления выборки на две подвыборки $(y_1,X_1,Z_1)$ и $(y_2,X_2,Z_2)$. Первая выборка используется для получения оценки  $\hat{\Pi}$ из регрессии $X_1$ на $Z_1$. Вторая выборка используется для получения оценки метода инструментальных переменных, где инструмент $\tilde{X}_2=Z_2 \hat{\Pi}_1$ использует $\hat{\Pi}_1$, полученную из отдельной первой выборки. Ангрист и Крюгер (1995) определяют несмещённую оценку метода инструментальных переменных с делением выборки как:
\[
\tilde{\beta}_{USSIV}=(\tilde{X}_2' X_2)^{-1} \tilde{X}_2' y_2.
\]
Оценки метода инструментальных переменных с делением выборки $\tilde{\beta}_{USSIV}=(\tilde{X}_2' X_2)^{-1} \tilde{X}_2' y_2$ --- вариант, основанный на интерпретации Тейла двухшагового МНК. Эти оценки имеют смещение в конечной выборке в сторону нуля, в отличие от оценки двухшагового МНК, которая смещена по направлению к МНК. Однако есть значительные потери эффективности, поскольку только половина выборки используется на заключительном этапе.

\begin{center}
Джекнайф метод инструментальных переменных 
\end{center}

Более эффективный вариант этой оценки применяет аналогичную процедуру, но генерирует инструменты наблюдение за наблюдением.

Пусть индекс $(-i)$ обозначает выбрасывание одного $i$-го наблюдения. Тогда для $i$-ого наблюдения получаем оценку $\hat{\Pi}_i$ из регрессии $X_{(-i)}$ на $Z_{(-i)}$ и будем использовать в качестве инструмента $\tilde{x}'_i=z'_i \hat{\Pi}_i$. Повтор $N$ раз даёт вектор инструментов, который обозначают как $\tilde{X}_{(-i)}$ с $i$-ой строчкой $\tilde{x}_i'$. Таким образом мы получаем джекнайф оценку  метода инструментальных переменных:
\[
\tilde{\beta}_{JIV}=(\tilde{X'}_{(-i)}X)^{-1} \tilde{X'}_{(-i)} y_2.
\]

Эта оценка была первоначально предложена Филлипсом и Хейлом (1977). Ангрист, Имбенс и Крюгер (1999) и Бломквист и Дальберг (1999) называли её джекнайф оценкой, она (см. Раздел 11.5.5) является методом кросс-валидации с исключением отдельных наблюдений для снижения смещения. Вычисления требующиеся для получения $N$ предсказанных значений $\tilde{x'}_i$ методом джекнайф небольшие при использовании рекурсивной формулы, приведённой в Разделе 11.5.5. Результаты экспериментов Монте-Карло, приведённые в двух последних работах является неоднозначными, они указывают на возможность уменьшения смещения, но и на риск  увеличения дисперсии. Таким образом, метод джекнайф может быть не лучше, чем обычная версия с точки зрения среднеквадратичной ошибки. Предыдущая работа Филлипса и Хейла (1977) представляет аналитические результаты, что смещение в конечной выборке оценки метода инструментальных переменных джекнайф меньше, чем у оценки двухшагового МНК только для сверх-идентифицированных моделей с $r > 2(K+1)$. См. также Хана, Хаусмана и Керштейнера (2001).

\begin{center}
Независимо взвешенный двухшаговый МНК
\end{center}

Методом, близким к методу инструментальных переменных с делением выборки, является независимо взвешенный ОММ Алтоньи и Сегала (1996), приведённый в Разделе 6.3.5. Разделение выборки на $G$ групп и специализация на линейном методе инструментальных переменных даёт оценку независимо взвешенного метода инструментальных переменных:
\[
\hat{\beta}_{IWIV}=\frac{1}{G} \sum_{g=1}^{G} [X'_g Z_g {\hat{S}_{(-g)}}^{-1} Z'_g X_g]^{-1} X'_g Z_g {\hat{S}_{(-g)}}^{-1} Z'_g y_g,
\]
где $\hat{S}_{(-g)}$ вычисляется с использованием $\hat{S}$, определённой в (6.40) за исключением того, что наблюдения из $g$-ой группы исключаются. В применении к панелям Зилиак (1997) обнаружил, что оценка независимо взвешенного метода инструментальных переменных даёт результаты, которые намного лучше, чем несмещённые оценки метода инструментальных переменных с делением выборки.

\section{Нелинейный метод инструментальных переменных}
 
Нелинейные методы инструментальных переменных, в частности нелинейный двухшаговый МНК, предложенный Амэмия (1974), дают состоятельные оценки нелинейных регрессионных моделей в ситуациях, когда оценки НМНК не состоятельны, потому что регрессоры коррелируют с остаточным членом. Мы приводим эти методы, как прямое расширение подхода ОММ для линейных моделей.

В отличие от линейного случая оценки не имеют явной формулы, но асимптотическое распределение может быть получено как частный случай результатов Раздела 6.3. В этом разделе представлены результаты с одним уравнением, а результаты для систем уравнений приведены в Разделе 6.10.4. Фундаментально важным результатом является то, что естественное расширение двухшагового МНК Тейла для линейных моделей на нелинейные модели может привести к несостоятельным оценкам параметров (см. Раздел 6.5.4). Вместо этого должен быть использован подход ОММ.

Альтернативная нелинейность может возникнуть, когда модель для зависимой переменной является линейной моделью, но приведенная форма для эндогенных регрессоров является нелинейной моделью благодаря особенностям зависимой переменной. Например, эндогенный регрессор может быть счётным или бинарным. В этом случае линейные методы предыдущего раздела применимы. Один из подходов заключается в том, чтобы игнорировать особый характер эндогенного регрессора и просто делать регулярный линейный двухшаговый МНК или оптимальный ОММ. Кроме того, получить оценённые значения эндогенных регрессоров с помощью соответствующей нелинейной регрессии такой, как регрессия Пуассона на все инструменты, если эндогенный регрессор является счётным, а затем делать регулярный линейный метод инструментальных переменных с использованием этих оценённых значений в качестве инструмента для счётной переменной, следуя подходу Басманна. Обе оценки состоятельны, хотя они имеют различные асимптотические распределения. Первый более простой подход является более распространённым.

\subsection{Нелинейный ОММ с инструментами}

Рассмотрим достаточно общую нелинейную регрессионную модель, где ошибка может быть аддитивной или неаддитивной (см. Раздел 6.2.2). Таким образом,
\begin{equation}
u_i=r(y_i,x_i,\beta),
\end{equation}
где нелинейная модель с аддитивной ошибкой является частным случаем
\begin{equation}
u_i=y_i-g(x_i,\beta),
\end{equation}
где $g(\cdot)$ является заданной функцией. Оценки, приведённые в Разделе 6.2.2, являются несостоятельными, если $\E[u_i|x_i] \not=0$.

Предположим существование $r$ инструментов $z$, где $r \ge K$, которые удовлетворяют
\begin{equation}
\E[u_i|z_i]=0.
\end{equation}
Это то же самое уравнение для условного момента, как и в линейном случае, за исключением того, что $u_i=r(y_i,x_i,\beta)$, а не $u_i=y_i-x'_i\beta$.

\begin{center}
Оценки нелинейного ОММ
\end{center}

По закону повторного математического ожидания, (6.54) приводит к
\begin{equation}
\E[z_i u_i]=0.
\end{equation}
Оценка ОММ минимизирует квадратичную форму для соответствующих условий выборочного момента.

В матричном виде пусть $u$ обозначает вектор ошибок размера $N \times 1$ с $i$-ым элементом равным $u_i$, приведённом в (6.52) и пусть $Z$ --- матрица инструментов размера $N \times r$ с $i$-ой строчкой $z'_i$. Тогда $\sum_i z_i u_i =Z'u$ и оценка ОММ в нелинейной модели метода инструментальных переменных $\hat{\beta}_{GMM}$ минимизирует
\begin{equation}
\mathcal{Q}_{N}(\beta)= \left( \frac{1}{N} u'Z \right) W_N \left( \frac{1}{N} Z'u \right),
\end{equation}
где $W_N$ является матрицей весов размера $r \times r$. В отличие от линейного ОММ условия первого порядка не приводят к явному решению для $\hat{\beta}_{GMM}$.

\begin{center}
Распределение оценок нелинейного ОММ
\end{center}

Оценка ОММ состоятельна для $\beta$, заданного в (6.54), и асимптотически нормально распределена с оценкой асимптотической ковариационной матрицы:
\begin{equation}
\widehat{\Var}[\hat{\beta}_{GMM}]=N[\hat{D'} Z W_N Z' \hat{D}]^{-1} [\hat{D'} Z W_N \hat{S}_N  W_N Z' \hat{D}] [\hat{D'} Z W_N Z' \hat{D}]^{-1} 
\end{equation}
Здесь мы использовали результаты из Раздела 6.3.3 с $h(\cdot)=zu$, где $\hat{S}$ приведена далее и $\hat{D}$ является матрицей производных случайной ошибки размера $N \times K$ 
\begin{equation}
\hat{D}= \left. \frac{\partial u}{\partial \beta'} \right|_{\hat{\beta}_{GMM}}.
\end{equation}
С неаддитивными ошибками  $i$-ая строка матрицы $\hat{D}$ имеет вид $\partial r(y_i,x_i,\beta)/ \partial \beta'|_{\hat{\beta}}$. С аддитивными ошибками $i$-ая строка имеет вид  $\partial g(x_i,\beta)/ \partial \beta'|_{\hat{\beta}}$, знак минуса сокращается в (6.57).

Для независимых гетероскедастичность ошибок
\begin{equation}
\hat{S}=N^{-1} \sum_i {\hat{u}_i}^2 z_i z'_i,
\end{equation}
что похоже на линейный случай, только теперь $\hat{u}_i=r(y_i,x_i,\hat{\beta})$ или $\hat{u}_i=y_i-g(x,\hat{\beta})$.

Поэтому оценка асимптотической ковариационной матрицы оценки ОММ в нелинейной модели такая же, как и в линейном случае, приведённом в (6.39), с отличием в том, что матрица регрессоров $X$ заменяется на производную $\partial u / \partial \beta'|_{\hat{\beta}}$. Это точно такое же изменение, как в Разделе 5.8 при переходе от линейного к нелинейному методу наименьших квадратов. По аналогии с линейным методом инструментальных переменных условие рангов для идентификации состоит в том, что матрица $\plim N^{-1} Z' \partial u / \partial \beta'|_{\beta_0}$ имеет ранг $K$ и более слабое условие порядка --- $r \ge K$.

\subsection{Различные оценки нелинейных ОММ}

Две основных оценки ОММ, отличающиеся выбором матрицы весов, --- это оптимальный ОММ с $W_N={\hat{S}}^{-1}$, и оценка нелинейного двухшагового МНК с $W_N=(Z'Z)^{-1}$. В Таблице 6.3 приведены эти оценки и связанные с ними оценки ковариационных матриц, в предположении независимых гетероскедастичных ошибок, и приведены результаты для общей $W_N$ и результаты для нелинейного метода инструментальных переменных в точно идентифицированной модели.

\begin{table}[h]
\begin{center}
\caption{\label{tab:GMMestnl} Оценки ОММ в нелинейной модели инструментальных переменных и оценки их асимптотических ковариационных матриц}
\begin{minipage}{\textwidth}
\begin{tabular}[t]{ll}
\hline
\hline
\bf{Оценка}\footnote{Уравнения основаны на нелинейной регрессионной модели c ошибкой $u$, определённой в (6.53) или (6.52), и инструментами $Z$. $\hat{D}$ --- производная вектора ошибок по $\beta'$ в точке $\hat{\beta}$ и может быть упрощена для моделей с аддитивными ошибками до вида производной функции условного математического ожидания по $\beta'$ в точке $\hat{\beta}$. $\hat{S}$ определена в (6.59). Все оценки ковариационных матриц построены для ошибок, которые являются независимыми и гетероскедастичными, кроме упрощения до гомоскедастичных ошибках для оценки нелинейного двухшагового МНК.} & \bf{Определение и оценка} \\
& \bf{асимптотической ковариационной матрицы} \\
\hline
ОММ & $\mathcal{Q}_{GMM}(\beta)= u' Z W_N Z' u$ \\
(общая $W_N$) & $\widehat{\Var}[\hat{\beta}]= N [\hat{D'} Z W_N Z' \hat{D}]^{-1} [\hat{D'} Z W_N \hat{S} W_N Z' \hat{D}] [\hat{D'} Z W_N Z' \hat{D'}]^{-1}$ \\
Оптимальный ОММ & $\mathcal{Q}_{OGMM}(\beta)= u' Z {\hat{S}}^{-1} Z' u $ \\
$(W_N={\hat{S}}^{-1})$ & $\widehat{\Var}[\hat{\beta}]=N[\hat{D'} Z {\hat{S}}^{-1} Z' \hat{D}]^{-1}$ \\
Нелинейный двухшаговый МНК & $\mathcal{Q}_{NL2SLS}(\beta)=u' Z (Z' Z)^{-1} Z' u $ \\
$(W_N=[N^{-1} Z' Z]^{-1})$ & $\widehat{\Var}[\hat{\beta}]=N[\hat{D'} Z (Z' Z)^{-1} Z' \hat{D}] ^{-1} [\hat{D'} Z (Z' Z)^{-1} \hat{S} (Z' Z)^{-1} Z' \hat{D}]$ \\ 
& $\times [\hat{D'} Z (Z' Z)^{-1} Z' \hat{D}]^{-1} $ \\
& $\widehat{\Var}[\hat{\beta}]=s^2  [\hat{D'} Z (Z' Z)^{-1} Z' \hat{D}]^{-1}$ \\
& \text{если ошибки гомоскедастичны}\\
Нелинейный метод & $\hat{\beta}_{NLIV} \text{ решает} Z' u=0$\\ 
инструментальных переменных & \\
(точно идентифицирована) & $\widehat{\Var}[\hat{\beta}]=N(Z'\hat{D})^{-1} \hat{S} (\hat{D'} Z)^{-1}$\\
\hline
\hline
\end{tabular}
\end{minipage}
\end{center}
\end{table}

\begin{center}
Нелинейный метод инструментальных переменных
\end{center}

В точно идентифицированном случае можно напрямую использовать условия для выборочных моментов, соответствующие (6.55). Это даёт оценку метода моментов в нелинейной
модели метода инструментальных переменных $\hat{\beta}_{NLIV}$, которая является решением:
\begin{equation}
\frac{1}{N} \sum_{i=1}^{N} z_i u_i=0,
\end{equation}
или $Z'u=0$ с оценкой асимптотической ковариационной матрицы, приведённой в таблице 6.3.

Нелинейные оценки часто рассчитываются с использованием итерационных методов, которые находят оптимум целевой функции, а не решают нелинейные системы оценочных уравнений. Для точно идентифицированного случая $\hat{\beta}_{NLIV}$ может быть вычислена как оценка ОММ при минимизации (6.56) при любой матрице весов, наиболее просто случай $W_N=I$, что приводит к
такой же оценке.

\begin{center}
Оптимальный нелинейный ОММ
\end{center}

Для сверх-идентифицированных моделей оценка оптимального ОММ использует матрицу весов $W_N={\hat{S}}^{-1}$. Оценка оптимального ОММ в нелинейной модели метода инструментальных переменных $\hat{\beta}_{OGMM}$ минимизирует
\begin{equation}
\mathcal{Q}_{N}(\beta)= \left( \frac{1}{N} u' Z \right) {\hat{S}}^{-1} \left( \frac{1}{N} Z' u \right).
\end{equation}

Оценка асимптотической ковариационной матрицы, приведённая в таблице 6.3, имеет относительно простую форму, так как (6.57) упрощается, когда $W_N={\hat{S}}^{-1}$.

Как и в линейном случае оценка оптимального ОММ представляет собой двухшаговую оценку, когда ошибки гетероскедастичны. При расчёте оценки ковариационной матрицы можно использовать $\hat{S}$, представленную в таблице 6.3, но более распространено использование вместо этого оценку $\tilde{S}$, которая также рассчитывается с использованием (6.59), но оценивает остатки при оптимальной оценке ОММ, а не оценке первого шага, которая используется для формирования $\hat{S}$ в (6.61).

\begin{center}
Нелинейный двухшаговый МНК
\end{center}

Частный случай оценки ОММ с инструментами использует $W_N=(N^{-1}Z'Z)^{-1}$ в
(6.56). Это даёт оценку нелинейного двухшагового метода наименьших квадратов $\hat{\beta}_{NL2SLS}$, которая минимизирует
\begin{equation}
\mathcal{Q}_{N}(\beta)= \frac{1}{N} u'Z(Z'Z)^{-1}Z'u.
\end{equation}
Данная оценка является оценкой оптимального ОММ, если ошибки гомоскедастичны, поскольку тогда $\hat{S}=s^2 Z'Z /N$, где $s^2$ --- состоятельная оценка константы $\Var[u|z]$, поэтому матрица ${\hat{S}}^{-1}$ пропорциональна $(Z'Z)^{-1}$.

С гомоскедастичными ошибками эта оценка имеет более простую оценку асимптотической ковариационной матрицы, указанную в таблице 6.3. Данный результат  часто приводится в учебниках. Тем не менее, в микроэконометрических приложениях часто допускают гетероскедастичные ошибки и используют более сложные скорректированные оценки, также приведённые в таблице 6.3.

Оценка нелинейного двухшагового МНК, предложенная Амэмия (1974), была важным предшественником оценки ОММ. Идея построения оценки аналогична идее построения линейного двухшагового МНК, приведённого в Разделе 6.4.3. Таким образом, мы умножаем ошибки модели $u$ на инструменты $Z'$ для получения $Z'u$, где $\E[Z'u]=0$, так как $\E[u|Z]=0$. После этого мы оцениваем нелинейную регрессию ОМНК. При гомоскедастичных ошибках она минимизирует 
\[
\mathcal{Q}_{N}(\beta)= u'Z[\sigma^2 Z'Z]^{-1}Z'u,
\]
так из $\Var[u|Z]=\sigma^2 I$ следует $\Var[Z'u|Z]=\sigma^2 Z'Z$. Эта целевая функция пропорциональна выражению (6.62).

Интерпретация Тейла линейного двухшагового метода МНК не всегда может быть перенесена на нелинейные модели (см. Раздел 6.5.4). Кроме того, оценка нелинейного двухшагового МНК является  оценкой одного шага. Амэмия выбрал термин оценка нелинейного двухшагового МНК, потому что, как и в линейном случае, она даёт состоятельные оценки с использованием инструментальных переменных. Название не следует понимать буквально, и более точным термином было бы --- оценка нелинейного метода инструментальных переменных или оценка обобщённого нелинейного метода инструментальных переменных.

\begin{center}
Выбор инструментов в нелинейных моделях
\end{center}

Предыдущие оценки предполагают существование инструментов таких, что $\E[u|z]=0$, и оценивание наиболее эффективно, если оно основано на уравнениях для  безусловного момента $\E[zu]=0$. Рассмотрим нелинейную модель с аддитивными ошибками, где $u=y-g(x,\beta)$. Чтобы быть релевантным, инструмент должен быть коррелирован с регрессорами $x$, тем не менее, чтобы быть годным он не может быть прямой причиной изменений $y$. Из ковариационной матрицы, приведённой в (6.57), имеет значение на самом деле корреляция $z$ с $\partial g/\partial \beta$, а не столько с $x$. Данная корреляция нужна для того, чтобы величина $\hat{D'}Z$ была большой. Сомнения по поводу слабых переменных здесь очень уместны, как и в линейном случае, изученном в Разделе 4.9.

Учитывая очень вероятную гетероскедастичность наилучшее моментное условие, на котором следует строить оценку, при условии что $\E[u|z]=0$, это не $\E[zu]=0$. Из Раздела 6.3.7, однако, оптимальное моментное условие требует дополнительных предположений о моментах, которые трудно сделать. Стандартно используют $\E[zu]=0$, как это было сделали мы.

Альтернативным способом учесть гетероскедастичность будет построение ОММ оценки на ошибке, близкой к гомоскедастичной. Например, со счётными данными, вместо $u=y-\exp(x'\beta)$, лучше использовать стандартизированные ошибки $u^*=u/\sqrt{\exp(x'\beta)}$ (см. Раздел 6.2.2). Однако следует отметить, что $\E[u^*|z]=0$ и $\E[u|z]=0$ --- это разные предположения.

Часто только одна компонента $x$ коррелирует с $u$. Тогда, как и в линейном случае, экзогенные компоненты могут быть использованы в качестве инструментов для самих себя, и задача состоит в том, чтобы найти дополнительный инструмент, который не коррелирует с $u$. Есть некоторые нелинейные приложения, которые возникают из формальных экономических моделей, как в Разделе 6.2.7, и в этом случае многие имеющиеся переменные доступны в качестве инструментов.

\subsection{Пример инструментальных переменных для распределения Пуассона}

Регрессионная модель Пуассона с экзогенными регрессорами специфицирует $\E[y|x]=\exp(x'\beta)$. Это уравнение можно рассматривать как модель с аддитивными ошибками $u=y-\exp(x'\beta)$. Если регрессоры являются эндогенными, то $\E[u|x] \not=0$ и оценки ММП Пуассона тогда будут несостоятельными. Состоятельная оценка предполагает наличие инструментов $z$, которые удовлетворяют условию $\E[u|z]=0$ или, что эквивалентно,
\[
\E[y-\exp(x'\beta)|z]=0.
\]

Предыдущие результаты могут быть применены напрямую. Целевая функция:
\[
\mathcal{Q}_{N}(\beta)= \left[ N^{-1} \sum_i z_i u_i \right]' W_N \left[ N^{-1} \sum_i z_i u_i \right],
\]
где $u_i=y_i-\exp(x'_i\beta)$. Тогда условия первого порядка:
\[
\left[ \sum_i \exp(x'_i \beta) x_i z'_i \right]' W_N \left[ \sum_i z_i (y_i -\exp(x'_i \beta))\right]=0.
\]

Асимптотическое распределение приведено в таблице 6.3, с $\hat{D}'Z = \sum_i \exp(x_i'\hat{\beta}) x_i z_i'$, так как $\partial{g}/\partial{\beta} = \exp(x'\beta)x$, а также $\hat{S}$ определено в (6.39) с $\hat{u}_i = y_i - \exp(x_i'\hat{\beta})$. Оптимальные оценки ОММ и оценки нелинейного двухшагового МНК отличаются матрицами весов $\hat{S}^{-1}$ или $(N^{-1}Z'Z)^{-1}$, где $Z'Z = \sum_i z_iz_i'$.

Другую состоятельную оценку можно получить с помощью подхода Басманна. Во-первых, с помощью МНК необходимо оценить сокращённую форму $x_i = \Pi z_i + v_i$, что даёт $K$ предсказаний $\hat{x}_i = \hat{\Pi}z_i$. Во-вторых, оценим с помощью нелинейного метода инструментальных переменных, как в (6.60) с инструментами $\hat{x}_i$ вместо $z_i$. Учитывая формулу МНК для $\hat{\Pi}$, эта оценка является решением следующего равенства: 

\[
\left[ \sum_i x_iz_i' \right] \left[ \sum_i z_iz_i' \right]^{-1} \left[ \sum_i (y_i - exp(x_i'\beta))z_i \right] = 0.
\]

Эта оценка отличается от оценки нелинейного двухшагового МНК, потому что первое слагаемое в левой части другое. Потенциальные проблемы, связанные с использованием здесь обобщения метода Тейла для линейной модели, подробно описаны в следующем разделе.

Аналогичные вопросы возникают и других в нелинейных моделях, например, в моделях для бинарных данных.

\subsection{Двухшаговое оценивание нелинейных моделей}

Обычная интерпретация линейного двухшагового МНК может не сработать в нелинейных моделях. Итак, пусть у $y$ математическое ожидание равно $g(x,\beta)$ и есть инструменты $z$ для регрессоров $x$. Тогда МНК регрессия $x$ на инструменты $z$, чтобы получить оценённые значения $\hat{x}$ с последующей регрессии НМНК $y$ на $g(\hat{x},\beta)$, может привести к несостоятельным оценкам параметра $\beta$, что мы сейчас покажем. Вместо этого, нужно использовать оценки нелинейного двухшагового МНК, представленного в предыдущем разделе.

Рассмотрим простую модель, основанную на модели, представленной Амэмия (1984), то есть нелинейную по переменным, но линейную по параметрам. Пусть
\begin{equation}
\begin{split}
y=\beta x^2 +u, \\
x=\pi z + \upsilon,\\
\end{split}
\end{equation}
где ошибки $u$ и $\upsilon$ имеют нулевое математическое ожиданием и коррелируют. Регрессоры $x^2$ эндогенны, так как $x$ является функцией от $\upsilon$ и по предположению $u$ и $\upsilon$ коррелируют. В результате оценка МНК $\beta$ несостоятельна. Если $z$ генерируется независимо от других случайных переменных в модели, то он будет годным инструментом, поскольку тогда ясно, что он не зависит от $u$, но коррелирует с $x$. 

Оценка метода инструментальных переменных $\hat{\beta}_{IV}= \left( \sum_i z_i x^2_i  \right)^{-1} \sum_i z_i y_i$. Эту оценку можно реализовать с помощью обычной регрессии метода инструментальных переменных $y$ на $x^2$ с инструментами $z$. Немного алгебраических преобразований, и как и ожидалось, $\hat{\beta}_{IV}$ равна оценке нелинейного метода инструментальных переменных, определённой в (6.60).

Предположим, вместо этого мы выполняем следующее двухшаговое оценивание наименьших квадратов. Во-первых, построим регрессию $x$ на $z$, чтобы получить $\hat{x}=\hat{\pi}z$, а затем построим регрессию $y$ на ${\hat{x}}^2$. Тогда $\hat{\beta}_{2SLS}= \left( \sum_i {\hat{x}_i}^2 {\hat{x}_i}^2 \right)^{-1} \sum_i {\hat{x}_i}^2 y_i$, где ${\hat{x}_i}^2$ --- квадрат прогноза $\hat{x}_i$, полученный из МНК регрессии $x$ на $z$. Мы получаем несостоятельные оценки. Адаптируя доказательство для линейного случая в Разделе 6.4.3, получаем:
\[
y_i=\beta x^2_i + u_i
\]
\[
=\beta {\hat{x_i}}^2 + w_i
\]
где $w_i=\beta({x_i}^2-{\hat{x_i}}^2)+u_i$. МНК регрессия $y_i$ на ${\hat{x_i}}^2$ несостоятельна для $\beta$, потому что регрессоры ${\hat{x_i}}^2$ асимптотически коррелируют с составной ошибкой $w_i$. Из $(x^2_i-{\hat{x_i}}^2)=(\pi z_i + \upsilon_i)^2-(\hat{\pi} z_i )^2=\pi^2 z^2_i + 2 \pi z_i \upsilon_i + \upsilon^2_i - {\hat{\pi}}^2{z_i}^2$, используя тот факт, что $\plim \hat{\pi} =\pi$, после некоторых преобразований мы получаем $\plim N^{-1} \sum_i {\hat{x_i}}^2 (x^2_i-{\hat{x_i}}^2) = \plim N^{-1} \sum_i {\pi}^2 z^2_i \upsilon^2_i \not =0$, даже если $z_i$ и $\upsilon_i$ независимы. Следовательно, $\plim N^{-1} \sum_i {\hat{x_i}}^2 w_i \not=  \plim N^{-1} \sum_i {\hat{x_i}}^2 \beta (x_i-\hat{x_i})^2=0$.

Вариант, который даёт состоятельную оценку, однако, заключается в построении регрессии $x^2$, а не $x$ на $z$ на первом шаге и использовании прогноза $\hat{x}^2 \not = (\hat{x})^2$ на втором этапе. Можно показать, что мы получим $\hat{\beta}_{IV}$. Инструмент для $x^2$ должен быть прогнозом значения $x^2$, а не квадратом прогноза $x$.

Данный пример обобщается на другие нелинейные модели, где нелинейность присутствует только по регрессорам, то есть на
\[
y=g(x)' \beta+u,
\]
где $g(x)$ --- нелинейная функция от $x$. Типичным примером является использование степенной функции и натурального  логарифма. Пусть $\E[u|z]=0$. Несостоятельные оценки получают путём построения регрессии $x$ на $z$, для получения прогнозов $\hat{x}$, а затем построения регрессии $y$ на $g(\hat{x})$. Состоятельные оценки могут быть получены путём регрессии $g(x)$ на $z$, для получения прогнозов $\hat{g}(x)$, а затем регрессии $y$ на $\hat{g}(x)$ на втором этапе. Мы используем $\hat{g}(x)$, а не $g(\hat{x})$ как инструмент для $g(x)$. Даже тогда регрессия на втором этапе даёт неверные стандартные ошибки, поскольку МНК будет использовать остатки $\hat{u}=y-\hat{g}(x)'\hat{\beta}$, а не $\hat{u}=y-g(x)'\hat{\beta}$. Лучше всего сразу использовать ОММ или нелинейный двухшаговый МНК.

В более общем случае модели могут быть нелинейными и по переменным, и по параметрам. Рассмотрим одноиндексную модель с аддитивной ошибкой
\[
y=g(x'\beta)+u.
\]
Несостоятельные оценки могут быть получены с помощью регрессии МНК $x$ на $z$, для получения прогнозов $\hat{x}$, а затем оценивания НМНК регрессии $y$ на $g(\hat{x'}\beta)$. Нужно использовать либо ОММ, либо нелинейный двухшаговый МНК. По сути, для обеспечения состоятельности необходимо использовать $\hat{g}(x'\beta)$, не $g(\hat{x'}\beta)$.

\begin{center}
Пример нелинейного двухшагового МНК
\end{center}

Мы оцениваем с помощью нелинейного двухшагового МНК модель с простой нелинейностью. Квадрат эндогенной переменной выступает в роли регрессора, как и в предыдущем разделе.

Процесс, генерирующий данные, --- (6.63), поэтому $y=\beta x^2 + u$ и $x=\pi z + \upsilon$, где $\beta=1$, $\pi=1$, и $z=1$ для всех наблюдений. Вектор $(u,\upsilon)$ является совместно нормально распределёнными с математическим ожиданием $0$, дисперсией $1$ и корреляцией $0.8$. Берётся выборка размера 200. Результаты приведены в таблице 6.4.

\begin{table}[h]
\begin{center}
\caption{\label{tab:GMMestnlex} Пример нелинейного двухступенчатого метода наименьших квадратов}
\begin{minipage}{12.5cm}
\begin{tabular}[t]{cccc}
\hline
\hline
& \multicolumn{3}{c}{\bf{Оценка}} \\
\hline
\bf{Переменная}\footnote{Процесс, генерирующий данные, приведённый в тексте, имеет истинный коэффициент, равный единице. Размер выборки $N=200$.} & \bf{МНК} & \bf{Нелинейного} & \bf{Двушагового} \\
 & & \bf{двухшагового МНК} & \bf{метода} \\
\hline
$x^2$ & 1.189 & 0.960 & 1.642 \\
 & (0.025) & (0.046) & (0.172) \\
$R^2$ & 0.88 & 0.85 & 0.80 \\
\hline
\hline
\end{tabular}
\end{minipage}
\end{center}
\end{table}

Нелинейность здесь довольно мягкая, в роли регрессора выступает квадрат $x$, а не сам $x$. Цель заключается в оценке коэффициента $\beta$ при квадрате $x$. МНК-оценка несостоятельная, в то время как оценка нелинейного двухшагового МНК состоятельна. Двухшаговый метод, где сначала МНК регрессия $x$ на $z$ используется для формирования $\hat{x}$, а затем оценивается МНК-регрессия $y$ на $(\hat{x})^2$, даёт оценку, которая более чем на две стандартных ошибки отличается от истинного значения $\beta=1$. Симуляции также указывают на потерю в качестве подгонки и точности с большими стандартными ошибками и более низким $R^2$, как и в случае линейного метода инструментальных переменных.

\section{Последовательная двухшаговая М-оценка}

В процедуре последовательного двухступенчатого оценивания финальная оценка целевого параметра основана на первоначальной оценке неизвестного параметра. В качестве примера может послужить ДОМНК, где ошибка имеет условную дисперсию $\exp(z'\gamma)$. При наличии оценки $\tilde{\gamma}$ для $\gamma$, оценка ДОМНК $\hat{\beta}$ является решением $\sum_{i=1}^{N}(y_i-x'_i\hat{\beta})^2/\exp(z'_i \tilde{\gamma})$. Вторым примером является двухступенчатый оценка Хекмана, приведённая в Разделе 16.10.2.

Эти оценки привлекательны тем, что они могут обеспечить относительно простой способ получения состоятельных оценок параметров. Однако для правильных статистических выводов может потребоваться изменить оценку асимптотической ковариационной матрицы второго шага для того, чтобы учесть наличие первого шага. Мы представляем результаты для специального случая, когда оценочные уравнения для оценок обоих первого и второго шага приравнивают выборочное среднее к нулю, как это имеет место для М-оценок, оценок метода моментов и оценок методом оценочных уравнений.

Разделим вектор параметров $\theta$ на две части $\theta_1$ и $\theta_2$, основная цель заключается в оценивании $\theta_2$. Модель оценивается последовательно с получением сначала $\hat{\theta}_1$, которая является решением $\sum_{i=1}^{N} h_{1i}(\hat{\theta}_1)=0$, и тогда при заданной $\hat{\theta}_1$ получаем $\hat{\theta}_2$, которая является решением $N^{-1} \sum_{i=1}^{N} h_{2i}(\hat{\theta}_1,\hat{\theta}_2)=0$. В целом распределение $\hat{\theta}_2$, учитывая использование оценки $\hat{\theta}_1$, отличается от распределения $\hat{\theta}_2$ и сложнее, чем распределение $\hat{\theta}_2$, если $\theta_1$ известно. Статистический вывод является неправильным, если он не принимает во внимание это осложнение за исключением некоторых особых случаев, приведённых в конце этого раздела.

Следующий вывод приведён в книге Ньюи (1984) с аналогичными результатами, полученными Мёрфи и Топелем (1985) и Паганом (1986). Двухшаговая оценка может быть переписана в виде одношаговой оценки, где $(\theta_1,\theta_2)$  являются решениями уравнений:
\begin{equation}
\begin{split}
N^{-1} \sum_{i=1}^{N} h_1(w_i,\hat{\theta}_1)=0, \\
N^{-1} \sum_{i=1}^{N} h_2(w_i,\hat{\theta}_1,\hat{\theta}_2)=0.\\
\end{split}
\end{equation}
Определив $\theta=(\begin{matrix} \theta'_1 & \theta'_2 \end{matrix})'$ и $h_i=(\begin{matrix} h'_{1i} & h'_{2i} \end{matrix})'$, можно записать уравнения как:
\[
N^{-1} \sum_{i=1}^{N} h(w_i,\hat{\theta})=0.
\]
В этой постановке предполагается, что $\dim(h_1)=\dim(\theta_1)$ и $\dim(h_2)=\dim(\theta_2)$, чтобы количество оценочных уравнений было равно числу параметров. Тогда (6.64) является оценкой оценочных уравнений или оценкой ММ.

Для состоятельности необходимо условие  $\plim N^{-1} \sum_i h(w_i,\theta_0)=0$, где $\theta_0=[\theta^1_{10},\theta^1_{20}]$. Это условие будет выполнено, если $\hat{\theta}_1$ состоятельна для $\theta_{10}$ на первом шаге, и если оценка $\hat{\theta}_2$ второго шага с известной $\theta_{10}$ (а не оценённой с помощью $\hat{\theta}_1$) приведёт к состоятельной оценке $\theta_{20}$. В рамках метода моментов мы требуем $\E[h_{1i}(\theta_1)]=0$ и $\E[h_{2i}(\theta_1,\theta_2)]=0$. Мы предполагаем, что состоятельность установлена.

Для асимптотического распределения мы применяем общий результат, что 
\[
\sqrt{N} (\hat{\theta}-\theta_0) \xrightarrow{d} \mathcal{N}[0,G^{-1}_0 S_0 (G^{-1}_0)'],
\] 
где $G_0$ и $S_0$ определены в утверждении 6.1. Разбиение $G_0$ и $S_0$ происходит аналогичным образом, как и разбиение $\theta$ и $h_i$. Тогда
\[
G_0= \lim \frac{1}{N} \sum_{i=1}^{N} \E \begin{bmatrix} \partial h_{1i} / \partial \theta'_1 & 0 \\ \partial h_{2i} / \partial \theta'_1 & \partial h_{2i} / \partial \theta'_2
\end{bmatrix} = \begin{bmatrix} G_{11} & 0 \\  G_{21} &  G_{22} \end{bmatrix},
\]
здесь $\partial h_{1i} / \partial \theta'_2 =0$, так как $h_{1i}(\theta)$ не является функцией от $\theta_2$ из (6.64). Поскольку $G_0$, $G_{11}$, $G_{22}$--- квадратные матрицы
\[
G^{-1}_0 = \begin{bmatrix} G^{-1}_{11} & 0 \\  -G^{-1}_{22} G_{21} G^{-1}_{11} &  G^{-1}_{22} \end{bmatrix}.
\]
И ежy понятно,
\[
S_0= \lim \frac{1}{N} \sum_{i=1}^{N} \E \begin{bmatrix} h_{1i} h_{1i}' &  h_{1i} h_{2i}' \\ h_{2i} h_{1i}' &  h_{2i} h_{2i}' \end{bmatrix}= \begin{bmatrix} S_{11} & S_{12} \\ S_{21} & S_{22} \end{bmatrix}.
\]
Асимптотической ковариационной матрицей $\hat{\theta}_2$ является подматрица  ковариационной матрицы $\hat{\theta}$. После некоторых алгебраических преобразований мы получаем:
\begin{equation}
\Var[\hat{\theta}_2]= G^{-1}_{22} \left\{ \begin{matrix} S_{22} + G_{21} [G^{-1}_{11} S_{11} G^{-1}_{11}] G'_{21} \\ -G_{21} G^{-1}_{11} G_{12} - S_{21} G^{-1}_{11} G'_{21} \end{matrix} \right\} G^{-1}_{22}.
\end{equation}

Обычный компьютерный алгоритм выдаёт неправильные стандартные ошибки, которые  занижают истинные стандартные ошибки, так как $\Var[\hat{\theta}_2]$  предполагается равной $G^{-1}_{22} S_{22} G^{-1}_{22}$, и можно показать, что она меньше истинной дисперсии, приведённой в (6.65).

Нет необходимости учитывать дополнительную изменчивости на втором шаге, вызванную использованием оценки на первом этапе в особом случае, если $\E[\partial h_{2i} (\theta) / \partial \theta_1]=0$, поскольку тогда $G_{21}=0$ и $\Var[\hat{\theta}_2]$ в (6.65) сводится к $G^{-1}_{22} S_{22} G^{-1}_{22}$.

Известным примером с $G_{21}=0$ является ДОМНК. Тогда для гетероскедастичных ошибок
\[
h_{2i}(\theta)= \frac{x_{2i}(y_i-x'_i \theta_2)}{\sigma(x_i, \theta_1)},
\]
где $\Var[y_i|x_i]=\sigma^2(x_i, \theta_1)$ и
\[
\E[\partial h_{2i} (\theta) / \partial \theta_1] =\E \left[ -x_{2i} \frac{(y_i-x'_i \theta_2)}{\sigma(x_i, \theta_1)^2} \frac{\partial \sigma(x_i, \theta_1)}{\partial \theta_1} \right],
\]
что равно нулю, поскольку $\E[y_i|x_i]=x'_i \theta_2$. Кроме того, для ДОМНК состоятельность $\hat{\theta}_2$ не требует, чтобы $\hat{\theta}_1$ была состоятельной, так как для $\E[h_{2i}(\theta)]=0$ необходимо, чтобы $\E[y_i|x_i]=x'_i \theta_2$, что не зависит от $\theta_1$.

Второй пример $G_{21}=0$ --- оценка ММП с блочно-диагональной матрицей с $\E[\partial^2 \mathcal{L}(\theta) / \partial \theta_1 \partial \theta'_2]=0$. Например, так будет в регрессии при нормальности ошибок, где $\theta_1$ являются параметрами дисперсии и $\theta_2$ являются параметрами регрессии.

В других примерах, однако, $G_{21} \not = 0$ и должно быть использовано более громоздкое выражение (6.65). Это делается автоматически в компьютерных пакетах для некоторых стандартных двухшаговых оценок, в первую очередь для двухшаговой оценки Хекмана в моделях с самоотбором выборки, представленной в Разделе 16.5.4. В противном случае, $\Var[\hat{\theta}_2]$ должна быть вычислена вручную. Многие компоненты уже посчитаны ранее. В частности $G^{-1}_{11} S_{11} G^{-1}_{11}$ --- скорректированная ковариационная матрица для $\hat{\theta}_1$ и $G^{-1}_{22} S_{22} G^{-1}_{22}$ --- скорректированная ковариационная матрица оценки $\hat{\theta}_2$, которая ошибочно игнорирует оценку $\hat{\theta}_1$. Для данных, независимых по $i$, подкомпоненты подматрицы $S_0$ могут быть состоятельно оценены с помощью $\hat{S}_{jk}= N^{-1} \sum_i \hat{h}_{ji} \hat{h}_{ki}'$. Тогда сложность заключается в вычислении $\hat{G}_{21}= N^{-1} \sum_i \partial h_{2i} / \partial \theta'_1|_{\hat{\theta}}$.

Рекомендуется более простой подход, который заключается в получении бутстрэп стандартных ошибок (см. Раздел 16.2.5), или непосредственно совместно оценить $\theta_1$ и $\theta_2$ в комбинированной модели (6.64), предполагая доступ к обычному ОММ.

Данные более простые подходы также могут быть применены к последовательным оценкам, которые являются оценками ОММ, а не М-оценками. Тогда объединение двух оценок приведёт к набору условий более сложному, чем (6.64), и мы больше не получим (6.65). Тем не менее, можно использовать бутстрэп или оценить совместно, а не последовательно.

\section{Оценивание методом минимального расстояния}

Оценивание методом минимального расстояния --- это способ оценки структурных параметров $\theta$, которые зависят от параметров в приведенной форме $\pi$, при заданной состоятельной оценке $\hat{\pi}$ для $\pi$.

Стандартно обращаются к книге Фергюсона (1958). Ротенберг (1973) применил этот метод к моделям линейных одновременных уравнений, хотя альтернативные методы, приведённые в Разделе 6.9.6, являются стандартными методами, используемыми на практике. Оценка минимального расстояния чаще всего используется в анализе панельных данных. В начальной работе Чемберлен (1982, 1984) (см. Раздел 22.2.7) задаёт $\hat{\pi}$ как оценку МНК из линейной регрессии зависимой переменной текущего периода на регрессоры во всех периодах. Последующие приложения к моделям ковариации (см. Раздел 22.5.4) предполагают, что $\hat{\pi}$ --- оценки дисперсий и автокорреляций в панельных данных. См. также косвенный метод логического вывода (Раздел 12.6).

Предположим, что соотношение между $q$ структурными параметрами и $r > q$ параметрами в приведенной форме --- $\pi_0=g(\theta_0)$. Далее предположим, что у нас есть состоятельная оценка $\hat{\pi}$. Очевидной оценкой является $\hat{\theta}$ такая, что $\hat{\pi}=g(\hat{\theta})$, но это невозможно, так как $q < r$. Вместо этого, оценка минимального расстояния (Minimal distance, MD) $\hat{\theta}_{MD}$ минимизирует целевую функцию по $\theta$:
\begin{equation}
\mathcal{Q}_{N}(\theta)=(\hat{\pi}-g(\theta))' W_N (\hat{\pi}-g(\theta)),
\end{equation}
где $W_N$ --- матрица весов размера $r \times r$.

Если $\hat{\pi} \xrightarrow{p} \pi_0$ и $W_N \xrightarrow{p} W_0$, где $W_0$ конечна и положительно полуопределена, тогда $\mathcal{Q}_{N}(\hat{\theta}) \xrightarrow{p} \mathcal{Q}_{0}(\theta)=(\pi_0-g(\theta))' W_0 (\pi_0-g(\theta))$. Отсюда следует, что $\theta_0$ локально идентифицируема, если 
$ \rank [W_0 \times \partial g(\theta) / \partial \theta']=q$, а для состоятельность в основном необходимо условие $\pi_0=g(\theta_0)$.

Для оценки минимального расстояния $\sqrt{N} (\hat{\theta}_{MD}-\theta_0) \xrightarrow{d} \mathcal{N}[0, \Var[\hat{\theta}_{MD}]]$, где
\begin{equation}
\Var[\hat{\theta}_{MD}]=(G'_0 W_0 {G_0}^{-1})^{-1} (G'_0 W_0 \Var[\hat{\pi}] W_0 G_0) (G'_0 W_0 G_0)^{-1},
\end{equation}
$G_0=\partial g(\theta) / \partial \theta'|_{\theta_0}$, и предполагается, что параметры в приведенной форме $\hat{\pi}$ имеют предельное распределение $\sqrt{N} (\hat{\pi}-\pi_0) \xrightarrow{d} \mathcal{N}[0,\Var[\hat{\theta}]]$. Более эффективные оценки в приведенной форме приводят к более эффективным оценкам минимального расстояния, поскольку меньшее значение $\Var[\hat{\pi}]$ приводит к меньшим значениям $\Var[\hat{\theta}_{MD}]$ в (6.67).

Чтобы получить результат (6.67), начнём со следующего масштабирования условий первого порядка для оценки минимального расстояния:
\begin{equation}
G_N(\hat{\theta})' W_N \sqrt{N} (\hat{\pi}-g(\theta))=0,
\end{equation}
где $G_N(\theta)=\partial g(\theta) / \partial \theta'$. Точное разложение в ряд Тейлора первого порядка в точке $\theta_0$ даёт
\begin{equation}
\sqrt{N} h(\hat{\pi}-g(\hat{\theta}))= \sqrt{N} (\hat{\pi}-\pi_0) - G_N (\theta^+) \sqrt{N} (\hat{\theta}-\theta_0),
\end{equation}
где $\theta^+$ лежит между $\theta$ и $\theta_0$, и мы используем $g(\theta_0)=\pi_0$. Подставляя (6.69) обратно в (6.68) и решая относительно $\sqrt{N} (\hat{\theta}-\theta_0)$ получаем:
\begin{equation}
\sqrt{N} (\hat{\theta}-\theta_0)= [G_N(\hat{\theta})' W_N G_N(\theta^+)]^{-1} G_N(\hat{\theta})' W_N \sqrt{N} (\hat{\pi}-\pi_0),
\end{equation}
что приводит напрямую к (6.67).

Для заданных оценок в приведенной форме $\hat{\pi}$ наиболее эффективные оценки минимального расстояния используют матрицу весов $W_N=\widehat{\Var}[\hat{\pi}]^{-1}$ в (6.66). Эта оценка называется оценкой оптимального метода минимального расстояния, а иногда минимальной оценкой хи-квадрат по Фергюсону (1958).

Распространённым альтернативным частным случаем является оценка равновзвешенного минимального расстояния, у которой $W_N=I$. Она менее эффективна, чем оценка ОМР, но у неё нет проблемы смещения в конечных выборках, аналогичной той, которая обсуждалась в Разделе 6.3.5 и которая возникает при использовании оптимальный матрицы весов. Равновзвешенная оценка может быть получена с помощью регрессии НМНК $\hat{\pi}_j$ на $g_j(\theta), j=1, \dots, r$, так как минимизация $(\hat{\pi}-g(\hat{\theta}))'(\hat{\pi}-g(\hat{\theta}))$ даёт такие же условия первого порядка, как в (6.68) с $W_N=I$.

Максимальное значение целевой функции для оптимального метода наименьшего расстояния распределено по хи-квадрат. В частности,
\begin{equation}
(\hat{\pi}-g(\hat{\theta}_{OMD}))' \Var[\hat{\pi}]^{-1}(\hat{\pi}-g(\hat{\theta}_{OMD})) 
\end{equation}
асимптотически распределена по $\chi^2(r-q)$ при $H_0:g(\theta_0)=\pi_0$. Можно провести тест на спецификацию модели, аналогичный тесту OIR в Разделе 6.3.8.

Оценки минимального расстояния качественно похожи на оценки ОММ. Подход ОММ является более стандартными. Оценка минимального расстояния чаще всего используется в панельных исследованиях ковариационных структур, поскольку $\hat{\pi}$ включает легко оцениваемые выборочные моменты (дисперсии и ковариации) и может быть использована для получения оценки $\hat{\theta}$.

\section{Эмпирический метод правдоподобия}

Подходы ММ и ОММ не требуют полной спецификации условной плотности. Вместо этого, оценка основана на моментных условиях вида $\E[h(y,x,\theta)]=0$. Подход эмпирического метода правдоподобия, по Оуэну (1988), является альтернативной процедурой оценки, основанной на тех же моментных условиях.

Привлекательностью оценки эмпирического метода правдоподобия является то, что, хотя она асимптотически эквивалентна оценке ОММ, у неё другие свойства в конечных выборках, и в некоторых случаях она превосходит оценку ОММ.

\subsection{Оценки математического ожидания генеральной совокупности с помощью эмпирического метода правдоподобия}

Начнём с эмпирического метода правдоподобия в случае скалярных нормальных и независимо распределённых случайных величин $y$ с плотностью $f(y)$ и функцией максимального правдоподобия для выборки $\Pi_i f(y_i)$. Отличие, которое мы рассматриваем, состоит в том, что плотность $f(y)$ не специфицирована, так что обычный подход ММП невозможен.

В полностью непараметрическом подходе оценивают плотность $f(y)$, вычисленную в каждом из выборочных значений $y$. Пусть $\pi_i=f(y_i)$ обозначает вероятность того, что $i$-ое наблюдение $y$ принимает реализовавшееся значение $y_i$. В таких обозначениях цель состоит в максимизации так называемый эмпирической функции правдоподобия $\Pi_i \pi_i$, или что равносильно, максимизации эмпирической логарифмической функции правдоподобия $N^{-1} \sum_i \ln \pi_i$, которая является мультиномиальной моделью без определенной структуры для $\pi_i$. Эта логарифмическая функция правдоподобия неограничена, если не наложить  ограничения на $\pi_i$. Используется нормализация $\sum_i \pi_i=1$. Она приводит к стандартной оценке функции распределения для полностью непараметрического случая, как мы сейчас покажем.

Оценки эмпирического метода правдоподобия максимизируют Лагранжиан по $\pi$ и $\eta$
\begin{equation}
\mathcal{L}_{EL}(\pi,\eta)= \frac{1}{N} \sum_{i=1}^{N} \ln \pi_i - \eta \left( \sum_{i=1}^{N} \pi_i - 1 \right),
\end{equation}
где $\pi=[\pi_1 \dots \pi_N]'$ и $\eta$ --- множитель Лагранжа. Хотя данные $y_i$ не появляются явно в (6.72), они появляются неявно, поскольку $\pi_i=f(y_i)$. Взяв производные по $\pi_i (i=1, \dots, N)$ и приравняв $\eta$ к единице, получаем $\hat{\pi}_i=1/N$ и $\eta=1$. Таким образом, оцененная функция вероятности $\hat{f}(y)$ присваивает вес $1/N$ для каждого реализованного значения $y_i, i=1, \dots, N$. В результате функция распределения --- $\hat{F}(y) = N^{-1} \sum_{i=1}^N {\bf{1}} (y \le y_i)$, где ${\bf{1}}(A)=1$, если событие $A$ происходит, и $0$ в противном случае. Значит $\hat{F}(y)$ --- обычная эмпирическая функция распределения.

Теперь введём параметры. Для простого примера предположим, что мы вводим моментное ограничение, $\E[y-\mu]=0$, где $\mu$ является неизвестным математическим ожиданием генеральной совокупности. В контексте эмпирического метода правдоподобия этот теоретический момент заменяется выборочным моментом, взвешивающим выборочные значения с помощью вероятностей $\pi_i$. Таким образом, мы вводим ограничение, что $\sum_i \pi_i (y_i -\mu)=0$. Функция Лагранжа для оценки эмпирического метода максимального правдоподобия имеет вид
\begin{equation}
\mathcal{L}_{EL}(\pi,\eta,\lambda,\mu) = \frac{1}{N} \sum_{i=1}^{N} \ln \pi_i - \eta \left( \sum_{i=1}^{N} \pi_i -1 \right) - \lambda \sum_{i=1}^{N} \pi_i (y_i -\mu),
\end{equation}
где $\eta$ и $\lambda$ --- множители Лагранжа.

Начнём с дифференцирования функции Лагранжа по $\pi_i (i=1, \dots, N)$, $\eta$ и
$\lambda$, но не $\mu$. Приравняем эти производные к нулю, и получим уравнения, которые являются функциями от $\mu$. Решая их находим $\pi_i=\pi_i(\mu)$ и, следовательно, функция эмпирического правдоподобия $N^{-1} \sum_i \ln \pi_i (\mu)$ максимизируется по $\mu$. Этот метод решения приводит к нелинейным уравнениям, которые должны быть решены численно.

Для этой конкретной проблемы есть более  простой способ найти оценку для $\mu$. Заметим, что максимальное значение $\mathcal{L}(\pi,\eta,\lambda,\mu)$ должно быть меньше или равно $N^{-1} \sum_i \ln N^{-1}$, так как это максимальное значение без последнего ограничения. Однако  $\mathcal{L}(\pi,\eta,\lambda,\mu)$ равно $N^{-1} \sum_i \ln N^{-1}$, если $\pi_i = 1/ N$ и $\hat{\mu}=N^{-1} \sum_i y_i = \bar{y}$. Поэтому оценка математического ожидания генеральной совокупности при использовании эмпирического метода максимального правдоподобия --- это выборочное среднее.

\subsection{Оценки параметров регрессии эмпирического метода правдоподобия}

Теперь рассмотрим данные регрессии, которые являются одинаково и независимо распределёнными по $i$. Структура модели задани  $r$ моментными условиями:
\begin{equation}
\E[h(w_i,\theta)]=0,
\end{equation}
где $h(\cdot)$ и $w_i$ определены в Разделе 6.3.1. Например, $h(w,\theta)=x(y-x'\theta)$ для оценки МНК и $h(y,x,\theta)=(\partial g / \partial \theta)(y-g(x,\theta))$ для оценки НМНК.

Подход эмпирического метода правдоподобия максимизирует эмпирическую функцию правдоподобия $N^{-1} \sum_i \ln \pi_i$ при ограничении $\sum_i \pi_i=1$ (см. (6.72)) и при дополнительном выборочном ограничении, основанном на условии для теоретического момента:
\begin{equation}
\sum_{i=1}^{N} \pi_i h(w_i,\theta)=0.
\end{equation}
Таким образом, мы максимизируем по $\pi,\eta,\lambda$ и $\theta$
\begin{equation}
\mathcal{L}_{EL}(\pi,\eta,\lambda,\theta) = \frac{1}{N} \sum_{i=1}^{N} \ln \pi_i - \eta \left( \sum_{i=1}^{N} \pi_i -1 \right) - \lambda' \sum_{i=1}^{N} \pi_i h(w_i,\theta),
\end{equation}
где множители Лагранжа --- это скаляр $\eta$ и вектор-столбец $\lambda$ той же размерности, что и $h(\cdot)$.

Во-первых, найдем $N$ параметров $\pi_1, \dots, \pi_N$. Продифференцировав $\mathcal{L}(\pi,\eta,\lambda,\theta)$ по $\pi_i$, получим $1/(N \pi_i) -\eta-\lambda'h_i=0$. Тогда получаем $\eta=1$ с помощью умножения на $\pi_i$ и суммируя по $i$ и используя $\sum_i \pi_i h_i=0$. Отсюда следует, что
\begin{equation}
\pi_i(\theta,\lambda)=\frac{1}{N(1+\lambda'h(w_i,\theta))}.
\end{equation}

Теперь задача сведена к максимизации по $(r+q)$ переменных $\lambda$ и $\theta$. Это множители Лагранжа, связанные с $r$ условиями моментов (6.74), и $q$ параметров $\theta$.

Решение на этом этапе требует применения численных методов, даже для точно идентифицированных моделей. Можно максимизировать по $\theta$ и $\lambda$ функцию $N^{-1} \sum_i \ln [1/N(1+\lambda'h(w_i,\theta))]$.

Можно пойти другим путём и сначала выразить $\lambda$. Дифференцирование $\mathcal{L}(\pi(\theta,\lambda),\eta,\lambda)$ по $\lambda$ даёт $\sum_i \pi_i h_i = 0$. Определим $\lambda(\theta)$ как неявное решение системы из $\dim(\lambda)$ уравнений:
\[
\sum_{i=1}^{N} \frac{1}{N(1+\lambda'h(w_i,\theta))} h(w_i,\theta) =0.
\]
При практическом применении нужны численные методы для получения $\lambda(\theta)$. Тогда(6.77) упрощается до:
\begin{equation}
\pi_i(\theta)=\frac{1}{N(1+\lambda(\theta)' h(w_i,\theta))}.
\end{equation} 
Подставляя (6.78) в функцию эмпирического метода правдоподобия $N^{-1} \sum_i \ln \pi_i$, логарифмическая функция эмпирического метода правдоподобия, оценённая в $\theta$, имеет вид,
\[
\mathcal{L}_{EL}(\theta)=-N^{-1} \sum_{i=1}^N \ln [N(1+\lambda(\theta)' h(w_i,\theta))].
\]

Оценка эмпирического метода максимального правдоподобия $\hat{\theta}_{MEL}$ максимизирует эту функцию по $\theta$.

Квин и Лоулесс (1994) показали, что
\[
\sqrt{N} (\hat{\theta}_{MEL}-\theta_0) \xrightarrow{d} \mathcal{N}[0, A(\theta_0)^{-1} B(\theta_0) A(\theta_0)'^{-1}],
\]
где $A(\theta_0)=\plim \E[\partial h(\theta) / \partial \theta'|_{\theta_0}]$ и $B(\theta_0)= \plim \E[h(\theta)h(\theta)'|_{\theta_0}]$. Это то же предельное распределение, как и у метода моментов (см. (6.13)). Однако в конечных выборках $\hat{\theta}_{MEL}$ отличается от $\hat{\theta}_{GMM}$, и статистические выводы строятся на выборочных оценках:
\[
\hat{A}= \sum_{i=1}^N \hat{\pi}_i \left. \frac{\partial h'_i}{\partial \theta} \right|_{\hat{\theta}},
\]
\[
\hat{B}= \sum_{i=1}^N \hat{\pi}_i h_i(\hat{\theta}) h_i(\hat{\theta})'
\]
которые взвешены с помощью оценённых вероятностей $\hat{\pi}_i$, а не с помощью констант $1/N$.

Имбенс (2002) сравнивает эмпирический метод правдоподобия с ОММ в своём обзоре. Различные вариации метода включают, например, замену $N^{-1} \sum_i \ln \pi_i$ в (6.26) на $N^{-1} \sum_i \pi_i \ln \pi_i$. Эмпирический метод правдоподобия  вычислительно более сложен, см. Имбенс (2002) для обсуждения. Преимущество метода в том, что асимтотические результаты дают лучшую аппроксимацию в конечных выборках, чем для оценки ОММ. Это дополнительно рассмотрено в Разделе 11.6.4.

\section{Линейные системы уравнений}

Предыдущая теория оценивания охватывает методы оценки с одним уравнением, которые используются в большинстве прикладных исследований. Рассмотрим теперь одновременную оценку нескольких уравнений. Уравнения, линейные по параметрам с аддитивными ошибками, представлены в этом разделе, с обобщением до нелинейных систем, приведённом в следующем разделе.

Основным преимуществом совместного оценивания является увеличение эффективности,  связанной с учётом корреляции ненаблюдаемых переменных по всем уравнениям для данного индивида. Кроме того, совместная оценка может быть необходима, если есть ограничения на параметры уравнений. Оценка систем с экзогенными регрессорами является небольшим расширением МНК с одним уравнением и оценки НМНК, в то время как с эндогенными регрессорами адаптируется метод инструментальных переменных с одним уравнением.

Один из основных примеров системы уравнений --- уравнения для наблюдаемого спроса на нескольких товаров в определённый момент времени для множества индивидов. Для внешне не связанных уравнений все регрессоры являются экзогенными, тогда как для системы одновременных уравнений некоторые регрессоры являются эндогенными.

Второй распространенный пример --- панельные данные, где одно уравнение наблюдается в нескольких моментах времени для множества индивидов, и каждый период времени рассматривается как отдельное уравнение. Рассматривая модель панельных данных в качестве примера системы, можно повысить эффективность, получить скорректированные стандартные ошибки для панели и получить инструменты, когда некоторые регрессоры являются эндогенными.

Многие эконометрические тексты содержат подробное изложение  систем линейных уравнений. Рассмотрение здесь очень краткое. Оно в основном направлено на обобщение на нелинейные системы (см. Раздел 6.10) и применение на панельных данных (см. Главы 21-23).

\subsection{Системы  линейных уравнений}

Линейная модель с одним уравнением задаётся как $y_i=x'_i\beta+u_i$, где $y_i$ и $x_i$ ---скаляры и $x_i$ и $\beta$ --- вектор-столбцы. Линейная модель с несколькими уравнениями, или многомерная линейная модель, с $G$ зависимыми переменными задаётся как:
\begin{equation}
y_i=X_i \beta + u_i, i=1, \dots, N,
\end{equation}
где $y_i$ и $u_i$ --- векторы размера $G \times 1$, $X_i$ --- матрица размера $G \times K$, и $\beta$ --- вектор-столбец размера $K \times 1$.

В этом разделе мы делаем предположение типичное для пространственных данных, что вектор ошибки $u_i$ независим по $i$, поэтому $\E[u_i u'_j]=0$ для $i \not = j$. Тем не менее, элементы $u_i$ для заданного $i$ могут коррелировать и имеют дисперсию и ковариации, которые изменяются по $i$, что приводит к условной ковариационной матрицы ошибок для $i$-ого индивида равной:
\begin{equation}
\Omega_i=\E[u_i u'_i|X_i].
\end{equation}

Существуют различные причины, по которым может возникнуть модель множественных уравнених. Модель внешне не связанных уравнений содержит $G$ уравнений таких, например, как спросы на различные потребительские товары, где параметры отличаются в разных уравнениях и регрессоры могут варьироваться или не варьироваться в уравнениях. Панельная модель содержит $G$ наблюдений во времени для одного и того же уравнения с параметрами, которые постоянны по периодам и регрессорам, которые варьируются или не варьируются по периодам. Эти два случая подробно изложены в Разделах 6.9.3 и 6.9.4.

Разместив данные (6.79)  по $N$ индивидам в столбец, получаем:
\begin{equation}
\begin{bmatrix} y_1 \\  \vdots \\ y_N \end{bmatrix} = \begin{bmatrix} X_1 \\  \vdots \\ X_N \end{bmatrix} \beta + \begin{bmatrix} u_1 \\  \vdots \\ u_N \end{bmatrix},
\end{equation}
или
\begin{equation}
y=X \beta + u,
\end{equation}
где $y$ и $u$ векторы размера $NG \times 1$ и $X$ представляет собой матрицу $NG \times K$.

Результаты, приведённые далее, могут быть получены путём рассмотрения модели в виде столбцов
(6.82) так же, как и в случае одного уравнения. Таким образом, оценка МНК равна $\hat{\beta}=(X'X)^{-1}X'y$, а в случае точной идентификации с матрицей инструментов $Z$ оценка метода инструментальных переменных равна $\hat{\beta}=(Z'X)^{-1}Z'y$. Единственным существенным изменением является то, что обычное пространственное предположение о диагональной ковариационной матрице ошибок заменяется предположением о блочно-диагональной матрице ошибок. Эта блочная диагональность должна быть принята во внимание при оценке ковариационной матрицы оценок и в формировании оценок ДОМНК и эффективных оценок ОММ.

\subsection{Оценки МНК и ДОМНК для систем уравнений}

МНК оценка системы (6.82) даёт оценку МНК для систем уравнений $(X'X)^{-1}X'y$. Используя (6.81), сразу получаем, что
\begin{equation}
\hat{\beta}_{SOLS}= \left[ \sum_{i=1}^N X'_i X_i \right]^{-1} \sum_{i=1}^N X'_i y_i.
\end{equation}
Оценка является асимптотически нормальной и при предположении, что данные независимы по $i$, обычная робастная сэндвич форма применима и здесь и:
\begin{equation}
\widehat{\Var}[\hat{\beta}_{SOLS}]= \left[ \sum_{i=1}^N X'_i X_i \right]^{-1} \sum_{i=1}^N X'_i \hat{u}_i \hat{u}'_i X_i \left[ \sum_{i=1}^N X'_i X_i \right]^{-1},
\end{equation}
где $\hat{u}_i=y_i-X_i \hat{\beta}$. Эта оценка ковариационной матрицы допускает, что условные дисперсии и ковариации ошибок отличаются по индивидам.

С учётом корреляции элементов вектора ошибок для данного индивида, более эффективная оценка достигается с помощью НМНК или ДОМНК. Если наблюдения независимы по $i$, оценка НМНК системы является оценкой МНК системы, применяемой к преобразованной системе:
\begin{equation}
{\Omega_i}^{-1/2} y_i ={\Omega_i}^{-1/2} X_i \beta + {\Omega_i}^{-1/2}u_i,
\end{equation}
где $\Omega_i$ --- ковариационная матрица ошибок, определённая в (6.80). У преобразованных ошибок ${\Omega_i}^{-1/2}u_i$ нулевое математическое ожидание и следующая ковариационная матрица:
\[
\E \left[ \left( \Omega^{-1/2}_i u_i \right)' \left( \Omega^{-1/2}_i u_i \right) |X_i \right]= \Omega^{-1/2}_i \E[u'_i u_i|X_i] \Omega^{-1/2}_i
\]
\[
= \Omega^{-1/2}_i \Omega_i \Omega^{-1/2}_i
\]
\[
=I_G.
\]
Таким образом, у преобразованной системы ошибки гомоскедастичны и коррелируют в $G$ уравнениях и оценка МНК является эффективной.

Для реализации этой оценки модель для $\Omega_i$ должна быть специфицирована, например, $\Omega_i=\Omega_i(\gamma)$. Затем проводится оценка системы МНК в преобразованной системе, с $\Omega_i$ заменённой на $\Omega_i(\hat{\gamma})$, где $\hat{\gamma}$ --- состоятельная оценка для $\gamma$. Мы получаем оценку ДОМНК системы:
\begin{equation}
\hat{\beta}_{SFGLS} = \left[ \sum_{i=1}^N X'_i {\hat{\Omega}_i}^{-1} X_i \right]^{-1} \sum_{i=1}^N X'_i {\hat{\Omega}_i}^{-1} y_i.
\end{equation}
Эта оценка является асимптотически нормальной и для защиты от возможной неправильной спецификации $\Omega_i(\gamma)$ мы можем использовать робастную сэндвич оценку ковариационной матрицы:
\begin{equation}
\widehat{\Var}[\hat{\beta}_{SFGLS}]= \left[ \sum_{i=1}^N X'_i {\hat{\Omega}_i}^{-1} X_i \right]^{-1} \sum_{i=1}^N X'_i {\hat{\Omega}_i}^{-1} \hat{u}_i \hat{u}'_i  {\hat{\Omega}_i}^{-1} X_i  \left[ \sum_{i=1}^N X'_i {\hat{\Omega}_i}^{-1} X_i \right]^{-1},
\end{equation}
где $\hat{\Omega}_i=\Omega_i(\hat{\gamma})$.

Чаще всего предполагают, что  $\Omega_i$ не зависит от $i$. Тогда $\Omega_i=\Omega$ является матрицей размера $G \times G$, которую можно состоятельно оценить для конечного $G$ и $N \rightarrow \infty$ с помощью:
\begin{equation}
\hat{\Omega}= \sum_{i=1}^N \hat{u}_i \hat{u}'_i 
\end{equation}
где $\hat{u}_i=y_i-X_i \hat{\beta}_{SOLS}$. Тогда оценка системы ДОНМНК совпадает с формулой (6.86) с $\hat{\Omega}$ вместо $\hat{\Omega}_i$, и после некоторых алгебраических преобразований оценка системы ДОНМНК также может быть записана в виде:
\begin{equation}
\hat{\beta}_{SFGLS} = \left[ X' \left( {\hat{\Omega}}^{-1} \otimes I_N \right) X \right]^{-1} X' \left( {\hat{\Omega}}^{-1} \otimes I_N \right) y',
\end{equation}
где $\otimes$ означает Кронекерово произведение. Предположение, что $\Omega_i=\Omega$ исключает, например, гетероскедастичность по $i$. Это сильное предположение, и во многих случаях на практике лучше использовать робастные стандартные ошибки, посчитанные с применением формулы (6.87). При этом будут получаться  правильные стандартные ошибки, даже если $\Omega_i$ зависит от $i$.

\subsection{внешне не связанные уравнения}

В модели внешне не связанных уравнений (Seemingly Unrelated Regressions, SUR) $g$-ое из $G$ уравнений
для $i$-го из $N$ индивидов задается формулой:
\begin{equation}
y_{ig}=x'_{ig} \beta_g + u_{ig}, g=1, \dots, G, i=1, \dots, N,
\end{equation}
где $x_{ig}$ --- регрессоры, предполагаемые экзогенными, и $\beta_g$ --- векторы параметров размера $K_g \times 1$. Например, для данных по спросу на $G$ товаров для $N$ индивидов, $y_{ig}$ может быть расходами $i$-ого индивида на товар $g$ или доля бюджета на товар $g$. Далее $G$ предполагается фиксированным и относительно небольшим при $N \rightarrow \infty$. Обратите внимание, что мы индексируем $y$ в порядке $y_{ig}$, поскольку результаты тогда можно легко использовать для панельных данных с переменной $y_{it}$ (см. Раздел 6.9.4). Другие авторы используют обратный порядок $y_{gi}$.

Модель SUR была предложена Зеллнером (1962). Термин внешне не связанные уравнения обманчив, поскольку ясно, что уравнения связаны, если ошибки $u_{ig}$ в разных уравнениях коррелируют. Для модели SUR связь между $y_{ig}$ и $y_{ih}$ является косвенной, она возникает из-за корреляции между ошибками в разных уравнениях.

При оценивании сочетаются наблюдения и по уравнениям, и по индивидам. В микроэконометрических приложениях, где предполагается независимость по $i$, наиболее удобно сначала расположить все уравнения для данного индивида друг под другом. Поставив таким образом $G$ уравнений для $i$-ого индивида получаем:
\begin{equation}
\begin{bmatrix} y_{i1} \\ \vdots \\ y_{iG} \end{bmatrix} = \begin{bmatrix}
x'_{i1} & 0 & 0 \\ 0 & \ddots & 0 \\ 0 & 0 & x'_{iG} \end{bmatrix} \begin{bmatrix} \beta_{1} \\ \vdots \\ \beta_{G} \end{bmatrix} + \begin{bmatrix} u_{i1} \\ \vdots \\ u_{iG} \end{bmatrix}.
\end{equation}
Данное выражение имеет вид $y_i=X_i \beta + u_i $ формулы (6.79), где $y_i$ и $u_i$ --- векторы размера $G \times 1$ с $g$-ыми элементами $y_{ig}$ и $u_{ig}$, $X_i$ --- матрица размера $G \times K$ с $g$-ой строкой $[0 \dots x'_{ig} \dots 0]$, и $\beta=[\beta'_1 \dots \beta'_G]'$ --- вектор размера $K \times 1$, где $K=K_1+ \cdots + K_G$. Некоторые авторы наоборот упорядочивают наблюдения по  всем индивидам для данного уравнения, что приводит к различным алгебраическим выражениям для одинаковых оценок.

Учитывая определения $X_i$ и $y_i$, легко показать, что $\hat{\beta}_{SOLS}$ в (6.83) имеет вид
\[
\begin{bmatrix} \hat{\beta}_1 \\ \vdots \\ \hat{\beta}_G \end{bmatrix} = \begin{bmatrix} \left[ \sum_{i=1}^N x_{i1} x'_{i1} \right]^{-1} \sum_{i=1}^N x_{i1} y_{i1} \\ \vdots \\ \left[ \sum_{i=1}^N x_{iG} x'_{iG} \right]^{-1} \sum_{i=1}^N x_{iG} y_{iG} \end{bmatrix}.
\]
То есть МНК для систем уравнений даёт такие же оценки как МНК применяемый по отдельности к каждому уравнению.  Как и следовало ожидать априори, если единственная связующая между уравнениями --- это ошибка, и ошибки рассматриваются как некоррелированные, то совместная оценка сводится к оценке отдельных уравнения.

Лучшая оценка --- это оценка ДОМНК, определённая в (6.86), с использованием $\hat{\Omega}$ в (6.88) и статистических выводов, которые основаны на асимптотической дисперсии, приведённой в (6.87). Эта оценка, как правило, более эффективна, чем оценка системы МНК, хотя можно показать, что она сходится к оценке МНК, если ошибки в разных  уравнениях не коррелированы или если одни и те же регрессоры используются в каждом уравнении.

В моделях внешне не связанных уравнений могут накладываться ограничения на параметры в разных уравнениях. Например, ограничение симметричности может означать, что коэффициент второго регрессора в первом уравнении равен коэффициенту первого регрессора во втором уравнении. Если такие ограничения --- равенства, можно легко оценить модель с помощью соответствующего переопределения $X_i$ и $\beta$, приведённых в (6.79). Например, если есть два уравнения и ограничение $\beta_2=-\beta_1$, тогда определяем $X_i=[x_{i1} \, -x_{i2}]'$ и $\beta=\beta_1$. Также можно воспользоваться обобщением МНК или НМНК для систем уравнений с линейными ограничениями на параметры.

В системах уравнений возможно, что при наличии балансовых ограничений ковариационная матрица вектора ошибок $u_i$ становится вырожденной. Например, предположим, что $y_{ig}$ --- $i$-ая доля бюджета и модель имеет вид $y_{ig}= \alpha_g + z'_i \beta_g + u_{ig}$, где во всех уравнениях одинаковые регрессоры. Тогда $\sum_g y_{ig}=1$, так как сумма долей бюджета равна 1, что требует $\sum_g \alpha_{g}=1$, $\sum_g \beta_{g}=0$ и $\sum_g u_{ig}=0$. Последнее ограничение означает что матрица $\Omega_i$ является вырожденной и, следовательно, необратимой. Можно убрать одно уравнение, скажем последнее, и оценить модель для системы оставшихся $G-1$ уравнения. Тогда оценки параметров для $G$-ого уравнения могут быть получены с помощью ограничений. Например, $\hat{\alpha}_G=1-(\hat{\alpha}_1 + \cdots + \hat{\alpha}_{G-1})$. Кроме того, можно ввести ограничения в виде равенств на параметры на этом этапе. В литературе описываются методы гарантирующие, что получаемые оценки инвариантны к выбору удаляемого уравнения; см., например, Берндт и Савин (1975).

\subsection{Панельные данные}

Другим важным применением методов НМНК для систем уравнений являются панельные данные, где скалярная зависимая переменная наблюдается в каждом из $T$ периодов времени для $N$ индивидов. Панельные данные можно рассматривать как систему уравнений, либо $T$ уравнений для $N$ индивидов или $N$ уравнений для $T$ периодов времени. В микроэконометрике мы предполагаем короткую панель с малым $T$ и $N \rightarrow \infty$, поэтому вполне естественно взять скалярную зависимую переменную $y_{it}$, где $g$-ое уравнение в предыдущем обсуждении теперь интерпретируется как $t$-ый период времени и $G=T$.

Простая модель панельных данных имеет вид
\begin{equation}
y_{it}=x'_{it} \beta + u_{it}, t=1, \dots, T, i=1, \dots, N,
\end{equation}
частный случай (6.90) с постоянным вектором $\beta$. Тогда в (6.79) матрица регрессоров приобретает вид $X_i=[x_{i1} \dots x_{iT}]'$. После некоторых алгебраических преобразований оценки МНК для систем уравнений, определённые в (6.83), можно выразить как:
\begin{equation}
\hat{\beta}_{POLS}= \left[ \sum_{i=1}^{N} \sum_{i=1}^{T} x_{it} x'_{it} \right]^{-1} \sum_{i=1}^{N} \sum_{i=1}^{T} y_{it}x'_{it}.
\end{equation}

Эта оценка называется сквозной МНК оценкой, так как она объединяет или сочетает в себе аспекты cross-section данных и временных рядов.

Сквозная МНК оценка может быть получена с помощью МНК регрессии  $y_{it}$ на $x_{it}$. Однако
если $u_{it}$ коррелируют по $t$ для заданного $i$, стандартные ошибки МНК, которые даются по умолчанию и которые предполагают независимость ошибки и по $i$, и по $t$, являются неправильными и могут быть в значительной степени смещены. Вместо этого статистические выводы должны основываться на робастной форме  ковариационной матрицы, приведённой в (6.84). Это подробно описано в Разделе 21.2.3. На практике оцениваются более сложные модели, чем (6.92), которые включают отдельные специфические эффекты (см. Раздел 21.2).

\subsection{Метод инструментальных переменных для систем уравнений}

Оценка одного линейного уравнения с эндогенными регрессорами была представлена в Разделе 6.4. Теперь мы обобщим её на многомерную линейную модель (6.79), для случая $\E[u_i|X_i] \not= 0$. Бранди и Йоргенсон (1971) рассматривали оценку метода инструментальных переменных в применении к системам уравнений для получения состоятельных и эффективных оценок.

Мы предполагаем существование матрицы инструментов $Z_i$ размера $G \times R$, которая удовлетворяет условию $\E[u_i|Z_i] = 0$ и, следовательно,
\begin{equation}
\E[Z'_i(y_i -X_i \beta)]=0.
\end{equation}
Эти инструменты могут быть использованы для получения оценок параметров для каждого уравнения по отдельности, но совместная оценка уравнений может повысить эффективность. Оценка ОММ систем минимизирует:
\begin{equation}
\mathcal{Q}_{N}(\beta)= \left[  \sum_{i=1}^{N} Z'_i(y_i -X_i \beta) \right]' W_N \left[  \sum_{i=1}^{N} Z'_i(y_i -X_i \beta) \right],
\end{equation}
где $W_N$ --- матрица весов размера $r \times r$. Некоторые алгебраические преобразования приводят к
\begin{equation}
\hat{\beta}_{SGMM}=[X' Z W_N Z' X]^{-1} [X' Z W_N Z' y],
\end{equation}
где $X$ представляет собой матрицу размера $NG \times K$, полученную путём постановки друг под другом $X_1, \dots, X_N$ (см. (6.81)), и $Z$ является матрицей размера $NG \times r$, полученной аналогичной постановкой друг под другом $Z_1, \dots, Z_N$. Оценка ОММ систем имеет точно такой же вид, как и (6.37), и асимптотическая ковариационная матрица --- это та, которая дана в (6.39). Отсюда следует, что робастная оценка ковариационной матрицы --- это
\begin{equation}
\widehat{\Var}[\hat{\beta}_{SGMM}]= N \left[ X' Z W_N Z' X \right]^{-1} \left[ X' Z W_N \hat{S} W_N Z' X \right] \left[ X' Z W_N Z' X \right]^{-1}.
\end{equation}
Предполагая независимость по $i$ для систем уравнений мы получаем
\begin{equation}
\hat{S}= \frac{1}{N} \sum_{i=1}^{N} Z'_i \hat{u}_i \hat{u}_i' Z_i.
\end{equation}

Нескольким вариантам матрицы весов уделяют особое внимание.

Во-первых, оценка оптимального ОММ для систем уравнений --- (6.96) с $W_N={\hat{S}}^{-1}$, где $\hat{S}$ определена в (6.98). Тогда ковариационная матрица упрощается до
\[
\widehat{\Var}[\hat{\beta}_{SGMM}]= N \left[ X' Z  {\hat{S}}^{-1} Z' X \right]^{-1}.
\]
Эта оценка является наиболее эффективной оценкой ОММ, основанной на условиях моментов (6.94). Выигрыш в эффективности возникает из двух факторов: (1) оценивание системы уравнений целиком. При этом допускается корреляция ошибок между разными уравнениями, т.е. $\Var[u_i|Z_i]$ необязательно должна быть блочно-диагональной. (2) допущение достаточно общего вида гетероскедастичности и корреляции, так что $\Omega_i$ может меняться по $i$.

Во-вторых, оценка двухшаговым МНК для систем уравнений.  Для неё $W_N=(N^{-1} Z' Z)^{-1}$. Рассмотрим модель внешне не связанных уравнений, определённую в (6.91), с некоторыми эндогенными регрессорами $x_{ig}$. Тогда двухшаговый МНК для систем уравнений сводится к двухшаговому МНК для каждого уравнения по отдельности с инструментами $z_g$ для $g$-ого уравнения и матрицей инструментов
\begin{equation}
Z_i= \begin{bmatrix} z'_{i1} & 0 & 0 \\ 0 & \ddots & 0 \\ 0 & 0 & z'_{iG} \end{bmatrix}.
\end{equation}
Во многих приложениях $z_1=z_2= \cdots = z_g$. В этом случае общий набор инструментов используется во всех уравнениях, но мы не должны ограничивать анализ только этим случаем. Для модели панельных данных (6.92) двухшаговый МНК для систем уравнений сводится к сквозной оценке двухшагового МНК, если мы определим $Z=[z_{i1} \dots z_{iT}]'$.

В-третьих, предположим, что $\Var[u_i|Z_i]$ не меняется по $i$, т.е. $\Var[u_i|Z_i]=\Omega$. Это аналог предположения о гомоскедастичности для системы уравнений. Тогда, как и с (6.88), состоятельная оценка для $\Omega$ --- это $\hat{\Omega}=N^{-1} \sum_i \hat{u}_i \hat{u}'_i$, где $\hat{u}_i$ --- остатки, основанные на состоятельной оценке метода инструментальных переменных, например, подойдёт оценка двухшагового МНК для системы уравнений. Тогда оценка оптимального ОММ --- (6.96) с $W_N = I_N \otimes \hat{\Omega}$. Отметим, что эта оценка отличается от оценки трёхшагового метода наименьших квадратов, представленной в конце следующего равздела.

\subsection{Системы линейных одновременных уравнений}

Модель линейных одновременных уравнений, введённая в Разделе 2.4, очень важна и часто достаточно подробно излагается в вводных курсах эконометрики бакалаврского уровня. В этом разделе мы приводим очень краткое резюме. Обсуждение идентификации перекликается с тем, которое приведено в Главе 2. В связи с наличием эндогенных переменных оценки МНК и для внешне не связанных уравнений несостоятельны. Состоятельные методы оценки лежат в рамках ОММ, хотя стандартные методы были разработаны задолго до ОММ.

Модель одновременных уравнений определяет $g$-ое из $G$ уравнений для $i$-го из $N$ индивидов, как
\begin{equation}
y_{ig}=z'_{ig} \gamma_g + Y'_{ig} \beta_g + u_{ig}, g=1, \dots, G,
\end{equation}
где порядок индексов --- тот, что в Разделе 6.9, а не в Разделе 2.4, $z_g$ --- вектор экзогенных регрессоров, которые, как предполагается, не коррелируют с ошибками $u_g$, и $Y_g$ --- вектор, содержащий подмножество зависимых переменных $y_1, \dots, y_{g-1}, y_{g+1}, \dots, y_G$ других $G-1$ уравнений. Вектор $Y_g$ является эндогенным, так как он коррелирует с ошибками модели. Модель для $i$-го индивида может быть так же записана в виде:
\begin{equation}
y'_i B + z'_i \Gamma =u_i,
\end{equation}
где $y=[y_{i1} \dots y_{iG}]'$ --- вектор эндогенных переменных размера $G \times 1$, $z_i$ --- вектор экзогенных переменных размера $r \times 1$, что является объединением $z_{i1}, \dots, z_{iG}, u_i=[u_{i1} \dots u_{iG}]'$ --- вектор ошибок размера $G \times 1$, $B$ ---матрица параметров размера $G \times G$ с диагональными элементами единицами, $\Gamma$ --- матрица параметров размера $r \times G$, а некоторые из членов $B$ и $\Gamma$ равны единице. Предполагается, что $u_i$ нормально распределены и независимо  по $i$ с математическим ожиданием $0$ и ковариационной матрицей $\Sigma$.

Модель (6.101) называется структурной формой.  Различные ограничения на $B$ и $\Gamma$ задают различные структуры. Если выразить эндогенные переменные через экзогенные, мы получим приведенную форму:
\begin{equation}
\begin{split}
y_i=-z'_i \Gamma B^{-1} + u_i  B^{-1} \\
= z'_i \Pi+ v_i,
\end{split}
\end{equation}
где $\Pi=-\Gamma B^{-1}$ --- матрица параметров в приведенной форме размера $r \times G$ и $ v_i=u_i B^{-1}$ --- приведенная форма вектора ошибок с ковариационной матрицей $\Omega=(B^{-1})' \Sigma B^{-1}$.

Приведённая форма может быть состоятельна оценена с помощью МНК, который даёт оценки $\Pi=-\Gamma B^{-1}$ и $\Omega=(B^{-1})' \Sigma B^{-1}$. Проблема идентификации, см. Раздел 2.5, заключается в получении уникальных оценок структурной формы параметров $B$, $\Gamma$ и $\Sigma$. Это требует некоторых ограничений на параметры, поскольку без ограничений $B$, $\Gamma$, и $\Sigma$ содержат на $G^2$ параметров больше, чем $\Pi$ и $\Omega$. Необходимым условием для идентификации параметров в $g$-ом уравнении является условие порядка для того, чтобы число экзогенных переменных, исключённых из $g$-ого уравнения, было не менее, чем число включённых эндогенных переменных. Это тоже самое, что и условие порядка, указанное в Разделе 6.4.1. Например, если $Y_{ig}$ в (6.100) имеет одну компоненту, то есть одна эндогенная переменная в уравнении, тогда по крайней мере одна из компонент $x_i$ не должна быть включена. Это гарантирует, что существует столько инструментов, сколько регрессоров.
Достаточное условие идентификации --- более сильное условие ранга. Оно даётся во многих книгах таких, как книга Грина (2003), и для краткости оно не приводится здесь. Другие ограничения такие, как ограничения на ковариации, могут также привести к идентификации.

С учётом идентификации структурные параметры модели могут быть состоятельно оценены с помощью отдельной оценки каждого уравнения двухшаговым методом наименьших квадратов, определённым в (6.44). Тот же набор инструментов $z_i$ используется для каждого уравнения. В $g$-ом уравнении подкомпонента $z_{ig}$ используется как инструмент для себя, а остальная часть $z_i$ используется в качестве инструмента для $y_{ig}$.

Более эффективные оценки системы могут быть получены с использованием оценки трёхшагового метода наименьших квадратов Зеллнера и Тейла (1962), которая предполагает, что ошибки гомоскедастичны, но коррелируют в уравнениях. Во-первых, оценим приведенную форму коэффициентов $\Pi$ в (6.102) с помощью МНК регрессии $y$ на $z$. Во-вторых, получим оценки двухшагового метода МНК с помощью регрессии МНК (6.100), где $Y_g$ заменяется на прогнозы для приведенной формы $\hat{Y}_g=z' \hat{\Pi}_g$. Это МНК регрессия $y_g$ на $\hat{Y}_g$ и $z_g$, или, что эквивалентно, $y_g$ на $\hat{x}_g$, где $\hat{x}_g$ --- прогнозы $Y_g$ и $z_g$ из МНК регрессии на $z$. В-третьих, получим оценку трёхшагового метода наименьших квадратов с помощью МНК для систем, построив регрессию  $y_g$ на $\hat{x}_g, g=1, \dots, G$. Тогда из (6.89):
\[
\hat{\theta}_{3SLS} = \left[ \hat{X'} \left( {\hat{\Omega}_1}^{-1} \otimes I_N \right) \hat{X} \right]^{-1} \hat{X'} \left( {\hat{\Omega}_1}^{-1} \otimes I_N \right)y,
\]
где $\hat{X}$ получается так: сначала формируются блочно-диагональные матрицы $\hat{X}_i$ с  блоками $\hat{x}_{i1}, \dots, \hat{x}_{iG}$, а затем $\hat{X}_1, \dots, \hat{X}_N$ ставятся друг на друга, $\hat{\Omega}=N^{-1} \sum_i \hat{u}_i \hat{u}'_i$, $\hat{u}_i$ --- векторы остатков, рассчитанные с использованием оценок двухшагового МНК.

Эта оценка совпадает с оценкой ОММ для систем уравнений с $W_N= I_N \otimes \hat{\Omega}$ в случае, когда оценка ОММ системы использует одинаковые инструменты в каждом уравнении. В противном случае, оценки трёхшагового МНК и оценки ОММ для систем отличаются, хотя оба метода дают состоятельные оценки, если $\E[u_i|z_i]=0$.

\subsection{Оценка ММП для систем уравнений}

Чаще всего для систем уравнений используют оценки МНК или метода инструментальных переменных с выводами, основанными на скорректированных стандартных ошибках. Теперь дополнительно предположим, что ошибки независимы и одинаково распределены, $u_i \sim \mathcal{N}[0,\Omega]$.

Для систем с экзогенными регрессорами оценки ММП асимптотически эквивалентны оценкам НМНК. Однако эти оценки используют разные оценки $\Omega$ и, следовательно, $\beta$, поэтому в небольших выборках появляются различия между оценками ММП и НМНК. Например, см. Главу 21 для модели случайных эффектов по панельным данным.

Для линейной модели одновременных уравнений (6.101) оценка ММП с ограниченной информацией, оценка ММП для отдельных уравнений, асимптотически эквивалентна оценке двухшагового МНК. Оценка ММП с полной информацией, оценка ММП для системы уравнений, асимптотически эквивалентна оценке трёхшагового МНК. См., например, Шмидт (1976) и Грин (2003).

\section{Нелинейные системы уравнений}

Рассмотрим теперь системы уравнений, которые нелинейны по параметрам. Например, система уравнений спроса, полученных по заданной прямой или косвенной функции полезности, может быть нелинейна по параметрам. Вообще, если отдельная переменная описывается нелинейной моделью, например, логит моделью или моделью Пуассона, то любая совместная модель для двух или более таких переменных обязательно будет нелинейной.

Мы начинаем с обсуждения полностью параметрического моделирования систем уравнений до рассмотрения вопроса о частично параметрическом моделировании. Как и в линейном случае мы приводим модели с экзогенными регрессорами, прежде чем рассматривать эндогенные регрессоры.

\subsection{ММП оценка нелинейных систем}

Оценка ММП для одной зависимой переменной была представлена в Разделе 5.6. Эти результаты могут быть применены к совместным моделям нескольких зависимых переменных с очень небольшим изменением, что условная плотность с одной зависимой переменной $f(y_i|x_i,\theta)$ становится $f(y_i|X_i,\theta)$, где $y_i$ обозначает вектор зависимых переменных, $X_i$ обозначает все регрессоры, и $\theta$ обозначает все параметры.

Например, если $y_1 \sim \mathcal{N}[\exp(x'_1 \beta_1),{\sigma_1}^2]$ и $y_2 \sim \mathcal{N}[\exp(x'_2 \beta_2),{\sigma_2}^2]$, тогда подходящая совместная модель заключаться в том, чтобы считать, что $(y_1,y_2)$ имеют двумерное нормальное распределение с математическим ожиданием $\exp(x'_1\beta_1)$ и $\exp(x'_2\beta_2)$, дисперсиями $\sigma^2_1$ и $\sigma^2_2$ и корреляцией $\rho$.

Для данных, которые не нормально распределены, могут быть сложности в спецификации и выборе достаточно гибкого совместного распределения. Например, для одномерных счётных данных стандартная исходная модель --- отрицательная биномиальная (см. главу 20). Для обобщения этой модели на двумерные или многомерные счётные данные существует несколько альтернативных отрицательных биномиальных моделей на выбор. Они могут отличаться, например, тем, какое распределение  предполагается отрицательным биномиальным: одномерное условное  или одномерное предельное. Для многомерного нормального распределения и условное и предельное распределение являются  нормальными. Все эти многомерные отрицательные биномиальные распределения накладывают некоторые ограничения на диапазон корреляции. Например,  ограничение на положительность корреляцию. А для многомерного нормального распределения нет таких ограничений.

К счастью, современные прорывы в вычислительных методах позволяют специфицировать более сложные модели. Например, разумно гибкая модель для коррелированных двумерных счётных данных заключается в предположении, что зависимый от ненаблюдаемых $\varepsilon_1$ и $\varepsilon_2$, $y_1$ распределён по Пуассону с математическим ожиданием $\exp(x'_1 \beta_1 + \varepsilon_1)$ и $y_2$ распределён по Пуассону с математическим ожиданием $\exp(x'_1 \beta_1 + \varepsilon_2)$. Двумерное распределение для исходных данных может быть получено в предположении, что ненаблюдаемые переменные $\varepsilon_1$ и $\varepsilon_2$ являются двумерными нормально распределёнными. Явного решения для этого двумерного распределения не существует, но параметры, тем не менее, можно оценить, используя симуляционный метод максимального правдоподобия, представленный в Разделе 12.4.

Ряд примеров нелинейных совместных моделей приведён в 4-ой части книги. Простейшие совместные модели могут быть негибкими, таким образом, состоятельность может быть основана на предположениях о распределении, которые являются слишком строгими. Тем не менее, как правило, теоретически нет препятствий для спецификации более гибких моделей, которые можно оценить с помощью методов, активно использующих вычисления.

В частности, два основных метода построения сложных многомерных параметрических моделей представлены подробно в Разделе 19.3. Эти методы приведены в контексте модели данных по длительности, но они имеют гораздо более широкое применение. Во-первых, можно ввести коррелированную ненаблюдаемую гетерогенность, как в только что приведённом примере двумерных данных. Во-вторых, можно использовать копулы, обеспечивающие способ генерации совместного распределения при заданных  одномерных предельных распределениях.

Для оценки ММП более простой, хотя и менее эффективной, используется подход квази-ММП, который заключается в спецификации отдельных параметрических моделей для $y_1$ и $y_2$ и получении оценки ММП, предполагая независимость $y_1$ и $y_2$, но при построении статистических выводов учитывается корреляция между $y_1$ и $y_2$. Это подход был  представлен в Разделе 5.7.5. В оставшейся части этого раздела мы рассмотрим аналогичные частично параметрические подходы.

Проблем больше, если есть эндогенность, то есть зависимая переменная в одном уравнении появляется как регрессор в другом уравнении. Существует мало моделей для нелинейных систем одновременных уравнений, помимо нелинейных регрессионных моделей с аддитивными нормальными ошибками.

\subsection{Нелинейные системы уравнений}

Для линейной регрессии переход от одного уравнения к нескольким уравнениям ясен, так как отправной точкой является линейная модель $y=x' \beta +u $ и оценивание ведётся методом наименьших квадратов. Эффективные оценки системы тогда могут быть получены как оценки  НМНК для системы уравнений. Для нелинейных моделей намного больше разнообразия в отправной точке и методе оценки.

Мы определяем многомерную нелинейную модель с $G$ зависимыми переменными следующим образом:
\begin{equation}
r(y_i,X_i,\beta)=u_i,
\end{equation}
где $y_i$ и $u_i$ --- векторы размера $G \times 1$, $r(y_i,X_i,\beta)$ --- векторная функция размера $G \times 1$, $X_i$ --- матрица размера $G \times L$, а $\beta$ --- вектор-столбец размера $K \times 1$. В этом разделе мы делаем обычное  предположение для пространственных данных, что вектор ошибок $u_i$ независим по $i$, но компоненты $u_i$ для заданного $i$ могут коррелировать, а дисперсии и ковариации могут зависеть от $i$.

Один из примеров (6.103) --- нелинейная модель внешне не связанных регрессий. Тогда $g$-ое из $G$ уравнений для $i$-го из $N$ индивидов задаётся таким образом:
\begin{equation}
r_g(y_{ig},x_{ig},\beta_g)=u_{ig}, g=1, \dots, G.
\end{equation}
Например, $u_{ig}=y_{ig}-\exp(x'_{ig} \beta_g)$. Тогда $u_i$ и $r(\cdot)$ в (6.103) --- векторы размера $G \times 1$ с $g$-ыми элементами $u_{ig}$ и $r_g(\cdot)$, $X_i$ такая же блочно-диагональная матрица как в (6.91), и $\beta$ получается путём постановки друг под другом $\beta_1$, $\beta_G$.

Вторым примером является нелинейная модель панельных данных. Тогда для отдельного индивида $i$ в период $t$
\begin{equation}
r(y_{it},x_{it},\beta)=u_{it}, t=1, \dots, T.
\end{equation}
Тогда $u_i$ и $r(\cdot)$ в (6.103) --- векторы размера $T \times 1$, поэтому $G=T$, с $i$-ыми элементами $u_{it}$ и $r(y_{it},x_{it},\beta)$. Модель панельных данных отличается от модели внешне не связанных уравнений тем, что имеет одинаковую функцию $r(\cdot)$ и параметры $\beta$ в каждом периоде.

\subsection{Оценка нелинейных систем}

Когда регрессоры $X_i$ в модели (6.103) являются экзогенными:
\begin{equation}
\E[u_i|X_i]=0,
\end{equation}
где $u_i$ --- ошибки, определенные в (6.103). Мы предполагаем, что ошибки независимы по $i$, а ковариационная матрица выглядит так:
\begin{equation}
\Omega_i = \E[u_iu_i'|X_i].
\end{equation}

\begin{center}
Аддитивнные ошибки
\end{center}

Оценивание нелинейным систем --- это прямое обобщение оценивания систем линейных моделей с помощью МНК и ДОМНК, если ошибки аддитивны. В таком случае можно преобразовать (6.103) в
\begin{equation}
u_i = y_i - g(X_i,\beta).
\end{equation}
В этом случае оценка НМНК для систем минимизирует сумму квадратов остатков $\sum_i u_i'u_i$, в то время как оценка ДОНМНК систем минимизирует
\begin{equation}
Q_N(\beta) = \sum_i u_i'\hat{\Omega}_i^{-1} u_i, 
\end{equation}
где мы задаём модель $\Omega_i(\gamma)$ для $\Omega_i$ и $\hat{\Omega}_i = \Omega_i(\hat{\gamma})$. Чтобы  предотвратить возможную неправильную спецификацию $\Omega_i$, можно использовать скорректированные стандартные ошибки, которые требуют лишь, чтобы $u_i$ были независимы и удовлетворяли (6.106). Тогда оценённая дисперсии оценки ДОНМНК систем такая же, что и для оценки ДОМНК линейных систем из (6.87). Только в данном случае $X_i$ заменён на $\partial{g(y_i,\beta)}/\partial{\beta'}|_{\hat{\beta}}$, и теперь $\hat{u}_i = y_i - g(X_i,\hat{\beta})$. Можно получить более простую оценку дисперсии с помощью НМНК для систем уравнений, заменяя $\hat{\Omega}_i$ на $I_G$.

Основная сложность состоит в том, чтобы специфицировать удачную модель для $\Omega_i$. В качестве примера предположим, что мы хотим хотим совместно моделировать две счетные переменные. В Главе 20 мы покажем, что стандартная модель для счетных данных, немного более общая, чем модель Пуассона, задаёт условное математическое ожидание как $\exp(x'\beta)$ и задаёт условную дисперсию так, чтобы она была кратна $\exp(x'\beta)$. Совместную модель может задать в виде $u = \begin{bmatrix} u_1 & u_2 \end{bmatrix}'$, где $u_1 = y_1 - \exp(x_1'\beta_1)$ и $u_2 = y_2 - \exp(x_2'\beta_2)$. Тогда ковариационная матрица $\Omega_i$ имеет диагональные элементы $\alpha_1 \exp(x_{i1}'\beta_1)$ и $\alpha_2 \exp(x_{i2}'\beta_2)$. В таком случае одной из возможных параметризацией ковариации будет $\alpha_3[\exp(x_{i1}'\beta_1)\exp(x_{i2}'\beta_2)]^{1/2}$. Оценка $\hat{\Omega}_i$ требует наличия оценок $\beta_1, \beta_2, \alpha_1, \alpha_2$ и $\alpha_3$, которые могут быть получены из первого шага оценивания  уравнений по отдельности.

\begin{center}
Неаддитивные ошибки
\end{center}

Регрессия наименьших квадратов с неаддитивными ошибками больше не применима, как было показано в случае одного уравнения в разделе 6.2.2. Вулдридж (2002) представляет состоятельный метод  моментов.

Ограничение на условный момент (6.106) приводит к многим возможным условиям на безусловные моменты, которые можно использовать для оценивания. Очевидная отправная точка --- основывать оценивание на моментных условиях $\E[X_i'u_i] = 0$. Однако можно использовать другие моментные условия. В более общем случае мы рассматриваем оценивание, основанное на $K$ моментных условиях
\begin{equation}
\E[R(X_i,\beta)'u_i] = 0,
\end{equation}
где $R(X_i,\beta)$ --- матрица функций $X_i$ и $\beta$ имеет размер $K \times G$. Спецификация $R(X_i,\beta)$ и возможная зависимость от $\beta$ описаны далее.

По построению существует столько же условий моментов, сколько параметров. Оценка ММ для систем уравнений $\hat{\beta}_{SMM}$ является решением соответствующих выборочных моментных условий 
\begin{equation}
\frac{1}{N} \sum_{i=1}^N R(X_i,\beta)'r(y_i,X_i,\hat{\beta}_{SMM}) = 0,
\end{equation}
где на практике $R(X_i, \beta)$ оценивается в точке $\tilde{\beta}$, являющейся предварительной оценкой. Оценка ММ для систем уравнений асимптотически нормальна с ковариационной матрицей
\begin{equation}
\hat{V}[\hat{\beta}_{SMM}] = \left[ \sum_{i=1}^N \hat{D}_i'\hat{R}_i \right]^{-1} \sum_{i=1}^N \hat{R}_i'\hat{u}_i\hat{u}_i'\hat{R}_i \left[ \sum_{i=1}^N \hat{R}_i'\hat{D}_i \right]^{-1},
\end{equation}
где $\hat{D}_i = \partial{r_i}/\partial{\beta'}|_{\hat{\beta}}$, $\hat{R}_i = R(X_i, \hat{\beta})$ и $\hat{u}_i = r(y_i, X_i, \hat{\beta}_{SMM})$.

Основной вопрос --- спецификация $R(X, \beta)$ из (6.110). Из Раздела 6.3.7 наиболее эффективная оценка, основанная на (6.106), задаёт
\begin{equation}
R^*(X_i, \beta) = \E \left[ \frac{\partial{r(y_i, X_i, \beta)'}}{\partial{\beta}}|X_i \right] \Omega_i^{-1}
\end{equation}

В общем случае первое ожидание правой части требует сильных предположений о распределении, что затрудняет оптимальное оценивание. 

Однако выражение можно упростить, если нелинейная модель содержит аддитивные ошибки из (6.108). Тогда $R^*(X_i, \beta) = \partial{g(X_i,\beta)'}/\partial{\beta} \times \Omega_i^{-1}$, и оцениваемые уравнения (6.110) превращаются в
\[
N^{-1} \sum_{i=1}^N \frac{\partial{g(X_i,\beta)'}}{\partial{\beta}}\Omega_i^{-1}(y_i - X_i'\hat{\beta}_{SMM}) = 0.
\]
Эта оценка асимптотически эквивалентна оценке ДОМНК системы, которая минимизирует (6.109).

\subsection{Оценка метода инструментальных переменных для нелинейных систем}

Когда регрессоры $X_i$ из модели (6.103) эндогенны и $\E[u_i|X_i] \not= 0$, мы предполагаем существование матрицы инструментов $Z_i$ размером $G \times r$ такой, что
\begin{equation}
\E[u_i|Z_i] = 0,
\end{equation}
где $u_i$ --- ошибки, которые определены в (6.103). Мы предполагаем, что ошибки независимы по $i$ и ковариационная матрица имеет вид: $\Omega_i = \E[u_iu_i'|Z_i]$. Для нелинейной  модели внешне не связанных уравнений $Z_i$ определено в (6.99).

Этот подход аналогичен тому, который используется в предыдущем разделе для оценки ММ систем с дополнительной трудностью в том, что теперь может быть избыток инструментов. Это приводит к необходимости оценивания с помощью ОММ, а не просто ММ. Ограничение на условные моменте (6.106) приводит ко многим возможным условиям на безусловные моменте, которые могут использоваться для оценивания. Здесь мы поступаем, как и многие другие, основывая оценивание на моментных  условиях $\E[Z_i'u_i] = 0$. Тогда оценка ОММ систем минимизирует
\begin{equation}
Q_N(\beta) = \left[ \sum_{i=1}^N Z_i'r(y_i,X_i, \beta) \right]'W_N \left[ \sum_{i=1}^N Z_i'r(y_i,X_i, \beta) \right].
\end{equation}
Эта оценка асимптотически нормальна с оценкой ковариационной матрицы
\begin{equation}
\hat{\Var}[\hat{\beta}_{SGMM}] = N\left[ \hat{D}'ZW_N Z'\hat{D} \right]^{-1} \left[ \hat{D}'ZW_N\hat{S}W_NZ'\hat{D} \right] \left[ \hat{D}'ZW_N Z'\hat{D} \right]^{-1},
\end{equation}
где $\hat{D}'Z = \sum_i \partial{r_i'}/\partial{\beta}|_{\hat{\beta}}Z_i$ и $\hat{S} = N^{-1}\sum_i Z_i\hat{u}_i\hat{u}_i'Z_i'$. Мы также предполагаем, что $u_i$ независимы по $i$ с ковариационной матрицей $V[u_i|X_i] = \Omega_i$.

Выбор $W_N = [N^{-1}\sum_i Z_iZ_i']^{-1}$ соответствует нелинейному двухшаговому МНК в случае, если $r(y_i,X_i, \beta)$ получено из нелинейной  модели внешне не связанных уравнений. Выбор $W_N = [N^{-1}\sum_i Z_i\hat{\Omega}Z_i']^{-1}$, где $\hat{\Omega} = N^{-1}\sum_i \hat{u}_i\hat{u}_i'$, называется оценкой нелинейного трёхшагового МНК. Эта оценка является наиболее эффективной оценкой, основанной на моментных условиях $\E[Z_i'u_i] = 0$, если $\Omega_i = \Omega$. Выбор $W_N = \hat{S}^{-1}$ даёт наиболее эффективную оценку при более общем предположении, что $\Omega_i$ может меняться в зависимости от $i$. Однако, как правило, моментные условия, отличные от $\E[Z_i'u_i] = 0$, могут приводить к более эффективным оценкам.

\subsection{Нелинейные системы одновременных уравнений}

Модель нелинейных одновременных уравнений задаёт $g$-ое из $G$ уравнений для $i$-ого из $N$ индивидов следующим образом:
\begin{equation}
u_{ig} = r_g(y_i,x_{ig},\beta_g), g = 1, \dots, G.
\end{equation}
Это похоже на нелинейная модель внешне не связанных уравнений, но сейчас среди регрессоров могут быть зависимые переменные из других уравнений. В отличие от модели линейных одновременных уравнений существует мало полезных с практической точки зрения результатов, которые гарантируют, что модель нелинейных одновременных уравнений идентифицируема.

Если модель идентифицируема, то можно получить состоятельные оценки с использованием оценок ОММ, которые описаны в предыдущем разделе. С другой стороны, мы можем предположить, что $u_i \sim \mathcal{N}[0, \Omega]$, и получить нелинейную оценку ММП с полной информацией. В отличие от модели линейных одновременных уравнений, нелинейная оценка ММП с полной информацией в целом имеет асимптотическое распределение, которое отличается от распределения оценки нелинейного трёхшагового МНК. Также состоятельность нелинейной оценки ММП с полной информацией требует, чтобы ошибки были нормально распределены. Дополнительные подробности можно посмотреть у Амэмия (1985).

Бороться с эндогенностью в нелинейных моделях может быть сложно. Раздел 16.8 рассматривает одновременность в моделях тобит, где анализ проще, когда модель является линейной по латентным переменным. Раздел 20.6.2 представляет пример более сложной нелинейности и эндогенные регрессоры в моделях счётных данных.

\section{Практические соображения}

В идеальном случае можно найти оценки ОММ с помощью эконометрических пакетов, при этом требуются не сильно более глубокия знания, чем для оценивания при  помощи нелинейного метода наименьших квадратов с гетероскедастичными ошибками. Однако, не все широко применяемые эконометрические пакеты могут реализовывать оценивание с помощью ОММ. В зависимости от конкретного применения ОММ оценивание может подразумевать использование более подходящего пакета или использование матричного языка программирования и знание алгебры ОММ.

Распространённое применение ОММ --- оценивание с помощью метода инструментальных переменных. Большинство эконометрических пакетов включает линейный метод инструментальных переменных, но не включают нелинейные методы инструментальных переменных. По умолчанию стандартные ошибки могут предполагаться гомоскедастичнымы, а не скорректированными на гетероскедастичность. Как уже подчёркивалось в Главе 4, может быть трудно получить инструменты, которые не коррелируют с ошибкой, но в то же время достаточно коррелирует с регрессором, или в нелинейном случае соответствующие производные ошибки по параметрам.

Эконометрические пакеты обычно включают методы для  линейных систем, но не для  нелинейных. Опять же по умолчанию стандартные ошибки могут не иметь поправки на гетероскедастичность.

\section{Библиографические заметки}

Описание ОММ представлено в книгах Дэвидсона и МакКиннона (1993, 2004), Гамильтона (1994) и Грина (2003). Более свежие книги Хаяши (2000) и Вулдриджа (2002) делают упор на оценивание с помощью ОММ. Бера и Билиас (2002) описывают связь и историю многих оценок, представленных в главах 5 и 6.

\begin{itemize}
\item [$6.3$] Изначальное описание представлено у Хансена (1982). Хорошее объяснение оптимальных моментных условий для ОММ приведено в приложении у Арельяно (2003). Октябрьский выпуск 2002 года Журнала экономической и бизнес статистики (Journal of Business and Economic Statistics) посвящён оцениванию с помощью ОММ.
\item [$6.4$] Использование линейной оценки инструментальных переменных, описаное Сарганом (1958), является одним из основных предшественником ОММ.
\item [$6.5$] Оценка нелинейного ДМНК, которая были введены Амэмия (1974), легко обобщается до ОММ оценки.
\item [$6.6$] При описании последовательного двухшагового оценивания ссылаются на Ньюи (1984), Мерфи и
Топела (1985), и Пагана (1986).
\item [$6.7$] Описывая оценивание с помощью минимального расстояния, ссылаются на Чемберлена (1982).
\item [$6.8$] Хороший обзор эмпирического метода максимального правдоподобия приведён у Миттелхаммера, Джаджа и Миллера (2000). Ключевые работы, связанные с этой темой, --- работы Оуэна (1988, 2001) и Кьюина и Лоулесса (1994). Имбенс (2002) приводит обзор и применение этого сравнительно нового метода.
\item [$6.9$] Такие работы, как работа Грина (2003), содержат более детальное описание оценивания систем, чем  представленное здесь. Это особенно касается линейных моделей внешне не связанных уравнений и моделей линейных одновременных уравнений.
\item [$6.10$] Амэмия (1985) подробно описывает нелинейные системы уравнений.
\end{itemize}

\section{Упражнения}

\begin{enumerate}
\item [$6 - 1$] Для гамма-регрессионной модели из упражнения 5.2 $\E[y|x] = \exp(x'\beta)$ и $\Var[y|x] = (\exp(x'\beta))^2/2$.
\begin{enumerate}
\item Покажите, что из этих условий следует $\E[x\{(y-x'\beta)^2 - (\exp(x'\beta))^2/2\}] = 0$.
\item Используя моментное условие из пункта (а), выведите оценку ММ $\hat{\beta}_{MM}$.
\item Приведите асимптотическое распределение $\hat{\beta}_{MM}$, используя результат (6.13).
\item Предположим, что мы используем моментное условие  $\E[x(y-\exp(x'\beta))]$ в дополнение к тому, которое было
в пункте (а). Приведите целевую функцию для ОММ-оценки $\beta$.
\end{enumerate}
\item [$6 - 2$] Рассмотрим линейную регрессионную модель для независимых по $i$ данных с $y_i = x_i'\beta + u_i$. Пусть $\E[u_i|x_i] \not= 0$, но есть доступные инструменты $z_i$ с $\E[u_i|z_i] = 0$ и $\Var[u_i|z_i] = \sigma_i^2$, где $\dim(z) > \dim(x)$. Рассмотрим ОММ-оценку $\hat{\beta}$, которая минимизирует
\[
Q_N(\beta) = \left[ N^{-1} \sum_i z_i(y_i - x_i'\beta) \right]'W_N \left[  N^{-1} \sum_i z_i(y_i - x_i'\beta) \right].
\]
\begin{enumerate}
\item Выведите предельное распределение $\sqrt{N}(\hat{\beta} - \beta_0)$, используя общий ОММ результат из (6.11).
\item Укажите, как получить состоятельную оценку асимптотической дисперсии $\hat{\beta}$.
\item Если ошибки гомоскедастичны, то какое $W_N$ Вы бы использовали? Ваш ответ обоснуйте.
\item Если ошибки гетероскедастичны, то какое $W_N$ Вы бы использовали? Ваш ответ обоснуйте.
\end{enumerate}
\item [$6 - 3$] Рассмотрим пример Лапласа только с константой, который приведён в конце раздела 6.3.6. Таким образом, $y = \mu + u$. Тогда оценивание ОММ основывается на $\E[h(\mu)] = 0$, где $h(\mu) = [(y-\mu), (y-\mu)^3]'$.
\begin{enumerate}
\item Используя знания о центральных моментах $y$, приведённых в Разделе 6.3.6, покажите, что $G_0 = \E[\partial{h}/\partial{\mu}] = [-1,-6]'$ и что $S_0 = \E[hh']$ имеет диагональные элементы 2 и 720, а также внедиагональные элементы равные 24.
\item Покажите, что $G_0'S_0^{-1}G_0 = 252/432$.
\item Покажите, что $\hat{\mu}_{OGMM}$ имеет асимптотическую дисперсию $1.7143/N$.
\item Покажите, что ОММ-оценка $\mu$ с $W = I_2$ имеет асимптотическую дисперсию $19.14/N$.
\end{enumerate}
\item [$6 - 4$] Этот вопрос использует пробит-модель, но требует мало знаний о модели. Пусть $y$ --- бинарная переменная, которая принимает значение 0 или 1 в соответствии с тем, происходит ли событие или нет. Пусть $y$ ---вектор регрессоров, и предположим, что наблюдения независимы.
\begin{enumerate}
\item Предположим, что $\E[y|x] = \Phi(x'\beta)$, где $\Phi(\cdot)$ --- функция стандартного нормального распределения. Покажите, что $\E[(y-\Phi(x'\beta))x] = 0$. Приведите оцениваемые уравнения для ОММ-оценки $\beta$.
\item Даст ли эта оценка дают те же самые оценки, что и оценки ММП для пробит-модели? [Только для этой части Вам необходимо прочитать раздел 14.3.]
\item Приведите целевую функцию ОММ, которая соответствует оценке из пункта (а). То есть приведите целевую функцию, которая приводит к тем же самым условиям первого порядка, вплоть до преобразования  с помощью матрицы полного ранга, что и условия, полученные в пункте (а).
\item Теперь предположим, что из-за эндогенности в некоторых компонентах $\E[y|x] \not= \Phi(x'\beta)$. Предположим, существует вектор $z$ с $\dim[z] > \dim[x]$, такой что $\E[y - \Phi(x'\beta)|z] = 0$. Приведите целевую функцию для состоятельной оценки $\beta$. Оценка необязательно должна быть полностью эффективной.
\item Для Вашей оценки из пункта (г) приведите асимптотическое распределение оценки. Чётко обозначьте любые предположения о процессе, порождающем данные, чтобы получить этот результат.
\item Приведите матрицу весов и способ её получить для оценки оптимальным ОММ из пункта (г).
\item Приведите  пример из жизни для пункта (г). То есть, приведите смысловой пример пробит-модели с эндогенным(и) регрессором(ами) и подходящими инструментом(ами). Обозначьте зависимую переменную, эндогенный(ые) регрессор(ы) и инструменты(ы), которые используются для получения состоятельной оценки. [Эта часть на удивление сложная.]
\end{enumerate}
\item [$6 - 5$] Предположим, что мы накладываем ограничение, что $\E[w_i] = g(\theta)$, где $\dim[w] > \dim[\theta]$. 
\begin{enumerate}
\item Получите целевую функцию для ОММ оценки.
\item Получите целевую функцию для оценки минимального расстояния (см. раздел 6.7) с $\pi = \E[w_i]$ и $\hat{\pi} = \bar{w}$.
\item Покажите, что метод минимального расстояния и ОММ эквивалентны для этого примера.
\end{enumerate}
\item [$6 - 6$] Оценка минимального расстояния (см. раздел 6.7) использует ограничение $\pi-g(\theta)=0$. Предположим, в более общем случае, что ограничение имеет вид $h(\theta, \pi)=0$ и мы оцениваем с помощью оценки обобщённого минимального расстояния, которая минимизирует $\mathcal{Q}_{N}(\theta)=h(\theta, \hat{\pi})' W_N h(\theta, \hat{\pi})$. Измените (6.68) - (6.70) таким образом, чтобы показать, что (6.67) выполняется с $G_0= \partial h(\theta, \pi) / \partial \theta|_{\theta_0,\pi_0}$ и $\Var[\hat{\pi}]$, заменённой на $H'_0 \Var[\hat{\pi}] H_0$, где $H_0= \partial (\theta,\pi) / \partial \pi\theta|_{\theta_0,\pi_0}$.
\item [$6 - 7$] Для данных, порожденных процесса из раздела 6.6.4 с $N = 1 000$, получите оценки нелинейного двухшагового МНК и сравните их с оценками двухшагового метода.
\end{enumerate}


\chapter{Проверка гипотез}
\section{Введение}

В этой главе мы рассмотрим тесты на проверку гипотез, в том числе нелинейных по параметрам, используя оценки для нелинейных моделей.

Распределение тестовой статистики может быть получено, используя ту же самую статистическую теорию, что и для получения оценки, так как тестовые статистики, как и оценки, --- это статистики, выбор которых зависит от результатов наблюдений. При подходящей линеаризации оценок и гипотез результаты похожи на результаты тестирования линейных ограничений в линейной регрессионной модели. Однако результаты основываются на асимптотической теории, а также $t$-статистика и $F$-статистика для линейной модели при предположении о нормальности заменяются на статистики, которые распределены асимптотически нормально ($z$-тесты) или распределены по хи-квадрат.

Существуют две основные проблемы, связанные с проверкой гипотез. Во-первых, тесты могут быть неверного размера, так, проверяя гипотезу на номинальном уровне значимости, например, 5\%, истинная вероятность отвергнуть нулевую гипотезу может быть гораздо больше или меньше, чем 5\%. Такая проблема пратически всегда возникает в выборках среднего размера, так как базовая асимптотическая теория является лишь апроксимацией. Одним из решений этой проблемы является метод бутстрэп, который вводится в этой главе, но он настолько важен, что будет рассмотрен отдельно в главе 11. Во-вторых, мощность тестов может быть низкой, то есть вероятность отвергнуть нулевую гипотезу, когда она должна быть отвергнута, будет маленькой. Многие часто не учитывают этот потенциальный недостаток. В этой книге размеру и мощности уделяется большее внимание, чем в других книгах.

Тест Вальда, наиболее широко применяемая тестовая процедура, будет определён в разделе 7.2. В разделе 7.3 дополнительно описаны тест отношения правдоподобия и тест множителей Лагранжа, которые применяются при оценивании с помощью метода максимального правдоподобия. Различные тесты проиллюстрированы в разделе 7.4. Раздел 7.5 представляет методы оценивания, отличные от метода максимального правдоподобия, включая формы тестов с поправками. В разделах 7.6, 7.7 и 7.8 описаны мощность теста, метод Монте-Карло и метод бутстрэп.

Методы для определения спецификации и выбора модели рассмотрены отдельно в главе 8.

\section{Тест Вальда}

Тест Вальда, предложенный Вальдом в 1943 году, --- это выдающийся тест в микроэконометрике. Он требует оценивания неограниченной модели, то есть модели без наложенных на неё ограничений, которые указаны в нулевой гипотезе. Тест Вальда широко применяется, так как современное программное обеспечение обычно позволяет оценить неограниченную модель, даже если она является более сложной, чем ограниченная модель. Более того, современное программное обеспечение позволяет найти оценку ковариационной матрицы с поправкой, которая даёт возможность применять тест Вальда при довольно слабых предпосылках о распределении. Статистика теста на значимость регрессоров, которая рассчитывается различными статистическими пакетами, --- это и есть статистика Вальда. 

В этом разделе детально рассматривается тест Вальда для проверки нелинейных гипотез, с теорией и примерами. Также в нём представлен дельта-метод, который используется для построения доверительных интервалов и областей нелинейных функций параметров. Минусом теста Вальда является его инвариантность к алгебраически эквивалентной параметризации нулевой гипотезы. Он будет подробно освещён в конце раздела.

\subsection{Линейные гипотезы в линейных моделях}

Сначала мы вспомним результаты стандартной линейной модели, так как тест Вальда --- это обобщение обычного теста на линейные ограничения в линейной регрессионной модели.

Нулевая и альтернативная гипотезы для двустороннего теста на линейные ограничения на параметры регрессии в линейной регрессионной модели $y = X'\beta + u$ выглядят так:

\begin{equation}
H_0: R\beta_0 - r = 0,
H_a: R\beta_0 - r \not= 0,
\end{equation}
где есть $h$ ограничений, $R$ --- это матрица констант полного ранга $h$ и размера $h \times K$, $\beta$ --- это вектор параметров размера $K \times 1$, $r$ --- это вектор констант размера $h \times 1$, и $h \leq K$.

Например, совместный тест на то, что $\beta_1 = 1$, $\beta_2 - \beta_3 = 2$ при $K = 4$, может быть сформулирован как нулевая и альтернативная гипотезы, представленные выше при условии, что
\[
R = \begin{bmatrix} 1 & 0 & 0 & 0 \\ 0 & 1 & -1 & 0 \end{bmatrix}, r = \begin{bmatrix} 1 \\ 2 \end{bmatrix}.
\]

Тест Вальда на то, что $R\beta_0 - r = 0$, --- это тест на близость к нулю выборочного аналога $R\hat{\beta} - r$, где $\hat{\beta}$ --- это вектор параметров оценённой с помощью МНК неограниченной модели. При сильном предположении о том, что $u \sim N[0,\sigma_0^2I]$ и оценка $\hat{\beta} \sim N[\beta_0, \sigma_0^2(X'X)^{-1}]$, получается, что при нулевой гипотезе
\[
R\hat{\beta} - r \sim N[0,\sigma_0^2R(X'X)^{-1}R'],
\]
где $R\beta_0 - r = 0$ привело к тому, что математическое ожидание стало равно нулю. Получение квадратичной формы приводит к тестовой статистике
\[
W_1 = (R\hat{\beta} - r)'[\sigma_0^2R(X'X)^{-1}R'](R\hat{\beta} - r),
\]
которая при нулевой гипотезе распределена по $\chi^2(h)$. Однако на практике тестовая статистика $W_1$ не может быть рассчитана, так как $\sigma_0^2$ неизвестна.

В больших выборках замена $\sigma_0^2$ на $s^2$ не влияет на предельное распределение $W_1$, так как это эквивалентно домножению слева $W_1$ на $\sigma_0^2/s^2$ и $\plim(\sigma_0^2/s^2) = 1$ (см. теорему преобразования А.12). Таким образом,
\begin{equation}
W_2 = (R\hat{\beta} - r)'[s^2R(X'X)^{-1}R'](R\hat{\beta} - r)
\end{equation}
сходится к распределению $\chi^2(h)$ при нулевой гипотезе.

Тестовая статистика $W_2$ распределена по хи-квадрат только асимптотически. В этом линейном примере с нормальным распределением ошибок для малой выборки может быть получен альтернативный результат. Стандартный результат, который выводится во многих вводных книгах
\[
W_3 = W_2/h,
\] 
распределён по $F(h,N - K)$ при нулевой гипотезе, если $s^2 = (N - K)^{-1}\sum_i \hat{u}_i^2$, где $\hat{u}_i$ --- это остатки в модели, оценённой с помощью МНК. Это $F$-статистика, которая часто записывается в виде суммы квадратов остатков.

Точные результаты такие, как для $W_3$, невозможно получить в нелинейных моделях, и даже в линейных моделях они требуют очень сильных предпосылок. Вместо этого применяется нелинейный аналог $W_2$ с асимптотическими результатами.

\subsection{Нелинейные гипотезы}

Мы рассмотрим гипотезы об $h$ ограничениях, возможно, нелинейных по параметрам, на векторе параметров $\theta$ размера $q \times 1$, где $h \leq q$. Для линейной регрессии $\theta = \beta$ и $q = K$.

Нулевая и альтернативная гипотезы для двустороннего теста выглядят так:
\begin{equation}
H_0: h(\theta_0) = 0,
H_a: h(\theta_0) \not= 0,
\end{equation}
где $h(\cdot)$ --- векторная функция $\theta$ размера $h \times 1$. Обратите внимание, что в этой главе $h(\theta)$ используется для обозначения ограничений, которые указаны в нулевой гипотезе. Важно это не путать с $h(w, \theta)$, которое использовалось в предыдущей главе для обозначения моментных условий, применяемых для метода моментов и для обобщённого метода моментов.

Обычные линейные примеры включают тест на значимость одного коэффициента $h(\theta) = \theta_j = 0$ и тест на подгруппу коэффициентов $h(\theta) = \theta_2 = 0$. Нелинейным примером одного ограничения может быть $h(\theta) = \theta_1/\theta_2 - 1 = 0$. Эти примеры рассматриваются в дальнейших главах.

Предполагается, что $h(\theta)$ --- это матрица размера $h \times q$
\begin{equation}
R(\theta) = \frac{\partial{h(\theta)}}{\partial{\theta'}},
\end{equation}
которая имеет полный ранг в оцениваемой точке $\theta = \theta_0$. Это предположение равнозначно линейной независимости ограничений в линейной модели, в которой $R(\theta) = R$ не зависит от $\theta$ и имеет полный ранг $h$. Также предполагается, что параметры не находятся на границе области параметров при нулевой гипотезе. Это исключает проверку гипотезы $H_0: \theta_1 = 0$, если для модели необходимо условие $\theta_1 \geq 0$.

\subsection{Статистика теста Вальда}

Логика, которая лежит в основе теста Вальда, довольно проста. Тест на проверку гипотезы $h(\theta_0) = 0$ заключается в том, чтобы получить оценку $\hat{\theta}$ без наложения ограничений и проверить $h(\hat{\theta}) \simeq  0$ или нет. Если $h(\hat{\theta}) \stackrel{a}{\sim} N[0, \V[h(\hat{\theta}]]$ при нулевой гипотезе, тогда тестовая статистика будет такой:
\[
W = h(\hat{\theta})'[\V[h(\hat{\theta})]]^{-1}h(\hat{\theta}) \stackrel{a}{\sim} \chi^2(h).
\]

Единственное затруднение, которое может возникнуть --- это нахождение $\V[h(\hat{\theta})]$, которое зависит от ограничений $h(\cdot)$ и оценки $\hat{\theta}$.

С помощью разложения в ряд Тейлора до первого члена (см. раздел 7.4.2) при нулевой гипотезе $h(\hat{\theta})$ имеет то же самое предельное распределение, что и $R(\theta_0)(\hat{\theta} - \theta_0)$, где $R(\theta)$ было определено в (7.4). Тогда $h(\theta)$ распределён асимптотически нормально при нулевой гипотезе с математическим ожиданием, равным нулю, и ковариационной матрицей $R(\theta_0)\V[\hat{\theta}]R(\theta_0)'$. Состоятельная оценка имеет вид $\hat{R}N^{-1}\hat{C}\hat{R}'$, где $\hat{R} = R(\hat{\theta})$. Предполагается, что для оценки $\hat{\theta}$ верно следующее:
\begin{equation}
\sqrt{N}(\hat{\theta} - \theta) \stackrel{d}{\rightarrow} N[0, C_0],
\end{equation}
и $\hat{C}$ --- любая состоятельная оценка $C_0$.

\begin{center}
Стандартные версии теста Вальда
\end{center}

Из предыдущих обсуждений следует, что статистика теста Вальда выглядит следующим образом:
\begin{equation}
W = N\hat{h}'[\hat{R}\hat{C}\hat{R}']^{-1}\hat{h},
\end{equation}
где $\hat{h} = h(\hat{\theta})$ и $\hat{R} = \partial{h(\theta)}/\partial{\theta'}|_{\hat{\theta}}$. Это эквивалентно выражению $W = \hat{h}'[\hat{R}\hat{\V}[\hat{\theta}]\hat{R}']^{-1}\hat{h}$, где $\hat{\V}[\hat{\theta}] = N^{-1}\hat{C}$ --- это оценка асимптотической ковариационной матрицы $\hat{\theta}$.

Тестовая статистика $W$ асимптотически распределена по $\chi^2(h)$ при нулевой гипотезе. Таким образом, нулевая гипотеза отвергается в пользу альтернативной гипотезы на уровне значимости $\alpha$, если $W > \chi_{\alpha}^2(h)$, во всех остальных случаях она не отвергается. Аналогично нулевая гипотеза отвергается на уровне значимости $\alpha$, если $p$-значение, равное $\Pr[\chi^2(h) > W]$, меньше $\alpha$.

Статистику Вальда можно использовать и для $F$-статистики. Асимптотическая $F$-статистика Вальда
\begin{equation}
F = W/h
\end{equation}
имеет асимпотическое распределение $F(h, N - q)$. Это даёт то же самое $p$-значение, что и для первоначальной статистики Вальда $W$, так как $N \rightarrow \infty$. Однако для конечных выборок $p$-значения будут отличаться. Для нелинейных моделей чаще всего смотрят на $W$, хотя $F$ тоже применяется в надежде, что она даст лучшую апроксимацию для малых выборок.

Для теста только на одно ограничение, квадратный корень из хи-квадрат теста Вальда --- это стандартная тестовая статистика. Этот результат полезен, так как он позволяет проверять односторонние гипотезы. В частности, для скаляра $h(\theta)$ $z$-тестовая статистика Вальда выглядит так:
\begin{equation}
W_z = \frac {\hat{h}}{\sqrt{\hat{r}N^{-1}\hat{C}\hat{r}'}}
\end{equation}
где $\hat{h} = h(\hat{\theta})$ и $\hat{r} = \partial{h(\theta)}/\partial{\theta'}|_{\hat{\theta}}$ --- это вектор размера $1 \times k$. Результат (7.6) подразумевает, что $W_z$ имеет асимптотическое нормальное распределение при нулевой гипотезе. Соответственно, $W_z$ имеет асимптотическое $t$-распределение с $(N-q)$ степенями свободы, так как $t$-распределение сходится к нормальному при $N \rightarrow \infty$.

\begin{center}
Обсуждение
\end{center}

Статистика Вальда (7.6) для нелинейного случая имеет тот же самый вид, что и статистика $W_2$ для линейной модели (7.2). Оцениваемое отклонение от нулевой гипотезы равно $h(\hat{\theta})$, а не $(R\hat{\beta} - r)$. Матрица $R$ заменена оценённой матрицей $\hat{R}$, а предположение о том, что матрица $R$ имеет полный ранг, заменяется предположением о том, что матрица $R_0$ имеет полный ранг. Наконец, оценка ковариационной матрицы равна $N^{-1}\hat{C}$, а не $s^2(X'X)^{-1}$.

Существует множество возможных состоятельных оценок $C_0$ (см. раздел 5.5.2), которые приводят на практике к расчёту  различных значений $W$, $F$ или $Wz$, которые асимптотически эквивалентны. В частности, $C_0$ часто имеет форму сэндвича $A_0^{-1}B_0A_0^{-1}$, оценка с поправкой которого равна $\hat{A}^{-1}\hat{B}\hat{A}^{-1}$. Преимущество теста Вальда состоит в том, что к нему легко применить поправку для обеспечения достоверных статистических заключений при относительно слабых предположениях о распределении таких, как наличие потенциальной гетероскедастичности ошибок.

Вероятность отвергнуть нулевую гипотезу тем выше, чем больше значения $W$ или $F$, или для двустронних тестов $Wz$. Чем дальше значение $h(\hat{\theta})$ от 0 в нулевой гипотезе, тем более эффективна оценка $\hat{\theta}$ (так как $\hat{C}$ принимает маленькие значения) и тем больше размер выборки, так как в этом случае $N^{-1}$ мало. Последний результат является следствием проверки гипотезы на неизменном уровне значимости $\alpha$ при увеличении размера выборки. В принципе можно было бы уменьшить $\alpha$ при увеличении размера выборки. Такие штрафы для полностью параметрических моделей представлены в разделе 8.5.1.


\subsection{Вывод статистики Вальда}

Разложив в ряд Тейлора до первого члена в точке $\theta_0$, получаем
\[
h(\hat{\theta}) = h(\theta_0) + \frac{\partial{h}}{\partial{\theta'}}|_{\theta^{+}} (\hat{\theta} - \theta_0)
\]
для некоторой $\theta^{+}$, которая лежит между $\hat{\theta}$ и $\theta_0$. Отсюда следует, что
\[
\sqrt{N}(h(\hat{\theta}) - h(\theta_0)) = R(\theta^{+})\sqrt{N}(\hat{\theta} - \theta_0),
\]
где $R(\theta)$ представляет собой то, как оно было определено в (7.4), откуда получается, что
\begin{equation}
\sqrt{N}(h(\hat{\theta}) - h(\theta_0)) \stackrel{d}{\rightarrow} N[0, R_0C_0R_0'],
\end{equation}
с помощью теоремы о нормальности предела произведения (Теорема A.17), так как $R(\theta^{+}) \stackrel{p}{\rightarrow} R_0 = R(\theta_0)$ и применяя предельное распределение для $\sqrt{N}(\hat{\theta} - \theta_0)$, которое дано в (7.5).

При нулевой гипотезе (7.9) можно упростить, так как $h(\theta_0) = 0$, отсюда при нулевой гипотезе
\begin{equation}
\sqrt{N}h(\hat{\theta}) \stackrel{d}{\rightarrow} N[0, R_0C_0R_0'].
\end{equation}

Теоретически можно было бы использовать это многомерное нормальное распределение для определения области, в которой гипотеза отвергается, но гораздо проще его привести к хи-квадрат распределению. Напомним, что $z \sim N[0,\Omega]$, где матрица $\Omega$ имеет полный ранг. Тогда $z'\Omega^{-1}z \sim \chi^2(dim(\Omega))$. В таком случае из (7.10) следует, что при нулевой гипотезе
\[
Nh(\hat{\theta})'[R_0C_0R_0']^{-1}h(\hat{\theta}) \stackrel{d}{\rightarrow} \chi^2(h),
\]
где для этого выражения существует обратная матрица в силу предположения, что матрицы $R_0$ и $C_0$ имеют полный ранг. Статистика Вальда, определённая в (7.6), получается при замене $R_0$ и $C_0$ состоятельными оценками.

\subsection{Примеры теста Вальда}

Наиболее распространённые тесты --- тесты на одно или несколько исключающих ограничений. Мы также приводим пример теста для проверки нелинейной гипотезы.

\begin{center}
Тесты на исключающие ограничения
\end{center}

Рассмотрим случай, когда последние $h$ компонент $\theta$ равны нулю. Тогда $h(\theta) = \theta_2 = 0$, где мы разобьём $\theta = (\theta_1', \theta_2')'$. Отсюда следует, что
\[
R(\theta) = \frac{\partial{h(\theta)}}{\partial{\theta'}} = \begin{bmatrix} \frac{\partial{\theta_2}}{\partial{\theta_1'}} & \frac{\partial{\theta_2}}{\partial{\theta_2'}} \end{bmatrix} = \begin{bmatrix} 0 & I_h \end{bmatrix},
\]
где $0$ --- это матрица из нулей размера $(q - h) \times q$ и $h$ --- это единичная матрица размера $h \times h$. Тогда
\[
R(\theta)C(\theta)R(\theta)' = \begin{bmatrix} 0 & I_h \end{bmatrix} \begin{bmatrix} C_{11} & C_{12} \\ C_{21} & C_{22}\end{bmatrix} \begin{bmatrix} 0 \\ I_h \end{bmatrix} = C_{22}.
\]

Таким образом, статистика Вальда для теста на исключающие ограничения выглядит следующим образом:
\begin{equation}
W = \hat{\theta_2}'[N^{-1}\hat{C}_{22}]^{-1}\hat{\theta_2},
\end{equation}
где $N^{-1}\hat{C}_{22} = \hat{\V}[\hat{\theta}_2]$. При нулевой гипотезе $W$ имеет асмиптотическое $\chi^2(h)$ распределение.

Эта тестовая статистика является обобщением теста на подгруппу регрессоров в модели линейной регрессии. В этом случае результаты доступны для малых выборок, если ошибки имеют нормальное распределение, тогда используется соответствующий $F$-тест.

\begin{center}
Тесты на статистическую значимость
\end{center}

Тест на значимость одного коэффициента --- это тест на то, отличается ли $\theta_j$ ($j$-ая компонента $\theta$) от нуля. Тогда $h(\theta) = \theta_j$, и $r(\theta) = \partial{h}/\partial{\theta}'$ --- это вектор, состоящий из нулей, кроме $j$-ой компоненты, которая равна 1, тогда можно упростить (7.8) до выражения
\begin{equation}
W_z = \frac{\hat{\theta}_j}{se[\hat{\theta}_j]},
\end{equation}
где $se[\hat{\theta}_j] = \sqrt{N^{-1}\hat{c}_{jj}}$ --- это стандартная ошибка $\hat{\theta}_j$, а $\hat{c}_{jj}$ --- это диагональный элемент матрицы $\hat{C}$.

Тестовую статистику $W_z$ (7.12) часто называют <<$t$-статистикой>> из-за результатов для линейной регрессионной модели при условии нормальности, но, строго говоря, она является асимптотической <<$z$-статистикой>>.

Для двустороннего теста нулевая гипотеза $H_0: \theta_{j0} = 0$ против альтернативной гипотезы $H_a: \theta_{j0} \not= 0$ отвергается на уровне значимости $\alpha$, если $|W_z| > z_{\alpha/2}$, в остальных случаях она не отвергается. Это даёт те же самые результаты, что и хи-квадрат статистика Вальда, так как $W_z^2 = W$, где $W$ определено в (7.6) и $z^2_{\alpha/2} = \chi^2(h)$.

Нередко бывает, что есть предварительная информация о знаке $\theta_j$. Тогда следует проверять одностороннюю гипотезу. Например, предположим, что, исходя из экономического обоснования или предыдущих исследований, $\theta_j > 0$. В данном случае имеет значение, соответствует ли $\theta_j > 0$ нулевой или альтернативной гипотезе. Для односторонних гипотез принято брать предыдущее утверждение в качестве альтернативной гипотезы, так как можно показать, что в этом нужны более веские доказательства, для того чтобы подтвердить утверждение. В этом случае на уровне значимости $\alpha$, если $W_z < z_{\alpha}$, нулевая гипотеза $H_0: \theta_{j0} \leq 0$ отвергается в пользу альтернативной $H_a: \theta_{j0} > 0$. Аналогично для утверждения $\theta_{j0} < 0$ на уровне значимости $\alpha$, если $W_z < - z_{\alpha}$, нулевая гипотеза $H_0: \theta_{j0} \geq 0$ отвергается в пользу альтернативной $H_a: \theta_{j0} < 0$.

Компьютер обычно приводит $p$-значение для двустороннего теста, но во многих случаях более целесообразно использовать односторонний тест. Если $\hat{\theta}_j$ имеет <<правильный>> знак, то $p$-значение для одностороннего теста в два раза меньше, чем для двустороннего теста.

\begin{center}
Тесты на нелинейное ограничение
\end{center}

Рассмотрим тест на одно нелинейное ограничение
\[
H_0: h(\theta) = \theta_1/\theta_2 - 1 = 0.
\]
Тогда $R(\theta)$ --- это вектор размера $1 \times q$, первый элемент которого равен $\partial{h}/\partial{\theta_1} = 1/\theta_2$, второй элемент равен $\partial{h}/\partial{\theta_2} = - \theta_1/\theta_2^2$, а все остальные равны нулю. Обозначая $\hat{c}_{jk}$ как $j$-ый элемент матрицы $\hat{C}$, (7.6) преобразовывается в
\[
W = N\left(\frac{\hat{\theta}_1}{\hat{\theta}_2} - 1\right)^2 \begin{pmatrix} \begin{bmatrix} \frac{1}{\hat{\theta}_2} & -\frac{\hat{\theta}_1}{\hat{\theta}_2^2} & 0 \end{bmatrix} & \begin{bmatrix} \hat{c}_{11} & \hat{c}_{12} & \dots \\ \hat{c}_{21} & \hat{c}_{22} & \dots \\ \vdots & \vdots & \ddots \end{bmatrix} & \begin{bmatrix} 1/\hat{\theta}_2 \\ - \hat{\theta}_1 / \hat{\theta}_2^2 \\ 0 \end{bmatrix} \end{pmatrix}^{-1},
\]
где $0$ --- это матрица из нулей размера $(q - 2) \times q$, которая даёт 
\begin{equation}
W = N[\hat{\theta}_2(\hat{\theta}_1 - \hat{\theta}_2)]^2(\hat{\theta}_2^2\hat{c}_{11} - 2\hat{\theta}_1\hat{\theta}_2\hat{c}_{12} + \hat{\theta}_1^2\hat{c}_{22})^{-1},
\end{equation}
распределённую асимптотически по $\chi^2(1)$ при нулевой гипотезе. Аналогично $\sqrt{W}$ имеет асимптотическое нормальное распределение.

\subsection{Тесты на неправильную спецификацию модели}

Большинство процедур проверки гипотез, в том числе приведённых в главах 7 и 8 этой книги, предполагают, что модель, которая получается при нулевой гипотезе, верно специфицирована, кроме случая небольшой неверной спецификации, которая не влияет на оценку, но требует, чтобы была поправка для стандартных ошибок.

На практике это является значительным упрощением. Например, при проверке теста на гетероскедастичность ошибок предполагается, что это единственный недостаток, который может иметь регрессия. Однако если условное математическое ожидание неверно специфицировано, то истинный размер теста будет отличаться от номинального размера, даже асимптотически. Кроме того, больше не будет асимптотической эквивалентности тестов, таких как тест Вальда, тест отношения правдоподобия и тест множителей Лагранжа. Однако чем лучше специфицирована модель, тем более полезны тесты.

Кроме того, стоит обратить внимание на то, что зачастую тесты имеют мощность не только против явно заявленной альтернативной гипотезы, но и против других. Например, предположим, что модель, которая получается при нулевой гипотезе, имеет вид $y = \beta_1 + \beta_2x + u$, где ошибки $u$ гомоскедастичны. Тест на то, включать ли $z$ в качестве регрессора или нет, также будет иметь мощность против альтернативы, что модель является нелинейной по $x$, например, $y = \beta_1 + \beta_2x + \beta_3x^2 + u$, если $x$ и $z$ коррелированы. Аналогично тест на гетероскедастичность ошибок также будет иметь некоторую мощность против нелинейности по $x$. Если нулевая гипотеза отвергается, это не означает, что модель, которая получается при альтернативной гипотезе, является единственной возможной моделью.

\subsection{Совместные тесты против отдельных тестов}

В прикладных работах исследователь часто хочет знать, какие коэффициенты из множества коэффициентов <<значимы>>. Когда проверяется несколько гипотез с помощью теста, можно провести совместный или одновременный тест для проверки всех интересующих гипотез или провести отдельные тесты для каждой из гипотез.

Яркий пример для линейной регрессии касается использования отдельных $t$-тестов для проверки нулевых гипотез $H_{10}: \beta_1 = 0$ и $H_{20}: \beta_2 = 0$ вместо использования $F$-теста для проверки совместной гипотезы $H_0: \beta_1 = \beta_2 = 0$, где альтернативой является то, что по крайней мере один из параметров не равен нулю. $F$-тест --- явный совместный тест, при котором $H_0$ отвергается, если оценённая точка $(\hat{\beta}_1, \hat{\beta}_2)$ выходит за пределы линии уровня уровня линии уровня функции плотности. В качестве альтернативы можно провести два отдельных $t$-теста. Эта процедура является неявным совместным тестом, который называют индуцированным тестом (Савин, 1984). Отдельные тесты отвергают $H_0$, если отвергается или $H_{10}$, или $H_{20}$. Это происходит, если точка $(\hat{\beta}_1, \hat{\beta}_2)$ выходит за прямоугольник, границами которого являются критические значения двух тестовых статистик. Даже если для проверки $H_0$ используется один и тот же уровень значимости такой, чтобы эллипс и прямоугольник имели одинаковую площадь, области, в которых гипотеза отвергается, для совместных и отдельных тестов отличаются, и существует возможность конфликта между ними. Например, точка $(\hat{\beta}_1, \hat{\beta}_2)$ может лежать внутри эллипса, но за пределами прямоугольника.

Пусть $e_1$ и $e_2$ обозначают существование ошибки первого рода (см. раздел 7.5.1) в двух отдельных тестах, и пусть $e_I = e_1 \cup e_2$ обозначает существование ошибки первого рода в индуцированном совместном тесте. Тогда $\Pr[e_I] = \Pr[e_1] + \Pr[e_2] - \Pr[e_1 \cap e_2]$, откуда следует, что
\begin{equation}
\alpha_I \leq \alpha_1 + \alpha_2,
\end{equation}
где $\alpha_I$, $\alpha_1$ и $\alpha_2$ обозначают размеры индуцированного совместного теста, первого отдельного теста и второго отдельного теста соответственно. В частном случае, когда отдельные тесты статистически независимы, $\Pr[e_1 \cap e_2] = \Pr[e_1]\Pr[e_2] = \alpha_1 \alpha_2$, и, следовательно $\alpha_I = \alpha_1 + \alpha_2 -\alpha_1 \alpha_2$. Для распространённых значений $\alpha_1$ и $\alpha_2$, например, 0.05 или 0.01, значение $\alpha_1 \alpha_2$ очень мало, и верхняя граница (7.14) является хорошим индикатором размера теста.

Большое число книг по индуцированным тестам рассматривает проблему выбора критических значений для отдельных тестов таких, чтобы индуцированный тест имел известный размер. Мы не рассматриваем этот вопрос подробно, но приводим $t$-тест Бонферрони в качестве примера. Критические значения этого теста представлены в таблице; см. Савин (1984).

Статистически независимые тесты возникают в линейной регрессии с ортогональными регрессорами и при тестировании с помощью метода максимального правдоподобия (см. раздел 7.3), если соответствующие части информационной матрицы диагональные. Тогда тестовая статистика индуцированного совместного теста основывается на двух статистически независимых отдельных тестовых статистиках, тогда как тестовая статистика явного совместного нулевого теста является суммой двух отдельных тестовых статистик. Гипотеза о совместном нуле может быть отвергнута, потому что одна или обе компоненты нулевой гипотезы отвергаются. Использование отдельных тестов покажет, какая из двух ситуаций имеет место.

В более общем случае коррелированных регрессоров или недиагональной информационной матрицы явный совместный тест имеет указанный недостаток. Тот факт, что нулевая гипотеза отвергается, не указывает причину, из-за которой она была отвергнута. Если используется индуцированный совместный тест, то выбор размера теста требует применения некоторого варианта теста Бонферони или приближения с использованием верхней границы из (7.14). Аналогичные проблемы также возникают, когда последовательно проводятся отдельные тесты, каждый следующий этап которых зависит от результатов предыдущего этапа. В разделе 18.7.1 представлен пример с обсуждением совместного теста двух гипотез, где две компоненты теста коррелированы.

\subsection{Дельта метод для доверительных интервалов}

Метод, используемый для вывода статистики Вальда, называется дельта-методом, так как апроксимация с помощью ряда Тейлора $h(\hat{\theta})$ влечёт за собой взятия производной от $h(\theta)$. Этот метод также может быть использован для получения распределения нелинейной комбинации параметров и, следовательно, для построения доверительных интервалов или областей.

Примером может послужить получения для соотношения $\theta_1/\theta_2$ оценки $\hat{\theta}_1/\hat{\theta}_2$. Вторым примером является прогнозирование условного математического ожидания $g(x'\beta)$, например, с помощью $g(x'\hat{\beta})$. Третьим примером является расчёт эластичности по изменению одной компоненты $x$.

\begin{center}
Доверительные интервалы
\end{center}

Рассмотрим вывод вектора параметров $\gamma = h(\theta)$, оценка которого выглядит так:
\begin{equation}
\hat{\gamma} = h(\hat{\theta}),
\end{equation}
где предельное распределение $\sqrt{N}(\hat{\theta} - \theta_0)$ --- то же самое распределение, которое было дано в (7.5). Тогда прямое применение (7.9) даёт $\sqrt{N}(\hat{\gamma} - \gamma_0) \stackrel{d}{\rightarrow} N[0, R_0C_0R_0']$, где $R(\theta)$ определено в (7.4). Аналогично мы говорим, что $\hat{\gamma}$ имеет асимптотическое нормальное распределение со следующей оценкой коварианционной матрицы
\begin{equation}
\hat{\V}[\hat{\gamma}] = \hat{R}N^{-1}\hat{C}\hat{R}'.
\end{equation}
Этот результат может быть использован для построения доверительных интервалов.

В частности, $100(1-\alpha)\%$ доверительный интервал для скалярного параметра $\gamma$ будет иметь вид
\begin{equation}
\gamma \in \hat{\gamma} \pm z_{\alpha/2}se[\hat{\gamma}],
\end{equation}
где 
\begin{equation}
se[\hat{\gamma}] = \sqrt{\hat{r}N^{-1}\hat{C}\hat{r}'},
\end{equation}
где $\hat{r} = h(\hat{\theta})$ и $r(\theta) = \partial{\gamma}/\partial{\theta}' = \partial{h(\theta)}/\partial{\theta}'$.

\begin{center}
Примеры доверительных интервалов
\end{center}

В качестве примера предположим, что $\E[y|x] = \exp(x'\beta)$, и мы хотим построить доверительный интервал для спрогнозированного условного математического ожидания, когда $x = x_p$. Тогда $h(\beta) = \exp(x_p'\beta)$, так, $\partial{h}/\partial{\beta}' = \exp(x_p'\beta)x_p$ и (7.18) дают
\[
se[\exp(x_p'\beta)] = \exp(x_p'\beta)\sqrt{x_p'N^{-1}\hat{C}x_p},
\]
где $\hat{C}$ --- это состоятельная оценка ковариационной матрицы при предельном распределении $\sqrt{N}(\hat{\beta} - \beta_0)$.

В качестве второго примера предположим, что мы хотим построить доверительный интервал для $e^{\beta}$, а не для $\beta$, скалярного коэффициента. Тогда $h(\beta) = e^{\beta}$, так, $\partial{h}/\partial{\beta} = e^{\beta}$ и (7.18) дают $se[e^{\hat{\beta}}] = e^{\hat{\beta}}se[\hat{\beta}]$. Это даёт $95\%$ доверительный интервал для $e^{\beta}$: $e^{\hat{\beta}} \pm 1.96e^{\hat{\beta}}se[\hat{\beta}]$.

Дельта метод не всегда является лучшим способом, чтобы построить доверительный интервал, потому что он ограничивает доверительный интервал так, чтобы он был симметричным относительно $\hat{\gamma}$. Более того, в предыдущем примере доверительный интервал мог включать в себя отрицательные значения, хотя $e^{\beta} > 0$. Альтернативный доверительный интервал получается экспоненцированием слагаемых в доверительном интервале для $\beta$. Тогда
\[
\Pr[\hat{\beta} - 1.96se[\hat{\beta}] < \beta < \hat{\beta} + 1.96se[\hat{\beta}]] = 0.95
\]
\[
\Rightarrow \Pr[\exp(\hat{\beta} - 1.96se[\hat{\beta}]) < e^{\beta} < \exp(\hat{\beta} + 1.96se[\hat{\beta}])] = 0.95.
\]

Этот доверительный интервал имеет то преимущество, что он асимметричный и включает в себя только положительные значения. Такое преобразование часто используется для доверительных интервалов для параметров наклона в моделях бинарного выбора и в моделях длительности. Подход может быть обобщён и на другие преобразования $\gamma = h(\theta)$ при условии монотонности $h(\cdot)$.

\subsection{Отсутствие инвариантности теста Вальда}

Тестовую статистику Вальда легко получить при условии, что оценки модели без ограничений могут быть получены. Она не менее мощная, чем другие возможные процедуры тестов, которые будут рассмотрены в следующих разделах. По этой причине тест Вальда является наиболее часто используемым тестом.

Однако у теста Вальда есть фундаментальная проблема: он не инвариантен к алгебраически эквивалентным параметризациям нулевой гипотезы. Например, рассмотрим на примере раздела 7.2.5. Тогда $H_0: \theta_1/\theta_2 - 1 = 0$ можно эквивалентным образом выразить как $H_0: \theta_1 - \theta_2 = 0$, что приводит к следующей хи-квадрат тестовой статистике Вальда:
\begin{equation}
W^* = N(\hat{\theta}_1 - \hat{\theta}_2)^2(\hat{c}_{11} - \hat{c}_{12} + \hat{c}_{22})^{-1},
\end{equation}
которая отличается от $W$ из (7.13). Статистики $W$ и $W^*$ могут существенно отличаться для конечных выборок, хотя асимптотически они эквивалентны. Для малых выборок разница может быть весьма существенной, как было показано в методе Монте-Карло Грегори и Виллом (1985), которые рассматривали очень похожий пример. Для тестов с номинальным размером 0.05, один вариант теста Вальда имел истинный размер между 0.04 и 0.06 для всех моделей. Так, асимптотическая теория дала хорошую апроксимацию для малых выборок, тогда как альтернативный асимптотически эквивалентный вариант теста Вальда имел истинный размер, который в некоторых моделях превышал 0.20.

Филлипс и Парк (1988) объяснили различия, показав, что, хотя различные представления ограничений нулевой гипотезы  имеют то же хи-квадрат распределение с использованием обычных асимптотических методов, они имеют разные асимптотические распределения при использовании более сложной асимптотической теории, основанной на разложениях Эджуорта (см. раздел 11.4.3). Кроме того, в особых условиях таких, как предыдущий пример, разложения Эджуорта могут быть использованы для указания параметризации нулевой гипотезы и областей пространства параметров, где стандартная асимптотическая теория, скорее всего, может обеспечить плохую апроксимацию для малых выборок.

Урок заключается в том, что необходимо проявлять осторожность при проверке гипотез о нелинейных ограничениях. В качестве проверки на устойчивость можно провести несколько тестов Вальда с использованием различных алгебраически эквивалентных представлений ограничений, налагаемых нулевой гипотезой. Если это приводит к существенно другим выводам, то может возникнуть проблема. Одним из решений является выполнение бутстрэп версии теста Вальда. Это может дать лучший результат для малых выборок и устранить большую разницу между тестами Вальда, которые используют различные представления нулевой гипотезы, потому что, как следует из раздела 11.4.4, бутстрэп по сути реализует разложение Эджурта. Второе решение заключается в использовании других методов тестирования, которые приведены в следующем разделе и которые не зависят от различных представлений нулевой гипотезы.

\section{Тесты, основанные на методе максимального правдоподобия}

В этом разделе мы рассмотрим проверку гипотез, когда функция правдоподобия известна, то есть распределение полностью специфицировано. Существуют три классические статистические методики для проверки гипотез --- тест Вальда, тест отношения правдоподобия и тест множителей Лагранжа. Четвёртый тест, $C(\alpha)$ тест, был предложен Нейманом (1959), менее широко используется и здесь не приводится; см. Дэвидсон и МакКиннон (1993). Все четыре теста асимптотически эквивалентны, поэтому выбор между ними основан на простоте расчёта и на том, какие результаты они дают на конечных выборках. Мы также не приводим гладкий критерий Неймана (1937), который, как утверждают Бера и Гош (2002), является оптимальным и столь же фундаментальным, как и другие тесты.

Эти результаты предполагают правильную спецификацию функции правдоподобия. Более подробное описание тестов, основанных на оценках метода квази-максимального правдоподобия, а также на $m$-оценках и эффективных оценках, полученных с помощью обобщённого МНК, приведены в разделе 7.5.

\subsection{Тест Вальда, тест отношения правдоподобия и тест множителей Лагранжа}

Пусть $L(\theta)$ обозначает функцию правдоподобия, совместную условную плотность $y$ при заданных $X$ и параметрах $\theta$. Мы хотим проверить нулевую гипотезу, которая дана в (7.3), то есть  $h(\theta_0) = 0$.

Тесты, кроме теста Вальда, требуют оценивание, которое подразумевает ограничения, накладываемые нулевой гипотезой. Определим оценки следующим образом:

\begin{equation}
\hat{\theta}_u,
\tilde{\theta}_r,
\end{equation}
где $\hat{\theta}_u$ --- оценка ММП для неограниченной модели, $\tilde{\theta}_r$ --- оценка ММП для ограниченной модели.

Оценка ММП для неограниченной модели $\hat{\theta}_u$ получается из максимизации $\ln L(\theta)$, ранее при рассмотрении теста Вальда она обозначалась просто $\hat{\theta}$. Оценка ММП для ограниченной модели $\tilde{\theta}_r$ получается из максимизации Лагранжиана $\ln L(\theta) - \lambda'h(\theta)$, где $\lambda$ --- это вектор множителей Лагранжа размера $h \times 1$. В простом случае исключающего ограничения $h(\theta) = \theta_2 = 0$, где $\theta = (\theta_1', \theta_2')'$, а при включении ограничения $\tilde{\theta}_r = (\tilde{\theta}_{1r}',0')$, где $\tilde{\theta}_{1r}'$ получается как максимум по $\theta_1$ функции максимального правдоподобия для ограниченной модели $\ln L(\theta_1, 0)$ и $0$, который является вектором из нулей размера $(q - h) \times 1$.

Здесь мы вводим и описываем три тестовые статистики, а вывод откладываем до раздела 7.3.3. Все три тестовые статистики сходятся по распределению к $\chi^2(h)$ при нулевой гипотезе. Таким образом, нулевая гипотезы отвергается на уровне значимости $\alpha$, если вычисленная тестовая статистика превышает $\chi_{\alpha}^2(h)$. Аналогично нулевая гипотеза отвергается на уровне значимости $\alpha$, если $p \leq \alpha$, где $p = \Pr[\chi^2(h) > t]$ --- это $p$-значение, а $t$ --- это вычисленное значение тестовой статистики.

\begin{center}
Тест отношения правдоподобия
\end{center}

Мотивацией для статистики теста отношения правдоподобия является то, что если нулевая гипотеза не отвергается, то безусловный и условный максимумы функции максимального правдоподобия должны совпадать. Это предполагает использование функции разницы между $\ln L(\hat{\theta}_u)$ и $\ln L(\tilde{\theta}_r)$.

Для реализации метода необходимо полученить предельное распределение этой разницы. Можно показать, что разница, домноженная на два, имеет асимптотическое хи-квадрат распределение при нулевой гипотезе. Это приводит к статистике теста отношения правдоподобия
\begin{equation}
LR = -2[\ln L(\tilde{\theta}_r) - \ln L(\hat{\theta}_u)].
\end{equation}

\begin{center}
Тест Вальда
\end{center}

Мотивацией для теста Вальда является то, что если нулевая гипотеза не отвергается, то оценка ММП для неограниченной модели $\hat{\theta}_u$ должна удовлетворять ограничениям, накладываемым нулевой гипотезой, поэтому значение $h(\hat{\theta}_u)$ должно быть близко к нулю.

Для реализации метода необходимо получить асимптотическое распределение $h(\hat{\theta}_u)$. В общем виде тест Вальда приведён в (7.6). Для метода максимального правдоподобия происходит специализация с помощью равенства информационных матриц $\V[\hat{\theta}_u] = -N^{-1}A_0^{-1}$, где
\begin{equation}
A_0 = \plim N^{-1} \frac{\partial^2 \ln L}{\partial{\theta}\partial{\theta}'}|_{\theta_0}.
\end{equation}

Это приводит к тестовой статистике Вальда
\begin{equation}
W = -N\hat{h}'[\hat{R}\hat{A}^{-1}\hat{R}']^{-1}\hat{h},
\end{equation}
где $\hat{h} = h(\hat{\theta}_u)$, $\hat{R} = R(\hat{\theta}_u)$, $R(\theta) = \partial{h(\theta)}/\partial{\theta'}$, и $\hat{A}$ --- это состоятельная оценка $A_0$. Знак минуса появляется, так как $A_0$ отрицательно определена.

\begin{center}
Тест множителей Лагранжа
\end{center}

Мотивацией для статистики теста множителей Лагранжа является то, что градиент $\partial{\ln L}/\partial{\theta}|_{\hat{\theta}_u} = 0$ в точке максимума функции максимального правдоподобия. Если нулевая гипотеза не отвергается, то этот максимум будет максимумом и для функции максимального правдоподобия в случае ограниченной модели (например, $\partial{\ln L}/\partial{\theta}|_{\tilde{\theta}_r} \simeq 0)$, потому что наложение ограничений будет иметь незначительное влияние на оценку $\theta$. Используя эту мотивацию, тест множителей Лагранжа называется  скор-тестом, потому что $\partial{\ln L}/\partial{\theta}$ является скор-вектором.

Альтернативные мотивации заключаются в измерении близости к нулю множителей Лагранжа из оптимизационной задачи в случае ограниченной модели. Максимизация по $\theta$ функции $\ln L(\theta) - \lambda'h(\theta)$ подразумевает, что
\begin{equation}
\frac{\partial{\ln L}}{\partial{\theta}}|_{\tilde{\theta}_r} = \frac{\partial{h(\theta)'}}{\partial{\theta}}|_{\tilde{\theta}_r} \times \tilde{\lambda}_r.
\end{equation}

Отсюда следует, что тесты, основанные на оценённых множителях Лагранжа $\tilde{\lambda}_r$, эквивалентны тестам, основанным на значении $\partial{\ln L}/\partial{\theta}|_{\tilde{\theta}_r}$, так как предполагается, что $\partial{h}/\partial{\theta}'$ имеет полный ранг.

Применение этого метода требует нахождения асимптотического распределения $\partial{\ln L}/\partial{\theta}|_{\tilde{\theta}_r}$. Это приводит к получению статистики теста множителей Лагранжа
\begin{equation}
LM = -N^{-1}\frac{\partial{\ln L}}{\partial{\theta}'}|_{\tilde{\theta}_r}\tilde{A}^{-1}\frac{\partial{\ln L}}{\partial{\theta}}|_{\tilde{\theta}_r}, 
\end{equation}
где $\tilde{A}$ --- это состоятельная оценка $A_0$ из (7.22), оценённая в точке $\tilde{\theta}_r$, а не $\hat{\theta}_u$.

Тест множителей Лагранжа, предложенный Аинчисоном и Сильви (1958) и Сильви (1959), эквивалентен скор-тесту, что показал Рао (1947). Статистика теста множителей Лагранжа обычно выводится путём получения аналитического выражения для скор-вектора, а не для множителей Лагранжа. Эконометристы обычно называют этот тест тестом множителей Лагранжа, хотя более правильно было бы называть его скор-тестом.

\begin{center}
Обсуждение
\end{center}

Хорошее объяснение обеспечивается разъяснительным графическим анализом трёх тестов, который предложил Бьюз (1982) и который рассматривает все три теста, измеряя изменение логарифма функции правдоподобия. Здесь мы приводим словесное описание.

Рассмотрим скалярный параметр и тест Вальда для проверки того, $\theta_0 - \theta^* = 0$ или нет. Тогда данное отклонение $\hat{\theta}_u$ от $\theta^*$ приведёт к тем большему изменению $\ln L$, чем более изогнутым является  логарифм функции правдоподобия. Обычной мерой кривизны функции является вторая производная $H(\theta) = \partial^2{\ln L}/\partial{\theta}^2$. Таким образом, $W = -(\hat{\theta}_u - \theta^*)^2H(\hat{\theta}_u)$. Статистику $W$ из (7.23) можно рассматривать как обобщение вектора $\theta$ и более общее ограничение $h(\theta_0)$, где $N\hat{A}$ используется для измерения кривизны.

Для скор-теста Бьюз показывает, что заданное значение $\partial{\ln L}/\partial{\theta}|_{\tilde{\theta}_r}$ приводит к тем большему изменению $\ln L$, чем менее изогнут логарифм функции правдоподобия. Это приводит к использованию $(N\tilde{A})^{-1}$ из (7.25). И статистика теста множителей Лагранжа напрямую сравнивает логарифмы функций правдоподобия.

\begin{center}
Пример
\end{center}

Чтобы проиллюстрировать три теста, рассмотрим пример, где $y_i$ независимы и одинаково распределены, $y_i \sim N[\mu_0, 1]$. Необходимо проверить гипотезу $H_0: \mu_0 = \mu^*$. Тогда $\hat{\mu}_0 = \bar{y}$ и $\tilde{\mu}_r = \mu^*$.

Для теста отношения правдоподобия $\ln L(\mu) = -\frac{N}{2}\ln 2\pi - \frac{1}{2}\sum_i (y_i - \mu)^2$. После некоторых алгебраических преобразований получается
\[
LR = 2[\ln L(\bar{y}) - \ln L(\mu^*)] = N(\bar{y} - \mu^*)^2.
\]

Тест Вальда основывается на том, $\bar{y} - \mu^* \simeq 0$ или нет. Не представляет сложности показать, что $\bar{y} - \mu^* \sim N[0, 1/N]$ при нулевой гипотезе, что приводит к квадратичной форме
\[
W = (\bar{y} - \mu^*)[1/N]^{-1}(\bar{y} - \mu^*).
\]
Это выражение можно упростить до $N(\bar{y} - \mu^*)^2$, и тогда получается, что $W = LR$.

Тест множителей Лагранжа основывается на близости к нулю $\partial{\ln L(\mu)}/\partial{\mu}|_{\mu^*} = \sum_i (y_i - \mu)|_{\mu^*} = N(\bar{y} - \mu^*)$, что является всего лишь масштабированием $(\bar{y} - \mu^*)$, поэтому $LM = W$. Более формально $\tilde{A}(\mu^*) = -1$, так как $\partial^2{\ln L(\mu)}/\partial{\mu^2} = -N$, и (7.25) даёт
\[
LM = N^{-1}(N(\bar{y} - \mu^*))[1]^{-1}(N(\bar{y} - \mu^*)).
\]
Это тоже можно упростить до $N(\bar{y} - \mu^*)^2$, что подтверждает, что $LM = W = LR$.

Несмотря на совершенно разные мотивации, три тестовые статистики эквивалентны в данном случае. Эта точная эквивалентность свойственна этому примеру с постоянной кривизной вследствие того, что логарифм функции правдоподобия является квадратичной функцией относительно $\mu$. В более общем случае три тестовые статистики отличаются для конечных выборок, но они асимптотически эквивалентны (см. раздел 7.3.4).

\subsection{Пример регрессии Пуассона}

Рассмотрим тесты на исключающие ограничения в регрессионной модели Пуассона, которая была определена в разделе 5.2. Этот пример в основном обучающий, так как на практике следует делать статистические выводы для  данных при более слабых предположениях о распределении, чем тех, которые есть в модели Пуассона (см. главу 20).

Если $y$ при заданном $x$ распределён по Пуассону с условным математическим ожиданием $\exp(x'\beta)$, то логарифм функции правдоподобия будет иметь следующий вид

\begin{equation}
\ln L(\beta) = \sum_{i=1}^N \{- \exp(x_i'\beta) + y_ix_i'\beta - \ln y_i!\}.
\end{equation}
Для $h$ исключающих ограничений нулевая гипотеза $H_0: h(\beta) = \beta_2 = 0$, где $\beta = (\beta_1',\beta_2')'$.

Оценка ММП для неограниченной модели $\hat{\beta}$ получается из максимизации (7.26) по $\beta$, а условие первого порядка вылядит так: $\sum_i (y_i - \exp(x_i'\beta))x_i = 0$. Предельной ковариационной матрицей является $-A^{-1}$, где
\[
A = - \plim N^{-1} \sum_i \exp(x_i'\beta)x_ix_i'.
\]  

Оценка ММП для ограниченной модели $\tilde{\beta} = (\tilde{\beta}_1', 0')'$, где $\tilde{\beta}$ получается при максимизации (7.26) по $\beta_1$, где $x_i'\beta$ заменено на $x_{1i}'\beta_1$, так как $\beta_2 = 0$. Таким образом, $\tilde{\beta}_1$ является решением условия первого порядка $\sum_i (y_i - \exp(x_{1i}'\beta_1))x_{1i} = 0$.

Статистику ММП (7.21) легко рассчитать с помощью подстановки оценок в логарифмы функций максимального правдоподобия для ограниченной и неограниченной моделей.

Тестовая статистика Вальда для исключающих ограничений из раздела 7.2.5 равна $W = -N\hat{\beta}_2'\hat{A}^{22}\hat{\beta}_2$, где $\hat{A}^{22}$ --- это элемент (2,2) из матрицы $\hat{A}^{-1}$, а $\hat{A} = -N^{-1}\sum_i \exp(x_i'\beta)x_ix_i'$.

Тест множителей Лагранжа основывается на $\partial{\ln L(\beta)}/\partial{\beta} = \sum_i x_i(y_i - \exp(x_i'\beta))$. В случае ММП для ограниченной модели это выражение можно переписать как $\sum_i x_i\tilde{u}_i$, где $\tilde{u}_i = y_i - \exp(x_{1i}'\tilde{\beta}_1)$ --- это остатки из ограниченной модели. Статистика теста множителей Лагранжа (7.25) выглядит так:
\begin{equation}
LM = \begin{bmatrix} \sum_{i=1}^n x_i\tilde{u}_i \end{bmatrix}'\begin{bmatrix} \sum_{i=1}^n \exp(x_{1i}'\tilde{\beta}_1)x_ix_i'\end{bmatrix}^{-1}\begin{bmatrix} \sum_{i=1}^n x_i\tilde{u}_i \end{bmatrix}.
\end{equation}

Предыдущее выражение можно упростить ещё больше, так как $\sum_{i=1}^n x_i\tilde{u}_i = 0$ из условия первого порядка в случае ММП для ограниченной модели, который был приведён выше. Здесь тест множителей Лагранжа основывается на корреляции между пропущенными регрессорами и остатками, этот результат распространяется и на другие примеры, которые приведены в разделе 7.3.5.

В общем случае может быть трудно получить алгебраическое выражение для теста множителей Лагранжа. Для стандартных применений теста это было сделано и включено в статистические пакеты. Также возможно применение вспомогательной регрессии для расчёта (см. раздел 3.5).

\subsection{Вывод тестов}

Распределение теста Вальда было формально выведено в разделе 7.2.4. Доказательства для теста отношения правдоподобия и теста множителей Лагранжа являются более сложными, и здесь мы всего лишь кратко опишем их.

\begin{center}
Тест отношения правдоподобия
\end{center}

Для простоты рассмотрим особый случай, где нулевая гипотеза заключается в том, $\theta = \bar{\theta}$ или нет, чтобы не было ошибки оценивания при $\tilde{\theta}_r = \bar{\theta}$. Разложение $\ln L(\bar{\theta})$ в ряд Тейлора до второго члена в окрестности точки $\ln L(\hat{\theta}_u)$ даёт
\[
\ln L(\bar{\theta}) = \ln L(\hat{\theta}_u) + \frac{\partial{\ln L}}{\partial{\theta}'}|_{\hat{\theta}_u}(\bar{\theta} - \hat{\theta}_u) + \frac{1}{2}(\bar{\theta} - \hat{\theta}_u)'\frac{\partial^2{\ln L}}{\partial{\theta}\partial{\theta}'}|_{\hat{\theta}_u}(\bar{\theta} - \hat{\theta}_u) + R,
\]
где $R$ --- это остаточный член. Так как $\partial{\ln L}/\partial{\theta}|_{\hat{\theta}_u} = 0$ из-за условия первого порядка, то после перегруппировки получается
\begin{equation}
-2[\ln L(\bar{\theta}) - \ln L(\hat{\theta}_u)] = - (\bar{\theta} - \hat{\theta}_u)'\frac{\partial^2{\ln L}}{\partial{\theta}\partial{\theta}'}|_{\hat{\theta}_u}(\bar{\theta} - \hat{\theta}_u) + R.
\end{equation}
Правая сторона (7.28) распределена по $\chi^2(h)$ при $H_0: \theta = \bar{\theta}$, так как с помощью стандратных результатов $\sqrt{N}(\hat{\theta}_u) \stackrel{d}{\rightarrow} N[0, -[\plim N^{-1}\partial^2{\ln L}/
\partial{\theta}\partial{\theta}']^{-1}]$. Для того чтобы ознакомиться с тем, как выводится предельное распределение для теста отношения правдоподобия в общем случае, смотрите, например, Амемия (1985, стр. 143).

Причина, из-за которой предпочитается тест отношения правдоподобия, заключается в том, что по лемме Неймана-Пирсона (1933) наиболее мощным тестом для проверки простой нулевой гипотезы против простой альтернативной гипотезы является функция отношения правдоподобия $L(\tilde{\theta}_r)/L(\hat{\theta}_u)$, хотя необязательно, чтобы эта функция совпадала с функцией $-2\ln(L(\tilde{\theta}_r)/L(\hat{\theta}_u))$, которая отражает тест отношения правдоподобия, приведённый в (7.21), и даёт тестовой статистике своё название.

\begin{center}
Тест множителей Лагранжа или скор-тест
\end{center}

С помощью разложения в ряд Тейлора до первого члена получим
\[
\frac{1}{\sqrt{N}} \frac{\partial{\ln L}}{\partial{\theta}}|_{\tilde{\theta}_r} = \frac{1}{\sqrt{N}} \frac{\partial{\ln L}}{\partial{\theta}}|_{\theta_0} + \frac{1}{N} \frac{\partial^2{\ln L}}{\partial{\theta}\partial{\theta}'}\sqrt{N}(\tilde{\theta}_r - \theta_0),
\]
и оба члена правой части выражения входят в предельное распределение. Тогда следует $\chi^2(h)$ распределение метода множителей Лагранжа, которое было определено в (7.25), так как можно показать, что
\begin{equation}
R_0A_0^{-1} \frac{1}{\sqrt{N}} \frac{\partial{\ln L}}{\partial{\theta}}|_{\tilde{\theta}_r} \stackrel{d}{\rightarrow} N[0, R_0A_0^{-1}B_0A_0^{-1}R_0'],
\end{equation}
для которого все детали рассматриваются, например, Вулдриджем (2002, стр. 365), а $A_0$ и $R_0$ заданы в (7.4) и (7.22) и
\begin{equation}
B_0 = \plim N^{-1}\frac{\partial{\ln L}}{\partial{\theta}}\frac{\partial{\ln L}}{\partial{\theta}'}|_{\theta_0}.
\end{equation}

Результат (7.29) приводит к хи-квадрат статистике, которая является гораздо более сложной, чем (7.25), но существует упрощение (7.25) с помощью равенства информационных матриц.

\subsection{Выбор теста}

Тест, как правило, выбирается на основании существования версий с поправками, результатов для конечных выборок и простоты вычислений.

\begin{center}
Асимптотическая эквивалентность
\end{center}

Все три тестовые статистики имеют асимптотическое $\chi^2(h)$ распределение при нулевой гипотезе. Более того, можно показать, что все три теста могут иметь нецентральное  $\chi_2(h,\lambda)$ распределение с одним и тем же параметром нецентральности при локальных альтернативах. Более подробная информация о тесте Вальда представлена в разделе 7.6.3. Таким образом, тесты имеют одинаковую асимптотическую мощность против локальных альтернатив.

Распределения трёх тестовых статистик отличаются для конечных выборок. В линейной регрессионной модели с нормальным распределением остатков вариант тестовой статистики Вальда для $h$ линейных ограничений на $\theta$ в точности равен статистике $F(h, N - K)$ (см. раздел 7.2.1), тогда как невозможно аналитически выразить статистику теста отношения правдоподобия и статистику теста множителей Лагранжа. В целом, в нелинейных моделях для малых выборок их невозможно выразить в явном виде.

В некоторых случаях можно упорядочить значения, применяемые тремя тестовыми статистиками. В частности, для тестов на линейные ограничения для линейной регрессионной модели с нормальным распределением остатков Берндт и Савин (1977) показали, что $W \geq LM \geq LR$. Этот результат имеет небольшие теоретические последствия, так как 
тест, при котором нулевая гипотеза будет отвергнута с наименьшей вероятностью, будет иметь наименьший истинный размер, но и в то же время у него будет наименьшая мощность. Однако это имеет подследствие для линейной модели, так как это означает, что при тестировании на фиксированном номинальном размере $\alpha$ тест Вальда всегда будет отвергать нулевую гипотезу чаще, чем тест отношения правдоподобия, который, в свою очередь, будет отвергать нулевую гипотезу чаще, чем тест множителей Лагранжа. Тест Вальда будет предпочтитаться исследователем, который хочет отвергнуть нулевую гипотезу. Тем не менее, этот результат является верным только для линейных моделей.

\begin{center}
Инвариантность к параметризации 
\end{center}

Тест Вальда не является инвариантным к алгебраически эквивалентным параметризациям нулевой гипотезы (см. раздел 7.2.9), тогда как тест отношения правдоподобия является инвариантным. Некоторые, но не все версии теста множителей Лагранжа, инвариантны. Тест множителей Лагранжа, как правило, является инвариантным, если для оценки $A_0$ используется математическое ожидание Гессиана (см. раздел 5.5.2), и является неинвариантным, если используется сам Гессиан. Тест множителей Лагранжа$^*$, который будет описан позже в (7.34), инвариантный. Отсутствие инвариантности теста Вальда является одним из его основных недостатков.

\begin{center}
Версии с поправками
\end{center}

В некоторых случаях с неверно специфицированной плотностью метод квази-максимального правдоподобия (см. раздел 5.7) остаётся состоятельным. Тогда легко сделать поправку для теста Вальда (см. раздел 7.2). Для теста множителей Лагранжа можно сделать поправку, но это будет более сложно, см. (7.38) в разделе 7.5.1 общий результат для $m$-оценок и в разделе 8.4 некоторые примеры теста множителей Лагранжа с поправкой. Тест отношения правдоподобия уже не будеть иметь хи-квадрат распределение, за исключением частного случая, который будет приведён в (7.39). Вместо этого, тест отношения правдоподобия представляет собой смесь хи-квадратов (см. раздел 8.5.3).

\begin{center}
Удобство
\end{center}

Удобство расчётов также стоит принимать во внимание. Тест отношения правдоподобия требует оценки модели дважды, с ограничениями нулевой гипотезы и без них. Если расчёт производится статистическим пакетом, это можно легко реализовать, так как пакет выдаст значения для логарифмов фунций правдоподобия, надо вычесть один из другого и умножить на два. Тест Вальда требует оценивания только при альтернативной гипотезе, и его лучше всего использовать, когда легко оценить неограниченную модель. Например, этому соответствует случай ограничений на параметры условного математического ожидания в нелинейных моделях, таких как НМНК, пробит, Тобит и логит. Статистика теста множителей Лагранжа требует оценивания только при нулевой гипотезе, и лучше всего её использовать, когда легче всего оценить ограниченную модель. Примерами могут послужить тесты на автокорреляцию и гетероскедастичность, где легче всего оценить модель, которая получается при нулевой гипотезе и которая не имеет этих усложнений.

Тест Вальда часто используется для тестов на значимость, в то время как тест множителей Лагранжа часто используется для теста на верную спецификацию модели.

\subsection{Интерпретация и расчёт теста множителей Лагранжа}

Тест множителей Лагранжа имеет дополнительные преимущества из-за простой интерпретации в ряде ведущих примеров и расчёта с помощью вспомогательной регрессии.

В этом разделе будет рассматриваться стандартный случай пространственных данных со скалярной независимой по $i$ переменной такой, что $\partial{\ln(\theta)}/\partial{\theta} = \sum_i s_i(\theta)$, где 
\begin{equation}
s_i(\theta) = \frac{\partial{\ln f(y_i|x_i,\theta)}}{\partial{\theta}}
\end{equation}
отражает вклад $i$-ого наблюдения в скор-вектор неограниченной модели. Из (7.25) тест множителей Лагранжа --- тест $\sum_i s_i(\tilde{\theta}_r)$ на близость к нулю.

\begin{center}
Простая интерпретация теста множителей Лагранжа
\end{center}

Предположим, что плотность такова, что $s(\theta)$ факторизуется как
\begin{equation}
s(\theta) = g(x, \theta)r(y, x, \theta)
\end{equation}
для некоторой векторной функции $g(\cdot)$ размера $q \times 1$ и скалярной функции $r(y, x, \theta)$, которую можно интерпретировать как обобщённый остаток, так как $y$ присутствует в $r(\cdot)$, но не в $g(\cdot)$. Например, для регрессии Пуассона $\partial{\ln f}/\partial{\theta} = x(y - \exp(x'\beta))$.

\begin{center}
Тесты на проверку гипотез
\end{center}

Учитывая (7.32) и независимость по $i$, $\partial{\ln L}/\partial{\theta}|_{\tilde{\theta}_r} = \sum_i \tilde{g}_i\tilde{r}_i$, где $\tilde{g}_i = g(x_i, \tilde{\theta}_r)$ и $\tilde{r}_i = r(y_i, x_i, \tilde{\theta}_r)$. Таким образом, тест множителей Лагранжа можно легко интерпретировать как скор-тест на корреляцию между $\tilde{g}_i$ и остатками $\tilde{r}_i$. Эта интерпретация была приведена в разделе 7.3.2 про тест множителей Лагранжа в случае регрессии Пуассона, где $\tilde{g}_i = x_i$ и $\tilde{r}_i = y_i - \exp(x_{1i}'\tilde{\beta}_1)$.

Выражение из (7.32) будет напоминать о себе всякий раз, когда $f(y)$ будет основан на однопараметрической плотности. В частности, многие распространённые модели правдоподобия основываются на плотностях из экспоненциалього семейства с единственным параметром $\mu$, который затем моделируется как функция от $x$ и $\beta$. В случае экспоненциального семейства  $r(y, x, \theta) = (y - \E[y|x]) $ (см. раздел 5.7.3), поэтому обобщённые остатки $r(\cdot)$ из (7.32) являются обычными остатками.

В более общем случае выражение, аналогичное (7.32), также возникает, когда $f(y)$ основан на двухпараметрической плотности, информационная матрица которой имеет блочно-диагональный вид для этих двух параметров, а эти два параметра зависят от регрессоров и векторов параметров $\beta$ и $\alpha$, которые отличаются друг от друга. Тогда тест множителей Лагранжа на $\beta$ --- тест на корреляцию между $\tilde{g}_{\beta_i}$ и $\tilde{r}_{\beta_i}$, где $s(\beta) = g_{\beta}(x, \theta)r_{\beta}(y, x, \theta)$ с той же самой интерпретацией теста множителей Лагранжа на уровне значимости $\alpha$.

Ярким примером является линейная регрессия с нормальным распределением остатков и с двумя параметрами $\mu$ и $\sigma^2$, которые представлены так: $\mu = x'\beta$ и $\sigma^2 = \alpha$ или $\sigma^2 = \sigma^2(z, \alpha)$. Для исключающих ограничений в линейной регрессии при условии распределённых по нормальному закону остатков $s_i(\beta) = x_i(y_i - x_i'\beta)$ и тест множителей Лагранжа --- тест на корреляцию между регрессорами $x_i$, а $\tilde{u}_i = y_i - x_{1i}'\tilde{\beta}_1$ --- остатки из ограниченной модели. Для тестов на гетероскедастичность с $\sigma^2_i = \exp(\alpha_1 + z_i'\alpha_2)$, $s_i(\alpha) = \frac{1}{2}z_i(((y_i - x_i'\beta)^2/\sigma^2_i) - 1)$. Тест множителей Лагранжа --- это тест на корреляцию между $z_i$, и квадраты остатков $\tilde{u}^2_i = (y_i - x_i'\tilde{\beta})^2$, так как $\sigma^2_i$ постоянна при нулевой гипотезе о том, что $\alpha_2 = 0$.

\begin{center}
Вариант теста множителей Лагранжа с внешним произведением градиента
\end{center}

Теперь вернёмся к общему случаю $s_i(\theta)$, который был определён в (7.31). Как будет показано далее, асимптотически эквивалентную версию статистики теста множителей Лагранжа (7.25) можно получить с помощью вспомогательной или искусственной регрессии
\begin{equation}
1 = \tilde{s}_i'\gamma + v_i,
\end{equation}
где $\tilde{s}_i = s_i(\tilde{\theta}_r)$. Теперь получается
\begin{equation}
LM^* = NR_u^2,
\end{equation}
где $R_u^2$ --- это нецентрированный $R^2$, который был определён в (7.36). $LM^*$ имеет асимптотическое $\chi^2(h)$ распределение при нулевой гипотезе. Это равносильно тому, $LM^*$ равна $ESS_u$, нецентрированная объясняемая сумма квадратов (сумма квадратов оценённых значений), или равна $N - RSS$, где $RSS$ --- сумма квадратов остатков из регрессии (7.33).

Этот результат можно распространить и на другие случаи, так как во многих приложениях может быть довольно легко аналитически получить $s_i(\theta)$, сгенерировать данные для $q$ компонент $\tilde{s}_{1i}, \dots,\tilde{s}_{qi}$ и регрессировать 1 на $\tilde{s}_{1i}, \dots,\tilde{s}_{qi}$. Заметьте, что здесь $f(y_i| x_i, \theta)$ из (7.31) --- это плотность неограниченной модели.

Для исключающих ограничений в модели Пуассона был приведён пример в разделе 7.3.2, $s_i(\beta) = (y_i - \exp(x_i'\beta))x_i$ и $x_i'\tilde{\beta}_r = x_{1i}'\tilde{\beta}_{1r}$. Отсюда следует, что можно посчитать $LM^*$ как $NR_u^2$, регрессируя 1 на $(y_i - \exp(x_{1i}'\tilde{\beta}_{1r}))x_i$, где $x_i$ содержит $x_{1i}$ и $x_{2i}$, а $\tilde{\beta}_{1r}$ получается из регрессии Пуассона, когда $y_i$ регрессируют только на $x_{1i}$.

Уравнения (7.33) и (7.34) требуют только независимость по $i$. Можно проводить другие вспомогательные регрессии только при введении предположений о дальнейшей структуре. В частности, можно рассмотреть случаи, когда $s(\theta)$ факторизуется, как в (7.32), и задать $r(y, x, \theta)$ так, чтобы $\V[r(y, x, \theta)] = 1$. Тогда альтернативной асимптотически эквивалентной версией теста множителей Лагранжа будет $NR_u^2$ из регрессии $\tilde{r}_i$ на $\tilde{g}_i$. Это включает в себя тесты множителей Лагранжа для линейной регрессии с нормальным распределением остатков, например, тест множителей Лагранжа Бройша-Пагана на гетероскедастичность.

Эти альтернативные версии теста множителей Лагранжа называются тестами множителей Лагранжа с внешним произведением градиента, так как в них $-A_0$ из (7.22) заменяют на оценки, полученные с помощью внешнего произведения градиента, или на BHHH оценки $B_0$. Хотя они легко вычисляются, варианты тестов множителей Лагранжа с внешним произведением градиента могут иметь плохие свойства для малых выборок с большим числом искажений. По этой причине многие отказываются от этого варианта теста множителей Лагранжа. Проблемы такого рода могут быть значительно уменьшены с помощью метода бутстрэп (см. раздел 11.6.3). Дэвидсон и МакКиннон (1984) предлагают вспомогательные регрессии двойной длины, которые также показывают лучшие результаты на конечных выборках.

\subsection{Вывод варианта с внешним произведением градиента}

Чтобы вывести $LM^*$, сначала заметим, что в (7.25), $\partial{\ln L(\theta)/\partial{\theta}}|_{\tilde{\theta}_r} = \sum \tilde{s}_i$. Во-вторых, с помощью равенства информационных матриц $A_0 = - B_0$ можно получить состоятельную оценку $B_0$ из раздела 5.5.2 при нулевой гипотезе с помощью оценки с внешним произведением градиента или BHHH оценки $N^{-1}\sum \tilde{s}_i\tilde{s}_i'$. Объединение этих результатов даёт асимптотически эквивалентную версию статистики теста множителей Лагранжа (7.25):
\begin{equation}
LM^* = \begin{pmatrix} \sum_{i=1}^n \tilde{s}_i' \end{pmatrix} \begin{bmatrix} \sum_{i=1}^n \tilde{s}_i' \end{bmatrix}^{-1} \begin{pmatrix} \sum_{i=1}^n \tilde{s}_i \end{pmatrix}.
\end{equation}

Это тестовая статистика может быть рассчитана из вспомогательной регрессии 1 на $\tilde{s}_i$. Определим $S$ как матрицу размера $N \times q$ с $i$-ым рядом $\tilde{s}_i'$ и определим $I$ как вектор из единиц размера $N \times 1$. Тогда
\begin{equation}
LM^* = I'S[S'S]^{-1}S'I = ESS_u = NR_u^2
\end{equation}
В целом для регрессии $y$ на $X$ нецентрированная объяснённая сумма квадратов $(ESS_u)$ имеет вид: $y'X(X'X)^{-1}X'y$, что в точности имеет вид (7.36), в то время как нецентрированный $R^2$ является $R_u^2 = y'X(X'X)^{-1}X'y/y'y$, который здесь (7.36) делится на $I'I = N$. Термин <<нецентрированный>> используется, потому что в $R_u^2$ имеет место деление на сумму квадратов отклонений $y$ в окрестности нуля, а не вокруг выборочного среднего.

\section{Пример тестов, основанных на методе максимального правдоподобия}

Различные тестовые процедуры --- тест Вальда, тест отношения правдоподобия, тест множителей Лагранжа --- проиллюстрированы, используя сгенерированные данные. Использовался процесс порождающий данные $y|x$, распределённые по Пуассону с математическим ожиданием $\exp(\beta_1 + \beta_2 x_2 + \beta_3 x_3 + \beta_4 x_4)$, где $\beta_1 = 0$  и $\beta_2 = \beta_3 = \beta_4 = 0.1$ и три регрессора независимы и одинаково распределены по нормальному закону $N[0,1]$.

\begin{table}[h]
\begin{center}
\caption{\label{tab:pred}Тестовые статистики для примера регрессии Пуассона}
\begin{minipage}{16.5cm}
\begin{tabular}[t]{l*{7}{{c}}}
\hline
\hline
  & \multicolumn{4}{c}{\bf{Тестовая статистика}} &  &  \bf{Результат} \\
\bf{Нулевая гипотеза}\footnote{Процесс порождающий данные для $y$ --- распределение Пуассона с параметром $\exp(0.0 + 0.1x_2 + 0.1x_3 + 0.1x_4)$ и размером выборки $N = 200$. Тестовые статистики приведены с соответствующими $p$-значениями в скобках. Тесты на проверку второй гипотезы имеют распределение $\chi^2(2)$, а другие тесты распределены по $\chi^2(1)$. Логарифмы функций правдоподобия для ограниченной модели также приведены; логарифм функции правдоподобия для неограниченной модели равен $-238,772$.} & \bf{Wald} & \bf{LR} & \bf{LM} & \bf{LM*} & \bf{$\ln L$} & \bf{на 5\% (ур. зн.)} \\
\hline
$H_{10}: \beta_3 = 0$ & 5.904 & 5.754 & 5.916 & 6.218 & -241.648 & Отвергнуть \\
 & (0.015) & (0.016) & (0.016) & (0.013) &  &  \\
$H_{20}: \beta_3 = 0, \beta_4 = 0$ & 8.570 & 8.302 & 8.575 & 9.186 & -242.922 & Отвергнуть \\
 & (0.014) & (0.016) & (0.014) & (0.010) &  &  \\
$H_{30}: \beta_3 - \beta_4 = 0$ & 0.293 & 0.293 & 0.293 & 0.315 & -238.918 & Не отвергать \\
 & (0.588) & (0.589) & (0.588) & (0.575) &  &  \\
$H_{30}: \beta_3 / \beta_4 - 1 = 0$ & 0.158 & 0.293 & 0.293 & 0.315 & -238.918 & Не отвергать \\
 & (0.691) & (0.589) & (0.588) & (0.575) &  &  \\
\hline
\hline
\end{tabular}
\end{minipage}
\end{center}
\end{table}

Регрессия Пуассона, когда $y$ регрессируется на константу, $x_2, x_3, x_4$ для сгенерированной выборки размером в 200 наблюдений в случае метода максимального правдоподобия для неогранченной модели даёт
\[
\hat{\E}[y|x] = \exp(-\underset{(-2.14)}{0.165} - \underset{(-0.36)}{0.028}x_2 + \underset{(2.43)}{0.163}x_3 + \underset{(0.08)}{0.103}x_4),
\]
где в скобках внизу указаны значения $t$-статистик, а значение логаримфма функции правдоподобия для неограниченной модели равно $-238.772$.

Сравнение тестов четырёх различных гипотез описано в первой колонке таблицы 7.1. Оценка является нелинейной, тогда как гипотезы являются примерами одного исключающего ограничения, нескольких исключающих ограничений, линейных ограничений и нелинейных ограничений соответственно. В оставшейся части таблицы приведены четыре асимптотически эквивалентные тестовые статистики для этих гипотез и соответствующие им $p$-значения. Для данной выборки все тесты отвергают первые две гипотезы и не отвергают две оставшиеся на уровне значимости 0.05.

Статистика теста Вальда вычисляется с использованием (7.23). Для этого оценивается неограниченная модель, учитывая ранее, что необходимо получить оценку ковариационной матрицы для неограниченной модели с помощью метода максимального правдоподобия. Тесты Вальда на проверку различных гипотез затем требуют расчёта различных $h$ и $R$ и упрощения в некоторых случаях. Хи-квадрат тест Вальда на одно исключающее ограничение --- это просто квадрат обычного $t$-теста, с $2.43^2 \simeq 5.90$. Статистика теста Вальда на совместные исключающие ограничения подробно описана в разделе 7.2.5. Здесь $x_3$ является статистически значимым и $x_4$ является статистически незначимым, в то время как совместно $x_3$ и $x_4$ статистически значимы на уровне значимости 0.05. Тест Вальда для третьей гипотезы приведён в (7.19) и приводит к тому, что гипотеза не отвергается. Третья и четвёртая гипотезы эквивалентны, поскольку из $\beta_3/\beta_4 - 1 = 0$ следует, что $\beta_3 = \beta_4$, но тестовая статистика Вальда для четвёртой гипотезы, которая приведена в (7.13), отличается от (7.19). Тестовая статистика (7.13) была рассчитана с использованием матричных операций, так как большинство статистических пакетов в лучшем случае рассчитает тест Вальда для линейных гипотез.

Статистику теста отношения правдоподобия особенно легко вычислить, используя (7.21), учитывая оценивание ограниченной модели. Для трёх первых гипотез ограниченная модель оценена с помощью регрессии Пуассона, когда $y$ регрессируется на регрессоры $(1, x_2, x_4)$, $(1, x_2)$ и $(1, x_2, x_3 + x_4)$ соответственно, где в третьей регрессии используется $\beta_3 x_3 + \beta_4 x_4$, если $\beta_3 = \beta_4$. В качестве примера теста отношения правдоподобия для второй гипотезы $LR = - 2 [-238.772 - (-242.922)] = 8.30$. Четвёртая ограниченная модель в теории требует оценивания методом максимального правдоподобия при нелинейном ограничении на параметры. Довольно немного статистических пакетов могут это реализовать. Тем не менее, ограниченное оценивание методом максимального правдоподобия инвариантно к тому, как выражаются ограничения, так что здесь получаются те же самые оценки, что и для третьей ограниченной модели, которая приводит к той же самой статистике теста отношения правдоподобия.

Статистика теста множителей Лагранжа вычисляется с помощью (7.25), которая преобразуется для модели Пуассона в (7.27). Эта статистика вычисляется с использованием матричных операций с различными ограничениями, которые ведут к различным  оценкам $\tilde{\beta}$, полученным с помощью метода максимального правдоподобия для ограниченной модели. Что касается теста множителей Лагранжа, то он инвариантен к преобразованиям, поэтому этот тест для третьей и четвёртной гипотезы эквивалентен.

Асимптотически эквивалентной версией статистики теста множителей Лагранжа является модифицированная статистика теста множителей Лагранжа, которая приведена в (7.35). Она может быть вычислена как объясняемая сумма квадратов из вспомогательной регрессии (7.33). Для модели Пуассона $s_{ji} = \partial{\ln f(y_i)}/\partial{\beta}_j = (y_i - \exp(x_i'\beta))x_{ji}$ при оценивании с помощью метода максимального правдоподобия для ограниченной модели, которая зависит от рассматриваемых гипотез. Модифицированная статистика теста множителей Лагранжа вычисляется проще, чем стандартная статистика теста множителей Лагранжа, хотя они обе требуют оценок, полученных методом максимального правдоподобия для ограниченной модели.

В этом примере с сгенерированными данными значения различных тестовых статистик очень похожи. Это не всегда так. В частности, модифицированная статистика теста множителей Лагранжа может иметь более плохие результаты на конечной выборке, чем стандартный тест множителей Лагранжа, даже если процесс порождающий данные известен. Кроме того, в приложениях к реальными данными процесс порождающий данные вряд ли будет хорошо специфицирован, что приводит к расходимости различных тестовых статистик даже для бесконечно больших выборок.

\section{Тесты без использования метода максимального правдоподобия}

Тест Вальда является стандартным тестом, который не использует метод максимального правдоподобия. Из раздела 7.2 следует, что это стандартная тестовая процедура, которая всегда может быть реализована с использованием соответствующей сэндвич оценки ковариационной матрицы оценок параметров. Единственным ограничением является то, что в некоторых случаях оценки для неограниченной модели может быть гораздо сложнее получить, чем оценки для ограниченной модели.

Тест множителей Лагранжа или скор-тест, основанный на отклонении от нуля вектора-градиента из неограниченной модели, оценённой с помощью оценок ограниченной модели, также может быть обобщён на оценки, которые получены не методом максимального правдоподобия. Однако вид теста множителей Лагранжа, как правило, в этом случае значительно сложнее, чем в случае метода максимального правдоподобия. Кроме того, простейшие формы статистики теста множителей Лагранжа, основанные на вспомогательных регрессиях, как правило, не устойчивы к неверной спецификации распределения.

Тест отношения правдоподобия основан на разности между максимальными значениями целевой функции с и без ограничений. Это обычно не обобщается на все целевые функции, кроме функции правдоподобия, так как эта разность, как правило, не имеет хи-квадрат распределение.

Для полноты картины мы приводим расширение тестов множителей Лагранжа для $m$-оценок и для эффективных оценок, полученных методом моментов. Как уже отмечалось, в большинстве случаев достаточно использовать более простой тест Вальда.

\subsection{Тесты, основанные на $m$-оценках}

Тесты для $m$-оценок являются расширенными вариантами тестов для оценок, полученных методом максимального правдоподобия. Однако для них уже невозможно использовать равенство информационных матриц для упрощения тестовой статистики, а также тест отношения правдоподобия может быть обобщён только в особых случаях. Результирующая тестовая статистика имеет асимптотическое $\chi^2(h)$ распределение при $H_0: h(\theta) = 0$, и она имеет то же нецентральное хи-квадрат распределение при локальных альтернативах.

Рассмотрим $m$-оценки, которые получаются из максимизации $Q_N(\theta) = N^{-1}\sum_i q_i(\theta)$ из условия первого порядка $N^{-1}\sum_i s_i(\theta) = 0$. Пусть матрица $A(\theta) = N^{-1}\sum_i \partial{s_i(\theta)}/\partial{\theta}'$ и $B(\theta) = N^{-1}\sum_i s_i(\theta)s_i(\theta)'$ --- матрицы размера $q \times q$, а $R(\theta) = \partial{\ln h(\theta)}/\partial{\theta}'$ --- матрица размера $h \times q$. Пусть $\hat{\theta}_u$ и $\tilde{\theta}_r$ --- это оценки для ограниченной и неограниченной моделей соответственно, и заданы $\hat{A} = A(\hat{\theta}_u)$ и $\tilde{A}= A(\tilde{\theta}_r)$ для $B$ и $R$ соответственно. Также пусть $\hat{h} = h(\hat{\theta}_u)$ и $\tilde{s}_i = s_i(\tilde{\theta}_r)$.

Тестовая статистика Вальда основана на близости $\hat{h}$ к нулю. Здесь
\begin{equation}
W = \hat{h}'[\hat{R}N^{-1}\hat{A}^{-1}\hat{B}\hat{A}^{-1}\hat{R}']^{-1}\hat{h},
\end{equation}
так как из раздела 5.5.1 оценка ковариационной матрицы с поправкой $\hat{\theta}_u$ равна $N^{-1}\hat{A}^{-1}\hat{B}\hat{A}^{-1}$.

Статистические пакеты со встроенной опцией скорректированных стандартных ошибок используют более общий вид для расчёта тестов Вальда на статистическую значимость.

Пусть $g(\theta) = \partial{\ln Q_N(\theta)}/\partial{\theta}$ обозначает вектор-градиент и $\tilde{g} = g(\tilde{\theta}_r) = \sum_i \tilde{s}_i$. Статистика теста множителей Лагранжа основана на близости $\tilde{g}$ к нулю и имеет вид
\begin{equation}
LM = N\tilde{g}'\begin{bmatrix} \tilde{A}^{-1} \tilde{R}'\begin{pmatrix}\tilde{R} \tilde{A}^{-1} \hat{B}\tilde{A'}^{-1}\tilde{R}'\end{pmatrix}^{-1}\tilde{R}\tilde{A}^{-1}\end{bmatrix}^{-1}\tilde{g}.
\end{equation}

Этот результат получен с помощью хи-квадрат тестовой статистики, основанной на (7.29), где $N\tilde{g}$ заменяет $|\partial{\ln L}/\partial{\theta}|_{\tilde{\theta}_r}$. Этот тест не так просто реализовать в качестве теста Вальда с поправкой. Некоторые примеры вычисления тестов метода Лагранжа с поправкой приведены в разделе 8.4.
Стандартные тесты множителей Лагранжа в статистических пакетах часто не являются версиями теста множителей Лагранжа с поправкой.

Тест отношения правдоподобия нелегко обобщить. Его можно обобщать дл $m$-оценок, если $B_0 = -\alpha A_0$  для некоторого скалярного $\alpha$, это более слабая версия равенства информационных матриц. В таких особых случаях статистика теста отношения квази-правдоподобия равна
\begin{equation}
QLR = -2N[Q_N(\tilde{\theta}_r) - Q_N(\hat{\theta}_u) ]/\hat{\alpha}_u,
\end{equation}
где $\hat{\alpha}_u$ является состоятельной оценкой, полученной из оценивания неограниченной модели (см. Вулдридж, 2002, с. 370). Условие $B_0 = -\alpha A_0$ справедливо для обобщённых линейных моделей (см. раздел 5.7.4). Тогда статистика теста отношения квази-правдоподобия эквивалентна разнице вариаций для ограниченной и неограниченной моделей, обобщённому $F$-тесту, который основан на разности между суммами квадратов остатков ограниченной и неограниченной моделей для МНК и НМНК оценок с гетероскедастичными ошибками. Для общего случая оценивания с помощью метода максимального правдоподобия, где $B_0 \not= - \alpha A_0$, статистика теста отношения правдоподобия может быть распределена как взвешенная сумма хи-квадратов (см. раздел 8.5.3).

\subsection{Тесты, основанные на эффективных оценках обобщенного метода моментов}

Простейшим вариантом различных тестовых статистик общённого метода моментов является применение эффективных оценок, полученных обобщённым методом моментов, что означает оценивание с помощью оптимальных взвешенных матриц. Это не накладывает больших практических ограничений, так как оптимальная взвешенная матрица всегда может быть оценена, как показано в разделе 6.3.5.

Рассмотрим оценивание с помощью обобщённого метода моментов, основанного на условии, что $\E(m_i) = \theta$. (Обратите внимание на другое обозначение по сравнению с главой 6: $H(\theta)$ используется в текущей главе для
обозначения ограничений, накладываемых нулевой гипотезой).

Используя обозначения, введённые в разделе 6.3.5, эффективная оценка, полученная обобщённым методом моментов в случае неограниченной модели, $\hat{\theta}_u$ получается из минимизации $Q_N(\theta) = g_N(\theta)'S_N^{-1}g_N(\theta)$, где $g_N(\theta) = N^{-1}\sum_i m_i(\theta)$ и $S_N$ состоятельна для $S_0 = \V[g_N(\theta)]$. Предполагается, что оценка, полученная обобщённым методом моментов в случае ограниченной модели, $\tilde{\theta}_r$ является минимумом $Q_N(\theta)$ с той же взвешенной матрицей $S_N^{-1}$ при ограничении $h(\theta) = 0$.

Три следующие ниже тестовые статистики, описанные Ньюи и Вестом (1987а), имеют асимптотическое $\chi_2(h)$ распределение при $H_0: h(\theta) = 0$ и имеют то же нецентральное хи-квадрат распределение при локальных альтернативах.

Статистика теста Вальда, как и в стандартном случае, основана на близости $\hat{h}$ к нулю, что даёт
\begin{equation}
W = \hat{h}'\begin{bmatrix} \hat{R}N^{-1}\begin{pmatrix} \hat{G}'S^{-1}\hat{G} \end{pmatrix}^{-1} \hat{R}'\end{bmatrix}^{-1}\hat{h},
\end{equation}
так как вариация эффективной оценки, полученной с помощью обобщённого метода моментов, равна $N^{-1}(\hat{G}'S^{-1}\hat{G})^{-1}$ из раздела 6.3.5, где $G_N(\theta) = \partial{g_N(\theta)}/\partial{\theta}'$ и крышка означает оценивание в точке $\hat{\theta}_u$.

Условие первого порядка для этой оценки равно $\hat{G}'S^{-1}\hat{g} = 0$. Статистика тестов множителей Лагранжа, проверяющая близок ли вектор-градиент к нулю, оценивается в точке $\tilde{\theta}_r$, что даёт
\begin{equation}
LM = N\tilde{g}'S^{-1}\tilde{G}(\tilde{G}'S^{-1}\tilde{G})^{-1}\tilde{G}'S^{-1}\tilde{g},
\end{equation}
где тильда обозначает оценивание в точке $\tilde{\theta}_r$ и мы используем предположение из раздела 6.3.3 о том, что $\sqrt{N}g_N(\theta_0) \stackrel{d}{\rightarrow} N[0, S_0]$, откуда $\sqrt{N G'S^{-1}g} \stackrel{d}{\rightarrow} N[0, \plim N^{-1}G'S^{-1}G]$.

Для эффективной оценки, полученной с помощью обобщённого метода моментов, можно найти разность значений целевой функции в точках, соответствующих максимуму функции для ограниченной и неограниченной моделей, что даст формулу тестовой статистики
\begin{equation}
D = N[Q_N(\tilde{\theta}_r) - Q_N(\hat{\theta}_u)].
\end{equation}

Так же, как и статистика теста Вальда и статистика теста множителей Лагранжа, статистика $D$ тоже имеет асимптотическое $\chi^2(h)$ распределение при $H_0: h(\theta) = 0$.

Даже в случае применения метода максимального правдоподобия последняя статистика отличается от статистики отношения правдоподобия, так как она использует другую целевую функцию. В методе максимального правдоподобия минимизируется $Q_N(\theta) = - N^{-1}\sum_i \ln f(y_i|\theta)$. Из раздела 6.3.7 в случае асимптотически эквивалентной эффективной оценки, полученной с помощью обобщённого метода моментов, вместо этой функции минимизируется квадратичная форма  $Q_N(\theta) = N^{-1}(\sum_i s_i(\theta))'(\sum_i s_i(\theta))$, где $s_i(\theta) = \partial{\ln f(y_i|\theta)}/\partial{\theta}$. Статистику $D$ можно обобщить при условии, что оценка, полученная с помощью обобщённого метода моментов, является эффективной, тогда как тест отношения правдоподобия может быть только обобщён для некоторых частных случаев $m$-оценок, о которых рассказано после (7.39).

Для оценок, полученных с помощью метода моментов, в модели, которая была описана выше, выполняется равенство $D = LM = NQ_N(\tilde{\theta}_r)$, поэтому тест множителей Лагранжа и тесты разности эквивалентны. Для $D$ можно применить это упрощение, так как $g_N(\hat{\theta}_u) = 0$ и $Q_N(\hat{\theta}_u) = 0$. То, как можно упростить выражение для теста множителей Лагранжа, приведено в (7.41), так как матрица $\tilde{G}_N$ обратима.

\section{Мощность и размер тестов}

В остальных разделах этой главы исследуются два ограничения использования статистических пакетов для проверки гипотез.

Во-первых, тест может иметь низкую способность отличать нулевую гипотезу от альтернативной гипотезы. В этом случае тест имеет низкую мощность, то есть существует низкая вероятность отвергнуть нулевую гипотезу, когда она неверна. Стандартный результат, который выдаёт компьютер, не рассчитывает мощность теста, но её можно оценить с помощью асимптотических методов (см. данный раздел) или с помощью метода Монте-Карло для конечных выборок (см. раздел 7.7). Если главный вклад эмпирической работы заключается в том, отвергается ли конкретная гипотеза или нет, то нет никаких оснований для того, чтобы не привести дополнительно мощность теста против некоторых важных альтернативных гипотез.

Во-вторых, истинный размер теста может существенно отличаться от номинального размера теста, который получен из асимптотической теории. Правило большого пальца состоит в том, что размер выборки $N > 30$ является достаточным для того, чтобы асимптотическая теория обеспечила хорошую апроксимацию для вывода для одной переменной, но это правило не распространяется на модели с регрессорами. 

Плохая апроксимация, скорее всего, может быть в хвостах распределения, но хвосты используются для получения критических значений тестов на стандартных уровнях значимости таких, как 5\%. На практике критическое значение тестовой статистики, полученное с помощью апроксимации на большой выборке, часто меньше, чем истинное критическое значение, основанное на неизвестном истинном распределении. Уточнения для малых выборок --- это попытки приблизиться к истинному критическому значению. Для линейной регрессии с нормальным распределением остатков точные критические значения могут быть получены с использованием $t$ вместо $z$ и $F$ вместо $\chi^2$ распределени, но аналогичные результаты могут быть неточными для нелинейной регрессии. Вместо этого уточнения для малых выборок могут быть получены с помощью метода Монте-Карло (см. раздел 7.7) или с помощью метода бутстрэп (см. раздел 7.8 и главу 11).

С современными компьютерами относительно легко исправить размер и исследовать мощность тестов, используемых в прикладных исследованиях. Мы представляем эту тему, которой обычно пренебрегают, более подробно.

\subsection{Размер теста и мощность}

Проверка гипотез приводит либо к тому, что нулевая гипотеза отвергается, либо нет. Правильное решение принимается, если нулевая гипотеза отвергается, когда она неверна, или если она не отвергается, когда она верна. 

Есть также два возможных неправильных решения: (1) если нулевая гипотеза отвергается, когда она верна, то это ошибка первого рода, и (2) если она не отвергается, когда она неверна, то это ошибка второго рода. В идеальном случае вероятности обоих ошибок будут низкими, но на практике уменьшение вероятности ошибки одного рода происходит за счёт увеличения вероятности ошибки другого рода. Классическое решение для тестирования гипотезы --- закрепить вероятность ошибки первого рода на определённом уровне, как правило, 0.05, оставив при этом вероятность ошибки второго рода неуказанной.

Размер теста или уровень значимости определяется так:
\begin{equation}
\alpha = \Pr[\text{ошибка I рода}] = \Pr[ \text{отвергнуть } H_0 | H_0 \text{ верна} ],
\end{equation}
со стандартными вариантами $\alpha$ 0.01, 0.05 или 0.10. Гипотеза отвергается, если значение тестовой статистики попадает в область альтернативной гипотезы, определённую так, чтобы уровень значимости теста равнялся указанному значению $\alpha$. Близкий к этому эквивалентный метод вычисляет $p$-значение теста, предельный уровень значимости, при котором нулевая гипотеза отвергается, и отвергает нулевую гипотезу, если $p$-значение меньше указанного значения $\alpha$. Оба способа требуют только знание распределения тестовой статистики при нулевой гипотезе, которое представлено в разделе 7.2 для статистики теста Вальда.

Также стоит рассмотреть вероятность ошибки второго рода. Мощность теста определяется как
\begin{multline}
\text{Мощность} = \Pr[\text{отвергнуть } H_0|H_a \text{ верна}] = \\
= 1 - \Pr[\text{принять } H_0|H_a \text{ верна}] = 1 - \Pr[\text{ошибка II рода}].
\end{multline}

В идеальном случае мощность теста близка к единице, поскольку тогда вероятность ошибки второго рода близка к нулю. Определение мощности требует знания распределения тестовой статистики при альтернативной гипотезе.

Анализ мощности теста, как правило, не проводится в эмпирических работах. Кроме того, тестовые процедуры, как правило, выбираются таким образом, чтобы заранее из теории было известно, что мощность, которой они обладают, при заданном уровне $\alpha$ является высокой по сравнению с другими альтернативными тестовыми статистиками. В идеальном случае используется равномерно наиболее мощный критерий. Это тест, который имеет наибольшую мощность при заданном уровне $\alpha$ для всех альтернативных гипотез. Равномерно наиболее мощные критерии существуют при проверке простой нулевой гипотезы против простой альтернативной гипотезы. Более того, лемма Неймана-Пирсона даёт результат, что равномерно наиболее мощный критерий является функцией теста отношения правдоподобия. Для более общих ситуаций проверки гипотез, связанных с тестированием сложных гипотез обычно не существует равномерно наиболее мощного критерия. Также накладываются дальнейшие ограничения такие, как равномерно наиболее мощные односторонние тесты. На практике право размышлять о мощности предоставляют эконометристам-теоретикам, которые используют теорию и моделирование к различным процедурам проверки гипотез, чтобы выяснить, какие тесты являются самыми мощными.

Тем не менее, можно определить мощность теста в любом заданном случае. В дальнейшем мы подробно рассмотрим, как вычислить асимптотическую мощность теста Вальда, которая равна мощности теста отношения правдоподобия и теста множителей Лагранжа в полностью параметрическом случае.

\subsection{Локальные альтернативные гипотезы}

Поскольку мощность --- это вероятность отвергнуть нулевую гипотеза, когда альтернативная гипотеза верна, вычисление мощности требует получения распределения тестовой статистики при альтернативной гипотезе. Для хи-квадрат теста Вальда на уровне значимости $\alpha$ мощность равна $\Pr[W > \chi_{\alpha}^2(h)|H_a]$. Расчёт этой вероятности требует уточнения конкретной альтернативной гипотезы, потому что вариант $H_a: h(\theta) \not= 0$ слишком широкий.

Очевидным выбором является фиксированная альтернатива $h(\theta) = \delta$, где $\delta$ --- конечный вектор, состоящий из  ненулевых констант и имеющий размер $q \times 1$. Величину $\delta$ иногда называют ошибкой гипотезы, и большие ошибки гипотезы приводят к большей мощности. Для фиксированной альтернативы статистика теста Вальда асимптотически имеет мощность, равную единице, так как она всегда отвергает нулевую гипотезу. Чтобы убедиться в этом, заметим, что если $h(\theta) = \delta$, то статистика Вальда становится стремится к бесконечности, так как 
\begin{equation}
W = \hat{h}(\hat{R}N^{-1}\hat{C}\hat{R}')^{-1}\hat{h} \stackrel{p}{\rightarrow} \delta'(R_0N^{-1}C_0R_0')^{-1}\delta,
\end{equation}
используя $\hat{\theta} \stackrel{p}{\rightarrow} \theta_0$, так, $\hat{h} = h(\hat{\theta}_u) \stackrel{p}{\rightarrow} h(\theta) = \delta$ и $\hat{C} \stackrel{p}{\rightarrow} C_0$. Отсюда следует, что $W \stackrel{p}{\rightarrow} \infty$, так как все члены, кроме $N$ конечные и отличные от нуля. Это бесконечное значение приводит к тому, что нулевая гипотеза всегда отвергается, как и должно быть. Следовательно, тест имеет абсолютную мощность, равную единице.

Статистика теста Вальда --- это состоятельная тестовая статистика, её мощность стремится к единице при $N \rightarrow \infty$. Как и многие оценки являются состоятельными, так и многие тестовые статистики могут быть состоятельными. Необходимы более строгие критерии для выбора между тестовыми статистиками, в то время как для выбора между оценками используется относительная эффективность.

Для оценок, которые являются $\sqrt{N}$ состоятельными, рассмотрим последовательность локальных альтернатив
\begin{equation}
H_a: h(\theta) = \delta/\sqrt{N},
\end{equation}
где $\delta$ --- вектор, состоящий из конечных констант с $\delta \not= 0$. Эта последовательность альтернативных гипотез, называемая дрейфом Питмана, становится тем ближе к значению нулевой гипотезы, равному нулю, чем больше размер выборки. Причём данное стремление происходит с той же самой скоростью $\sqrt{N}$, который применяется для нормирования оценки $\hat{\theta}$, необходимого для получения невырожденного распределения состоятельной оценки. Следовательно, значение $h(\theta)$ для альтернативной гипотезы стремится к нулю со скоростью, не допускающей  повышения эффективности с увеличением размера выборки. Более подробно о локальных альтернативах написано в книге МакМануса (1991).

\subsection{Асимптотическая мощность теста Вальда}

При последовательности локальных альтернатив (7.45) статистика теста Вальда имеет невырожденное распределение, нецентральное хи-квадрат распределение. Это позволяет определить мощность теста Вальда.

В частности, как показано в разделе 7.7.4, при $H_a$ статистика теста Вальда, приведённая в (7.6), имеет асимптотическое $\chi^2(h; \lambda)$ распределение, где $\chi^2(h; \lambda)$ --- это нецентральное хи-квадрат распределение с параметром нецентральности
\begin{equation}
\lambda = \frac{1}{2}\delta'(R_0C_0R_0')^{-1}\delta,
\end{equation}
где $R_0$ и $C_0$ были заданы в (7.4) и (7.5). Мощность теста Вальда, вероятность отвергнуть нулевую гипотезу, когда локальная альтернативная гипотеза верна, тогда имеет вид:
\begin{equation}
\text{Мощность} = \Pr[W > \chi_{\alpha}^2(h)| W \sim \chi_{\alpha}^2(h; \lambda)].
\end{equation}

График 7.1 отражает мощность против $\lambda$ тестов на проверку скалярных гипотез $(h = 1)$ на часто используемых размерах или уровнях значимости 10\%, 5\% и 1\%. Для $\lambda$, близких к нулю, мощность равна размеру, а для крупных $\lambda$ мощность стремится к одному.

Эти особенности имеют место и для $h > 1$. В частности, мощность монотонно возрастает по параметру нецентральности $\lambda$, который определён в (7.46). Отсюда следует несколько общих результатов.

Во-первых, мощность увеличивается с расстоянием между нулевой и альтернативной гипотезами, так как тогда $\delta$ и, следовательно, $\lambda$ увеличиваются.


\vspace{5cm}



График 7.1. Мощность хи-квадрат теста Вальда с одной степенью свободы для трёх различных размеров тестов с параметром нецентральности от 0 до 20.

Во-вторых, для заданной альтернативы $\delta$ мощность увеличивается вместе с эффективностью оценки $\hat{\theta}$, так как тогда $C_0$ меньше, и, следовательно, $\lambda$ больше.

В-третьих, вместе с увеличением размера теста увеличивается и мощность, а вероятность ошибки второго рода снижается.

В-четвертых, если несколько различных тестовых статистик все имеют $\chi^2(h)$ распределение при нулевой гипотезе и имеют нецентральное $\chi^2(h)$ распределение при альтернативной гипотезе, то предпочтительнее является та тестовая статистика, у которой наибольший параметр нецентральности $\lambda$, так как в этом случае мощность наиболее высокая. Кроме того, два теста, которые имеют тот же параметр нецентральности, асимптотически эквивалентны при локальных альтернативах.

Наконец, в реальных приложениях мощность можно рассчитать как функцию от $\delta$. В особенности, для заданной альтернативы $\delta$ оценку параметра нецентральности $\lambda$ можно вычислить с помощью (7.46), используя оценку параметра $\hat{\theta}$ с соответствующими значениями $\hat{R}$ и $\hat{C}$. Расчёт мощности представлен в разделе 7.6.5.

\subsection{Вывод асимптотической мощности}

Чтобы получить распределение $W$ при альтернативной гипотезе, надо начать с разложения в ряд Тейлора из (7.9). Его можно упростить до
\begin{equation}
\sqrt{N}h(\hat{\theta}) \stackrel{d}{\rightarrow} N[\delta, R_0C_0R_0'],
\end{equation}
при альтернативной гипотезе, так как тогда $\sqrt{N}h(\theta) = \delta$. Таким образом, квадратичная форма, центрированная по $\delta$, будет иметь хи-квадрат распределение при альтернативной гипотезе.

Статистика теста Вальда, определённая в (7.6), образует квадратичную форму, центрированную относительно нуля, и больше не имеет хи-квадрат распределение при альтернативной гипотезе. В общем случае, если $z \sim N[\mu, \Omega]$, где $\rank(\Omega) = h$. Тогда $z'\Omega^{-1}z \sim \chi^2(h; \lambda)$, где $\chi^2(h; \lambda)$ --- это нецентральное хи-квадрат распределение с параметром нецентральности $\lambda = \frac{1}{2}\mu'\Omega^{-1}\mu$. Применение этого результата к (7.48) даёт
\begin{equation}
Nh(\hat{\theta})'(R_0C_0R_0')^{-1}h(\hat{\theta}) \stackrel{d}{\rightarrow} \chi^2(h; \lambda)
\end{equation}
при альтернативной гипотезе, где $\lambda$ опеределено в (7.49).

\subsection{Расчёт асимптотической мощности}

Чтобы прояснить, как мощность меняется с изменением $\delta$, рассмотрим тест на значимость коэффициента в скалярном случае. Тогда параметр нецентральности, который был определён в (7.46), будет иметь вид:
\begin{equation}
\lambda = \frac{\delta^2}{2c} \simeq \frac{(\delta/\sqrt{N})^2}{2(se[\hat{\theta}])^2},
\end{equation}
где возникает приближение из-за оценивания $c$, предельной дисперсии $\sqrt{N}(\hat{\theta} - \theta)$ с помощью $N(se[\hat{\theta}])^2$, где $se[\hat{\theta}]$  --- это стандартная ошибка $\hat{\theta}$.

Рассмотрим хи-квадрат тест Вальда на проверку нулевой гипотезы $H_0: \theta = 0$ против альтернативной гипотезы
\[
H_a: \theta = a \times se[\hat{\theta}],
\]
где $se[\hat{\theta}]$ рассматривается как константа. Тогда $\delta/\sqrt{N}$ из (7.45) равняется $a \times se[\hat{\theta}]$, и (7.50) можно успростить до $\lambda = a^2/2$. Таким образом, тест Вальда имеет асимптотическое $\chi_{\alpha}^2(1; \lambda)$ распределение при альтернативной гипотезе, где $\lambda = a^2/2$.

Из графика 7.1 ясно, что в общем случае для тестов на значимость на уровне значимости 5\%,
если $a = 2$, мощность значительно ниже 0.5, если $a = 4$, мощность около 0.5, и если $a = 6$,
мощность ниже 0.9. Следовательно, пограничный тест на статистическую значимость может иметь низкую мощность против альтернатив, которые находятся на расстоянии нескольких стандартных ошибок от нуля. Интуитивно, если $\hat{\theta} = 2se[\hat{\theta}]$, тогда тест на проверку гипотезы $\theta = 0$ против альтернативной гипотезы $\theta = 4se[\hat{\theta}]$ имеет мощность около 0.5, так как $95\%$-ный доверительный интервал для $\theta$ примерно $(0, 4se[\hat{\theta}])$, если предположить, что значения $\theta = 0$ и $\theta = 4se[\hat{\theta}]$ равновероятны.

В качестве более конкретного примера предположим, что $\theta$ измеряет процентное увеличение заработной платы
в результате программы повышения квалификации и что исследование выявило, что $\hat{\theta} = 6$ с $se[\hat{\theta}] = 4$. Тогда тест Вальда на уровне значимости 5\% приводит к тому, что $H_0$ не будет отвергнута, так как $W = (6/4)^2 = 2.25 < \chi_{0.05}^2(1) = 3.96$. Исследование будет часто делать вывод, что программа повышения квалификации не является статистически значимой. Однако не надо воспринимать это как то, что существует высокая вероятность того, что программа не влияет, так как этот тест имеет низкую мощность. Например, предыдущий анализ показывает, что тест на проверку гипотезы $H_0: \theta = 0$ против альтернативной гипотезы $H_a: \theta = 16$, что говорит об относительно большом эффекте программы, имеет мощность, равную лишь 0.5, так как $4 \times se[\hat{\theta}] = 16$. Причины низкой мощности включают в себя небольшой размер выборки, большую дисперсию ошибок модели и малый разброс регрессоров.

В простых случаях можно решить обратную задачу оценивания минимального размера выборки, необходимого для достижения заданного желаемого уровня мощности. Это особенно популярно в медицинских исследованиях. 

Эндриус (1989) даёт более формальное изложение способов применения параметра нецентральности для определения значений параметров, при которых в практических ситуациях тесты могут иметь низкую мощность. Он предоставляет множество практических примеров, где можно легко определить, что тесты имеют низкую мощность против значимых альтернатив.

\section{Метод Монте-Карло}

Наше обсуждение статистических выводов до сих пор полагалось на асимптотические результаты. Для малых выборок аналитические результаты редко доступны, помимо тестов на линейные ограничения в линейной регрессионной модели при условии нормального распределения остатков. Однако результаты для малых выборок могут быть получены с помощью метода Монте-Карло.

\subsection{Обзор}

Приведём пример исследования свойств тестовой статистики для малых выборок с помощью метода Монте-Карло. Установим размер выборки $N = 40$ и, например, случайным образом сгенерируем 10 000 выборок размером 40 для модели, получаемой при нулевой гипотезе. Для каждой выборки выпишем тестовую статистику и проверим нулевую гипотезу, которая отвергается, если значение тестовой статистики попадает в область альтернативной гипотезы, которая, как правило, определяется с помощью асимптотических результатов.

Истинный или фактический размер тестовой статистики --- это доля выборок, для которых тестовая статистика попадает в область альтернативной гипотезы. В идеальном случае он близок к номинальному размеру, который является выбранным уровнем значимости теста. Например, при тестировании на уровне 5\% номинальный размер теста равен 0.05, и ожидается, что истинный размер близок к 0.05.

Определение мощности теста для небольших выборок требует дополнительного моделирования с выборками, сгенерированными при одной или нескольких конкретных спецификациях возможных моделей, которые заключены в композитной альтернативной гипотезе. Мощность вычисляется как доля выборок, на которых нулевая гипотеза отвергается, используя либо тот же самый метод, который применяется для определения истинного размера, либо версию теста с поправкой на размер выборки. Эта версия использует область альтернативной гипотезы, то есть ту, где номинальный размер равен истинному размеру.

Исследование с помощью метода Монте-Карло просто реализовать, но есть много тонкостей, используемых при моделировании хорошего исследования Монте-Карло. Отличный обзор представлен в книге Дэвидсона и МакКиннона (1993).


\subsection{Особенности метода Монте-Карло}

В качестве примера исследования с помощью метода Монте-Карло мы рассматриваем статистическое влияние на коэффициент наклона в пробит-модели. Приведённый ниже анализ не опирается на знание пробит-модели. Процесс порождающий данные --- это пробит-модель с бинарным регрессором $y$, который равен одному с вероятностью
\[
\Pr[y = 1|x] = \Phi(\beta_1 + \beta_2 x),
\]
где $\Phi(\cdot)$ --- стандартная нормальная функция распределения, $x \sim N[0,1]$ и $(\beta_1,\beta_2) = (0, 1)$.

Для этого процесса порождающего данные нетрудно сгенерировать данные $(y,x)$. Регрессор $x$ сначала получают случайным образом с помощью стандартного нормального распределения. Затем из раздела 14.4.2 зависимая переменная $y$ устанавливается равной 1, если $x + u > 0$, и устанавливается равной 0 в противном случае, где $u$ получено случайным образом с помощью стандартного нормального распределения. Для этого процесса порождающего данные $y = 1$ примерно в половине случаев
и $y = 0$ для другой половины.

В каждой модели есть $N$ новых наблюдений $x$ и $y$. С помощью метода максимального правдоподобия получаются оценки для пробит-регрессии $y$ на $x$. В качестве альтернативы можно использовать те же $N$ вариантов регрессора $x$ для каждого моделирования и только менять $y$. Первый способ соответствует простой случайной выборке, а второй соответствует анализу при заданном $x$, или <<фиксированному в повторяемых экспериментах>>, см. раздел 4.4.7.

В метода Монте-Карло часто рассматривают различные размеры выборок. Здесь мы устанавливаем $N = 40$. Можно также проверять на выборках с очень большим значением $N$, например, $N = 10 000$, так как тогда результаты, полученные с помощью метода Монте-Карло, дают результаты, близкие к асимптотическим.

Для определения истинного размера теста необходимо провести проверку на большом числе выборок, поскольку размер зависит от поведения в хвостах распределения, а не в центре. Если проведено $S$ экспериментов для поиска истинного размера теста $\alpha$, то доля случаев, когда нулевая гипотеза была отвергнута верно, является результатом $S$ биномиальных экспериментов с математическим ожиданием $\alpha$ и дисперсией $\alpha(1 - \alpha)/S$. Так, $95\%$ экспериментов Монте Карло дадут результат, что тест размера должен быть в интервале $\alpha \pm 1.96 \sqrt{\alpha(1 - \alpha)/S}$. Недостаточно провести 100 экспериментов, так как, например, этот интервал может иметь вид $(0.007, 0.093)$ при $\alpha = 0.05$. Для 10 000 экспериментов $95\%$ интервал будет гораздо более точным $(0.008, 0.012)$, $(0.046, 0.054)$, $(0.094, 0.106)$ и $(0.192, 0.208)$ при $\alpha$, равной $0.01$, $0.05$, $0.10$ и $0.20$ соответственно.

Проблема, которая может возникнуть при использовании метода Монте-Карло, заключается в том, что для некоторых выборок, возможно, нельзя будет оценить модель. Рассмотрим, например, линейную регрессию только на константу и переменную-индикатор. Если переменная-индикатор всегда имеет одно и то же значение, например 0,  то его коэффициент не может быть отделить от коэффициента для константы. Аналогичная проблема возникает в пробит и в других моделях бинарного выбора, если в выборке все $y$-ки равны 0 или все $y$-ки равны 1. Стандартная процедура, которая может быть подвергнута критике, --- отбросить такие выборки и написать компьютерный код, который позволяет повторяет эксперимент при возникновении такой проблемы. В этом примере проблема не возникла для $N = 40$, но она появилась для $N = 40$.

\subsection{Смещение для малых выборок}

Перед тем как перейти к проверке гипотез, мы смотрим на свойства оценки $\hat{\beta}_2$, полученной с помощью метода максимального правдоподобия, и оценки её стандартной ошибки $se[\hat{\beta}_2]$ для малых выборок.

На 10 000 выборках $\hat{\beta}_2$ имела математическое ожидание 1.201 и стандартное отклонение 0.452,
в то время как среднее значение $se[\hat{\beta}_2]$ было 0.359. По этой причине метод максимального правдоподобия даёт смещение в сторону завышения для малых выборок, так как среднее значение $se[\hat{\beta}_2]$ гораздо меньше, чем стандартное отклонение $\hat{\beta}_2$.

\subsection{Размер теста}

Мы рассмотрим двусторонний тест на проверку гипотезы $H_0: \beta_2 = 1$ против альтернативной гипотезы $H_a: \beta_2 \not= 1$, используя тест Вальда
\[
z = W_z = \frac{\hat{\beta}_2 - 1}{se[\hat{\beta}_2]}, 
\]
где $se[\hat{\beta}_2]$ представляет собой стандартную ошибку оценки, полученной методом максимального правдоподобия с использованием ковариационной матрицы, приведённой в разделе 14.3.2, которая равна минус математическое ожидание обратной матрицы Гессе. При данном процессе порождающем данные, $z$ имеет асимптотически стандартное нормальное распределение, и $z^2$ имеет хи-квадрат распределение. Цель состоит в том, чтобы определить, насколько хорошо это апроксимируется к распределению для малых выборок.

График 7.2 отражает плотность для $S = 10 000$ вычисленных значений $z$, где плотность изображается с использованием оценки плотности ядра из главы 9, а не с ипользованием гистограммы. Это накладывается на плотность стандартного нормального распределения. 

Очевидно, что асимптотический результат не является точным, особенно в верхней части, где разница достаточно большая, чтобы привести к искажению размера при тестировании, например, на уровне значимости 5\%. Также $z$ имеет среднее значение $0.114 \not= 0$ и стандартное отклонение $0.956 \not = 1$.

\begin{table}[h]
\begin{center}
\caption{\label{tab:pred} Размер теста Вальда и мощность примера пробит-регрессии}
\begin{minipage}{13cm}
\begin{tabular}[t]{*{4}{{c}}}
\hline
\hline
\bf{Номинальный} & \bf{Истинный} & \bf{Истинная} & \bf{Асимптотическая} \\
\bf{размер $(\alpha)$}\footnote{Процесс, порождающий данные, для $y$ --- пробит с $\Pr[y = 1] = \Phi( 0 + \beta_2 x)$ и размером выборки $N = 40$. Тест --- это двустронний тест Вальда на проверку того, равен ли коэффициент наклона 1 или нет. Истинный размер рассчитан с помощью $S = 10000$ экспериментов при $\beta_2 = 1$. Мощность рассчитана из 10 000 экспериментов при $\beta_2 = 2$} & \bf{размер} & \bf{мощность} & \bf{мощность} \\
\hline
0.01 & 0.005 & 0.007 & 0.272 \\
0.05 & 0.029 & 0.226 & 0.504 \\
0.10 & 0.081 & 0.608 & 0.628 \\
0.20 & 0.192 & 0.858 & 0.755 \\
\hline
\hline
\end{tabular}
\end{minipage}
\end{center}
\end{table}

Первые два столбца таблицы 7.2 отражают номинальный размер и истинный размер теста Вальда для номинальных размеров $\alpha = 0.01$, 0.05, 0.10 и 0.20. Истинный размер --- это доля 10 000 экспериментов, в которых $|z| > z_{\alpha/2}$, что эквивалентно $z^2 > \chi_{\alpha}^2(1)$. Очевидно, что истинный размер теста намного меньше, чем номинальный размер при $\alpha \leq 0.10$. Специальная поправка для малых выборок состоит в предположении, что $z$ имеет $t$-распределение с 38 степенями свободы, и гипотеза отвергается, если $|z| > t_{\alpha/2}(38)$. Однако это приводит к ещё меньшему истинному размеру, так как $t_{\alpha/2}(38) > z_{\alpha/2}$.

Метод Монте-Карло также можно использовать для получения скорректированных на размер критических значений. Таким образом, нижние и верхние 2.5 процентные квантили 10 000 моделируемых значений $z$ равны $- 1.905$ и $2.003$. Отсюда следует, что асиметричная область альтернативной гипотезы с истинным размером будет иметь вид: $z < - 1.905$ и $z > 2.003$. Она больше, чем область альтернативной гипотезы $|z^2| > 1.960$.



\vspace{5cm}




График 7.2 Плотность тестовой статистики Вальда, коэффициент наклона которой равен 1, посчитанный с помощью метода Монте-Карло. Для сравнения изображена плотность стандартного нормального распределения. Данные взяты из модели пробит.

\subsection{Мощность теста}

Мы рассмотрим мощность теста Вальда при альтернативной гипотезе $H_a: \beta_2 = 2$. Мы ожидаем, что мощность будет разумной, так как значение $\beta_2$ лежит в интервале от двух до трёх стандартных ошибок от значения $\beta_2 = 1$ из нулевой гипотезы при условии, что $se[\hat{\beta}_2]$ имеет среднее значение 0.359. Истинная и номинальная мощности теста Вальда приведены в двух последних столбцах таблицы 7.2.

Истинная мощность получается таким же образом, как и истинный размер, который является долей экспериментов из 10 000, в которых $|z| > z_{\alpha/2}$. Единственное отличие состоит в том, что при получении $y$ в моделировании $\beta_2 =2$, а не 1. Истинная мощность очень низкая для $\alpha = 0.01$ и 0.05, то есть для случев, когда истинный размер значительно меньше, чем номинальный размер.

Номинальная мощность теста Вальда определяется с использованием асимптотического нецентрального $\chi^2 (1; \lambda)$ распределения при альтернативной гипотезе, где из (7.50) $\lambda = \frac{1}{2}(\delta \sqrt{N})^2/ se[\hat{\beta}_2]^2 = \frac{1}{2} \times 1^2/0.359^2 \simeq 3.88$, так как локальная альтернатива состоит в том, что  $H_a: \beta_2 - 1 = \delta / \sqrt{N}$, поэтому $\delta / \sqrt{N} = 1$ и $\beta_2 = 2$. Асимптотический результат не является точным, но он даёт полезную оценку мощности для  $\alpha = 0.10$ и 0.20, то есть для случаев, когда истинный размер близок к номинальному размеру.

\subsection{Монте-Карло на практике}

Предыдущее обсуждение делало упор на использование метода Монте-Карло для расчёта мощности и размера теста. Он также может быть очень полезен для определения смещения оценки в малых выборках при установлении больших $N$ для определения того, что на самом деле оценка является состоятельной. Такую процедуру Монте-Карло легко реализовать с использованием современных статистических пакетов.

Метод Монте-Карло может быть применён на реальных данных, если условное распределение $y$ при заданном $x$ полностью параметризовано. Рассмотрим, например, пробит-модель, оценённую на реальных данных. 

В каждой модели регрессоры устанавливаются равными выборочным значениям, если выборка является одной из выборок с фиксированными регрессорами в повторяющихся выборках, и необходимо получить новый набор значений для бинарной зависимой переменной $y$. Это будет зависеть от того, какие значения параметра $\beta$ используются. Пусть $\hat{\beta}_1, \dots, \hat{\beta}_k$ обозначают оценки пробит модели для изначальной выборки и рассмотрим тест Вальда $H_0: \beta_j = 0$. Чтобы вычислить размер теста, проведём $S$ экспериментов при условии $\beta_k = \hat{\beta}_k$ для $j \not= k$ и $\beta_j = 0$, и потом посчитаем количество случаев, когда нулевая гипотеза отвергается. Чтобы посчитать мощность теста Вальда против специфической альтернативной гипотезы $H_a: \beta_j = 1$, сгенерируем $y$ при условии $\beta_k = \hat{\beta}_k$ для $j \not= k$ и $\beta_j = 0$ для генерируемого $y$ и посчтитаем долю случаев, когда нулевая гипотеза отвергается.

На практике большая часть микроэконометрического анализа основывается на оценках, которые не основаны на полностью параметрических моделях. В таком случае необходимы дополные предположения о распределении, для того чтобы проводить эксперименты Монте-Карло. 

Альтернативным способом можно найти мощность с помощью асимптотических методов, а не методов для конечных выборок. Более того, метод бутстрэп, который будет представлен ниже, может быть применён для того, чтобы получить размер теста, используя более сложную асимптотическую теорию. 

\section{Пример метода бутстрэп}

Бутстреп --- это вариант метода Монте-Карло, плюсом которого является то, что его можно применить с меньшим числом параметрических предположений и несложным дополнительным компьютерным кодом, необходимым для оценивания модели. Для того чтобы выполнялись наиболее важные предположения для бутстрэпа, необходимо, чтобы оценки имели предельное распределение и чтобы выборки, получаемые для бутстрэпа, были независимые и одинаково распределённые.

Бутстреп имеет два основных применения. Во-первых, он может быть альтернативным способом для расчёта тестовых статистик без асимптотических предположений. Это особенно полезно для вычисления стандартных ошибок, когда аналитические формулы имеют сложный вид. Во-вторых, он может быть использован для определения через него стандартной асимптотической теории, которая может позволить получить более хорошую апроксимацию к распределению тестовой статистики для конечных выборок.

Мы покажем, как бутстрэп применим для теста Вальда далее в главе 11.
 
 
\subsection{Вывод при использовании стандартной асимптотической теории}

Рассмотрим опять пример пробит-регрессии с бинарным $y$, равным единице с вероятностью $p = \Phi (\gamma + \beta x)$, где $\Phi(\cdot)$ --- функция распределения стандартной нормальной случайной величины. Мы хотим проверить гипотезу $H_0: \beta = 1$ против альтернативной $H_a: \beta \not= 1$ на уровне значимости 0.05. Для проведения анализа знание пробит-модели не является необходимым.

Генерируем одну выборку размером $N = 30$. Оценивание пробит-модели с помощью метода максимального правдоподобия даёт $\hat{\beta} = 0.817$ и $s_{\hat{\beta}} = 0.294$, где стандартная ошибка основана на $- \hat{A}^{-1}$, поэтому тестовая статистика $z = (1 - 0.817)/0.294 = - 0.623$

Используя стандартную асимптотическую теорию, получаем критические значения для 5\% уровня значимости $- 1.96$ и 1.96, так как $z_{0.025} = 1.96$, поэтому нулевая гипотеза не отвергается.


\subsection{Бутстреп без асимптотической теории}

Отправной точкой метода бутстрэп является генерация повторной выборки из апроксимации к генеральной совокупности, см. раздел 11.2.1. Парный бутстрэп достигает этого путём генерации повторной выборки на базе имеющейся выборки.

Таким образом, можно создать $B$ псевдовыборок размера $N$, которые состоят из случайных комбинаций элементов имеющейся выборки $(y_i, x_i), i = 1, \dots, N$. Например, в первой псевдовыборке из 30 наблюдений может содержаться $(y_1, x_1)$ один раз, $(y_2, x_2)$ вообще может отсутствовать, а $(y_3, x_3)$ может встречаться два раза и так далее. Это даёт $B$ оценки параметров $\beta$ $\hat{\beta}_1^*, \dots, \hat{\beta}_B^*$, которые могут быть использованы для выявления особенностей распределения исходной оценки $\beta$. 

Например, допустим, что компьютерная программа, которая используется для оценивания пробит-модели, выдаёт $\hat{\beta}$, но не выдаёт стандартные ошибки $s_{\hat{\beta}}$. Бутстреп решает эту проблему, так как можно использовать оценки стандартных ошибок $s_{\hat{\beta}, boot}$ оценок $\hat{\beta}_1^*, \dots, \hat{\beta}_B^*$ из $B$ бутстрэп псевдовыборок. При наличии оценок стандартных ошибок можно проверить гипотезу о $\beta$  с помощью теста Вальда.

Для примера теста Вальда для пробит-модели получаемая с помощью бутстрэпа оценка стандартной ошибки $\hat{\beta}$ равна 0.376, поэтому $z = (1 - 0.817) / 0.376 = - 0.487$. Так как $- 0.487$ принадлежит интервалу от $- 1.96$ до 1.96, то на уровне значимости 5\% нулевая гипотеза не отвергается.

Это использование бутстрэпа для проверки гипотез не приводит к улучшению размера теста для малых выборок. Тем не менее, это может привести к большой экономии времени во многих случаях, если трудно иным образом получить стандартные ошибки для оценки.

\subsection{Бутстреп с асимптотической теорией}

Иногда бутстрэп может привести к лучшей асимптотической апроксимации к распределению $z$. Это может привести к получению критических значений для конечных выборок, которые лучше в том смысле, что истинный размер, скорее всего, будет ближе к номинальному размеру 0.05. Подробно об этом говорится в главе 11. Здесь мы приводим описание метода.

Снова создаём $B$ псевдовыборок размера $N$, которые состоят из случайных комбинаций элементов имеющейся выборки. Оцениваем пробит-модель для каждой псевдовыборки и для $b$-той псевдовыборки вычисляем $z_b^* = (\hat{\beta}_b^* - \hat{\beta}) / s_{\hat{\beta}_b^*}$, где $\hat{\beta}$ --- это первоначальная оценка. Распределение бутстрэпа для тестовой статистики $z$ --- эмпирическое распределение $z_i^*, \dots, z_B^*$, а не стандартное нормальное распределение. Нижний и верхний 2.5 процентные квантили этого эмпирического распределения дают критические значения бутстрэпа.

Для примера здесь при $B = 1 000$ нижний и верхний 2.5 процентные квантили эмпирического бутстрэп распределения  $z$ равны $- 2.62$ и 1.83. Критические значения бутстрэпа для тестирования на уровне значимости 5\% равны $- 2.62$ и 1.83, а не стандартным $\pm 1.96$. Так как изначальная тестовая статистика для исследуемой выборки $z = - 0.623$ лежит в промежутке от $- 2.62$ до 1.83, мы не отвергаем нулевую гипотезу $H_0: \beta = 1$. Для метода  
бутстрэп также можно вычислить $p$-значение.

В отличие от метода бутстрэп, описанного в предыдущем разделе, здесь есть асимптотическое улучшение, потому что стьюдентизированная тестовая статистика $z$ асимптотически значима (см. раздел 11.2.3), тогда как оценка $\hat{\beta}$ нет.

\section{Практические соображения}

Микроэконометрические исследовани делают акцент на статистических выводах, основанных на минимальных предположениях о распределении, и используют скорректированную оценку ковариационной матрицы оценки. Однако нет смысла в скоректированных выводах, если нарушение предположений о распределении приводит к более серьёзному осложнению несостоятельности оценки, что может произойти с некоторыми, хотя и не всеми, оценками, полученными методом максимального правдоподобия.

Многие пакеты имеют опцию скорректированных стандартных ошибок. В пакетах, связанных с микроэконометрикой, термин <<скорректированные>> часто означает устойчивость к гетероскедастичности и не рассматривает другие проблемы такие, как кластеризация, см. раздел 24.5, что также может привести к неверным статистическим выводам.

Скорректированный вывод обычно получают с помощью теста Вальда. Он имеет недостаток: он неинвариантный к репараметризации нелинейных гипотез. Однако эта проблема может быть решена с помощью применения метода бутстрэп. Стандартные вспомогательные регрессии для теста множителей Лагранжа и реализация теста с помощью статистических пакетов, как правило, не имеет поправки. Хотя в некоторых случаях существует относительно простая поправка для теста множителей Лагранжа (см. раздел 8.4).

Мощность тестов может быть слабой. В идеальном случае мощность против некоторых значимых альтернатив может быть получена. В противном случае, как указывается в разделе 7.6, следует быть осторожным с выводами при проверке гипотез кроме случая, когда параметры очень точно оценены.

Размер тестов для конечных выборок, полученный с помощью асимптотической теории, также является проблемой. Метод бутстрэп, который подробно описан в главе 11, может позволять проверять гипотезы и строить доверительные интервалы с гораздо более хорошими свойствами для конечных выборок.

Статистические выводы могут быть неустойчивыми, поэтому этот вопрос имеет важное значение для практиков. Рассмотрим двусторонний тест Вальда на статистическую значимость, когда $\hat{\theta} = 1.96$, и предположим, что тестовая статистика, действительно, имеет стандартное нормально распределение. Если $s_{\hat{\theta}} = 1.0$, то $t = 1.96$ и $p$-значение равно 0.050. Однако истинное $p$-значение намного выше 0.117, если стандартная ошибка была недооценена на 20\% (таким образом, правильное $t = 1.57$), и оно значительно ниже 0.014, если стандартная ошибка переоценена на 20\% (таким образом, $t = 2.35$).

\subsection{Библиографические заметки}

Книги по эконометрике Гурьеру и Монфора (1989), Дэвидсона и МакКиннона (1993) уделяют много внимания проверке гипотез. Описание, которое приведено в этой книге, рассматривает только ограничения типа равенств. Для тестов на ограничения типа неравенств можно посмотреть Гурьеру, Холли, и Монфора (1982) для линейных случаев и Волак (1991) для нелинейных случаев. Если параметры заданы на границе пространства параметров при нулевой гипотезе, не всегда можно провести тест на проверку гипотезы; см. Эндриус (2001).
\begin{itemize}
\item [$7.3$] Полезное графическое описание трёх классических тестовых процедур приведено Бьюзом (1982).
\item [$7.5$] Ньюи и Вест (1987a) описывают расширение классических тестов для оценивания обобщённым методом моментов.
\item [$7.6$] Дэвидсон и МакКиннон (1993) подробно рассматривают мощность и объясняют различие между явными и неявными нулевыми и альтернативными гипотезами.
\item [$7.7$] Про исследования методом Монте-Карло можно посмотреть книги Дэвидсона и МакКиннона (1993) и Хендри (1984). 
\item [$7.8$] Метод бутстрэп, введённый Эфроном (1979), подробно рассмотрен в главе 11.
\end{itemize}

\section{Упражнения}
\begin{enumerate}

\item [$7-1$] Предположим, что для выборки получены оценки $\hat{\theta}_1 = 5$, $\hat{\theta}_2 = 3$ с асимптотическими оценками вариации $4$ и $2$, коэффициент корреляции между $\hat{\theta}_1$ и $\hat{\theta}_2$  равен $0.5$. Предположим, что оценки параметров имеют асимптотическое нормальное распределение.
\begin{enumerate}
\item Проверьте гипотезу $H_0: \theta_1 e^{\theta_2} = 100$ против $H_a: \theta_1 \not= 100$ на уровне значимости 0.05.
\item Постройте 95\% доверительный интервал для $\gamma = \theta_1 e^{\theta_2}$.
\end{enumerate}
\item [$7-2$] Рассмотрим НМНК регрессию для модели $y = \exp(\alpha + \beta x) + \epsilon$, где $\alpha, \beta$  и $x$ являются скалярными, а $\epsilon \sim N[0,1]$. Для лёгкости вычислений $\sigma_{\epsilon}^2 = 1$, и его оценку не нужно получать. Мы хотим проверить гипотезу $H_0: \beta = 0$ против $H_a: \beta \not= 0$.
\begin{enumerate}
\item Приведите условие первого порядка для $\alpha$ и $\beta$, оцениваемых методом максимального правдоподобия для неограниченной модели.
\item Приведите асимптотическую ковариационную матрицу для $\alpha$ и $\beta$, оцениваемых методом максимального правдоподобия для неограниченной модели.
\item Найдите оценки $\alpha$ и $\beta$ с помощью метода максимального правдоподобия для ограниченной модели.
\item Приведите вспомогательную регрессию для того, чтобы найти вариант теста множителей Лагранжа с внешним произведением градиента.
\item Приведите полное выражение для стандартного вида теста множителей Лагранжа. Заметьте, что для этого необходимо найти производные логарифма функции правдоподобия для неограниченной модели, оценённого в $\alpha$ и $\beta$, полученных методом максимального правдоподобия для ограниченной модели. [Это сложнее, чем пункты (a) --- (d)].
\end{enumerate}
\item [$7-3$] Предположим, мы хотим выбрать из двух вложенных параметрических моделей. Соотношение между плотностями этих двух моделей выглядит так: $g(y|x, \beta, \alpha = 0) = f(y|x, \beta)$, где для простоты $\beta$ и $\alpha$ --- скаляры. Если $g$ --- это верная плотность, то ММП оценка $\beta$, основанная на плотности $f$, несостоятельная. Тест на модель $f$ против модели $g$ --- это тест $H_0: \alpha = 0$ против $H_a: \alpha \not= 0$. Предположим, что оценивание методом маскимального правдоподобия даёт следующие результаты: (1) модель $f$: $\hat{\beta} = 5.0, se[\hat{\beta}] = 0.5$ и $\ln L = - 106$; (2) модель $g$: $\hat{\beta} = 3.0, se[\hat{\beta}] = 1.0, \hat{\alpha} = 2.5, se[\hat{\alpha}] = 1.0$ и $\ln L = - 103$. Не все тесты возможны при данных выше условиях. Если информации достаточно, проведите тесты и сделайте вывод. Если не хватает информации, то обозначьте это.
\begin{enumerate}
\item Проверьте нулевую гипотезу с помощью теста Вальда на уровне значимости 0.05. 
\item Проверьте нулевую гипотезу с помощью множителей Лагранжа на уровне значимости 0.05. 
\item Проверьте нулевую гипотезу с помощью теста отношения правдоподобия на уровне значимости 0.05. 
\item Проверьте нулевую гипотезу с помощью теста Хаусмана на уровне значимости 0.05. 
\end{enumerate}
\item [$7-4$] Проверьте гипотезу $H_0: \mu = 0$ против $H_a: \mu \not= 0$ на уровне значимости 0.05, если процесс порождающий данные --- это $y \sim N[\mu, 100]$, то есть стандартное отклонение равно 10, а размер выборки $N = 10$. Тестовая статистика --- это стандартная $t$-тестовая статистика $t = \hat{\mu}/\sqrt{s/10}$, где $s^2 = (1/9) \sum_i (y_i - \bar{y})^2$. Проведите 1000 симуляций, чтобы ответить на следующие вопросы.
\begin{enumerate}
\item Получите истинный размер $t$-теста, если верные критические значения для конечных выборок равны $\pm t_{0.025}(8) = \pm 2.306$. Есть ли искажения из-за размера?
\item Получите истинный размер $t$-теста, если асимптотически апроксимированные критические значения равны $\pm t_{0.025}(8) = \pm 2.306$. Есть ли искажения из-за размера?
\item Получите мощность $t$-теста против альтернативной гипотезы $H_a: \mu = 1$, когда используются критические значения $\pm t_{0.025}(8) = \pm 2.306$. Мощный ли тест против именно этой альтернативы?
\end{enumerate}
\item [$7-5$] Используйте данных о расходах на здравоохранение из раздела 16.6. Модель представляет собой пробит-регрессию DMED, переменной-индикатора положительных расходов на здравоохранение, на 17 регрессоров, перечисленных во втором абзаце в разделе 16.6. Вы должны получить оценки, приведённые в первой колонке таблицы 16.1. Рассмотрим совместный тест на статистическую значимость показателей здоровья HLTHG, HLTHF и HLTHP на уровне значимости 0.05.
\begin{enumerate}
\item Проведите тест Вальда.
\item Проведите тест отношения правдоподобия.
\item Проведите вспомогательную регрессию, чтобы провести тест множителей Лагранжа. [Это потребует написания дополнительного кода].
\end{enumerate}
\end{enumerate}



\chapter{Тесты на спецификацию и выбор моделей}
\section{Введение}

Два важных практических аспекта микроэконометрического моделирования --- это выявление, является ли модель верно специфицированной, и выбор между альтернативными моделями. Для этих целей можно использовать методы проверки гипотез, представленные в предыдущей главе, особенно когда модели являются вложенными. В этой главе мы представим некоторые другие методы.

Во-первых, такие М-тесты, как тест на условные моменты, --- это тесты на проверку того, выполняются ли моментные условия, наложенные моделью, или нет. Такой подход близок к обобщённому методу моментов, за исключением того, что моментные условия не накладываются для оценивания, а наоборот используются для тестирования. Такие тесты концептуально очень отличаются от тестов на проверку гипотез, представленных в главе 7, так как нет явной формулировки модели, которая получается при альтернативной гипотезе.

Во-вторых, тесты Хаусмана --- это тесты на разницу между двумя оценками, обе из которых состоятельны, если модель верно специфицирована, но они расходятся, если модель неверно специфицирована.

В-третьих, тесты невложенных моделей требуют специальных методов, поскольку стандартный подход к проверке гипотез может быть применён только, когда одна модель получается при наложении ограничений на другую.

Наконец, бывает полезно рассчитывать статистику адекватности модели, которая не является тестовой статистикой. Например, можно сконструировать аналог $R^2$ для измерения качества подгонки нелинейной модели.

В идеальном случае эти методы используются в процессе спецификации, оценивания, тестирования и трактовки модели, что может способствовать переходу от общей к конкретной модели или от конкретной к более общей модели, которая отражает наиболее важные особенности данных.

В разделе 8.2 описаны М-тесты, в том числе тесты на условный момент, тест информационных матриц и хи-квадрат тест на качество подгонки. Тест Хаусмана рассмотрен в разделе 8.3. Тесты на несколько часто встречающихся неверных спецификаций приведены в разделе 8.4. Выбор между невложенными моделями подробно описан в разделе 8.5. Часто используемые тесты, приведённые в разделах 8.2 --- 8.5, могут опираться на сильные предположения о распределении и/или могут давать плохие результаты на малых выборках. Эти недостатки останавливают многих от применения этих тестов, но эти проблемы устарели, потому что во многих случаях метод бутстрэп, изложенный в главе 11, может исправить эти недостатки. Раздел 8.6 рассматривает последствия тестирования модели на последующие выводы. Диагностика модели представлена отдельно в разделе 8.7.

\section{М-тесты}

М-тесты такие, как тесты на условный момент, являются тестовой процедурой проверки общей спецификации, которая охватывает много стандартных тестов на спецификацию. Тесты легко реализовать, используя вспомогательные регрессии, когда оценки получают с помощью метода максимального правдоподобия. Это тот случай, когда особенно желательно провести тест модельных предположений. Реализация тестов, как правило, затруднена, если оценки основаны на минимальных предположениях о распределении.

Мы сначала определим тестовую статистику и вычислительные методы, а затем приведём примеры и проиллюстрируем тесты.

\subsection{М-тестовая статистика}

Предположим, что модель предполагает условие для теоретического момента
\begin{equation}
H_0: \E[m_i(w_i, \theta)] = 0,
\end{equation}
где $w$ представляет собой вектор наблюдаемых переменных, как правило, зависимой переменной $y$ и регрессоров $x$, а иногда и дополнительных переменных $z$, $\theta$ --- вектор параметров размера $q \times 1$, и $m_i(\cdot)$ --- вектор размера $h \times 1$. Например, в простом случае, $\E[(y = x'\beta)z] = 0$, если $z$ может быть опущен в линейной модели $y = x'\beta + u$. Особенно для полностью параметрических моделей существует много вариантов для $m_i(\cdot)$.

М-тест --- это тест на близость к нулю соответствующего момента выборки 
\begin{equation}
\hat{m}_N (\hat{\theta}) = N^{-1} \sum_{i=1}^N m_i(w_i, \hat{\theta}).
\end{equation}
Этот подход аналогичен тесту Вальда, где $h(\theta) = 0$ проверяется путём тестирования близости $h(\hat{\theta})$ к нулю.

Тестовую статистику получают методом, подобным методу для теста Вальда, описанному в разделе 7.2.4. В разделе 8.2.3 показано, что если (8.1) выполнено, то
\begin{equation}
\sqrt{N}\hat{m}_N (\hat{\theta}) \stackrel{d}{\rightarrow} \mathcal{N}[0,V_m],
\end{equation} 
где $V_m$, которая будет определена далее в (8.10), имеет более сложную структуру, чем в случае теста Вальда, потому что $m_i(w_i, \hat{\theta})$ имеет два источника стохастической вариации: $w_i$ и $\hat{\theta}$ являются случайными величинами. Хи-квадрат тестовая статистика может быть получена с помощью соответствующей квадратичной формы. Таким образом, тестовая М-статистика для (8.1) имеет вид:
\begin{equation}
M = N\hat{m}_N (\hat{\theta})'\hat{V}_m^{-1}\hat{m}_N (\hat{\theta}),
\end{equation}
и имеет асимптотическое $\chi^2(\rank(V_m))$ распределение, если моментные условия (8.1) являются верными. М-тест отвергает моментные условия из (8.1) на уровне значимости $\alpha$, если $M > \chi_{\alpha}^2(h)$, и не отвергает иначе.

Сложность может заключается в том, что матрица $V_m$ может не иметь полного ранга $h$. Например, такое может быть, если оценка $\hat{\theta}$ получается при приравнивании линейной комбинации компонент $\hat{m}_N (\hat{\theta})$ к нулю. В некоторых случаях таких, как тест на сверх-идентифицирующие ограничения (OIR), $\hat{V}_m$ по-прежнему имеет полный ранг и можно вычислить $M$, но хи-квадрат статистика имеет только $\rank[V_m]$ степеней свободы. В других случаях матрица $V_m$ не будет иметь полный ранг. Тогда проще всего откинуть $(h - \rank[V_m])$ моментных условий и провести М-тест, используя только это подмножество моментных условий. Полный набор моментных условий может быть использован, но $\hat{V}_m^{-1}$ из (8.4) заменяется $\hat{V}_m^{-}$, обобщённой обратной матрицей $\hat{V}_m$. Обобщённая матрица Мура-Пенроуза $V^{-}$, обратная $V$,  удовлетворяет условиям $VV^{-}V = V$, $V^{-}VV^{-} = V^{-}$, $(VV^{-})' = VV^{-}$ и $(V^{-}V)' = V^{-}V$. Когда $V_m$ имеет неполный ранг, строго говоря, (8.3) больше не выполняется, так как невырожденное многомерное нормальное распределение требует полного ранга $V_m$. Однако (8.4) по-прежнему выполняется с учётом этих корректировок.
 
Подход М-теста концептуально очень прост. Моментные ограничения (8.1) отвергаются, если квадратичная форма выборочной оценки (8.2) находится достаточно далеко от нуля. Проблемы, которые могут возникнуть, связаны с расчётом $M$, так как $\hat{V}_m$ может иметь достаточно сложный вид (см. раздел 8.2.2), с выбором моментов $m(\cdot)$ для проверки (см. основные примеры из разделов 8.2.3 --- 8.2.6) и интерпретацией причин, из-за которых (8.1) отвергается (см. раздел 8.2.8).

\subsection{Расчёт М-статистики}

Есть несколько способов вычислить М-статистику.

Во-первых, всегда можно вычислить непосредственно $\hat{V}_m$ и, соответственно, $M$, используя состоятельные оценки компонент $V_m$, которые приведены в разделе 8.2.3. Большинство практиков уклоняются от такого подхода, поскольку он требует матричных вычислений.

Во-вторых, всегда можно использовать бутстрэп (см. раздел 11.6.3), так как этот метод позволяет найти оценку $V_m$, которая отвечает за все возможные случайности в $\hat{m}_N (\hat{\theta}) = N^{-1}\sum_i m_i(w_i, \hat{\theta})$.  

В-третьих, чтобы вычислить асимптотически эквивалентные версии $M$, которые не требуют вычисления $\hat{V}_m$, в некоторых случаях можно провести вспомогательные регрессии, аналогичные регрессиям для теста множителей Лагранжа, которые приведены в разделе 7.3.5. Эти вспомогательные регрессии могут быть проведены с методом бутстрэп, чтобы получить асимптотическое уточнение (см. раздел 11.6.3). Мы приведём несколько основных вспомогательных регрессий.

\begin{center}
Вспомогательные регрессии с использованием оценок, полученных методом максимального правдоподобия
\end{center}

Тесты на спецификацию моделей особенно желательно проводить, когда вывод делается в рамках метода правдоподобия, так как в общем случае любая неверная спецификация плотности может привести к несостоятельности метода максимального правдоподобия. К счастью, легко реализовать М-тест, когда оценку получают методом максимального правдоподобия.

В частности, если $\hat{\theta}$ --- оценка, полученная методом максимального правдоподобия, обобщение результата теста множителей Лагранжа из раздела 7.3.5 (см. раздел 8.2.3) даёт асимптотически эквивалентную версию М-теста при проведении вспомогательной регрессии
\begin{equation}
1 = \hat{m}_i' \delta + \hat{s}_i' \gamma + u_i,
\end{equation}
где $\hat{m}_i = m_i(y_i, x_i, \hat{\theta}_{ML})$, $\hat{s}_i = \partial{lnf(y_i|x_i, \theta)}/\partial{\theta}|_{\hat{\theta}_{ML}}$ --- это вклад $i$-того наблюдения в скор-функцию, и $f(y_i|x_i, \theta)$ --- это условная функция плотности при расчёте
\begin{equation}
M^{*} = NR_u^2,
\end{equation}
где $R_u^2$ --- нецентрированный $R^2$, который определён в конце раздела 7.3.5. Другими словами, $M^{*}$ равно $ESS_u$, нецентрированной объяснённой сумме квадратов (сумме квадратов оценочных значений) из регрессии (8.5), или $M^{*}$ равно $N - RSS$, где $RSS$ --- это сумма квадратов остатков из регрессии (8.5). Статистика $M^{*}$ имеет асимптотическое $\chi^2(h)$ распределение при нулевой гипотезе.

Тестовая статистика $M^{*}$ --- это внешнее произведение градиента для М-теста, и это является обобщением вспомогательной регрессии для теста множителей Лагранжа (см. раздел 7.3.5). Несмотря на то, что внешнее произведение градиента можно легко вычислить, на малых выборках статистика имеет плохие свойства с большими искажениями размера теста. Однако аналогично тесту множителей Лагранжа эти проблемы для малых выборок могут быть значительно уменьшены с помощью использования метода бутстрэп (см. раздел 11.6.3).

Тестовая статистика $M^{*}$ также может быть уместна в условиях, отличных от условий метода максимального правдоподобия. Вспомогательная регрессия применяется всегда, когда выполняется $\E[\partial{m}/\partial{\theta}'] = - \E[ms']$ (см. раздел 8.2.3). С помощью обобщённого равенства информационных матриц (см. раздел 5.6.3), это условие выполняется для метода максимального правдоподобия, когда математическое ожидание считается по указанной функции плотности $f(\cdot)$. Оно также может выполняться при более слабых предположениях о распределении в некоторых случаях.

\begin{center}
Вспомогательные регрессии в случае $\E[\partial{m}/\partial{\theta}'] = 0$
\end{center}

В некоторых случаях $m_i(w_i, \theta)$ удовлетворяет условию
\begin{equation}
\E[\partial{m_i(w_i,\theta)}/\partial{\theta}'|_{\theta_0}] = 0,
\end{equation}
в дополнении к (8.1).

Тогда можно показать, что асимптотическое распределение $\sqrt{N} \hat{m}_N(\hat{\theta})$ не отличается от $\sqrt{N}m_N(\theta_0)$. Таким образом, $V_m = \plim N^{-1}\sum_i m_{i0}m_{i0}'$, а её оценка $\hat{V}_m = \plim N^{-1}\sum_i \hat{m}_i\hat{m}_i'$. Тестовую статистику можно вычислить способом, аналогичным (8.5), но вспомогательная регрессия в данном случае будет более простой:
\begin{equation}
1 = \hat{m}_i \delta + u_i,
\end{equation}
с тестовой статистикой $M^{**}$, которая равна $NR_u^2$.

Эта вспомогательная регрессия действительна для любой $\sqrt{N}$ состоятельной оценки $\hat{\theta}$, а не только
для оценок, полученных методом максимального правдоподобия, при условии, что (8.7) выполняется. Условие (8.7) выполняется в нескольких примерах, см. раздел 8.2.9.

Даже если (8.7) не выполняется, можно провести более простую регрессию (8.8), так как она задаёт нижнюю границу верного значения $M$ для М-тестовой статистики. Если эта более простая регрессия приводит к тому, что нулевая гипотеза отвергается, то (8.1) точно отвергается.

\begin{center}
Другие вспомогательные регрессии
\end{center}

Возможны регрессии, которые альтернативны (8.5) и (8.8), если $m(y, x, \theta)$ и $s(y, x, \theta)$ могут быть разложены на сомножители.

Во-первых, если $s(y, x, \theta) = g(x, \theta)r(y, x, \theta)$ и $m(y, x, \theta) = h(x, \theta)r(y, x, \theta)$ для некоторой простой скалярной функции $r(\cdot)$ с $\V[r(y, x, \theta)] = 1$ и если оценивание происходит с помощью метода максимального правдоподобия, тогда асимптотически эквивалентной (8.5) регрессией будет $NR_u^2$ из регрессии $\hat{r}_i$ на $\hat{g}_i$ и $\hat{h}_i$.

Во-вторых, если $m(y, x, \theta) = h(x, \theta)v(y, x, \theta)$ для некоторой скалярной функции $v(\cdot)$ с $\V[v(y, x, \theta)] = 1$ и $\E[\partial{m}/\partial{\theta}'] = 0$, тогда асимптотически эквивалентной (8.8) регрессией будет $NR_u^2$ из регрессии $\hat{v}_i$ на $\hat{h}_i$. Подробности представлены в книге Вулдриджа (1991).

Дополнительные вспомогательные регрессии существуют и в других частных случаях. Примеры приведены в разделе 8.4, а также Уайт (1994) приводит довольно общее описание.

\subsection{Вывод тестовых М-статистик}

Чтобы избежать необходимости вычислять $V_m$, ковариационную матрицу из (8.3), обычно проводятся М-тесты с помощью вспомогательных регрессий или метода бутстрэп. Для полноты в этом разделе представлены фактическое выражение для $V_m$ и обоснование вспомогательных регрессий (8.5) и (8.8).

Основная задача --- это получение распределения $\hat{m}_N(\hat{\theta})$ из (8.2). Задача осложняется тем, что $m_N(\hat{\theta})$ является стохастической матрицей по двум причинам: из-за случайных величин $w_i$ и оценивания в точке $\hat{\theta}$.

Предположим, что $\hat{\theta}$ является М-оценкой, т.е. является решением 
\begin{equation}
\frac{1}{N} \sum_{i=1}^N s_i(w_i, \hat{\theta}) = 0,
\end{equation}
для некоторой функции $s(\cdot)$, возможно несовпадающей с $\partial{\ln f(y|x, \theta)}/\theta$ в данном случае. Также необходимо сделать обычное для пространственных данных предположение о том, что данные независимы по $i$. Тогда мы покажем, что $\sqrt{N}\hat{m}_N(\hat{\theta}) \stackrel{d}{\rightarrow} \mathcal{N}[0, V_m]$, как и в (8.3), где
\begin{equation}
V_m = H_0J_0H_0',
\end{equation} 
матрица размера $h \times (h + q)$
\begin{equation}
H_0 = [I_h - C_0A_0^{-1}],
\end{equation}
где $C_0 = \plim N^{-1} \sum_i \partial{m_{i0}}/\partial{\theta}'$ и $A_0 = \plim N^{-1} \sum_i \partial{s_{i0}}/\partial{\theta}'$, и матрица размера $(h + q) \times (h + q)$
\begin{equation}
J_0 = \plim N^{-1} \begin{bmatrix} \sum_{i=1}^N m_{i0} m_{i0}' &  \sum_{i=1}^N m_{i0} s_{i0}' \\ \sum_{i=1}^N s_{i0} m_{i0}' & \sum_{i=1}^N s_{i0} s_{i0}' \end{bmatrix},
\end{equation}
где $m_{i0} = m_i(w_i, \theta_0)$ и $s_{i0} = s_i(w_i, \theta_0)$.

Чтобы вывести (8.10), возьмём разложение в ряд Тейлора первого порядка в окрестности точки $\theta_0$ и получи
\begin{equation}
\sqrt{N}\hat{m}_N(\hat{\theta}) = \sqrt{N}m_N(\theta_0) + \frac{\partial{m_N(\theta_0)}}{\partial{\theta}'} \sqrt{N}(\hat{\theta} - \theta_0) + o_p(1).
\end{equation}

Для $\hat{\theta}$, определённой в (8.9), это означает, что
\begin{equation}
\sqrt{N}\hat{m}_N(\hat{\theta}) = \frac{1}{\sqrt{N}} \sum_{i=1}^N m_i(\theta_0) - C_0A_0^{-1}\frac{1}{\sqrt{N}} \sum_{i=1}^N s_{i0} + o_p(1), 
\end{equation}
где $m_N = N^{-1} \sum_i m_i$, $\partial{m_N}/\partial{\theta}' =  N^{-1} \sum_i \partial{m_i}/\partial{\theta}' \stackrel{p}{\rightarrow} C_0$, и $\sqrt{N}(\hat{\theta} - \theta_0)$ имеет то же самое предельное распределение, что и $A_0^{-1}N^{-1/2} \sum_i s_{i0}$, при применении обычного разложения в ряд Тейлора до первого члена для (8.9). Уравнение (8.14) можно переписать как
\begin{equation}
\sqrt{N}\hat{m}_N(\hat{\theta}) = \begin{bmatrix} I_h & - C_0A_0^{-1} \end{bmatrix} \begin{bmatrix}  \frac{1}{\sqrt{N}}\sum_{i=1}^N m_{i0} \\ \frac{1}{\sqrt{N}}\sum_{i=1}^N s_{i0} \end{bmatrix} + o_p(1).
\end{equation}

Уравнение (8.10) вытекает из применения теоремы о нормальности предела произведения (теорема А.17), так как второй член произведения в (8.15) имеет предельное нормальное распределение при нулевой гипотезе с нулевым математическим ожиданием и дисперсией $J_0$.

Для вычисления $M$ из (8.4) может быть получена состоятельная оценка $\hat{V}_m$ для $V_m$ с помощью замены каждого компонента $V_m$ на его состоятельную оценку. Например, для $C_0$ можно получить состоятельную оценку $\hat{C} = N^{-1}\sum_i \partial{m}_i/\partial{\theta'}|_{\hat{\theta}}$. Вариант с использованием вспомогательных регрессий реализовать легче, если они доступны.

Во-первых, рассмотрим вспомогательную регрессию (8.5), где $\hat{\theta}$ --- это оценка, полученная методом максимального правдоподобия. С помощью обобщённого равенства информационных матриц (см. раздел 5.6.3) $\E[\partial{m_{i0}}/\partial{\theta'}] = - \E[m_{i0}s_{i0}']$, где для случая метода максимального правдоподобия мы рассмотрим $s_i = \partial{\ln f(y_i, x_i, \theta)}/\partial{\theta'}$. Можно значительно упростить выражение, так как $C_0 = - \plim N^{-1}\sum_i m_{i0}s_{i0}'$ и $A_0 = - \plim N^{-1}\sum_i s_{i0}s_{i0}'$, которые тоже встречаются в матрице $J_0$. Это приводит к тесту в виде внешнего произведения градиента. Более подробно можно посмотреть в книгах Ньюи (1985) или Пагана и Велла (1989).

Во-вторых, для вспомогательной регрессии (8.8) обратите внимание, что $\E[\partial{m_{i0}}/\partial{\theta'}] = 0$, и тогда $C_0 = 0$, то $H_0 = \begin{bmatrix} I_h & 0\end{bmatrix}$ и $H_0J_0H_0' = \plim N^{-1}\sum_i m_{i0}m_{i0}'$.

\subsection{Тесты на условный момент}

Тесты на условный момент, введённые Ньюи (1985) и Таушеном (1985), --- это М-тесты на безусловные моментные ограничения, которые получаются из лежащего в их основе условного моментного ограничения.

В качестве примера рассмотрим модель линейной регрессии $y = x'\beta + u$. Стандартное предположение о состоятельности МНК-оценки состоит в том, что ошибка имеет нулевое условное математическое ожидание или, что  условное моментное ограничение имеет вид:
\begin{equation}
\E[y - x'\beta|x] = 0.
\end{equation}
В главе 6 мы рассматривали использование некоторых накладываемых безусловных моментных ограничений в качестве основы для оценивания методом моментов или обобщённым методом моментов. В частности (8.16) подразумевает, что $\E[x(y - x'\beta)] = 0$. Решение соответствующего выборочного моментного условия $\sum_i x_i(y_i - x_i'\beta) = 0$ приводит к получению МНК-оценки $\beta$. Тем не менее, (8.16) требует многих других моментных условий, которые не используются для оценивания. Рассмотрим безусловное моментное ограничение
\[
\E[g(x)(y - x'\beta)] = 0,
\]
где вектор $g(x)$ должен отличаться от $x$, который уже используется в оценивании МНК. Например, $g(x)$ может содержать квадраты и смешанные произведения компонентов вектора $x$. Это наводит на мысль о тесте, основанном на том, близок ли соответствующий выборочный момент $\hat{m}_N(\hat{\theta}) = N^{-1}\sum_i g(x_i)(y_i - x_i'\hat{\beta})$ к нулю или нет.

В более общем случае рассмотрим условное моментное ограничение
\begin{equation}
\E[r(y, x, \theta)|x] = 0,
\end{equation}
для некоторой скалярной функции $r(\cdot)$. Тест на условный момент --- М-тест, основанный на безусловных моментных ограничениях
\begin{equation}
\E[g(x)r(y, x, \theta)] = 0,
\end{equation}
где $g(x)$ и/или $r(y, x, \theta)$ выбраны так, чтобы эти ограничения ранее не использовались при оценивании.

Модели, основанные на методе максимального правдоподобия, приводят к многим потенциальным ограничениям. Для неполностью параметрических моделей примеры $r(y, x, \theta)$ включают $y - \mu(x, \theta)$, где $\mu(\cdot)$ --- заданная функция условного математического ожидания, и $(y - \mu(x, \theta))^2 - \sigma^2(x, \theta)$, где $\sigma^2(x, \theta)$ --- заданная функция условной дисперсии.

\subsection{Информационный матричный тест Уайта}

Для оценивания методом максимального правдоподобия равенство информационных матриц приводит к моментным ограничениям, которые могут быть использованы и в М-тесте, так как они, как правило, не используются при получении оценки метода максимального правдоподобия.

В частности, из раздела 5.6.3 равенство информационных матриц приводит к тому, что
\begin{equation}
\E[\Vech[D_i(y_i, x_i, \theta_0)]] = 0,
\end{equation}
где матрица $D_i$ имеет размера $q \times q$ и представляется в виде:
\begin{equation}
D_i(y_i, x_i, \theta_0) = \frac{\partial^{2}{\ln f_i}}{\partial{\theta}\partial{\theta}'} + \frac{\partial{\ln f_i}}{\partial{\theta}} \frac{\partial{\ln f_i}}{\partial{\theta'}},
\end{equation}
и математическое ожидание берётся согласно предполагаемой условной функции плотности $f_i(y_i| x_i, \theta)$. Здесь $\Vech$ --- оператор, который ставит столбцы матрицы $D_i$ таким же образом, как и $Vec$ оператор. Отличие состоит в том, что только $q(q + 1)/2$ уникальных элементов симметричной матрицы $D_i$ размещаются в столбец.

Уайт (1982) предложил информационно-матричный тест, который проверяет, близок ли соответствующий выборочный момент
\begin{equation}
\hat{D}_N(\hat{\theta}) = N^{-1}\sum_{i=1}^N \Vech[D_i(y_i, x_i, \hat{\theta}_{ML})]
\end{equation}
к нулю или нет. Применяя (8.4), получаем, что тестовая статистика в данном случае
\begin{equation}
IM = N\hat{d}_N(\hat{\theta})'\hat{V}^{-1}\hat{d}_N(\hat{\theta}),
\end{equation}
где выражение для $\hat{V}$, предложенное Уайтом (1982), имеет довольно сложный вид. Гораздо проще провести тест, как утверждают Ланкастер (1984) и Чешер (1984), при помощи вспомогательной регрессии (8.5), которую можно проводить, так как в (8.21) используется метод максимального правдоподобия.

Информационно-матричный тест также может быть применён и для подмножества ограничений (8.19). Так следует поступать, если $q$ велико, так как в этом случае количество проверяемых ограничений $q(q + 1)/2$ очень велико.

Большие значения статистики информационно-матричного теста приводят к отвержению ограничений равенства информационных матриц и выводу, что плотность неверно специфицирована. В общем случае это означает, что оценка, полученная методом максимального правдоподобия, является несостоятельной. В некоторых особых случаях, описанных в разделе 5.7, оценка, полученная методом максимального правдоподобия, всё же может быть состоятельной, хотя стандартные ошибки в этом случае должны быть основаны на сэндвич форме ковариационной матрицы.

\subsection{Хи-квадрат тест на качество подгонки}

Полезный тест на спецификацию для полностью параметрических моделей --- сравнение предсказанных вероятностей с относительными частотами из выборки. Модель является плохой, если эти величины существенно различаются.

Начнём с дискретных одинаково распределённых случайных переменных $y$, которые могут принимать одно из $J$ возможных значений с вероятностью $p_1, p_2, \dots, p_j$, $\sum_{j=1}^J p_j = 1$.

Верная спецификация вероятностей может быть проверена с помощью теста на равенство теоретических частот $Np_j$ и наблюдаемых частот $N\bar{p}_j$, где $\bar{p}_j$ --- доля выборки, которая принимает $j$-ое возможное значение. Статистика хи-квадрат теста Пирсона на качество подгонки имеет вид:
\begin{equation}
PCGF = \sum_{j=1}^J \frac{(N\bar{p}_j - Np_j)^2}{Np_j}.
\end{equation}
Эта статистика имеет асимптотическое $\chi^2(J-1)$ распределение при нулевой гипотезе о том, что вероятности $p_1, p_2, \dots, p_j$ являются верными. Тест может быть обобщен на прогнозирование вероятностей из регрессии (см. упражнение 8.2). Рассмотрим мультиномиальную модель для дискретных $y$ с вероятностями $p_{ij} = p_{ij}(x_i, \theta)$. Тогда $p_j$ из (8.23) заменяется на $\hat{p}_j = N^{-1}\sum_i F_j(x_i, \hat{\theta})$ и, если $\hat{\theta}$ --- это мультиномиальная оценка, полученная методом максимального правдоподобия, мы опять получаем хи-квадрат распределение, но уже с ограниченным числом степеней свободы $(J - \dim(\theta) - 1)$, которые получаются вследствие оценивания $\theta$ (см. Эндриус, 1988а).

Для регрессионных моделей отличных от мультиномиальных, статистика $PCGF$ из (8.23) может быть вычислена путём группировки $y$ в интервалы, но в этом случае статистика $PCGF$ больше не имеет хи-квадрат распределение. Вместо этого применяется похожая статистика М-теста. Чтобы вывести эту статистику, надо разбить диапазон $y$ на $J$ взаимоисключающих интервалов, где $J$ интервалов охватывают все возможные значения $y$. Пусть $d_{ij}$ --- переменная-индикатор, равная единице, если $y_i \in$ интервале $j$, и равная нулю в противном случае. Пусть $p_{ij}(x_i, \theta) = \int_{y_i \in cell j} f(y_i| x_i, \theta)dy_i$ --- спрогнозированная вероятность того, что наблюдение $i$ попадает в интервал $j$, где $f(y|x, \theta)$ --- условная плотность $y$. Также для начала мы предполагаем, что вектор параметров $\theta$ известен. Если условная плотность верно специфицирована, то
\begin{equation}
\E[d_{ij}(y_i) - p_{ij}(x_i, \theta)] = 0, j = 1, \dots, J.
\end{equation}
Объединяя все $J$ моментов в единый вектор, мы получаем
\begin{equation}
\E[d_i(y_i) - p_i(x_i, \theta)] = 0,
\end{equation}
где $d_i$ и $p_i$ --- это векторы размера $J \times 1$ с $j$-тыми элементами $d_{ij}$ и $p_{ij}$. Мы приходим к  М-тесту на близость к нулю соответствующего выборочного момента
\begin{equation}
\widehat{dp}_N(\hat{\theta}) = N^{-1}\sum_{i=1}^N (d_i(y_i) - p_i(x_i, \hat{\theta})),
\end{equation}
который представляет собой разницу между вектором относительных выборочных частот $N^{-1}\sum_i d_i$ и вектором предсказанных частот $N^{-1}\sum_i \hat{p}_i$. Используя (8.5), мы получаем хи-квадрат статистику теста Эндриуса на качество подгонки (1988а, 1988б):
\begin{equation}
CGF = N\widehat{dp}_N(\hat{\theta})'\hat{V}^{-1}\widehat{dp}_N(\hat{\theta}), 
\end{equation}
где выражение для $\hat{V}$ имеет довольно сложный вид. Статистику $CGF$ теста можно легко посчитать, используя вспомогательную регрессию (8.5), с $\hat{m}_i = d_i - \hat{p}_i$. Эта вспомогательная регрессия здесь уместна, потому что тестируется полностью параметрическая модель и поэтому $\hat{\theta}$ будет получена с помощью метода максимального правдоподобия.

Необходимо отбросить одну из категорий из-за  ограничения, что сумма вероятностей равнялась единице. Мы получаем тестовую статистику, которая имеет асимптотическое $\chi^2(J - 1)$ распределение при нулевой гипотезе о том, что $f(y|x, \theta)$ верно специфицирована. Возможно, будет необходимо отбросить другие категории в некоторых особых случаях, таких, как мультиномиальный пример, который обсуждался после (8.23). В дополнение к рассчитанной тестовой статистике может быть полезно посмотреть на компоненты $N^{-1}\sum_i d_i$ и $N^{-1}\sum_i \hat{p}_i$.

Соответствующая асимптотическая теории рассматривается Эндриусом (1988a), а также более простая интерпретация и несколько приложений приведены в книге Эндриуса (1988б). Для простоты мы представили группировку, которая определяется значением $y$, но разделение может быть и по $y$, и по $x$. Группировка наблюдений должна быть проведена таким образом, чтобы в каждую группу попало достаточное число наблюдений. Более подробная информация и история этого этого теста представлены в этих статьях.

Для непрерывной случайной величины $y$ в случае одинаково распределённых $y$ более общим тестом, чем тест $SCGF$, является тест Колмогорова, который использует всё распределение $y$, а не только группы наблюдений, образованные $y$.  Эндриус (1997) представляет регрессионную версию теста Колмогорова, но её гораздо сложнее реализовать, чем тест $CGF$.

\subsection{Тест на сверх-идентифицирующие ограничения}

Тесты на сверх-идентифицирующие предположения (см. раздел 6.3.8) являются примерами М-тестов.

В обозначениях главы 6 оценки, полученные обобщённым методом моментов, основаны на предположении о том, что $\E[h(w_i,\theta)] = 0$. Если модель сверх-идентифицирована, то только $q$ из этих моментных ограничений используются для оценивания, что приводит к $(r - q)$ линейно зависимым ортогональным условиям, где $r = \dim[h(\cdot)]$. Их можно использовать для проведения М-теста. Тогда мы используем $M$ из (8.4), где $\hat{m}_N = N^{-1}\sum_i h(w_i, \hat{\theta})$. 

Как показано в разделе 6.3.9, если $\hat{\theta}$ --- оптимальная оценка, полученная обобщённым методом моментов, то $\hat{m}_N(\hat{\theta})'\hat{S}_N^{-1}\hat{m}_N(\hat{\theta})$, где $\hat{S}_N = N^{-1}\sum_{i=1}^N \hat{h}_i\hat{h}_i'$ имеет асимптотическое $\chi^2(r - q)$ распределение. Более интуитивный линейный пример метода инструментальных переменных приведён в разделе 8.4.4.

\subsection{Мощность и состоятельность тестов на условный момент}

Так как нет никаких явных альтернативных гипотез, М-тесты отличаются от тестов из главы 7.

Некоторые авторы приводили примеры, где может быть показано, что информационно матричный тест эквивалентен обычному тесту множителей Лагранжа нулевой гипотезы против альтернативной. Чешер (1984) интерпретировал информационно матричный тест как тест на неоднородность случайного параметра. Для линейной модели с нормальным распределением остатков А. Холл (1987) показал, что подкомпоненты информационно матричного теста соответствуют тестам множителей Лагранжа на гетероскедастичность, симметрию и эксцесс. Кэмерон и Триведи (1998) приводят некоторые дополнительные примеры и объяснения результатов для линейного экспоненциального семейства.

В более общем случае М-тесты могут быть проинтерпретированы в рамках условного момента. Начнём с теста на добавленную переменную в модели линейной регрессии. Предположим, мы хотим проверить, выполняется ли $\beta_2 = 0$ в модели $Y = x_1'\beta_1 + x_2'\beta_2 + u$. Это тест на проверку гипотезы $H_0: \E[y - x_1'\beta_1|x] = 0$ против альтернативной гипотезы $H_a: \E[y - x_1'\beta_1|x] = x_2'\beta_2$. Самый мощный тест $\beta_2 = 0$ в регрессии $y - x_1'\beta_1$ на $x_2$ основан на эффективной оценки, полученной с помощью обобщённого метода моментов,
\[
\hat{\beta}_2 = \left[ \sum_{i=1}^N \frac{x_{2i}x_{2i}'}{\sigma_i^2}\right]^{-1} \sum_{i=1}^N \frac{x_{2i}(y_i - x_{1i}'\beta_1)}{\sigma_i^2},
\]
где $\sigma_i^2 = \V[y_i|x_i]$ при нулевой гипотезе и предполагается независимость по $i$. Этот тест эквивалентен тесту, основанному на второй сумме в отдельности, который является М-тестом на
\begin{equation}
\E\left[ \frac{x_{2i}(y_i - x_{1i}'\beta_1)}{\sigma_i^2} \right] = 0.
\end{equation}
Смотря на этот процесс с обратной стороны, мы можем интерпретировать М-тест, основанный на (8.28), как тест на условный момент на проверку гипотезы $H_0: \E[y - x_1'\beta_1|x] = 0$ против альтернативной гипотезы $H_a: \E[y - x_1'\beta_1|x] = x_2'\beta_2$. Также М-тест, основанный на $\E[x_2(y - x_1'\beta_1)] = 0$, можно интерпретировать как тест на условный момент на проверку гипотезы $H_0: \E[y - x_1'\beta_1|x] = 0$ против альтернативной гипотезы $H_a: \E[y - x_1'\beta_1|x] = \sigma_{y|x}^2 x_2'\beta_2$, где $\sigma_{y|x}^2 = \V[y|x]$ при нулевой гипотезе.

В более общем случае предположим, что мы начнём с условного моментного ограничения
\begin{equation}
\E[r(y_i, x_i, \theta)|x_i] = 0,
\end{equation}
для некоторой скалярной функции $r(\cdot)$. Тогда М-тест, основанный на безусловном моментном ограничении
\begin{equation}
\E[g_i(x_i)r(y_i, x_i, \theta)] = 0,
\end{equation}
может быть интерпретирован как тест на условный момент с нулевой и альтернативной гипотезами
\begin{equation}
H_0: \E[r(y_i, x_i, \theta)|x_i] = 0, H_a: \E[r(y_i, x_i, \theta)|x_i] = \sigma_i^2 g(x_i)'\gamma,
\end{equation}
где $\sigma_i^2 = \V[r(y_i, x_i, \theta)|x_i]$ при нулевой гипотезе.

Такой подход позволяет понять, когда тест на условный момент имеет большую мощность. Хотя (8.30) предполагает, что мощность возрастает по $g(x)$, из (8.31) следует более точное утверждение о том, что мощность растёт с увеличение произведения  $g(x)$ на дисперсию $r(y, x, \theta)$. Это уточнение имеет важное значение для приложений к пространственным данным, потому что дисперсия не является постоянной для всех наблюдений. Более подробная информация и ссылки представлены в книге Кэмерона и Триведи (1998), которые называют этот подход основанным на регрессионном анализе с помощью теста на условный момент. Этот подход может быть обобщён и для вектора $r(\cdot)$, хотя для этого необходимы более громоздкие вычисления.

М-тест представляет собой тест на конечное число моментных условий. Таким образом, можно построить процесс, порождающий данные, для которого лежащее в основе условие для условного момента, например, такое, как в (8.29), неверно, хотя моментные условия выполняются. В этом случае тест на условный момент является несостоятельным, так как он не отвергается с вероятностью, равной единице, при $N \rightarrow \infty$. Биренс (1990) предложил способ задать $g(x)$ в (8.30) чтобы получить состоятельный тест на условный момент для тестов на функциональную форму в нелинейной регрессионной модели, где $r(y, x, \theta) = y - f(x, \theta)$. Однако состоятельность теста не означает высокую мощность против конкретных альтернатив.

\subsection{Пример М-тестов}

Чтобы проиллюстрировать различные М-тесты, мы рассмотрим пример регрессионной модели Пуассона, которая приведена в разделе 5.2, с функцией вероятности $f(y) = e^{-\mu}\mu^y/y!$ и $\mu = \exp(x'\beta)$.

Мы хотим протестировать
\[
H_0: \E[m(y, x, \beta)] = 0,
\]
для разных функций $m(\cdot)$. Этот тест будет проведён при предположении, что процесс, порождающий данные, --- это действительно распределение Пуассона.

\begin{center}
Вспомогательные регрессии
\end{center}

Так как оценивание производится с помощью метода максимального правдоподобия, мы можем использовать М-тестовую статистку $M^*$, рассчитанную как $NR_u^2$ из вспомогательной регрессии (8.5), где
\begin{equation}
1 = \hat{m}(y_i, x_i, \hat{\beta})'\delta + (y_i - \exp(x_i'\hat{\beta}))x_i'\gamma + u_i,
\end{equation}
так как $\hat{s} = |\partial{\ln f(y)}/\partial{\beta}|_{\hat{\beta}} = (y - \exp(x'\hat{\beta}))x$ и $\hat{\beta}$ --- это оценка, полученная методом максимального правдоподобия. При нулевой гипотезе статистика имеет $\chi^2(\dim(m))$ распределение.

Альтернативная $M^{**}$ статистика из вспомогательной регрессии
\begin{equation}
1 = \hat{m}(y,x,z,\hat{\beta})'\delta + u.
\end{equation}
Этот тест асимптотически эквивалентен $LM^*$, если $m(\cdot)$ такое, что $\E[\partial{m}/\partial{\beta}] = 0$, но в противном случае он не имеет хи-квадрат распределения.

\begin{center}
Тесты на моменты
\end{center}

Верная спецификация условной функции математического ожидания, $\E[y - exp(x'\beta)|x] = 0$, может быть протестирована с помощью М-теста на
\[
\E[(y - exp(x'\beta))z] = 0,
\]
где $z$ может быть функцией от $x$. Для Пуассона и других моделей экспоненциального семейства, $z$ не может равняться $x$, потому что 
условия первого порядка для $\hat{\beta}_{ML}$ накладывают ограничение, что $\sum_i (y_i - \exp(x_i'\hat{\beta}))x_i = 0$, что приводит к $M = 0$, если $z = x$. Вместо этого, $z$ может включать квадраты и произведения регрессоров.

Верная спецификация дисперсии также может быть проверена, так как распределение Пуассона предполагает равенство условной дисперсии и математического ожидания. Так как $\V[y|x] - \E[y|x] = 0$ и $\E[y|x] = \exp(x'\beta)$ мы можем провести М-тест на
\[
\E[\{(y - exp(x'\beta))^2 - \exp(x'\beta)\}x] = 0.
\]
Другая версия наоборот тестирует
\[
\E[\{(y - \exp(x'\beta))^2 - y\}x] = 0,
\]
так как $\E[y|x] = \exp(x'\beta)$. Тогда $m(\beta) = \{(y - \exp(x'\beta))^2 - y\}x$ обладает свойством $\E[\partial{m}/\partial{\beta}] = 0$. В этом случае (8.7) выполняется, и альтернативная регрессия (8.33) даёт асимптотически эквивалентный тест для регрессии (8.32).

Стандартная спецификация теста для параметрических моделей --- это информационно матричный тест. Для функции плотности Пуассона, $D$, которая определена в (8.19), становится равной $D(x,y,\beta) = \{(y - \exp(x'\beta))^2 - y\}xx'$, и мы тестируем 
\[
\E[\{(y - \exp(x'\beta))^2 - y\}\Vech[xx']] = 0.
\]
Очевидно, что для примера Пуассона информационно матричный тест является тестом на условия для первого и второго моментов, которые накладываются моделью Пуассона. Этот результат выполняется для более общего случая моделей из экспоненциального семейства. Тестовая статистика $M^{**}$ асимптотически эквивалентна $M^{*}$, так как здесь $\E[\partial{m}/\partial{\beta}]$.

Предположение Пуассона также может быть протестировано с помощью хи-квадрат теста на качество подгонки. Например, так как малое число частот превышает число три в последующем примере моделирования, создадим четыре группы, которые соответствуют $y = 0, 1, 2$ и $3$ и больше, где при выполнении теста группа с $y \geq 3$ отбрасывается, так как сумма вероятностей равна одному. Таким образом, для $j = 0, \dots, 2$ вычислим индикатор $d_{ij} = 1$, если $y_i = j$, и $d_{ij} = 1$ в противном случае. Потом вычислим спрогнозированную вероятность $\hat{p}_{ij} = e^{-\hat{\mu}_i}\hat{\mu}_i^j/j!$, где $\hat{\mu}_i = \exp(x_i'\hat{\beta})$. В этом случае мы тестируем
\[
\E[(d - p)] = 0,
\]
где $d_i = [d_{i0}, d_{i1}, d_{i2}]$ и $p_i = [p_{i0}, p_{i1}, p_{i2}]$ из вспомогательной регрессии (8.33), и $\hat{m}_i = d_i - \hat{p}_i$. 

\begin{center}
Результаты моделирования
\end{center}

Данные были сгенерированы по модели Пуассона с математическим ожиданием $\E[y|x] = \exp(\beta_1 + \beta_2x_2)$, где $x_2 \sim \mathcal{N}[0,1]$ и $(\beta_1, \beta_2) = (0,1)$. Оценивание методом максимального правдоподобия регрессии Пуассона $y$ на $x$ на выборке размером в 200 наблюдений даёт
\[
\hat{\E}[y|x] = \exp(-\underset{(0.089)}{0.165} + \underset{(0.069)}{1.124}x_2),
\]
где соответствующие стандартные ошибки представлены в скобках.

Результаты различных М-тестов представлены в таблице 8.1.

\begin{table}[h]
\begin{center}
\begin{scriptsize}
\caption{\label{tab:mtest} М-тесты на спецификацию для примера регрессии Пуассона}
\begin{minipage}{16cm}
\begin{tabular}[t]{l*{6}{{c}}}
\hline
\hline
\bf{Тип теста}\footnote{\begin{scriptsize} Процесс, порождающий данные для $y$ --- распределение Пуассона с параметром математического ожидания $\exp(0 + x_2)$ и размером выборки $N = 200$. М-тестовая статистика имеет хи-квадрат распределение с числом степеней свободы, которые представлены в столбце ст.св., и $p$-значения представлены в отдельном столбце. Альтернативная $M^{**}$ статистика имеет смысл только для тестов 3 и 4. \end{scriptsize}} & \bf{$H_0$, где $\mu = \exp(x' \beta)$} & \bf{$M^*$} & \bf{ст.св.} & \bf{$p$-значение} & \bf{$M^{**}$} \\
\hline
1. Верное математическое ожидание & $\E[(y - \mu)x_2^2] = 0$ & 3.27 & 1 & 0.07 & 0.44 \\
2. Дисперсия = математическое ожидание & $\E[\{(y - \mu)^2 - \mu\}|x] = 0$ & 2.43 & 2 & 0.30 & 1.89 \\
3. Дисперсия = математическое ожидание & $\E[\{(y - \mu)^2 - y\}|x] = 0$ & 2.43 & 2 & 0.30 & 2.41 \\
4. Информационная матрица & $\E[\{(y - \mu)^2 - y\}\Vech[xx']] = 0$ & 2.95 & 3 & 0.40 & 2.73 \\
5. Хи-квадрат критерия согласия & $\E[d - p] = 0$ & 2.50 & 3 & 0.48 & 0.75 \\
\hline
\hline
\end{tabular}
\end{minipage}
\end{scriptsize}
\end{center}
\end{table}

В качестве примера вычисления $M^*$, используя (8.32), рассмотрим информационно матричный тест. Так как $x = [1, x_2]'$ и $\Vech[xx'] = [1, x_2, x_2^2]'$, вспомогательная регрессия 1 на $\{(y - \hat{\mu})^2 - y\}$, $\{(y - \hat{\mu})^2 - y\}x_2$, $\{(y - \hat{\mu})^2 - y\}x_2^2$, $(y - \hat{\mu})$ и $(y - \hat{\mu})x_2$ даёт $R^2 = 0.01473$ при $N = 200$, что приводит к $M^* = 2.95$. Такое же значение $M^*$ можно получить и непосредственно из нецентрированной объясненной суммы квадратов, равной 2.95, так и косвенно как $N$ минус 197.05, остаточная сумма квадратов из этой регрессии. Тестовая статистика имеет $\chi^2(3)$ распределение с $p = 0.40$, поэтому нулевая гипотеза не отвергается на уровне значимости 0.05.

Для хи-квадрат теста на качество подгонки фактические частоты равны, соответственно, 0.435, 0.255 и 0.110,  соответствующие спрогнозированные вероятности равны 0.429, 0.241 и 0.124. Мы получаем $PCGF = 0.47$, используя (8.23), но эта статистика не имеет хи-квадрат распределения, так как она не учитывает ошибку при оценивании $\hat{\beta}$. Вспомогательная регрессия для правильной $CGF$ статистики из (8.27) приводит к $M^* = 2.50$, которая имеет хи-квадрат распределение.

В данном моделировании все пять моментных условий не отвергаются на уровне значимости 0.05, так как $p$-значение для $M^*$ превышает 0.05. Именно этот результат и ожидался, так как данные в этом примере  сгенерированы по заданному распределению, поэтому тесты на уровне значимости 0.05 должны отвергаться только в 5\% случаев. Альтернативная статистика $M^{**}$ имеет смысл только для тестов 3 и 4, так как только тогда $\E[\partial{m}/\partial{\beta}] = 0$, в противном случае она только даёт значение нижней границы для $M$.

\section{Тест Хаусмана}

Тесты, основанные на сравнении между двумя различными оценками, называются тестами Хаусмана, предложенными Хаусманом (1978), или тестами Ву-Хаусмана, или даже тестами Дарбина-Ву-Хаусмана, предложенными Ву (1973) и Дарбином (1954), который предложил аналогичные тесты.

\subsection{Тест Хаусмана}

Рассмотрим тест на эндогенность регрессора в одном уравнении. Две альтернативные оценки --- МНК- и ДМНК-оценки, где ДМНК-оценка использует инструменты для учёта возможной эндогенности регрессора. Если эндогенность присутствует, то МНК-оценка несостоятельна, так что две оценки будут иметь различные пределы по вероятности. Если нет эндогенности, то обе оценки состоятельны, поэтому они имеют и тот же предел по вероятности. Это говорит о том, что можно тестировать на эндогенность путём тестирования на разницу между МНК- и ДМНК-оценками, см. раздел 8.4.3 для подробностей.

В более общем случае рассмотрим две оценки $\hat{\theta}$ и $\tilde{\theta}$. Будем проводить тест на проверку приведённых ниже гипотез:
\begin{equation}
H_0: \plim(\hat{\theta} - \tilde{\theta}) = 0,
H_a: \plim(\hat{\theta} - \tilde{\theta}) \not= 0.
\end{equation}

Предположим, что разница между двумя $\sqrt{N}$ состоятельными оценками тоже $\sqrt{N}$-состоятельная при нулевой гипотезе с математическим ожиданием 0 и предельным нормальным распределением, и
\[
\sqrt{N}(\hat{\theta} - \tilde{\theta}) \stackrel{d}{\rightarrow} \mathcal{N}[0, V_H],
\]
где $V_H$ обозначает ковариационную матрицу для предельного распределения. Тогда статистика теста Хаусмана
\begin{equation}
H = (\hat{\theta} - \tilde{\theta})'(N^{-1}V_H)^{-1}(\hat{\theta} - \tilde{\theta})
\end{equation}
имеет $\chi^2(q)$ распределение при нулевой гипотезе. Мы отвергаем нулевую гипотезу на уровне значимости $\alpha$, если $H > \chi_{\alpha}^2(q)$.

В некоторых приложениях таких, как тесты на эндогенность, $\V[\hat{\theta} - \tilde{\theta}]$ имеет неполный ранг. Тогда используется обобщённая обратная матрица из (8.35) и хи-квадрат тест имеет число степеней
свободы равное рангу $\V[\hat{\theta} - \tilde{\theta}]$.

Тест Хаусмана может быть применён только для подмножества параметров. Так, например, интерес может заключаться исключительно в коэффициенте при, возможно, эндогенном регрессоре и в том, меняется ли он при переходе от МНК к ДМНК. Тогда лишь одна из компонент $\theta$ используется, и тестовая статистика имеет $\chi^2(1)$ распределение. Как и в других условиях, этот тест на подмножество параметров может привести к выводу, который отличается от теста на все параметры.

\subsection{Расчёт теста Хаусмана}

Посчитать тест Хаусмана в принципе несложно, но на практике это тяжело, так как необходимо получить состоятельную оценку $V_H$, предельной ковариационную матрицу $\sqrt{N}(\hat{\theta} - \tilde{\theta})$. В общем случае
\begin{equation}
N^{-1}V_H = \V[\hat{\theta} - \tilde{\theta}] = \V[\hat{\theta}] + \V[\tilde{\theta}] - 2\Cov[\hat{\theta},\tilde{\theta}].
\end{equation}
Первые две величины уже рассчитаны для обычного случая, но третье слагаемое --- нет.

\begin{center}
Расчёт для полностью эффективной оценки при нулевой гипотезе
\end{center}

Хотя в общем нулевая и альтернативная гипотезы для теста Хаусмана определены в (8.34), в приложениях, как правило, под нулевой и альтернативной гипотезами подразумеваются конкретные модели. Например, в сравнении МНК- и ДМНК-оценок нулевая гипотеза подразумевает, что все  регрессоры экзогенны, в то время как в альтернативной гипотезе допускаются эндогенные регрессоры.

Если $\hat{\theta}$ --- эффективная оценка для модели при нулевой гипотезе, то $\Cov[\hat{\theta},\tilde{\theta}] = \V[\hat{\theta}]$. Для доказательства смотрите упражнение 8.3. Это означает, что $\V[\hat{\theta} - \tilde{\theta}] = \V[\tilde{\theta}] - \V[\hat{\theta}]$, и
\begin{equation}
H = (\hat{\theta} - \tilde{\theta})'(\V[\tilde{\theta}] - \V[\hat{\theta}])^{-1}(\hat{\theta} - \tilde{\theta}).
\end{equation}

Эта статистика имеет значительное преимущество, так как ей необходимы только оценки асимптотических ковариационных матриц оценок параметров $\hat{\theta}$ и $\tilde{\theta}$. Полезно использовать программу, которая позволяет сохранить оценки параметров и оценки ковариационных матриц, а также в которой можно выполнять матричные вычисления.

Например, упрощение возможно для тестов на эндогенность в линейной регрессионной модели, если ошибки предполагаются гомоскедастичными. Тогда $\hat{\theta}$ --- МНК-оценка, которая полностью эффективна при нулевой гипотезе об отсутствии эндогенности, а $\tilde{\theta}$ --- ДМНК-оценка. Однако необходимо проявлять осторожность, чтобы состоятельные оценки ковариационных матриц оказались такими, что $\V[\tilde{\theta}] - \V[\hat{\theta}]$ была бы положительно определена (см. Рууд, 1984). При сравнении МНК и ДМНК-оценок в оценках ковариационных матриц $\V[\hat{\theta}]$ и $\V[\tilde{\theta}]$ должна использоваться одна и та же оценка дисперсии ошибки $\sigma^2$.

Версию (8.37) теста Хаусмана особенно легко рассчитать вручную, если $\theta$ является скаляром, или если проверяется только одна компонента вектора параметров. Тогда
\[
H = (\hat{\theta} - \tilde{\theta})^2/(\tilde{s}^2 - \hat{s}^2)
\]
имеет $\chi^2(1)$ распределение, а $\hat{s}$ и $\tilde{s}$ --- это стандартные ошибки $\hat{\theta}$ и $\tilde{\theta}$.

\begin{center}
Вспомогательные регрессии
\end{center}

В некоторых основных случаях тест Хаусмана может быть легко рассчитан как стандартный тест на значимость подмножества регрессоров в расширенной регрессии МНК, полученной при предположении о том, что $\hat{\theta}$ полностью эффективная оценка. Примеры приведены в разделе 8.4.3 и в разделе 21.4.3.

\begin{center}
Тесты Хаусмана с поправкой
\end{center}

Более простая версия (8.37) теста Хаусмана и стандартные вспомогательные регрессии требуют сильного предположения  о распределении --- эффективности $\hat{\theta}$. Это противоречит подходу робастного оценивания  при относительно слабых предположениях о распределении.

Прямое оценивание $Cov[\hat{\theta},\tilde{\theta}]$ и, следовательно, $V_H$, в принципе, возможно. Предположим, что $\hat{\theta}$ и $\tilde{\theta}$ --- М-оценки, которые получаются из решения $\sum_i h_{1i}(\hat{\theta}) = 0$ и $\sum_i h_{2i}(\tilde{\theta}) = 0$. Зададим $\hat{\delta}' = [\hat{\theta},\tilde{\theta}]$. Тогда $\V[\hat{\delta}] = G_0^{-1}S_0( G_0^{-1})'$, где $G_0$ и $S_0$ определены в разделе 6.6, c упрощением, что $G_{12} = 0$. В таком случае $\V[\hat{\theta} - \tilde{\theta}] = R\V[\hat{\delta}]R'$, где $R = [I_q, - I_q]$. На практике может потребоваться дополнительный программный код, который может быть особенным для каждого конкретного случая.

Более простой подход --- метод бутстрэп (см. раздел 11.6.3), хотя необходима осторожность в некоторых приложениях, для того посчитать правильное число степеней свободы хи-квадрат теста.

Другой возможный подход для неэффективной $\hat{\theta}$ --- использование вспомогательной регрессии, которая подходит в случае эффективных оценок. Но в данном случае надо провести тест на подмножество регрессоров, используя скорректированные стандартные ошибки. Этот тест с поправкой прост в реализации и будет иметь мощность при тестировании на неверную спецификацию, хотя он необязательно будет эквивалентным тесту Хаусмана, который использует более общий вид $H$, который приведён в (8.35). Пример приведён в разделе 21.4.3.

Наконец, могут быть рассчитаны границы, которые не требуют вычисления $\Cov[\hat{\theta},\tilde{\theta}]$. Для скалярных случайных величин $\Cov[x,y] \leq s_xs_y$. Для скалярного случая это приводит к верхней границе для $H$ $N(\hat{\theta} - \tilde{\theta})^2/(\tilde{s}^2 + \hat{s}^2 - 2\tilde{s}\hat{s})$, где $\hat{s}^2 = \hat{\V}[\hat{\theta}]$ и $\hat{s}^2 = \hat{\V}[\tilde{\theta}]$. Нижняя граница для $H$ $N(\hat{\theta} - \tilde{\theta})^2/(\tilde{s}^2 + \hat{s}^2)$ при предположении, что $\hat{\theta}$ и $\tilde{\theta}$ положительно коррелированы. На практике, однако, эти границы достаточно широкие.

\subsection{Мощность теста Вальда}

Тест Хаусмана --- это довольно общая процедура, которая не задаёт альтернативную гипотезу в явном виде, поэтому она необязательно имеет большую мощность против конкретных альтернатив. 

Рассмотрим, например, тесты на исключающие ограничения в полностью параметрических моделях. Нулевая гипотеза $H_0: \theta_2 = 0$, где $\theta = (\theta_1', \theta_2')'$. 


Очевидная спецификация --- тест Хаусмана на разницу $\hat{\theta}_1 - \tilde{\theta}_1$, где $(\hat{\theta}_1, \hat{\theta}_2)$ --- оценка, полученная с помощью метода максимального правдоподобия для неограниченной модели, и $(\tilde{\theta}_1, 0)$ --- оценка, полученная с помощью метода максимального правдоподобия для ограниченной модели.

Холли (1982) показал, что этот тест Хаусмана совпадает с классическим тестом ($Wald$, $LR$ или $LM$) c $H_0: \mathcal{I}_{11}^{-1}\mathcal{I}_{12}\theta_2 = 0$, где $\mathcal{I}_{ij} = \E[\partial{\mathcal{L}(\theta_1, \theta_2)}^2/\partial{\theta_i}\partial{\theta_i}]$, а не $H_0: \theta_2 = 0$. Два теста совпадают, если $\mathcal{I}_{12}$ имеет полный ранг и $\dim(\theta_1) \geq \dim(\theta_2)$, так как тогда $\mathcal{I}_{11}^{-1}\mathcal{I}_{12}\theta_2 = 0$. В противном случае они могут отличаться. Очевидно, что тест Хаусмана не будет мощным против $H_0$, если информационная матрица имеет блочно-диагональной вид, так как тогда $\mathcal{I}_{12} = 0$. Холли (1987) обобщил анализ на нелинейные гипотезы.

\section{Стандартные тесты на неверную спецификацию}

В этом разделе мы рассмотрим стандартные тесты на некоторые виды неверной спецификации модели. Внимание акцентируется на тестовой статистике, которую можно вычислить с помощью вспомогательной регрессии, используя минимальные предположения, позволяющие сделать вывод, основываясь на стандартных ошибках с поправкой на гетероскедастичность.

\subsection{Тест на пропущенные переменные}

Пропущенные переменные обычно приводят к несостоятельным оценкам параметров за исключением особых случаев таких, как ортогональность пропущенный регрессор к остальным в линейной модели. Поэтому всегда важно проверить модель на возможность пропуска переменных.

Наиболее часто используется тест Вальда, так как обычно оценить модель с  включенными потенциально пропущенными переменными, так же легко, как и с исключенными. Кроме того, этот тест может использовать скорректированные  стандартные ошибки в сэндвич-форме, хотя это имеет смысл, только если оценка остаётся состоятельной в подобной ситуациях.

В методе максимального правдоподобия в качестве альтернативы можно оценить модели с и без потенциально нерелевантных регрессоров и провести тест отношения правдоподобия.

Версию теста множителей Лагранжа с поправкой можно легко вычислить в некоторых условиях. Рассмотрим, например, тест $H_0: \beta_2 = 0$ в модели Пуассона с математическим ожиданием $\exp(x'\beta_1 + x'\beta_2)$. Статистика теста множителей Лагранжа основана на скор статистике $\sum_i x_i\tilde{u}_i$, где $\tilde{u}_i = y_i - \exp(x_{1i}'\tilde{\beta}_1)$ (см. раздел 7.3.2). Теперь оценка с поправкой на гетероскедастичность дисперсии $N^{-1/2}\sum_i x_iu_i$, где $u_i = y_i - \E[y_i|x_i]$, равна $N^{-1}\sum_i u_i^2x_ix_i'$, и можно показать, что
\[
LM^+ = \left[ \sum_{i=1}^n x_i\tilde{u}_i \right]' \left[\sum_{i=1}^n \tilde{u}_i^2x_ix_i'\right]^{-1} \left[ \sum_{i=1}^n x_i\tilde{u}_i\right]
\]
является статистикой  теста множителей Лагранжа с поправкой, которая не требует ограничения Пуассона, что $\V[u_i|x_i] = \exp(x_{1i}'\beta_1)$ при нулевой гипотезе. Статистику можно вычислить как $NR_u^2$ из регрессии 1 на $x_{1i}\tilde{u}_i$ и $x_{2i}\tilde{u}_i$. Такие тесты множителей Лагранжа с поправкой возможны в более общем случае для моделей из  экспоненциального семейства, так как скор статистика в таких моделях --- средневзвешенная сумма остатков $\tilde{u}_i$ (см. Вулдридж, 1991). Этот класс включает МНК, также возможны модификации, когда оценивание проводится с помощью ДМНК или НМНК, см. Вулдридж (2002).

\subsection{Тесты на гетероскедастичность}

Оценки параметров в линейной или нелинейной регрессионных моделях условного математического ожидания, которые оценивается с помощью метода наименьших квадратов или инструментальных переменных, сохраняют свою состоятельность при наличии гетероскедастичности. Единственное, что необходимо, --- скорректировать стандартные ошибки этих оценок. Это не требует моделирования гетероскедастичности, а стандартные ошибки с поправкой на гетероскедастичность можно вычислить при минимальных предположениях о распределении, используя результат Уайта (1980). Таким образом, нет необходимости проводить тест на гетероскедастичность кроме случая, когда эффективность оценки вызывает большое беспокойство. Тем не менее, мы обобщим некоторые результаты тестов на гетероскедастичность.

Начнём с оценивания методом наименьших квадратов модели линейной регрессии $y = x'\beta + u$. Предположим, что гетероскедастичность моделируется как $\V[u|x] = g(\alpha_1 + z'\alpha_2)$, где $z$, как правило, --- подмножество $x$ и $g(\cdot)$ --- экспоненциальная функция. В литературе основное внимание уделяется тестам на $H_0: \alpha_2 = 0$ методом множителей Лагранжа, потому что в отличие от теста Вальда и теста отношения правдоподобия он требуют только получения МНК-оценки $\beta$. Стандартный тест множителей Лагранжа Брейша и Пагана (1979) во многом основывается на предположении о нормальном распределении ошибок, так как он предполагает, что $\E[u^4|x^4] = 3\sigma^4$ при нулевой гипотезе. Коенкер (1981) предложил более надёжную версию теста множителей Лагранжа, $NR^2$ из регрессии $\hat{u}_i^2$ на 1 и $z_i$, где $\hat{u}_i$ --- МНК-остатки. Этот тест требует слабое предположение о том, что $\E[u^4|x]$ является константой. Как и тест Бройша-Пагана, он инвариантен к выбору функции $g(\cdot)$. Тест Уайта (1980a) на гетероскедастичность эквивалентен этому тесту множителей Лагранжа с $z = \Vech[xx']$. Тест можно обобщить, чтобы $\E[u^4|x]$ изменялось по $x$, хотя постоянство может быть разумным предположением для теста, так как нулевая гипотеза уже указывает, что $\E[u^2|x]$ является константой.

Качественно подобные результаты переносятся и на нелинейные модели условного математического ожидания, предполагающие определённую форму гетероскедастичности, которую можно проверить на неверную спецификацию. Например, в модели регрессии Пуассона $\V[y|x] = \exp(x'\beta)$. В более общем случае для моделей из линейного экспоненциального семейства, метод квази-максимального правдоподобия является состоятельным, несмотря на неверную спецификацию гетероскедастичности, и для него верны качественные результаты  аналогичные представленным здесь. При использовании скорректированных стандартных ошибок, представленных в разделе 5.7.4 будут сделаны верные статистические выводы, даже если вид гетероскедастичности неверно специфицирован. Если исследователь всё же хочет проверить гетероскедастичность на верную спецификацию, то можно провести тесты множителей Лагранжа с поправкой (см. Вулдридж, 1991).

Гетероскедастичность может привести к более серьёзным последствиям несостоятельности оценок параметров в некоторых нелинейных моделях. Ярким примером является модель тобит (см. главу 16), линейная регрессионная модель с нормальными гомоскедастичными ошибками, которая становится нелинейной в результате цензурирования или усечения. В таком случае проверка на гетероскедастичность становится более важной. Задав модель для $\V[u|x]$, можно выполнить тест Вальда, тест отношения правдоподобия и тест множителей Лагранжа, или можно использовать М-тесты на гетероскедастичность (см. Паган и Велла, 1989).

\subsubsection{Тесты Хаусмана на эндогенность}

Оценки инструментальных переменных следует использовать только там, где есть необходимость в них, так как оценки, полученные с помощью метода наименьших квадратов, являются более эффективными, если все регрессоры экзогенные и, из раздела 4.9, эта потеря эффективности может быть существенной. Может быть полезно проверить, нужен ли метод инструментальных переменных или нет. Тест на эндогенность регрессоров сравнивает оценки инструментальных переменных с оценками МНК. Если регрессоры являются эндогенными, то в пределе эти оценки будут отличаться. Если регрессоры экзогенны, то две оценки не будут отличаться. Таким образом, большие различия между МНК-оценками и оценками инструментальных переменных можно интерпретировать как свидетельство эндогенности.

Этот пример иллюстрирует первоначальную мотивацию для теста Хаусмана. Рассмотрим модель линейной регрессии
\begin{equation}
y = x_1'\beta_1 + x_2'\beta_2 + u,
\end{equation}
где $x_1$ потенциально эндогенный и $x_2$ экзогенный. Пусть $\hat{\beta}$ --- МНК-оценка, а $\tilde{\theta}$ --- ДМНК-оценка в (8.38). Предположим гомоскедастичность ошибок. МНК-оценка является эффективной при нулевой гипотезе об отсутствии эндогенности, тест Хаусмана на эндогенность $x_1$ можно вычислить с помощью тестовой статистики $H$, которая определена в (8.37). Поскольку матрица $\V[\hat{\beta}] - \V[\tilde{\beta}]$ может быть неполного ранга, необходима обобщённая обратная матрица, и число степеней свободы равно $\dim(\beta_1)$, а не $\dim(\beta)$.

Хаусман (1978) показал, что тест можно проще провести с помощь проверки $\gamma = 0$ в расширенной регрессии МНК
\[
y = x_1'\beta_1 + x_2'\beta_2 + \hat{x}_1'\gamma + u,
\]
где $\hat{x}_1$ --- предсказанное значение эндогенного регрессора $x_1$ из сокращённой формы многомерной регрессии $x_1$ на инструменты $z$. Эквивалентно мы можем проверить $\gamma = 0$ в расширенной регрессии МНК
\[
y = x_1'\beta_1 + x_2'\beta_2 + \hat{v}_1'\gamma + u,
\]
где $\hat{v}_1$ --- остатки из приведённой формы многомерной регрессии $x_1$ на инструменты $z$. Интуиция этих тестов состоит в том, что если $u$ из (8.38) не коррелирует с $x_1$ и $x_2$, то $\gamma = 0$. Если вместо этого $u$ коррелирует с $x_1$, то это будет подхвачено значимостью дополнительных преобразованных $x_1$ таких, как $\hat{x}_1$ и $\hat{v}_1$.

Для пространственных данных принято предполагать гетероскедастичность ошибок. Тогда МНК-оценка $\hat{\beta}$ является неэффективной в (8.38), и нельзя использовать более простую версию (8.37) теста Хаусмана. Однако можно использовать представленные выше расширенные регрессии МНК при условии, что $\gamma = 0$ проверяется с использованием оценки ковариационной матрицы, устойчивой к гетероскедастичности. Это эквивалентно тесту Хаусмана, в книге Дэвидсона и МакКиннона (1993, с. 239) показано, что $\hat{\gamma}_{OLS}$ в этих расширенных регрессиях равна $A_N(\hat{\beta} - \tilde{\beta})$, где $A_N$ --- матрица полного ранга с конечным пределом по вероятности.

Дополнительные тесты Хаусмана на эндогенность возможны. Пусть $y =  x_1'\beta_1 + x_2'\beta_2 + x_3'\beta_3 + u$, где $x_1$ потенциально эндогенный, $x_2$ считается эндогенным, а $x_3$ считается экзогенным. Тогда эндогенность $x_1$ может быть проверена с помощью сравнения ДМНК-оценки с инструментами только для $x_2$ и ДМНК-оценки с инструментами для $x_1$ и $x_2$. Тест Хаусмана также может быть обобщён на нелинейные регрессионные модели с МНК, заменённым на НМНК, и ДМНК, заменённым на ДНМНК. Дэвидсон и МакКиннон (1993) представили расширенные регрессии, которые могут быть использованы для вычисления релевантного теста Хаусмана, предполагая гомоскедастичность ошибок. Мроз (1987) приводит хорошие примеры тестов на эндогенность, включая примеры вычисления $\V[\hat{\theta} - \tilde{\theta}]$, когда $\hat{\theta}$ не является эффективной.

\subsubsection{OIR тесты на экзогенность}

Если используются оценки инструментальных переменных, то инструменты должны быть экзогенными, для того чтобы оценки были состоятельными. Для точно идентифицированных моделей невозможно провести тест на экзогенность инструментов. Вместо этого для подтверждения действенности инструмента должны использоваться априорные показатели. Некоторые примеры приведены в разделе 4.8.2. Однако для сверхидентифицированных моделей можно провести тест на экзогенность инструментов.

Начнём с линейной регрессии. Тогда $y = x'\beta + u$ и инструменты $z$ действенны, если $\E[u|z] = 0$ или если $\E[zu] = 0$. Очевидный тест на $H_0: \E[zu] = 0$ основан на том, насколько величина $N^{-1}\sum_i z_i\hat{u}_i$ далека от нуля. В точно идентифицированном случае оценка инструментальных переменных является решением $N^{-1}\sum_i z_i\hat{u}_i = 0$, поэтому этот тест не является полезным. 

В случае сверх-идентифицированной модели, тест на сверх-идентифицирующие  ограничения, представленный в разделе 6.3.8, имеет вид
\begin{equation}
OIR = \hat{u}Z\hat{S}^{-1}Z'\hat{u},
\end{equation}
где $\hat{u} = y - X\hat{\beta}$ --- это оптимальная оценка, полученная с помощью обобщённого метода моментов, и $S$ --- состоятельная оценка для $\plim N^{-1}\sum_i u_i^2z_iz_i'$. OIR тест Хансена (1982) является расширением теста, предложенного Сарганом (1958) для линейного метода инструментальных переменных, и тестовую статистику (8.39) часто называют тестом Саргана. Если значение $OIR$ велико, то моментные условия отвергаются и оценка метода инструментальных переменных несостоятельна. Отвержение нулевой гипотезы обычно интерпретируется как доказательство того, что инструменты $z$ являются эндогенными, но это также может быть свидетельством неверной спецификации модели, т.е. что на самом деле $y \not= x'\beta + u$. В любом случае тот факт, что гипотеза отвергается, указывает на проблемы с оценкой.

Как формально показано в разделе 6.3.9, $OIR$ имеет $\chi^2(r - K)$ распределение при нулевой гипотезе, где $(r - K)$ --- количество сверх-идентифицирующих ограничений. Чтобы получить некоторую интуицию для этого результата, полезно использовать гомоскедастичные ошибки. Тогда $\hat{S} = \hat{\sigma}^2Z'Z$, где $\hat{\sigma}^2 = \hat{u}'\hat{u}/(N - K)$, поэтому
\[
OIR = \frac{\hat{u}'P_z\hat{u}}{\hat{u}'\hat{u}/(N - K)},
\]
где $P_Z = Z(Z'Z)^{-1}Z'$. Таким образом, $OIR$ является отношением квадратичных форм $\hat{u}$. При нулевой гипотезе числитель имеет предел по вероятности $\sigma^2(r - K)$, а знаменатель имеет $\plim\hat{\sigma}^2 = \sigma^2$, так что дробь примерно равна $r - K$, т.е. математическому ожиданию $\chi^2(r - K)$ случайной величины.

Тестовая статистика (8.39) обобщается сразу на нелинейную регрессию, путем определения $u_i = y - g(x, \beta)$ или $u_i = r(y,x,\beta)$, как и в разделе 6.5, а также на линейные системы и на панельные оценки с помощью соответствующего определения $u$ (см. разделы 6.9 и 6.10).

Для линейного метода инструментальных переменных с гомоскедастичными ошибками были предложены тесты, альтернативные $OIR$ из (8.39). Мандалинос (1988) сравнивает такие тесты. Можно также использовать вариации $OIR$ тестов на подмножество сверх-идентифицирующих ограничений.

\subsubsection{RESET-тест}

Типичная ошибка  спецификации функциональной формы может состоять в игнорировании нелинейности зависимости от некоторых регрессоров. Рассмотрим регрессию $y = x'\beta + u$, где предполагается, что регрессоры линейны и асимптотически не коррелируют с ошибкой $u$. Прямой способ проверить нелинейность --- это добавить в список регрессоров степени экзогенных переменных, обычно квадраты,  и проверить статистическую значимость этих дополнительных переменных с помощью теста Вальда или $F$-теста. У исследователя должны быть конкретные основания предполагать нелинейность, и ясно, что метод не будет работать для категориальных переменных $x$.

Рамсей (1969) предложил тест на пропущенные переменные в регрессии, которые могут быть сформулированы в виде теста на функциональную форму. 

Предложение Рамсея --- найти оценённые значения из начальной регрессии и сгенерировать новые регрессоры, которые являются функциями от оценённых значений $\hat{y} = x'\hat{\beta}$, например, $w = [(x'\hat{\beta})^2, (x'\hat{\beta})^3, \dots, (x'\hat{\beta})^p]$. Затем необходимо оценить модель $y = x'\beta + w'\gamma + u $ и провести тест на нелинейность --- тест Вальда на $p$ ограничений, $H_0: \gamma = 0$ против $H_a: \gamma \not= 0$. Обычно используются небольшие значения $p$ такие, как 2 или 3. Для этого теста можно сделать поправку на  гетероскедастичность.

\section{Выбор между невложенными моделями}

Две модели являются вложенными, если одна является частным случаем другой. Они являются невложенными, если ни одна из них не может быть представлена как частный случай другой. Выбор между вложенными моделями можно делать с помощью стандартного теста на параметрические ограничения, которые сводят одну модель к другой. Однако в случае невложенных моделей должны быть разработаны альтернативные методы.

Особое внимание уделяется выбору между невложенными моделями в рамках метода максимального правдоподобия, где теория является хорошо разработанной. Краткое описание остальных случаев приведено в разделе 8.5.4. Байесовские методы для выбора между моделями приведены в разделе 13.8.

\begin{center}
Информационные критерии
\end{center}

Информационные критерии строятся на логарифме функции правдоподобия с учетом числа степеней свободы. Модель с наименьшим информационным критерий является предпочтительной.

Интуиция состоит в том, что есть выбор между качеством подгонки модели, измеряемым с помощью логарифма функции правдоподобия, и принципом бережливости, который выступает в пользу простой модели. Подгонка модели может быть улучшена за счёт увеличения сложности модели. Тем не менее, параметры добавляются, только если улучшение подгонки компенсирует потерю бережливости. Отметим, что с этой точки зрения необязательно, чтобы множество рассматриваемых моделей включало <<истинный>> процесс, порождающий данные. Различные информационные критерии отличаются тем, насколько сильно они штрафуют модель за сложность.
Акаике (1973) предложил информационный критерий Акаике (Akaire Information Criterion)
\begin{equation}
AIC = - 2\ln L + 2q,
\end{equation}
где $q$ --- число параметров. Выбирается модель с наименьшим $AIC$. Термин информационный критерий используется, потому что лежащая в основе теория, представленная проще всего в книге Амэмия (1980), выбирает между моделями, используя информационный критерий Кульбака-Лейблера $(KLIC)$.

Было предложено значительное количество модификаций $AIC$, которые имеют вид $ - 2\ln L + g(q, N)$ для указанной  функции-штрафа $g(\cdot)$, которая превышает $2q$. Самый популярный вариант --- Байесовский информационный критерий (Bayesian Information Criterion)
\begin{equation}
BIC = - 2\ln L + (\ln L)q,
\end{equation}
который был предложен Шварцем (1978). Шварц предполагал, что $y$ имеет плотность из экспоненциального семейства с параметром $\theta$, $j$-тая модель имеет параметр $\theta_j$ с $\dim(\theta_j) = q_j$, и априорные вероятности для всех моделей равны. Он показал, что при этих предположениях максимизация апостериорной вероятности (см. главу 13) является асимптотически эквивалентной выбору модели, для которой величина $- 2\ln L + (\ln L)q_j/2$ максимальна. Так как это эквивалентно минимизации (8.41), процедура Шварца была названа Байесовским информационным критерием. Версия $AIC$, основанная на минимизации $KLIC$, похожая на $BIC$, называется состоятельным (consistent) $AIC$, $CAIC = - 2\ln L + (1 + \ln N)q$. Некоторые авторы определяют такие критерии, как $AIC$ и $BIC$, путём дополнительного деления на $N$ правой части равенства (8.40) и (8.41).

Если важна простота модели, то чаще используется $BIC$, так как штраф за размер модели у $AIC$ является относительно маленьким. Рассмотрим две вложенные модели с $q_1$ и $q_2$ параметрами соответственно, где $q_2 = q_1 + h$. Тогда можно провести тест отношения правдоподобия, по результатам которого расширенная модель лучше на уровне значимости 5\%, если $2\ln L$ увеличивается на $\chi_{0.05}^2(h)$. По $AIC$ можно сделать вывод, что расширенная модель лучше, если $2\ln L$ возрастает больше, чем на $2h$. Штраф за размер модели у $AIC$ меньше, чем у теста отношения правдоподобия при $h < 7$. В частности, для $h = 1$, то есть при одном ограничении, на уровне значимости 5\% для теста отношения правдоподобия критическое значение равно 3.84, тогда как для $AIC$ оно намного меньше и равно 2. По $BIC$ можно сказать, что расширенная модель лучше, если $2\ln L$ увеличивается на $h\ln L$. Этот штраф гораздо больше, чем $AIC$ или тест отношения правдоподобия с уровнем значимости 0.05 (кроме случая, когда $N$ очень мало).

Байесовский информационный критерий увеличивает штраф с увеличением размера выборки, в то время как традиционные тесты гипотезы на уровне значимости, например, 5\% не имеют такой особенности. Для вложенных моделей с $q_2 = q_1 + 1$ выбор расширенной модели на основе более низкого значения $BIC$ эквивалентен использованию критического значения двустороннего $t$-теста $\sqrt{\ln N}$, которое равно 2.15, 3.03 и 3.72 для $N = 102$, 104 и 106 соответственно. Для сравнения тесты на традиционные гипотезы с размером 0.05 используют неизменное критическое значение 1.96. В более общем случае для тестовой статистики, имеющей $\chi^2(h)$ распределение, $BIC$ предлагает использовать критическое значение $h\ln L$, а не обычное $\chi^2(h)$.

Учитывая их простоту, критерии, которые накладывают штраф и которые основаны на методе максимального правдоподобия, часто используются для выбора <<лучшей модели>>. Тем не менее, нет чёткого ответа на вопрос, какому критерию, если таковой имеется, следует отдать предпочтение. Используется значительная аппроксимация при выводе формул для $AIC$ и связанных с ним показателей. Функции потерь, отличные от минимизации $KLIC$, или максимизация апостериорной вероятности в случае $BIC$, могут быть гораздо более уместными. С теоретической точки зрения, выбор модели из множества моделей должен зависеть от предполагаемого применения этой модели. Например, целью модели может быть краткое описание основных особенностей реальности или прогнозирование некоторых результатов, или проверка некоторой важной гипотезы. В прикладных работах редко можно увидеть ясное изложение предполагаемого использования эконометрической модели.

\subsection{Тест отношения правдоподобия Кокса на невложенные модели}

Рассмотрим выбор между параметрическими моделями. Пусть модель $F_{\theta}$ имеет функцию плотности $f(y|x, \theta)$, а модель $G_{\gamma}$ --- $g(y|x, \gamma)$.

Тест отношения правдоподобия для модели $F_{\theta}$ против модели $G_{\gamma}$ основан на
\begin{equation}
LR(\hat{\theta}, \hat{\gamma}) = \mathcal{L}_f(\hat{\theta}) - \mathcal{L}_g(\hat{\gamma}) = \sum_{i=1}^N \ln  \frac{f(y_i|x_i, \hat{\theta})}{g(y_i|x_i, \hat{\gamma})}.
\end{equation}
Если $G_{\gamma}$ вложена в $F_{\theta}$, то, согласно разделу 7.3.1,  $2LR(\hat{\theta}, \hat{\gamma})$ имеет хи-квадрат распределение при нулевой гипотезе, что $F_{\theta} = G_{\gamma}$. Тем не менее, этот результат не выполняется, если модели являются невложенными. Кокс (1961, 1962б) предложил решение этой задачи для случая, когда $F_{\theta}$ является истинной моделью, но модели не являются вложенными, с помощью применения центральной предельной теоремы.

Расчёты для такого подхода нелегко осуществить, если нельзя получить аналитически \\ $\E_f[\ln (f(y|x, \theta)/g(y|x, \gamma))]$, где $\E_f$ обозначает математическое ожидание. Кроме того, если получить подобную тестовую статистику с $F_{\theta}$ и  $G_{\gamma}$, помененными ролями, то можно получить, что модель $F_{\theta}$ отвергается в пользу $G_{\gamma}$ и что модель $G_{\gamma}$ отвергается в пользу $F_{\theta}$. По этой причине тест необязательно является тестом на выбор модели, поскольку он необязательно выбирает одну или другую, вместо этого он может выбрать одну модель, обе или ни одной.

В некоторых случаях статистика Кокса может быть получена аналитически. Про невложенные линейные регрессионные модели $y = x'\beta + u$ и и $y = z'\gamma + v$ с гомоскедастичными нормально распределёнными ошибками можно посмотреть книгу Песарана (1974). Про преобразования невложенных моделей $h(y) = x'\beta + u$ и $g(y) = z'\gamma + v$, где $h(y)$ и $g(y)$ --- известные преобразования, написано у Песарана и Песарана (1995), которые использует симуляционный подход. Это позволяет, например, выбрать между линейной и лог-линейной параметрическими моделями с тождественным преобразованием $h(\cdot)$  и логарифмическим преобразованием $g(\cdot)$. Песаран и Песаран (1995) применяют идею о выборе между логит и пробит-моделями, представленными в главе 14.

\subsection{Тест отношения правдоподобия Вуонга на невложенные модели}

Вуонг (1989) привёл общую теорию распределения для статистики теста отношения правдоподобия, которая охватывает как вложенные, так и невложенные модели, а также позволяет процессу, порождающему данные, иметь неизвестную плотность, которая отличается от $f(\cdot)$ и $g(\cdot)$.

Асимптотические результаты Вуонга, представленные здесь, чтобы облегчить понимание тестов описанных в статье Вуонга, являются относительно сложными, так как в некоторых случаях тестовая статистика --- взвешенная сумма хи-квадратов с весами, которые может быть сложно вычислить. Вуонг предложил тест
\begin{equation}
H_0: \E_0 \left[ \ln \frac{f(y|x, \theta)}{g(y|x, \gamma)} \right] = 0,
\end{equation}
где $\E_0$ обозначает математическое ожидание по отношению к истинному процессу, порождающему данные, $h(y|x)$, который может быть неизвестным. Это эквивалентно тестированию $\E_h[\ln (h/g)] - \E_h[\ln (h/f)] = 0$ или тестированию на то, имеют ли две плотности $f$ и $g$ одинаковый информационный критерий Кульбака-Лейблера (см. раздел 5.7.2). Возможны односторонние альтернативные варианты с $H_f: \E_0[\ln (f/g)] > 0$ и $H_g: \E_0[\ln (f/g)] < 0$.

Очевидным тестом для проверки нулевой гипотезы является М-тест на то, отличается ли от нуля выборочный аналог $LR(\hat{\theta}, \hat{\gamma})$, определённый в (8.42). Здесь необходимо получить распределение тестовой статистики с возможно неизвестным процессом, порождающим данные. Это возможно, поскольку из раздела 5.7.1 оценка, полученная с помощью метода квази-максимального правдоподобия, $\hat{\theta}$ сходится к псевдо-истинному значению $\theta^*$, и $\sqrt{N}(\hat{\theta} - \theta^*)$ имеет предельное нормальное распределение. Аналогичный результат верен для оценки $\hat{\gamma}$, полученной с помощью метода квази-максимального правдоподобия.

\begin{center}
Общий результат
\end{center}

Полученное распределение $LR(\hat{\theta}, \hat{\gamma})$ варьируется в зависимости от того, действительно ли две модели, возможно, обе некорректные, эквивалентны в том смысле, что $f(y|x, \theta_*) = g(y|x, \gamma_*)$, где $\theta_*$ и $\gamma_*$ являются псевдо-истинными значениями $\theta$ и $\gamma$.

Если $f(y|x, \theta_*) = g(y|x, \gamma_*)$, то
\begin{equation}
2LR(\hat{\theta}, \hat{\gamma}) \stackrel{d}{\rightarrow} M_{p + q}(\lambda_*),
\end{equation}
где $p$ и $q$ --- размеры $\theta$ и $\gamma$, а $M_{p + q}(\lambda_*)$ обозначает функцию распределения взвешенной суммы хи-квадрат переменных $\sum_{j=1}^{p+q} \lambda_{*j}Z_j^2$. Величины $Z_j^2$ независимы и одинаково распределены по $\chi^2(1)$ и $\lambda_{*j}$ являются собственными значениями матрицы размера $(p+q)(p+q)$
\begin{equation}
W = \begin{bmatrix} -B_f(\theta_*)A_f(\theta_*)^{-1} & -B_{fg}(\theta_*, \gamma_*)A_g(\gamma_*)^{-1} \\ -B_{gf}(\gamma_*,\theta_*)A_f(\theta_*)^{-1} & -B_g(\gamma_*)A_g(\gamma_*)^{-1} \end{bmatrix},
\end{equation}
где $A_f(\theta_*) = \E_0[\partial{\ln f}^2/\partial{\theta}\partial{\theta}']$, $B_f(\theta_*) = \E_0[(\partial{\ln f}/\partial{\theta})(\partial{\ln f}/\partial{\theta}')]$. Матрицы $A_g(\gamma_*)$ и $B_g(\gamma_*)$ аналогично определяются для плотности $g(\cdot)$, кросс-матрица $B_{fg}(\theta_*, \gamma_*) = \E_0[(\partial{\ln f}/\partial{\theta})(\partial{\ln g}/\partial{\gamma}')]$, и математические ожидания считаются по истинному процессу, порождающему данные. Объяснение и вывод этих результатов можно посмотреть у Вуонга (1989).

Если наоборот $f(y|x, \theta_*) \not= g(y|x, \gamma_*)$, то при нулевой гипотезе
\begin{equation}
N^{-1/2}LR(\hat{\theta},\hat{\gamma}) \stackrel{d}{\rightarrow} \mathcal{N}[0, w_*^2],
\end{equation}
где
\begin{equation}
w_*^2 = V_0 \left[ \ln \frac{f(y|x, \theta_*)}{g(y|x, \gamma_*)} \right],
\end{equation}
и дисперсия считается по истинному процессу, порождающему данные. Вывод можно посмотреть в работе Вуонга (1989).

Использование этих результатов зависит от вида вложенности двух моделей и от того, предполагается ли, что одна модель верно специфицирована, или нет.

Вуонг выбирал между тремя типами сравнения моделей. Модели $F_{\theta}$ и $G_{\gamma}$ являются (1) вложенными так, что $G_{\gamma}$ вложена в $F_{\theta}$, если $G_{\gamma} \subset F_{\theta}$; (2) строго невложенными моделями тогда и только тогда, когда $F_{\theta} \cap G_{\gamma} = \emptyset$, то есть ни одну из моделей нельзя привести к другой; и (3) пересекающимися, если $F_{\theta} \cap G_{\gamma} \not= \emptyset$, $F_{\theta} \varsubsetneq G_{\gamma}$ и $G_{\gamma} \varsubsetneq F_{\theta}$. Похожие методы выбора описаны Песараном и Песараном (1995).

И (2), и (3) являются невложенными моделями, но они требуют различных процедур тестирования. Примерами строго невложенных моделей являются линейные модели с различным распределением ошибок и нелинейные регрессионные модели с одним и тем же распределением ошибок, но разными функциональными формами условного математического ожидания. Для пересекающихся моделей некоторые частные случаи этих двух моделей совпадают. Примером являются линейные модели с частично совпадающим множеством регрессоров.

\begin{center}
Вложенные модели
\end{center}

Для вложенных моделей обязательно выполнено условие $f(y|x, \theta_*) = g(y|x, \gamma_*)$. Для $G_{\gamma}$, вложенной в $F_{\theta}$, $H_0$ проверяется с помощью $H_f: \E_0[\ln (f/g)] > 0$.

Для плотности, возможно, неверно специфицированной, результат (8.44) про взвешенную сумму хи-квадратов остается верным, если использовать собственные значения $\hat{\lambda}_j$ выборочного аналога $W$ из (8.45). Также можно использовать собственные значения $\tilde{\lambda}_j$ выборочного аналога для меньшей матрицы
\[
\underbar{W} = B_f(\theta_*)[D(\gamma_*)A_g(\gamma_*)^{-1}D(\gamma_*)' - A_f(\theta_*)^{-1}],
\]
где $D(\gamma_*) = \partial{\phi(\gamma_*)}/\partial{\gamma}$ и $\tilde{\theta} = \phi(\hat{\gamma})$ --- оценка, полученная с помощью ограниченного метода квази-максимального правдоподобия, как и у Вуонга (1989). Данный тест является робастной версией стандартного теста отношения правдоподобия на вложенные модели.

Если плотность $f(\cdot)$ на самом деле верно специфицирована, или в более общем случае удовлетворяет
равенству информационных матриц, мы получим ожидаемый результат, что $2LR(\hat{\theta}, \hat{\gamma}) \stackrel{d}{\rightarrow} \chi^2(p - q)$, так как $(p - q)$ собственных значений $W$ или $\underbar{W}$ равны единице, в то время как все остальные равны нулю.

\begin{center}
Строго невложенные модели
\end{center}

Для строго невложенных моделей выполнено условие $f(y|x, \theta_*) \not= g(y|x, \gamma_*)$. Результат (8.46) о нормальности распределения верен, а состоятельная оценка $w_*^2$ имеет вид:
\begin{equation}
\hat{w}^2 = \frac{1}{N} \sum_{i=1}^N \left( \ln \frac{f(y_i|x_i, \hat{\theta})}{g(y_i|x_i, \hat{\gamma})} \right)^2 - \left( \frac{1}{N} \sum_{i=1}^N \ln \frac{f(y_i|x_i, \hat{\theta})}{g(y_i|x_i, \hat{\gamma})} \right)^2.
\end{equation}
Поэтому
\begin{equation}
T_{LR} = N^{-1/2}LR(\hat{\theta}, \hat{\gamma})/\hat{w} \stackrel{d}{\rightarrow} \mathcal{N}[0,1].
\end{equation}
Для тестов с критическим значением $c$, $H_0$ отвергается в пользу $H_f: \E_0[\ln (f/g)] > 0$, если $T_{LR} > c$, $H_0$ отвергается в пользу $H_g: \E_0[\ln (f/g)] < 0$, если $T_{LR} < - c$, и выбор между двумя моделями невозможен, если $|T_{LR}| < c$. Данный тест может быть модифицирован, чтобы можно было вводить штрафы на базе логарифма функции правдоподобия аналогично $AIC$ и $BIC$, см. Вуонг (1989, стр. 316). Асимптотически эквивалентную статистику для (8.49) можно получить, заменив $\hat{w}^2$ на $\tilde{w}^2$, равной первому члену правой части выражения (8.48).

Данный тест предполагает, что обе модели имеют неверную спецификацию. Если вместо этого предполагается, что одна из моделей имеет верную спецификацию, то необходимо использовать подход Кокса из раздел 8.5.2.

\begin{center}
Пересекающие модели
\end{center}

Для пересекающихся моделей неясно априори, будет ли $f(y|x, \theta_*) = g(y|x, \gamma_*)$ или нет, и необходимо сначала проверить это условие.

Вуонг (1989) предлагает проверку на то, равна ли нулю дисперсия $w_*^2$, определённая в (8.47), так как $w_*^2 = 0$, если и только если $f(\cdot) = g(\cdot)$. Таким образом, необходимо вычислить $\hat{w}^2$ из (8.48). При нулевой гипотезе $H_0^w: w_*^2 = 0$
\begin{equation}
N\hat{w}^2 \stackrel{d}{\rightarrow} M_{p + q}(\lambda_*),
\end{equation}
где распределение $M_{p + q}(\lambda_*)$ определяется как (8.44). Гипотеза $H_0^w$ отвергается на уровне значимости $\alpha$, если $N\hat{w}^2$ превышает верхний процентный квантиль $\alpha$ у $M_{p + q}(\hat{\lambda})$ распределения, при этом используются собственные значения $\hat{\lambda}_j$ выборочного аналога $W$ из (8.45). Альтернативно и более просто можно проверить гипотезу, что $\theta_*$ и $\gamma_*$ удовлетворяют равенству $f(\cdot) = g(\cdot)$. Примеры приведены Лиеном и Вуонгом (1987).

Если $H_0^w$ не отвергается или условия для равенства $f(\cdot) = g(\cdot)$ не отвергаются, то можно сделать вывод, что невозможно выбирать между двумя моделями с учётом имеющихся данных. Если $H_0^w$ отвергается или  условия для равенства $f(\cdot) = g(\cdot)$  отвергаются, то необходимо протестировать $H_0$ против $H_f$ или $H_g$, используя $T_{LR}$, как указано в случае строго невложенных моделей. В этом случае уровень значимости не превышает максимального уровни значимости для каждого из двух тестов.

Этот тест предполагает, что обе модели имеют неверную спецификацию. Если вместо этого предполагается, что одна из моделей верно специфицирована, то другая модель также должна быть верно специфицирована, для того чтобы обе модели были эквивалентными. Таким образом, $f(y|x, \theta_*) = g(y|x, \gamma_*)$ при $H_0$, и можно перейти к тесту отношения правдоподобия, используя взвешенный хи-квадрат результат (8.44). Пусть $c_1$ и $c_2$ --- это верхнее и нижнее критические значения соответственно. Если $2LR(\hat{\theta}, \hat{\gamma}) > c_1$ , то $H_0$ отвергается в пользу $H_f$, если $2LR(\hat{\theta}, \hat{\gamma}) < c_2$, то $H_0$ отвергается в пользу $H_g$, и в противном случае тест не позволяет сделать однозначного  вывода.

\begin{center}
Другие сравнения невложенных моделей
\end{center}

Предшествующие методы можно использовать только для полностью параметрических моделей. Методы для выбора между моделями, которые лишь частично параметризованы, такие, как линейная регрессия без предположения о нормальном распределении, являются менее изученными.

Информационный критерий из раздела 8.5.1 можно заменить критериями, разработанными с использованием функций потерь, отличных от $KLIC$. Различные меры, соответствующие различным функциям потерь, представлены у Амэмия (1980). Эти меры часто предназначены для вложенных моделей, но также могут быть применимы и для невложенных моделей.

Простой подход заключается в сравнении способности прогнозирования, т.е. выбирает модель с наименьшим значением среднеквадратической ошибки $(N - q)^{-1}\sum_i (y_i - \hat{y}_i)^2$. Для линейной регрессии это эквивалентно выбору модели с наиболее высоким скорректированным $R^2$, который обычно рассматривается как ставящий слишком маленький штраф за сложность модели. Модификация для непараметрической регрессии --- кросс-валидация с исключением отдельных наблюдений (см. раздел 9.5.3).

Формальные тесты для выбора между невложенными моделями в случае, отличном от метода максимального правдоподобия, часто используют один из двух подходов. Искусственное вложение, предложенное Дэвидсоном и МакКинноном (1984), подразумевает вложение двух невложенных моделей в более общую искусственную модель и приводит к так называемым $J$ тестам, $P$ тестам и связанным с ними тестам. Принцип всеобщности, предложенный Мизоном и Ричардом (1986), приводит к общим условиям для тестирования одной модели против альтернативной невложенной модели. Уайт (1994) связывает этот подход с тестами на условные моменты. Обзор этих  источников можно почитать у Дэвидсона и МакКиннона (1993, глава 11).

\subsection{Пример невложенных моделей}

Сгенерируем выборку из 100 наблюдений с помощью модели Пуассона с математическим ожиданием $\E[y|x] = \exp(\beta_1 + \beta_2x_2 + \beta_3x_3)$, где $x_2, x_3 \sim \mathcal{N}[0,1]$ и $(\beta_1, \beta_2, \beta_3) = (0.5, 0.5, 0.5).$

\begin{table}[h]
\begin{center}
\caption{\label{tab:nonnestpoiss} Сравнение невложенных моделей для примера регрессии Пуассона}
\begin{minipage}{17cm}
\begin{tabular}[t]{llll}
\hline
\hline
\bf{Тип теста}\footnote{$N = 100$. Модель 1 --- регрессия Пуассона $y$ на константу и $x_2$. Модель 2 --- регрессия Пуассона $y$ на константу, $x_3$ и $x_3^2$. Две последние строчки приведены для теста Вуонга на непересекающиеся модели (см. текст).} & \bf{Модель 1} & \bf{Модель 2} & \bf{Вывод} \\
\hline
$-2\ln L$ & 366.86 & 352.18 & Предпочитается модель 2 \\
$AIC$ & 370.86 & 358.18 & Предпочитается модель 2 \\
$BIC$ & 376.06 & 366.00 & Предпочитается модель 2 \\
$N\hat{w}^2$ & 7.84 c $p = 0.000$ &  & $T_{LR}$ тест применим \\
$T_{LR} = N^{-1/2}LR/\hat{w}$ & $-0.883$ c $p = 0.377$ &  & Ни одна модель не предпочитается \\
\hline
\hline
\end{tabular}
\end{minipage}
\end{center}
\end{table}

Зависимая переменная $y$ имеет выборочное среднее 1.92 и стандартное отклонение 1.84. Две неверные невложенные модели были оценены с помощью регрессии Пуассона:
\[
\text{Модель 1}: \hat{\E}[y|x] = \exp(\underset{(8.08)}{0.608} + (\underset{(4.03)}{0.291}x_2),
\]
\[
\text{Модель 2}: \hat{\E}[y|x] = \exp(\underset{(5.14)}{0.493} + (\underset{(5.10)}{0.359}x_3 + (\underset{(1.78)}{0.091}x_3^2),
\]
где $t$-статистики приведены в скобках.

В первых трёх строках таблицы 8.2 представлены различные информационные критерии, а также модель с наименьшим предпочтительным значением критерия. Первый не штрафует число параметров и даёт результат в пользу модели 2. Второй и третий способы, определённые в (8.40) и (8.41), накладывают больший штраф на модель 2, которая имеет дополнительный параметр, но всё же дают результат в пользу модели 2.

Последние две строки таблицы 8.2 суммируют информацию о тесте Вуонга на пересекающиеся модели.

Сначала необходимо проверить условие равенства функций плотности при оценке псевдо-истинных значений. Статистика $\hat{w}^2$ из (8.48) легко вычисляется, если даны выражения для функций плотности. Сложность заключается в вычислении оценки матрицы $W$ из (8.45). Для распределения Пуассона мы можем использовать $\hat{A}$ и $\hat{B}$, которые были определены в конце раздела 5.2.3, и $\hat{B}_{fg} = N^{-1}\sum_i (y_i - \hat{\mu}_{fi})x_{fi} \times (y_i - \hat{\mu}_{gi})x_{gi}$. Собственные значения $W$ равны $\lambda_1 = 0.29$, $\lambda_2 = 1.00$, $\lambda_3 = 1.06$, $\lambda_4 = 1.48$ и $\lambda_5 = 2.75$. $P$-значение для тестовой статистики $N\hat{w}^2$, распределение которой приведено в (8.44), --- доля случаев в выборке в которых $\sum_{j=1}^5 \lambda_j z_j^2$, превышает $N\hat{w}^2 = 69.14$. Можно сгенерировать, например, 10000 случайных наблюдений. Здесь $p = 0.000 < 0.05$. Отсюда можно заключить, что можно сравнивать модели. Критическое значение на уровне значимости 0.05 в данном случае равно 16.10. Это немного выше, чем $\chi_{0.05}^2 (5) = 11.07$.

Учитывая то, что возможно сравнение, можно применять второй тест. Здесь $T_{LR} = - 0.883$ выступает в пользу второй модели, так это значение отрицательно. Однако, стандартный нормальный двусторонний тест, на уровне значимости 5\% говорит, что разница не является статистически значимой. В этом примере значение $\hat{w}^2$  довольно велико, поэтому первая тестовая статистика $N\hat{w}^2$ имеет большое значение, а вторая тестовая статистика $N^{-1/2}LR(\hat{\theta},\hat{\gamma})$ имеет маленькое значение.

\section{Последствия проверки гипотез}

На практике проводят несколько тестов, прежде чем выбирают предпочтительную модель. Это приводит к ряду осложнений, которые практики обычно игнорируют.

\subsection{Предварительное оценивание перед проведением тестов}

Использование тестов на спецификацию при выборе модели усложняет распределение оценки. Предположим, что мы выбираем между двумя оценками $\hat{\theta}$ и $\tilde{\theta}$ с помощью статистического теста на уровне значимости 5\%. Например, $\hat{\theta}$ и $\tilde{\theta}$ могут быть оценками для ограниченной и неограниченной моделей соответственно. 

Тогда фактическая оценка имеет вид $\theta^+ = w\hat{\theta} + (1 - w)\tilde{\theta}$, где случайная величина $w$  принимает значение 1, если тест выступает в пользу $\hat{\theta}$, и 0, если тест выступает в пользу $\tilde{\theta}$. То есть оценка зависит от оценок ограниченной и неограниченной моделей и от случайной величины $w$, которая, в свою очередь, зависит от уровня значимости теста. Следовательно $\theta^+$ --- оценка со сложными свойствами. Данная оценка называется оценкой с предварительным тестированием, или претест-оценкой (pretest estimator). Распределение $\theta^+$ было получено для модели линейной регрессии при условии нормального распределения остатков, и оно не является стандартным.

В теории статистические выводы должны быть основаны на распределении $\theta^*$. На практике вывод основывается на распределении $\hat{\theta}$, если $w = 1$, или $\tilde{\theta}$, если $w = 0$, при этом игнорируется наличие случайности в  $w$. Это делается для простоты, так как даже в простейших моделях распределение оценки становится крайне сложным, когда проводится несколько таких тестов.

\subsection{Порядок тестирования}

Можно сделать различные выводы в зависимости от порядка, в котором проводятся тесты.

Один из возможных порядков ---  от общих к частным моделям. Например, можно оценить общую модель спроса перед тестированием ограничений из теории потребительского спроса таких, как однородность и симметрия. Или  можно идти от частной к общей модели с добавлением регрессоров по мере необходимости и дополнительных обобщений таких, как учёт  эндогенности, если она присутствует. Эти два упорядочения являются естественными при выборе регрессоров для включения в модель, но когда проводятся тесты на спецификацию, часто используются оба способа в одном и том же исследовании.

Связанный с этим вопрос заключается в выборе между совместными и отдельными тестами. Например, значимость двух регрессоров может быть проверена либо с помощью двух отдельных $t$-тестов на значимость, либо с помощью совместного $F$-теста, или $\chi^2(2)$ теста на значимость. Общее описание было представлено в разделе 7.2.7. Пример приведён ниже в разделе 18.7.

\subsection{Интеллектуальный анализ данных}

Доведённое до крайности широкое использование тестов для выбора модели было названо добычей данных или интеллектуальным анализом данных (Ловелл, 1983). Например, можно искать среди нескольких сотен показателей, которые, возможно, предсказывают $y$, и выбрать только те, которые являются значимыми на уровне значимости 5\% при проведении двустороннего теста. Существуют компьютерные программы, которые автоматизируют такой процесс поиска, и они часто используются в некоторых отраслях прикладной статистики. К сожалению, такой широкий поиск приводит к обнаружению ложных соотношений, так как тест с уровнем значимости 0.05 приводит к ошибочному обнаружению зависимости в 5\% случаев. Ловелл отметил, что применение такой методологии, как правило, ведёт к завышению параметров качества подгонки (например, $R^2$) и к недооценке выборочных дисперсий коэффициентов регрессии, даже если она успешно определит переменные, которые отражают особенности процесса, порождающего данные. Использование стандартных тестов и приведение $p$-значений без учёта процедуры поиска модели может ввести в заблуждение, поскольку номинальные и истинные $p$-значения не являются одинаковыми. Уайт (2001б) и Салливан, Тиммерманн и Уайт (2001) показывают, как использовать метод бутстрэп, чтобы рассчитать истинную статистическую значимость регрессоров. Об этом также можно посмотреть у П. Хансена (2003).

Мотивацией для интеллектуального анализа данных иногда может быть сохранение степеней свободы или избежание сверхпараметризации. Что ещё более важно, многие аспекты спецификации, например, функциональная форма зависимости, остаются нерешёнными лежащей в основе теорией. Учитывая возможность ошибки спецификации,  существуют аргументы в пользу поиска спецификации (Сарган, 2001). Тем не менее, необходимо быть внимательным, особенно когда анализируются малые выборки и количество шагов при поиске спецификации велико по отношению размера выборки. Когда процесс поиска спецификации является пошаговым с большим числом шагов и каждый новый шаг зависит от результатов предыдущего теста, статистические свойства процедуры в целом являются сложными и их нельзя получить аналитически.

\subsection{Практический подход}

Прикладные микроэконометрические исследования обычно минимизирует проблему претест оценивания путём разумного использования проверки гипотез. Экономическая теория используется при выборе регрессоров, чтобы значительно уменьшить число потенциальных регрессоров. Если выборка имеет большой размер, то не имеет смысла исключать <<незначимые>> переменные. Финальная модель часто включает статистически незначимые регрессоры такие, как дамми на регион и отрасль  или профессию в регрессии дохода. Замусоривания отчёта можно избежать, не включив в него незначимые коэффициенты, но обратив внимание на сам факт их наличия.  Включение большого количества переменных может привести к некоторой потере точности оценки ключевых регрессоров таких, как длительность школьного образования в регрессии дохода, но защищает от смещения, вызванного ошибочным исключением переменных, которые должны быть включены.

Хорошей практикой является использование части выборки (<<обучающей выборки>>) для поиска спецификации и выбора модели, а затем сообщить о результатах оценивания выбранной модели, используя другую части выборки (<<тестовую выборка>>). В таком случае предварительное тестирование не влияет на распределение оценки, если подвыборки являются независимыми. Эта процедура, как правило, применяется, только когда размер выборки очень большой, потому что использование неполной выборки в конечном оценивании приводит к потере в точности оценки.

\section{Диагностика модели}

В этом разделе мы рассмотрим показатели качества подгонки и варианты определения остатков в нелинейных моделях. Полезные являются те показатели, который позволяют выявить недостатки модели.

\subsection{Псевдо-$R^2$ показатели}

Качество подгонки интерпретируется как близость оценённых значений к выборочным значениям зависимой переменной. 

Для линейных моделей с $K$ регрессорами наиболее прямой мерой является стандартная ошибка регрессии, которая является оценённым значением стандартного отклонения случайной ошибки,

\[
s = \left[ \frac{1}{N - K} \sum_{i=1}^N (y_i - \hat{y}_i)^2 \right]^{1/2}.
\]

Например, стандартная ошибка, равная 0.10, в регрессии логарифма дохода означает, что примерно 95\% от оценённых значений находятся в пределах 0.20 от фактического значения логарифма дохода, или в пределах 22\% от фактического дохода т.к. $e^{0.2} \simeq 1.22$. Этот показатель совпадает с выборочной среднеквадратическая ошибка, $\hat{y}_i$ рассматривается как прогноз $y_i$, только число степеней свободы скорректировано. В качестве альтернативы можно использовать среднюю абсолютную ошибку $(N - K)^{-1}\sum_i |y_i - \hat{y}_i|$. Можно использовать те же самые показатели для нелинейных регрессионных моделей при условии, что нелинейные модели дают прогнозные значения зависимой переменной $\hat{y}_i$.

Похожий показатель в линейных моделях --- $R^2$, коэффициент множественной детерминации. Он равен доле выборочной дисперсии зависимой переменной, которая объяснена регрессорами. Величина $R^2$ рассчитывается чаще, чем $s$, хотя $s$ может быть более информативной для оценки качества подгонки.

Псевдо-$R^2$ является обобщением $R^2$ для нелинейной регрессионной модели. Есть несколько интерпретаций $R^2$ в линейной модели. Они приводят к нескольким возможным псевдо-$R^2$ показателям, которые отличаются в нелинейных моделях и необязательно  лежат между нулём и единицей или увеличиваются при добавлении регрессоров. Мы представляем некоторые из этих показателей и  для простоты не корректируем их на степени свободы.

Один из подходов к $R^2$ основан на  разложении общей суммы квадратов $(TSS)$, 
\[
\sum_{i} (y_i - \bar{y})^2 = \sum_{i} (y_i - \hat{y}_i)^2 + \sum_{i} (\hat{y}_i - \bar{y})^2 + 2\sum_{i} (y_i - \hat{y}_i)(\hat{y}_i - \bar{y})
\]

Первая сумма в правой части --- сумма квадратов остатков $(RSS)$, а вторая --- объяснённая сумма квадратов $(ESS)$. Это приводит к двум возможным показателям:
\[
R_{RES}^2 = 1 - RSS/TSS
\]
\[
R_{EXP}^2 = ESS/TSS
\]

Для МНК регрессии в линейной модели с константой третья сумма равна нулю, поэтому $R_{RES}^2 = R_{EXP}^2$. Однако это упрощение невозможно в других моделях, и в общем случае $R_{RES}^2 \not= R_{EXP}^2$ в нелинейных моделях. Показатель $R_{RES}^2$ может быть меньше нуля, $R_{EXP}^2$ может превышать единицу, и оба показателя могут уменьшаться при добавлении регрессоров, хотя $R_{RES}^2$ увеличится для НМНК регрессии нелинейной модели, так как тогда оценка минимизирует $RSS$.

Другой показатель равен
\[
R_{COR}^2 = \widehat{\Cor}^2[y_i, \hat{y}_i],
\]
квадрату выборочной корреляции между фактическими и оценёнными значения. Показатель $R_{COR}^2$ лежит между нулём и единицей, и он равен $R^2$ в $OLS$ регрессии для линейной модели с константой. В нелинейных моделях $R_{COR}^2$ может уменьшиться при добавлении регрессоров. 

Третий подход использует взвешенные суммы квадратов, которые учитывают внутреннюю гетероскедастичность пространственных данных. Пусть $\hat{\sigma}_i^2$ --- оценка условной дисперсии $y_i$, где предполагается, что гетероскедастичность явно смоделирована как в случае с ДОМНК или для таких моделей, как логит и модель Пуассона. Тогда мы можем использовать
\[
R_{WSS}^2 = 1 - WRSS/WTSS,
\]
где взвешенные сумма квадратов остатков $WRSS = \sum_i (y_i - \hat{y}_i)^2/\hat{\sigma}_i^2$, $WTSS = \sum_i (y_i - \hat{\mu})^2/\hat{\sigma}^2$, $\hat{\mu}$ и $\hat{\sigma}^2$ --- оценка математического ожидания и оценка дисперсии в модели только с константой. Данный показатель можно назвать $R^2$ Пирсона, потому что $WRSS$ равно статистике Пирсона, которая без поправок на конечную выборку должна быть равна $N$, если гетероскедастичность правильно смоделирована. Следует отметить, что $R_{WSS}^2$ может быть меньше нуля, и оно уменьшается при добавлении регрессоров.

Четвёртый подход является обобщением $R^2$ до целевых функций помимо суммы квадратов остатков. Пусть $Q_N(\theta)$ обозначает целевую функцию, которая максимизируется. $Q_0$ --- её значение в модели только с константой, $Q_{fit}$ --- значение в оценённой модели, а $Q_{max}$ --- наибольшее возможное значение $Q_N(\theta)$. Тогда максимально возможный прирост целевой функции, который получается в результате включения регрессоров, --- $Q_{max} - Q_0$, а фактический прирост --- $Q_{fit} - Q_0$. Тогда показатель выглядит так:
\[
R_{RG}^2 = \frac{Q_{fit} - Q_0}{Q_{max} - Q_0} = 1 - \frac{Q_{max} - Q_{fit}}{Q_{max} - Q_0},
\]
где индекс $RG$ внизу означает относительный прирост (relative gain). Для оценивания методом МНК максимум функции потерь ---  минус сумма квадратов остатков. Тогда $Q_0 = - TSS$, $Q_{fit} = - RSS$ и $Q_{max} = 0$. Тогда $R_{RG}^2 = ESS/TSS$ для МНК или НМНК регрессии. Показатель $R_{RG}^2$ имеет то преимущество, что он лежит между нулём и единицей и он увеличивается при добавлении регрессоров. Для оценивания методом максимального правдоподобия функция потерь имеет вид $Q_N(\theta) = \ln L_N(\theta)$. Тогда $R_{RG}^2$ не может быть всегда применён, так как в некоторых моделях может и не быть границы $Q_{max}$. Например, для линейной модели с нормальными остатками $L_N(\beta, \sigma^2) \rightarrow \infty$ при $\sigma^ 2 \rightarrow 0$. Для метода максимального правдоподобия и метода квази-максимального правдоподобия при оценивание линейных моделей из экспоненциального семейства, например, логит или модели Пуассона, обычно известно значение $Q_{max}$. Можно показать, что $R_{RG}^2$ --- это $R^2$, основанный на остатках отклонений (deviance residuals), которые определены в следующем разделе.

Связанный с $R_{RG}^2$ показатель --- $R_{Q}^2 = 1 - Q_{fit}/Q_0$. Этот показатель увеличивается при добавлении регрессоров. Он равен $R_{RG}^2$, если $Q_{max} = 0$, что выполняется в случае МНК-регрессии, а также для бинарных и полиномиальных моделей. В противном случае для дискретных данных этот показатель может иметь верхнюю границу, которая меньше единицы, тогда как для непрерывных данных показатель не может быть ограничен нулём и единицей, так как логарифм функции правдоподобия может быть положительным или отрицательным. Например, для оценивания методом максимального правдоподобия с непрерывной функцией плотности возможно, что $Q_0 = 1$ и $Q_{fit} = 4$, что приводит к $R_Q^2 = -3$, или, что $Q_0 = - 1$ и $Q_{fit} = 4$, что приводит к $R_Q^2 = 5$.

Для нелинейных моделей, следовательно, не существует универсального псевдо-$R^2$. Самыми полезными показателями могут быть $R_{COR}^2$, так как коэффициенты корреляции легко интерпретируются, и $R_{RG}^2$ в особых случаях, когда $Q_{max}$ известно. Кэмерон и Виндмейер (1997) анализируют многие показатели, а Кэмерон и Виндмейер (1996) применяют эти показатели на реальных данных.

\subsection{Анализ остатков}

Микроэконометрический анализ фактически уделяет мало внимания анализу остатков по сравнению с некоторыми другими областями статистики. Если наборы данных малы, есть опасение, что анализ остатков может привести к модели с излишней подгонкой. Если набор данных большой, то существует мнение, что анализ остатков может оказаться ненужным, так как одно наблюдение мало повлияет на анализ. Поэтому мы приведём краткое описание. Более обширное описание приведено, например, в работе МакКуллаха и Нельдера (1989), а также Кэмерона и Триведи (1998, глава 5). Эконометристы много внимания уделяли способам  определения остатков в цензурированных и усеченных моделях.

Много вариантов остатков было предложены для нелинейной регрессионной модели. Рассмотрим скалярную зависимую переменную $y_i$ с оценёнными значениями $\hat{y}_i = \hat{\mu}_i = \mu(x_i, \hat{\theta})$. Обычные остатки --- это $r_i = y_i - \hat{\mu}_i$. Остатки Пирсона являются очевидной поправкой на гетероскедастичность --- $p_i = (y_i - \hat{\mu}_i)/\hat{\sigma}_i$, где $\hat{\sigma}_i$ является оценкой условной дисперсии $y_i$. Это требует  спецификации дисперсии для $y_i$, что сделано, например, в рамках модели Пуассона. Для распределения из экспоненциального семейства (см. раздел 5.7.3) определяют остатки отклонений --- это $d_i = \sign(y_i - \hat{\mu}_i)\sqrt{2[l(y_i) - l(\hat{\mu}_i)]}$, где $l(y)$ обозначает логарифм функции плотности $y|\mu$, оценённый в точке $\mu = y$, а $l(\hat{\mu})$ обозначает оценивание в точке $\mu = \hat{\mu}$. Логика остатков отклонений состоит в том, что сумма квадратов этих остатков является статистикой отклонения (deviance statistic) --- обобщением для моделей экспоненицального семейства суммы квадратов обычных остатков в линейной модели. Остатки Анскомба определяются как преобразование $y$, при котором получается распределение наиболее близкое к нормальному, а затем проводится масштабирование к нулевому математическому ожиданию и единичной дисперсии. Это преобразование было получено для распределений из экспоненциального семейства.

Поправки остатков для малых выборок были предложены для учёта ошибки оценивания в $\hat{\mu}_i$. В линейной модели используется деление остатков на $\sqrt{1 - h_{ii}}$, где $h_{ii}$ --- диагональный элемент матрицы $H = X(X'X)^{-1}X$. Считается, что эти остатки показывают более хорошие результаты на конечных выборках. Так как матрица $H$ имеет ранг $K$, равный числу регрессоров, среднее значение $h_{ii}$ равно $K/N$, а значения $h_{ii}$, превышающие $2K/N$, рассматриваются как имеющие большой рычаг. Эти результаты распространяются на модели экспоненциального семейства с $H = W^{1/2}X(X'WX)^{-1}XW^{1/2}$, где $W = \Diag[w_{ii}]$ и $w_{ii} = g'(x_i'\beta)/\sigma_i^2$ с $g(x_i'\beta)$ и $\sigma_i^2$ --- заданное условное математическое ожидание и заданная дисперсия соответственно. Маккуллаг и Нельдер (1989) приводят описание данного обобщения.

Кокс и Снелл (1968) определяют обобщённые остатки как любую скалярную функцию $r_i = r(y_i, x_i, \theta)$, которая удовлетворяет некоторым относительно слабым условиям. Причина, из-за которой такие остатки возникают, состоит в том, что многие оценки имеют условие первого порядка вида $\sum_i g(x_i, \theta)r(y_i, x_i, \hat{\theta}) = 0$, где $y_i$ встречается в скаляре $r(\cdot)$, но не в векторе $g(\cdot)$. Про этот сюжет можно также прочитать у Уайта (1994).

Для регрессионных моделей, основанных на нормально распределенной скрытой переменной, (см. главы 14 и 16), Чешер и Айриш (1987) предлагают использовать $\E[\e_i^*|y_i^*]$ в качестве остатокв, где $y_i^* = \mu_i + \e_i^*$ --- ненаблюдаемая скрытая переменная, и $y_i = g(y_i^*)$ --- наблюдаемая зависимая переменная. Частные случаи $g(\cdot)$ соответствуют пробит и тобит моделям. Гурьеру и другие (1987) обобщают этот подход для распределений из экспоненциального семейства. Естественный подход здесь заключается в рассмотрении  остатков как пропущенной переменной, по аналогии с  алгоритмом максимизации ожидания из раздела 10.3.

Обычно изображают графики, на которых по одной оси изображена интересующая переменная, а по другой --- остатки. Например, график остатков против оценённых значений может выявить плохую подгонку модели; график остатков против невключенных переменных может мотивировать включение дополнительных регрессоров в модель, а график остатков против включённых регрессоров может указать на то, что необходимо изменить функциональную форму. Может быть полезно включить линию непараметрической регрессии в таких графиках (см. главу 9). Если данные принимают всего несколько дискретных значений, то может быть трудно интерпретировать графики из-за скученности данных в нескольких точках. Поэтому может быть полезно добавить небольшой искусственный случайный шум к данным, чтобы уменьшить скученность точек на графике.

Некоторые параметрические модели подразумевают, что  остатки должны иметь нормальное распределение. Это можно проверить с помощью графика квантилей, который сортирует остатки $r_i$ от меньшего к большему и отображает на графике значения остатков против предсказанных значений остатков, в предположении, что остатки имеют именно нормальное распределение. Таким образом, график отсортированных $r_i$ против $\bar{r} + s_r\Phi^{-1}((i - 0.5)/N)$, где $\bar{r}$ и $s_r$ --- выборочное среднее значение и стандартное отклонение $r$ и $\Phi^{-1}(\cdot)$ --- обратное значение функции распределения стандартного нормального распределения.

\subsection{Пример диагностики}

Таблица 8.3 использует тот же самый процесс, порождающий данные, что и в разделе 8.5.5. Зависимая переменная $y$ имеет выборочное среднее, равное 1.92, и стандартное отклонение, равное 1.84. Регрессия Пуассона $y$ на $x_3$ и $y$ на $x_3$ и $x_3^2$ даёт
\[
\text{Модель 1}: \hspace{0.5cm} \hat{\E}[y|x] = \exp(\underset{(5.20)}{0.586} + \underset{(7.60)}{0.389}x_3)
\]
\[
\text{Модель 2}: \hspace{0.5cm} \hat{\E}[y|x] = \exp(\underset{(5.14)}{0.493} + \underset{(5.10)}{0.359}x_3 + \underset{(1.78)}{0.091}x_3^2),
\]
где $t$-статистики приведены в скобках.

В этом примере все показатели $R^2$ увеличиваются с добавлением $x_3^2$ в качестве регрессора, при этом прибавка отличается довольно сильно, несмотря на то, что все кроме последнего $R^2$ изначально имеют близкие значения. В более общем случае первые три $R^2$ увеличиваются в похожих пропорциях, а $R_{RES}^2$ и $R_{COR}^2$ могут быть довольно близки, но остальные три показателя меняются довольно по-разному. Только два последние показателя $R^2$ гарантированно увеличатся при добавлении регрессоров в случае, когда целевая функция не является суммой квадратов ошибок. Показатель $R_{RG}^2$ может быть посчитан в данном случае, т.к.  логарифма функции правдоподобия для регрессии Пуассона достигает максимума, когда оценённое среднее равно $\hat{\mu}_i = y_i$ для всех $i$, что приводит к $Q_{max} = \sum_i [y_i\ln y_i - y_i - \ln y_i!]$, где $y\ln y = 0$ при $y = 0$.

Кроме того, три вида остатков были рассчитаны для второй модели. Выборочное среднее и стандартное отклонение остатков составили соответственно 0 и 1.65 для обычных остатков, 0.01 и 1.97 для остатков Пирсона, $- 0.21$ и 1.22 для остатков отклонений. Нулевое среднее для обычных остатков является свойством регрессии Пуассона с константой и присутствует в небольшом числе других моделей. Большое стандартное отклонение обычных остатков отражает отсутствие масштабирования и тот факт, что здесь стандартное отклонение $y$ превышает 1. Все попарные корреляции между этими остатками превышают 0.96. Это часто происходит, когда $R^2$ маленький, так что  $\hat{y}_i \simeq \bar{y}$.


\begin{table}[h]
\begin{center}
\caption{\label{tab:pseudor2} Псевдо-$R^2$: пример регрессии Пуассона}
\begin{minipage}{13.5cm}
\begin{tabular}[t]{l*{4}{{c}}}
\hline
\hline
\bf{Тип теста}\footnote{$N = 100$. Модель 1 --- регрессия Пуассона $y$ на константу и $x_3$. Модель 2 --- регрессия Пуассона $y$ на константу, $x_3$ и $x_3^2$. $RSS$ --- сумма квадратов остатков $(SS)$, $ESS$ --- объяснённая сумма квадратов $SS$, $TSS$ --- общая сумма квадратов, $WRSS$ --- взвешенный $RSS$, $WTSS$ --- взвешенный $TSS$, $Q_{fit}$ --- оценённое значение целевой функции, $Q_0$ --- оценённое значения в модели только с константой, и $Q_{max}$ --- максимальное возможное значение целевой функции при имеющихся данных, и это значение существует лишь для некоторых целевых функций.} & \bf{Модель 1} & \bf{Модель 2} & \bf{Разница} \\
\hline
$s, \text{где} \hspace{0.2cm} s^2 = RSS/(N - K)$ & 0.1662 & 0.1661 & 0.0001 \\
$R_{RES}^2 = 1 - RSS/TSS$ & 0.1885 & 0.1962 & + 0.0077 \\
$R_{EXP}^2 = ESS/TSS$ & 0.1667 & 0.2087 & + 0.0402 \\
$R_{COR}^2 = \widehat{\Cor}^2[y_i, \hat{y}_i]$ & 0.1893 & 0.1964 & + 0.0067 \\
$R_{WSS}^2 = 1 - WRSS/WTSS$ & 0.1562 & 0.1695 & + 0.0233 \\
$R_{RG}^2 = (Q_{fit} - Q_0)/(Q_{max} - Q_0)$ & 0.1552 & 0.1712 & + 0.0160 \\
$R_Q^2 = 1 - Q_{fit}/Q_0$ & 0.0733 & 0.0808 & + 0.0075 \\
\hline
\hline
\end{tabular}
\end{minipage}
\end{center}
\end{table}

\section{Практические соображения}

М-тесты и тесты Хаусмана легко провести с помощью вспомогательных регрессий. Следует помнить, что эти вспомогательные регрессии имеют смысл только при предположениях о распределении, которые сильнее, чем те, которые были сделаны для получения обычных скорректированных стандартных ошибок коэффициентов регрессии. Некоторые тесты с поправками были представлены в разделе 8.4.

С достаточно большим набором данных и при фиксированном уровне значимости таким, как 5\%, выборочные моментные условия, накладываемые моделью, будут отвергнуты, за исключением нереалистичного случая, когда все аспекты функциональной формы модели, регрессоры и распределение верно специфицированы. В классической проверке гипотез часто это и является желаемым результатом. В частности, при достаточно большой выборке коэффициенты регрессии всегда будет существенно отличаться от нуля, и цель многих исследований --- добиться такого результата. Тем не менее, для тестов на спецификацию желаемый результат состоит в том, чтобы не отвергать нулевую гипотезу, потому что тогда можно утверждать, что модель верно специфицирована. Возможно, именно по этой причине нечасто проводят тесты на спецификацию.

В качестве примера рассмотрим тесты на верную спецификацию моделей жизненного цикла потребления. Кроме случая, когда выборки малые, исследователь скорее всего будет отвергать модель на уровне значимости 5\%. Например, предположим, что статистика теста на спецификацию модели имеет $\chi^2(12)$ распределение, когда он проводится на выборке с $N = 3000$ и имеет $p$-значение 0.02. Неясно, действительно ли модель жизненного цикла  плохо описывает данные, хотя эта гипотеза формально будет отвергнута на уровне значимости 5\%. Одной из возможностей является увеличение критического значения с увеличением размера выборки, как при  использовании  $BIC$ (см. раздел 8.5.1).

Другая причина неширокого использования тестов на спецификацию --- трудность вычислений и большая вероятность ошибки первого рода, если используются более удобные вспомогательные регрессии при построении асимптотически эквивалентной версии теста. Эти недостатки могут быть значительно снижены при использовании метода бутстрэп. Глава 11 представляет метод бутстрэп для проведения многих тестов, приведённых в этой главе.

\section{Библиографические заметки}

\begin{itemize}
\item [$8.2$] Тест на условный момент, предложенный Ньюи (1985) и Таушеном (1985), является обобщением информационно матричного теста Уайта (1982). Для оценивания с помощью метода максимального правдоподобия расчёт М-теста для вспомогательной регрессии обобщает методы Ланкастера (1984) и Чешера (1984) для информационно матричного теста. Хороший обзор М-тестов представлен у Пагана и Велла (1989). М-тест определяет общий подход к тестированию. Частными случаями М-теста являются  все тесты Вальда, тесты множителей Лагранжа, тесты отношения правдоподобия и тесты Хаусмана. Эта унификация представлена в работе Уайта (1994).
\item [$8.3$] Тест Хаусмана, предложенный Хаусманом (1978), был введён ранее в разделе 8.3, а также хорошее описание было сделано Руудом (1984).
\item [$8.4$] В эконометрических работах Грина (2003), Дэвидсона и Маккинона (1993), а также Вулдриджа (2002) описаны стандартные тесты на спецификацию. 
\item [$8.5$] Песаран и Песаран (1993) рассматривают в своей работе, каким образом можно применить тест Кокса на невложенные модели (1961, 1962б), когда нельзя получить аналитическое выражение для математического ожидания логарифма функции правдоподобия. В таком случае можно применять тест Вуонга (1989).
\item [$8.7$] Диагностика нелинейных моделей часто проводится с помощью расширения результатов для линейной регрессионной модели до обобщённой линейной модели такой, как, например, логит или модель Пуассона. Детальное обсуждение со ссылками на литературу представлено в работе Кэмерона и Триведи (1998, глава 5).
\end{itemize}


\section{Упражнения}

\begin{enumerate}
\item [$8 - 1$] 

Предположим, что $y = x'\beta + u$, где $u \sim \mathcal{N}[0,\sigma^2]$ с вектором параметром $\theta = [\beta', \sigma^2]'$ и функцией плотности $f(y|\theta) = (1/\sqrt{2\pi}\sigma)\exp[-(y - x'\beta)^2/2\sigma^2].$  У нас есть выборка из $N$ независимых наблюдений.
\begin{enumerate}
\item Объясните, почему тест на моментное условие $\E[x(y - x'\beta)^3]$ является тестом на проверку предположения о нормальном распределении ошибок.
\item Приведите выражения для $\hat{\mu}_i$ и $\hat{s}_i$, заданных в (8.5). Они необходимы для проведения М-теста, основанного на моментном условии из пункта (а).
\item Пусть $\dim[x] = 10$, $N = 100$, и вспомогательная регрессия из (8.5) даёт нецентрированный $R^2$, равный 0.2. Какой вывод Вы сделаете на уровне значимости 0.05?
\item Для этого примера приведите моментные условия, которые проверяются с помощью информационно матричного теста Уайта.
\end{enumerate}
\item [$8 - 2$] Рассмотрите мультиномиальную версию $PCGF$ теста из (8.23) c $p_j$, заменённым на $\hat{p}_j = N^{-1}\sum_i F_j(x_i, \hat{\theta})$. Покажите, что $PCGF$ можно выразить как $CGF$ из (8.27)  с $V = \Diag[N\hat{p}_j]$. [Сделайте вывод, что в мультиномиальном случае тестовую статистику Андриуса можно упростить до статистики Пирсона].
\item [$8 - 3$] (Переработанный вариант из Амэмия, 1985). Для теста Хаусмана, описанного в разделе 8.4.1, пусть $V_{11} = \V[\hat{\theta}]$, $V_{22} = \V[\tilde{\theta}]$ и $V_{12} = \Cov[\hat{\theta},\tilde{\theta}]$.
\begin{enumerate}
\item Покажите, что оценка $\bar{\theta} = \hat{\theta} + [V_{11} + V_{22} - 2V_{12}]^{-1}(\tilde{\theta},\hat{\theta})$ имеет асимптотическую ковариационную матрицу $\V[\bar{\theta}] = V_{11} - [V_{11} - V_{12}][V_{11} + V_{22} - 2V_{12}]^{-1}[V_{11} - V_{12}]$.
\item Покажите, что $\V[\bar{\theta}]$ меньше, чем $\V[\hat{\theta}]$ в матричном смысле кроме случая, когда $\Cov[\hat{\theta},\tilde{\theta}] = \V[\hat{\theta}]$.
\item Теперь предположим, что $\hat{\theta}$ полностью эффективна. Может ли $\V[\bar{\theta}]$ быть меньше $\V[\hat{\theta}]$? Каков Ваш вывод?
\end{enumerate}
\item [$8 - 4$] Предположим, что две модели являются невложенными и что есть $N = 200$ наблюдений. Для модели 1 количество параметров $q = 10$ и $\ln L = - 400$. Для модели 3 количество параметров $q = 10$ и $\ln L = - 380$.
\begin{enumerate}
\item Какая модель будет предпочитаться при использовании $AIC$?
\item Какая модель будет предпочитаться при использовании $BIC$?
\item Какая модель будет предпочитаться, если модели были на самом деле вложенными, а мы использовали тест отношения правдоподобия на уровне значимости 0.05?
\end{enumerate}
\item [$8 - 5$] Используйте данные о расходах на здравоохранение из раздела 16.6. Модель --- пробит-регрессия $DMED$, переменной-индикатора для положительных расходов на здравоохранение, на 17 регрессоров, которые были перечислены во втором параграфе раздела 16.6. Необходимо получить оценки, которые приведены в первой колонке таблицы 16.1.
\begin{enumerate}
\item Проведите совместный тест на статистическую значимость субъективных индикаторов  здоровья $HLTHG$, $HLTHF$ и $HLTHP$ на уровне значимости 0.05, используя тест Хаусмана. [Чтобы ответить на этот вопрос может потребоваться дополнительный код, который зависит от используемого программного обеспечения]
\item Является ли тест Хаусмана лучшим в данном случае?
\item Отвергает ли информационно матричный тест на уровне значимости 0.05 ограничения этой модели? [Чтобы ответить на этот вопрос может потребоваться дополнительный код]
\item Сравните модель, которая не учитывает $HLTHG$, $HLTHF$ и $HLTHP$, с моделью, которая не учитывает $LC$, $IDP$ и $LPI$ на основе $R_{RES}^2$, $R_{EXP}^2$, $R_{COR}^2$ и $R_{RG}^2$.
\end{enumerate}
\end{enumerate}




\chapter{Полупараметрические методы}
\section{Введение}

В этой главе мы рассматриваем методы для анализа данных, которые требуют меньше от спецификации модели, чем методы, описанные в предыдущих главах.

Начнём с непараметрического оценивания. Для него требуются минимальные предположения относительно процесса, порождающего данные. Одним из наиболее ярких примеров является оценивание непрерывной функции плотности с помощью ядерной оценки. 
Привлекательность этого способа состоит в том, что он позволяет получить более гладкую оценку, чем гистограмма. Вторым ярким примером является непараметрическая регрессия на скалярный регрессор, например, ядерная регрессия. Она накладывает гибкую кривую на диаграмму рассеяния в координатах $(x,y)$. В данном случае нет никаких параметрических ограничений на форму кривой. Непараметрические оценки имеют множество применений, включая описание исходных данных, анализ данных и оценённых остатков из регрессионной модели, а также описание оценок параметров, полученных с помощью метода Монте-Карло.

Эконометрический анализ уделяет особое внимание многомерной регрессии скалярного $y$ на многомерный регрессор $x$. Теоретически возможно применять непараметрические методы для множественной регрессии на очень больших выборках, но на практике это затруднительно, поскольку данные должны быть разделены по нескольким измерениям, что приводит к слишком малому количеству точек в каждой группе.

По этой причине эконометристы уделяли больше внимания полупараметрическим методам. Они сочетают параметрическую компоненту, которая значительно уменьшает размерность, и непараметрическую компоненту. Одним из важных применений этих методов является то, что они позволяют получать более гибкие модели условного математического ожидания. Например, условное математическое ожидание $\E[y|x]$ может быть параметризовано в одноиндексную форму $g(x'\beta)$, где функциональная форма для $g(\cdot)$ не специфицирована. Вместо этого она непараметрически оценена вместе с неизвестными параметрами $\beta$. Другое важное применение позволяет ослабить предположения о распределении, которые при неверной спецификации приводят к получению несостоятельных оценок параметров. Например, мы хотим получить состоятельные оценки $\beta$ в линейной регрессионной модели $y = x'\beta + \e$, когда данные $y$ усечены или цензурированы (см. главу 16). Но в то же время мы не хотим указывать конкретное распределение $\e$.

Асимптотическая теория для непараметрических методов отличается от теории для более параметрических методов. Оценки получаются путём разделения данных на очень маленькие части при $N \rightarrow \infty$. Оценивание локального поведения происходит отдельно для каждой части. Так как при оценивании каждой части используются не все $N$ наблюдений, скорость сходимости меньше, чем в предыдущих главах. Тем не менее, в простейших случаях непараметрические оценки по-прежнему имеют асимптотически нормальное распределение. В некоторых основных случаях полупараметрической регрессии оценки параметров $\beta$ имеют стандартные свойства сходимости со скоростью $N^{-1/2}$, поэтому масштабирование на $\sqrt{N}$ приводит к получению предельного нормального распределения, в то время как непараметрическая компонента модели сходится с меньшей скоростью $N^{-r}$, где $r < 1/2$.

Так как непараметрические методы --- это методы локального усреднения, различный выбор локальности приводит к различным результатам на конечных выборках. В некоторых случаях при наличии ограничений существуют правила или методы для определения ширины окна, которые используются для локального усреднения. Эти правила аналогичны правилам для определения количества столбцов в гистограмме с учётом количества наблюдений. Кроме того, обычной практикой является использование ненаучных методов для выбора ширины окна. Эти методы предлагают использовать график, который на глаз выглядит достаточно гладким, но в то же время обнаруживает особенности интересующей зависимости.

Описание непараметрических методов занимает основную часть этой главы, потому что они представляют отдельный интерес и потому что они необходимы для полупараметрических методов, которые изложены в первую очередь в главах, посвящённых моделям с дискретными и ограниченным зависимыми переменными. Ядерным методам уделяется особое внимание, так как они относительно просты, для того чтобы их представлять. Как говорил Хэрдл (1990, с. xi): <<Считается, что все методы сглаживания в асимптотическом смысле по сути эквивалентны ядерному сглаживанию>>.

Раздел 9.2 содержит примеры непараметрического оценивания плотности и примеры непараметрических регрессий, которые проводятся на реальных данных. Ядерное оценивание плотности представлено в разделе 9.3. Локальные регрессии рассмотрены в разделе 9.4, чтобы отразить логику использования ядерной регрессии, которая рассмотрена в разделе 9.5. Раздел 9.6 содержит непараметрические регрессионные методы, отличные от ядерных методов. Далее в разделе 9.7 представлена тема полупараметрической регрессии.

\section{Непараметрический пример: почасовая заработная плата}

В качестве примера рассмотрим почасовую заработную плату и образование для 175 женщин в возрасте 36 лет, которые работали в 1993 году. Данные взяты из Мичиганского панельного опроса динамики доходов (Michigan Panel Survey of Income Dynamics). Легко обнаружить, что распределение почасовой заработной платы скошено вправо. По этой причине мы моделируем $\ln wage$ --- натуральный логарифм почасовой заработной платы.

Мы приведём всего один пример непараметрического оценивания плотности и пример одной непараметрической регрессии, а также проиллюстрируем важную роль выбора ширины окна. В разделах 9.3 --- 9.6 описана лежащая в основе теория.

\subsection{Оценивание функции плотности с помощью непараметрических методов}

Гистограмма натурального логарифма заработной платы приведена на графике 9.1. Уточним, что на гистограмме изображены 30 столбцов, ширина каждого столбца составляет примерно 0.20. Это необычно узкая ширина столбцов для 175 наблюдений, но многие детали не видны при большей ширине столбцов. Логарифмированные данные по заработной плате выглядят довольно симметричными, хотя они, возможно, немного скошены влево.

\vspace{5cm}

График 9.1: Гистограмма натурального логарифма почасовой заработной платы. Данные для 175 американских женщин в возрасте 36 лет, которые работали в 1993.

Стандартная сглаженная непараметрическая оценка плотности --- это ядерная оценка плотности, которая определена в (9.3). В данном случае мы используем ядро Епанечникова, которое определено в таблице 9.1.

Важным решением на практике является выбор ширины окна. Для этого примера оценка Сильвермана, определённая в (9.13), даёт $h = 0.545$. Тогда ядерная оценка является средневзешенным тех наблюдений, у которых значение логарифма заработной платы находится в пределах 0.21 единиц логарифма заработной платы в текущей точке оценивания. Наибольший вес имеют наблюдения  ближайшие к текущей точке оценивания. График 9.2 отражает три ядерные оценки плотности с шириной окна 0.273, 0.545 и 1.091 соответственно. Они соответствуют ширине окна, равной половине оценки, оценке Сильвермана и оценке, умноженной на два. Очевидно, что наименьшая ширина окна слишком мала, так как она приводит к слишком негладкой оценке плотности. Наибольшая ширина окна наоборот слишком сглаживает данные. Средняя ширина окна, равная 0.545, кажется наилучшим выбором, так как она даёт достаточно гладкую оценку плотности.

\vspace{5cm}

График 9.2: Ядерная оценка плотности для логарифма заработной платы для трёх различных вариантов ширины окна, используя ядро Епанечникова. Выбранная ширина окна равна $h = 0.545$. Используются те же данные, что и для графика 9.1.

Что можно сделать с этой ядерной оценкой плотности? Один вариант состоит в том, чтобы сравнить эту плотность с плотностью нормального распределения путём наложения плотности нормального распределения с математическим ожиданием, равным выборочному среднему, и дисперсией, равной выборочной дисперсии. Здесь не приведён график, но он показывает, что ядерная оценка плотности с предпочтительной шириной окна 0.545 является более остроконечной, чем для нормального распределения. Второй вариант заключается в том, чтобы сравнить ядерную оценку плотности логарифма заработной платой для различных подгрупп, например, по уровню образования или по полному или неполному рабочему дню.


\subsection{Непараметрическая регрессия}

Рассмотрим взаимосвязь между логарифмом заработной платы и образованием. Непараметрический метод, который здесь используется, --- локальный линейный регрессионный метод $LOWESS$, локальная средневзвешенная оценка (см. уравнение (9.16) и раздел 9.6.2).

Линия локально взвешенной регрессии в каждой точке $x$ получается с помощью центрированного подмножества наблюдений, которое включает  по умолчанию ближайшие $0.8N$ наблюдений. Здесь $N$ --- размер выборки, а веса снижаются по мере удаления от $x$. Для значений $x$, близких к краям, используется нецентрированное подмножество меньшего размера.

График 9.3 отражает диаграмму рассеяния логарифма заработной платы и образования, а также три регрессионные кривые $LOWESS$ для ширины окна, равной 0.8, 0.4 и 0.1. Первые два варианта дают аналогичные кривые. Зависимость кажется квадратичной, но это может быть иллюзией, так как данных для низких уровней образования довольно мало. Тех, кто учился в школе менее 10 лет, менее 10\% выборки. Для большей части данных линейная зависимость тоже может хорошо подходить. Для простоты мы не приводим 95\% доверительные интервалы, хотя это можно сделать.

\vspace{5cm}

График 9.3: Непараметрическая регрессия логарифма заработной платы на образование для трёх различных вариантов ширины окна, используя локально-линейную регрессию $LOWESS$. Используются те же данные, что и для графика 9.1.

\section{Ядерное оценивание плотности}

Непараметрические оценки плотности могут быть полезны для сравнения закона распределения между различными группами и для сравнения с эталонной плотностью такой, как плотность нормального распределения. По сравнению с гистограммой они имеют то преимущество, что они обеспеченивают более гладкую оценку плотности. Ключевое решение, аналогичное выбору количества столбцов на гистограмме, --- выбор ширины окна. Мы концентрируемся на стандартной непараметрической оценке плотности, ядерной оценке плотности. Мы представляем подробное описание, так как для оценки плотности можно более просто получить результаты, которые также релевантны для регрессии.


\subsection{Гистограмма}

Гистограмма --- это оценка плотности, которая получается с помощью разделения $x$ на одинаковые по размеру интервалы и расчёта доли выборки в каждом интервале.

Мы рассматриваем более формальное представление гистограммы, то, которое можно расширить до более гладкой ядерной оценки плотности. Рассмотрим оценивание плотности $f(x_0)$ скалярной непрерывной случайной величины $x$, которую мы хотим оценить в точке $x_0$. Так как плотность --- производная функции распределения $F(x_0)$ (т.е., $f(x_0) = d F(x_0)/dx$), мы получаем
\[
f(x_0) = \underset{h \rightarrow 0}{\lim} \frac{F(x_0 + h) - F(x_0 - h)}{2h} = \underset{h \rightarrow 0}{\lim} \frac{\Pr[x_0 - h < x < x_ 0 + h]}{2h}.
\]
Для выборки $\{x_i, i = 1, \dots, N\}$ размера $N$ это означает использование оценки
\begin{equation}
\hat{f}_{HIST}(x_0) = \frac{1}{N} \sum_{i=1}^N \frac{{\bf{1}} (x_0 - h < x < x_ 0 + h)}{2h},
\end{equation}
где функция-индикатор
\[
{\bf{1}}(A) =  
\begin{cases} 
1 \hspace{0.2cm} \text{если событие А случается,} \\ 
0 \hspace{0.2cm} \text{в противном случае.} 
\end{cases}
\]

Оценка $\hat{f}_{HIST}(x_0)$ --- это высота столбика диаграммы, центрированного относительно $x_0$ с шириной $2h$, она равняется доле выборки, которая лежит в промежутке от $x_0 - h$ до $x_0 + h$. Если оценивать $\hat{f}_{HIST}(x_0)$ с постоянным шагом по $x$, через каждые $2h$ единиц, то можно получить гистограмму. 

Оценка $\hat{f}_{HIST}(x_0)$ присваивает одинаковый вес всем наблюдениям в $x_0 \pm h$. Это становится ясно, если переписать (9.1) как
\begin{equation}
\hat{f}_{HIST}(x_0) = \frac{1}{Nh} \sum_{i=1}^N \frac{1}{2} \times {\bf{1}} \left( \left|\frac{x_i - x_0}{h}\right| < 1 \right).
\end{equation}
Это приводит к оценке плотности, которая является ступенчатой функцией, даже если лежащая в основе плотность непрерывная. Более гладкие оценки могут быть получены с использованием взвешивающих функций, отличных от функции-индикатора, которая выбрана здесь.

\subsection{Ядерная оценка функции плотности}

Ядерная оценка плотности, введённая Росенблаттом (1956), обобщает оценку гистограммы (9.2) с помощью альтернативной взвешивающей функции. Таким образом,
\begin{equation}
\hat{f}(x_0) =  \frac{1}{Nh} \sum_{i=1}^N  K \left( \frac{x_i - x_0}{h} \right).
\end{equation}
Взвешивающая функция $K(\cdot)$ называется ядерной функцией и удовлетворяет ограничениям, которые приведены в следующем разделе. Параметр $h$ --- параметр сглаживания, который называется шириной окна, и $h$, умноженное на два, --- ширина окна, умноженная на два. Некоторые источники называет шириной окна именно $2h$. Плотность оценивается путём оценивания $\hat{f}(x_0)$ на более широком диапазоне значений $x_0$, чем диапазон значений, используемый для формирования гистограммы. В стандартном случае оценивание производится на выборочных значениях $x_1, \dots, x_N$. Это также позволяет получать оценку плотности, которая является более гладкой, чем оценка гистограммы.

\subsection{Ядерные функции}

Ядерная функция $K(\cdot)$ --- непрерывная функция, симметричная относительно нуля, которая имеет единичный интеграл и удовлетворяет дополнительным условиям ограниченности. Как и Ли (1996), мы предполагаем, что ядро удовлетворяет следующим условиям:
\begin{enumerate}
\item $K(z)$ симметрична относительно нуля и непрерывна.
\item $\int K(z)dz = 1$, $\int zK(z)dz = 0$ и $\int |K(z)|dz < \infty$.
\item Или (а) $K(z) = 0$, если $|z| \geq z_0$ для некоторых $z_0$, или (б) $|z|K(z) \rightarrow 0$ при $|z| \rightarrow \infty$.
\item $\int z^2K(z)dz = k$, где $k$ --- константа.
\end{enumerate}

На практике ядерные функции работают лучше, если они удовлетворяют условию (3a), а не только более слабому условию (3б). Тогда будет рассматриваться интервал $[-1,1]$, а не $[- z_0, z_0]$. Этот интервал является нормированием для удобства, и, как правило, $K(z)$ ограничена интервалом $z \in [-1,1]$.

Некоторые часто используемые ядерные функции приведены в таблице 9.1. Равномерное ядро использует те же веса, что и гистограмма с шириной столбцов $2h$, но оно даёт гистограмму, которая оценивается на серии точек $x_0$, а не с использованием фиксированных столбцов. Гауссово ядро удовлетворяет (3б), а не (3а), потому что оно не ограничивает $z \in [-1,1]$. Ядро порядка $p$ --- ядро, первый ненулевой момент которого является $p$ моментом. Первые семь ядер имеют второй порядок и удовлетворяют второму условию из пункта 2. Последние два ядра имеют четвёртый порядок. Ядра более высокого порядка могут увеличить скорость сходимости, если $f(x)$ дважды и более дифференцируема (см. раздел 9.3.10). Однако они могут принимать отрицательные значения. В таблице 9.1 для некоторых ядер также представлен параметр $\delta$, который определён в (9.11) и который используется в разделе 9.3.6 для выбора ширины окна.

Если есть $K(\cdot)$ и $h$, то легко найти оценку. Если ядерная оценка оценивается в $r$ различных значениях $x_0$, то вычисление ядерной оценки требует максимум $Nr$ операций, когда ядро имеет неограниченный носитель. На практике можно значительно уменьшить число необходимых операций, см., например, Хэрдл (1990, стр. 35).


\begin{table}[h]
\begin{center}
\caption{\label{tab:pred} Ядерные функции: наиболее часто используемые примеры}
\begin{minipage}{16.5cm}
\begin{tabular}[t]{lll}
\hline
\hline
\bf{Ядро}\footnote{Константа $\delta$ определена в (9.11). Она используется для получения оценки Сильвермана, которая приведена в (9.13).} & \bf{Ядерная функция $K(z)$} & \bf{$\delta$} \\
\hline
Равномерное (или прямоугольное) & $\frac{1}{2} \times {\bf{1}} (|z| < 1)$ & 1.3510 \\
Треугольное (или треугольник) & $(1 - |z|) \times {\bf{1}} (|z| < 1)$ & -- \\
Епанечникова (или квадратичное) & $\frac{3}{4} (1 - z^2) \times {\bf{1}} (|z| < 1)$ & 1.7188 \\
Квартическое (или биквадратичное) & $\frac{15}{16} (1 - z^2)^2 \times {\bf{1}} (|z| < 1)$ & 2.0362 \\
Триквадратичное & $\frac{35}{32} (1 - z^2)^3 \times {\bf{1}} (|z| < 1)$ & 2.3122 \\
Трикубическое & $\frac{70}{81} (1 - |z|^3)^3 \times {\bf{1}} (|z| < 1)$ & -- \\
Гауссово (или нормальное) & $(2\pi)^{-1/2}\exp(-z^2/2)$ & 0.7764 \\
Гауссово четвёртого порядка & $\frac{1}{2}(3-z)^2(2\pi)^{-1/2}\exp(-z^2/2)$ & -- \\
Квартическое четвёртого порядка & $\frac{15}{32}(3-10z^2+7z^4)\times {\bf{1}} (|z| < 1)$ & -- \\
\hline
\hline
\end{tabular}
\end{minipage}
\end{center}
\end{table}

\subsection{Пример ядерной плотности}

Выбор ширины окна $h$ уже был проиллюстрирован на графике 9.2.

Здесь мы проиллюстрируем выбор ядра, используя сгенерированные данные --- случайную выборку, которая состоит из 100 наблюдений и имеет нормальное распределение $\mathcal{N}[0,25^2]$. Для этой выборки выборочное среднее равно 2.81, а выборочное стандартное отклонение равно 25.27.

График 9.4 показывает эффекты  использования различных ядер. Для Епанечникова, Гауссова, квартического и равномерного ядер оценка Сильвермана, приведённая в (9.13), даёт ширину окна, которая равна 0.545, 0.246, 0.246 и 0.214 соответственно. Полученные ядерные оценки плотности очень похожи, даже равномерное ядро, которое порождает скользящую гистограмму. Различие оценок плотности для разных ядер гораздо меньше, чем для разной ширины окна. Это видно на графике 9.2.

\vspace{5cm}

График 9.4: Ядерные оценки плотности для логарифма заработной платы для четырёх различных ядер с использованием оценки Сильвермана для ширины окна. Используются те же данные, что и для графика 9.1.

\subsection{Статистические выводы}

Мы рассмотрим распределение ядерной оценки плотности $\hat{f}(x)$ при заданных $K(\cdot)$ и $h$, предполагая, что данные $x$ независимы и одинаково распределены. Оценка $\hat{f}(x)$ является смещённой. Это смещение стремится асимптотически к нулю, если ширина окна $h \rightarrow 0$ при $N \rightarrow \infty$, поэтому $\hat{f}(x)$ является состоятельной оценкой. Тем не менее, параметр смещения необязательно исчезает в асимптотическом нормальном распределении $\hat{f}(x)$. Это затрудняет статистические выводы.

\begin{center}
Математическое ожидание и дисперсия
\end{center}

Получение математического ожидания и дисперсии $\hat{f}(x)$ описано в разделе 9.8.1, в предположении, что вторая производная $f(x)$ существует и  ограничена. Также предполагается, что ядро удовлетворяет условию $\int zK(z)dz = 0$, которое является вторым свойством ядра из раздела 9.3.3.

Ядерная оценка плотности смещена с параметром смещения $b(x_0)$, который зависит от ширины окна и кривизны истинной плотности и используемого для оценки ядра:
\begin{equation}
b(x_0) = \E[\hat{f}(x_0)] - f(x_0) = \frac{1}{2}h^2 f''(x_0) \int z^2K(z)dz.
\end{equation}
Ядерная оценка смещена на порядок малости $O(h^2)$, где мы используем обозначение порядка малости, т.е. функция $a(h)$ имеет порядок малости $O(h^k)$, если $a(h)/h^k$ конечно. Смещение исчезает асимптотически, если $h \rightarrow 0$ при $N \rightarrow \infty$.

В предположении, что $h \rightarrow 0$ и $N \rightarrow \infty$, дисперсия ядерной оценки плотности имеет вид:
\begin{equation}
\V[\hat{f}(x_0)] = \frac{1}{Nh} f(x_0) \int K(z)^2dz + o\left( \frac{1}{Nh} \right),
\end{equation}
где функция $a(h)$ имеет порядок малости $O(h^k)$ при $a(h)/h^k \rightarrow 0$. Дисперсия зависит от размера выборки, ширины окна, истинной плотности и ядра. Дисперсия исчезает, если $Nh \rightarrow \infty$. Для этого необходимо, чтобы при $h \rightarrow 0$ дисперсия уменьшалась с меньшей скоростью, чем при $N \rightarrow \infty$.

\begin{center}
Состоятельность
\end{center}

Ядерная оценка поточечно соcтоятельна, то есть она состоятельна в конкретной точке $x = x_0$, если и смещение, и дисперсия исчезают. Это происходит в случае, если $h \rightarrow 0$ и $Nh \rightarrow \infty$.

Можно показать, что для оценивания $f(x)$ для всех значений $x$ может действовать более сильное условие равномерной сходимости $\sup_{x_0} |\hat{f}(x_0) - f(x_0)| \stackrel{p}{\rightarrow} 0$ при $Nh/\ln N \rightarrow \infty$. В этом случае $h$ должно быть больше, чем в случае поточечной сходимости.

\begin{center}
Асимптотическая нормальность
\end{center}

Предыдущие результаты показывают, что асимптотически $\hat{f}(x_0)$ имеет математическое ожидание $f(x_0) + b(x_0)$ и дисперсию $(Nh)^{-1}f(x_0)\int K(z)^2dz$. Отсюда следует, что если можно применить центральную предельную теорему, то ядерная оценка плотности имеет предельное распределение
\begin{equation}
\sqrt{Nh}(\hat{f}(x_0) - f(x_0) - b(x_0)) \stackrel{d}{\rightarrow} \mathcal{N} \left[ 0, f(x_0) \int K(z)^2dz \right].
\end{equation}
Тот вариант центральной предельной теоремы, который применяется в данном случае, не является стандартным и требует выполнения условия (4), см., например, Ли (1996, стр. 139) или Пагана и Улла (1999, стр. 40).

Важно отметить наличие параметра смещения $b(x_0)$, который был определён в (9.4). Для типичных вариантов ширины окна этот параметр не исчезает, усложняя получение доверительных интервалов (они представлены в разделе 9.3.7).

\subsection{Выбор ширины окна}

Выбор ширины окна $h$ гораздо более важен, чем выбор ядерной функции $K(\cdot)$. Существует разница между выбором небольшого значения $h$, чтобы уменьшить смещённость, и выбором большого значения $h$, чтобы обеспечить гладкость. Стандартный способ в данном случае --- использовать среднеквадратичную ошибку $(MSE)$, которая равна сумме квадрата смещения и дисперсии.

Из (9.4) смещение имеет порядок малости $O(h^2)$, и из (9.5) дисперсия имеет порядок малости $O((Nh)^{-1})$. Интуитивно $MSE$ минимизируется с помощью выбора $h$ так, чтобы квадрат смещения и дисперсия имели один и тот же порядок. Таким образом, $h^4 = (Nh)^{-1}$. Это означает, что оптимальная ширина окна $h = O(N^{-0.2})$ и $\sqrt{Nh} = O(N^{0.4})$. Теперь мы приведём более формальное описание, которое включает практическую оценку для $h$.
 
\begin{center}
Cредняя интегрированная квадратическая ошибка
\end{center} 

Локальный показатель эффективности ядерной оценки плотности в точке $x_0$ --- это $MSE$ 
\begin{equation}
MSE[\hat{f}(x_0)] = \E[(\hat{f}(x_0) - f(x_0))^2],
\end{equation}
где ожидание берётся от функции плотности $f(x)$. Так как $MSE$ равняется сумме дисперсии и квадрата смещения, (9.4) и (9.5) дают $MSE$ ядерной оценки плотности 
\begin{equation}
MSE[\hat{f}(x_0)] \simeq \frac{1}{Nh} f(x_0) \int K(z)^2dz + \Bigl\{ \frac{1}{2}h^2 f''(x_0) \int z^2K(z)dz \Bigr\}^2.
\end{equation}

Чтобы получить глобальный показатель эффективности во всех значениях $x_0$, мы начнём с определения интегрированной квадратической ошибки $(ISE)$
\begin{equation}
ISE(h) = \int (\hat{f}(x_0) - f(x_0))^2 dx_0,
\end{equation}
непрерывный аналог суммирования квадратов ошибок по всем $x_0$ в дискретном случае. Это можно записать как функцию от $h$, чтобы подчеркнуть зависимость от ширины окна. Затем мы устраняем зависимость $\hat{f}(x_0)$ от значений $x$, отличных от $x_0$, взяв математическео ожидаемое $ISE$ от плотности $f(x)$. Это даёт среднюю интегрированную квадратическую ошибку $(MISE)$,
\[
MISE(h) = \E[ISE(h)] = \E \left[ \int (\hat{f}(x_0) - f(x_0))^2 dx_0 \right] = \int \E \left[(\hat{f}(x_0) - f(x_0))^2 dx_0 \right] = \int MSE[\hat{f}(x_0)]dx_0,
\]
где $MSE[\hat{f}(x)]$ определено в (9.8). Из предыдущих алгебраических вычислений $MISE$ равно интегрированной среднеквадратической ошибке $(IMSE)$.

\begin{center}
Оптимальная ширина окна
\end{center}

Оптимальная ширина окна является минимумом $MISE$. Дифференцируя $MISE(h)$ по $h$ и приравнивая производную к нулю, можно получить оптимальную ширину окна
\begin{equation}
h^* = \delta \left( \int f''(x_0)^2 dx_0 \right)^{-0.2} N^{-0.2},
\end{equation}
где $\delta$ зависит от используемой ядерной функции,
\begin{equation}
\delta = \left( \frac{\int K(z)^2dz}{(\int z^2K(z)dz)^2} \right)^{0.2}.
\end{equation}
Этот результат получен Сильверманом (1986).

Так как $h^* = O(N^{-0.2})$, мы имеем $h^* \rightarrow 0 $ при $N \rightarrow \infty$ и $Nh^* = O(N^{0.8}) \rightarrow \infty$, что требуется для состоятельности.

Смещение в $\hat{f}(x_0)$ имеет порядок малости $O(h^{*2}) = O(N^{-0.4})$, но оно исчезает при $N \rightarrow \infty$. Для оценки гистограммы можно показать, что $h^* = O(N^{-0.2})$ и $MISE(h^*) = O(N^{-2/3})$ уступает  $MISE(h^*) = O(N^{-4/5})$ для ядерной оценки плотности.

Оптимальная ширина окна зависит от кривизны плотности, и $h$ меньше, если $f(x)$ может сильно меняться.

\begin{center}
Оптимальное ядро
\end{center}

Оптимальная ширина окна зависит от выбора ядра (см. (9.10) и (9.11)). Можно показать, что $MISE(h^*)$ мало отличается для  разных ядер при условии, что различные оптимальные $h^*$ используются для различных ядер (график 9.4 это иллюстрирует). Можно показать, что оптимальное ядро --- ядро Епанечникова, хотя его превосходство незначительно.

Выбор ширины окна гораздо более важен, чем выбор ядра, и из (9.10) он зависит от ядра.

\begin{center}
Оценка ширины окна 
\end{center}

Оценка ширины окна --- это простая формула для $h$, которая зависит от размера выборки $N$ и стандартного отклонения $s$.

Стоит начать с предположения, что данные имеют нормальное распределение. Тогда  \\ $\int f''(x_0)^2dx_0 = 3/(8\sqrt{\pi}\sigma^5) = 0.2116/\sigma^5$. В таком случае (9.10) можно преобразовать как
\begin{equation}
h^* = 1.3643\delta N^{-0.2}s,
\end{equation}
где $s$ --- выборочное стандартное отклонение $x$ и $\delta$ приведены в таблице 9.1 для различных ядер. Для Епанечникова ядра $h^* = 2.345N^{-0.2}s$, а для Гауссова ядра $h^* = 1.059N^{-0.2}s$. Для нормального ядра ширина окна гораздо меньше, потому что в отличие от большинства ядер нормальное ядро даёт некоторый вес $x_i$, даже если $|x_i - x_0| > h$. На практике используют оценку Сильвермана
\begin{equation}
h^* = 1.3643\delta N^{-0.2}\min(s,iqr/1.349),
\end{equation}
где $iqr$ --- выборочное межквартильное отклонение. При этом $iqr/1.349$ используется в качестве альтернативной оценки $\sigma$. Она защищает от выбросов, которые могут увеличить $s$ и привести к слишком большому значению $h$.

Эти оценки $h$ хорошо работают на практике, особенно для симметричных одномодальных плотностей, даже если $f(x)$ не является плотностью нормального распределения. Тем не менее, для проверки всегда следует попробовать различные варианты ширины окна такие, как дважды и половина оценки Сильвермана.

Для примера на графиках 9.2 и 9.4 мы имеем $177^{-0.2} = 0.3551$, $s = 0.8282$ и $iqr/1.349 = 0.6459$, поэтому (9.13) даёт $h^* = 0.3173\delta$. Для ядра Епанечникова, например, это получается $h^* = 0.545$, так как $\delta = 1.7188$ из таблицы 9.1.

\begin{center}
Кросс-валидация
\end{center}

Из (9.9) $ISE(h) = \int \hat{f}^2(x_0)dx_0 - 2\int \hat{f}(x_0)f(x_0)dx_0 + \int f^2(x_0)dx_0$. Третье слагаемое не зависит от $h$. Альтернативный подход позволяет оценить первые два члена $ISE(h)$ с помощью
\begin{equation}
CV(h) = \frac{1}{N^2h} \sum_i \sum_j K^{(2)} \left( \frac{x_i - x_j}{h} \right) - \frac{2}{N} \sum_{i=1}^N \hat{f}_{-i}(x_i),
\end{equation}
где $K^{(2)}(u) = \int K(u - t)K(t)dt$ --- свёртка $K$ с самим собой, а $\hat{f}_{-i}(x_i)$ --- ядерная оценка $f(x_i)$ с выбрасыванием одного наблюдения. Вывод можно посмотреть у Ли (1996, стр. 137) или Пагана и Улла (1999 стр.51). Оценка $h_{CV}$, полученная с помощью кросс-валидации, выбирается так, чтобы она минимизировала $\widehat{CV}(h).$ Можно показать, что $h_{CV} \stackrel{p}{\rightarrow} h^*$ при $N \rightarrow \infty$, но скорость сходимости очень низкая.

Вычислить $h_{CV}$ довольно тяжело, потому что необходимо вычислить $ISE(h)$ для ряда значений $h$. Часто нет необходимости проводить кросс-валидацию для ядерного оценивания плотности, так как оценка Сильвермана обычно является хорошей отправной точкой.

\subsection{Доверительные интервалы}

Ядерные оценки плотности, как правило, представлены без доверительных интервалов, но можно построить точечные доверительные интервалы для $f(x_0)$, где <<точечные>> означает построенные для определённого значенияы $x_0$. Простая процедура состоит в том, чтобы получить доверительные интервалы в небольшом количестве оцениваемых точек $x_0$, например, в десяти точках. Разумно взять эти точки на равных расстояниях по всему диапазону $x$. Потом надо нанести эти доверительные интервалы на график вместе с оценкой  плотности.

Результат (9.6) приводит к следующему 95\% доверительному интервалу для $f(x_0)$:
\[
f(x_0) \in \hat{f}(x_0) - b(x_0) \pm 1.96 \times \sqrt{\frac{1}{Nh}\hat{f}(x_0)\int K(z)^2dz}.
\]

Для большинства ядер $\int K(z)^2dz$ можно легко получить с помощью аналитических методов.

Ситуация усложняется из-за наличия параметра смещения, который нельзя игнорировать при конечных выборках, хотя асимптотически $b(x_0) \stackrel{p}{\rightarrow} 0$. Это так, потому что при оптимальной ширине окна $h^* = O(N^{-0.2})$ смещение нормированной случайной величины $\sqrt{Nh}(\hat{f}(x_0) - f(x_0))$, которая представлена в (9.6), не исчезает, так как $\sqrt{Nh^*} \times O(h^{*2}) = O(1)$. Можно оценить смещение, используя (9.4) и ядерную оценку $f''(x_0)$, но на практике оценка $f''(x_0)$ имеет много шумов. Вместо этого стандартный метод
состоит в уменьшении смещения при вычислении доверительного интервала, а не самой $\hat{f}(x_0)$. Это делается с помощью недосглаживания, то есть выбора $h < h^*$ такого, что $h^* = o(N^{-0.2})$. Другие подходы включают использование ядер более высокого порядка таких, как ядра четвёртого порядка, которые приведены в таблице 9.1, или использование метода бутстрэп (см. раздел 11.6.5).

Можно также вычислить доверительные интервалы для $f(x)$ для всех возможных значений $x$. Они будут шире, чем точечные доверительные интервалы для каждого значения $x_0$.

\subsection{Оценивание производных плотности}

В некоторых случаях необходимо получить оценки производных плотности. Например, оценивание параметра смещения $\hat{f}(x_0)$, который приведён в (9.4), требует оценить $f''(x_0)$.

Для простоты мы приводим оценки первой производной. Разностный подход использует $\hat{f}'(x_0) = [\hat{f}(x_0 + \Delta) - \hat{f}(x_0 - \Delta)]/2\Delta$. Альтернативный подход вместо этого подразумевает взятие первой производной $\hat{f}(x_0)$ из (9.3), что даёт $\hat{f}'(x_0) = - (Nh^2)^{-1} \sum_i K'((x_i - x_0)/h)$.

Интуитивно ясно, что необходимо использовать большую ширину окна для оценивания производных, которые могут
меняться быстрее, чем $f(x_0)$. Смещение $\hat{f}^{(s)}(x_0)$ сходится так же, как и раньше, но дисперсия сходится медленнее, что приводит к оптимальной ширине окна $h^* = O(N^{-1/(2s+2p+1)})$, если $f(x_0)$ дифференцируема $p$ раз. Для ядерного оценивания первой производной нам необходимо, что $p \geq 3$.

\subsection{Многомерная ядерная оценка плотности}

Предыдущее описание рассматривало ядерное оценивание плотности для скалярного $x$. Многомерная ядерная оценка плотности для плотности $k$-мерной случайной величины $x$ имеет вид: 
\[
\hat{f}(x_0) = \frac{1}{Nh^k} \sum_{i=1}^N K\left( \frac{x_i - x_0}{h} \right),
\]
где $K(\cdot)$ --- $k$-мерное ядро. Обычно $K(\cdot)$ является ядром-произведением, произведением одномерных ядер. Также можно использовать многомерные ядра такие, как плотность многомерного нормального распределения или сферические ядра, пропорциональные $K(z'z)$. Ядро $K(\cdot)$ удовлетворяет свойствам, аналогичным свойствам для одномерного случая, см. Ли (1996, стр. 125).

Аналитические результаты и выражения аналогичны тем, которые были раньше. Только дисперсия $\hat{f}(x_0)$ убывает со скоростью $O(Nh^k)$. При $k > 1$ она убывает медленнее, чем $O(Nh)$ в одномерном случае. Тогда

\[
\sqrt{Nh^k}(\hat{f}(x_0) - f(x_0) - b(x_0)) \stackrel{d}{\rightarrow} \mathcal{N}\left[ 0, f(x_0)\int K(z)^2dz \right].
\] 

Оптимальная ширина в данном случае $h = O(N^{-1/(k+4)})$, что больше, чем $O(N^{-0.2})$ в одномерном случае. Тогда получается, что $\sqrt{Nh^k} = O(N^{2/(k+4)})$. Оценка Сильвермана и кросс-валидация могут быть расширены и на многомерный случай. Для нормального ядра-произведения  оценка ширины окна Скотта для $j$-той компоненты $x$ $h_j = N^{-1/(k+4)}s_j$, где $s_j$ --- выборочное стандартное отклонение $x_j$. 

Проблема малочисленности данных, скорее всего, возникает в случае многомерного ядра. Это называется проклятием размерности, так как меньшее число наблюдений в непосредственной близости от $x_0$ получают существенный вес, когда $x$ имеет большую размерность. Даже когда в этом нет проблемы, изображение даже двумерной ядерной оценки плотности возможно лишь на трёхмерном графике. А его, возможно, будет трудно интерпретировать.

Одно из возможных применений многомерной ядерной оценки плотности состоит в том, чтобы сделать возможным оценивание условной плотности. Так как $f(y|x) = f(x,y)/f(x)$, оценка $\hat{f}(y|x) = \hat{f}(x,y)/\hat{f}(x)$ является естественной оценкой условной плотности. Здесь $\hat{f}(y|x)$ и $\hat{f}(x)$ являются двумерной и одномерной ядерными оценками плотности.

\subsection{Ядра высшего порядка}

В предыдущем анализе предполагается, что $f(x)$ дважды дифференцируема. Это является необходимой предпосылкой для получения параметра смещения из (9.4). Если $f(x)$ более чем дважды дифференцируема, то использование ядер более высокого порядка (см. раздел 9.3.3 примеры ядер четвёртого порядка) уменьшает величину смещения, что приводит к меньшему значению $h^*$ и более высокой скорости сходимости. Стандартное утверждение состоит в том, что если $x$ является $k$-мерным, $f(x)$ дифференцируема $p$ раз и используется ядро порядка $p$, то ядерная оценка $f(x_0)$, $\hat{f}(x_0)$, имеет оптимальную скорость сходимости $N^{-p/(2p+k)}$, когда $h^* = O(N^{-1/(2p+k)})$.

\subsection{Альтернативные непараметрические оценки плотности}

Ядерная оценка плотности является стандартной непараметрической оценкой. Другие оценки плотности представлены, например, у Пагана и Улла (1999). Такого рода оценки часто основываются на подходах таких, как методы ближайших соседей, и чаще используются в непараметрической регрессии. Они кратко описаны в разделе 9.6.

\section{Непараметрическая локальная регрессия}

Мы рассматриваем регрессию скалярной зависимой переменной $y$ на скалярный регрессор $x$. Регрессионная модель имеет вид:
\begin{equation}
y_i = m(x_i) + \e_i, i = 1, \dots, N,
\end{equation}
\[
\e_i \sim iid[0, \sigma_{\e}^2].
\]
Сложность заключается в том, что функциональная форма $m(\cdot)$ не задана. По этой причине невозможно оценивать модель с помощью НМНК.

В этом разделе приводится общее описание непараметрической регрессии с использованием локальных средневзвешенных. Описание ядерной регрессии приведено в разделе 9.5, а другие широко применяемые методы локально взвешенных методов представлены в разделе 9.6.

\subsection{Локальные средневзвешенные}

Предположим, что одному значению регрессора, например, $x_0$, соответствует несколько значений $y$, например, $N_0$ наблюдений. Тогда очевидная оценка $m(x_0)$ --- выборочное среднее значение $N_0$ $y$-ков. Обозначим эту оценку как $\tilde{m}(x_0)$. Отсюда следует, что $\tilde{m}(x_0) \sim [m(x_0), N_0^{-1}\sigma_{\e}^2]$, так как это среднее значение $N_0$ наблюдений, которые в силу (9.15) независимы и одинаково распределены с математическим ожиданием $m(x_0)$ и дисперсией $\sigma_{\e}^2$.

Оценка $\tilde{m}(x_0)$ несмещена, но она необязательно состоятельна. Для состоятельности необходимо, чтобы $N_0 \rightarrow \infty$ при $N \rightarrow \infty$ так, чтобы $\V[\tilde{m}(x_0)] \rightarrow 0$. При дискретных регрессорах эта оценка может иметь сильные шумы в конечных выборках, потому что $N_0$ может быть маленьким. Более того, для непрерывных регрессоров может быть только одно наблюдение, в котором $x_i$ принимает конкретное значение $x_0$, даже когда $N \rightarrow \infty$.

Проблема малочисленности данных может быть преодолена с помощью усреднения наблюдаемых значений $y$, когда $x$ близок к $x_0$ в дополнение к $x$ в точности равному $x_0$. Прежде всего отметим, что оценку $\tilde{m}(x_0)$ можно выразить как средневзвешенное зависимой переменной $\tilde{m}(x_0) = \sum_i w_{i0}y_i$, где веса $w_{i0}$ равны $1/N_0$, если $x_i = x_0$, и равны 0, если $x_i \not= x_0$. Таким образом, вес варьируется в зависимости от оцениваемой точки $x_0$ и выборочных значений регрессоров.

Мы рассматриваем локально средневзвешенную оценку
\begin{equation}
\hat{m}(x_0) = \sum_{i=1}^N w_{i0,h}y_i,
\end{equation}
где веса
\[
w_{i0,h} = w(x_i, x_0, h)
\]
в сумме равны единице, то есть $\sum_i w_{i0,h} = 1$. Веса так заданы, чтобы они увеличивались при приближении $x_i$ к $x_0$.

Дополнительный параметр $h$ является общим обозначением для параметра ширины окна. Он задан таким образом, что меньшие значения $h$ приводят к меньшей ширине окна, а также больший вес придаётся тем наблюдениям $x_i$, которые близки к $x_0$. Конкретно для ядерной регрессии $h$ --- это ширина окна. Другие методы, которые приведены в разделе 9.6, имеют альтернативные параметры сглаживания, которые играют ту же роль, что $h$ в данном случае. С уменьшением $h$ $\hat{m}(x_0)$ становится менее смещённой, так как используются только наблюдения, близкие к $x_0$. Но в то же время разброс её значений становится больше, так как используется меньше наблюдений.

Оценка МНК для линейной регрессионной задачи --- средневзвешенное $y_i$, так как после некоторых алгебраических выражений мы получаем
\[
\hat{m}_{OLS}(x_0) = \sum_{i=1}^N \Bigl\{\frac{1}{N}+\frac{(x_0-\bar{x})(x_i-\bar{x})}{\sum_j (x_j-\bar{x})^2} \Bigr\} y_i.
\]
Однако веса МНК могут увеличиваться с увеличением расстояния между $x_0$ и $x_i$, если, например, $x_i > x_0 > \bar{x}$. Вместо этого локальная регрессия использует веса, которые уменьшаются в $|x_i - x_0|$. 

\subsection{Пример: метод $k$ ближайших соседей}

Мы рассмотрим простой пример, невзвешенное среднее значений $y$, которые соответствуют ближайшим $(k - 1)/2$ наблюдениям $x$, меньшим $x_0$, и ближайшим $(k - 1)/2$ наблюдениям $x$, большим $x_0$.

Надо проранжировать наблюдения по мере увеличения значений $x$. Тогда оценивание в точке $x_0 = x_i$ даёт
\[
\hat{m}_k(x_i) = \frac{1}{k}(y_{i-(k-1)/2} + \dots + y_{i+(k-1)/2}),
\]
где для простоты $k$ нечётно, а также игнорируются возможные мофицикации из-за совпадающих значений и значений $x_0$, близких к крайним значениям $x_1$ и $x_N$. Эта оценка может быть представлена как частный случай (9.16) с весами
\[
w_{i0,k} = \frac{1}{k} \times {\bf{1}} \left( |i - 0| < \frac{k-1}{2} \right), x_1 < x_2 < \cdots < x_0 < \cdots < x_N.
\]

Эту оценку можно называть по-разному. Мы называем её (симметризованной) оценкой $k$ ближайших соседей ($k$ -- NN, Nearest Neighbors), которая определена в разделе 9.6.1. Она также является стандартной локальной скользящей средней или скользящей средней длины $k$, центрированной относительно $x_0$. Она используется, например, для изображения на графике временного ряда $y$ в зависимости от времени $x$. Параметр $k$ играет роль ширины окна $h$ из раздела 9.4.1.: небольшие значения $k$ соответствуют небольшим значениям $h$.

В качестве примера рассмотрим данные, сгенерированные из модели
\begin{equation}
y_i = 150 + 6.5x_i - 0.15x_i^2 + 0.001x_i^3 + \e_i, i = 1, \dots, 100,
\end{equation}
\[
x_i = i,
\]
\[
\e_i \sim \mathcal{N}[0, 25^2].
\]
Математическое ожидание $y$ является кубическим по $x$, а $x$ принимает значения $1,2,\dots,100$, с точками перегиба при $x = 20$ и $x = 80$. К этому добавляются ошибки, имеющие нормальное распределение с стандартным отклонением 25.

График 9.5 отображает симметризованную $k - NN$ оценку с $k = 5$ и $25$. Обе скользящие средние говорят о кубической зависимости. Вторая является более гладкой, чем первая, но она всё-таки достаточно зубчатая, несмотря на то что для формирования среднего используется одна четверть выборки. Также на диаграмме изображена линия регрессии МНК.

\vspace{5cm}

График 9.5: Кривая регрессии $k$ ближайших соседей для двух различных вариантов $k$, а также линия регрессии МНК. Данные сгенерированы с помощью кубической полиномиальной модели.

Наклон $\hat{m}_k(x)$ более пологий в конечных точках при $k = 25$, чем при $k = 5$. Это иллюстрирует проблему краевых значений при оценивании $m(x)$ в краевых точках. Например, для наименьшего значения регрессора $x_1$ нет значений $x$, которые были бы меньше. Тогда средняя становится односторонней средней $\hat{m}_k(x_1) = (y_1 + \cdots + y_{1 + (k-1)/2})/[(k + 1)/2]$. Так как для этих данных $m_k(x)$ возрастает по $x$ на этом промежутке, это приводит к тому, что $\hat{m}_k(x_1)$ является переоценённой, и завышение увеличивается по $k$. Такие граничные проблемы можно уменьшить, используя методы, приведённые в разделе 9.6.2.

\subsection{Пример регрессии $LOWESS$}

Использование весов, альтернативных тем, которые используются для формирования симметризованной $k - NN$ оценки,  может привести к получению улучшенных оценок $m(x)$.

Примером может служить $LOWESS$ оценка, которая определена в разделе 9.6.2. Эта оценка является более гладкой оценкой $m(x)$, так как в данном случае используются ядерные веса, а не функция-индикатор. Это аналогично тому, что ядерная оценка плотности более гладкая, чем гистограмма. Она также имеет меньшее смещение (см. раздел 9.6.2), что особенно полезно при оценивании $m(x)$ в крайних точках.

График 9.6 отражает $LOWESS$ оценку при $k = 25$ для данных, cгенерированных с помощью (9.17). Эта оценка локальной регрессии довольно близка к истинной кубической функции условного математического ожидания, которая также изображена на графике. Сравнивая графики 9.6 и 9.5, на которых изображены симметризованные $k - NN$ оценки с $k = 25$, мы видим, что регрессия $LOWESS$ приводит к получению намного более гладкой оценки и более точных оценках на границах.

\subsection{Статистические выводы}

Когда ошибки имеют нормальное распределение, и анализ зависит от $x_1, \dots, x_N$, легко получить точное распределение $\hat{m}(x_0)$ для малых выборок из (9.16).

\vspace{5cm}

График 9.6: Непараметрическая линия регрессии с использованием $LOWESS$, а также кубическая линия регрессии. Используются те же данные, что и для графика 9.5.

Подстановка $y_i = m(x_i) + \e_i$ в определение $\hat{m}(x_0)$ приводит к следующему результату:
\[
\hat{m}(x_0) - \sum_{i=1}^N w_{i0,h} m(x_i) =  \sum_{i=1}^N w_{i0,h}\e_i.
\]
Следовательно, при фиксированных регрессорах и при независимых и одинаково нормально распределенных $\e_i$,  $\e_i\sim \mathcal{N}[0, \sigma_{\e}^2]$, мы получаем
\begin{equation}
\hat{m}(x_0) \sim \mathcal{N}\left[\sum_{i=1}^N w_{i0,h} m(x_i), \sigma_{\e}^2 \sum_{i=1}^N w_{i0,h}^2\right].
\end{equation}

Заметим, что в общем случае $\hat{m}(x_0)$ смещена, и распределение необязательно центрировано относительно $m(x_0)$.

При стохастических регрессорах и ошибках, не имеющих нормального распределения, мы берём зависимость от $x_1, \dots, x_N$ и применяем центральную предельную теорему для $U$-статистики, которая подходит для двойного суммирования (см., например, Паган и Улл, 1999, стр. 359). Тогда для $\e_i$, которые независимы и одинаково распределены по нормальному закону $\mathcal{N}[0, \sigma_{\e}^2]$,

\begin{equation}
c(N)\sum_{i=1}^N w_{i0,h}\e_i \stackrel{d}{\rightarrow} \mathcal{N}\left[ 0, \sigma_{\e}^2 \lim c(N)^2 \sum_{i=1}^N w_{i0,h}^2 \right],
\end{equation}
где $c(N)$ --- функция размера выборки с $O(c(N)) < N^{1/2}$, которая может меняться вместе с локальной оценкой. Например, $c(N) = \sqrt{Nh}$ для ядерной регрессии и $c(N) = N^{0.4}$ для ядерной регрессии с оптимальной шириной окна. Тогда
\begin{equation}
c(N)(\hat{m}(x_0) - m(x_0) - b(x_0)) \stackrel{d}{\rightarrow} \mathcal{N}\left[ 0, \sigma_{\e}^2 \lim c(N)^2 \sum_{i=1}^N w_{i0,h}^2 \right],
\end{equation}
где $b(x_0) = m(x_0) - \sum_i w_{i0,h} m(x_i)$. Обратите внимание, что (9.20) даёт (9.18) для асимптотического распределения $\hat{m}(x_0)$. 

Очевидно, что распределение $\hat{m}(x_0)$, простого средневзвешенного, может быть получено при альтернативных предположениях о распределении. Например, для гетероскедастичных ошибках дисперсия из (9.19) и (9.20) заменяется на $\lim c(N)^2 \sum_i \sigma_{\e,i}^2 w_{i0,h}^2$, где для получения состоятельных оценок можно заменить $\sigma_{\e,i}^2$ на квадрат остатков $(y_i - \hat{m}_(x_i))^2$. Также можно использовать метод бутстрэп (см. раздел 11.6.5).

\subsection{Выбор ширины окна}

В этой главе мы следуем непараметрической терминологии, что оценка $\theta_0$ $\hat{\theta}$ имеет скорость сходимости $N^{-r}$, если $\hat{\theta} = \theta_0 + O_p(N^{-r})$. Таким образом, $N^r (\hat{\theta} - \theta_0) = O_p(1)$, и в идеальном случае $N^r (\hat{\theta} - \theta_0)$ имеет предельное нормальное распределение. Стоить обратить внимание, что $\sqrt{N}$-состоятельная оценка сходится со скоростью $N^{-1/2}$. Обычно непараметрические оценки имеют меньшую скорость сходимости, чем эта, при $r < 1/2$. Это так, потому что для устранения смещения необходима небольшая ширина окна $h$. Однако в таком случае для оценивания $\hat{m}(x_0)$ используются не все $N$ наблюдений.

В качестве примера рассмотрим $k - NN$ оценку из примера в разделе 9.4.2. Пусть $k = N^{4/5}$. Тогда, например, $k = 251$, если $N = 1 000$. В этом случае оценка является состоятельной, так как скользящая средняя использует $N^{4/5}/N = N^{-1/5}$ наблюдений из выборки, и при $N \rightarrow \infty$ она учитывает только наблюдения лежащие очень близко к $x_0$. Используя (9.18), дисперсия оценки скользящей средней равна $\sigma_{\e}^2 \sum_i w_{i0,k}^2 = \sigma_{\e}^2 \times k \times (1/k)^2 = \sigma_{\e}^2 \times 1/k = \sigma_{\e}^2 N^{-4/5}$. Поэтому в (9.19) $c(N) = \sqrt{k} = \sqrt{N^{4/5}} = N^{0.4}$, что меньше $N^{1/2}$. Другие значения $k$ также обеспечивают состоятельность при условии $k < O(N)$. 
 
В общем случае диапазон значений параметра ширины окна устраняет асимптотическое смещение, но небольшие значения ширины окна увеличивает разброс. В данной книге выбор между этими двумя показателями делается с помощью минимизации среднеквадратической ошибки, которая равна сумме дисперсии и квадрата смещения.

Стоун (1980) показал, что если $x$ является $k$-мерным и $m(x)$ дифференцируема $p$ раз, то наибольшая возможная скорость сходимости непараметрической оценки с производной $m(x)$ порядка $s$ равна $N^{-r}$, где $r = (p - s)/(2p + k)$. Эта скорость снижается по мере увеличения порядка производной, а также по мере увеличения размерности $x$. Она увеличивается, чем больше производных существует у $m(x)$, приближаясь к $N^{-1/2}$, если  $m(x)$ имеет производные порядка, близкого к бесконечности. Для получения скалярной оценки $m(x)$ в регрессии принято предполагать существование $m''(x)$. В этом случае $r = 2/5$ и наибольная скорость сходимости равна $N^{-0.4}$.

\section{Ядерная регрессия}

Ядерная регрессия является средневзвешенной оценкой с использованием ядерных весов. Такие проблемы, как смещённость и выбор ширины окна, представленные для ядерной оценки плотности, релевантны и в этом случае. Однако в случае ядерной оценки плотности существует меньше рекомендаций для выбора ширины окна, чем в случае регрессии. Кроме того, хотя мы описываем ядерную регрессию в педагогических целях, ядерные оценки локальной регрессии часто используются на практике (см. раздел 9.6).

\subsection{Оценка ядерной регрессии}

Задача ядерной регрессии состоит в том, чтобы оценить регрессионную функцию $m(x)$ в модели $y = m(x) + \e$ из (9.15).

Из раздела 9.4.1, очевидная оценка $m(x_0)$ --- среднее выборочных значений зависимой переменной $y_i$, соответствующих $x_i$, близким к $x_0$. Другой вариант --- найти среднее $y_i$ для всех наблюдений, для которых $x_i$ находится в пределах расстояния $h$ от $x_0$. Это можно выразить как
\[
\hat{m}(x_0) = \frac{\sum_{i=1}^N {\bf{1}} \left( \left. |\frac{x_i - x_0}{h}| \right. < 1 \right)y_i}{\sum_{i=1}^N {\bf{1}} \left( \left. | \frac{x_i - x_0}{h}| \right. < 1 \right)},
\]
где ${\bf{1}}(A)$ равно единице, если событие $A$ происходит, и равно нулю в противном случае. В числителе суммируются значения $y$, а знаменатель отражает количество $y$, которые суммируются. В этом выражении все наблюдения, близкие к $x_0$, имеют равный вес. Однако может быть лучше присваивать наибольший вес в $x_0$ и уменьшать вес при удалении от $x_0$. Таким образом, мы рассматриваем ядерную взвешивающую функцию $K(\cdot)$, которая была введена в разделе 9.3.2. Мы получаем ядерную оценку регрессии
\begin{equation}
\hat{m}(x_0) = \frac{\frac{1}{Nh}\sum_{i=1}^N {\bf{K}} \left( \frac{x_i - x_0}{h} \right)y_i}{\frac{1}{Nh}\sum_{i=1}^N {\bf{K}} \left( \frac{x_i - x_0}{h} \right)}.
\end{equation}
Несколько наиболее часто используемых ядерных функций --- равномерное, Гауссово, Епанечникова и квартичное --- уже были приведены в таблице 9.1.

Константа $h$ --- ширина окна, и $2h$ --- удвоенная ширина окна. Ширина окна играет ту же роль, что и $k$ в $k - NN$ примере из раздела 9.4.2.

Оценка (9.21) была предложена Надарайа (1964) и Уотсоном (1964), которые привели альтернативный вывод. Условное математическое ожидание равно $m(x) = \int yf(y|x)dy = \int y[f(y,x)/f(x)]dy$, его можно оценить с помощью $\hat{m}(y) = \int y[\hat{f}(y,x)/\hat{f}(x)]dy$, где $\hat{f}(y,x)$ и $\hat{f}(x)$ являются двумерной и одномерной ядерными оценками плотности. Можно показать, что это равняется оценке из (9.21). Литература по статистике также рассматривает ядерную регрессию в случае фиксированных регрессоров, где $f(x)$ известна и её не нужно оценивать. Однако мы рассматриваем только случай стохастических регрессоров, который возникает при наблюдаемых данных.

Ядерная оценка регрессии является частным случаем средневзвешенной (9.16) с весами
\begin{equation}
w_{i0,h} = \frac{\frac{1}{Nh}{\bf{K}} \left( \frac{x_i - x_0}{h} \right)}{\frac{1}{Nh}\sum_{i=1}^N {\bf{K}} \left( \frac{x_i - x_0}{h} \right)},
\end{equation}
сумма которых равна единице. Общие результаты раздела 9.4 релевантны, но мы приводим более подробный анализ. 

\subsection{Статистические выводы}

Мы приводим распределение ядерной оценки регрессии $\hat{m}(x)$ при заданных $K(\cdot)$ и $h$, предполагая, что данные $x$ независимы и одинаково распределены. Мы неявно предполагаем, что регрессоры непрерывны. При дискретных регрессорах $\hat{m}(x_0)$ будет по-прежнему стремиться к $m(x_0)$, функция $\hat{m}(x_0)$ в пределе, и функция $m(x_0)$ являются ступенчатыми.

\begin{center}
Состоятельность
\end{center}

Состоятельность $\hat{m}(x_0)$ для функции условного математического ожидания $m(x_0)$ требует, чтобы $h \rightarrow 0$. Таким образом, больший вес присваивается только $x_i$, очень близким к $x_0$. В то же время нам необходимо много $x_i$, близких к $x_0$, чтобы при расчёте средневзвешенной использовалось много наблюдений. Формально $\hat{m}(x_0) \stackrel{p}{\rightarrow} m(x_0)$, если $h \rightarrow 0$ и $Nh \rightarrow \infty$, так как $N \rightarrow \infty$.

\begin{center}
Смещение
\end{center}

Ядерная оценка регрессии смещена на $O(h^2)$ с параметром смещения
\begin{equation}
b(x_0) = h^2 \left( m'(x_0)\frac{f'(x_0)}{f(x_0)} + \frac{1}{2} m''(x_0) \right) \int z^2K(z)dz
\end{equation}
(см. раздел 9.8.2), предполагая, что $m(x)$ дважды дифференцируема. Что касается ядерной оценки плотности, то смещение варьируется в зависимости от используемой ядерной функции. Более того, смещение зависит от наклона и кривизны регрессионной функции $m(x_0)$ и наклона плотности $f(x_0)$ регрессоров, тогда как для оценки плотности смещение зависело только от второй производной $f(x_0)$. Смещение может быть особенно большим в краевых точках, как показано в разделе 9.4.2.

Можно уменьшить смещение, используя ядра более высокого порядка, которые определены в разделе 9.3.3, и модификации для оценки регрессии на границе, например, специальные граничные ядра. Локальная полиномиальная регрессия и такие модификации, как $LOWESS$ (см. раздел 9.6.2), привлекательны, так как показатель из (9.23), зависящий от $m'(x_0)$, сокращается. Также они дают хорошие результаты для краевых значений.

\begin{center}
Асимптотическая нормальность
\end{center}

В разделе 9.8.2 показано, что для $x_i$, которые независимы и одинаково распределены с плотностью $f(x_i)$, ядерная оценка регрессии имеет предельное распределение 
\begin{equation}
\sqrt{Nh}(\hat{m}(x_0) - m(x_0) - b(x_0)) \stackrel{d}{\rightarrow} \mathcal{N}\left[ 0, \frac{\sigma_{\e}^2}{f(x_0)} \int K(z)^2dz \right].
\end{equation}
Дисперсия из (9.24) больше для малых $f(x_0)$. Таким образом, как и ожидалось, дисперсия $\hat{m}(x_0)$ больше в областях, в которые попадает мало значений $x$.

\subsection{Выбор ширины окна}

Включение значений $y_i$, для которых $x_i \not= x_0$, в средневзвешенную приводит к появлению смещения, так как $\E[y_i|x_i] = m(x_i) \not= m(x_0)$ для $x_i \not= x_0$. Тем не менее, использование этих дополнительных точек уменьшает дисперсию оценки, так как мы усредняем по большему числу данных. Оптимальная ширина окна балансирует между увеличением смещения и снижением дисперсии, используя квадрат ошибок потерь. В отличие от ядерного оценивания плотности простые методы оценки ширины окна являются непрактичными, и поэтому более широко используется кросс-валидация.

Для простоты большинство исследований сосредоточено на выборе единой ширины окна для всех значений $x_0$. Существуют методы с меняющимися значениями ширины окна, в частности, $k - NN$ и $LOWESS$, приведены в разделе 9.6.

\begin{center}
Cредняя интегрированная квадратическая ошибка
\end{center} 

Локальное качество оценивания $\hat{m}(\cdot)$ в точке $x_0$ измеряется с помощью среднеквадратической ошибки, которая имеет вид:
\[
MSE[\hat{m}(x_0)] = \E[(\hat{m}(x_0) - m(x_0))^2],
\]
где математическое ожидание устраняет зависимость $\hat{m}(x_0)$ от $x$. Так как $MSE$ равняется сумме дисперсии и квадрата смещения, $MSE$ можно получить, используя (9.23) и (9.24).

Аналогично разделу 9.3.6 интегрированная квадратическая ошибка $(ISE)$ имеет вид:
\[
ISE(h) = \int (\hat{m}(x_0) - m(x_0))^2f(x_0) dx_0,
\]
где $f(x)$ обозначает плотность регрессоров $x$. Тогда средняя интегрированная квадратическая ошибка или интегрированная среднеквадратическая ошибка равна
\[
MISE(h) = \int MSE[\hat{m}(x_0)]f(x_0)dx_0.
\]

\begin{center}
Оптимальная ширина окна
\end{center}

Оптимальная ширина окна $h^*$ минимизирует $MISE(h)$. Мы получаем $h^* = O(N^{-0.2})$, поскольку смещение имеет порядок малости $O(h^2)$ из (9.23). Дисперсия имеет порядок малости $O((Nh)^{-1})$ из (9.24), так как можно получить дисперсию $O(1)$ после деления $\hat{m}(x_0)$ на $\sqrt{Nh}$. Для того чтобы квадрат смещения и дисперсия были одного и того же порядка, $(h^2)^2 = (Nh)^{-1}$ или $h = N^{-0.2}$. В этом случае ядерная оценка сходится к $m(x_0)$ со скоростью $(Nh^*)^{-1/2} = N^{-0.4}$, а не со стандартной скоростью $N^{-0.5}$ для параметрического анализа.

\begin{center}
Оценка ширины окна
\end{center}

Можно получить точное выражение для $h^*$, который минимизирует $MISE(h)$, используя методы расчёта, аналогичные методам для оценки плотности ядра из раздела 9.3.5. Тогда $h^*$ зависит от смещения и дисперсии из (9.23) и (9.24).

Метод подстановки предлагает вычислять $h^*$ с использованием оценок этих неизвестных. Тем не менее, оценивание $m''(x)$, например, требует непараметрических методов, которые, в свою очередь, требуют первоначального выбора ширины окна. Однако $h^*$ также зависит от таких неизвестных, как $m''(x)$. Эти усложнения следует принимать во внимании при оценивании методом подстановки. Более распространённым способом является применение кросс-валидации, которая представлена далее.

Также можно показать, что $MISE(h^*)$ минимизируется, если используется ядро Епанечникова (см. Хэрдл, 1990, стр. 186, или Хэрдл и Линтон, 1994, стр. 2321). Хотя, как и в случае ядерной регрессии, $MISE(h^*)$ не намного больше для других ядер. Ключевым вопросом является определение $h^*$, которое зависит от ядра и данных.

\begin{center}
Кросс-валидация
\end{center}

Эмпирическую оценку оптимальной $h$ можно получить с помощью  кросс-валидации с отбрасыванием отдельных наблюдений. Она выбирает такую $h^*$, которая минимизирует
\begin{equation}
CV(h) = \sum_{i=1}^N (y_i - \hat{m}_{-i}(x_i))^2\pi(x_i),
\end{equation}
где $\pi(x_i)$ --- взвешивающая функция (она будет рассмотрена далее) и 
\begin{equation}
\hat{m}_{-i}(x_i) = \sum_{j \not= i} w_{ji,h} y_j/\sum_{j \not= i} w_{ji,h}.
\end{equation}
$\hat{m}_{-i}(x_i)$ --- оценка $m(x_i)$ полученная при удалении одного наблюдения с помощью ядерной формулы (9.21) или в более общем случае с помощью взвешивающей процедуры (9.16) с модификацией, что $y_i$ не используется.

Кросс-валидация не требует такого большого числа вычислений, как кажется с первого взгляда. Можно показать, что
\begin{equation}
y_i - \hat{m}_{-i}(x_i) = \frac{y_i - \hat{m}(x_i)}{1 - [w_{ii,h}/\sum_j w_{ji,h}]}.
\end{equation}

Таким образом, для каждого значения $h$ кросс-валидация требует только расчёта средневзвешенной $\hat{m}(x_i)$, $i = 1, \dots, N$.

Веса $\pi(x_i)$ введены для того, чтобы уменьшить вес краевых точек, которые в противном случае могут получить слишком большое значение. Локальные взвешенные оценки могут быть сильно смещёнными в краевых точках, как было показано в разделе 9.4.2. Например, наблюдения $x_i$, которые находятся за пределами промежутка с 5-го по 95-й процентный квантиль, могут быть не использованы при расчёте $CV(h)$. В этом случае $\pi(x_i) = 0$ для этих наблюдений, и $\pi(x_i) = 0$ в противном случае. Термин кросс-валидация используется, так как он проверяет способность прогнозировать $i$-ое наблюдение, используя все остальные наблюдения в наборе данных. Само $i$-ое наблюдение не используется, потому что если бы оно было дополнительно использовано в прогнозе, то $CV(h)$ минимизировалось бы просто при $\hat{m}_h(x_i) = y_i$, $i = 1, \dots, N$. Величину $CV(h)$ также называют оценённой ошибкой прогнозирования.

Хэрдл и Маррон (1985) показали, что минимизация $CV(h)$ асимптотически эквивалентна минимизации модификации $ISE(h)$ и $MISE(h)$. Модификация включает взвешивающую функцию $\pi(x_0)$ под знаком интеграла, а также усреднённую квадратическую ошибку $(ASE)$ $N^{-1}\sum_i (\hat{m}(x_i) - m(x_i))^2 \pi(x_i)$, которая является дискретной выборочной аппроксимацией $ISE(h)$. Показатель $CV(h)$ сходится с медленной скоростью $O(N^{-0.1})$, поэтому $CV(h)$ может сильно варьироваться в малых выборках.

\begin{center}
Обобщённая кросс-валидация
\end{center}

Альтернативой поэлементной кросс-валидации может быть использование показателя, аналогичного $CV(h)$. Но он использует $\hat{m}(x_i)$ вместо $\hat{m}_{-i}(x_i)$, а также вводит штраф за усложнение модели, который увеличивается при уменьшении ширины окна $h$. Это приводит к 
\[
PV(h) = \sum_{i=1}^N (y_i - \hat{m}(x_i))^2\pi(x_i)p(w_{ii,h}),
\]
где $p(\cdot)$ --- функция штрафа, и $w_{ii,h}$ --- вес, который присваивается $i$-тому наблюдению в $\hat{m}(x_i) = \sum_j w_{ji,h} y_j$.

Распространённый пример --- обобщённая кросс-валидация, которая использует функцию штрафа $p(w_{ii,h}) = (1 - w_{ii,h})^2$. Другие варианты функции представлены у Хэрдла (1990, стр. 167), а также у Хэрдла и Линтона (1994, стр. 2323).

\begin{center}
Пример кросс-валидации
\end{center}

Для примера локальной скользящей средней из раздела 9.4.2 , $CV(k) =$ 54 811, 56 666, 63 456, 65 605 и 69 939 для $k = $ 3, 5, 7, 9 и 25 соответственно. В этом случае для расчёта $CV(k)$ были использованы все наблюдения с $\pi(x_i) = 1$, несмотря на возможные проблемы с краевыми точками. Там нет  реального выигрыша после $k = 5$, хотя на графике 9.5 это значение даёт слишком негладкую оценку, и на практике можно выбрать большую величину $k$, чтобы получить более гладкую кривую.

В более общем случае кросс-валидация не является идеальным способом. Зачастую, чтобы достичь желаемую степень гладкости для выбора $h$, выбор из оценённых непараметрических кривых осуществляется на глаз.

\begin{center}
Усечение
\end{center}

Знаменатель ядерной оценки из (9.21) --- это $\hat{f}(x_0)$, ядерная оценка плотности регрессора в $x_0$. В некоторых оцениваемых точках $\hat{f}(x_i)$ может принимать очень маленькие значения, что приводит к очень большой оценке $\hat{m}(x_i)$. Усечение устраняет или значительно уменьшает вес значений, для которых $\hat{f}(x_i) < b$, где, например, $b \rightarrow 0$ с подходящей скоростью при $N \rightarrow \infty$. Такие проблемы, скорее всего, могут возникнуть в хвостах распределения. Для непараметрического оценивания можно сосредоточиться на оценивании $m(x_i)$ для более близких к центру значений $x_i$, а вес значений в хвостах можно уменьшить при кросс-валидации. Тем не менее, полупараметрические методы из раздела 9.7 могут потребовать вычисление $\hat{m}(x_i)$ для всех значений $x_i$. В этом случае часто применяют усечение. В идеальном случае использование усечения не должно приводить к другим результатам асимптотически, хотя на малых выборках результаты могут отличаться.

\subsection{Доверительные интервалы}

Оценки ядерной регрессии в общем случае рекомендуется предъявлять вместе с поточечными доверительными интервалами. Простая процедура заключается в представлении поточечных доверительных интервалов для $f(x_0)$, например, во всех децилях $x_0$ с первого по девятый дециль $x$.

Если смещение $b(x_0)$ в $\hat{m}(x_0)$ игнорируется, то (9.24) приводит к следующему 95\% доверительному интервалу:
\[
m(x_0) \in \hat{m}(x_0) \pm 1.96 \sqrt{\frac{1}{Nh} \frac{\hat{\sigma}_{\e}^2}{\hat{f}(x_0)} \int K(z)^2dz},
\]
где $\hat{\sigma}_{\e}^2 = \sum_i w_{i0,h} \hat{\e}_i^2$ и $ w_{i0,h}$ из (9.22), и $\hat{f}(x_0)$ --- ядерная оценка плотности в точке $x_0$. Эта оценка предполагает гомоскедастичные ошибки, хотя, скорее всего, она устойчива к гетероскедастичности, так как наблюдениям, близким к $x_0$, придаётся наибольший вес. Альтернативно из рассуждений  после (9.20) можно получить 95\% доверительный интервал с поправкой на гетероскедастичность: $\hat{m}(x_0) \pm 1.96\hat{s}_0$, где $\hat{s}_0^2 = \sum_i w_{i0,h}^2 \hat{\e}_i^2$

Как и в случае ядерной плотности, смещение в $\hat{m}(x_0)$ нельзя игнорировать. Как уже отмечалось, оценивание смещения --- сложная задача. Здесь стандартная процедура --- недосглаживание с меньшей шириной окна $h$, где $h = o(N^{-0.2})$, а не оптимальной $h^* = O(N^{-0.2})$.

Хэрдл (1990) даёт подробное описание доверительных интервалов, в том числе полос постоянной ширины, а не точечных интервалов, и метода бутстрэп, который рассмотрен в разделе 11.6.5.

\subsection{Оценивание производных}

В регрессии нас часто интересует, как меняется условное математическое ожидание $y$ при изменении $x$, предельный эффект, а не условное математическое ожидание само по себе.

Ядерные оценки можно легко использовать для формирования производной. Общим результатом является то, что $s$-тая
производная ядерной оценки регрессии, $\hat{m}^{(s)}(x_0)$, состоятельна для $m^{(s)}(x_0)$, $s$-той производной условного математического ожидания $m(x_0)$. Можно применять либо дифференциирование, либо конечно-разностные методы.

В качестве примера рассмотрим оценивание первой производной для примера сгенерированных данных из предыдущего раздела. Пусть $z_1, \dots, z_N$ обозначают упорядоченные точки, в которых оценивается ядерная функция регрессии, а $\hat{m}(z_1), \dots, \hat{m}(z_N)$ обозначают оценки в этих точках. Конечно-разностная оценка --- это $\hat{m}'(z_i) = [\hat{m}(z_i) - \hat{m}(z_{i-1})]/[z_i - z_{i-1}]$. Это представлено на графике 9.7 вместе с истинной производной, которая для процесса, порождающего данные,  из (9.17) является квадратичной $m'(z_i) = 6.5 - 0.30z_i + 0.003z_i^2$. Как и ожидалось, оценка производной имеет шумы, но она отражает основные особенности. Оценки производных должны основываться на сверхсглаженных оценках условного математического ожидания. Более подробное описание представлено у Пагана и Улла (1999, глава 4). Хэрдл (1990, стр. 160) представляет вариант кросс-валидации для оценивания производных.

В дополнение к производной в точке $m'(x_0)$ мы также можем быть заинтересованы в оценивании математического ожидания $\E[m'(x)]$. Средняя оценка производной, приведённая в разделе 9.7.4, даёт $\sqrt{N}$ состоятельную и асимптотически нормальную оценку $\E[m'(x)]$.
 
\subsection{Оценивание условного момента}

Методы ядерной регрессии для условного математического ожидания $\E[y|x] = m(x)$ можно расширить и на непараметрическое оценивание других условных моментов.

\vspace{5cm}

График 9.7: Непараметрические оценки производной, одна получена с помощью  оценённой ранее $LOWESS$ регрессионной кривой, другая --- с помощью кубической регрессии. Используются те же данные, что и для графика 9.5.

Для условных моментов таких, как $\E[y^k|x]$ мы используем взвешенную среднюю
\begin{equation}
\hat{\E}[y^k|x_0] = \sum_{i=1}^N w_{i0,h}y_i^k,
\end{equation}
где веса $w_{i0,h}$ могут быть теми же весами, что и веса, использованные для оценивания $m(x_0)$. 

Центральные условные моменты можно вычислить, выразив их как взвешенные суммы моментов. Например, так как $\V[y|x] = \E[y^2|x] - (\E[y|x])^2$, оценкой условной дисперсии может быть $\hat{\E}[y^2|x_0] - \hat{m}(x_0)^2$. Можно ожидать, что условные моменты более высокого порядка будут оценены с большим шумом, чем оценка условного математического ожидания.

\subsection{Многомерная ядерная регрессия}

Мы рассматривали ядерную регрессию с одним регрессором. Для регрессии скалярного $y$ на $k$-мерный вектор $x$, то есть $y_i = m(x_i) + \e_i = m(x_{1i}, \dots, x_{ki}) + \e_i$, ядерная оценка регрессии становится
\[
\hat{m}(x_0) = \frac{\frac{1}{Nh^k}\sum_{i=1}^N {\bf{K}} \left( \frac{x_i - x_0}{h} \right)y_i}{\frac{1}{Nh^k}\sum_{i=1}^N {\bf{K}} \left( \frac{x_i - x_0}{h} \right)},
\]
где в данном случае $K(\cdot)$ --- многомерное ядро. Часто $K(\cdot)$ обозначает произведение $k$ одномерных ядер, хотя можно использовать многомерные ядра такие, как многомерная плотность нормального распределения.

Если используется ядро-произведение, регрессоры должны быть приведены к общей шкале путём деления на стандартное отклонение. В таком случае для определения стандартной оптимальной ширины окна $h^*$ можно использовать кросс-валидацию (9.25). Хотя, когда $x$ многомерный, сложнее определить, каким $x$ следует придать меньший вес из-за близости к краевым точкам. Есть другой способ, в котором нет необходимости масштабировать регрессоры. Однако тогда для каждого регрессора надо использовать различную ширину окна.

Асимптотические результаты и выражения аналогичны тем, что были рассмотрены ранее, так как оценка снова является локальным средним $y_i$. Смещение $b(x_0)$ снова имеет порядок малости $O(h^2)$, как и раньше. Дисперсия $\hat{m}(x_0)$ убывает со скоростью $O(Nh^k)$, медленнее, чем в одномерном случае, так как существенно меньшая часть выборки используется для формирования $\hat{m}(x_0)$. Тогда
\[
\sqrt{Nh^k}(\hat{m}(x_0) - m(x_0) - b(x_0)) \stackrel{d}{\rightarrow} \mathcal{N}\left[ 0, \frac{\sigma_{\e}^2}{f(x_0)}\int K(z)^2dz \right].
\]

Оптимальная ширина окна  $h^* = O(N^{-1/(k+4)})$ больше, чем $O(N^{-0.2})$ в одномерном случае. Соответствующая оптимальная скорость сходимости $\hat{m}(x_0)$ равна $N^{-2/(k+4)}$.

Этот результат и приведённый ранее результат для скалярного случая предполагают, что $m(x)$ дважды дифференцируема --- это необходимая предпосылка для получения величины смещения из (9.23). Если вместо этого $m(x)$ дифференцируема $p$ раз, то ядерное оценивание с использованием ядра $p$-того порядка (см. раздел 9.3.3) уменьшает величину смещения. Это приводит к меньшей $h^*$ и более высокой скорости сходимости, которые достигают границу Стоуна, которая приведена в разделе 9.4.5, дополнительное описание можно посмотреть у Хэрдла (1990, стр. 93). Другие непараметрические оценки, приведённые в следующем разделе, также могут достичь границу Стоуна.

Скорость сходимости уменьшается по мере увеличения числа регрессоров, приближаясь к $N^0$, когда число регрессоров стремится к бесконечности. Это проклятие размерности сильно ограничивает использование непараметрических методов в регрессионных моделях с несколькими регрессорами. Полупараметрические модели (см. раздел 9.7) накладывают дополнительные ограничения так, чтобы непараметрические компоненты имели малую размерность.

\subsection{Тесты для параметрических моделей} 

Очевидный тест на верную спецификацию параметрической модели условного математического ожидания --- сравнить оценённое математическое ожидание с полученным математическим ожиданием из непараметрической модели.

Пусть $\hat{m}_{\theta}(x)$ --- параметрическая оценка $\E[y|x]$, $\hat{m}_h(x)$ --- непараметрическая оценка такая, как  ядерная оценка. Один подход состоит в том, чтобы сравнить  $\hat{m}_{\theta}(x)$ и $\hat{m}_h(x)$ для ряда значений $x$. Это сложно из-за того, что необходимо делать поправку на смещение в $\hat{m}_h(x)$ (см. Хэрдл и Маммен, 1993). Второй подход состоит в том, чтобы проверять тесты на условный момент вида $N^{-1}\sum_i w_i(y_i - \hat{m}_{\theta}(x))$, где различные веса, основанные на ядерной регрессии, проверяют условие $\E[y|x] = m_{\theta}(x)$. Например, Хоровиц и Хэрдл (1994) используют $w_i = \hat{m}_h(x_i) - \hat{m}_{\theta}(x_i)$. Паган и Улл (1999, стр. 141-150) и Ячью (2003, стр. 119-124) рассматривают некоторые широко используемые методы.

\section{Некоторые альтернативные непараметрические оценки регрессии}

В разделе 9.4 описаны методы локальной регрессии, которые оценивают регрессионную функцию $m(x_0)$ с помощью локальной средневзешенной $\hat{m}(x_0) = \sum_i w_{i0,h}y_i$, где веса $w_{i0,h} = w_i(x_i, x_0, h)$ имеют различные оцениваемые точки $x_0$ и различные выборочные значения $x_i$. В разделе 9.5 представлены подробные результаты в случае ядерных весов.

Здесь мы рассматриваем часто используемые локальные оценки, для которых используются другие веса. Многие результаты раздела 9.5 остаются в силе с похожей оптимальной скоростью сходимости, также используется кросс-валидация для выбора ширины окна, хотя точные выражения для смещения и дисперсии отличаются от тех, что были даны в (9.23) и (9.24). Оценки, которые приведены в разделе 9.6.2, наиболее широко распространены.

\subsection{Оценка ближайших соседей}

Оценка $k$ ближайших соседей --- простое среднее арифметическое $y$ для $k$ наблюдений $x_i$, которые ближе всего к $x_0$. Пусть $N_k(x_0)$ --- множество, состоящее из $k$ наблюдений $x_i$, которые ближе всего к $x_0$. Тогда
\begin{equation}
\hat{m}_{k - NN}(x_0) = \frac{1}{k} \sum_{i=1}^N {\bf{1}}(x_i \in N_k(x_0))y_i.
\end{equation}
Эта оценка --- ядерная оценка с равномерными весами (см. таблицу 9.1). Однако в этом случае ширина окна изменяется. Здесь ширина окна $h_0$ в $x_0$ равняется расстоянию между $x_0$ и самым дальним из $k$ соседей, и более формально $h_0 \simeq k/(2Nf(x_0))$. Отношение $k/N$ называется размахом. Можно получить более гладкие кривые, используя ядерные веса из (9.29).

Достоинство данной оценки в том, что она даёт простое правило выбора ширины окна. Для ускорения вычислений рассчитывают симметризованную версию, которая использует $k/2$ ближних соседей слева и то же самое количество справа, как в методе локальной скользящей средней из раздела 9.4.2. Тогда можно использовать обновлённую версию формулы на отсортированных по увеличению $x_i$ наблюдениях, в которой одно наблюдение исключается и одно включается при увеличении $x_0$.  

\subsection{Локальная линейная регрессия и $LOWESS$}

Ядерная оценка регрессии --- локальная постоянная оценка, потому что предполагается, что $m(x)$ равняется константе в локальной окрестности $x_0$. Вместо этого можно задать $m(x)$ так, чтобы она была линейной в окрестности $x_0$, то есть $m(x_0) = a_0 + b_0(x - x_0)$.

Чтобы применить эту идею, заметим, что можно получить ядерную оценку регрессии $\hat{m}(x_0)$ с помощью минимизации $\sum_i K((x_i - x_0)/h)(y_i - m_0)^2$ по $m_0$. Оценка локальной линейной регрессии минимизирует 
\begin{equation}
\sum_{i=1}^N K \left( \frac{x_i - x_0}{h} \right) (y_i - a_0 - b_0(x_i - x_0))^2,
\end{equation}
по $a_0$ и $b_0$, где $K(\cdot)$ --- ядерная взвешивающая функция. Тогда $\hat{m}(x) = \hat{a}_0 + \hat{b}_0(x - x_0)$ в окрестности $x_0$. Оценка в точке $x_0$ тогда имеет вид $\hat{m}(x) = \hat{a}_0$, и $\hat{b}_0$ --- это оценка первой производной $\hat{m}'(x_0)$. В более общем случае локальная полиномиальная оценка степени $p$ минимизирует
\begin{equation}
\sum_{i=1}^N K \left( \frac{x_i - x_0}{h} \right) (y_i - a_{0,0} - a_{0,1}(x_i - x_0) - \dots - a_{0,p} \frac{(x_i - x_0)^p}{p!})^2,
\end{equation}
что даёт $\hat{m}^{(s)}(x_0) = \hat{a}_{0,s}$. 

Фан и Гайбельс (1996) указывают много свойств и достоинств этого метода. Оценивание подразумевает только взвешенную МНК регрессию в каждой оцениваемой точке $x_0$. Оценки можно выразить как средневзвешенное $y_i$, так как они являются оценками метода наименьших квадратов. Локальная линейная оценка имеет параметр смещения $b(x_0) = h^2(\frac{1}{2}m''(x_0))\int z^2 K(z)dz$, который, в отличие от смещения ядерной регрессии в (9.23), не зависит от $m'(x_0)$. Это особенно полезно при решении проблем с границами, которые описаны в разделе 9.4.2. Для оценивания производной $s$-того порядка хороший выбор $p$ -- это $p = s + 1$. Так, например, можно использовать локальную квадратичную оценку, чтобы оценить первую производную.

Распространённая локальная оценка регрессии --- локально взвешенное сглаживание диаграммы рассеяния  (locally weighted scatterplot smoothing, $LOWESS$) или оценка Кливленда (1979). Это вариант локального полиномиального оценивания из (9.31), который использует переменную ширины окна $h_{0,k}$. Она определяется расстоянием между $x_0$ и $k$-тым ближайшим соседом. В данном случае используется трикубическое ядро $K(z) = (70/81)(1 - |z|^3)^3{\bf{1}}(|z| < 1)$, при этом наблюдениям с большими остатками $y_i - \hat{m}(x_i)$ придаётся меньший вес. Для вычисления оценки приходится обсчитать данные $N$ раз. Обзор можно посмотреть у Фана и Гайбельса (1996, стр. 24). $LOWESS$ привлекательнее по сравнению с ядерной регрессией, так как этот метод использует локальную полиномиальную оценку, делает поправку на выбросы, а также использует локальную полиномиальную оценку, чтобы решить проблемы с границами. Однако это требует большого количества вычислений.

Другой популярный вариант --- супер-сглаживатель Фридмана (1984) (см. Хэрдл, 1990, стр. 181). Отправная точка --- симметризованная оценка $k - NN$. В данном случае для лучшего оценивания на границах используется локальное линейное, а не локальное постоянное оцениваниe. Вместо того, чтобы использовать фиксированный размах или фиксированные $k$, супер-сглаживатель --- сглаживатель с переменной шириной окна, которая определяется с помощью локальной кросс-валидации, для чего необходимо девять обсчётов данных. По сравнению с $LOWESS$ супер-сглаживатель не так устойчив к выбросам, но он использует переменную ширину окна. Более того, этот метод лёгок для расчётов.

\subsection{Оценка сглаженных сплайнов}

Кубическая оценка сглаженных сплайнов $\hat{m}_{\lambda}(x)$ минимизирует сумму квадратов остатков со штрафом
\begin{equation}
PRSS(\lambda) = \sum_{i=1}^N (y_i - m(x_i))^2 + \lambda \int (m''(x))^2dx,
\end{equation}
где $\lambda$ --- параметр сглаживания. В этой главе так же, как и в других, используется квадрат ошибок потерь. Первый член в отдельности приводит к недосглаженным результатам оценивания, так как тогда $\hat{m}(x_i) = y_i$. Второй член вводится, чтобы штрафовать за недосглаживание. Методы кросс-валидации из раздела 9.5.3 могут быть использованы, чтобы определить $\lambda$. При больших значениях $\lambda$ кривая будет более гладкая.

Хэрдл (1990, стр. 56-65) показывает, что $\hat{m}_{\lambda}(x)$ является кубическим полиномом между последовательными значениями $x$. Также он показывает, что можно представить оценку как локальное средневзвешенное $y$-ков и что она асимптотически эквивалентна ядерной оценке с определённым меняющимся ядром. В микроэконометрике сглаживающие сплайны используются реже, чем другие методы, которые представлены здесь. Этот подход может быть адаптирован для других штрафов за недосглаживание и для других функций потерь.

\subsection{Оценки с разложением в ряд}

Оценки с разложением в ряд аппроксимируют регрессионную функцию с помощью взвешенной суммы $K$ функций $z_1(x), \dots, z_K(x)$,
\begin{equation}
\hat{m}_k(x) = \sum_{j=1}^K \hat{\beta}_j z_j(x), 
\end{equation}
где коэффициенты $\hat{\beta}_1, \dots, \hat{\beta}_K$ получены с помощью МНК-регрессии $y$ на $z_1(x), \dots, z_K(x)$. Функции $z_1(x), \dots, z_K(x)$ образуют усечённые серии. Примеры включают в себя полиномиальную аппроксимацию $(K - 1)$-ого порядка или степенной ряд с $z_j(x) = x^{j-1}$, $j = 1, \dots, K$; ортогональные и ортонормированные варианты многочлена (см. раздел 12.3.1); усечённые ряды Фурье, где регрессор масштабируется так, чтобы $x \in [0,2\pi]$; гибкие функциональные формы Галланта для рядов Фурье (1981), которые представляют собой усечённые ряды Фурье c членами $x$ и $x^2$; сплайн регрессии, которые аппроксимируют регрессионную функцию $m(x)$ с помощью полиномиальных функций на заданном числе узлов.

Этот метод отличается от того, который был представлен в разделе 9.4. Он представляет собой подход, который заключается в глобальной аппроксимации оценки $m(x)$, а не локальной аппроксимации $m(x_0)$. Тем не менее, $\hat{m}_K(x) \stackrel{p}{\rightarrow} m(x_0)$ при $K \rightarrow \infty$ c соответствующей скоростью при $N \rightarrow \infty$. Ньюи (1997) показал, что если $x$ является $k$-мерным и $m(x)$ дифференцируема $p$ раз, средняя интегрированная квадратическая ошибка (см. раздел 9.5.3) $MISE(h) = O(K^{-2p/k} + K/N)$, где первый член отражает смещение, а второй --- дисперсию. Отсюда оптимальное $K^* = N^{k/(2p+k)}$. Таким образом, $K$ растёт с более медленным темпом, чем размер выборки. Скорость сходимости $\hat{m}_{K^*}(x)$ равна максимально возможной скорости Стоуна (1980), которая приведена в разделе 9.4.5. Логично, что оценки с разложением в ряд могут быть неустойчивы к выборосам, так как выбросы могут оказывать глобальное, а не только локальное влияние на $\hat{m}(x)$. Однако эта гипотеза не проверяется  в стандартных примерах, приводимых в учебниках.

Эндриус (1991) и Ньюи (1997) приводят хорошее общее описание, которое включает в себя многомерный случай, оценивание функционалов, отличных от условного математического ожидания, и расширение для полупараметрических моделей, где часто используется оценивание с разложением в ряд.

\section{Полупараметрическая регрессия}

Предшествующий анализ был сконцентрирован на регрессионных моделях, не имеющих структуры. В микроэконометрике, как правило, на регрессионную модель накладывается определённая структура.

Во-первых, экономическая теория накладывает некоторую структуру на функции спроса, например, ограничения на симметрию и однородность. Эту информацию можно включить в непараметрическую регрессию; см., например, Матцкин (1994).

Во-вторых, что случается довольно часто, эконометрические модели включают так много потенциальных регрессоров, что проклятие размерности делает полностью непараметрический анализ непрактичным. Вместо этого часто оценивают полупараметрические модели, которые содержат как параметрическую компоненту, так и непараметрическую компоненту; см. Пауэлл (1994), у которого приведено подробное описание термина полупараметрический.

Существует много различных полупараметрических моделей, и часто доступно бесчисленное число методов для получения состоятельных оценок этих моделей. В этом разделе мы приводим лишь несколько основных примеров. В этой книге также рассмотрены приложения, в том числе модели бинарного выбора и цензурированные регрессионные модели, представленные в главах 14 и 16.

\subsection{Примеры}

Таблица 9.2 отражает несколько основных примеры полупараметрических регрессий. Первые два примера, которые будут подробно описаны далее, являются обобщением $x'\beta$ из линейной модели с помощью добавления неспецифицированной компоненты $\lambda(z)$ или с помощью неспецифицированного преобразования $g(x'\beta)$. В то же время третий пример сочетает в себе первые два. Следующие три модели  используются в прикладной статистике чаще, чем в эконометрике. Они уменьшают размерность, предполагая аддитивность или сепарабельность регрессоров, а в остальном являются непараметрическими. Мы рассмотрим обобщённую аддитивную модель. С этими моделями связаны модели нейронных сетей, см. Куан и Уайт (1994). Последний пример, который также будет подробно описан далее, представляет собой гибкую модель условной дисперсии. Необходимо быть внимательным, чтобы обеспечить идентифицируемость полупараметрической модели. Например, посмотрите на обсуждение индентифицируемости одноиндексных моделей. Интересно также посмотреть не только на оценивание $\beta$, но и на предельные эффекты $\partial{\E}[y|x,z]/\partial{x}$.

\begin{table}[h]
\begin{center}
\begin{small}
\caption{\label{tab:pred} Полупараметрические модели: основные примеры}
\begin{tabular}[t]{llcc}
\hline
\hline
\bf{Название} & \bf{Модель} & \bf{Параметрическая} & \bf{Непараметрическая} \\
 & & \bf{компонента} & \bf{компонента} \\
\hline
Частично линейная & $\E[y|x,z] = x'\beta + \lambda(z)$ & $\beta$ & $\lambda(\cdot)$ \\
Одноиндексная & $\E[y|x] = g(x'\beta)$ & $\beta$ & $g(\cdot)$ \\
Обобщённая частично & $\E[y|x,z] = g(x'\beta + \lambda(z))$ & $\beta$ & $g(\cdot),\lambda(\cdot)$ \\
линейная & & & \\
Обобщённая аддитивная & $\E[y|x] = c + \sum_{j=1}^k g_j(x_j)$ & --- & $g_j(\cdot)$ \\
Частично аддитивная & $\E[y|x,z] = x'\beta + c + \sum_{j=1}^k g_j(x_j)$ & $\beta$ & $g_j(\cdot)$ \\
Поиск наилучшей & $\E[y|x] = \sum_{j=1}^M g_j(x_j'\beta_j)$ & $\beta_j$ & $g_j(\cdot)$ \\
проекции & & & \\
Линейная с гетероскедастичностью & $\E[y|x] = x'\beta$; $\V[y|x] = \sigma^2(x)$ & $\beta$ & $\sigma^2$ \\
\hline
\hline
\end{tabular}
\end{small}
\end{center}
\end{table}

\subsection{Эффективность полупараметрических оценок}

Мы рассмотрим потерю эффективности при оценке с помощью полупараметрических вместо параметрических методов. Далее мы представим результаты для нескольких основных полупараметрических моделей.

Наше описание аналогично описанию Робинсона (1988б), который рассматривает полупараметрическую модель с параметрической компонентой, которая обозначается как $\beta$, и  с непараметрической компонентой, которая обозначается как $G$. Компонента $G$ зависит от бесконечного числа мешающих параметров. Примеры $G$ включают форму распределения независимых, одинаково и симметрично распределённых ошибок и одноиндексную функцию $g(\cdot)$, которая приведена в (9.37) в разделе 9.7.4. Оценка $\hat{\beta} = \beta(\hat{G})$, где $\hat{G}$ --- непараметрическая оценка $G$.

В идеальном случае оценка $\hat{\beta}$ является адаптивной, то есть нет потери эффективности при оценивании $G$ с помощью непараметрических методов, т.е.
\[
\sqrt{N}(\hat{\beta} - \beta) \stackrel{d}{\rightarrow} \mathcal{N}[0,V_G],
\]
где $V_G$ --- ковариационная матрица для функции $G$ в конкретном рассматриваемом классе. В рамках метода правдоподобия $V_G$ --- нижняя граница Крамера-Рао. В контексте второго момента $V_G$ определяется из теоремы Гаусса-Маркова или из обобщений до, например, обобщённого метода моментов. Ярким примером адаптивной оценки является оценка с заданной функцией условного математического ожидания, но с неизвестной функциональной формой гетероскедастичности (см. раздел 9.7.6).

Если оценка $\hat{\beta}$ не является адаптивной, то следующая лучшая оценка в соответствии с критерием оптимальности будет оценка, достигающая полупараметрической границы эффективности $V_G^*$ так, чтобы
\[
\sqrt{N}(\hat{\beta} - \beta) \stackrel{d}{\rightarrow} \mathcal{N}[0,V_G^*],
\]
где $V_G^*$ является обобщением нижней границы Крамера-Рао или её аналогом второго момента, который обеспечивает наименьшую возможную ковариационную матрицу с учётом заданной полупараметрической модели. Для адаптивной оценки $V_G^* = V_G$, но обычно $V_G^*$ превышает $V_G$. Полупараметрические границы эффективности вводятся в разделе 9.7.8. Они могут быть получены только для некоторых полупараметрических моделей, и даже в этом случае может не существовать оценки, которая достигает границы. Пример, который достигает границу --- это оценка модели бинарного выбора Кляйна и Спэйди (1993) (см. раздел 14.7.4).

Если полупараметрическая граница эффективности недостижима или неизвестна, то следующее лучшее свойство заключается в том, выполняется ли $\sqrt{N}(\hat{\beta} - \beta) \stackrel{d}{\rightarrow} \mathcal{N}[0,V_G^{**}]$ для некоторой $V_G^{**}$, большей, чем $V_G^*$. Это позволяет делать обычные статистические выводы. В более общем случае $\sqrt{N}(\hat{\beta} - \beta) = O_p(1)$, но это выражение необязательно имеет нормальное распределение. В конце концов оценки, которые менее состоятельны, чем $\sqrt{N}$ состоятельные оценки имеют свойство $N^r(\hat{\beta} - \beta) = O_p(1)$, где $r < 0.5$. Часто невозможно достичь асимптотическую нормальность. Это часто возникает, если параметрическая и непараметрическая части рассматриваются одинаково, то есть максимизация происходит совместно по $\beta$ и $G$. Существует много примеров, особенно для дискретных и усечённых моделей выбора.

Несмотря на потенциальную неэффективность, полупараметрические оценки привлекательны тем, что
они могут сохранить состоятельность в условиях, когда полностью параметрическая оценка является несостоятельной. Пауэлл (1994, стр. 2513) приводит таблицу, которая кратко описывает существование состоятельных и $\sqrt{N}$ состоятельных асимптотических нормальных оценок для ряда полупараметрических моделей.

\subsection{Частично линейная модель}

Частично линейная модель задаёт условное математическое ожидание как обычную регрессионную функцию с добавлением неспецифицированной нелинейной компоненты:
\begin{equation}
\E[y|x,z] = x'\beta + \lambda(z),
\end{equation}
где скалярная функция $\lambda(\cdot)$ не задана.

Приведём пример оценивания функции спроса на электричество, где $z$ отражает показатель времени суток или индикаторы погоды такие, как температура. Второй пример --- модель самоотбора выборки, которая представлена в разделе 16.5. Игнорирование $\lambda(z)$ приводит к несостоятельным $\beta$ из-за смещения вызванного существованием пропущенных переменных кроме случая, когда $\Cov[x,\lambda(z)] = 0$. В прикладных исследованиях особое внимание может уделяться $\beta$, $\lambda$ или и тому, и другому. Полностью непараметрическое оценивание $\E[y|x,z]$ возможно, но оно приводит к оценкам, которые менее состоятельные, чем $\sqrt{N}$ состоятельные оценки $\beta$.

\begin{center}
Оценка разностей Робинсона
\end{center}

Робинсон (1988а) предложил следующий метод. Регрессионная модель имеет вид:
\[
y = x'\beta + \lambda(z) + u,
\]
где ошибка $u = y - \E[y|x,z]$. Из этого следует, что
\[
\E[y|z] = \E[x|z]'\beta + \lambda(z),
\]
так как $\E[u|x,z] = 0$ подразумевает, что $\E[u|z] = 0$. Вычитая одно из другого, получим
\begin{equation}
y - \E[y|z] = (x - \E[x|z])'\beta + u.
\end{equation}
Условные моменты из (9.35) неизвестны, но их можно заменить на непараметрические оценки.

Таким образом, Робинсон предложил с помощью МНК оценить регрессию
\begin{equation}
y_i - \hat{m}_{yi} = (x - \hat{m}_{xi})'\beta + v,
\end{equation}
где $\hat{m}_{yi}$ и $\hat{m}_{xi}$ --- прогнозы непараметрической регрессии $y_i$ и $x_i$ на $z_i$ соответственно. При условии независимости по $i$ МНК-оценка оценка $\beta$ из (9.36) $\sqrt{N}$ состоятельная и асимптотически нормальная с 
\[
\sqrt{N}(\hat{\beta}_{PL} - \beta) \stackrel{d}{\rightarrow} \mathcal{N} \left[ 0, \sigma^2 \left( \plim \frac{1}{N} \sum_{i=1}^N (x_i - \E[x_i|z_i])(x_i - \E[x_i|z_i])' \right)^{-1} \right],
\]
при предположении, что $u_i$ независимы и одинаково распределены с параметрами $[0,\sigma^2]$. Если $\lambda(z)$ не задана, то часто происходит потеря эффективности. Однако этой потери нет, если $\E[x|z]$ является линейным по $z$. Для того чтобы оценить $\V[\hat{\beta}_{PL}]$, можно просто заменить $(x_i - \E[x_i|z_i])$ на $(x_i - \hat{m}_{xi})$. Асимптотический результат можно обобщить и на гетероскедастичные ошибки. В этом случае используют обычные стандартные ошибки Эйкера-Уайта из регрессии МНК (9.36). Так как $\lambda(z) = \E[y|z] - \E[x|z]'\beta$, то можно получить её состоятельную оценку $\hat{\lambda}(z) = \hat{m}_{yi} - \hat{m}_{xi}'\hat{\beta}$.

Можно использовать разнообразные непараметрические оценки $\hat{m}_{yi}$ и $\hat{m}_{xi}$. Робинсон (1988a) использовал ядерные оценки, которые требуют сходимость со скоростью не меньшей, чем $N^{-1/4}$. По этой причине, если размерность $z$ велика, необходимо использовать сверхсглаживание или ядра более высокого порядка,; см. Паган и Улла (1999, стр. 205). Отметим также, что ядерные оценки тоже могут быть усечены (см. раздел 9.5.3).

\begin{center}
Другие оценки
\end{center}

Некоторые другие методы приводят к $\sqrt{N}$ состоятельным оценкам $\beta$ в частично линейной модели. Спекман (1988) также использовал ядра. Энгл и другие (1986) использовали обобщение оценки кубического сглаживания сплайнами. Эндриус (1991) представил регрессию $y$ на $x$ и аппроксимацию с помощью рядов для $\lambda(z)$, что приведено в разделе 9.6.4. Ячью (1997) представил простую оценку разностей.

\subsection{Одноиндексные модели}

Одноиндексная модель задаёт условное математическое ожидание как неизвестную скалярную функцию линейной комбинации регрессоров с 
\begin{equation}
\E[y|x] = g(x'\beta),
\end{equation}
где скалярная функция $g(\cdot)$ не задана. Преимущества одноиндексных моделей были представлены в разделе 5.2.4. В данном случае функция $g(\cdot)$ оценивается по имеющимся данным, в то время как в предыдущих примерах она была задана, например, как $\E[y|x] = \exp(x'\beta)$.

\begin{center}
Идентификация
\end{center}

Ичимура (1993) привёл условия идентификации для одноиндексной модели. При неизвестной функции $g(\cdot)$ для одноиндексной модели $\beta$ заданы только с точностью до сдвига и масштабирования. Чтобы это увидеть, обратите внимание, что для скалярного $v$ функцию $g^*(a + bv)$ можно всегда выразить как $g(v)$. Таким образом, функция $g^*(a + bx'\beta)$ эквивалентна $g(x'\beta)$. Кроме того, $g(\cdot)$ должна быть дифференцируема. В простейшем случае все регрессоры являются непрерывными. Но если некоторые регрессоры являются дискретными, то по крайней мере один регрессор должен быть непрерывным. Также если $g(\cdot)$ монотонна, то можно получить границы для $\beta$.

\begin{center}
Оценка средней производной
\end{center}

Для непрерывных регрессоров Стокер (1986) заметил, что если условное математическое ожидание является одноиндексным, то вектор средних производных условного математического ожидания определяет $\beta$ с точностью до масштаба, так как для $m(x_i) = g(x_i'\beta)$
\begin{equation}
\delta \equiv \E\left[ \frac{\partial{m(x)}}{\partial{x}} \right] = \E[g'(x'\beta)]\beta,
\end{equation}
и $\E[g'(x_i'\beta)]$ --- скаляр. Более того, из обобщённого равенства информационных матриц, которое приведено в разделе 5.6.3, следует, что для любой функции $h(x)$ $\E[\partial{h(x)}/\partial{x}] = - \E[h(x)s(x)]$, где $s(x) = \partial{\ln f(x)}/\partial{x} = f'(x)/f(x)$ и $f(x)$ --- плотность $x$. Таким образом, 
\begin{equation}
\delta = -\E[m(x)s(x)] = - \E[\E[y|x]s(x)].
\end{equation}
Отсюда следует, что $\delta$ и соответственно $\beta$ с точностью до масштаба могут быть оценены с помощью оценки средней производной $(AD)$
\begin{equation}
\hat{\delta}_{AD} = - \frac{1}{N} \sum_{i=1}^N y_i \hat{s}(x_i),
\end{equation}
где $\hat{s}(x_i) = \hat{f}'(x_i)/\hat{f}(x_i)$ можно получить с помощью ядерного оценивания плотности $x_i$ и её первой производной. Оценка $\hat{\delta}$ является $\sqrt{N}$ состоятельной, и её асимптотическое нормальное распределение были получено Хэрдлом и Стокером (1989). Функцию $g(\cdot)$ можно оценить с помощью непераметрической регрессии $y_i$ на $x_i'\hat{\delta}$. Обратите внимание, что $\hat{\delta}_{AD}$ даёт оценку $\E[m'(x)]$ вне зависимости от того, является ли одноиндексная модель релевантной или нет.

Недостаток $\hat{\delta}_{AD}$ заключается в том, что $\hat{s}(x_i)$ может быть очень большой, если $\hat{f}(x_i)$ маленькая. Один из вариантов --- усечь те значения, в которых $\hat{f}(x_i)$ маленькая. Пауэлл, Сток и Стокер (1989) обратили внимание, что результат из (9.38) можно обобщить до взвешенных производных с $\delta \equiv \E[w(x)m'(x)]$. Особенно удобно выбирать $w(x) = f(x)$, что даёт средневзвешенную оценку производной плотности $DWAD$
\begin{equation}
\hat{\delta}_{DWAD} = - \frac{1}{N} \sum_{i=1}^N y_i \hat{f}'(x_i),
\end{equation}
которую больше нет необходимости делить на $\hat{f}(x_i)$. Это позволяет получить $\sqrt{N}$ состоятельную и асимптотически нормальную оценку $\beta$ с точностью до масштаба. Например, первая компонента $\beta$ нормируется к единице, когда $\hat{\beta}_1 = 1$ и $\hat{\beta}_j = \hat{\delta}_j/\hat{\delta}_1$ для $j > 1$.

Эти методы требуют непрерывных регрессоров для того, чтобы существовали производные. Хорвитц и Хэрдл (1996) приводят обобщение и на дискретные регрессоры.

\begin{center}
Полупараметрический метод наименьших квадратов
\end{center}

Альтернативная оценка одноиндексной модели была предложена Ичимура (1993). Начнём с предположения, что $g(\cdot)$ известно. В этом случае $WLS$-оценка $\beta$ минимизирует
\[
S_N(\beta) = \frac{1}{N} \sum_{i=1}^N w_i(x)(y_i - g(x_i'\beta))^2.
\]
Для неизвестной $g(\cdot)$ Ичимура предложил заменить $g(x_i'\beta)$ непараметрической оценкой $\hat{g}(x_i'\beta)$, что приводит к взвешенной полупараметрической оценке наименьших квадратов $(WSLS)$ --- $\hat{\beta}_{WSLS}$, которая минимизирует
\[
Q_N(\beta) = \frac{1}{N} \sum_{i=1}^N \pi(x_i)w_i(x)(y_i - \hat{g}(x_i'\beta))^2,
\]
где $\pi(x)$ является функцией усечения, которая удаляет наблюдения, если ядерная оценка регрессии скаляра $x_i'\beta$ очень маленькая, и $\hat{g}(x_i'\beta)$ ---  ядерная оценка с выбрасывание отдельного наблюдения из регрессии $y_i$ на $x_i'\beta$. Эта $\sqrt{N}$ состоятельная и асимптотически нормальная оценка $\beta$ с точностью до масштаба. В общем случае она более эффективная, чем оценка $DWAD$. Для гетероскедастичных данных наиболее эффективная оценка --- аналог ДОМНК, который использует оценку взвешивающей функции $\hat{w}_i(x) = 1/\hat{\sigma}_i^2$, где $\hat{\sigma}_i^2$ --- ядерная оценка из (9.43) раздела 9.7.6 и где $\hat{u}_i = y_i - \hat{g}(x_i'\hat{\beta})$, а $\hat{\beta}$ получена из первоначальной минимизации $Q_N(\beta)$ с $w_i(x) = 1$.

$WSLS$ оценка вычисляется с помощью итерационных методов. Начнём с первоначальной оценки $\hat{\beta}^{(1)}$, например, $DWAD$ оценки с нормированной к единице первой компонентой. Получим ядерную оценку $\hat{g}(x_i'\hat{\beta}^{(1)})$ и соответственно $Q_N(\hat{\beta}^{(1)})$. Несильно поменяем $\hat{\beta}^{(1)}$, чтобы получить градиент $g_N(\hat{\beta}^{(1)}) = \partial{Q_N(\beta)}/\partial{\beta}|_{\hat{\beta}^{(1)}}$ и $\hat{\beta}^{(2)} = \hat{\beta}^{(1)} + A_Ng_N(\hat{\beta}^{(1)})$ и так далее. Эту оценку значительно сложнее вычислить, чем $DWAD$ оценку, особенно если $Q_N(\beta)$ может быть невыпуклой и мультимодальной.
 
\subsection{Обобщённые аддитивные модели}

Обобщённые аддитивные модели задают $\E[y|x] = g_1(x_1) + \cdots + g_k(x_k)$, частный случай полностью непараметрической модели $\E[y|x] = g(x_1, \dots, x_k)$. Этот частный случай даёт оценки подфункций $\hat{g}_j(x_j)$, которые сходятся с той же скоростью, что и одномерная непараметрическая регрессия, а не с более медленной скоростью $k$-мерной непараметрической регрессии. 

Существует хорошо развитая методология для оценивания таких моделей (см. Хасти и Тибшарани, 1990). Это автоматизировано в некоторых статистических пакетах таких, как S-Plus. Графики оценённых подфункций $\hat{g}_j(x_j)$ в зависимости от $x_j$ отражают предельные эффекты $x_j$ на $\E[y|x]$. По этой причине аддитивная модель может быть полезным инструментом для анализа данных. Эта модель редко используется в микроэконометрике отчасти потому, что многие приложения такие, как цензурированные, усечённые и дискретные переменные приводят к одноиндексным и частично линейным моделям.

\subsection{Гетероскедастичная линейная модель}

Гетероскедастичная линейная модель задаёт 
\[
\E[y|x] = x'\beta,
\]
\[
\V[y|x] = \sigma^2(x),
\]
где функция дисперсии $\sigma^2(\cdot)$ не задана.

Предположение, что ошибки гетероскедастичны, является стандартным предположением для пространственных данных в современной микроэконометрике. Можно получить состоятельные, но неэффективные оценки $\beta$ с помощью МНК и используя устойчивую к гетероскедастичности состоятельную оценку Эйкера-Уайта ковариационной матрицы МНК-оценки. Крэгг (1983) и Амемия (1983) предложили оценку инструментальных переменных, которая является более эффективной, чем МНК, но всё ещё не полностью эффективной. ДОМНК позволяет получить полностью эффективную оценку второго момента. Однако этот способ не является привлекательным, так как он требует наличия заданной функциональной формы для $\sigma^2(x)$ такой, как $\sigma^2(x) = \exp(x'\gamma)$.

Робинсон (1987) предложил вариант ДОМНК с использованием непараметрической оценки $\sigma_i^2 = \sigma^2(x_i)$. Тогда
\begin{equation}
\hat{\beta}_{HLM} = \left( \sum_{i=1}^N \hat{\sigma}_i^{-2} x_ix_i' \right)^{-1} \left( \sum_{i=1}^N \hat{\sigma}_i^{-2} x_iy_i \right),
\end{equation}
где Робинсон (1987) использовал $k - NN$ оценку $\sigma_i^2$ с равными весами. Тогда
\begin{equation}
\hat{\sigma}_i^2 = \frac{1}{k} \sum_{j=1}^N {\bf{1}} (x_j \in N_k(x_i))\hat{u}_j^2,
\end{equation}
где $\hat{u}_i = y_i - x_i'\hat{\beta}_{OLS}$ --- это остатки из первого шага МНК регрессии $y_i$ на $x_i$, и $N_k(x_i)$ --- множество из $k$ наблюдений $x_j$, которые наиболее близки к $x_i$ во взвешенной Евклидовой норме. В этом случае
\[
\sqrt{N}(\hat{\beta}_{HLM}  - \beta) \stackrel{d}{\rightarrow} \mathcal{N}\left[ 0, \left( \plim \frac{1}{N} \sum_{i=1}^N \sigma^{-2}(x_i)x_ix_i' \right)^{-1} \right],
\] 
предполагая, что $u_i$ независимы и одинаково распределены с параметрами $[0, \sigma^2(x_i)]$. Эта оценка адаптивная, так как она достигает границы Гаусса-Маркова так же, как и ОМНК-оценка при известной $\sigma_i^2$. Состоятельная оценка ковариационной матрицы $(N^{-1}\sum_i \hat{\sigma}_i^{-2} x_ix_i')^{-1}$.

Можно использовать другие непараметрические оценки $\sigma^2(x_i)$, но Кэрролл (1982) и другие изначально предложили использовать ядерную оценку $\sigma_i^2$ и обнаружили, что можно доказать эффективность только при  наложении ограничений на $x_i$. Метод Робинсон можно обобщить и на модели с нелинейной функцией математического ожидания.

\subsection{Полунепараметрический метод максимального правдоподобия}

Пусть $y_i$ независимы и одинаково распределены с заданной плотностью $f(y_i|x_i,\beta)$. В общем случае неверная спецификация плотности приводит к несостоятельным оценкам параметров. Галлант и Ничка (1987) предложили аппроксимировать неизвестную истинную плотность с помощью разложения в степенной ряд в окрестности плотности $f(y|x,\beta)$. Чтобы обеспечить положительную плотность, они фактически используют разложение в степенной ряд в квадрате в окрестности $f(y|x,\beta)$. Это даёт
\begin{equation}
h_p(y|x,\beta, \alpha) = \frac{(p(y|\alpha))^2f(y|x,\beta)}{\int (p(z|\alpha))^2f(y|z,\beta)dz},
\end{equation}
где $p(y|\alpha)$ --- многочлен порядка $p$, $\alpha$ --- вектор коэффициентов многочлена. Деление на знаменатель гарантирует, что интеграл плотности или  сумма вероятностей равна единице. Оценка $\beta$ и $\alpha$ максимизирует логарифм функции правдоподобия $\sum_{i=1}^N \ln h_p(y_i|x,\beta, \alpha)$. Подход можно сразу обобщить и на многомерный $y_i$. Эта оценка --- полунепараметрическая оценка максимального правдоподобия, потому что это непараметрическая оценка, которую можно оценить таким же образом, что и оценку метода максимального правдоподобия. Галлант и Ничка (1987) показали, что при достаточно общих условиях получаются состоятельны оценки плотности, если порядок $p$ многочлена увеличивается с размером выборки $N$ с соответствующей скоростью.

Этот результат даёт обоснование для использования (9.44), чтобы получить класс гибких распределений для каких-либо конкретных данных. Этот способ особенно прост, если полиномиальный ряд $p(y|\alpha)$ является ортогональным или ортонормированный полиномиальным рядом (см. раздел 12.3.1) для базовой плотности $f(y|x,\beta)$. Тогда можно выбрать более простой нормирующий множитель в знаменателе. Порядок многочлена можно выбрать с помощью информационных критериев с показателями, которые штрафуют за сложность модели сильнее, чем $AIC$, который используется на практике. Стандартные статистические выводы метода максимального правдоподобия можно делать, если проигнорировать, что выбор порядка многочлена зависит от данных, и предполагать, что плотность $h_p(y|x,\beta,\alpha)$ верно специфицирована. Пример такого подхода для регрессии счётных данных приведён у Камерона и Йоханссона (1997).

\subsection{Полупараметрические границы эффективности}

Полупараметрические оценки эффективности расширяют такие границы эффективности, как Крамер-Рао или теорему Гаусса-Маркова, для тех случаев, когда процесс, порождающий данные, имеет непараметрическую компоненту. Лучшие полупараметрические методы достигают эту границу эффективности.

Мы используем $\beta$ для обозначения параметров, которые мы хотим оценить. Они могут включать дисперсию $\sigma^2$. Мешающие параметры обозначены буквой $\eta$. Для простоты мы рассмотрим оценку метода максимального правдоподобия с непараметрической компонентой.

Начнём с полностью параметрического случая. Оценка метода максимального правдоподобия $(\hat{\beta},\hat{\eta})$ максимизирует $\mathcal{L}(\beta, \eta) = \ln L(\beta, \eta)$. Пусть $\theta = (\beta, \eta)$, и пусть $\mathcal{I}_{\theta \theta}$ --- информационная матрица, которая определена в (5.43). Тогда $\sqrt{N}(\hat{\theta} - \theta) \stackrel{d}{\rightarrow} \mathcal{N}[0, \mathcal{I}_{\theta \theta}^{-1}]$. Для $\sqrt{N}(\hat{\beta} - \beta)$ блочное обращение $\mathcal{I}_{\theta \theta}$ приводит к 
\begin{equation}
V^* = (\mathcal{I}_{\beta \beta} - \mathcal{I}_{\beta \eta} \mathcal{I}_{\eta \eta}^{-1} \mathcal{I}_{\eta \beta})^{-1}
\end{equation}
в качестве границы эффективности для оценки $\beta$ при неизвестном $\eta$. Существует потеря эффективности
при неизвестном $\eta$ кроме случая, когда информационная матрица является блочно-диагональной, и $\mathcal{I}_{\beta \eta} = 0$, а дисперсия сводится к $\mathcal{I}_{\beta \beta}^{-1}$.

Теперь рассмотрим обобщение на непараметрический случай. Предположим, что у нас есть параметрическая подмодель, например, $\mathcal{L}_0(\beta)$, которая включает в себя только $\beta$. Рассмотрим семейство всех возможных параметрических моделей $\mathcal{L}(\beta, \eta)$ таких, что $\mathcal{L}_0(\beta)$ являются их вложенными моделями при некотором значении $\eta$. Полупараметрическая граница эффективности --- наибольшее значение $V^*$ из (9.45) для всех возможных параметрических моделей $\mathcal{L}(\beta, \eta)$. Однако её сложно получить.

Возможно упростить эту ситуацию, рассматривая
\[
\tilde{s}_{\beta} = s_{\beta} - \E[s_{\beta}|s_{\eta}],
\]
где $s_{\theta}$ означает скор-функцию $\partial{\mathcal{L}}/\partial{\theta}$ и $\tilde{s}_{\beta}$ --- скор-функция для $\beta$ после удаления $\eta$. Для $\eta$ с конечной размерностью можно показать, что $\E[N^{-1}\tilde{s}_{\beta}\tilde{s}_{\beta}'] = V^*$. В данном случае наоборот $\eta$ имеет бесконечную размерность. Предположим, что данные независимы и одинаково распределены. Пусть $s_{\theta_i}$ обозначает $i$-тую компоненту суммы, которая приводит к скор-функции $s_{\theta}$. Беган и другие (1983) определяют касательное множество как все линейные комбинации $s_{\eta_i}$. Когда оно является линейным и замкнутым наибольшее значение $V^*$ из (9.45) равно
\[
\Omega = (\plim N^{-1}\tilde{s}_{\beta}\tilde{s}_{\beta}')^{-1} = (\E[\tilde{s}_{\beta_i} \tilde{s}_{\beta_i}'])^{-1}.
\]
В этом случае матрица $\Omega$ будет являться полупараметрической границей эффективности.

В прикладных исследованиях сначала получают $s_{\eta} = \sum_i s_{\eta_i}$. Потом получают $\E[s_{\beta_i}| s_{\eta_i}]$, для чего могут потребоваться разные  предположения, например, симметрия ошибок. Они накладывают ограничения на класс рассматриваемых полупараметрических моделей. Это даёт $\tilde{s}_{\beta_i}$ и соответственно $\Omega$. Более подробное описание представлено у Ньюи (1990б), Пагана и Улла (1999), Северини и Трипати (2001).

\section{Вывод математического ожидания и дисперсии ядерных оценок}

Непараметрическое оценивание подразумевает баланс между гладкостью (дисперсией) и смещением (математическим ожиданием). Здесь мы выводим математическое ожидание и дисперсию ядерной плотности и ядерных оценок регрессии. Вывод аналогичен тому, как это делал М. Дж. Ли (1996).

\subsection{Математическое ожидание и дисперсия ядерной оценки плотности}

Так как $x_i$ независимы и одинаково распределены, каждый член суммы имеет одинаковое ожидаемое значение и
\[
\E[\hat{f}(x_0)] = \E \left[ \frac{1}{h} K \left( \frac{x-x_0}{h} \right) \right] = \int \frac{1}{h} K \left( \frac{x-x_0}{h} \right) f(x)dx.
\]
Заменив переменную на $z = (x - x_0)/h$ так, чтобы $x = x_0 + hz$ и $dx/dz = h$, мы получим
\[
\E[\hat{f}(x_0)] = \int K(z)f(x_0 + hz)dz.
\]
Разложение $f(x_0 + hz)$ в ряд Тейлора второго порядка в окрестности $f(x_0)$ даёт  
\[
\E[\hat{f}(x_0)] = \int K(z)\{ f(x_0) + f'(x_0)hz + \frac{1}{2}f''(x_0)(hz)^2\}dz 
\]
\[
= f(x_0) \int K(z)dz + hf'(x_0)\int zK(z)dz + \frac{1}{2} h^2f''(x_0)\int z^2K(z)dz.
\]
Так как интеграл $K(z)$ равен единице, то 
\[
\E[\hat{f}(x_0)] - f(x_0) = hf'(x_0) \int zK(z)dz + \frac{1}{2}h^2f''(x_0) \int z^2K(z)dz.
\]
Если также ядро удовлетворяет условию $\int zK(z)dz = 0$, что предполагалось в условии (2) в разделе 9.3.3, и вторая производная $f$ ограничена, то первый член с правой стороны исчезает. Тогда получается $\E[\hat{f}(x_0)] - f(x_0) = b(x_0)$, где граница $b(x_0)$ задана в 9.4.

Чтобы получить дисперсию $\hat{f}(x_0)$, заметим, что если $y_i$ независимы и одинаково распределены, то $\V[\bar{y}] = N^{-1}\V[y] = N^{-1}\E[y^2] - N^{-1}(\E[y])^2$. Таким образом,
\[
\V[\hat{f}(x_0)] = \frac{1}{N} \E\left[ \left( \frac{1}{h} K \left( \frac{x - x_0}{h} \right) \right)^2 \right] - \frac{1}{N} \left( \E \left[ \frac{1}{h} K \left( \frac{x - x_0}{h} \right) \right] \right)^2.
\]
Теперь с помощью замены переменных и разложения в ряд Тейлора первого порядка получим
\[
\E\left[ \left( \frac{1}{h} K \left( \frac{x - x_0}{h} \right) \right)^2 \right] = \int \frac{1}{h} K(z)^2 \{f(x_0) + f'(x_0)hz\}dz
\]
\[
= \frac{1}{h}f(x_0)\int K(z)^2dz + f'(x_0)\int zK(z)^2dz.
\]
Отсюда следует, что 
\[
\V[\hat{f}(x_0)] = \frac{1}{Nh} f(x_0) \int K(z)^2 dz + \frac{1}{N} f'(x) \int zK(z)^2dz
\]
\[
- \frac{1}{N} \left[f(x_0) + \frac{h^2}{2} f''(x_0)\left[ \int z^2 K(z)dz\right]\right]^2.
\]
Для $h \rightarrow 0$ и $N \rightarrow \infty$ это доминируется первым членом, что приводит к уравнению (9.5).

\subsection{Распределение ядерной оценки регрессии}

Мы получаем распределение для регрессоров $x_i$, которые являются независимыми и одинаково распределёнными с плотностью $f(x)$. Из раздела 9.5.1 ядерная оценка --- средневзвешенное $\hat{m}(x_0) = \sum_i w_{i0,h}y_i$, где ядерные веса $w_{i0,h}$ приведены в (9.22). Так как сумма весов равна единице, мы получаем $\hat{m}(x_0) - m(x_0) = \sum_i w_{i0,h}(y_i - m(x_0))$. Заменяя (9.15) на $y_i$ и нормируя на $\sqrt{Nh}$, так как в случае ядерной оценки плотности мы получаем
\begin{equation}
\sqrt{Nh}(\hat{m}(x_0) - m(x_0)) = \sqrt{Nh} \sum_{i=1}^N w_{i0,h}(m(x_i) - m(x_0) + \e_i).
\end{equation}

Один подход, позволяющий получить предельное распределение (9.46), состоит в том, чтобы разложить $m(x_i)$ в ряд Тейлора второго порядка в окрестности $x_0$. Такой подход используют не всегда, так как веса $w_{i0,h}$ имеют сложный вид из-за нормирования, чтобы их сумма равнялась единице (см. (9.22)).

Вместо этого мы применяем подход Ли (1996, стр. 148-151), который аналогичен подходу Биренса (1987, стр. 106-108). Обратите внимание, что знаменатель взвешивающей функции --- ядерная оценка плотности $x_0$, так как $\hat{f}(x_0) = (Nh)^{-1} \sum_i K((x_i - x_0)/h)$. Тогда (9.46) даёт
\begin{equation}
\sqrt{Nh}(\hat{m}(x_0) - m(x_0)) = \frac{1}{\sqrt{Nh}} \left( \sum_{i=1}^N K \left( \frac{x_i - x_0}{h} \right)(m(x_i) - m(x_0) + \e_i) \right) / \hat{f}(x_0)
\end{equation}
Мы применяем теорему о преобразовании (Теорема А.12) к (9.47), используя $\hat{f}(x_0) \stackrel{p}{\rightarrow} f(x_0)$ в знаменателе. Получение предельного распределения для числителя происходит в несколько этапов:
\begin{multline}
\frac{1}{\sqrt{Nh}} \sum_{i=1}^N K \left( \frac{x_i - x_0}{h} \right)(m(x_i) - m(x_0) + \e_i) = \\
 \frac{1}{\sqrt{Nh}} \sum_{i=1}^N K \left( \frac{x_i - x_0}{h} \right)(m(x_i) - m(x_0)) +  \frac{1}{\sqrt{Nh}} \sum_{i=1}^N K \left( \frac{x_i - x_0}{h} \right)\e_i
\end{multline}

Рассмотрим первую сумму из (9.48). Если можно применить закон больших чисел, то эта сумма сходится по вероятности к своему математическому ожиданию:

\begin{multline}
\E \left[ \frac{1}{\sqrt{Nh}} \sum_{i=1}^N K \left( \frac{x_i - x_0}{h} \right)(m(x_i) - m(x_0)) \right] = \\
\frac{\sqrt{N}}{\sqrt{h}} \int K \left( \frac{x - x_0}{h} \right)(m(x) - m(x_0))f(x)dx = \\
\sqrt{Nh} \int K(z)(m(x_0 + hz) - m(x_0))f(x_0 + hz)dz = \\
\sqrt{Nh} \int K(z) \left( hzm'(x_0)+ \frac{1}{2}h^2z^2m''(x_0) \right) (f(x_0) + hzf'(x_0))dz =  \\
\sqrt{Nh} \left\{ \int K(z)h^2z^2 m'(x_0)f'(x_0)dz + \int K(z) h^2z^2m''(x_0)f(x_0)dz \right\} =  \\
\sqrt{Nh}h^2 \left( m'(x_0)f'(x_0) + \frac{1}{2} m''(x_0) f(x_0) \right) \int z^2 K(z)dz  = \\
\sqrt{Nh}f(x_0)b(x_0),
\end{multline}
где $b(x_0)$ определено в (9.23). Первое равенство использует независимость и одинаковость распределения $x_i$. Второе равенство отражает замену переменных на $z = (x - x_0)/h$. Третье равенство применяет разложение в ряд Тейлора до второго порядка к $m(x_0 + hz)$ и разложение в ряд Тейлора до первого порядка к $f(x_0 + hz)$. Четвёртое равенство следует, потому что почленное перемножение даёт четыре члена, два из которых доминируют над остальными (см., например, Ли, 1996, стр. 150).

Сейчас рассмотрим вторую сумму из (9.48). Математическое ожидание каждого из членов суммы равно нулю, а дисперсия каждого из них, опуская индекс $i$, имеет вид:
\begin{equation}
\begin{split}
\V \left[ K \left( \frac{x - x_0}{h} \right)\e \right] &= \E \left[ K^2 \left( \frac{x - x_0}{h} \right)\e^2 \right] \\
 &= \int K^2 \left( \frac{x - x_0}{h} \right) \V[\e|x]f(x)dx \\
 &= h \int K^2(z)\V[\e|x_0 + hz]f(x_0 + hz)dz \\
 &= h\V[\e|x_0]f(x_0) \int K^2(z)dz,
\end{split}
\end{equation}
заменяя переменные на $z = (x - x_0)/h$ с $dx = hdz$ в члене на третьей строке и устремляя $h \rightarrow 0$, получим выражение из последней строки. Применив центральную предельную теорему получаем, что 
\begin{equation}
\frac{1}{\sqrt{Nh}} \sum_{i=1}^N K \left( \frac{x_i - x_0}{h} \right)\e_i \stackrel{d}{\rightarrow} \mathcal{N} \left[ 0, \V[\e|x_0]f(x_0) \int K^2(z)dz \right].
\end{equation}

Соединяя (9.49) и (9.51), мы получаем, что $\sqrt{Nh}(\hat{m}(x_0) - m(x_0))$ из (9.47) сходится к $1/f(x_0)$, умноженному на $\mathcal{N}[\sqrt{Nh}f(x_0)b(x_0),\V[\e|x_0]f(x_0) \int K^2(z)dz]$. Деление математического ожидания на $f(x_0)$ и дисперсии на $f(x_0)^2$ приводит к получению предельного распределения, которое определено в (9.24).

\section{Практические сообращения}

Статистические пакеты, которые содержат различные регрессии, предлагают адекватные методы для одномерного непараметрического оценивания плотности. Язык программирования Xplore используется для непараметрических и графических методов. Подробную информацию о многих методов можно найти на веб-сайте этого языка программирования.

Непараметрическое оценивание одномерной плотности можно осуществить, используя ядерную оценку плотности на основе таких ядер, как Гауссово или Епанечникова. Легко вычислить оценки ширины окна с помощью правила Сильвермана. Оно представляет хорошую отправную точку, потому что потом можно взять половину этого значения или умножить его на два, чтобы проверить наличие улучшения.

Непараметрическую одномерную регрессию также легко осуществить, но в данном случае возникает проблема выбора ширины окна. Если мы хотим получить относительно несмещённые оценки регрессионной функции в граничных точках, то оценки локальной линейной регрессии или $LOWESS$ лучше, чем ядерная регрессия. Сложнее получить оценки ширины окна с помощью метода аналогичного оценке Сильвермана, поэтому вместо него применяется кросс-валидация (см. раздел 9.5.3 ), а также визуальный анализ с помощью диаграммы рассеяния, на которой изображена оценённая линия регрессии. Степень требуемой гладкости может изменяться в зависимости от желания исследователя. Для непараметрической многомерной регрессии подобного рода визуальный анализ может оказаться невозможным.

Полупараметрическую регрессию сложнее осуществить. Её применение может повлечь за собой такие неоднозначные моменты, как усечение и недосглаживание непараметрической компоненты, так как обычно оценивание параметрической компоненты включает в себя усреднение непараметрической компоненты. Для таких целей обычно используют специализированный код, написанный на таких языках, как Gauss, Matlab, Splus или Xplore. Для непараметрического оценивания компоненты можно значительно сократить вычисления за счёт использования быстрых вычислительных алгоритмов таких, как разбиение значений случайной величины на интервалы и пошагового обновления, см., например, Фан и Гайбельс (1996), Хэрдл и Линтон (1994).

На определённом этапе все методы требуют спецификации ширины окна. Различные варианты приводят к различным оценкам на конечных выборках. Эти различия могут быть довольно большими, как было показано на многих графиках в этой главе. С другой стороны, в полностью параметрических рамках разные исследователи, которые оценивают одну и ту же модель с помощью метода максимального правдоподобия, будут получать одни и те же оценки параметров. Эта неопределённость является недостатком непараметрических методов, хотя есть надежда, что по крайней мере в полупараметрических методах побочные эффекты для параметрической компоненты модели могут быть небольшими.

\section{Библиографические заметки}

Непараметрическое оценивание хорошо описано во многих статистических текстах, в том числе в работах Фана и Гайбельса (1996). Рупперт, Ванд, и Кэрролл (2003) представляют способы применения многих полупараметрических методов. Книги по эконометрики Хэрдла (1990), М. Дж. Ли (1996), Хоровица (1998б), Пагана и Улла (1999), и Ячью (2003) охватывают как непараметрическое, так и полупараметрическое оценивание. Паган и Улл (1999), в частности, приводят наиболее подробное описание. Ячью (2003) больше ориентировался на эконометриста-практика. Он уделяет особое внимание частично линейным и одноиндексным моделям, а также практическим аспектам их реализации, например, построению доверительных интервалов.

\begin{itemize}
\item [$9.3$] Наиболее ранее описание ядерного оценивания плотности приведено у Росенблатта (1956) и Парзена (1962). Книга Сильвермана (1986) является классической книгой о непараметрическом оценивании плотности.
\item [$9.4$] Довольно общее описание оптимальной скорости сходимости для непараметрических оценок приведено у Стоуна (1980).
\item [$9.5$] Ядерное оценивание было предложено Надарайа (1964) и Уотсоном (1964). Очень полезное и довольно ясное описание ядерной регрессии и метода ближайших соседей было приведено Альтманом (1992). В статистической литературе есть ещё много других полезных работ. Хэрдл (1990, глава 5) привёл подробное рассмотрение вопроса выбора ширины окна и построения доверительных интервалов.
\item [$9.6$] У Стоуна приведено много подходов к непараметрической локальной регрессии (1977). Об оценках серий можно посмотреть у Эндриуса (1991) и Ньюи (1997). 
\item [$9.6$] Границы полупараметрической эффективности описаны у Ньюи (1990б), а также в более недавних работах Северини и Трипати (2001). Эконометрическое применение было описано Чемберленом (1987).
\item [$9.7$] Книги по эконометрике подробно рассматривают полупараметрическую регрессию. Об этом писали Пауэлл (1994), Робинсон (1988б) и для более начального уровня Ячью (1998). Также в этой книге приведены дополнительные ссылки, особенно в разделах 14.7, 15.11, 16.9, 20.5 и 23.8. Практическое исследование Беллемара, Меленберга и Ван Соеста (2002) иллюстрирует некоторые полупараметрические методы. 
\end{itemize}


\section{Упражнения}

\begin{enumerate}
\item [$9 - 1$] Предположим, что мы получаем ядерную оценку плотности, используя равномерное ядро (см. таблицу 9.1), с $h = 1$ и размером выборки $N = 100$. Предположим, что данные имеют нормальное распределение $x \sim \mathcal{N}[0,1]$.
\begin{enumerate}
\item Рассчитайте смещение ядерной оценки плотности в точке $x_0 = 1$, используя (9.4).
\item Велика ли величина смещения относительно истинного значения $\phi(1)$, где $\phi(\cdot)$ --- функция плотности стандартного нормального распределения?
\item Рассчитайте дисперсию ядерной оценки плотности в точке $x_0 = 1$, используя (9.5).
\item У кого больший вклад в $MSE$ в точке $x_0 = 1$, у дисперсии или квадрата смещения?
\item Используя результаты из раздела 9.3.7, постройте 95\% доверительный интервал для плотности в точке $x_0 = 1$ с помощью ядерной оценки $\hat{f}(1)$.
\item Определите оптимальную ширину окна $h^*$ из (9.10) для этого примера.
\end{enumerate}
\item [$9 - 2$] Предположим, что мы получаем ядерную оценку плотности, используя равномерное ядро (см. таблицу 9.1), с $h = 1$ и размером выборки $N = 100$. Предположим, что данные имеют нормальное распределение $x \sim \mathcal{N}[0,1]$ и функция условного математического ожидания --- это $m(x) = x^2$.
\begin{enumerate}
\item Рассчитайте смещение ядерной оценки плотности в точке $x_0 = 1$, используя (9.23).
\item Велика ли величина смещения относительно истинного значения $m(1) = 1$?
\item Рассчитайте дисперсию ядерной оценки плотности в точке $x_0 = 1$, используя (9.24).
\item У кого больший вклад в $MSE$ в точке $x_0 = 1$, у дисперсии или квадрата смещения?
\item Используя результаты из раздела 9.5.4, постройте 95\% доверительный интервал $\E[y|x_0 = 1]$ с помощью ядерной оценки регрессии $\hat{m}(1)$.
\item Определите оптимальную ширину окна $h^*$ из (9.10) для этого примера.
\end{enumerate}
\item [$9 - 3$] Этот вопрос предполагает доступ к программе непараметрического оценивания плотности. Используйте данные о расходах на здравоохранение из раздела 4.6.4. Используйте ядерную оценку плотности с Гауссовым ядром (если оно доступно).
\begin{enumerate}
\item Получите ядерную оценку плотности для расходов на здравоохранение, выбирая подходящую ширину окна с помощью визуального анализа и метода проб и ошибок. Укажите выбранную ширину окна.
\item Получите ядерную оценку плотности для натурального логарифма расходов на здравоохранение, выбирая подходящую ширину окна с помощью визуального анализа и метода проб и ошибок. Укажите выбранную ширину окна.
\item Сравните ответ из пункта (б) с соответствующей гистограммой.
\item Если возможно, наложите плотность нормального распределения на этот же график, что и
ядерную оценку плотности из пункта (б). Имеют ли расходы на здравоохранение лог-нормальное распределение?
\end{enumerate}
\item [$9 - 4$] Этот вопрос предполагает доступ к программе, где есть непараметрическая регрессия или другой	 непараметрический сглаживатель. Используйте полную выборку данных из раздела 4.6.4, где есть натуральный логарифм расходов на здравоохранение $(y)$ и натуральный логарифм общих расходов $(x)$.
\begin{enumerate}
\item Получите ядерную оценку плотности из регрессии для расходов на здравоохранение, выбирая подходящую ширину окна с помощью визуального анализа и метода проб и ошибок. Укажите выбранную ширину окна.
\item Смотря на результат пункта (а), является ли здоровье нормальным благом?
\item Смотря на результат пункта (а), является ли здоровье благом высшей категории?
\item Сравните Ваши непараметрические оценки с прогнозами из линейной и квадратичной регрессии.
\end{enumerate}
\end{enumerate}



\chapter {Методы численной оптимизации}
\section{Введение}
Теоретическое вопросы  состоятельности и асимптотического распределения оценок были рассмотрены в качестве решения проблемы оптимизации в главах 5 и 6. В этой главе рассматриваются практические аспекты применения вычислительных методов для решения задач оптимизации, а именно, описаны методы, которые применяются для определения оценок параметров при отсутствии явно заданной формулы.


С точки зрения прикладных исследований оценка параметров нелинейных моделей, таких как логит, пробит, тобит-моделей, моделей пропорционального риска и Пуассона, не отличается от обычного МНК. Статистические пакеты позволяют определить коэффициенты регрессии, стандартные ошибки, t-статистики и p-значения. Вычислительные проблемы могут появиться по причинам сходным невозможности применить МНК:  мультиколлинеарность, ошибки в исходных данных.

Для оценки параметров в нестандартных моделях и в некоторых вариантах стандартных моделей, может потребоваться специальная программа. В случае если возможности стандартного статистического пакета не позволяют написать такую программу, то возможно использовать язык программирования. При этом необходимо знакомство с методами оптимизации.


Общие методы оптимизации рассмотрены в разделе 10.2., а ряд итерационных методов, например, градиентные методы Ньютона-Рафсона и Гаусса-Ньютона, в разделе 10.3. Особенности практического применения данных методов, в том числе типичные ошибки, рассмотрены в разделе 10.4. Данные особенности становятся особенно значимыми в том случае, если методы оптимизации не дают оценок параметров.

\section{Обзор методов численной оптимизации}

Зачастую в основе микроэкономического анализа лежит оценка параметра $\hat{\theta}$, которая максимизирует значение стохастически-заданной целевой функции $Q_{N}(\theta)$ и, как правило, значения $\hat{\theta}$ являются решением условий первого порядка $\partial{Q_{N}(\theta)}/\partial\theta=0$. Задачу минимизации можно привести к задаче максимизации путем умножения целевой функции на минус единицу. В случае нелинейно заданной целевой функции условия первого порядка, как правило, не будут иметь явного решения. Нелинейно заданная функция представляет собой систему из $q$ уравнений с $q$ неизвестными параметрами $\theta$.

Метод поиска на сетке, рассмотренный далее, редко применяется на практике. Вместо него применяются итерационные методы, например, градиентные.

\subsection{Метод поиска на сетке}

Процедура поиска на сетке включает в себя перебор множества значений параметра $\theta$ вдоль сетки, расчет значений функции $Q_N(\theta)$ для каждого параметра и выбор в качестве оценки того значения параметра $\hat{\theta}$, которое является локальным или глобальным (в зависимости от приложения) максимумом функции $Q_N(\theta)$. 

Данный метод всегда даст результат, если выбран достаточно мелкий шаг сетки. Однако на практике применение данного метода может сопровождаться рядом трудностей, поскольку выбор мелкого шага сетки осложнен. Например, если нам необходимо оценить 10 параметров и на сетке расположено 10 значений для каждого параметра, то получается что нам необходимо сделать $10^{10}$, т.е. 10 миллиардов расчетов. Кроме того данная сетка является довольно редкой.

Тем не менее, методы поиска на сетке могут использоваться при необходимости найти оптимальные значения только для некоторого подмножества параметров. Кроме того данные методы позволяют изучить поверхность отклика, чтобы убедиться в отсутствии множественности максимумов. По данному принципу работают многие статистические пакеты при оценки AR(1) коэффициента в регрессии с $\AR(1)$ ошибками. Кроме того, поиск на сетке позволяет найти оптимальные значения для скалярного параметра (inclusive value) во вложенной логит-модели (см. раздел 15.6). Поиск на сетке может быть использован в случае, если все остальные методы невозможно применить.

\subsection{Итерационные методы}

На практике наибольшее применение в микроэкономическом анализе получили итерационные методы. В данных методах оценка параметра $\theta$ пересчитывается согласно определенным правилам. На основе  $s$-ой оценки $\hat{\theta}_s$, итерационные методы строят $(s+1)$-ую оценку $\hat{\theta}_{s+1}$, здесь $\hat{\theta}_s$ --- оценка параметра, полученная на $s$-том шаге, а не $s$-тый компонент вектора оценки $\hat{\theta}$. Теоретически, новая оценка параметра должна быть быть более оптимальное, т.е. $Q_N(\hat{\theta}_{s+1})>Q_N(\hat{\theta}_{s})$, однако это не всегда выполняется. Следует отметить, что результатом оценивания градиентным методом может стать нахождение локального, а не глобального максимума.

\subsection{Градиентные методы}

Наибольшее количество итерационных методов являются градиентными. Они предполагают изменение оценки параметра $\hat{\theta}_s$ в соответствии с направлением градиента. Формула пересчета основана на взвешенном с помощью матрице градиенте

\begin{equation}
\hat{\theta}_{s+1}=\hat{\theta}_{s}+A_{s}g_{s}, s=1,\ldots ,S,
\end{equation}
где $A_s$ --- матрица размерностью $q \times q$ значения которой зависят от $\hat{\theta}_s$ и

\begin{equation}
g_s=\left. \dfrac{\partial{Q_N}{(\theta)}}{\partial\theta}\right|_{\hat{\theta}_{s}}
\end{equation}
где $g_s$ --- вектор градиента размерностью $q \times 1$, значения которого получены в точке $\hat{\theta}_s$. Применение различных градиентных методов предполагает использование различных матриц $A_s$, подробно данный вопрос рассматривается в разделе 10.3. Показательным примером является метод Ньютона-Рафсона, согласно которому матрица $A_s=-{H_s}^{-1}$, где $H_s$ --- это Гессиан (определение матрицы Гессе дано в разделе 10.6). Следует отметить, что в  этой главе обозначения $A$ и $g$ обозначают не то, что в остальных главах.  В остальных главах учебника матрица $A$ --- матрица, используемая для описания предельного распределения оценок параметров, а $g$ --- условное среднее  $y$ в модели нелинейной регрессии.

Теоретически,  матрицу $A_s$ следует взять положительно определенную для нахождения максимума (отрицательно определенную для нахождения минимума). При выполнение данного условия часто будет выполнено неравенство: $Q_N(\hat{\theta}_{s+1})>Q_N({\hat{\theta}}_s)$. Это следует из разложения целевой функции в ряд Тейлора, $Q_N(\hat{\theta}_{s+1})=Q_N(\hat{\theta}_s)+g'_s(\hat{\theta}_{s+1}-\hat{\theta}_s)+R$  где $R$ --- остаточный член. Подставляя данное выражение в формулу пересчета (10.1) получим, что

\[
Q_N(\hat{\theta}_{s+1})-Q_N(\hat{\theta}_s)=g'_sA_sg_s+R.
\]
Данное выражение больше нуля, если матрица $A_s$ положительно определена и остаток $R$ является малой величиной, потому как для положительно определенной квадратной матрицы $A$ выполнено неравенство $x'Ax>0$ для всех вектор-столбцов $x\neq0$. Слишком маленькие значения матрицы $A_s$ приводят к замедлению итерационных процессов; в тоже время слишком большие значения матрицы $A_s$ могут привести к <<перелёту>>, даже если матрица $A_s$ положительна определена. Связано это с тем, что при больших изменениях нельзя игнорировать остаточный член.

Стандартным способом предотвращения <<перелёта>> или слишком медленной скорости  градиентных методов является изменение размера шага:

\begin{equation}
\hat{\theta}_{s+1}=\hat{\theta}_s+\hat{\lambda}_sA_sg_s
\end{equation}
где скаляр $\hat{\lambda}_s$ --- размер шага, значения которого выбраны таким образом, чтобы максимизировать значение функции $Q_N(\hat{\theta}_{s+1})$. В первую очередь при итерации $s$ необходимо рассчитать $A_s g_s$, что может потребовать большого количества вычислений. Далее, необходимо определить $Q_N(\hat{\theta})$, где $\hat{\theta} = \hat{\theta}_s + \lambda A_s g_s$ для ряда значений $\lambda$ (так называемый, линейный поиск) и выбрать такое значение $\hat{\lambda}_s$, которое максимизирует значение функции $Q_N(\hat{\theta})$. Возможна существенная экономия времени вычислений, поскольку значения вектора градиента и матрицы $A_s$ не пересчитываются в ходе процесса линейного поиска.

Альтернативный способ корректировки применяется в случае, если матрица $A_s$ является обратной к матрице $B_s$, т.е $A_s=B_s^{-1}$. Тогда, если матрица $B_s$ близка к вырожденной, в качестве корректировки можно прибавлять или вычитать некую матрицу констант $C$, чтобы разрешить проблему вырожденности матрицы, т.е. $A_s = ({B_s + C})^{-1}$. Аналогичную корректировку можно применить для  матрицы $A_s$, не являющейся положительно определенной. Более подробно расчет матрицы $A_s$ представлен в разделе 10.3.

Как правило, градиентные методы позволяют найти локальный максимум функции в диапазоне близком к начальному значению. В случае если целевая функция имеет множество локальных максимумов, следует использовать множество начальных значений с целью увеличения вероятности определения глобального максимума.

\subsection{Пример градиентного метода}

Рассмотрим пример расчета МНК-оценки в экспоненциальной регрессионной модели, где единственным независимым параметром является константа. Тогда математическое ожидание $y$ будет равно $\E[y]=e^{\beta}$ и значение градиента будет равно $g=N^{-1}\Sigma_{i}(y_i-e^\beta)e^{\beta}$. Предположим, что в формуле (10.1) мы использовали матрицу $A_s=e^{-2\hat{\beta}_s}$, что является скоринговой вариацией алгоритма метода Ньютона-Рафсона. Данный метод будет рассмотрен подробнее в разделе 10.3.2. Итерационная формула для $\hat{\beta}_{s+1}$ упрощается до $\hat{\beta}_s = \hat{\beta} + (\bar{y} --- e^{\hat{\beta}_s})/e^{\hat{\beta}_s}$. 

Для иллюстрации работы данного алгоритма предположим, что $\bar{y}=2$ и начальное значение параметра рано единице, $\hat{\beta}_{1}=0$. Результаты итераций представлены в таблице 10.1. Как можно видеть существует очень быстрая сходимость к нелинейной МНК оценке и для приведенного примера оценка может быть получена аналитически поскольку $\hat{\beta} = \ln \bar{y} = \ln 2 = 0.693147$. Значения целевой функции растут с ростом числа итераций, из-за использования алгоритма Ньютона-Рафсона для глобально вогнутой функции. Отметим, что <<перелёт>> оценок происходит при первой итерации, от $\hat{\beta}_1 = 0.0$ до $\hat{\beta}_2 = 1.0$, что превышает значение $\hat{\beta} = 0.693$.

\begin{table}[h]
\begin{center}
\caption{\label{tab:gradres} Результаты градиентного метода}
\begin{tabular}{cccc}
\hline 
\hline
Итерация & Оценка & Градиент & Целевая функция \\ 
\hline 
s & $\hat{\beta}_s$& $g_s$ & $Q_N(\hat{\beta}_s)=-\frac{1}{2N}\sum_i(y_i-e^{\beta})^2$ \\ 
1 & 0.000000& 1.000000& 1.500000 -- $\sum_i{y_i}^2/2N$ \\ 
2 & 1.000000 & -1.952492 & 1.742036 -- $\sum_i{y_i}^2/2N$ \\ 
3 & 0.735758 & -0.181711 & 1.996210 -- $\sum_i{y_i}^2/2N$ \\ 
4 & 0.694042 & -0.003585 & 1.999998 -- $\sum_i{y_i}^2/2N$ \\ 
5 & 0.693147 & -0.000002 & 2.000000 -- $\sum_i{y_i}^2/2N$ \\ 
\hline 
\hline
\end{tabular} 
\end{center}
\end{table}

Быстрая сходимость, как правило, возникает при использовании алгоритма Ньютона-Рафсона и при условии, что целевая функция является глобально вогнутой. На практике проблема может возникнуть из-за того, что в нестандартных нелинейных моделях целевая функция может не быть глобально вогнутой.

\subsection{Метод моментов. Оценка параметра обобщенным методом моментов}

Для М-оценок значение функции и значение градиента равны соответственно $Q_N(\theta)=N^{-1}\sum_iq_i(\theta)$ и $g(\theta)=N^{-1}\sum_{i}\partial{q_i}(\theta)/\partial{\theta}$.

Для обобщенного метода моментов (GMM) функция $Q_N({\theta})$ имеет квадратичный вид (см. раздел 6.3.2) и значение градиента вычисляется по более сложной формуле

\[
g{(\theta)}=\left[N^{-1}\sum_{i}\partial{h_i}(\theta)'/\partial{\theta}\right] \times W_N \times \left[N^{-1}\sum_{i}h_{i}(\theta)\right].
\]
Следовательно, градиентные методы, работающие для среднего не могут быть применены. В тоже время, методы, которые будут рассмотрены в разделе 10.3, такие как метод Ньютона-Рафсона, наискорейшего подъема, DFP, BFG и метод имитации отжига могут быть применены.

Метод моментов и оценка оценивающих уравнений по сути представляют собой решение системы уравнений, но в тоже время может быть переформулированы как проблемы численной оптимизации аналогично обобщенному методу моментов. Оценка параметра $\hat{\theta}$, которая является решением системы $q$ уравнений вида $N^{-1}\sum_{i}h_{i}(\theta)=0$ может быть получена путем минимизации функции $Q_N(\theta)=[N^{-1}\sum_{i}h_{i}(\theta)]'\times$ $[N^{-1}\sum_{i}h_{i}(\theta)]$.

\subsection{Критерий сходимости}

Итерационный процесс продолжается до тех пор, пока изменения не прекратятся. Программа завершает цикл когда: (1) происходят незначительные изменения целевой функции $Q_N(\hat{\theta}_s)$; (2) незначительно меняются значения вектора градиента $g_s$, относительно Гессиана; (3) происходят относительно небольшие изменения в оценка параметра $\hat{\theta}_s)$. Как правило, статистические пакеты выбирают стандартное пороговое значение относительно которого определяется изменения вышеперечисленных пунктов ((1)-(3)). Это пороговое значение получило название критерий сходимости. Значение критерия зачастую может быть изменено исследователем. В качестве консервативного порогового значения  принято брать $10^{-6}$.

Следует отметить, что, как правило, максимальное количество итераций определено заранее. После проведения максимального количества итераций выводятся результаты оценивания. Однако, полученные результаты не могут быть использованы, если не выполняется условие сходимости.

Напротив, если сходимость есть, это говорит о том, что был получен локальный максимум. В тоже время, если целевая функция не является глобально вогнутой, локальный максимум не обязательно будет глобальным максимумом.

\subsection{Исходные значения}

Количество итераций значительно сокращается в случае, если исходные значения $\hat{\theta}_1$ близки к значениям $\hat{\theta}$. Естественно, что использование состоятельных оценок  параметров  в качестве исходных значений хорошо. Неудачный выбор начальных значений может привести к тому, что  итерационный метод не сработает. В частности, для некоторых методов оценивания и градиентных методов невозможно рассчитать $g_1$ или $A_1$, если исходные значения равны нулю, $\hat{\theta}_1=0$.

В случае, если целевая функция не является вогнутой глобально необходимо использовать более множество стартовых значений для того, чтобы увеличить вероятность получения глобального максимума.

\subsection{Численная и аналитическая производные}

Любой градиентный метод по определению предполагает вычисление производных целевой функции. Как численные, так и аналитические производные могут быть использованы.

Численные производные вычисляются по формуле 

\begin{equation}
\dfrac{\Delta{Q_N}(\hat{\theta}_s)}{\Delta\theta_j} = \dfrac{Q_N(\hat{\theta}_s + he_j) --- Q_N(\hat{\theta}_s --- he_j)}{2h}, j = 1,\ldots,q
\end{equation}
где $h$ мало и $e_j = \begin{pmatrix} 0 &\ldots & 0 & 1 & 0 &\ldots & 0 \end{pmatrix}'$ вектор с 1 в $j$-ой строке и нулями во всех остальных строках.

В теории $h$ должно быть мало, поскольку теоретически $\partial{Q_N}(\theta)/\partial{\theta_j}$ равно пределу $\Delta{Q_N}(\theta)/\Delta\theta_j$ при $h \rightarrow 0$. На практике слишком малые значения $h$ приводят к ошибкам округления. В связи с этим, расчеты, в которых используется численная производная, должны быть сделаны с двойной или с четверной точностью, а не с одинарной. Хотя в программе может быть заложено начальное значение, к примеру, $h=10^{-6}$, для конкретной задачи лучше использовать свои исходные значения. Например, малые значения $h$ могут быть использованы, если переменная $y$ в нелинейной МНК регрессии измеряется не в долларах, а в тысячах долларов (при неизменности масштаба остальных переменных), тогда оценка параметра $\theta$ в тысячных.

Недостаток использования численных производных заключается в том, что производные необходимо считать много раз --- для всех $q$ параметров, $N$ наблюдений и $S$ итераций. Следовательно, необходимо оценить целевую функцию $2qNS$ раз, где каждое вычисление целевой функции может быть вычислительно трудным.

Альтернативой численным производным являются аналитические. Расчет аналитических производных сопровождается меньшим количеством вычислительных ошибок и его быстрее сделать, особенно, если производные заданы более простыми функциями, чем целевая функция. Более того, необходимо произвести расчеты только $qNS$ раз.

В методах, где требуется расчет вторых производных для построения матрицы $A_s$, использование аналитических производных является хорошей альтернативой. Даже если дана только первая аналитическая производная, вторая производная может быть посчитана быстрее и с меньшей вероятностью ошибок как численная производная от аналитической производной. В статических пакетах заложена функция расчета аналитических производных первого и второго порядка.

Расчет численных производных имеет преимущество в том, что не требует кодирования, кроме как записи целевой функции. Это сохраняет время и исключает один из возможных источников ошибки пользователей, хотя в некоторых статистических пакетах заложено вычисление аналитических производных.

Если время на проведение расчетов является критичным фактором или существуют опасения насчет точности вычислений, то стоит использовать аналитические производные. Считается хорошей практикой проверить правильность кода для аналитических производных, путем расчета оценок параметров с помощью численных производных, при этом исходные значения берутся те же, что используются при расчете аналитических производных.

\subsection{Неградиентные методы}

Применение градиентных  методов предполагает, что целевая функция должна быть достаточно гладкой, иначе градиента функции может не существовать. В некоторых случаях, таких как метод наименьших абсолютных отклонений, оценивание квантильной регрессии, метод максимального скоринга, невозможно вычислить градиент и следует применить иные итерационные методы.

Например, для оценки недифференцируемой целевой функции, такой как $Q_N(\theta_s)=N^{-1}\sum_i|y_i-x_i\beta|$ в случае наименьших абсолютных отклонений (LAD), можно использовать методы линейного программирования. Поскольку подобные примеры достаточно редко встречаются в микроэконометрике, основное внимание мы уделим градиентным методам.

Для целевых функций, задача максимизации которых может быть затруднена, например наличием нескольких локальных оптимумов, возможно применение неградиентных методов, таких как метод иммитации отжига (см. раздел 10.3.8) и генетические алгоритмы (смотри Дорси и Майер, 1995).

\section{Специальные методы}

Основным методом поиска глобального минимума вогнутой функции является итерационный метод Ньютона-Рафсона. Иные методы, такие как метод наискорейшего спуска и DFP, как правило, используются, если невозможно применить метод Ньютона-Рафсона. Для нелинейного МНК широкое распространение получил метод Гаусса-Ньютона. Однако, данный метод не такой универсальный  как  метод Ньютона-Рафсона, поскольку с помощью метода Гаусса-Ньютона можно решить только задачи наименьших квадратов и, кроме того, данный метод можно получить небольшим изменением метода Ньютона-Рафсона. Вышеперечисленные методы направлены на нахождение локального оптимума на основе стартовых значений параметров модели.

В данном разделе описан метод ожиданий, который хорошо применим при наличии пропусков в данных, а также метод имитации отжига, Последний является неградиентным методом и, как правило, позволяет найти глобальный, а не локальный максимум.


\subsection{Метод Ньютона-Рафсона}

Метод Ньютона-Рафсона (NR) один из наиболее частоприменяемых градиентных методов особенно, если целевая функция глобально вогнутая по $\theta$. В  методе Ньютона-Рафсона

\begin{equation}
\hat{\theta}_{s+1} = \hat{\theta}_s --- H^{-1}_sg_s,
\end{equation}
где $g_s$ определено уравнением (10.2) и

\begin{equation}
H_s = \left. \dfrac{\partial^2Q_N(\theta)}{\partial\theta\partial\theta'}\right|_{\hat{\theta}_s}
\end{equation}
является матрицей Гессе размера $q{\times}q$ оцененной в точке $\hat{\theta}_s$. Вышеприведенные формулы используются как для задач максимизации, так и для задач минимизации функции $Q_N (\theta)$, для решения задачи минимизации целевая функция $Q_N (\theta)$ домножается на минус единицу, что меняет знак перед $H^{-1}_s$ и $g_s$.

Чтобы объяснить метод Ньютона-Рафсона, начнем  с оценки параметра $\theta$ на $s$-ой итерации, т.е. с $\hat{\theta}_s$. Тогда, согласно согласно разложению в ряд Тейлора в окрестности точки $\hat{\theta}_s$, получим

\[
Q_N(\theta) = \left. Q_N(\hat{\theta}_s)+\dfrac{\partial{Q_N(\theta)}}{\partial{\theta}'}\right|_{\hat{\theta}_s}(\theta --- \hat{\theta}_s)+\left. \dfrac{1}{2}(\theta --- \hat{\theta}_s)'
\dfrac{\partial^2{Q_N(\theta)}}{\partial{\theta}\partial{\theta}'}\right|_{\hat{\theta}_s}(\theta --- \hat{\theta}_s)+R.
\]
Опуская остаточный член $R$ и используя более краткие обозначения, получим следующее выражение для приближенного значения $Q_N(\theta)$

\[
Q^*_N(\theta) = Q_N(\hat{\theta}_s)+g'_s(\theta --- \hat{\theta}_s)+\dfrac{1}{2}(\theta --- \hat{\theta}_s)'H_s(\theta --- \hat{\theta}_s).
\]
где $g_s$ и $H_s$ определены выражениями (10.2) и (10.6), соответственно. Для максимизации $Q^*_N(\theta)$ по $\theta$ приравниваем производную к нулю. Таким образом, $g_s+H_s(\theta --- \hat{\theta}_s) = 0$, и решая уравнение для $\theta$ получим $\hat{\theta}_{s+1} = \hat{\theta}_s-H^{-1}_sg_s$, что эквивалентно выражению (10.5). Таким образом, обновленный метод Ньютона-Рафсона решает задачу максимизации для функции $Q_N(\theta)$ разложенную в ряд Тейлора второго порядка в точке $\hat{\theta}_s$.

Для того, чтобы проверить увеличивают ли итерации Ньютона-Рафсона значения функции $Q_N(\theta)$ подставим значение оценки параметра на $(s+1)$-м шаге в разложение в ряд Тейлора, тогда

\[
Q_N(\hat{\theta}_{s+1}) = Q_N(\hat{\theta}_s) --- \dfrac{1}{2}(\hat{\theta}_{s+1} --- \hat{\theta}_s)'H_s(\hat{\theta}_{s+1} --- \hat{\theta}_s)+R.
\]
Опустив остаточный член, увидим, что значение функции будет возрастать (убывать), если $H_s$ положительно определена (отрицательно определена). В точке локального максимума матрица Гессе отрицательно полуопределена, однако вдалеке от точки максимума, это не всегда верно даже для корректно поставленных задач. Если метод Ньютона-Рафсона попадает в такую область, то необязательно, что он движется в направлении максимума. Кроме того, это приводит к вырожденности матрицы Гессе, следовательно, невозможно посчитать $H^{-1}_s$ в (10.5). Таким образом, метод Ньютона-Рафсона лучше всего работает при решении задач максимизации (минимизации), при условии, что целевая функция глобально вогнута (выпукла), поскольку $H_s$ всегда отрицательно-(положительно-) определена. В этом случае сходимости часто можно достигнуть уже после 10 итераций.

Дополнительное преимущество метода Ньютона-Рафсона обнаруживается, если исходное значение параметра $\hat{\theta}_1$ $\sqrt{N}$-состоятельная оценка, т.е. если выражение $\sqrt{N}(\hat{\theta}_1 --- \theta_0)$ имеет соответствующее предельное распределение. Тогда можно продемонстрировать, что оценка после второй итерации $\hat{\theta}_2$ имеет такое же асимптотическое распределение как и оценка, полученный при итерировании до достижения сходимости. Следовательно, не существует теоретической выгоды от продолжения итераций. Примером является  доступный обобщенный метод наименьших квадратов (ДОМНК), где начальные значения МНК приводят к состоятельным оценкам параметров регрессии, и в дальнейшем, эти значения используются для получения состоятельных оценок дисперсий параметров, а те, в свою очередь, --- для расчета эффективных оценок ОМНК. Вторым примером является использование легко получаемых состоятельных оценок в качестве исходных значений перед тем как максимизировать сложную функцию максимального правдоподобия. Несмотря на то что необходимость дальнейшего итерирования отсутствует, на практике большинство исследователей предпочитают итерировать до сходимости, если это не занимает существенно больше времени. Преимуществом итерирования до сходимости является то, что разные исследователи должны получить при этом одинаковые оценки параметров, в то время как различные начальные значения $\sqrt{N}$-состоятельных оценок приводят к разным значениям оценок после второй итерации, несмотря на то что последние асимптотически эквивалентны.

\subsection{Метод скоринга}

Наиболее часто используемая модификация Ньютона-Рафсона метода --- это метод скоринга (method of scoring, MS). В методе скоринга матрица Гессе заменяется на свое ожидаемое значение

\begin{equation}
H_{MS,s} =\left. \E \left[\dfrac{\partial^2 Q_N(\theta)}{\partial{\theta}\partial{\theta}'}\right] \right|_{\hat{\theta}_s}. 
\end{equation}
Такая замена особенно выгодна, когда применяется метод максимального правдоподобия, т.е. $Q_N(\theta) = N^{-1}\mathcal{L}_N(\theta)$, поскольку математическое ожидание  матрицы должно быть отрицательно определено, что следует из равенства информационных матриц (см. раздел 5.6.3), $-H_{MS,s} = \E [\partial{\mathcal{L}_N}/\partial{\theta} \times \partial{\mathcal{L}_N}/\partial{\theta}']$. % похоже в английском тексте пропущен минус
Эта матрица положительно определена, так как она является ковариационной матрицей. Получить значение математического ожидания в (10.7) возможно только для М-оценок и даже в этом случае при расчете могут возникнуть аналитеческие затруднения.

Метод скоринга для оценки методом максимального правдоподобия обобщенных линейных моделей, таких как модель Пуассона, пробит и логит-модели может быть реализован с использованием итерационного взвешенного МНК (см. МакКуллах и Нелдер, 1989). Ранее, когда исследователи имели доступ только к программам оценивающим МНК, этот способ имел  существенное достоинство.

Метод скоринга также может быть применен для оценки другим М-оценкам, а  не только к оценкам ММП, в этом случае матрица $H_{MS,s}$ может уже не быть отрицательно определенной.


\subsection{BHHH метод}

BHHH метод (Берндт, Холл, Холл и Хаусман, Berndt, Hall, Hall и Hausman, 1974) используют (10.1) с взвешивающей матрицей $A_s = -H^{-1}_{BHHH,s}$, где 

\begin{equation}
H_{BHHH,s} = \left. -\sum^N_{i=1}\dfrac{\partial{q_{i}}(\theta)}{\partial\theta}\dfrac{\partial{q_{i}}(\theta)}{\partial{\theta}'} \right|_{\hat{\theta}_s}
\end{equation}
и $Q_N(\theta) = \sum_{i}q_{i}(\theta)$. В сравнении с методом Ньютона-Рафсона, BHHH метод имеет преимущество в том, что необходимо считать производные только первого порядка, что значительно уменьшает количество вычислений.

Для обоснования использования данного метода, начнем с метода скоринга для ММП. Тогда $Q_N(\theta) = \sum_{i} \ln f_{i}(\theta)$, где $f_i(\theta)$ является логарифмом плотности. Информационное матричное равенство можно записать следующим образом

\[
\E\left[ \dfrac{\partial^2\mathcal{L}_N(\theta)}{\partial{\theta}\partial{\theta}'}\right] = -\E\left[ \sum^N_{i=1}\dfrac{\partial{\ln f_i(\theta)}}{\partial{\theta}}\sum^N_{i=1}\dfrac{\partial{\ln f_j(\theta)}}{\partial{\theta}'}\right] 
\]
и независимость по $i$ приводит к
\[
\E\left[ \dfrac{\partial^2{\mathcal{L}_N(\theta)}}{\partial{\theta}\partial{\theta}'}\right] = -\sum^N_{i=1} \E\left[ \dfrac{\partial{\ln f_i(\theta)}}{\partial{\theta}} \dfrac{\partial {\ln f_i(\theta)}}{\partial{\theta}'}\right]. 
\]
Отбрасывание математического ожидания приводит к выражению (10.8).

Можно также применять BHHH метод к оценкам, отличным от оценок ММП. В этом случае его можно рассматривать как метод, который предлагает другой вариант матрицы $A_s$ в (10.1), отличный от оценки матрицы Гессе $H_s$.

BHHH метод часто применяют для М-оценок пространственных данных, так как они обладают хорошими свойствами и требуют расчета только первых производных.

\subsection{Метод скорейшего подъема}

Метод скорейшего подъема предполагает использование матрицы $A_s = I_q$, использование единичной матрицы является самым простым вариантом взвешивающей матрицы. Далее выполняется линейный поиск (см. раздел 10.3) чтобы отмасштабировать единичную матрицу $I_q$ на константу $\lambda_s$.

Линейный поиск может быть сделан вручную. На практике принято использовать оптимальное значение $\lambda$, которое определяется как $\lambda_s= -g'_s g_s / g'_s H_s g_s$, где $H_s$ --- матрица Гессе. Для определения оптимального значения $\lambda_s$ необходимо вычисление Гессиана, что может вызвать затруднения. В таком случае, метод наискорейшего спуска можно заменить методом Ньютона-Рафсона. Однако, преимущество метода скорейшего подъема состоит в том, что $H_s$ может быть вырожденной матрицей, тем не менее необходимо, чтобы $H_s$ была отрицательно определена чтобы $\lambda_s<0$, и матрица  $\lambda_s I_s$ была отрицательно определена.

\subsection{DFP метод и BFGS метод}

Алгоритм DFP (Дэвидсон, Флетчер и Пауэлл, Davidson, Fletcher и Powell) --- градиентный метод с положительно-определенной взвешивающей матрицей $A_s$. Также, требует расчета только первых производных, в отличии от NR, который предполагает расчет Гессиана. В данном пункте метод DFP описан без доказательств.

Взвешивающая матрица $A_S$ рассчитывается рекурсивным способом

\begin{equation}
A_s = A_{s-1} + \dfrac{\delta_{s-1}\delta'_{s-1}}{\delta'_{s-1}\gamma_{s-1}} + \dfrac{A_{s-1}\gamma_{s-1}\gamma'_{s-1}A_{s-1}}{\gamma'_{s-1}A_{s-1}\gamma_{s-1}},
\end{equation}
где $\delta_{s-1} = A_{s-1}g_{s-1}$ и $\gamma_{s-1} = g_s --- g_{s-1}$. Если посмотреть на правую часть выражения (10.9), можно увидеть, что матрица $A_s$ будет положительно-определена в случае, если начальная матрица $A_0$ будет положительно-определена (например, $A_0 = I_q$).

Процедура сходится быстро во многих статистических приложениях. В результате матрица $A_s$ сходится к матрице Гессе $- H^{-1}_s$ --- более предпочтительной с точки зрения теории. В принципе данный метод может позволять получить приближенную оценку обратной матрицы к матрице Гессе для расчета стандартных ошибок, без расчета второй производной или обратной матрицы. Однако, полученная оценка может быть довольно плохой на практике.

Уточненным алгоритмом DFP является алгоритм BFGS (Бойден, Флетчер, Голдфарб и Шеннон, Boyden, Fletcher, Goldfarb и Shannon), с

\begin{equation}
A_s = A_{s-1}+\dfrac{\delta_{s-1}\delta'_{s-1}}{\delta'_{s-1}\gamma_{s-1}}+\dfrac{A_{s-1}\gamma_{s-1}\gamma'_{s-1}A_{s-1}}{\gamma'_{s-1}A_{s-1}\gamma_{s-1}}
-(\gamma'_{s-1}A_{s-1}\gamma_{s-1})\eta_{s-1}\eta'_{s-1},
\end{equation}
где $\eta_{s-1} = (\delta_{s-1}/\delta'_{s-1}\gamma_{s-1})-(A_{s-1}\gamma_{s-1}/\gamma'_{s-1}A_{s-1}\gamma_{s-1})$.


\subsection{Метод Гаусса-Ньютона}

Метод Гаусса-Ньютона --- итерационный метод для получения оценок нелинейного МНК (НМНК), в котором оценки получаются путем нескольких итераций с помощью обычного МНК. 

Для нелинейного МНК с функцией условного среднего $g({x_i},\beta)$, метод Гаусса-Ньютона берёт в качестве вектора разностей оценок параметров $(\hat{\beta}_{s+1}-\hat{\beta}_s)$ оценки коэффициентов, полученные путем оценки МНК вспомогательной регрессии

\begin{equation}
y_{i}-g(x_{i},\hat{\beta_s})= \left. \frac{\partial{g_i}}{\partial{\beta}'} \right|_{\hat{\beta}_s} \beta+v_{i}
\end{equation}
Также оценку коэффициента $\hat{\beta}_{s+1}$, можно получить применив МНК для вспомогательной регрессии

\begin{equation}
\left. y_{i}-g(x_{i},\hat{\beta}_s)-\dfrac{\partial{g_i}}{\partial{\beta}'} \right|_{\hat{\beta_s}}\hat{\beta}_s= \left. \frac{\partial{g_i}}{\partial{\beta}'} \right|_{\hat{\beta}_s} \beta+v_{i}
\end{equation}

Для объяснения данного метода, предположим, что $\hat{\beta}_s$ является начальным значением и разложим функцию $g(x_{i},\beta)$ в ряд Тейлора первого порядка в окрестности данной точки. Получим, что

\[
g(x_{i},\beta) = \left. g(x_i,\hat{\beta}_s)+\dfrac{\partial{g_i}}{\partial{\beta}'} \right|_{\hat{\beta}_s}(\beta-\hat{\beta}_s).
\]
Подставим данное значение в целевую функцию $Q_N (\beta)$ и получим приближенное значение функции

\[
Q^*_N(\beta) =  \sum^N_{i=1} \left(y_{i}-g(x_{i},\hat{\beta}_s) --- \left. \dfrac{\partial{g_i}}{\partial{\beta}'} \right|_{\hat{\beta}_s} (\beta-\hat{\beta}_s) \right)^2.
\]
Данное выражение эквивалентно сумме квадратов остатков, полученных с помощью МНК при  регрессии $y_{i}-g(x_{i},\hat{\beta}_s)$ на $\left. \dfrac{\partial{g_i}}{\partial{\beta}'} \right|_{\hat{\beta}_s}$ с вектором параметров $(\beta-\hat{\beta}_s)$, что в результате дает выражение (10.11). Более точно,

\begin{equation}
\hat{\beta}_{s+1} = \hat{\beta}+\left[ \sum_i \left. \dfrac{\partial{g_i}}{\partial{\beta}} \right|_{\hat{\beta}_s} \left. \dfrac{\partial{g_i}}{\partial{\beta}'} \right|_{\hat{\beta}_s} \right] ^{-1} \left. \sum_i \dfrac{\partial{g_i}}{\partial{\beta}} \right|_{\hat{\beta}_s} (y_i-g(x_i,\hat{\beta}_s)).
\end{equation}
Данное выражение есть ни что иное как градиентный метод (10.1), где вектор $g_s$ задан выражением $g_s=\sum_i \partial{g_i} / \partial{\beta}|_{\hat{\beta}_s} (y_i-g(x_i,\hat{\beta}_s))$, а матрица $A_s$ равна $A_s = [\sum_i \partial{g_i} / \partial{\beta} \times \partial{g_i} / \partial{\beta}'|_{\hat{\beta}_s}]^{-1}$.

Оценка (10.13), рассчитанная итерационным методом эквивалента оценке, вычисленной по скоринговому варианту метода Ньютона-Рафсона для нелинейного МНК (см. раздел 5.8). В правой части выражения вторая сумма есть вектор градиента, а первая сумма равна  ожидаемому значению матрицы Гессе, домноженному на -1 (см. раздел 10.3.9). Таким образом, алгоритм Гаусса-Ньютона является частным случаем алгоритма Ньютона-Рафсона. В этой главе наибольшее внимание уделено методу Ньютона-Рафсона, поскольку он более универсален, чем метод Гаусса-Ньютона.

\subsection{Метод максимизации ожидания}

В этой книге определенный класс данных и моделей можно представить с помощью пропущенных или неполных данных. Например, зависимая переменная (издержки или промежуток времени) может быть цензурирована справа. Следовательно, точные значения переменных в определенный момент времени известно лишь частино, в остальных случаях нам известно, что результат не превысит заданного значения, к примеру, $c^{*}$. Также, примером неполных данных может служить множественная модель для которой матрица данных имеет вид:

\[
\begin{bmatrix} y_{i}&X_{1} \\ ?&X_{2} \end{bmatrix},
\]
где знак вопроса обозначает пропущенные данные. Далее предположим, что мы хотим оценить линейную регрессионную модель $y=X\beta+u$, где $y'= \begin{bmatrix} y_1 & ? \end{bmatrix}$, $X'= \begin{bmatrix} X_1 & X_2 \end{bmatrix}$, и часть данных зависимой переменной ненаблюдаемы. Также, примером проблемы ненаблюдаемых переменных может быть оценка параметров $(\theta_1,\theta_2,\ldots ,\theta_c,\pi_1,\ldots ,\pi_C)$ смеси $C$ распределений, так называемая модель со скрытыми или латентными классами, $h(y|X)=\sum_{j=1}^{C}\pi_{j}f_{j}(y_i|X_{j},\theta_j)$, где $f_{j}(y_{j}|X_{j},\theta_{j})$ --- функция плотности. Переменная $\pi_{j}$ $(j=1,\ldots ,C)$ обозначает неизвестную долю с которой в выборке встречается $j$-ый класс из всех $C$ возможных классов. Эта проблема может также быть отнесена к проблеме скрытых данных, в том смысле, что если бы были заданы $\pi_{j}$, то оценку модели было бы проще произвести.

Метод максимизации ожидания (EM, expectation maximization) рассматривает все задачи с ненаблюдаемыми переменными в рамках универсального подхода. Частные случаи моделей ненаблюдаемых переменных изложены в литературе достаточно давно, а Демпстер, Лэйрд и Рубин (1977) приводят систематическое изложение.

Обозначим за $y$ вектор зависимых переменных, а за $y^{*}$ скрытые переменные. Пусть $f^{*}(y^{*}|X,\theta)$ условная плотность совместного распределения ненаблюдаемых переменных, при  заданном $X$ и $f(y|X,\theta)$ обозначает условную совместную плотность наблюдаемых переменных. Допустим, что между $y$ и $y^{*}$ отношения много к одному, т.е. для каждого $y$ существует уникальное значение $y^{*}$, но обратное не верно. Согласно правилу Байеса условная плотность $f(y^{*}|y)=f(y, y^{*})/f(y)=f^{*}(y^{*})/f(y)$, где последнее равенство получается из равенства $f(y^{*},y)=f^{*}(y^{*})$, поскольку каждому $y$ соответствует уникальное значение $y^{*}$, и, следовательно, $f(y|X,\theta)=f^{*}(y^{*}|X,\theta)/f(y^{*}|y,X,\theta)$. В результате приведения, получим выражение $f(y)=f^{*}(y^{*})/f(y^{*}|y)$.

ММП максимизирует значение функции

\begin{equation}
Q_{N}(\theta)=\dfrac{1}{N}\mathcal{L}_{N}(\theta)=\dfrac{1}{N} \ln (f^{*}y^{*}|X,\theta)-\dfrac{1}{N} \ln f(y^{*}|y,X,\theta).
\end{equation}
Поскольку значения $y^{*}$ неизвестны, первое слагаемое опускается, от второго члена берется математическое ожидание. В выражении для математического ожидания отсутствует $y^{*}$ и значение выражения рассчитывается на $s$-той итерации в точке $\theta=\hat{\theta}_s$.

На <<Е>>-шаге EM алгоритма, ожидание $\E$ рассчитывается по формуле:

\begin{equation}
Q_{N}(\theta)=-\E\left[\dfrac{1}{N} \ln f(y^{*}|y,X,\theta)|y,X,\hat{\theta}_s\right],
\end{equation}
где ожидание берется по плотности $f(y^{*}|y,X,\hat{\theta}_s)$. На <<М>>-шаге максимизируется значение функции $Q_{N}(\theta|\hat{\theta}_s)$, чтобы получить $\hat{\theta}_{s+1}$.

EM --- это итерационный алгоритм. Максимизируется функции правдоподобия при заданных значениях ненаблюдаемых (скрытых) переменных; далее вновь оцениваются ожидаемые значения при заданных значениях $\theta$. Итерирование продолжается до наступления сходимости. Преимущество EM алгоритма заключается в том, что в результате значения функции $Q_{N}(\theta)$ увеличиваются или становятся менее изменчивыми, см. Амэмия (1985, стр. 376). Применение EM алгоритма к моделям с латентными классами рассмотрено в разделе 18.5.3 и к моделям с пропущенным данным данными в разделе 27.5.

В литературе широко представлено применение EM алгоритма к решению задач оптимизации, хотя и не ко всем задачам он применим. EM алгоритм можно легко запрограммировать для многих случаев, и его использование было подкреплено идеей ограниченной возможности расчетов, что в наше время уже не является актуальным. Тем не менее, для моделей с цензурированными данными и моделей со скрытыми переменными наиболее быстрым и эффективным, с точки зрения объема вычислений, считается итерационный метод Ньютона-Рафсона.

\subsection{Метод имитации отжига}

Метод имитации отжига относится к неградиентным итерационным методам, которые были рассмотрены Гоффе, Феррье и Роджерс (1994). В отличие от градиентных методов метод имитации отжига допускает снижение целевой функции, а не только её рост. Таким образом метод не попадает в ловушку постепенного движения в сторону ближайшего локального экстремума.

При заданных значениях $\hat{\theta}_s$ на $s$-том шаге меняем значение $j$-того компонента вектора $\hat{\theta}_s$ для получения новых пробных значений $\theta^{*}$

\begin{equation}
\theta^{*}_{s}=\hat{\theta}_s+ \begin{bmatrix} 0 \cdots 0 & (\lambda_{j}r_{j}) & 0 \cdot 0 \end{bmatrix}',
\end{equation}
где $\lambda_j$ это зафиксированная длина шага, а $r_j$ значение, взятое из равномерного распределения $(-1;1)$. Далее используется новое значение $\hat{\theta}_{s+1}=\theta_s^{*}$, если происходит увеличение целевой функции или, при отсутствии увеличения, проверяется  критерий Метрополиса

\begin{equation}
\exp \left( (Q_{N}(\theta^{*}_s)-Q_{N}(\hat{\theta}_s))/T_s \right) >u,
\end{equation} 
где $u$ генерируется из равномерного распределения $(0,1)$ и $T_s$ --- коэффициент масштабирования, называемый температурой. Следовательно, допустимо не только увеличение, но и уменьшение функции, с вероятностью которая падает с ростом разницы $Q_{N}(\theta^{*}_s)$ и $Q_{N}(\hat{\theta}_s)$, и растёт с ростом температуры. Понятия имитации отжига и температуры взяты из физики, по аналогии с понятием минимизации расходования тепловой энергии за счет медленного охлаждения (отжига) расплавленного металла. 

Тестирующему необходимо установить значение параметра $\lambda_{j}$. Гоффе и др. (1994) предложили периодически корректировать значения $\lambda_{j}$ так, чтобы $50\%$ всех перестановок были приняты. Значение температуры тоже нужно выбирать, и понижать температуру в ходе всего процесса итерирования. Таким образом, алгоритм заключается в поиске решения из широкого диапазона параметров, прежде чем сосредоточиться в узком диапазоне.

Метод ускоренного отжига (fast simulated annealing, FSA), предложен Сзу и Хартли (1987). Согласно этому методу случайная переменная $r_j$, значение которой равномерно и одинаково распределено на отрезке $(-1,1)$, заменяется на случайную переменную с распределение Коши $r_j$, домноженную на температуру; длину шага, $v_j$ можно фиксировать. Значение температуры $T_s$ равно значению начальной температуры, деленному на количество итераций алгоритма, где одна итерация подразумевает полный цикл по $q$ компонентам вектора $\theta$.

Кэмерон и Йоханссон (1997) исследовали и применяли метод имитации отжига в соответствии с методикой Хоровица (1992). Авторы начинали с метода ускоренного отжига и из соображений экономии времени переходили к градиентным методам (BFGS), когда незначительно меняется значение функции $Q_{N}(\cdot)$ при нескольких итерациях отжига подряд, или при большом (250) количестве итераций. В результате исследования авторы сделали вывод, что метод Ньютона-Рафсона с разными начальными значениями дает лучшие результаты, чем Ньютона-Рафсона с одним начальным значением,  а ускоренный отжиг  с разными начальными значениями --- еще лучше.

\subsection{Пример экспоненциальной регрессии}

Рассмотрим нелинейную регрессионную модель с экспоненциальным условным средним

\begin{equation}
\E[y_i|x_i] = \exp(x'_i\beta),
\end{equation}
где $x_i$ и $\beta$ --- векторы размерностью $K \times 1$. Оценка $\hat{\beta}$ нелинейного МНК минимизирует

\begin{equation}
Q_N(\beta) = \sum_i(y_i --- \exp(x'_i\beta))^2,
\end{equation}
где для упрощения обозначений умножение на $2/N$ опускается. Условия первого порядка нелинейны относительно параметра $\beta$ и нет явного решения для $\beta$. Следовательно, можно использовать градиентный метод.

Получим, что градиент и гессиан соответственно равны:

\begin{equation}
g = -2\sum_i(y_i --- e^{x'_i\beta})e^{x'_i\beta}x_i
\end{equation}
и
\begin{equation}
H = 2\sum_i\lbrace{e^{x'_i\beta} e^{x'_i\beta} x_{i}x'_{i} --- 2(y_i --- e^{x'_i \beta})e^{x'_i\beta} x_{i}x'_{i}} \rbrace.
\end{equation}
Значения $g_s$ и $H_s$, которые заданы выражениями (10.20) и (10.21), соответственно, в итерационном методе Ньютона-Рафсона (10.5) оцениваются в точке $\hat{\beta}_s$.

Более простой вариант метода Ньютона-Рафсона --- метод скоринга основан на том, что из выражения (10.18) следует что

\begin{equation}
\E[H] = 2\sum_{i} e^{x'_i\beta}e ^{x'_i\beta} x_{i}x'_{i}.
\end{equation} 

Используя $\E[H_s]$ вместо $H_s$, получим

\[
\hat{\beta}_{s+1} --- \hat{\beta}_s =\left[\sum_{i}e^{x'_i \hat{\beta}_s} e^{x'_i \hat{\beta}_s} x_{i}x'_{i} \right]^{-1} \sum_{i} e^{x'_{i}\hat{\beta}_s} x_{i} (y_i --- e^{x'_{i}\hat{\beta}_s}).
\]
Из этого следует, что $\hat{\beta}_{s+1} --- \hat{\beta}_s$ может быть посчитана с помощью МНК регрессии $(y_i --- e^{x'_i\hat{\beta}_s})$ на $e^{x'_i\hat{\beta}_s}x_i$. Что также эквивалентно регрессии Гаусса-Ньютона (10.11), поскольку $\partial{g(x_{i},\beta)}/\partial{\beta} = \exp(x'_i\hat{\beta}_s)x_i$ для экспоненциального условного среднего (10.18). Если взять $\exp(x'_i\beta) = \exp(\beta)$, то результат будет  таким же, как и в разделе 10.2.4. 

\section{Практические рекомендации}

Некоторые аспекты применения методов на практике были рассмотрены в разделе (10.2), например, критерий сходимости, модификации в виде адаптации размера шага $h$ и замена аналитической производной численной. В данном разделе представлен краткий обзор статистических пакетов, а также обсуждаются наиболее часто встречающиеся ошибки при расчете нелинейных оценок. 

\subsection{Пакеты статистических данных}

Во всех стандартных микроэконометрических пакетах, таких как Limdep, Stata, PCTSP, и SAS, заложены алгоритмы оценки стандартных нелинейных моделей, например, логит и пробит. Данные пакеты достаточно просты в применении и не требуют специальных знаний об итерационных методах или даже о модели, которая используется. Например, для логит модели может использоваться команда <<logit y x>> (по аналогии как для МНК <<ols y x>>). При записи нелинейного МНК необходимо задать функциональную форму $g(y,x)$. Оценивание должно быть быстрым и точном, для этого программа должна использовать особенности структуры модели. Например, для оценки вогнутой функции можно использовать метод скоринга.

В случае, если в статистическом пакете не предусмотрена функция для оценки отдельной модели, то необходимо писать дополнительный программный код. Необходимость в написании программы может возникнуть даже при малейших изменениях стандартной модели. Например, при наложении ограничений на параметры или использовании многоиндексных функциональных форм. Код можно написать, используя любой удобный статистический пакет или более специализированный язык программирования. Возможные варианты включают $(1)$ встроенные в статистический пакет процедуры оптимизации, которые требуют спецификации целевой функции и, возможно, ее производных; $(2)$ встроенные в статистический пакет матричные команды для расчета $A_s$, $g_s$ и итераций; $(3)$ матричный язык программирования такой, как Gauss, Matlab, OX, SAS/IML или S-Plus, и, возможно, дополнительные пакеты для оптимизации; (4) язык программирования такой, как  Fortran или C++; и $(5)$ пакет для оптимизации такой, как GAMS, GQOPT, or NAGLIB.

Первый и второй методы привлекательны, потому что они не вынуждают пользователя изучать новую программу. Первый метод особенно прост для М-оценивания, так как он требует только спецификацию подфункции $q_i(\theta)$ для $i$-того наблюдения, а не спецификации $Q_N(\theta)$. Однако на практике процедуры оптимизации для функций, которые задаются пользователем, в стандартных пакетах скорее столкнутся с проблемами расчетов, чем в случае применения специализированных программ. Более того, для некоторых пакетов второй метод может требовать изучения сложных форм матричных команд.

Для нелинейных задач третий метод является наилучшим, хотя это может требовать, чтобы пользователь изучил язык программирования с азов. В таком случае можно решить почти любую эконометрическую задачу, а методы оптимизации, которые используются в языках программирования с помощью матриц, обычно хорошие. Более того, многие авторы делают доступным код, который используется в своих работах.

Четвертный и пятый методы обычно требуют более высокого уровня сложности программ, чем третий метод. Четвертый метод может привести к более быстрым расчетам, в то время как пятый может решить большинство сложных с точки зрения расчетов оптимизационных задач.

Другие практические вопросы касаются стоимости программного обеспечения, также программного обеспечения, которым пользуются коллеги, и того факта, есть ли у этого программного обеспечения четкие сообщения об ошибках и полезные инструменты, которые позволяют бороться с ошибками, такие, как инструмент, который отслеживает пошаговое исполнение команд. Также не стоит недооценивать важность того, каким программным обеспечением пользуются коллеги.

\begin{table}[h]
\begin{center}
\caption{\label{tab:troubles} Сложности при расчетах: перечень}
\begin{tabular}{ll}
\hline 
\hline
Проблема & Проверка \\ 
\hline 
Неверно загружены данные & Посмотрите на все описательные статистики полностью. \\
Неточность расчета & Используйте аналитические производные или численные \\
& производные с другим размером шага $h$. \\
Мультиколлинеарность & Проверьте собственные значения матрицы $X'X$. \\
& Попробуйте использовать подмножества регрессоров. \\ 
Вырожденная матрица при итерациях & Попробуйте метод, который не требует обращения \\
& матрицы (например, DFP). \\
Плохие начальные значения & Попробуйте различные начальные значения. \\
Модель не идентифицируема & Тяжело проверить. Очевидная ловушка --- излишняя \\
& дамми-переменная. \\
Странные значения параметров & Включена ли константа? Сошелся ли алгоритм \\
& после интераций? \\
Разные стандартные ошибки & Какой метод был применен для расчета \\
& ковариационной матрицы? \\
\hline 
\hline
\end{tabular} 
\end{center}
\end{table}

\subsection{Вычислительные трудности}

Вычислительные трудности на практике проявляются в невозможности получения оценок параметров. Например, в сообщении об ошибке может быть сказано, что расчет невозможно произвести из-за вырожденности матрицы Гессе. Существует множество возможных причин вырожденности матрицы Гессе, информация представлена в таблице 10.2. Эти причины могут также выступать объяснением для аналогичных ситуаций, когда параметр оценен, но оценка явно является ошибочной. 

Во-первых, данные могут быть неверно прочитаны в память. Это достаточно распространенная ошибка. При обработке большого массива данных, не целесообразно выводить все данные на печать. Тем не менее, следует внимательно посмотреть описательные статистики и проверить правильность определения диапазона значений --- наличие крайне больших или маленьких значений среднего, а также очень большого или малого стандартного отклонения (в т.ч. равенство стандартного отклонения нулю, что будет означать отсутствие дисперсии). Более подробно этот вопрос рассмотрен в разделе 3.5.4.

Во-вторых, возможны ошибки вычислений. Для минимизации ошибок вычислений все расчеты должны производиться с двойной или даже с четверной точностью. В некоторых случаях следует изменить масштаб данных так, чтобы среднее и дисперсия независимых переменных были одного порядка. Например, может иметь смысл рассчитывать ежегодный доход в тысячах долларов, а не в долларах. При использовании численных производных может понадобиться изменить значение $h$ в (10.4). Следует обращать внимание какой метод применялся для вычисления функции. Например, функцию $\ln \Gamma(y)$, где $\Gamma(\cdot)$ --- гамма-функция, лучше считать с помощью специальной лог-гамма функции. Если сначала посчитать обычную гамма-функцию, а затем взять логарифм, то можно получить значительные вычислительные ошибки даже при небольшом  $y$.

В-третьих, может возникнуть проблема мультиколлинеарности. В одноиндексных моделях (см. раздел 5.2.4) мультиколлинеарность проверяется по стандартной процедуре. Можно посмотреть на корреляционную матрицу регрессоров, но она учитывает только попарные корреляции. Лучше всего использовать индекс обусловленности матрицы $X'X$,  равный корню из отношения наибольшего и наименьшего собственных значений матрицы $X'X$. Если отношение больше 100, тогда существует большая вычислительных трудностей. Для более сложных, чем одноиндексные,  нелинейных моделей, проблемы могут проявиться даже при небольшом значении индекса обусловленности. Если предполагается что мультиколлинеарность может привести к проблемам  вычислений, тогда необходимо рассмотреть возможность оценить модель используя только часть переменных, для которых ниже вероятность мультиколлинеарности. 

В-четвертых, из необратимости Гессиана во время итерацией не следует его сингулярность в точке экстремума. Для решения проблемы, наряду с методом Ньютона-Рафсона, возможно произвести расчет различными итерационными методами, например, методом наискорейшего подъема с линейным поиском или методом DFP. Отсутствие единственности максимума может быть вызвано мультиколлинеарностью.

В-пятых, возможно менять начальные значения. Итерационные градиентные методы ориентированы на расчет локального, а не глобального максимума. Для борьбы с этим возможно проводить итерации  при различных начальных значениях. Вторым способом является поиск на сетке. Теоретически оба подхода предполагают оценивание целевой функции в большом количестве точек, если  размерность вектора $\theta$ велика, однако в некоторых случаях достаточно провести детальный анализ упрощенной модели, где рассматриваются только несколько наиболее важных регрессоров.

Вместе с тем может возникнуть проблема идентифицируемости модели. Необходимым условием идентифицируемости модели является обратимость матрицы Гессе. Как и в случае линейной модели, нужно проверить, не столкнулись ли мы с ловушкой дамми-переменных, а при использовании части выборки также убедиться, что каждая из переменных используемого набора непостоянна.  Допустим, что данные отсортированы по полу, возрасту или региону, тогда проблема идентификации может возникнуть, если перечисленные переменные используются в виде дамми, а выборка состоит из данных по индивидам определенного пола, возраста или региона. Иногда для нелинейных моделей трудно определить, является ли модель идентифицируемой теоретически. Как правило в такой ситуации сначала пытаются исключить все другие потенциальные проблемы, а затем уже проводить теоретический анализ модели на идентифицируемость.

Даже после получения оценок параметров, возможно проявление проблем, поскольку в некоторых случаях невозможно рассчитать ковариационную матрицу, $A^{-1}BA'^{-1}$. Оценка ковариационной матрицы, $A^{-1}BA'^{-1}$ может не получаться, если в итерационном методе матрица Гессе $A^{-1}$ не используется в качестве взвешивающей матрицы при итерировании, как, например, в алгоритме DFP. Во-первых, необходимо проверить, что итерационный процесс сходится, а не останавливается из-за превышения количества допустимых итераций. Если итерационный процесс сходится, то следует рассчитать альтернативную оценку матрицы $A$, например, используя ожидаемый гессиан или более точные способы вычислений, можно попробовать использовать аналитическую производную вместо численной. Если и эти решения не помогают, то вычислительные трудности могут свидетельствовать о неидентифицируемости модели, бывает что неидентифицируемость проявляется на этапе вычисления оценки при использовании итерационного метода, не использующего матрицу Гессе.

Вычислительной трудностью может также восприниматься несовпадение оценок параметров и дисперсий с априорным мнением. Анализ полученной оценки параметра подразумевает проверку правильности выбора модели (включение или исключение свободного члена, в зависимости от ситуации), проверку сходимости и проверку того, что был достигнут глобальный, а не локальный экстремум (это можно сделать выбрать различные стартовые значения). Стандартные ошибки, рассчитанные в нескольких статистических пакетах могут отличаться, даже если оценки параметров совпадают, поскольку существует отличие в построении вариационной матрицы (см. раздел 5.5.2).

Хорошей методологией расчетов считается начать с выборки небольшого размера и количества регрессоров, например, 1 регрессор и 100 наблюдений. Это упрощает трассировку программы вручную, например программа может печатать ключевые показатели на каждом шаге, или с помощью встроенной функции пошагового исполнения, если статистический пакет это позволяет. Если программа проходит такой тест, то велика вероятность, что вычислительные трудности с полной моделью на полном наборе данных вызваны не ошибками при чтении данных или ошибками в коде, а являются истинными вычислительными трудностями, вызванными мультиколлинеарностью или плохим выбором стартовых значений.

Программу возможно протестировать построив ряд симуляционных данных, где истинные значения параметров известны. Для выборки большого размера, где $N=10,000$, оцененные значения параметров должны быть близки к истинным значениям. 

В заключении следует отметить, что правдоподобная оценка нелинейной модели еще не говорит о правильности расчета. Например, во многих ранее опубликованных работах представлен правдоподобный результат оценки пробит-модели множественного выбора, однако, позже было установлено, что модель не идентифицируема (см. раздел 15.8.1).


\section{Библиографические заметки}

Проблемы с расчетами могут возникнуть даже в линейных моделях, и очень полезно ознакомиться с работами Дэвидсона и МакКиннона (1993, раздел 1.5) и Грина (2003, приложение E). Стандартные ссылки на работы о статистических вычислениях --- работы Кеннеди и Джентла (1980) и особенно Пресса и других (1993), а также другие книги Пресса в соавторстве. У Абрамовица и Стегуна (1971) изложены вопросы вычисления функций. Квандт (1983) описывает много вопросов, связанных с вычислениями, включая оптимизацию.

\begin{enumerate}
\item [$5.3$] Описание итерационных методов приведено у Амэмия (1985, раздел 4.4), Дэвидсона и МакКиннона (1993, раздел 6.7), Маддалы (1977, раздел 9.8) и особенно Грина (2003, приложение E.6). Харви (1990) приводит много применений алгоритма Гаусса-Ньютона, который, благодаря своей простоте, является стандартным итерационным методом при оценивании с помощью нелинейного МНК. Чтобы изучить EM алгоритм, можно обратиться к работе Амэмия (1985, с. 375–378). Алгоритм имитации отжига хорошо описан у Гоффе и других (1994).
\end{enumerate}

\begin{center}
Упражнения
\end{center}

\begin{enumerate}
\item [$10 --- 1$] Найдите оценку коэффициента регрессии в логит-модели по методу максимального правдоподобия, где единственной независимой переменной в модели является константа. Тогда $\E[y]=1/(1+e^{-\beta})$ и градиент масштабированной логарифмической функции правдоподобия будет вычисляться по формуле $g(\beta)=(y-1/(1+e^{-\beta}))$. Предположим, что выборочное среднее значение $\bar{y}$ равно 0.8 и начальное значение $\beta=0.0$.
\begin{enumerate}
\item Расчитайте значение коэффициента $\beta$ для шести первых итераций по алгоритму Ньютона-Рафсона.
\item Рассчитайте шесть первых итераций с использованием метода градиента, предполагая, что $A_s=1$ (10.1) и $\hat{\beta}_{s+1}=\hat{\beta}_s+g_s$.
\item Сравните результаты, полученные в пунктах $(a)$ и $(b)$.
\end{enumerate}


\item [$10 --- 2$] Рассмотрим нелинейную регрессионную модель $y = \alpha{x_1}+\gamma/(x_2-\delta)+u$, где $x_1$ и $x_2$ экзогенно-заданные величины с нормально-распределенными независимыми ошибками $u \sim \mathcal{N}[0,\sigma^2]$.
\begin{enumerate}
\item Выведите уравнение для вычисления параметров $(\alpha, \gamma, \delta)$  алгоритмом Гаусса-Ньютона.
\item Выведите уравнение для вычисления параметров $(\alpha, \gamma, \delta)$ алгоритмом Ньютона-Рафсона.
\item Объясните, почему важно аккуратно выбирать начальные значения для параметров?
\end{enumerate}

\item [$10 --- 3$] Предположим, что функция плотности  $y$ является смесью распределений, $f(y|\pi) = \sum_{j=1}^C \pi_j f_j(y)$, где $\pi = (\pi_1, \dots, \pi_C)$, $\pi_j > 0$, $\sum_{j=1}^C \pi_j = 1$. $\pi_j$ --- это неизвестные веса, в то время как предполагается, что параметры плотностей $f_j(y)$ известны.
\begin{enumerate}
\item Если дана случайная выборка $y_i$, $i = 1, \dots, N$ выпишите общий вид логарифма функции правдоподобия и выпишите условие первого порядка $\hat{\pi}_{ML}$. Проверьте, что для $\hat{\pi}_{ML}$ нет решения в явном виде.
\item Пусть $z_i$ --- это вектор скрытых качественных переменных размера $C \times 1$ $i = 1, \dots, N$ таких, что $z_{ji} = 1$, если $y$ получен из $j$-той компоненты, и $z_{ji} = 0$ в противном случае. Выпишите функцию правдоподобия с помощью наблюдаемых и скрытых переменных, предполагая, будто скрытые переменные являются наблюдаемыми.
\item Предложите EM алгоритм для оценивания $\pi$. [Подсказка: если $z_{ji}$ были бы наблюдаемы, то ММП-оценка $\hat{\pi}_j = N^{-1} \sum_i z_{ji}$. На шаге $E$ необходимо также рассчитать $\E[z_{ji}|y_i]$, а на шаге $M$ надо заменить $z_{ji}$ на $\E[z_{ji}|y_i]$ и потом решить это уравнение относительно $\pi$.]
\end{enumerate}


\item [$10 --- 4$] Предположим, что $(y_{i1}, y_{i2})$, для $i = 1,\dots,N$ имеют двумерное нормальное распределение с заданными средними значениями $(\mu_1, \mu_2)$, параметрами ковариации $(\sigma_{11}, \sigma_{12}, \sigma_{22})$ и коэффициентом корреляции $\rho$. Предположим, что для $y_1$ доступны все $N$ наблюдений, в то время как для $y_2$ доступно только $m$ наблюдений, где $m<N$. При условии, что безусловное частное распределение $y_i$ является нормальным, а условное распределение задано как $y_j \sim \mathcal{N} [\mu_j, \sigma_{jj}]$, $y_2|y_1 \sim \mathcal{N}[\mu_{2.1}, \sigma_{22.1}]$, где $\mu_{2.1}=\mu_2+\sigma_{12}/\sigma_{22}(y_1-\mu_1)$, $\sigma_{22.1}=(1-\rho^2)\sigma_{22}$, разработайте алгоритм нахождения пропущенных значений $y_1$ методом максимизации ожиданий (EM).
\end{enumerate}



\part{Методы симуляционного моделирования}

В первой части говорилось о том, что, как правило, микроэконометрические модели нелинейны и для оценки этих моделей используются объемные, гетерогенные данные, полученные в ходе трудоемких эмпирических исследований, для которых характерно смещение в выборке. При такой постановке проблемы для более реалистичного описания экономической ситуации часто используются модели с трудоемкой процедурой оценивания и следующих из нее статистических выводов. Достижения в области вычислительной техники и программирования позволяют справиться с трудностями вычислений. В третьей части представлены современные, требующие больших затрат вычислительных ресурсов, методы оценки, которые позволяют облегчить процесс расчетов. Для освоения материала этой главы необходимо знание материала предыдущих частей, в особенности методов наименьших квадратов и максимального правдоподобия.

В главе 11 рассмотрены методы бутстрэпа, применяемые для статистических исследований. Преимущество этих методов состоит в упрощении процедуры вычисления стандартных ошибок в то время как формула для расчета следующая из асимптотической теории очень сложная, к примеру, расчет может состоять из двухшаговой процедуры. Вместе с тем, применение бутстрэпа, позволяет уточнить асимптотическую теорию, что может привести к улучшению статистических выводов для выборок малого размера.

В главе 12 рассмотрены методы симуляционного моделирования. С помощью этих методов производится оценка параметров, когда стандартные методы не работают из-за наличия интеграла по вероятностному распределению, наличие которого не позволяет получить решение в аналитическом виде.

В главе 13 рассмотрены байесовские методы. Байесовские подходы к оценке параметров и статистическим выводам значительно отличаются от классических подходов. Несмотря на отличие байесовского подхода, для выборок большого размера результаты аналогичны выводам, полученным в результате использования классических методов. Кроме того, байесовские методы часто оказываются более эффективны с точки зрения вычислений.


\chapter{Методы бутстрэп}
\section{Введение}

Для большинства микроэконометрических методов и соответствующих им тестовых статистик невозможно получить точные результаты для конечной выборки. Статистические выводы, которые рассматривались ранее и основанные на асимптотической теории, как правило, приводили к нормальному или хи-квадрат распределению.

Альтернативным асимптотике способом приближенной оценки является бутстрэп, впервые предложенный Эфроном (Эфрон, 1979, 1982). Бутстрэп позволяет построить приближенное к истинному распределение статистики с помощью метода Монте-Карло. Искусственная случайная выборка строится на основе эмпирического распределения или оцененного распределения для полученных данных. Дополнительные расчеты легко сделать с помощью современной вычислительной техники. Тем не менее, как и в стандартных методах, в основе бутстрэпа лежит асимптотическая теория, за счёт которой получаются точные оценки для бесконечно больших выборок.

Методы бутстрэпа можно разделить на две группы. К первой группе относят наиболее простые подходы, которые позволяют сделать статистические выводы, например, рассчитать стандартные ошибки, когда традиционные методы не справляются. Ко второй группе относят более сложные подходы, дополнительное преимущество которых заключается в возможности асимптотических уточнений, приводящих к более точным выводам для малых выборок. Для прикладных исследований наибольший интерес представляют подходы первой группы. Теоретические исследования отдают предпочтение второй группе, особенно в случаях, когда стандартные асимптотические методы плохо работают на конечной выборке.

В литературе по эконометрике бутстрэп рассматривается применительно к тестированию гипотез, и основан на приближенной оценке вероятности в хвостах распределения статистик. Также бутстрэп можно применять для расчета доверительных интервалов, оценки стандартных ошибок и коррекции смещения оценок. Бутстрэп легко реализовать для гладких  $\sqrt{N}$-состоятельных оценок, полученных на основе независимо одинаково распределенных выборок. Не смотря на это,  методы бутстрэпа с асимптотическими уточнениями используются слишком редко. В других задачах необходима осторожность, в том числе для расчета негладких оценок, таких как медиана, непараметрических оценок и при работе с  данными, которые не являются независимыми и одинаково распределенными.

Вполне полное описание бутстрэпа сделано в разделе 11.2, пример использования бутстрэпа рассмотрен в разделе 11.3, и некоторые теоретические вопросы представлены в разделе 11.4. Прочие варианты бутстрэпа рассмотрены в разделе 11.5. В разделе 11.6 рассмотрено применение бутстрэпа для специфического типа данных и специфических методов, используемых в микроэконометрике.
 
\section{Бутстрэп. Краткий обзор}

В этом разделе представлены основные методы бутстрэпа для оценки параметра $\hat{\theta}$ на независимо и одинаково распределенных данных $\{w_1, \ldots, w_N \}$, где, как правило, $w_i=(y_i,x_i)$ и $\hat{\theta}$ --- гладкая асимптотически нормальная и $\sqrt{N}$-состоятельная оценка. Для простоты обозначений рассмотрим скалярную $\theta$. В большинстве случаев для вектора $\theta$, $\theta$ достаточно просто заменить на $j$-ый элемент вектора $\theta$, $\theta_j$.

Интересующие нас статистики включают стандартные регрессионные показатели: оценку $\hat{\theta}$; стандартные ошибки $s_{\hat{\theta}}$; $t$-статистику $t=(\hat{\theta}-\theta_0)/s_{\hat{\theta}}$, где $\theta_0$ предполагаемое в нулевой гипотезе значение; соответствующее критическое значение $t$-статистики, точное P-значение и доверительный интервал. Далее рассмотрено применение бутстрэпа для каждой из вышеперечисленных статистик, теоретические аспекты см. в разделе 11.4.

\subsection{Бутстрэп без уточнения}

Рассмотрим оценку дисперсии выборочного среднего $\hat{\mu}=\overline{y}=N^{-1}\sum^{N}_{i=1}y_i$, где скалярные случайные величины $y_i$ независимы и одинаково распределены с параметрами $[\mu, \sigma^2]$, и известно, что $\V[\hat{\mu}]=\sigma^2/N$.

Оценка дисперсии $\hat{\mu}$ может быть получена путем создания $S$ выборок из генеральной совокупности размерностью $N$, и, следовательно, будут получены $S$ средних значений и $S$ оценок $\hat{\mu}_s=\overline{y}_s$, $s = 1,\ldots, S$. Таким образом, возможно оценить $\V[\hat{\mu}]$ через $(S - 1)^{-1}\sum^{S}_{s=1}(\hat{\mu}_s - \overline{\hat{\mu}})^2$, где $\overline{\hat{\mu}} = S^{-1}\sum^{S}_{s=1}\hat{\mu}_s$.

Естественно, что данный подход невозможно применить, поскольку у нас есть только одна выборка. Для использования бутстрэпа представим, что данная выборка есть генеральная совокупность. В таком случае, конечная выборка может быть представлена реальными данными $y_1,\ldots, y_N$. Распределение $\hat{\mu}$ можно получить построив $B$ бутстрэповских выборок из генеральной совокупности размера $N$, где каждая выборка размера $N$ получена путем случайной выборки из $y_1,\ldots , y_N$ с повторениями. В результате получим $B$ выборочных средних и $B$ оценок $\hat{\mu}_b = \overline{y}_b$, $b = 1,\ldots, B$. Тогда оценка $\V[\hat{\mu}]$ может быть получена из выражения $(B-1)^{-1}\sum^B_{b=1}(\hat{\mu}_b-\overline{\hat{\mu}})^2$, где $\overline{\hat{\mu}}=B^{-1}\sum^B_{b=1}\hat{\mu}_b$. Выборка с повторениями может казаться нарушением стандартных правил формирования выборки, тем не менее в действительности  стандартная теория формирования выборки подразумевает возможность повторений (см. раздел 24.2.2).

При наличии дополнительной информации возможно  создать бутстрэповскую выборку другими способами. Например, если известно, что $y_i$ распределено нормально $y_i \sim \cN [\mu,\sigma^2]$, то возможно сформировать $B$ выборок размера $N$ из распределения $\cN [\hat{\mu},s^2]$. Такой бутстрэп является примером параметрического бутстрэпа, в то время как предыдущий вид бутстрэпа создается на основе эмпирического распределения.

Для оценки $\hat{\theta}$ могут быть использованы аналогичные виды бутстрэпа, например, для оценки $\V[\hat{\theta}]$ и, соответствующих им стандартных ошибок, в случаях, когда аналитические формулы расчета $\V[\hat{\theta}]$ сложны. Как правило, такие методы бутстрэпа применимы для наблюдений $w_i$, независимо и одинаково распределенных по $i$, и порождают оценки со свойствами, похожими на свойства оценок, получаемых обычными асимптотическими методами.


\subsection{Асимптотические уточнения}

При решении некоторого класса задач методом бутстрэпа возможно достичь более точного результата, и, соответственно, оценки будут эквивалентны полученным на основе уточненной асимптотической теории, и будут лучше описывать распределение  $\hat{\theta}$ в конечной выборке. Основу данной главы составляет рассмотрение асимптотических уточнений.
 
Как правило, асимптотическая теория использует результат: $\sqrt{N}(\hat{\theta}-\theta_0) \stackrel{d}{\rightarrow} \cN[0,\sigma^2]$. Таким образом, 

\begin{equation}
\Pr[\sqrt{N}(\hat{\theta}-\theta_0)/ \sigma \leq z] = \Phi(z)+R_1,
\end{equation}
где $\Phi(\cdot)$ функция стандартного нормального распределения и $R_1$ остаточный член, значение которого стремится к нулю при $N \rightarrow \infty$. 

Этот результат получен с использованием асимптотической теории, подробно рассмотренной в разделе 5.3, который также включает центральную предельную теорему (ЦПТ). В основе ЦПТ лежит урезанное разложение в степенной ряд. Разложение Эджворта детально рассмотрено в разделе 11.4.3. При добавлении еще одного члена получаем:

\begin{equation}
\Pr[\sqrt{N}(\hat{\theta}-\theta_0)/ \sigma \leq z] = \Phi(z)+\dfrac{g_1(z)\phi(z)}{\sqrt{N}}+R_2,
\end{equation}
где $\phi(\cdot)$ плотность стандартного нормального распределения, $g_1(\cdot)$ ограниченная функция согласно (11.13) в разделе 11.4.3 и $R_2$ остаточный член, который исчезает при $N \rightarrow \infty$.

Разложение Эджворта сложно применять с теоретической точки зрения, поскольку функция $g_1(\cdot)$ зависит от данных сложным образом. Бутстрэп с асимптотическим уточнением предлагает простой вычислительный метод для применения разложения Эджворта. Теоретическое рассмотрение вопроса дано в разделе 11.4.4.

Поскольку $R_1=O(N^{-1/2})$ и $R_2=O(N^{-1})$, асимптотически $R_2<R_1$, и мы получаем более удачное приближение при $N \rightarrow \infty$. Однако в конечных выборках возможно что $R_2>R_1$. Бутстрэп с асимптотическим уточнением дает улучшенное асимптотическое приближение, что должно способствовать получению более точного приближения в конечных выборках типичного размера. Тем не менее, нет гарантии что такое улучшение произойдет и симуляционное моделирование часто используется, чтобы проверить действительно ли выгоды имеют место.

\subsection{Статистика, асимптотически не зависящая от неизвестных параметров}

Чтобы получить асимптотическое улучшение, предельное распределение бутстрэпируемой статистики не должно зависеть от неизвестных параметров. Данный результат объясняется в разделе 11.4.4.

В качестве примера рассмотрим процесс формирования выборки из $y_i \sim [\mu, \sigma^2]$. Тогда оценка $\hat{\mu}=\overline{y} \stackrel{a}{\sim} \cN[\mu, \sigma^2/N]$ не является статистикой в узком смысле слова, даже при заданном значении нулевой гипотезы $\mu = \mu_0$, поскольку данное распределение зависит от неизвестного параметра $\sigma^2$. Статистика Стьюдента $t=(\hat{\mu} - \mu_0)/s_{\hat{\mu}} \stackrel{a}{\sim} \cN[0,1]$ асимптотически не зависит от неизвестных параметров. 

Оценки неизвестных параметров, как правило асимптотически не являются статистиками в узком смысле слова. Традиционные тестовые статистики, имеющие стандартное нормальное или хи-квадрат распределение, в том числе статистика Вальда, статистика множителей Лагранжа, а также тест отношения правдоподобия и соответствующие им доверительные интервалы асимптотически не зависят от неизвестных параметров.  

\subsection{Бутстрэп}

В этом разделе мы приводим общее описание бутстрэпа, а детали изложены в последующих разделах.


\subsubsection*{Алгоритм бутстрэпа}


Общий алгоритм бутстрэпа состоит из следующих шагов:

\begin{enumerate}
\item  На основе данных $w_1,\ldots, w_N$ и выбранного метода строится бутстрэповская выборка размером $N$, $w_1^*,\ldots ,w_N^*$.


\item На основе бутстрэповской выборки рассчитывается статистика, а именно: (a) оценки параметра $\theta$ --- это $\hat{\theta}^*$, (b) стандартные ошибки $s_{\hat{\theta}^*}$ оценки $\hat{\theta}^*$,(c) $t$-статистика, $t^*=(\hat{\theta}^*-\hat{\theta})/s_{\hat{\theta}^*}$, центрированная на исходную оценку $\hat{\theta}$. Таким образом, $\hat{\theta}^*$ и $s_{\hat{\theta}^*}$ рассчитываются обычным методом, но с использованием новой бутстрэповской выборки.


\item Шаги 1 и 2  повторяются $B$ раз, при этом $B$ достаточно велико, в результате мы получаем $B$ значений интересующей нас статистики, скажем, $\hat{\theta}_1^*,\ldots ,\hat{\theta}_B^*$ или  $t_1^*,\ldots ,t_B^*$.

 
\item Используя полученные $B$ значений статистики, мы получаем  бутстрэповский вариант статистики, согласно описанию данному в последующих подразделах.
\end{enumerate}

Данный алгоритм допускает много вариаций, которые отличаются способом получения бутстрэповской выборки, количеством бутстрэповских выборок, бутстрэпировуемой статистикой и тем, зависит ли асимптотически данная статистика  от неизвестных параметров.


\subsubsection*{Процесс формирования выборки бутстрэпом}


Бутстраповский процесс порождающий данные на шаге 1 используется в качестве аппроксимации истинного процесса порождающего данные.

Самый простой бутстрэп предполагает использование эмпирического распределения данных, т.е. исходная выборка выступает в роли генеральной совокупности. Таки образом $w_1^*,\ldots ,w_N^*$ формируются путем случайной выборки из  $w_1,\ldots ,w_N$ с повторениями. В каждой бутстрэповской выборке, полученной таким образом, некоторые элементы исходной выборки могут встречаться несколько раз, а другие могут не встречаться вообще. Такой бутстрэп называют бустрэпом эмпирического распределения (empirical distribution function bootstrap) или непараметрическим бутстрэпом. Вместе с тем, этот подход получил название парного бутстрэпа, поскольку в регрессионных моделях с одним уравнением $w_i=(y_i,x_i)$, и новые выборки создаются для обеих переменных $y_i$ и $x_i$. 

Предположим, что задано условное распределение данных, $y|x\sim F(x,\theta_0)$ и имеется оценка $\hat{\theta} \stackrel{p}{\rightarrow} \theta_0$. Тогда, на первом шаге можно сформировать бутстрэповскую выборку, используя исходные $x_i$ и формируя новые значения $y_i$ случайным образом, учитывая заданное распределение $F(x_i,\hat{\theta})$. Такая процедура соответствует регрессорам с фиксированными значениями в повторных выборках (см. раздел 4.4.5). В качестве варианта, возможно сначала сформировать $x_i^*$ случайной выборкой из исходных значений $x_1,\ldots, x_N$ и, затем, сгенерировать $y_i$ случайно, используя распределение $F(x_i^*,\hat{\theta})$, $i=1,\ldots, N$. Оба вышеприведенных способа относятся к параметрическому бутстрэпу, который может быть применен в полностью  параметрических моделях.


Для регрессионной модели с независимо и одинаково распределенной аддитивной ошибкой, например, $y_i=g(x_i,\beta)+u_i$ можно получить предсказанные значения остатков $\hat{u}_1,\ldots ,\hat{u}_N$, где $\hat{u}_i=y_i-g(x_i,\hat{\beta})$. Тогда, на первом шаге бутстрэпа сделаем из этих остатков новую случайную выборку $(\hat{u}_1^*,\ldots ,\hat{u}_N^*)$, итогом будет формирование бутстрэповской выборки $(y_1^*,x_1),\ldots,(y_N^*,x_n)$, где $y_i^*=g(x_i,\hat{\beta})+u_i^*$. Такой бутстрэп получил название бутстрэпа остатков. Этот подход занимает промежуточное положение между непарметрическим и параметрическим бутсрэпом. Бутстрэп остатков может применяться, если распределение остатков не зависит от неизвестных параметров.

Мы обращаем особое внимание на парный бутстрэп. Он прост, применим ко многим нелинейным моделям, опирается на довольно слабые предпосылки о распределении.  В то же время, другие виды бутстрэпа обычно дают лучшую аппроксимацию, чем парный бутстрэп (см. Хоровиц, 2001, стр.3185) и должны применяться при выполнении необходимых предпосылок.

\subsubsection*{Количество выборок бутстрэпа}

Асимптотические свойства бутстрэпа основаны на предпосылке $N \rightarrow \infty$, поэтому бутстрэповские статистики могут быть асимптотически корректными и для малых $B$. Тем не менее, что бутстрэп более точен при $B \rightarrow \infty$. Какое значение $B$ считать достаточно большим? Ответ на этот вопрос зависит от приемлемого уровня бустраповской ошибки, вызванной симуляциями, и от цели использования бутстрэпа.

Эндрюс и Бучински (2000) предложили практический численный метод для определения количества репликаций $B$ необходимых для достижения требуемой точности, или метод для определения степени точности, получаемой при заданном значении $B$. Допустим, что $\lambda$ это искомое значение, а именно, стандартная ошибка или критическое значение статистики, тогда $\hat{\lambda}_\infty$ обозначает идеальную бутстрэповскую оценку при $B=\infty$, а $\hat{\lambda}_B$ --- оценку с количеством бутстрэповских выборок равным $B$. Эндрюс и Бучински (2000) показали, что

\[
\sqrt{B}(\hat{\lambda}_B-\hat{\lambda}_{\infty})/\hat{\lambda}_{\infty} \stackrel{d}{\rightarrow} \cN[0,w],
\]
где $w$ меняется в зависимости от приложения и задано в таблице III Эндрюс и Бучински (2000). Следовательно, $\Pr[\delta \leq z_{\tau/2} \sqrt{w/B}]=1-\tau$, где $\delta=|\hat{\lambda}_B-\hat{\lambda}_{\infty}|/\hat{\lambda}_{\infty}$ обозначает относительное отклонение зависящее от количества репликаций, $B$. Таким образом, выполнение неравенства $B \geq wz_{\tau/2}^2 / \delta^2$ означает, что относительные отклонения меньше $\delta$ с вероятностью не ниже $1-\tau$. Иными словами, при заданном количестве репликаций $B$ относительные отклонения меньше $\delta=z_{\tau/2}\sqrt{w/B}$.

Для практического руководства мы предлагаем правило <<большого пальца>>
\[
B=384w.
\]

Данное правило обеспечивает относительное отклонение меньше $10\%$ с вероятностью не менее $95\%$, поскольку $z_{.025}^2/0.1^2=384$. Единственная трудность в применении такого подхода заключается в оценке $\omega$, значение которого может меняться в зависимости от приложения.

Для оценивания стандартной ошибки $\omega=(2+\gamma_4)/4$, где $\gamma_4$ коэффициент эксцесса бутстрэповской оценки $\hat{\theta}^*$. Интуитивно,  тяжелые хвосты распределения приводят к более вероятным выбросам, и затрудняют оценивание  стандартной ошибки. Следовательно, если $\gamma_4=0$, то достаточно, чтобы $B=384{\times}(1/2)=192$, в то время как при $\gamma_4=8$ количество $B$ должно быть равно 960. Таким образом, это значение превышают $B=200$, ранее предложенное Эфроном и Тибшарани (1993, стр.52) как вполне достаточное.

Для проведения симметричного двустороннего теста или построения двустороннего доверительного интервала на уровне значимости $\alpha$, $w$ должна быть равна $\alpha(1-\alpha)/[2z_{\alpha/2}\phi(z_{\alpha/2})]^2$. Следовательно, $B=348$ при $\alpha=0.05$ и $B=685$ при $\alpha=0.01$. Как и ожидалось, чем дальше мы залезаем в хвосты распределения, тем больше должно быть количество бутстрэповских выборок.

Для проведения одностороннего или несимметричного двустороннего теста или построения доверительного интервала на уровне значимости $\alpha$ используется $w=\alpha(1-\alpha)/[z_{\alpha}\phi(z_{\alpha})]^2$. Следовательно, $B=634$ при $\alpha=0.05$ и $B=989$ при $\alpha=0.01$. Для одностороннего теста $B$ должно быть больше. Для хи-квадрат тестов с $h$ степенями свободы $w=\alpha(1-\alpha)/[\chi_{\alpha}^2(h)f(\chi_{\alpha}^2(h))]^2$, где $f(\cdot)$ плотность распределения $\chi^2(h)$.

Для точных $p$-значение используется $w$, равное $(1-p)/p$. Например, если $p=0.05$ тогда $w=19$ и $B=7,296$. Для расчета $p$-значения необходимо гораздо большее количество $B$ по сравнению с отвержением гипотезы при превышении критического значения.

Для оценки параметра $\theta$, скорректированной на смещение, есть простое правило, при котором $\hat{w}=\hat{\sigma}^2/\hat{\theta}^2$, где стандартная ошибка $\hat{\theta}$ равна $\hat{\sigma}$. Например, если обычная $t$-статистика равна $t=\hat{\theta}/\hat{\sigma}=2$, то $\hat{w}=1/4$ и $B=96$. В работе Эндрюс и Бучински (2000) рассмотрено намного больше деталей и уточнений этих результатов.

Для тестирования гипотез, Дэвидсон и МакКиннон (2000) предложили альтернативный подход. Они исследовали понижение мощности теста, которое происходит из-за  конечного количества $B$ бутстрэповских выборок. Отметим, что мощность теста не снижается, если $B=\infty$. На основе проведенных симуляций Дэвидсон и МакКиннон рекомендуют как минимум $B=399$ для тестов с $\alpha=0.05$ и как минимум $B=1,499$ для $\alpha=0.01$. Они приводят аргументы, что для тестирования гипотез их поход лучше, чем подход Эндрюса и Бучински.

Ряд работ Дэвидсона и МакКиннона, результаты которых обобщил МакКиннон (2002), сосредоточены на практических аспектах применения бутстрэпа. Для тестирования гипотез на уровне значимости $\alpha$ следует выбирать $B$ таким образом, чтобы значение выражения $\alpha(B+1)$ было целым числом. Например, при $\alpha=0.05$ лучше взять $B=399$, а не 400. Если все же взять $B=400$, тогда будет непонятно какая по счёту $t$-статистика, 20-ая или 21-ая, является критическим значением. Для нелинейных моделей количество расчетов можно сократить проведя только несколько итераций Ньютона-Рафсона в каждой бутстрэповской выборке, где начальные значения равны исходным оценкам параметров.

\subsection{Оценка стандартных ошибок}

Для оценивания дисперсии оценки параметра в бутстрэпе используется обычная формула оценки дисперсии, применяемая для $B$ репликаций $\hat{\theta}_1^*,\ldots ,\hat{\theta}_B^*$.

\begin{equation}
s_{\hat{\theta},Boot}^2=\dfrac{1}{B-1}\sum_{b=1}^B(\hat{\theta}_b^*-\overline{\hat{\theta}}^*)^2,
\end{equation} 
где

\begin{equation}
\overline{\hat{\theta}^*}=B^{-1}\sum_{b=1}^B\hat{\theta}_b^*.
\end{equation}
Извлекая квадратный корень, получим бутстрэповскую оценку стандартной ошибки, $s_{\hat{\theta},Boot}$.

Этот бутстрэп не предполагает асимптотических уточнений. Тем не менее, этот метод может быть полезен, когда трудно получить оценку стандартной ошибки применяя стандартные методы. Существует большое количество примеров. Оценка параметра $\hat{\theta}$ может быть последовательной двухшаговой М-оценкой, стандартную ошибку которой трудно рассчитать используя результаты раздела 6.8. Также $\hat{\theta}$ может быть рассчитана методом двухшагового МНК, при использовании статистического пакета, предполагающего гомоскедастичность стандартных ошибок, в то время как они гетероскедастичны. Оценка $\hat{\theta}$ может быть функцией других оценок, например, $\hat{\theta}=\hat{\alpha}/\hat{\beta}$, в таком случае может быть использован бутстрэп вместо дельта метода. Для кластеризованных данных со  множеством маленьких кластеров, к примеру, для разбиения на короткие панели, могут быть получены робастные кластерные стандартные ошибки через ресэмплинг кластеров.

Поскольку оценка бутстрэпа $s_{\hat{\theta},Boot}$ состоятельна, она может быть использована вместо $s_{\hat{\theta}}$ в обычной асимптотической формуле для построения доверительных интервалов и при проведении различных асимптотических тестов. Таким образом, использование асимптотических статистических выводов возможно в случае, когда трудно получить стандартные ошибки другими методами. Тем не менее, улучшения оценки для конечных выборок не произойдет. Для получения асимптотического уточнения необходимо использовать методы из следующего раздела.

\subsection{Тестирование гипотез}

В данном разделе рассматриваются тесты для отдельного коэффициента $\theta$. Тест может быть как с правой односторонней альтернативой $H_0:\theta \leq \theta_0$, $H_{\alpha}:\theta > \theta_0$, так и с двусторонней альтернативой $H_0: \theta = \theta_0$, $H_a: \theta \not= \theta_0$. Другие тесты будут рассмотрены в разделе 11.6.3.

\subsubsection*{Тесты с асимптотическими уточнениями}

Как правило базовой для асимптотических уточнений является обычная тестовая статистика $T_N=(\hat{\theta}-\theta_0)/s_{\hat{\theta}}$, поскольку асимптотическое распределение $T_N$ не зависит от неизвестных параметров. Создадим $B$ бутстрэповских репликаций и получим $B$ тестовых статистик $t_1^*,\ldots ,t_B^*$, где 

\begin{equation}
t_b^*=({\hat{\theta}_b}^*-\hat{\theta})/s_{{\hat{\theta}}_b^*}.
\end{equation}
Оценки $t_b^*$ сосредоточены около первоначальной оценки $\hat{\theta}$ поскольку ресэмплинг делается на основе значений, сосредоточенных около $\hat{\theta}$. Эмпирическое распределение $t_1^*, \ldots, t_B^*$, упорядоченных от меньшего к большему, используется для определения распределение $T_N$.

Для правосторонней альтернативной гипотезы бутстраповское критическое значение (при уровне значимости $\alpha$) равно выборочному верхнему квантилю уровня $\alpha$ среди $B$ упорядоченных статистик. Например, если $B=999$ и $\alpha=0.05$ тогда критическим значением будет 950-ое наибольшее значение статистики $t^*$, поскольку тогда $(B+1)(1-\alpha)=950$. Аналогично, для левосторонней альтернативной гипотезы  бутстраповское критическое значение  будет равно 50-му наименьшему значению $t^*$.

Также, используя бутстрэп можно рассчитать точное $p$-значения. Например, если первоначальное исходная $t$-статистика лежит между 914-м и 915-м наибольшим значением среди 999 бутстрэповских репликаций, тогда P-значение для правосторонней альтернативной гипотезы будет равно $1-914/(B+1)=0.086$.

Для двустороннего теста необходимо учитывать различие между симметричным и несимметричным тестами. Для несимметричного теста или теста с равными хвостами бутстрэповские критические значения на уровне значимости $\alpha$ составляют нижний $\alpha/2$ и верхний $\alpha/2$ квантили упорядоченных тестовых статистик $t^*$. Нулевая гипотеза отвергается на уровне значимости $\alpha$, если первоначальная $t$-статистика лежит за пределами обозначенного интервала. Для симметричного теста упорядочиваются абсолютные значения статистик, $|t^*|$ и бутстрэповское критическое значение (на уровне значимости $\alpha$) будет равно верхнему $\alpha$ квантилю упорядоченных $|t^*|$. Нулевая гипотеза отвергается на уровне значимости $\alpha$, если $|t|$ превышает критическое значение.

Вышеприведенные тесты при использовании метода $t$-перцентилей позволяют использовать асимптотические уточнения. Для одностороннего $t$-теста и несимметричного двустороннего $t$-теста истинная вероятность ошибки первого рода равна $\alpha+O(N^{-1/2})$ при использовании стандартных асимптотических критических значений, а при использовании бутстрэповских критических значений  истинная вероятность ошибки первого рода равна  $\alpha+O(N^{-1})$. Для двустороннего симметричного $t$-теста и асимптотического хи-квадрат теста аппроксимация работает лучше, здесь истинная вероятность ошибки первого рода равна $\alpha+O(N^{-1})$, если используются стандартные асимптотические критические значения, и равна  $\alpha+O(N^{-2})$ если используются бутстрэповские критические значения. 


\subsubsection*{Тесты без асимптотических уточнений}

Существуют альтернативные методы бутстрапирования, которые являются асимптотически корректными, но не приводят к асимптотическим уточнениям.

Один из подходов, мы упомянули его в конце раздела 11.2.5 заключается в том, чтобы рассчитать $t=(\hat{\theta}-\theta_0)/s_{\hat{\theta},boot}$, где бутстрэп оценка $s_{\hat{\theta},boot}$, вычисленная по формуле (11.3), заменяет обычную оценку $s_{\hat{\theta}}$ и затем данная статистика  сравнивается с критическими значениями стандартного нормального распределения.

Второй подход, излагаемый здесь применительно к  двустороннему тесту гипотезы $H_0:\theta=\theta_0$ c альтернативной $H_a:\theta \neq \theta_0$, сначала рассчитывает нижний $\alpha/2$ и верхний $\alpha/2$ квантили оценок бутстрэпа $\hat{\theta}_1^*,\ldots ,\hat{\theta}_B^*$. Гипотеза $H_0$ отвергается, если значение $\theta_0$ лежит за пределами границ этого интервала. Этот подход получил название процентильного метод. Асимптотическое уточнение  возникает из-за использования  статистики $t_b^*$ в формуле (11.5), при этом центрирование происходит вокруг  $\hat{\theta}$, а не вокруг $\theta_0$, и своей стандартной ошибки $s_{\hat{\theta}}^*$ в каждом бутстрэпе.

Привлекательность двух вышеприведённых методов состоит в отсутствии необходимости рассчитывать $s_{\hat{\theta}}$, т.е. оценки стандартной ошибки, основанной на асимптотической теории.

\subsection{Доверительные интервалы}

Как правило, в литературе по статистике рассматривается оценка доверительных интервалов, а не тестирование гипотез. В этом разделе основное внимание сосредоточено на тестировании гипотез, поэтому вопрос построения доверительных интервалов будет изложен в краткой форме. 

Уточнения асимптотики основаны на $t$-статистике, которая асимптотически не зависит от неизвестных параметров. Таким образом, следуя по шагам 1-3 из раздела 11.2.4, получим реплицированные бутстрэпом $t$-статистики $t_1^*,\ldots ,t_B^*$. Обозначим через $t_{[1-\alpha/2]}^*$ и $t_{[\alpha/2]}^*$ нижний и верхний квантиль степени свободы $\alpha/2$ этих $t$-статистик. Тогда, согласно $t$-перцентильному методу $100(1-\alpha)$ процентный доверительный интервал равен

\begin{equation}
(\hat{\theta}-t_{[1-\alpha/2]}^*{\times}s_{\hat{\theta}},\hat{\theta}+t_{[\alpha/2]}^*{\times}s_{\hat{\theta}}),
\end{equation}
где $\hat{\theta}$ и $s_{\hat{\theta}}$ оценка и стандартная ошибка, соответственно, из первоначальной выборки.

Альтернативным $t$-перцентильному методу  является ускоренный метод коррекции смещения (bias corrected and accelerated method, $BC_a$), он подробно рассмотрен в работе Эфрона (1987). Данный метод позволяет получить асимптотические уточнения в более широком классе задача, чем $t$-перцентильный метод.

Другие методы позволяют построить асимптотически верный доверительный интервал, но без асимптотических уточнений. Во-первых, для построения доверительного интервала возможно использовать оценку бутстрэпа стандартной ошибки в обычной формуле, так мы получим интервал вида $(\hat{\theta}-z_{[1-\alpha/2]} \times s_{\hat{\theta},boot},\hat{\theta}+z_{[\alpha/2]}{\times}s_{\hat{\theta},boot})$. Во-вторых, согласно перцентильному методу доверительный интервал определяется как промежуток между нижним $\alpha/2$ и верхним $\alpha/2$ квантилями $B$ оценок бутстрэпа $\hat{\theta}_1^*,\ldots, \hat{\theta}_B^*$ параметра $\theta$.

\subsection{Корректировка смещения}

Как правило, в конечных выборках нелинейные оценки смещены, однако данное смещение  асимптотически стремится к нулю, если оценка состоятельна. Например, если оценка $\mu^3$ получена как $\hat{\theta}=\overline{y}^3$, где $y_i$ независимо и одинаково распределены с параметрами $[\mu,\sigma^2]$, тогда $\E[\hat{\theta}-\mu^3]=3\mu\sigma^2/N+\E[(y-\mu)^3]/N^2$.

В более общем виде, для $\sqrt{N}$-состоятельных оценок

\begin{equation}
\E[\hat{\theta}-\theta_0]=\dfrac{a_N}{N}+\dfrac{b_N}{N^2}+\dfrac{c_N}{N^3}+\ldots ,
\end{equation}
где $a_N,b_N$ и $c_N$ --- это ограниченные константы, значения которых зависят от данных и метода оценивания (см. Холл, 1992, стр. 53). Альтернативная оценка $\tilde{\theta}$ является асимптотическим уточнением, если 

\begin{equation}
\E[\tilde{\theta}-\theta_0]=\dfrac{B_N}{N^2}+\dfrac{C_N}{N^3}+\ldots,
\end{equation}
где $B_N$ и $C_N$ ограниченные константы. Для обеих оценок смещение исчезает при $N \rightarrow \infty$. Оценка $\tilde{\theta}$ более привлекательна, поскольку смещение быстрее устремляется в ноль и, таким образом, получаем  асимптотическое уточнение, однако в конечных выборках возможна обратная ситуация, т.е. $(B_N/N^2)>(a_N/N+b_N/N^2)$.

Мы хотим оценить смещение $\E[\hat{\theta}]-\theta$. Значение данного выражения равно расстоянию между ожидаемым средним значением оценки параметра и значением параметра, в процессе, порождающем данные. Бутстрэп заменяет генеральную совокупность на выборку, поэтому бутстрэповские выборки создаются с использованием параметра $\hat{\theta}$, среднее значение которого по выборкам равно $\overline{\hat{\theta}}^*$ . 

Тогда оценка смещения бутстрэпом равна
\begin{equation}
Bias_{\hat{\theta}}=(\overline{\hat{\theta}}^*-\hat{\theta}),
\end{equation}
где $\overline{\hat{\theta}}$ определена в (11.4).

Например, предположим, что $\hat{\theta}=4$ и $\overline{\hat{\theta}}^*=5$, тогда оценка смещения равна $(5-4)=1$. Следовательно имеем смещение в сторону увеличения на 1, это в свою очередь означает, что $\hat{\theta}$ переоценена на 1. Для корректировки смещения необходимо вычесть единицу из $\hat{\theta}$, что даст скорректированную на смещение оценку равную 3. В более общем виде, скорректированная на смещение оценка параметра $\theta$ может быть записана как

\begin{equation}
\hat{\theta}_{Boot}=\hat{\theta}-(\overline{\hat{\theta}}^*-\hat{\theta})
\end{equation}
\[
=2\hat{\theta}-\overline{\hat{\theta}}^*
\]

Следует обратить внимание, что оценка $\overline{\hat{\theta}}^*$ не скорректирована на смещение сама по себе. Для более подробного изучения вопроса корректировки, см. Эфрон и Тибшарани (1993, стр. 138). Для типовых $\sqrt{N}$-состоятельных оценок асимптотическое смещение параметра $\hat{\theta}$ равно $O(N^{-1})$ в то время, как асимптотическое смещение бутстрэповской оценки $\hat{\theta}_{Boot}$ равно $O(N^{-2})$.

На практике корректировка смещения редко применяется для $\sqrt{N}$-состоятельных оценок, поскольку бутстрэпэвская оценка может быть более изменчивой, чем первоначальная оценка $\hat{\theta}$ и, как правило, смещение меньше чем стандартная ошибка оценки. Бустраповская корректировка смещения  применяется, если скорость сходимости оценки ниже, чем $\sqrt{N}$, в особенности для непараметрических оценок регрессии и  плотности.

\section{Пример бутстрэпа}

В качестве примера бутстрэпа рассмотрим модель экспоненциальной регрессии, введенной в разделе 5.9. Для  примера,  сгенерируем экспоненциальное распределение с экспоненциальным средним и двумя регрессорами:

\[
y_i|x_i\sim exponential(\lambda_i), i=1,\ldots ,50,
\]
\[
\lambda_i= \exp(\beta_1+\beta_{2}x_{2i}+\beta_{3}x_{3i}),
\]
\[
(x_{2i},x_{3i})\sim \cN[0.1,0.1;0.1^2,0.1^2,0.005],
\]
\[
(\beta_1,\beta_2,\beta_3)=(-2,2,2).
\]

По выборке, состоящей из 50-ти наблюдений, методом максимального правдоподобия получаем следующие результаты: $\hat{\beta}_1=-2.192$; $\hat{\beta}_2=0.267,s_2=1.417$, и $t_2=0.188$; и $\hat{\beta}_3=4.664, s_3=1.741$, и $t_3=2.679$. Для данного примера ММП в основе расчетов стандартных ошибок лежит минус обратная оценённая матрица Гессе, $-A^{-1}$.

Основное внимание обратим на статистические выводы для $\beta_3$ и покажем применение бутстрэпа для расчета стандартных ошибок, проведения теста статистической значимости, формирования доверительных интервалов и корректировки смещения. В данном примере разница между бутстрэпом и обычными асимптотическими оценками мала, тем не менее эта разница может быть больше в других примерах.

Результаты получены методом парного бутстрэпа (см. раздел 11.2.4) с совместным ресэмплингом $(y_i,x_{2i},x_{3i})$, проведенным $B=999$ раз. Как следует из Таблицы 11.1, среднее значение и стандартное отклонение 999 реплицированных оценок бутстрэпа $\hat{\beta}_{3,b}^*$, $b=1,\ldots ,999$, равны 4.716 и 1.939, соответственно. Также в Таблице 11.1 приведены значения основных перцентилей для $\hat{\beta}_3^*$ и $t_3^*$.

Парный бутстрэп может быть заменен на параметрический бутстрэп. При параметрическом бутстрэпе выборка для $y_i$ будет формироваться из экспоненциального распределения с параметром $\exp(\hat{\beta}_1+\hat{\beta}_{2}x_{2i}+\hat{\beta}_{3}x_{3i})$. При тестировании гипотезы $H_0:\beta_3=0$ параметр экспоненциального распределения может быть равен $\exp(\tilde{\beta}_1+\tilde{\beta}_2x_{2i})$, где $\tilde{\beta}_1$ и $\tilde{\beta}_2$ ограниченные оценки ММП по первоначальной выборки.

Стандартные ошибки: по формуле (11.3) бутстрэповская оценка стандартной ошибки  рассчитывается по обычной формуле для  стандартного отклонения выборки, получаемой в результате 999-ти репликаций $\beta_3$. Оценка стандартной ошибки равна 1.939, что превышает обычную асимптотическую оценку стандартной ошибки --- 1.741. Заметим, что при параметрическом бутстрэпе отсутствует возможность асимптотических уточнений. Поэтому он используется в качестве способа проверки или в случае, если нахождение стандартных ошибок другими способами  затруднено.

\begin{table}[h]
\begin{center}
\caption{\label{tab:pred} Вывод коэффициента наклона с помощью бутсрэпа: пример}
\begin{minipage}{9cm}
\begin{tabular}{ccccc}
\hline
\hline
& $\hat{\beta}^*_3$ & $t^*_3$ & $z = t(\infty)$ & $t(47)$ \\ 
\hline 
Среднее \footnote{Описательные статистики и перцентили, основанные на 999 парных бутстрэпированных выборках для $(1)$ оценки $\hat{\beta}^*_3$; $(2)$ статистики $t^*_3 = (\hat{\beta}^*_3 - \hat{\beta})/s_{\hat{\beta}^*_3}$;
 $(3)$ $t$-распределения Стьюдента с 47-ю степенями свободы; $(4)$ стандартного нормального распределения. Процесс, порождающий данные, --- выборка из экспоненциального распределения, которое приведено в тексте. Размер выборки равен 50.} & 4.716 & 0.026 & 1.021 & 1.000 \\ 
Ст.от.\footnote{Стандартное отклонение} & 1.939 & 1.047 & 1.000 & 1.021 \\
1\% & -0.336 & -2.664 & -2.326 & -2.408 \\
2.5\% & 0.501 & -2.183 & -1.960 & -2.012 \\
5\% & 1.545 & -1.728 & -1.645 & -1.678 \\
25\% & 3.570 & -0.621 & -0.675 & -0.680 \\
50\% & 4.772 & 0.062 & 0.000 & 0.000 \\
75\% & 5.971 & 0.703 & 0.675 & 0.680 \\
95\% & 7.811 & 1.706 & 1.645 & 1.678 \\
97.5\% & 8.484 & 2.066 & 1.960 & 2.012 \\
99.0\% & 9.427 & 2.529 & 2.326 & 2.408 \\
\hline
\hline
\end{tabular}
\end{minipage}
\end{center}
\end{table} 

Проведение теста с асимптотическими уточнениями: рассмотрим тестирование гипотезы $H_0:\beta_{3}=0$ с альтернативной гипотезой $H_a: \beta_3 \neq 0$ на уровне значимости $0.05$. Тест с асимптотическим уточнениями основан на $t$-статистике, которая асимптотически не зависит от неизвестных параметров. Из раздела $11.2.6$ для каждой бутстрэповской выборки мы рассчитали $t_3^*=({\hat{\beta}_3}^*-4.664)/s_{\hat{\beta}_3^*}$, значения которой центрированы около оценки $\hat{\beta}_3=4.664$, рассчитанной на основе первоначальной выборки. Для несимметричного теста критические значения бутстрэпа равны нижнему и верхнему $2.5\%$-ому перцентилю для 999-ти значений $t_3^*$, т.е. 25-ому наименьшему и 25-ому наибольшему значению. Из Таблицы 11.1 следует, что эти значения равны -2.183 и 2.066, соответственно. Поскольку значение $t$-статистики рассчитанное на основе первоначальной выборки равно $t_3=(4.664-0)/1.741=2.679 > 2.066$, нулевая гипотеза отвергается. При проведении симметричного теста, в котором используется верхний $5\%$ перцентиль абсолютного значения статистики $|t_3^*|$, критическое значение бутстрэпа равно 2.078, что вновь приводит к отвержению нулевой гипотезы на уровне значимости 0.05.

В этом примере критические значения бутстрэпа превышают критические значения, вычисленные с помощью асимптотической аппроксимации стандартным нормальным распределением или с помощью $t(47)$ распределения, ad-hoc корректировки для конечной выборки, верной в линейной регрессии с нормальными ошибками. Как правило, обычные асимптотические результаты в этом примере приводят к отверганию нулевой гипотезы слишком часто и фактический размер теста превышает номинальный. Например, на $5\%$-ом уровне область, где $H_0$ не отвергается с использованием стандартного нормального распределения, $(-1.960, 1.960)$, уже бутстрэповской области $(-2.183, 2.066)$. На рисунке 11.1 изображен график ядерной оценки функции плотности обычного $t$-теста, основанный на $t_3^*$. Для сравнения на графике изображена плотность стандартного нормального распределения. Плотности похожи, но левый хвост распределения заметно тяжелее для бутстрэповских оценок. В Таблице 11.1 подчеркнуты отличия в хвостах.

Тестирование гипотез без асимптотических уточнений: Существуют иные бутстрэповкие способы тестирования, но эти методы не предполагают асимптотических уточнений. Во-первых, можно воспользоваться бутстрэповской оценкой стандартной ошибки, равной 1.939, вместо обычной асимптотической оценки, равной 1.7741. При этом  будет получено значение $t_3$ равное $t_3=(4.664-0)/1.939=2.405$. Это приводит к отвержению нулевой гипотезы на уровне значимости 0.05, где критическое значение может быть как стандартным нормальным, так и t(47). Во-вторых, из Таблицы 11.1 $95\%$ бутстрэповских оценок $\hat{\beta}_3^*$ лежат в промежутке (0.501, 8.484),следовательно нулевая гипотеза $H_0:\beta_3=0$ отвергается.

\vspace{3cm}

График 11.1 Плотность $t$-статистики бутстрэпа для угла наклона, равного нулю, которая получена с помощью 999 бутстрэповских оценок. Для сравнения изображена плотность стандартного нормального распределения. Данные сгенерированы с помощью экспоненциальной регрессионной модели.


Доверительные интервалы: Асимптотическое уточнение получается с использованием $t$-перцентильного $95\%$-ного
доверительного интервала. Используя (11.6) получим  $(4.664-2.183 \times 1.741,4.664+2.066 \times 1.741)$ или $(0.864,8.260)$. Для  сравнения приведем традиционный $95\%$-ти доверительный интервал $4.664 \pm 1.960 \times 1.741$ или $(1.25, 8.08)$.

Также могут быть построены иные доверительные интервалы, но уже без асимптотических уточнений. Используя оценку стандартной ошибки, рассчитанную бутстрэпом, получим $95\%$ доверительный интервал $4.664 \pm 1.960 \times 1.939=(0.864,8.464)$. Метод перцентилей использует нижний и верхний 2.5 перцентиль 999-ти оценок коэффициента, рассчитанных методом бутстрэпа, в результате чего границы доверительного интервала становятся $(0.501,8.484)$.
 
Корректировка смещения: Среднее значение 999-ти бутстрэповских оценок $\beta_3$ равно $4.716$ в сравнении с первоначальной оценкой 4.664. Оцененное смещение (4.716-4.664)=0.052 достаточно мало, особенно в сопоставлении со стандартной ошибкой $s_3=1.741$. Таким образом, оценка смещена вверх и согласно (11.10) скорректированная на смещение оценка равна 4.664-0.052=4.612.

В основе бутстрэпа лежит асимптотическая теория и, для конечных выборок, результаты применения бутстрэпа могут быть хуже результатов традиционных методов. Для того, чтобы определить действительно ли бутстрэп ведет к улучшению результатов необходимо провести анализ Монте-Карло, скажем создать 1,000 выборок размером 50 согласно экспоненциальному процессу, порождающему данные и для каждой из созданных выборок сделать ресэмплинг, скажем, 999 раз.


\section{Теория бутстрэпа}

Изложенный материал в этом разделе следует изложению Хоровица (2001). Основные результаты: оценки бутстрэпа состоятельны, и, если бутстрэпируется статистика не зависящая от неизвестных параметров, то возможно сделать асимптотическое уточнение.

\subsection{Бутстрэп}

Мы используем общее обозначение $X_1,\ldots ,X_N$, где для простоты обозначений не используем выделение жирным шрифтом для $X_i$, хотя он обычно является вектором, как например пара $(y_i,x_i)$. Предположим, что данные независимы и порождены распределением $F_0(x)=\Pr[X \leq x]$. В простых приложениях $F_0$ зависит от конечного количества параметров, $F_0=F_0(x,\theta_0)$.

Обозначим интересующую нас статистику через $T_N=T_N(X_1,\ldots ,X_N)$. Точное распределение $T_N$ на конечной выборке есть функция $G_N=G_N(t,F_0)=\Pr[T_N \leq t]$. Проблема состоит в том, чтобы найти хорошую аппроксимацию для $G_N$.

Обычная асимптотическая теория использует асимптотическое распределение $T_N$, обозначенное $G_{\infty}=G_{\infty}(t,F_0)$. Теоретически данное распределение может зависеть от неизвестной функции $F_0$, в таком случае будет использована состоятельная оценка $F_0$. Например, можно использовать $\hat{F}_0=F_0(\cdot,\hat{\theta})$, где $\hat{\theta}$ состоятельная оценка $\theta_0$.

В эмпирическом бутстрэпе используется иной подход к аппроксимации $G_N(\cdot,F_0)$. Вместо замены $G_N$ на $G_{\infty}$ производится замена функции распределения генеральной совокупности $F_0$ на состоятельную оценку $F_0$ --- функцию $F_N$, например, на  эмпирическую функцию  распределения.

Функция $G_N(\cdot,F_N)$ не может быть выписана явно, но возможно рассчитать приближение к ней бутстрэпированием. В результате применения бутстрэпа один раз получим статистику $T_N^*=T_N(X_1^*,\ldots ,X_N^*)$. Повторяя этот шаг $B$ независимых раз получим $T_{N,1}^*,\ldots ,T_{N,B}^*$. Эмпирическая функция распределения $T_{N,1}^*,\ldots ,T_{N,B}^*$ является бутстрэповской оценкой распределения статистики $T$ и

\begin{equation}
\hat{G}_N(t,F_N)=\dfrac{1}{B}\sum_{b=1}^{B}1(T_{N,b}^* \leq t),
\end{equation}
где $1(A)$ равно 1, если событие $A$ происходит и, в противном случае, равно нулю. По сути эта функция равна доле бутстрэповских выборок, для которых выполняется неравенство $T_N^* \leq t$.

Все обозначения собраны в Таблице $11.2$.


\subsection{Состоятельность бутстрэпа}

Очевидно, что оценка бутстрапа $\hat{G}_N(t,F_N)$ сходится к $G_N(t,F_N)$ при $B \rightarrow \infty$. Для того, чтобы оценка $\hat{G}_N(t,F_N)$ функции $G_N(t,F_0)$ была состоятельно необходимо выполнение условия:

\[
G_N(t,F_N) \stackrel{p}{\rightarrow} G_N(t,F_0),
\]
равномерно по $t$ и для всех $F_0$ в пространстве возможных функций  распределения.

\begin{table}[h]
\begin{center}
\caption{\label{tab:pred} Обозначения для теоремы о бутсрэпе}
\begin{minipage}{\textwidth}
\begin{tabular}{ll}
\hline
\hline
Показатель & Обозначение \\ 
\hline 
Выборка\footnote{Выборка состоит из независимых и одинаково распределенных $X_i$} & $X_1, \ldots, X_N$, где $X_i$ --- это обычно вектор \\ 
Теоретическая функция распределения $X$ & $F_0 = F_0(x, \theta_0) = \Pr[X \leq x]$ \\
Статистика & $T_N = T_N(X_1, \ldots, X_N)$ \\
Функция распределения $T_N$ для конечных выборок & $G_N = G_N(t, F_0) = \Pr[T_N \leq t]$ \\
Предельная функция распределения $T_N$ & $G_{\infty} = G_{\infty}(t, F_0)$ \\
Асимптотическая функция распределения $T_N$ & $\hat{G}_{\infty} = G_{\infty}(t, \hat{F}_0)$, где $\hat{F}_0 = F_0(x, \hat{\theta})$ \\
Бутстрэпированная функция распределения $T_N$ & $\hat{G}_N(t, F_N) = B^{-1} \sum_{b=1}^B 1(T^*_{N,b} \leq t)$ \\
\hline
\hline
\end{tabular}
\end{minipage}
\end{center}
\end{table} 

Конечно, $F_N$ должна быть состоятельна для $F_0$. Кроме того, необходима гладкость процесса, порождающего данные, с распредлением $F_0(x)$, чтобы функции $F_N(x)$ и $F_0(x)$ были  близки равномерно по наблюдениям $x$ при больших $N$. Более того, необходима гладкость $G_N(\cdot,F)$, как фукционала от $F$, чтобы  $G_N(\cdot,F_N)$ и $G_N(\cdot,F_0)$ были близки при больших $N$. 

Хоровиц (2001, стр. 3166-3168) сформулировал две теоремы, одну общую и вторую --- для одинаково и независимых распределенных данных. Хоровиц приводит примеры, когда бутстрэп может не работать, например, при оценке медианы, а также наличии ограничений на  значения параметров.

При условии, что оценка $F_N$ состоятельна для $F_0$ и выполняются условия гладкости функций $F_0$ и $G_N$, бутстрэп дает состоятельные оценки и асимптотически верные выводы. В целом, бутстрэп дает состоятельные оценки для широкого круга задач.

\subsection{Разложение Эджворта}

Дополнительным преимуществом бутстрэпа является возможность асимптотических уточнений. Сингх (1981) предложил доказательство с использованием разложения Эджворта.

Рассмотрим предельное поведение $Z_N=\sum_{i}X_i/\sqrt{N}$, где для простоты обозначим за $X_i$ независимые и одинаково распределенные с параметрами $[0, 1]$ случайные величины. Следовательно, применение центральной предельной теоремы (ЦПТ) приводит к  стандартному нормальному распределению предела $Z_N$. Более точно,  функция распределения $Z_N$ имеет вид

\begin{equation}
G_N(z)=\Pr[Z_N \leq z] = \Phi(z) + O(N^{-1/2}),
\end{equation} 
где $\Phi(\cdot)$ функция стандартного нормального распределения. Отбрасывая остаточный член, обычная асимптотическая теория говорит, что $G_N(z)$ приблизительно равна $G_{\infty}(z)=\Phi(z)$.

ЦПТ, приводящая к формуле  (11.2), формально доказывается путем простой аппроксимации характеристической функции $Z_N$, $\E[e^{isZ_N}]$, где $i=-\sqrt{1}$. Более точная аппроксимация разлагает эту характеристическую функцию по степеням $N^{-1/2}$. В результате разложения Эджворта добавляется два дополнительных члена

\begin{equation}
G_N(z)=\Pr[Z_N \leq z]=\Phi(z)+\dfrac{g_1(z)}{\sqrt{N}}+\dfrac{g_2(z)}{N}+O(N^{-3/2}),
\end{equation}
где $g_1(z) = -(z^2-1)\phi(z) \kappa_3/6$, $\phi(\cdot)$ --- плотность стандартного нормального распределения, $\kappa_3$ --- третья кумулянта $Z_N$ и полное выражение для $g_2(\cdot)$ дано у Розенберга (1984, стр. 895) и Амэмия (1985, стр. 93). В общем случае, $r$-ая кумулянта $\kappa_r$ --- это $r$-ый коэффициент разложения ряда $\ln(\E[e^{isZ_N}])=\sum_{r=0}^{\infty} \kappa_r(is)^r/r!$ логарифмической характеристической  функции или, иначе, производящей функция кумулянт.

Опуская остаточный член в (11.13), разложение Эджворта даст приближенное значение функции $G_N(z,F_0)$, т.е. $G_{\infty}(z,F_0)=\Phi(z)+N^{-1/2}g_1(z)+N^{-1}g_2(z)$. Если принять $Z_N$ --- это тестовая статистика, то значение $Z_N$ можно использовать для расчета P-значений и критических значений. Иначе, (11.3) можно записать, как

\begin{equation}
\Pr \left[ Z_N+\dfrac{h_1(z)}{\sqrt{N}}+\dfrac{h_2(z)}{N} \leq z \right] \simeq \Phi(z),
\end{equation}
где выражения для функций $h_1(z)$ и $h_2(z)$ даны у Розенберга (1984, стр. 895). В левой части находится модифицированная статистика, лучше приближаемая стандартным нормальным распределением, нежели чем  первоначальная статистика $Z_N$.

Трудность использования такого подхода заключается в необходимости вычисления кумулянтов $Z_N$ для расчета значений функций $g_1(z)$ и $g_2(z)$ или $h_1(z)$ и $h_2(z)$. Составление выражения для расчета кумулянт в аналитическом виде может вызвать затруднения (см. например, Сурган, 1980 и Филлипс, 1983). Бутстрэп позволяет использовать разложение Эджворта без расчета кумулянт.

\subsection{Асимптотические уточнения с помощью бутстрэпа}

Вернемся к рассмотрению  вопроса в более общем виде, как в разделе 11.4.1., и сделаем дополнительное предположение о том, что $T_N$  имеет в пределе стандартное нормальное распределение и скорость сходимости $\sqrt{N}$.

Традиционные методы асимптотики используют предельную  функцию распределения $G_{\infty}(t,F_0)$ как приближение истинной функции распределения $G_N(t,F_0)$. Для $\sqrt{N}$-состоятельных асимтотически нормальных оценок такая аппроксимация дает ошибку в пределе кратную $N^{-1/2}$. Мы можем записать вышесказанное как

\begin{equation}
G_N(t,F_0)=G_{\infty}(t,F_0)+O(N^{-1/2}),
\end{equation}
где в нашем пример $G_{\infty}(t,F_0)=\Phi(t)$.

Более точная аппроксимация возможна с использованием разложения Эджворта. Тогда

\begin{equation}
G_N(t,F_0)=G_{\infty}(t,F_0)+\dfrac{g_1(t,F_0)}{\sqrt{N}}+\dfrac{g_2(t,F_0)}{N}+O(N^{-3/2}).
\end{equation}
К сожалению, как это было отмечено ранее, могут возникнуть трудности при построении функций $g_1(\cdot)$ и $g_2(\cdot)$, находящихся в правой части выражения.

Далее рассмотрим оценку $G_N(t,F_N)$ полученную бутстрэпом. Согласно разложению Эджворта получим 

\begin{equation}
G_N(t,F_N)=G_{\infty}(t,F_N)+\dfrac{g_1(t,F_N)}{\sqrt{N}}+\dfrac{g_2(t,F_N)}{N}+O(N^{-3/2});
\end{equation}
для более детального изложения см. Холл (1992). Оценка бутстрэпа $G_N(t,F_N)$ используется для аппроксимации функции распределения для конечной выборки $G_N(t,F_0)$. Вычитая (11.16) из (11.17), получим

\begin{equation}
G_N(t,F_N)-G_N(t,F_0)=[G_{\infty}(t,F_N)-G_{\infty}(t,F_0)]+\dfrac{[g_1(t,F_N)-g_1(t,F_0)]}{\sqrt{N}}+O(N^{-1}).
\end{equation}
Предположим, что $F_N$ является $\sqrt{N}$-состоятельной для истинной функции распределения $F_0$, т.е. $F_N-F_0=O(N^{-1/2})$. Для непрерывной функции $G_{\infty}$ первый член в правой части выражения (11.18), $[G_{\infty}(t,F_N)-G_{\infty}(t,F_0)]$, равен $O(N^{-1/2})$, таким образом $G_N(t,F_N)-G_N(t,F_0)=O(N^{-1/2})$.

Следовательно, в общем случае значение $G_N(t,F_N)$ не является асимптотически более близким к значению $G_N(t,F_0)$, чем $G_{\infty}(t,F_0)$; см. (11.15).

Теперь предположим, что статистика $T_N$ и, следовательно, её функция распределения $G_{\infty}$ асимптотически не зависят от неизвестных параметров. Это, например, возможно, если $T_N$ стандартизирована таким образом, что предельное распределение статистики является стандартным нормальным. В таком случае, $G_{\infty}(t,F_N)=G_{\infty}(t,F_0)$ и выражение (11.18) упрощается до 

\begin{equation}
G_N(t,F_N)-G_N(t,F_0)=N^{-1/2}[g_1(t,F_N)-g_1(t,F_0)]+O(N^{-1}).
\end{equation}
Поскольку $F_N-F_0=O(N^{-1/2})$, получим, что $[g_1(t,F_N)-g_1(t,F_0)]=O(N^{-1/2})$ для $g_1$ непрерывно зависящей от $F$. После упрощения имеем $G_N(t,F_N)=G_N(t,F_0)+O(N^{-1})$. Приближение, полученное при помощи бутстрэпа, $G_N(t,F_N)$ является более точным асимптотическим приближением для $G_N(t,F_0)$, поскольку теперь ошибка приближения равна $O(N^{-1})$.

Подводя итог, отметим, что в результате применения бутстрэпа для статистики, асимптотически независящей от неизвестных параметров, мы получаем выражение

\begin{equation}
G_N(t,F_0)=G_N(t,F_N)+O(N^{-1}),
\end{equation} 
которое свидетельствует об улучшении традиционной аппроксимации $G_N(t,F_0)=G_{\infty}(t,F_0)+O(N^{-1/2})$.

Таким образом, применение бутстрэпа для статистики асимптотически  не зависящей от неизвестных параметров повышает точность работы с малой выборкой в следующем смысле. Допустим, что $\alpha$ это номинальный размер теста. При использовании обычных асимптотических результатов,  фактический размер $t$-тестов равен $\alpha+O(N^{-1/2})$, в то время как при применении бутстрэпа размер $t$-тестов составляет $\alpha+O(N^{-1})$.

Можно показать, что для симметричного теста двусторонней гипотезы и построения доверительного интервала, в результате применения бутстрэпа для статистики в узком смысле слова ошибка приближения будет равна $O(N^{-3/2})$. В то время как  при использовании обычной теории асимптотики ошибка равна $O(N^{-1})$. 

Предыдущие результаты ограничены статистикой с асимптотически нормальным распределением. Для тестовой статистики, имеющей распределение хи-квадрат выигрыш от асимптотики аналогичен выигрышу для симметричного двустороннего теста гипотезы. Доказательство для коррекции смещения с помощью бутстрапа можно найти у  Хоровица (2001, стр. 3172).

Теоретический анализ приводит к следующим выводам. При  бутстрэпа должно использоваться  распределение $F_N$, состоятельное для $F_0$. Одним из требований применения бутстрэпа является гладкость и непрерывность функций $F_0$ и $G_N$. Если, например, существует разрыв из-за ограничений на параметры, к примеру, $\theta \geq 0$, то необходима модификация обычного бутстрэпа. Бутстрэп предполагает наличие моментов низшего порядка, поскольку кумулянты низшего порядка присутствуют в разложении Эджворта функции $g_1$. Для того, чтобы сделать асимптотические уточнения необходимо использовать статистику асимптотически не зависящую от неизвестных параметров. В рассмотренных выше уточнениях предполагалось, что данные независимо и одинаково распределены, так что модификация необходима даже если ошибки гетероскедастичны. Более подробное обсуждение см. у Хоровица (2001). 


 \subsection{Мощность тестов бутстрэпа}

Анализ бутстрэпа сосредоточен на получении тестов правильного размера на малых выборках. Изменение размера бутстрэп теста приведет к изменению мощности теста, как и при любом другом изменении размера. 

Интуитивно, если реальный размер теста, полученного с использованием асимптотики первого порядка, превышает его номинальный размер, то бутстрэпирование с асимптотическим уточнениям не только  сократит размер теста до номинального, но и приведет к сокращению мощности теста, поскольку отвергание гипотезы будет  происходить реже. Наоборот, в случае, если реальный размер меньше номинального, тогда бутстрэпирование приведет к увеличению мощности. Это видно в численном эксперименте Хоровица (1994, стр. 409). Стоит отметить, что проведя эксперименты Хоровиц сделал вывод, что при  бутстрэпировании асимптотически эквивалентных тестов, получаются тесты примерно равных фактических размеров  (близких к номинальному размеру), но возможны значительные различия в мощности полученных тестов. 


\section{Обобщения бутстрэпа}

Рассмотренные ранее методы бутстрэпа применялись к гладкой $\sqrt{N}$-состоятельной асимптотически нормальной оценке на независимо и одинаково распределенных данных. Приводимые обобщения позволяют получить для широкого класса задач асимптотически состоятельные бутстраповские оценки (Разделы 11.5.1 и 11.5.2) или состоятельного с асимптотическим уточнением (Разделы 11.5.3-11.5.5). Данные методы мы излагаем кратко. Некоторые из расширенных методов применяются в Разделе 11.6.

\subsection{Метод подвыборок}

В методе подвыборок используется выборка размера $m$, которая значительно меньше, чем размер выборки $N$. Подвыборка может быть как с повторениями (Бикель, Готце и ван Цвет, 1997) или без повторений (Политис и Романо, 1994).

В результате подвыборки с повторениями получаются случайные выборки из  генеральной совокупности, а не случайные выборки оценок как  при парном бутстрэпе. Подвыборка с повторениями может дать состоятельный результат в тех случаях, когда условия гладкости раздела 11.4.2 не выполняются и  бутстрэп по всей выборке несостоятелен. Однако, соответствующая асимптотическая ошибка при тестировании или построении  доверительных интервалов выше, чем $O(N^{-1/2})$, получаемая при использовании бутстрэпа без уточнений, когда он возможен.

Бутстрэп по подвыборке следует использовать, когда оценки бутстрэпа с использованием полной выборки ошибочны, или для проверки последнего. Результаты отличаются в зависимости от размера подвыборки. Кроме того, значительный рост ошибок может быть вызван сокращением размера выборки. В действительность нужны условия $(m/N) \rightarrow 0$ и $N{\rightarrow}\infty$. Для более подробного рассмотрения вопроса см. Политис, Романо, а также Вульф (1999) и Хоровиц (2001).

\subsection{Блочный бутстрэп}

Блочный бутстрэп используются для зависимых данных. Согласно данному методу выборка разбивается на $r$ неперекрывающих друг друга блоков длинной $l$, где $rl \simeq N$. Сначала  строится выборка с повторениями из этих блоков, полученные новые $r$ блоков будут расположены в порядке, который отличается от их изначального порядка. Далее производится оценка параметров с использованием  этой выборки. 

В блочном бутстрэпе  предполагается независимость случайно выбранных блоков, однако внутри блока данные могут быть зависимы. Аналогичное разбиение на блоки  было применено Андерсоном (1971) для докательства ЦПТ для слабозависимых случайных процессов. Для блочного бутстрапа нужны условия $r \rightarrow \infty$ при $N \rightarrow \infty$, чтобы вероятность получения соседних некоррелированных блоков была высока. Также необходимо, чтобы длина блока $l \rightarrow \infty$ при $N \rightarrow \infty$. К примеру, см. Готце и Кунш (1996).

\subsection{Вложенный бутстрэп}

Вложенный бутстрэп был предложен Холлом (1986), Бераном (1987) и Ло (1987) и  представляет собой бутстрэп в бутстрэпе. Данный метод особенно полезен, если бутстрэп построен для статистики асимптотически зависящей от неизвестных параметров.   В частности, при трудностях в расчете стандартных ошибок оценок возможно бутстрэпировать уже имеющуюся выборку бутстрэпа для того, чтобы получить стандартные ошибки $s_{\hat{\theta}^*,Boot}$ и сформировать статистику $t^*=(\hat{\theta}^*-\hat{\theta})/s_{\hat{\theta}^*,Boot}$, и затем применить  $t$-перцентильный метод для бутстрэповских репликаций  $t_1^*,\ldots ,t_B^*$. Такой подход позволяет получить асимптотические уточнения в тех случаях, где при однократном бутстрэпе таких уточнений сделать нельзя.

В более общем смысле, итерационное бутстрэпирование это способ улучшить бутстрэп через оценку и исправление ошибок (смещения), которые возникают, если бутстрэп применяется один раз. В общем случае, при каждой последующей итерации бутстрэпа происходит корректировка смещения на $N^{-1}$, если используется статистика асимптотически не зависящая от неизвестных параметров, и на $N^{-1/2}$ иначе. Этот вопрос подробно рассмотрен в работе Холл и Мартин (1988). Если количество бутстрэпов равно $B$ для одной итерации, тогда при количестве итераций равном $k$ необходимо провести $B^k$ бутстрэпов. Из-за этого, как правило, проводят не более двух итераций и такой бутстрэп получил название двойного или калиброванного бутстрэпа.

Дэвидсон, Хинкли и Шехтман (1986) предложили  сбалансированный бутстрэп. Данный подход позволяет использовать выборочные данные одинаковое количество по всем $B$ бутстрэпам, что в итоге должно привести к улучшению оценок, полученных с помощью бутстрэпа. Практическое применение см. в работе Глиcон (1988), чьи алгоритмы не сильно увеличивают время расчетов по сравнению с обычным несбалансированным бутстрэпом.

\subsection{Центрирование и изменение масштаба}

Для  асимптотических уточнений в основе бутстрэпа должна быть оценка $\hat{F}$ функции $F_0$, которая накладывает все необходимые предпосылки на рассматриваемую модель. Основным примером здесь служит бутстрэп остатков.

В нелинейных моделях, а также в линейных моделях без свободного члена сумма МНК остатков не равна нулю. Для бутстрэпа остатков (см. Раздел 11.2.4), в котором используются МНК остатки, не выполняется условие $\E[u_i]=0$. Поэтому при бутстрэпе остатков нужно использовать центрированные остатки $\hat{u}_i-\overline{u}$, где $\overline{u}=N^{-1}\sum_{i=1}^N\hat{u_i}$. Аналогичное центрирование должно быть проведено для парного бутстрэпа, когда с помощью ОММ оценивается сверх-идентифицированная модель (см. Раздел 11.6.4).

Кроме того иногда полезно изменить масштаб остатков. Например, в линейной регрессионной модели с независимо и одинаково распределенными ошибками можно делать ресэмплинг масштабированных остатков $(N/(N-K))^{1/2}\hat{u}_i$, поскольку они имеют дисперсию равную $s^2$. Возможны и другие корректировки, например, использование стандартизованных остатков $\hat{u}_i/\sqrt{(1-h_{ii})s^2}$, где $h_ii$ стоит на i-том месте главной диагонали в проекционной матрице $X(X'X)^{-1}X'$.

\subsection{Джекнайф}

Для корректировки смещения может быть использован бутстрэп (см. Раздел 11.2.8). Альтернативным методом ресэмплинга является джекнайф (jackknife), предшественник бутстрэпа. В методе джекнайф используются $N$ детерминистических подвыборок размера $N-1$, каждая из которых генерируется путем отбрасывания по очереди одного из $N$ наблюдений и затем производится пересчет оценки.

Для того, чтобы увидеть как работает джекнайф, обозначим за $\hat{\theta}_N$ оценку параметра $\theta$, полученную на основе выборки размера $N$ и за $\hat{\theta}_{N-1}$ оценку параметра $\theta$, полученную на основе выборки по первым $N-1$ наблюдению. Если 11.7 выполняется, тогда $\E[\hat{\theta}_N]=\theta+a_N/N+b_N/N^2+O(N^{-3})$ и $\E[\hat{\theta}_{N-1}]=\theta+a_N/(N-1)+b_N/(N-1)^2+O(N^{-3})$, следовательно, $\E[N\hat{\theta}_N-(N-1)\hat{\theta}_{N-1}]=\theta+O(N^{-2})$. Таким образом, смещение оценки $N\hat{\theta}_N-(N-1)\hat{\theta}_{N-1}$ меньше смещения $\hat{\theta}_N$.

Однако данная оценка может иметь большую дисперсию, поскольку для расчета $N\hat{\theta}_N-(N-1)\hat{\theta}_{N-1}$ используется меньше данных. В граничном случае, когда $\hat{\theta}=\overline{y}$, оценка будет равна $y_N$, т.е. $N$-тому наблюдению. Дисперсия может быть уменьшена путем последовательного отбрасывания отдельных наблюдений и усредняя.

Рассмотрим процедуру расчета $\hat{\theta}$, оценки вектора параметров $\theta$, полученную на основе выборки размера $N$ с использованием независимо и одинаково распределенных данных. Из всего массива $i=1,\ldots ,N$ исключается $i$-ое наблюдение, затем рассчитывается $N$ репликаций джекнайф  оценки $\hat{\theta}_{(-i)}$, используя $N$ джекнайф выборок размера $(N-1)$. Джекнайф оценка смещения $\hat{\theta}$ равна $(N-1)(\overline{\hat{\theta}}-\hat{\theta})$, где $\hat{\theta}=N^{-1}\sum_{i}\hat{\theta}_{(-i)}$ среднее значение $N$ репликаций джекнайф  $\hat{\theta}_{(-i)}$. Смещение кажется большим из-за домножения на $(N-1)$, вместе с тем разница $(\hat{\theta}_{(-i)}-\hat{\theta})$ намного меньше, чем в случае применения бутстрэпа, поскольку джекнайф выборка отличается от первоначальной выборки только на одно наблюдение. 

Мы получаем джекнайф оценку параметра $\theta$ с коррекцией смещения:

\begin{equation}
\hat{\theta}_{Jack}=\hat{\theta}-(N-1)(\overline{\hat{\theta}}-\hat{\theta})
\end{equation}

\[
=N\hat{\theta}-(N-1)\overline{\hat{\theta}}.
\]
Корректировка позволяет сократить смещение от $O(N^{-1})$ до $O(N^{-2})$, что равно порядку сокращения смещения для бутстрэпа. Также предполагается, что как и в случае бутстрэпа, оценка является  гладкой и $\sqrt{N}$-состоятельной. Джекнайф оценка может иметь большую вариацию, чем $\hat{\theta}$, примеры, где джекнайф не работает даны в Миллер (1974).

Рассмотрим простой пример оценки дисперсии $\sigma^2$ независимо и одинаково распределенных величин выборки $y_i \sim [\mu,\sigma^2]$. ММП оценка дисперсии для нормального распределения, $\hat{\sigma}^2=N^{-1} \sum_{i}(y_i-\overline{y})^2$, имеет математическое ожидание равное $\E[\hat{\sigma^2}]=\sigma^2(N-1)/N$, при этом смещение порядка  $O(N^{-1})$  равно $\hat{\sigma}^2/N$. В данном примере джекнайф оценка упрощается до $\hat{\sigma}^{2}_{Jack}=(N-1)^{-1}\sum_{i}(y_i-\overline{y})^2$, так что не возникает необходимости рассчитывать $N$ оценок дисперсии $\hat{\sigma}^{2}_{(-i)}$ по-отдельности. Джекнайф оценка дисперсии $\sigma^2$ является несмещенной, таким образом, смещение равно нулю, а не $O(N^{-2})$ как в общем случае.

Джекнайф впервые описан Кенуй (1956). Тьюки (1958) рассматривал применение данного метода к различным  статистикам. В частности, джекнайф оценка стандартной ошибки оценки параметра, $\hat{\theta}$, равна 

\begin{equation}
\hat{se}_{Jack}[\hat{\theta}]=\left[\dfrac{N-1}{N}\sum^{N}_{i=1}(\hat{\theta}_{(-i)}-\overline{\hat{\theta}})^2\right]^{1/2}.
\end{equation}
Тьюки предложил термин джекнайф (складной нож) по следующей аналогии: с помощью складного ножа можно  решить различные проблемы, каждая из которых может быть решена более рационально, если применить специально разработанный способ. Джекнайф является приближенным методом для снижения смещения во многих случаях, вместе с тем, данный метод не является идеальным. Джекнайф может быть рассмотрен как линейная аппроксимация бутстрэпа (Эфрон и Тибшарани, 1993, p.146). При использовании выборок малого размера джекнайф требует меньше расчетов, чем  бутстрэп, поскольку тогда вероятно, что $N<B$, однако джекнайф уступает бутстрэпу  при $B \rightarrow \infty$.

Рассмотрим линейную регрессионную модель $y=X\beta+u$, где $\hat{\beta}=(X'X)X'y$. В качестве примера для получения смещенной оценки, возьмем модель временного ряда с лаговой зависимой переменной в качестве регрессора и оценим ее с помощью МНК. Оценка коэффициентов рассчитанная на основе $i$-ой джекнайф выборки  $(X_{(-i)},y_{(-i)})$ равна:

\[
\hat{\beta}_{(-i)}=[X'_{(-i)}X_{(-i)}]^{-1}X'_{(-i)}y_{(-i)}
\]


\[
=[X'X-x_{i}x'_i]^{-1}(X'y-x_{i}y_i)
\]


\[
=\hat{\beta}-[X'X]^{-1}x_i(y_i-x'_{i}\hat{\beta}_{(-i)}).
\]
В третьем равенстве отсутствует необходимость обращать матрицу $X'_{(-i)}X_{(-i)}$, для того, чтобы получить его, заметим, что

\[
[X'X]^{-1}=[X'_{(-i)}X_{(-i)}]^{-1}
-\dfrac{[X'_{(-i)}X_{(-i)}]^{-1} x_i x'_i[X'_{(-i)} X_{(-i)}]^{-1}}{1+x'_{i}[X'_{(-i)}X_{(-i)}]^{-1}x_{i}}.
\]
В данном случае псевдо-значения равны $N\hat{\beta}-(N-1)\hat{\beta}_{(-i)}$ и джекнайф оценка $\hat{\beta}$ равна 

\begin{equation}
\hat{\beta}_{Jack}=N\hat{\beta}-(N-1)\dfrac{1}{N}\sum^{N}_{i=1}\hat{\beta}_{(-i)}.
\end{equation}

Интересным применением джекнайф оценки для снижения смещения является джекнайф оценка для инструментальных переменных (см. раздел 6.4.4).


\section{Практическое применение бутстрэпа}

В данном разделе будет рассмотрено практическое применение бутстрэпа с учетом типичных микроэконометрических усложнений таких, как гетероскедастичность и кластеризация, а также более сложные случаи, которые могут привести к невозможности применить обычный бутстрэп.

\subsection{Гетероскедастичные ошибки}

При применении метода наименьших квадратов для оценки моделей с гетероскедастичными аддитивными ошибками, стандартная процедура предполагает использование ковариационной гетероскедастично-состоятельной матрицы Уайта для оценок (HCCME). Для малых выборок данная матрица плохо работает. При верном применении бутстрэп может способствовать улучшению оценок.

Парный бутстрэп приводит к верным результатам, поскольку предположение о независимости и одинаковом распределении $(y_i,x_i)$ позволяет значениям $\V[u_i|x_i]$ зависеть от $x_i$ (см. Раздел 4.4.7). Однако, парный бутстрэп не позволяет сделать асимптотические уточнения поскольку этот метод не накладывает условия, что $\E[u_i|x_i]=0$.

Бутстрэп остатков приводит к ошибочным результатам, поскольку в этом методе делается предположение о  независимости и одинаковом распределения условных ошибок $u_i|x_i$, а значит ошибочно налагается условие гомоскедастичности ошибок. В обозначениях Раздела 11.4, оценка функции $\hat{F}$ будет несостоятельна для $F$. Возможно специфицировать конкретную модель гетероскедастичности остатков, например, $u_i=\exp(z'_i\alpha)\e_i$, где $\e_i$ независимы и одинаково распределены, затем рассчитать оценку $\exp(z'_i \hat{\alpha})$ и  бутстрэпировать соответствующие значения остатков $\hat{\e}_i$. Состоятельность и асимптотические уточнения для такого бутстрэпа требуют верной спецификации функциональной формы для гетероскедастичности.

Дикий бутстрэп впервые был предложен Ву (1986) и Лю (1988) и дальнейшее развитие получил в работах Маммена (1993), в которых были рассмотрены асимтотические уточнения  без определения структуры гетероскедастичности. В диком бутстрэпе происходит замена МНК остатков, $\hat{u}_i$ на нижеследующие:

\[
\hat{u}^{*}=
\begin{cases} \dfrac{1-\sqrt{5}}{2}\hat{u}_i \simeq -0.6180\hat{u}_i \hspace{1.1cm} \text{с вероятностью} \hspace{0.5cm} \dfrac{1+\sqrt{5}}{2\sqrt{5}} \simeq 0.7236, \\
[1-\dfrac{1-\sqrt{5}}{2}]\hat{u}_i \simeq 1.6180\hat{u}_i \hspace{0.5cm} \text{с вероятностью} \hspace{0.5cm} 1-\dfrac{1+\sqrt{5}}{2\sqrt{5}} \simeq 0.2764.
\end{cases}
\]

Взяв математическое ожидание по отношению к этому распределению, принимающему всего два значения, и после преобразований, получим $\E[\hat{u}^{*}_i]=0$, $\E[\hat{u}^{*2}_i]=\hat{u}^{2}_i$ и $\E[\hat{u}^{*3}_i]=\hat{u}^{3}_i$. Таким образом,  остатки $\hat{u}^{*}_i$,  как и планировалось, имеют нулевое условное среднее, поскольку из условия $\E[\hat{u}^{*}_i|\hat{u}_i,x_i]=0$ следует, что $\E[\hat{u}^*_i|x_i]=0$, в то время как значения второго и третьего моментов не меняются.

При ресэмплинге в диком бутстрэпе  $i$-ое наблюдение равно $(y^{*}_i,x_i)$, где $y^{*}_i=x'\hat{\beta}+\hat{u}^{*}_i$. Значения $y^{*}$ в повторной выборке меняются поскольку меняются значения $\hat{u}^{*}_i$. В симуляциях Хоровиц (1997, 2001)  показал, что дикий бутстрэп работает гораздо лучше, чем парный бутстрэп, когда остатки гетероскедастичны, и хорошо работает на фоне других видов бутстрэпа, даже при отсутствии гетероскедастичности остатков.

Кажется удивительным, что бутстрэп работает, потому что для $i$-того наблюдения он выбирает всего из двух возможных значений для остатков, $-0.6180\hat{u}_i$ или $0.6180\hat{u}_i$. Однако берется одна и та же выборка, которая выбирается из всех $N$ наблюдений и всех $B$ бутстрэп интераций. Вспомним оценку Уайта, в которой $\E[u_i^2]$ заменяется на $\hat{u}_i^2$. Эта оценка, хотя и неверна для одного наблюдения, достоверна при усреднении выборки. Дикий бутстрэп вместо этого выбирает из значений случайной величины, которая принимает всего два значения, с математическим ожиданием 0 и дисперсией $\hat{u}_i^2$.  

\subsection{Панельные и кластеризованные данные}

Рассмотрим регрессионную модель панельных данных

\[
\tilde{y}_{it}=\tilde{w}'_{it}\theta+\tilde{u}_{it},
\]
где $i$ означает индивида и $t$ обозначает время. Следуя обозначениям Раздела 21.2.3 тильда добавлены, т.к. исходные данные $y_{it}$ и $x_{it}$ могли быть преобразованы для исключения фиксированных эффектов. Предположим, что ошибки $\tilde{u}_{it}$ независимы по $i$, при этом ошибки могут быть зависимы и коррелированы по $t$ для заданного $i$. 

Если панель короткая, $T$ конечно и асимптотическая теория опирается на $N \rightarrow \infty$, то состоятельные стандартные ошибки оценки $\hat{\theta}$ могут быть получены при помощи парного бутстрэпа или бутстрэпа эмпирического распределения, в которых создание повторной выборки происходит относительно $i$, а не относительно $t$. Тогда, $w_i$ равно $[y_{i1},x_{i1},\ldots ,y_{iT},x_{iT}]$, и ресэмплинг по $i$ даёт $T$ наблюдений для каждого выбранного $i$.

Такой панельный бутстрэп также называется блочным бутстрэпом, и может быть использован для нелинейных панельных моделей Главы 23. Основным предположением является то, что панель должна быть короткой и данные должны быть независимы по $i$. В общем случае, блочный бутстрэп может быть применен на любых кластеризованных данных (см. Раздел 24.5), при условии, что размер кластера конечен и количество кластеров стремится к бесконечности.

Панельный бутстрэп дает стандартные ошибки асимптотически эквивалентные панельным робастным сэндвич стандартным ошибкам (см. Раздел 21.2.3). Для блочного бутстрапа асимптотическое уточнение не возникает. При этом данный бутстрэп легко реализовать и он очень полезен, поскольку многие статистические пакеты не рассчитывают панельные робастные стандартных ошибок даже для базовых способов оценки панельных данных, например, для оценки модели с фиксированным эффектом. В зависимости от ситуации, иные виды бутстрэпа, такие как параметрический бутстрэп или  бутстрэп остатков, иногда возможно применить, при этом ресэмплинг снова проводится по $i$.

В случае если ошибки независимо и одинаково распределены асимптотические уточнения получить легко. Однако, более правильно считать, что $\tilde{u}_{it}$ гетероскедастичны и коррелированы по $t$, а не по $i$. Дикий бутстрэп (см. Раздел 11.6.1) дает асимптотические уточнения  для линейных моделей на коротких панелях. При ресэмплинге дикого бутстрэпа $(i,t)$-ое наблюдение выглядит как $(\tilde{y}^{*}_{it},\tilde{w}_{it})$, где $\tilde{y}^{*}_{it}=\tilde{w}'_{it}\theta+\hat{\tilde{u}}_{it}^*$, $\hat{\tilde{u}}_{it}=\tilde{y}_{it}-\tilde{w}'_{it}\hat{\theta}$ и остатки $\hat{\tilde{u}}_{it}^*$ сгенерированы с помощью распределения с двумя значениями, описанного в Разделе 11.6.1.

\subsection{Тестирование гипотез и спецификация тестов}

В Разделе 11.2.6 основное внимание уделено тестированию гипотезы $\theta=\theta_0$. В данном разделе рассматриваются более общие тесты. Также как и в Разделе 11.2.6, для тестирования гипотез может использоваться бутстрэп как с асимтотическим уточнением, так и без. 


\subsubsection*{Тесты без асимптотических уточнений}

Основной пример пользы бутстрэпа --- это  тест Хаусмана (см. Раздел 8.3). Стандартное применение теста Хаусмана требует оценки $\V[\hat{\theta}-\tilde{\theta}]$, где $\hat{\theta}$ и $\tilde{\theta}$ две сравниваемых оценки. Получить  оценку дисперсии достаточно трудно если не предположить, что одна из оценок полностью эффективна при верной  $H_0$. Вместо этого может быть использован парный бутстрэп, который дает состоятельную оценку

\[
\hat{V}_{Boot}[\hat{\theta}-\tilde{\theta}]=\dfrac{1}{B-1}\sum^{B}_{b=1}[(\hat{\theta}^{*}_b-\tilde{\theta}^{*}_{b})-(\overline{\hat{\theta}}^*-\overline{\tilde{\theta}})][(\hat{\theta}^{*}_b-\tilde{\theta}^{*}_b)-(\overline{\hat{\theta}}^*-\overline{\tilde{\theta}}^*)]',
\]
где $\overline{\hat{\theta}}^*=B^{-1}\sum_{b}\hat{\theta}^{*}_b$ и $\overline{\tilde{\theta}}^*=B^{-1}\sum_{b}\tilde{\theta}^{*}_b$. Далее рассчитаем 

\begin{equation}
H=(\hat{\theta}-\tilde{\theta})'(\hat{V}_{Boot}[\hat{\theta}-\tilde{\theta}])^{-1}(\hat{\theta}-\tilde{\theta})
\end{equation}
и полученное значение сравниваем с критическим значением хи-вадрат. Как упоминалось ранее в Главе 8, может потребоваться обобщённая обратная матрица, а также необходима аккуратность при определения критических значений хи-квадрат распределения, т.к. нужно  верно определить степени свободы.


В общем случае, такой подход может быть применен для любого теста со стандартным нормальным или хи-квадрат распределением, реализация которого затруднена необходимостью оценки ковариационной матрицы. Примером может служить двухшаговый способ оценивания и М-тесты, рассмотренные в Главе 8. 

\subsubsection*{Тесты с асимптотическими уточнениями}

Многие тесты, особенно те, которые используются для полностью параметрических моделей, такие как тест множителей Лагранжа и критерий информационной матрицы могут быть легко реализованы с помощью  вспомогательной регрессии (см. Раздел 7.3.5 и 8.2.2). Однако, получаемая тестовая статистика плохо работает на конечных выборках, как это отмечено во многих исследованиях с помощью Монте-Карло. Такие тестовые статистики легко рассчитываются и асимптотически не зависят от неизвестных параметров, т.к. имеют хи-квадрат распределение. Таким образом, эти статистики являются главными кандидатами на асимптотические уточнения с помощью бутстрэпа.

Рассмотрим М-тест гипотезы $H_0:\E[m_i(y_i|x_i,\theta)]=0$ при альтернативной $H_a:\E[m_i(y_i|x_i,\theta)] \neq 0$ (см. Раздел 8.2). Сначала рассчитывается ММП  оценка параметра $\hat{\theta}$  и вычисляется тестовая статистика $M$. Используя параметрический бутстрэп, сгенерируем $y^*_i$ из условного распределения $f(y_i|x_i,\hat{\theta})$ для фиксированных регрессоров в повторных выборках, или используя $f(y_i|x_i^*,\hat{\theta})$. Далее вычисляется $M^*_b, b=1,\ldots, B$ в повторных выборках, созданных при помощи бутстрэпа. Гипотеза $H_0$ отвергается на уровне значимости $\alpha$, если изначально рассчитанная статистика $M$ превышает квантиль уровня $\alpha$ для $M^{*}_b, b=1,\ldots, B$.

Хоровиц (1994) предложил использовать вышеописанный бутстрэп для критерия информационной матрицы и продемонстрировал на примере симуляций, что данный бутстрэпа существенно повышает точность в малых выборках. Друккер (2002) подробно исследует тесты на спецификацию тобит-моделей и приходит к выводу, что тесты на спецификацию с использованием условных моментных ограничений легко реализовать в полностью параметрических моделях,  поскольку любое искажение размера теста во вспомогательной регрессии может быть скорректировано при помощи бутстрэпа.

Отметим, что тесты бутстрэпа без асимптотических уточнений, к примеру, рассмотренный тест Хаусмана, могут быть улучшены при помощи вложенного бутстрэпа, рассмотренного в Разделе 11.5.3.

\subsection{ОММ, метод минимального расстояния и эмпирическое правдоподобие в сверх-идентифицированных моделях}


Обобщенный метод моментов (ОММ) основан на моментных условиях для генеральной совокупности: $\E[h(w_i,\theta)]=0$, (см. Раздел 6.3.1). В точно идентифицированной модели состоятельная оценка является решением уравнения $N^{-1}\sum_{i}h(w_i,\hat{\theta}) = 0$. В сверх-идентифицированной модели такое решение не существует. Вместо  этого используется ОММ оценка (см. Раздел 6.3.2).

Рассмотрим использование парного бутстрэпа или  бутстрэп эмпирического распределения. Для ОММ в сверх-идентифицированных моделях $N^{-1}\sum_{i}h(w_i,\hat{\theta}) \neq 0$, поэтому на бутстрэповские выборки  не накладывается ограничение  $\E[h(w_i,\theta)]=0$. В результате чего, даже при использовании  $t$-статистики асимптотически не зависящей от неизвестных параметров не происходит асимптотического уточнения.  При этом бутстрэп оценки $\hat{\theta}$ и соответствующие доверительные интервалы и $t$-статистики остаются состоятельными. Более фундаментально, можно доказать, что применение бутстрэпа к тестам на сверх-идентифицирующие ограничений (OIR-тест) (см. раздел 6.3.8) приводит к несостоятельным результатам. В данном разделе основное внимание уделяется пространственным данным, но аналогичная проблема может возникнуть и при использовании  ОММ для панельных данных (см. Главу 22) в сверх-идентифицированных моделях.

Холл и Хоровиц (1996) предложили бороться со сверх-идентифицированностью при помощи повторного центрирования. Согласно их подходу в основе бутстрэпа лежит $h^{*}(w_i,\hat{\theta})=h(w_i,\hat{\theta})-N^{-1} \sum_{i} h(w_i,\hat{\theta})$, асимптотические уточнения  можно сделать для статистик, рассчитанных на основе $\hat{\theta}$, включая OIR тест на сверх-идентифицированность ограничений. 

Хоровиц (1998) провел аналогичное повторное центрирование для метода минимального расстояния (см. Раздел 6.7). Далее автор применил бутстрэп к ковариационной структуре, рассмотренной Альтонжи и Сигал (1996), см. Раздел 6.3.5.

Альтернативная корректировка была предложена в работе Браун и Ньюи (2002), идея заключается в том, чтобы вместо повторного центрирования делать ресэмплинг наблюдений $w_i$ с вероятностями не всегда равными $1/N$, а отличающихся для разных наблюдений. К примеру, допустим, что $\Pr[w^{*}=w_i]=\hat{\pi}_i$, где $\hat{\pi}_i=(1+\hat{\lambda}'\hat{h}_i)$, $\hat{h}_i=h(w_i,\hat{\theta})$, и $\hat{\lambda}$ максимизирует выражение $\sum_{i}\ln(1+\hat{\lambda}'\hat{h}_i)$. Обоснование такого расчета состоит в том, что вероятности $\hat{\pi}_i$ также являются решением задачи эмпирического правдоподобия (см. Раздел 6.8.2), когда максимизируется $\sum_{i}\ln{\pi}$ по $\pi_1,\ldots ,\pi_N$, при ограничениях $\sum_{i}\pi_{i}\hat{h}_i=0$ и $\sum_{i}\pi_{i}=1$. Таким образом, бутстрэп эмпирического правдоподобия оценок ОММ накладывает ограничение вида $\sum_{i}\hat{\pi}_{i}\hat{h}_{i}=0$.

Также возможно изначально использовать оценку эмпирического правдоподобия $\hat{\theta}$, а не ОММ оценку. Преимущество подхода Брауна и Ньюи (2002) состоит в отсутствии трудных вычислений, возникающих при эмпирическом правдоподобии. Вместо этого  необходимо только рассчитать оценку ОММ и решить задачу  минимизации  вогнутой функции $\sum_{i}\ln(1+\hat{\lambda}'\hat{h}_i)$. 

\subsection{Непараметрическая регрессия}

Непараметрические оценки функции плотности  регрессии сходятся со скоростью меньше, чем $\sqrt{N}$ и асимптотически смещены. Это усложняет статистические выводы, например, построение доверительных интервалов (см. Раздел 9.3.7 и 9.5.4).

Рассмотрим ядерную оценку регрессии $\hat{m}(x_0)$ функции $m(x_0)=\E[y|x=x_0]$ для независимо и одинаково распределенных данных $(y,x)$ , при этом допускается условная гетероскедастичность остатков. Как было отмечено у Хоровиц (2001, p.3204),  статистикой асимптотически независящая от неизвестных параметров будет:

\[
t=\dfrac{\hat{m}(x_0)-m(x_0)}{s_{\hat{m}(x_0)}},
\] 
где $\hat{m}(x_0)$ недосглаженая ядерная оценка  регрессии с шириной окна $h=o(N^{-1/3})$ отличной от оптимальной $h^{*}=O(N^{-1/5})$ и 

\[
s^{2}_{\hat{m}(x_0)}=\dfrac{1}{Nh{[\hat{f}(x_0)]}^2}\sum^{N}_{i=1}{(y_i-\hat{m}(x_i))}^{2}K{\left(\dfrac{x_i-x_0}{h}\right)}^2,
\]
где $\hat{f}(x_0)$ --- ядерная оценка функции распределения $f(x)$ в точке $x=x_0$. Применение парного бутстрэпа дает ресэмплинг пар $(y^{*},x^{*})$ и статистики $t^{*}_{b}=[\hat{m}^{*}_{b}(x_0)-m(x_0)]/s^{*}_{\hat{m}(x_0),b}$, где $s^{*}_{\hat{m}(x_0),b}$ рассчитана с использованием  бутстрэповской выборки  ядерных оценок $\hat{m}^{*}_b (x_i)$ и $\hat{f}^{*}_{b}(x_0)$. Метод $t$-перцентильного доверительного интервала, рассмотренный в Разделе 11.2.7, дает асимптотическое уточнение. Для симметричного доверительного интервала или симметричного теста уровня значимости $\alpha$ ошибка равна $o((Nh^{-1}))$ вместо $O((Nh^{-1}))$, которая получается при использовании асимптотического приближения первого порядка.

Возможно несколько вариантов представленного бутстрэпа. Вместо недосглаживания для исключения смещения возможно непосредственно оценивать смещение, указанное в Разделе 9.5.2. Также, вместо $s^{2}_{\hat{m}(x_0)}$ можно использовать явную оценку дисперсии из Раздела 9.5.2.

В работе Ячью (2003) представлено много деталей применения бутстрэпа для параметрических и полупараметрических регрессий.


\subsection{Негладкие оценки}

В Разделе 11.4.2 одним из предположений при использовании бутстрэпа была гладкость оценок и статистик. Если данная предпосылка не выполняется, то в результате бутстрэп может не привести к асимптотическим уточнениям, и даже может быть полностью неверным.

В качестве примера рассмотрим метод наименьших абсолютных отклонений и бинарные данные. В методе наименьших абсолютных отклонений (см. Раздел 4.6.2) целевая функция имеет вид $\sum_{i}|y_i-x'_{i}\beta|$, первая производная этой функции разрывна. В таком случае применение бутстрэпа может дать верное асимптотическое приближение, но не асимптотическое уточнение. Для бинарных результатов метод наименьших абсолютных значений сводится к методу максимального скоринга Мански (1975) (см. Раздел 14.7.2). Для бинарных данных бутстрэп несостоятелен.

В приведенных примерах бутстрэп с асимптотическими уточнениями может быть получен при использовании сглаженного варианта целевой функции. К примеру, сглаженный метод максимального скоринга Хоровица (1992) рассмотрен в Разделе 14.7.2.


\subsection{Временные ряды}

Методология бутстрэпа построена на создании повторной выборки из независимо и одинаково распределенных величин. Использование временных рядов вызывает очевидные трудности поскольку для этого типа данных характерна зависимость наблюдений.

Применение бутстрэпа не вызывает затруднений в линейных моделях с ARMA ошибками, в которых делают ресэмплинг ошибки, являющейся белым шумом. Например, предположим, что $y_t=\beta{x}_t+u_t$, где $u_t=\rho{u}_{t-1} + \e_t$ и $\e_t$ --- белый шум. Тогда, используя полученные оценки $\hat{\beta}$ и $\rho$ мы можем рекурсивно посчитать остатки $\hat{\e}_t=\hat{u}_t-\hat{\rho} \hat{u}_{t-1}=y_t-x_{t}\hat{\beta}-\hat{\rho}(y_{t-1}-x_{t-1}\hat{\beta})$. Бутстрэпируя эти остатки, можем получить новые значения $\hat{\e}^{*}_{t},t=1,\ldots ,T$ и рекурсивно рассчитать $\hat{u}^{*}_{t}=\rho\hat{u}^{*}_{t-1}+\hat{\e}^{*}_{t}$, тогда новые значения зависимой переменной будут равны $y^{*}_{t}=\hat{\beta}x_{t}+\hat{u}^{*}_{t}$. Далее строим регрессию $y^*_t$ на $x_t$ с AR(1) ошибками. Ранее аналогичный пример был приведен в работе Фридмана (1984), который использовал бутстрэп для динамической модели системы линейных уравнений, оцененной с помощью двухшагового МНК. С учетом линейности, одновременность добавляет мало проблем. Динамическая структура модели приводит к рекурсивному построению  $y^{*}_t=f(y^{*}_{t-1},x,u^{*}_t)$, где $u^{*}_t$ получены путем ресэмплинга остатков структурных уравнений, оцененных двухшаговым МНК и $y^{*}_0=y_0$. Далее применяем двухшаговый МНК на каждой выборки бутстрэпа.

В рассмотренном методе предполагается, что ошибки  независимо и одинаково распределены. Для общего случая, без ARMA спецификации, например для  нестационарных данных, может быть применен блочный бутстрэп (см. Раздел 11.5.2).

При проведении теста на наличие единичного корня или коинтеграции необходимо особая осторожность, поскольку при наличии единичного корня тестовая статистика разрывна, см. Ли и Маддала (1997). Не смотря на то, что при наличии единичного корня возможно применить бутстрэп, в настоящее время имеющиеся виды бутстрэпа не дают асимптотические уточнения.

\section{Практические соображения}

Бутстрэп без асимптотических уточнений может быть очень полезен для исследователя в тех ситуациях, когда задачу трудно решить другими способами. 
Эта необходимость зависит от доступного программного обеспечения и методов. 
В настоящее время самое распространенное применение бутстрэпа --- это вычисления оценок стандартных ошибок для теста Вальда. 
Приложения включают в себя тесты устойчивые к гетероскедастичности, робастные тесты для панельных данных, тесты для двухшаговых оценок и тесты для функций от оцениваемых параметров.
Также возможно применение бутстрэпа для М-тестов, например, для теста Хаусмана.

Кроме этого бутстрэп даёт возможность получить асимптотические уточнения.
Большое количество исследований Монте-Карло говорит о том, что популярные методы могут плохо работать на малых выборках.
Похоже здесь открывается широкая перспектива для асимптотических уточнений с помощью бутстрэпа, пока еще мало освоенная.
В некоторых случаях можно добавить точности существующим методам, например, это касается дикого бутстрэпа в моделях с аддитивными гетероскедастичными ошибками.
В других случаях можно использовать методы, которые пока еще не так часто используются.
Например, можно реализовать тесты на спецификацию моделей с хорошими свойствами в малых выборках с помощью бутстрэпа на этапе вспомогательной регрессии.

Существует два препятствия для использования бутстрэпа.
Во-первых, бутстрэп не всегда реализован в статистических пакетах по умолчанию.
Со временем эта ситуация изменится, и сейчас написание кода для бутстрэпа не является слишком сложным, если программа позволяет писать циклы и сохранять результаты оценивания регрессий.
Во-вторых, есть определенные тонкости.
Асимптотические уточнения возможны только для статистик не зависящих асимптотически от неизвестных параметров.
Простые виды бутстрэпа предполагают независимые и одинаково распределенные данные гладкость оценок и статистик. 
Этого достаточно для большинства, но не для всех приложений.


\section{Библиографические примечания}

Впервые бутстрэп был предложен Эфроном (1979) для независимо и одинаково распределенных данных. Ранняя теория бутстрэпа была представлена в работах Сингха (1981) и Бикеля и Фридман (1981). Хорошие статистическое введение можно найти в работе Эфрона и Тибшарани (1993), более углубленное --- в работе Холла (1992). Случай регрессии также был рассмотрен довольно рано, например,  у Фридмана (1984). Большинство  работ эконометристов появилось в последние 10 лет. Обширный обзор был составлен Хоровицом (2001), его хорошо дополняют работа Браунстоуна и Казими (1998), рассматривающая много эконометрических приложений, а также статья МакКиннона (2002).


\subsection{Упражнения}

\begin{enumerate}
\item [$11 - 1$] Рассмотрим регрессионную модель $y=\alpha+\beta x+\e$, где $\alpha, \beta$, и $x$ скаляры и $\e$ распределено нормально, $\e \sim \cN[0,\sigma^2]$. Сгенерируйте выборку размером $N=20$ c $\alpha=2, \beta=1$, и $\sigma^2=1$, и $x \sim \cN[2,2]$. Мы хотим протестировать $H_0:\beta=1$ против $H_a:\beta \neq 1$ на уровне значимости 0.05, используя $t$-статистику, $t=(\hat{\beta}-1)/se[\hat{\beta}]$. Сделайте столько заданий, сколько позволяет ваше программное обеспечение. Количество репликаций бутстрэпа $B=499$.
\begin{enumerate}
\item Оцените модель методом МНК и рассчитайте $\hat{\beta}$.
\item Рассчитайте стандартную ошибку методом парного бутстрэпа и сравните полученное значение с первоначальным. Протестируйте гипотезу $H_0$ с использованием вычисленной стандартной ошибки.\item  Протестируйте гипотезу $H_0$, используя парный бутстрэп с асимптотическими уточнениями.
\item Рассчитайте стандартную ошибку методом бутстрэпа остатков и сравните полученное значение с первоначальным. Протестируйте гипотезу $H_0$ с использованием вычисленной стандартной ошибки.
\item Протестируйте гипотезу $H_0$, используя  бутстрэп остатков с асимптотическими уточнениями.
\end{enumerate}

\item [$11 - 2$] Сгенерируйте выборку размера 20 в соответствии со следующим процессом порождающим данные. Имеются два регрессора $x_1\sim \chi^2(4)-4$ и $x_2 \sim 3.5+\mathcal{U}[1,2]$; ошибки рассчитываются из смешения двух нормальных распределений $u\sim \cN[0,25]$ с вероятностью 0.3 и $u \sim \cN[0,5]$ с вероятностью 0.7; и зависимая переменная $y$ задана уравнением $y=1.3x_1+0.7x_2+0.5u$.
\begin{enumerate}
\item Оцените модель $y=\beta_0+\beta_1{x_1}+\beta_2{x_2}+u$ с помощью МНК.
\item Предположим, что на основе данных нам необходимо оценить $\gamma=\beta_1+\beta^2$. Для оценки параметра используйте метод наименьших квадратов. Используйте дельта-метод для того, чтобы получить приближенную оценку стандартной ошибки данной функции.
\item Оцените стандартную ошибку $\hat{\gamma}$ с помощью парного бутстрэпа. Сравните полученное значение с $se[\hat{\gamma}]$, полученной в пункте (b) и объясните разницу. Для бутстрэпа используйте $B=25$ и $B=200$. 
\item Протестируйте гипотезу $H_0: \gamma=1.0$ на уровне значимости 0.05 используя парный бутстрэп с $B=200$. Проведите тесты бутстрэпа с асимптотическими уточнениями и без.
\end{enumerate}

\item [$11 - 3$] Используя 200 наблюдений из Раздела 4.6.4 по данным логарифма уровня затрат на здоровье $(y)$ и натурального логарифма общих затрат $(x)$, оцените модель $y=\alpha+\beta x +u$ с помощью МНК. Используйте парный бутстрэп, $B=999$.
\begin{enumerate}
\item Получите оценку стандартной ошибки параметра $\hat{\beta}$ методом бутстрэпа.
\item Используя оценку стандартной ошибки, протестируйте гипотезу $H_0:\beta=1$ при альтернативой гипотезе $H_a:\beta \neq 1$.
\item Проведите  бутстрэп тест гипотезы $H_0:\beta=1$ с уточнениями  против альтернативной $H_a:\beta \neq 1$, предполагая что остатки $u$ гомоскедастичны.
\item Какие результаты будут получены, если в пункте (c) предположить, что остатки $u$ гетероскедастичны? Является ли тест асимптотически верным? Если да, то возможно ли получить асимптотические уточнения?
\item Примените бутстрэп для получения оценки параметра $\beta$, скорректированной на смещение.
\end{enumerate}
\end{enumerate}



\chapter {Методы симуляционного моделирования}


\section{Введение}
Оценивание параметров нелинейными методами, рассмотренными ранее, не предполагало  наличия решений в явном виде. Тем не менее, теория сильно опирается на аналитическое представление результатов. В частности, предполагалось, что целевая функция может быть задана аналитически и асимптотическое распределение формировалось на основе линеаризации оцениваемых уравнений. 

В этой главе описаны методы, основанные на симуляционном моделировании. При расчете ММП-оценки, которая была изложена в Главе 5, предполагалось, что плотность $f(y|x,\theta)$ может быть записана аналитически. При отсутствии аналитического выражения для плотности, возможно использовать ММП-оценку при наличии хорошей аппроксимации $\hat{f}(y|x,\theta)$ для $f(y|x,\theta)$, при построении функции правдоподобия.
Главная причина отсутствия аналитического выражения для плотности заключается в наличии трудности невыразимого аналитически  математического ожидания  в определении  $f(y|x,\theta)$. Например, в модели со случайными коэффициентами могут возникнуть трудности при интегрировании по случайным параметрам. Если значение математического ожидания заменить аппроксимирующим значением, полученным методом Монте-Карло, тогда результирующая оценка будет называться оценкой, рассчитанной при помощи симуляционного моделирования. Аналогичный симуляционный подход может быть применен к оценке методом моментов в основе которого лежит оценка момента условного среднего, для которого не существует решения в аналитической форме. При использовании метода моментов возможно получить состоятельные оценки параметров с меньшим количеством симуляций, чем в случае с ММП-оценками. 

Имитационные методы оценки требуют больших объемов вычислений, поскольку в них применяется метод Монте-Карло для построения выборки. При использовании этих методов ставятся вопросы о точности приближенных значений, эффективности расчетов и свойствах выборочных оценок, рассчитанных на основе приближенных значений.

В Разделе 12.2 представлены примеры на которых показывается целесообразность применения методов имитационного моделирования. В Разделе 12.3 даны основы расчета интегралов, это необходимо поскольку математическое ожидание случайной величины задается интегралом. В Разделах 12.4 и 12.5 рассмотрены симуляционный метод максимального правдоподобия и симуляционный метод моментов; В Разделе 12.6 представлены следствия применениях этих методов. Для симуляционного моделирования важны основные техники (раздел 12.7) и псевдо-случайные числа (раздел 12.8).

\section{Примеры}

Рассмотрим примеры, где условная плотность $y$ при заданных значениях $x$ и вектора параметров $\theta$ рассчитывается при помощи следующего интеграла

\begin{equation}
f(y|x,\theta)=\int{h(y|x,\theta,u)g(u)du},
\end{equation}
где функциональная форма $h(\cdot)$ и $g(\cdot)$ известна и $u$ обозначает случайную величину, не обязательно ошибки модели, по которой производится интегрирование. Если для интеграла нет аналитического решения, а следовательно отсутствует аналитическое выражение для функции правдоподобия, следует использовать методы симуляционного моделирования.

\subsection{Модель со случайными коэффициентами}

В модели \textbf{со случайными параметрами} или модели \textbf{со случайными коэффициентами} возможно изменение значений коэффициентов регрессии между отдельными наблюдениями в соответствии с некоторым случайным распределением. Полностью параметрическая модель со случайными коэффициентами определяет переменную $y_i$ как зависимую от $x_i$ при заданных параметрах $\gamma_i$, функция распределения зависимой переменной представлена как $f(y_i|x_i,\gamma_i)$, где $\gamma_i$ независимы и одинаково распределены  с плотностью $g(\gamma_i|\theta)$. Выводы построены на условной плотности распределения $y_i$ при заданных $x_i$ и  $\theta$,

\begin{equation}
f(y|x,\theta)=\int{f(y|x,\gamma)g(\gamma|\theta)d\gamma}.
\end{equation}

Данный интеграл не имеет аналитического решения за исключением некоторых случаев. Чаще всего предполагается, что случайные параметры нормально распределены, т.е. $\gamma_{i}{\sim }\mathcal{N}[\mu,\sum]$.  Тогда $\gamma_{i}=\mu+\Sigma^{-1/2}u_{i}$, где $u_{i}{\sim }\mathcal{N}[0,I]$ и значит выражение (12.2) можно записать в форме (12.1), где $\theta$ вектор, состоящий из $\mu$ и отдельных компонент $\sum$, а $g(u)$ имеет $\mathcal{N}(0,I)$ плотность.

Простой пример модели случайных параметров --- скрытая гетерогенность. При скрытой гетерогенности  один из  заданных параметров, как правило свободный член, считается случайным. Рассмотренный интеграл является одномерным и легко аппроксимируется численно. В общем случае размерность интеграла может быть выше. 

Среди примеров моделей со случайными коэффициентами и скрытой гетерогенности можно обозначить: (1) нормально распределенные случайные параметры в мультиномиальной логит-модели (логит-модель со случайными параметрами ; см. Главу 15); (2) ненаблюдаемая гетерогенность, имеющая гамма-распределение в моделях дюрации Вейбулла (см. Главу 19), (3) ненаблюдаемая гетерогенность, имеющая гамма-распределение в пуассоновской модели со счетными  данными (см. Главу 20); и (4) индивидуальные случайные эффекты в моделях с панельными данными (см. Главу 21). Явные выражения для частной функции плотности после интегрирования по гетерогенной составляющей можно найти в примере 3 и при нормальной линейной модели в примере 4. В примерах 1 и 2, а также в нелинейных моделях примера 4 явные решения получить невозможно.


\subsection{Модели с ограниченной зависимой переменной}

Ограниченная зависимая переменная (limited dependent variable) --- это зависимая переменная, у которой мы наблюдаем только часть значений из-за цензурирования или усечения. В таком случае плотность наблюдаемой переменной --- интеграл, который может не иметь явного выражения.

Основной класс моделей с ограниченной зависимой переменной --- модели дискретного выбора, которые подробны описаны в главах 14 и 15. Здесь мы вводим модели дискретного выбора, так как они являются предметом исследования литературы по эконометрике, которая посвящена оцениванию с помощью симуляций.

В качестве примера рассмотрим выбор потребителя из трёх взаимоисключающих альтернатив, например, из трёх товаров длительного пользования, только один из которых будет выбран индивидом.  Предположим, что потребитель максимизирует полезность, и пусть полезность каждой из альтернатив 1, 2 и 3 задана $U_1$, $U_2$ и $U_3$ соответственно. Эти полезности являются ненаблюдаемыми величинами. Вместо этого мы можем только наблюдать дискретную переменную исхода $y = 1, 2$ или 3 в зависимости от того, какая альтернатива выбрана.

Предположим, что выбрана альтернатива 1, так как она обладает наибольшей полезностью. Тогда функция вероятности задана следующим образом $p_1 = \Pr[y=1]$, где

\[
p_1=\Pr[U_1-U_2 \geq  0, U_1-U_3 \geq  0]
\]

\[
=\Pr[(x_1-x_2)' \beta+\e_1-\e_2 \geq  0, (x_1-x_2)' \beta+\e_1-\e_3 \geq  0],
\]
если мы сделаем стандартное предположение (см. раздел 15.5.1) о том, что $U_j = x_j'\beta + \e_j$, $j = 1, 2, 3$, где регрессор $x$ измеряет различные качества трех товаров, а ошибка $\e$ может быть в промежутке $(-\infty, \infty)$. Определяя $u_1 = U_1 --- U_2$ и $u_2= U_1 --- U_3$,  мы получаем
 
\begin{equation}
p_1=\int^{\infty}_{0}\int^{\infty}_{0}g(u_1,u_2)du_1 du_2,
\end{equation}
где $g(u_1, u_2)$, более формально $g(u_1, u_2| x, \theta)$ --- двумерная плотность для вектора $(u_1, u_2)$ или эквивалентно
\begin{equation}
p_1=\int^{\infty}_{-\infty}\int^{\infty}_{-\infty}{\bf{1}}[u_1 \geq  0,u_2 \geq  0]g(u_1,u_2)du_1 du_2,
\end{equation}
где ${\bf{1}}[A]$ --- функция-индикатор, который равен 1, если событие $A$ происходит, и равно 0 в противном случае.

Интеграл из (12.4) можно представить в виде (12.1). Так как границы интеграла покрывают только часть значений $(u_1, u_2)$ (см. (12.3)), решение в явном виде может не существовать, несмотря на то что мы знаем, что $\int \int g(u_1, u_2)du_1du_2 = 1$, когда интегрирование проводится по всем значениям $(u_1, u_2)$.

В частности, если ошибки $\e$ имеют нормальное распределение, как в мультиномиальной пробит-модели, интеграл (12.3) считается по положительному ортанту двумерного нормального распределения. 
Не существует решения для $p$ в неявном виде, и поэтому не существует простого выражения для плотности $f(y|x,\theta)$. На практике размерность интеграла может быть очень большой, что усложняет нахождение численной аппроксимации, так при выборе  среди $m$  взаимоисключающих альтернатив размерность интеграла равна $m-1$. До того, как было развито оценивание с помощью симуляций, исследователи либо использовали модели с $m \leq 4$, либо выбирали другое распределение ошибок такое, чтобы оно приводило к более ограниченной мультиномиальной логит-модели.

\subsection{Оценка максимального правдоподобия}

Для простоты рассмотрим метод максимального правдоподобия. Предположим, что наблюдения независимы и условная плотность $y$ имеет вид $f(y|x,\theta)$.

Усложнение двух предыдущих примеров состоит в неприменимости оценки максимального правдоподобия на практике, поскольку нет аналитического решения для функции плотности $f(y|x,\theta)$, заданной сложным интегралом. Вместо этого, заменим интеграл, используя численное приближение $\hat{f}(y|x,\theta)$ и максимизируем следующее выражение по $\theta$.

\[
\ln\hat{L}_N(\theta)=\sum^{N}_{i=1}\ln\hat{f}(y_i|x_i,\theta)
\]

Оценка будет состоятельной и иметь такое же асимптотическое распределение как и оценка максимального правдоподобия, если $\hat{f}(y|x,\theta)$ является хорошей аппроксимацией для $f(y|x,\theta)$.

Полученные условия первого порядка как правило заданы нелинейно и решаются итерационными методами. Поскольку $\hat{f}(y_{i}|x_{i},\theta)$ меняется по $i$ и $\theta$, оценка градиента с использованием численных производных потребует $Nqr$ оценок, где $N$ размер выборки, $q$ размерность вектора $\theta$ и $r$ число итераций. К примеру, 1000 наблюдений, 10 параметров и 50 итераций дают 500 000 оценок функции. 

Стандартное время вычисления для нелинейных моделей теперь умножается на количество оцениваний, необходимых для расчета приближенного значения интеграла $f(y|x,\theta)$. Очевидно, что желательно методы с малым количеством вычислений. 

\subsection{Байесовские методы}

Байесовские методы рассмотрены отдельно в Главе 13. Отметим, что в байесовских методах рассчитывается интеграл, по форме аналогичный 12.2. Но при этом, в отличие от метода максимального правдоподобия, определяется (апостериорное) распределение параметров, а не точечная оценка.

\section{Основы расчета интегралов}

Рассмотрим интеграл

\begin{equation}
I=\int^{b}_{a}f(x)dx,
\end{equation}
где функция $f(\cdot)$ непрерывна на отрезке $[a,b]$ и, возможно, что границы интеграла могут быть разширены до плюс и минус бесконечности, соответственно, т.е. $a=-\infty$ и/или $b=\infty$. В этом разделе изначально считается, что $x$ --- скаляр и обозначает переменную интегрирования. В задачах регрессии обычно используется интегрирование по $u$, а $x$ обозначает регрессоры (см. 12.1). Предположим, что интеграл существует, условие существование интеграла необходимо проверять, поскольку методы аппроксимации дают конечную оценку даже расходящемуся интегралу.

Для начала рассмотрим численное интегрирование, как правило применяемое к интегралам малой размерности. Для интегралов большой размерности аналогом численного интегрирования выступает метод Монте-Карло.

Материал данного раздела описывает реализацию методов симуляционного моделирования; в связи с чем, некоторые читатели вначале могут предпочесть ознакомиться с разделами 12.4-12.6.

\subsection{Численное интегрирование детерминистическим методом}

Интеграл можно рассматривать как меру площади или объема. Численное интегрирование детерминистическим методом или просто интегрирование предполагает расчет площади фигуры в три шага: деление изначально заданной фигуры на несколько частей, расчет площади отдельных фигур и далее суммирование площадей. Иначе говоря, производится оценка подинтегральной функции в нескольких точках и рассчитывается средневзвешенная сумма полученных значений. Префикс <<детерминистическое>> используется для обозначения отсутствия симуляций при использовании приближенного метода расчета интегралов.

\begin{center}
Правило Симпсона
\end{center}

По определению интеграла 

\begin{equation}
I=\lim_{\Delta{x_i} \rightarrow 0}\sum^{n}_{j=1}f(x_j)\Delta{x_j},
\end{equation}
где промежуток $[a,b]$ разбит на $(n+1)$ промежутков, $x_{0}<x_{1}<\ldots <x_{n}$ и $n \rightarrow \infty$. Стандартные методы аппроксимации являются уточнениями выражения  12.6 и, соответственно, дают более точные результаты для конечного $n$. Ниже представлены выводы для равноотдаленных точек. В общем случае интегрирование детерминистическим методом может применяться и для неравноотдаленных точек. Для простоты допустим, что значения $f(x)$ могут быть рассчитаны в конечных точках $a$ и $b$. 

Согласно правилу прямоугольников сначала вычисляются значения в средней точке $\overline{x}_{j}=\dfrac{1}{2}(x_{j-1}+x_j)$ интервала $x_{j-1},x_j$ и затем суммируются $n$ прямоугольников с основанием $(b-a)/n$ и высотой $f(\overline{x}_j)$. Таким образом, приближенное значение интеграла равно 

\begin{equation}
\hat{I}_M=\sum^{n}_{j=1}\dfrac{b-a}{n}f(\overline{x}_j).
\end{equation}

В свою очередь, правило трапеций дает более точное приближение. Чертится прямая линия от $f(x_{j-1})$ до $f(x_j)$ и затем суммируются $n$ трапеций с основанием $(b-a)/n$  и средней высотой $(f(x_{j-1})+f(x_j))/2$. В этом случае, приближенное значение $I$ будет равно 

\begin{equation}
\hat{I}_T=\sum^{n}_{j=1}\dfrac{b-a}{n}\dfrac{f(x_{j-1})+f(x_j)}{2}.
\end{equation}

Правило парабол (правило Симпсона) предполагает использование парабол проходящих через $f(x_{j-1}), f(x_j)$ и $f(x_{j+1})$, в то время как в правиле трапеций строится линия, соединяющая идущие друг за другом значения функции. 

Правило Симпсона дает следующие результаты:

\begin{equation}
\hat{I}_S=\sum^{n}_{j=0}\dfrac{(b-a)}{3n}w_{j}f(x_j),
\end{equation}
где $n$ четное число, $w_j=4$, если $j$ четно и $w_j=2$, если $n$ нечетно, кроме исключений $w_0=w_n=1$. Дальнейшие обобщения используют полином степени $p$ по $(p+1)$ следующим друг за другом точкам.

Границы ошибок интегрирования растут по экспоненциальному закону с расширением диапазона интегрирования,$b-a$, и экспоненциально падают при сокращении количества интервалов. Для правила Симпсона, $|I_S-I|\leq M_{4}(b-a)^{5}/180n^4$, где $M_4$ максимум абсолютного значения четвертой производной по $x$ на отрезке $[a,b]$. Для правила трапеции справедливо неравенство $|I_T-I|\leq M_{2}(b-a)^{3}/12n^2$, где $M_2$ максимальное абсолютное значение второй производной по $x$ на отрезке $[a,b]$. Очевидно, что количество интервалов должно увеличиваться с ростом количества значений $x$, в связи с чем необходимо определить чувствительность диапазона значений к изменению количества интервалов.

Правило Симпсона и связанные с ним правила хорошо применимы к интегралам, определенным на отрезке, трудности могут возникнуть, если интеграл будет неопределенными интегралами из-за отсутствия граничных значений. Например, предположим что $[a,b]=[0,\infty)$. Тогда при выборе $x_n$ необходимо будет принять компромиссное решение. С одной стороны, верхнее значение границы $x_n$ должно быть большим, с другой стороны это приведет к большим расстояниям между точками. Также полезно определить чувствительность ответа к увеличению $x_n$.

\begin{center}
Интегрирование по Гауссу
\end{center}

Интегрирование по Гауссу является численным методом и было предложено Гауссом в 1814 году. Данные метод предлагает правило выбора точек оценки $x_j$, которые теперь неравномерно расположены. Метод особенно полезен для оценки неопределенных интегралов. 

Для начала преобразуем выражение 12.5 

\begin{equation}
I=\int^{d}_{c}w(x)r(x)dx,
\end{equation}

где как правило $w(x)$ это одна из следующих функций от $x$. Метод Гаусса-Эрмита полагает  $w(x)=e^{-x^2}$, и используется для интервала вида $[c,d]=(-\infty,\infty)$. Метод Гаусса-Лежандра полагает  $w(x)=e^{-x}$ для случая $[c,d]=(0,\infty)$. Метод Гаусса-Лежандра полагает $w(x)=1$ для случая $[c,d]=[-1,1]$.

В самом простом случае выражение 12.10 можно получить из 12.5, определив $r(x)=f(x)/w(x)$. В общем случае, может потребоваться преобразование $x$, чтобы, к примеру, отрезок $[2,\infty)$ в 12.5 преобразуется в отрезок $[0,\infty)$ в выражении 12.10. Некоторые процедуры позволяют исследователю выбрать $f(x)$ и промежуток интегрирования, а все необходимые преобразования производятся автоматически.

С помощью интегрирования по Гаусу рассчитывается приближенное значение интеграла 12.10, с помощью взвешенной суммы

\begin{equation}
\hat{I}_{G}=\sum^{m}_{j=1}w_{j}r(x_j),
\end{equation}
где исследователь выбирает $m$. Формулы для $m$ точек оценки $x_j$ и весов $w_j$ можно найти в литературе, например у  Абрамовица и Стегуна (1971) или получить при помощи компьютерной программы, например см. Пресс и др. (1993).

В теории, согласно которой рассчитываются приближенные значения, за основу берутся ортогональные многочлены $p_j(x), j=0,\ldots ,m$ с весами $w(x)$, удовлетворяющие следующему равенству:

\[
I=\int^{d}_{c}w(x)p_{j}(x)p_{k}(x)dx=0, j{\neq}k, j,k=0,\ldots ,m.
\] 
Если к тому же $\int^{d}_{c}w(x)p^{2}_{j}(x)dx=1$ полиномы называются ортонормальными. Приближение (12.11) является точным, если порядок полинома $r(x)$ меньше или равен $2m-1$, иными словами наилучшее приближение достигается, если значения $r(x)$ в (12.10) хорошо аппроксимируется с помощью полинома порядка $2m-1$. Хороший выбор количества точек оценки $m$ требует экспериментирования, однако во многих приложениях $m$ не более, чем 20 или 30.

В качестве примера рассмотрим интегрирование Гаусса-Эрмита, частый способ интегрирования в эконометрике, где границы интегрирования часто равны $(-\infty, \infty)$. Для $w(x)=e^{{-x}^2}$ ортогональные полиномы $p_{j}(x)$ являются полиномами Эрмита $H_{j}(x)$ и ортонормальная форма $H_{j}(x)$ задается рекурсивно $H_{j+1}(x)=\sqrt{2/(j+1)}xH_{j}(x)-\sqrt{j/(j+1)}H_{j-1}(x)$, $j=1,\ldots ,m$, где $H_{-1}=0$ и $H_0=\pi^{-1/4}$. При этом $m$ значений абсцисс $x_j$ являются $m$ корнями уравнений $H_{m}=0$ и для ортонормальных полиномов Эрмита веса равны $w_j=1/[jH_{j-1}(x_j)^2]$. Как отмечалось ранее $x_j$ и $w_j$ для заданного значения $m$ даны в таблицах или их можно рассчитать при помощи статистического пакета.

Для определенных интегралов применение интегрирования Гаусса-Лежандра, как правило дает результаты лучше, чем правило Симпсона. Тем не менее, главное преимущество интегрирования по Гауссу заключается в возможности применения данного метода для вычисления неопределенных интегралов. Заметим, что если границы интегрирования равны $(-\infty, \infty)$, то иногда возможно преобразование переменных, приводящее границы интегрирования к $(0,\infty)$ и, тогда вместо метода Гаусса-Эрмита будет применяться метод Гаусса-Лежандра.

Существует большое количество других детерминистических методов оценки интегралов, в т.ч. аппроксимация Лапласа (Тьерни, Касс и Кадэйн, 1989).

\subsection{Интегрирование методом Монте-Карло} 

Интегрирование с помощью метода Монте-Карло является альтернативой детерминистическому численному интегрированию. В общем случае оценка интеграла $I=\int^{b}_{a}f(x)dx$ имеет вид

\begin{equation}
\hat{I}_{MC}=\sum^{S}_{s=1}f(x^s),
\end{equation}
где $x^1,\ldots ,x^s$ равномерно распределенные величины на отрезке $[a,b]$. По сравнению с правилом прямоугольников, значение функции $f(x)$ рассчитывается в $S$ точках, выбранных случайным, а не детерминистическим  образом. 

Наше изложение метода уделяет особое внимание регрессионным задачам, таким как были рассмотрены в Разделе 12.2. Интегрирование применяется для расчета, например, математического ожидания $\E[h(x)]$, где случайная величина $x$ имеет функцию плотности распределения $g(x)$. При непрерывности функции оценивается интеграл 

\begin{equation}
\E[h(x)]=\int^{b}_{a}h(x)g(x)dx,
\end{equation}

На протяжении всей главы предполагается, что $\E[h(x)]<\infty$, т.е. интеграл сходится. $\E[h(x)]$ может быть оценена при помощи простого интегрирования по методу Монте-Карло

\begin{equation}
\hat{I}_{DMC}=\hat{\E}[h(x)]=S^{-1}\sum^{S}_{s=1}h(x^s),
\end{equation}
где ${x^{s}, s=1,\ldots ,S}$ выборка Монте-Карло, состоящая из $S$ псевдо-случайных  значений с распределением $g(x)$. Данную выборку можно построить с помощью методов, изложенных в разделе 12.8. Оценка 12.14 рассчитывает $h(x)$, используя  значения $x$ сгенерированные согласно плотности $g(x)$. Оценка 12.12 рассчитывает $h(x)g(x)$, используя равномерно распределенные $x$. Преимущество (12.14) состоит в возможности использования (12.14) для расчета неопределенных интегралов, в то время как равномерное распределение в (12.12) может вызвать затруднения, если значения $a$ и $b$ не ограничены.


Оценка $\hat{\E}[h(x)]$ --- это среднее значений функции $f()$ в каждой из случайных точек $x^s$. Другими словами, $\hat{\E}[h(x)]$ --- это среднее арифметическое случайных величин $h(x^s)$, и его свойства при $S\to\infty$ могут быть получены с помощью закона больших чисел или центральной предельное теоремы. Здесь $x^s$ независимы и одинаково распределены, поэтому $h(x^s)$ независимы и одинаково распределены и мы можем применить закон больших чисел Колмогорова (приложение А, теорема А.8) поскольку конечность $\E[h(s)]$ уже была предположена. Следовательно,

\[
\hat{\E}[h(x)]\stackrel{p}{\rightarrow}\E[h(x)] \text{при} S \rightarrow \infty
\]

Вместе с тем, поскольку $h(x^s)$ независимо и равномерно распределены дисперсия $\hat{\E}[h(x)]$ равна $S^{-1}\V[h(x)]$, если существует $\V[h(x)]$. Вероятно, что приближенное значение будет близко к истинному для небольших значений $S$, если $S^{-1}V[h(x^s)]$ мало. 

\subsection{Пример расчета интеграла}

Предположим, что $x{\sim } \mathcal{N}[0,1]$. Рассчитаем математическое ожидание $x$ 

\[
\E[x]=(\sqrt{2\pi})^{-1}\int^{\infty}_{-\infty}x\exp(-x^{2}/2)dx
\]

а также момент $\E[\exp(-\exp(x))]$, который задан интегралом 

\[
\E[\exp(-\exp(x))]=(\sqrt{2\pi})^{-1}\int^{\infty}_{-\infty}\exp(-\exp(x))\exp(-x^{2}/2)dx.
\]

Для $\E[x]$ существует аналитическое выражение и $\E[x]=0$. В отличие от этого для $\E[\exp(-\exp(x))]$ отсутствует аналитическое решение. Перед тем как искать численную аппроксимацию, необходимо убедиться, что интеграл действительно сходится. 

Поскольку выражение $\exp(-\exp(x))$ строго положительно и монотонно убывает при максимальном значении 1, следовательно $|\exp(-\exp(x))|<1$, поэтому, $\E[\exp(-\exp(x))]$ $<\E[1]=1$ и интеграл сходится.

Одномерные интегралы просто рассчитать, используя детерминистскую численную аппроксимацию. Например, рассмотрим применение правила прямоугольников при $n=20$, где точки равномерно распределены между $x_0=-5$ и $x_{20}=5$. Тогда

\[
\hat{\E}[x]=(\sqrt{2\pi})^{-1}\sum^{20}_{j=1}\dfrac{10}{20}\overline{x}_{i}\exp(-\overline{x}^{2}_{j}/2),
\]

\[
\hat{\E}[\exp(-\exp(x))]=(\sqrt{2\pi})^{-1}\sum^{20}_{j=1}\dfrac{10}{20}\exp(-\exp(\overline{x}_j))\exp(-\overline{x}^{2}_{j}/2),
\]
где $\overline{x}_j=-5.25+j/2$. Как и ожидалось, $\hat{\E}[x]=0$ до большого количества десятичных знаков после запятой, в то время как $\hat{\E}[\exp(-\exp(x))]=0.38175656$. Последняя оценка меняется незначительно, после восьмого знака, если взять $n=200$ и  равноудаленные точки между -10 и 10. Очевидно, что в этом случае детерминистическая численная аппроксимация дает адекватные результаты.

Эти интегралы возможно рассчитать, используя метод Монте-Карло со следующими параметрами

\[
\hat{\E}=\dfrac{1}{S}\sum^{S}_{s=1}x^{s},
\]

\[
\hat{\E}[\exp(-\exp(x))]=\dfrac{1}{S}\sum^{S}_{s=1}\exp(-\exp(x^s)),
\]
где $x^s$ это $s$-ая точка из $S$ распределенных нормально с параметрами $\mathcal{N}[0,1]$. Способ получения таких точек рассмотрен в Приложении B. В таблице 12.1 даны значения оценок $\hat{\E}[x]$ и $\hat{\E}[\exp(-\exp(x))]$ для разного количества симуляций $S$. Можно увидеть, что при $S \rightarrow \infty$ оценки более стабильны и их значения близки к истинным, 0 и 0.38175656, где последняя получена с помощью детерминистической численной аппроксимации. Однако, даже при $S=10^6$ оценка $\hat{\E}[x]$ будет отлична от нуля до четвертого знака после запятой. В таком случае, $\V[\hat{\E}[x]]=S^{-1}\V[x^s]=1/S$, поскольку $\V[x^s]=1$ так, что даже если $S=10^6$ стандартное отклонение $\hat{\E}[x]$ довольно велико (0.001). Альтернативные методы дающие оценку Монте-Карло с меньшей дисперсией  рассмотрены в разделе 12.7.
 
\begin{table}[h]
\begin{center}
\caption{\label{tab:mcintegral} Интегрирование Монте-Карло: пример для стандартного нормального $x$}
\begin{tabular}{lll}
\hline
\hline
$S$ = Количество симуляций & $\hat{\E}[x]$ & $\hat{\E}[\exp(-\exp(x))]$ \\ 
\hline 
10 & 0.145 & 0.336 \\ 
25 & -0.209 & 0.435 \\ 
50 & 0.050 & 0.369 \\ 
100 & -0.120 & 0.409 \\ 
500 & -0.059 & 0.398 \\ 
1,000 & 0.005 & 0.382 \\ 
10,000 & -0.007 & 0.383 \\ 
100,000 & -0.000 & 0.382 \\ 
1,000,000 & -0.000 & 0.381 \\ 
\hline 
\end{tabular}
\end{center}
\end{table} 

\subsection{Интегралы высокой размерности}

Интегралы высокой размерности можно оценить детерминистическими методами или методом Монте-Карло, последний позволяет становится более предпочтительным с ростом размерности.

Наилучшие результаты интегрирования детерминистическим методом получаются при интегрировании с помощью многомерного метода Гаусса или обычного метода Гаусса, если границы интегрирования не слишком сложные, поскольку тогда возможно понизить размерность интеграла и свести решение к оценке $m$ одномерных интегралов. Однако, из определения интеграла следует, что количество вычислений вырастет экспоненциально с ростом $m$. К примеру, если необходимо 20 оцениваний функции для одномерного интеграла, тогда для расчета интегралов пятой размерности может потребоваться $5^{20}$ или 95 биллионов оцениваний. Такая высокая точность может не понадобится, если похожие вычисления  с последующим суммированием производятся для каждой точки. Тем не менее, даже в таком случае количество расчетов может значительно увеличится с ростом размерности интеграла.

Оценка интегралов высокой размерности методом Монте-Карло не вызывает затруднений: необходимо задать $x$ в 12.13 и 12.14 как вектор и генерировать случайную выборку согласно многомерной плотности распределения $g(x)$. Казалось бы отсутствует негативное влияние размерности. 
Тем не менее, необходимо учитывать, что упрощенный метод Монте-Карло может не сработать, если подинтегральная функция имеет острые вершины, очень возможно, что острые вершины будут иметь место для интегралов высокой размерности. 
В частности, в примере дискретного выбора в разделе 12.2.2 подинтегральное выражение в (12.4) принимает ненулевые значения  только на небольшом диапазоне $(u,\nu)$, эти проблемы рассмотрены в Разделе 12.7. Кроме того, строить выборку из многомерного распределения сложнее, чем из одномерного. 

\section{Симуляционный метод максимального правдоподобия}

Рассмотрим способы интегрирования, которые применимы для расчета оценок максимального правдоподобия при отсутствии аналитической формы записи плотности. Основной результат симуляционных методов моделирования --- оценки параметров могут иметь такое же распределение, как оценки, полученные обычным методом максимального правдоподобия, при условии, что количество симуляций, проводимых при оценивании функции плотности для каждого наблюдения, стремится к бесконечности.

\subsection{Способы симуляций}

Предположим, что условная плотность распределения $f(y|x,\theta)$, включает интеграл, значение которого трудно посчитать. Например, предположим, как и в (12.1),

\begin{equation}
f(y_i|x_i,\theta)=\int{h(y_i|x_i,\theta,u_i)g(u_i)du_i}
\end{equation}

Прямым способом симулирования условной плотности $f(y_i|x_i,\theta)$ выступает метод Монте-Карло, согласно которому:

\begin{equation}
\hat{f}(y_i|x_i,u_{is},\theta)=\dfrac{1}{S}\sum^{S}_{s=1}h(y_i|x_i, \theta, u^{s}_i),
\end{equation}
где $u_{is}$ вектор состоящий из $S$ независимых значений $u^{s}_{i}=1,\ldots ,S$, сгенерированных согласно плотности $g(u_i)$. Это по сути просто среднее арифметическое $h(y_i|x_i, \theta, u^{s}_{i})$ по $S$ значениям. Согласно выводам раздела 12.3.2 следует, что $\hat{f}_i$ несмещенная и состоятельная оценка для $f_i$ при $S \rightarrow \infty$.

Вместе с тем, возможно использовать иные методы симулирования, которые подробно рассмотрены в разделе 12.7. С помощью непрямых методов возможно получить оценку функции $\hat{f_i}$, лучше аппроксимирующую значение функции $f_i$, чем прямые методы, при конечном количестве симуляций. Например, допуская наличие корреляции между симуляциями, при условии, что частное распределение симуляций по-прежнему задано функцией плотности $g(u_i)$. В общем случае выражение для оценки условно заданной функции $f(y_i|x_i,\theta)$ методом Монте-Карло имеет следующий вид:

\begin{equation}
\hat{f}(y_i|x_i,u_{is},\theta)=\dfrac{1}{S}\sum^{S}_{s=1}\tilde{f}(y_i|x_i,\theta,u^{s}_i),
\end{equation}
где $u^{s}_i=1,\ldots ,S$ это $S$ случайных значений сгенерированных согласно плотности $g(u_i)$, но необязательно независимых по $s$. Для того, чтобы оценка $\hat{f_i}$ была полезной, необходимо выполнение условия $\hat{f}_i \stackrel{p}{\rightarrow} f_i$ при $S \rightarrow \infty$. Вероятно, что последнее выполняется, если вспомогательная оценка $\tilde{f}(\cdot)$ является несмещенной и справедливо:

\begin{equation}
\E[\tilde{f}(y|x,\theta,u^s)]=f(y|x,\theta).
\end{equation}

Желательно, чтобы оценка $\hat{f_i}$ было дифференцируема по $\theta$, тогда для оценки параметра $\theta$ будет возможно использовать стандартные итерационные градиентные методы. Для устранения <<шума>>, вызванного симуляциями, а также для обеспечения численной сходимости, полученные с помощью Монте-Карло значения, используемые для построения $\hat{f}_i$, не должны меняться при изменении значений $\theta$ между итерациями. 

\subsection{Оценивание с помощью симуляционного метода максимального правдоподобия}

При выполнении условии независимости по $i$, оценки максимального правдоподобия, $\hat{\theta}_{ML}$, максимизируют значение функции $\ln{L_N}(\theta)=\sum^{N}_{i=1}\ln{f}(y_i|x_i,\theta)$. В свою очередь, оценка симуляционного максимального правдоподобия (maximum simulated likelihood), $\hat{\theta}_{MSL}$ максимизирует логарифмическую функцию максимального правдоподобия, построенную на основе оцененных значений плотности, то есть

\begin{equation}
\ln\hat{L}_{N}(\theta)=\sum^{N}_{i=1}\ln\hat{f}(y_i|x_i,u_{is},\theta),
\end{equation}
где вспомогательная оценка $\hat{f}(\cdot)$ задана в (12.7). Если $\hat{f}(\cdot)$ дифференцируема по $\theta$, тогда $\hat{\theta}_{MSL}$ может быть рассчитана стандартными градиентными методами  Главы 10, с использованием как аналитической, так и численной производной.

\subsection{Распределение оценок симуляционного максимального правдоподобия}

Согласно общему методу доказательства состоятельности оценок (см. раздел 5.3.2), оценка симуляционного метода максимиального правдоподобия будет иметь такой же предел по вероятности, как и ММП-оценка, если пределы по вероятности функции $N^{-1}\ln{\hat{L}_N(\theta)}$, аппроксимирующей целевую, и исходной целевой функции $N^{-1}\ln{\hat{L}_N}(\theta)$ равны. Это произойдет, если $\ln{\hat{f}_i}-\ln{f_i} \stackrel{p}{\rightarrow} 0$, для чего достаточно условия $\hat{f}_i-f_i \stackrel{p}{\rightarrow} 0$ при $S \rightarrow \infty$.

Даже если симуляционная ММП-оценка состоятельна, возможно что случайность симуляций  приведет к росту дисперсии по сравнению с ММП-оценкой. В качестве примера формальных условий, при которых оценка симуляционного правподобия полностью эффективна, дадим утверждение, которое является переформулировкой теоремы, которую приводят Гурьеру и Монфорт (1991).

\begin{proposition}
[Распределение оценок симуляционного правдоподобия](Гурьеру и Монфорт, 1991)

Предположим, что
\begin{enumerate}
\item Значения взяты из случайной выборки, а условная плотность распределения $f(y|x,\theta_0)$ удовлетворяет регуляторным условиям так, что оценка максимального правдоподобия состоятельна и асимптотически нормально распределена с предельной матрицей дисперсий $A^{-1}(\theta_0)$, где
\[
A(\theta_0)=-\left. \plim\left[ N^{-1}\sum^{N}_{i=1}\dfrac{{\partial}^{2}\ln f(y_i|x_i,\theta)}{\partial\theta\partial\theta^{'}} \right|_{ \theta_{0}}\right] 
\]

\item Плотность $f$ рассчитывается с помощью вспомогательной оценки $\hat{f}$, приведенной в (12.17) с $\tilde{f}$ несмещенной для $f$.
\end{enumerate}

Тогда оценка симуляционного правдоподобия, определение которой дано в (12.19) асимптотически эквивалентна ММП-оценке если $S,N \rightarrow \infty$ и $\sqrt{N}/S \rightarrow 0$ и имеет предельное нормальное распределение 


\begin{equation}
\sqrt{N}(\hat{\theta}_{MSL}-\theta_0)\stackrel{d}{\rightarrow} \mathcal{N}[0,A^{-1}(\theta_0)].
\end{equation}
\end{proposition}

Оценка симуляционного правдоподобия фактически является состоятельной, если взять более слабое условие, $S,N \rightarrow \infty$. Это условие будет удовлетворено, если, к примеру, $S=N^{0.4}/a$ для некоторой константы $a$. В таком случае, $\sqrt{N}/S=aN^{0.1} \rightarrow \infty$, т.е., по утверждению 12.1 оценка симуляционного правдоподобия не будет эффективной. Используя стандартную процедуру разложения в ряд Тейлора можно увидеть, что предельное распределение $\sqrt{N}(\hat{\theta}_{MSL}-\theta_0)$ с точностью до умножения на матрицу задано выражением $N^{-1/2}\sum_{i}\partial{\ln\hat{f}_i/\partial{\theta}}|_{\theta_0}$. Оно зависит от изменчивости $\partial{\ln{f_i}}/\partial\theta$ и от ошибки симуляций при аппроксимации $\hat{f}_i$. Согласно утверждению 12.1 ошибка симуляций асимптотически исчезает при условии, что количество симуляций растет с увеличением размера выборки со скоростью, превышающей $\sqrt{N}$.

Для расчета ковариационной матрицы симуляционного правдоподобия необходимо получить оценку $A(\theta_0)$. Для этого можно просто применить симуляционный вариант BHHH-метода, рассмотренный в разделе 5.5.2. Поскольку $\partial{\ln{f_i}}/\partial\theta=(\partial{f_i}/\partial\theta)/f_i$, оценка информационной матрицы по BHHH-методу будет равна

\[
\hat{B}=\dfrac{1}{N}\sum^{N}_{i=1}\dfrac{\partial{f_i}(\hat{\theta})/\partial\theta}{f_{i}(\hat{\theta})}\dfrac{\partial{f_i}(\hat{\theta})/\partial\theta^{'}}{f_{i}(\hat{\theta})}.
\]
Поскольку не существует аналитического решения для функции $f_i$ и $\partial{f_i}/\partial\theta$, невозможно вычислить значение данного выражения. Следовательно, заменим $f_i$ на $\hat{f_i}$, заданное в (12.17) и получим симуляционную оценку асимптотической вариации
\begin{equation}
\hat{\V}[\hat{\theta}_{MSL}]=\left( \sum^{N}_{i=1}\left(\dfrac{\sum^{S}_{s=1}\partial\tilde{f}^{s}_{i}(\hat{\theta})/\partial\theta}{\sum^{S}_{s=1}f^{s}_{i}(\hat{\theta})}\dfrac{\sum^{S}_{s=1}\partial\tilde{f}^{s}_{i}(\hat{\theta})/\partial\theta^{'}}{\sum^{S}_{s=1}f^{s}_{i}(\hat{\theta})} \right) \right)^{-1},
\end{equation}
где $\tilde{f}^{s}_{i}(\hat{\theta})=\tilde{f}(y_i|x_i,u^{s}_i,\hat{\theta}_{MSL})$. Альтернативные варианты оценок могут быть получены по аналогии с оценкой с помощью Гессиана или сэндвич-оценкой, определенными в разделе 5.5.2.

С практической точки зрения важна проблема определения количества симуляций. Возможно увеличивать количество симуляций с увеличением размера выборки, но абсолютное значение $S$ остается неопределенным. Если разница в оценке параметров при разном количестве симуляций несуществена, например, при 2400 или 2600, то это можно считать свидетельством того, что 2400 симуляций достаточно. Предположим, что размер выборки увеличился в четыре раза. Тогда, насколько должно измениться количество симуляций? Согласно Утверждению 12.1 количество симуляций $S$ должно увеличиться более чем в два раза, т.е. должно быть больше 4800, тогда соотношение $\sqrt{N}/S$ станет ближе к нулю. Однако нельзя утверждать, что значение $\sqrt{N}/S$ достаточно близко к нулю, так при $S=2,400$ и $N=6,400$ соотношение равно $1/30$. Таким образом, возникают затруднения при ответе на вопрос о достаточном количество симуляций. Многие эмпирические исследователи полагаются на приближенные показатели сходимости точечных оценок, неформально основанных на расчете градиентов $L_N(\theta)$. Формальный подход выбора $S$, основанный на проведении теста, рассмотрен у Хаживассилиу (2000). 

\subsection{Оценка симуляционного правдоподобия, скорректированная на смещение}

Оценка симуляционного правдоподобия является несостоятельной или асимптотически смещенной когда количество симуляций $S<{\infty}$. Смещение существует для конечных значений $S$, поскольку оценка $\ln{\hat{f}_i}$ смещена для $\ln{f_i}$, даже если вспомогательная оценка $\hat{f}_i$ несмещена для $f_i$. Смещение возникает из-за взятия натурального логарифма. Тогда, $N^{-1}\ln{\hat{L}_N}$ и $N^{-1}\ln{L_N}(\theta)$ имеют разные пределы по вероятности для конечного $S$. Это подталкивает к поиску альтернативных оценок, полученных на основе симуляций, так как мы не можем установить $S = \infty$, а устанавливать большое $S$ может быть затруднительным в вычислительном плане.

Очевидный подход --- найти несмещенную вспомогательную оценку для логарифма плотности $\ln f_i$, а не для $f_i$, но на практике это невозможно. Вместо этого, в этом разделе мы представляем вариант оценки симуляционного правдоподобия с поправкой на смещение, а в следующем разделе мы представляем альтернативу, менее эффективную оценку, чем оценку симуляционного правдоподобия, которая является состоятельной для конечного $S$.

Гурьеру и Монфорт (1991) приводят выражение для смещения оценки симуляционного правдоподобия. При фиксированном $S$ оценки симуляционного ММП  могут быть несостоятельными из-за того, что в этом случае $\ln \hat{f}$ --- несостоятельная оценка $\ln f$. Для уменьшения несостоятельности возможно использовать скорректированную на смещение логарифмическую функцию правдоподобия. Запишем

\[
\ln\hat{f}=\ln[f+(\hat{f}-f)].
\]

Разложение в ряд Тейлора до второго порядка в окрестности $\ln{f}$ даст

\[
\ln\hat{f} {\sim eq} \ln{f}+\dfrac{\hat{f}-f}{f}-\dfrac{1}{2}\dfrac{(\hat{f}-f)^2}{f^2}.
\]

Проинтегрируем согласно плотности распределения $u$ и вычислим значение $\ln{f}$

\begin{equation}
\ln{f} {\sim eq} \E_{u}[\ln\hat{f}]+\dfrac{1}{2}\dfrac{\E_{u}[(\hat{f}-f)^2]}{f^2},
\end{equation}

предполагая, что вспомогательная оценка $\hat{f}$ несмещенная и, следовательно, $\E_{u}[\hat{f}]=f$. Это выражение показывает, что если вспомогательная оценка $\hat{f}$ имеет маленькую дисперсию, то смещение незначительно.

Расчет оценок, скорректированных на смещение предполагает использование уточненной логарифмической функции максимального правдоподобия, основу которой составляет правая часть выражения (12.22). Для вспомогательной оценки (12.17), $\hat{f}$ равно $S^{-1}\sum_{s}\tilde{f}^{s}$ и $\E_{u}[(\hat{f}-f)^2]$ равно $S^{-1}\sum_{s}\E_{u}[(\tilde{f}^s-f)^2]$. Если симуляции независимы по $s$, то приближенное значение последнего выражения равно $S^{-1}\sum_{s}(\tilde{f}^s-\hat{f})^2$. Тогда, выражение (12.22) дает  оценку симуляционного ММП, скорректированную на смещение первого порядка, $\hat{\theta}_{BCMSL}$, которая максимизирует

\[
\ln\hat{L}_{B,N}(\theta)=\sum^{N}_{i=1}\left[\ln\hat{f}(y_i|x_i,u_{iS},\theta)+\dfrac{1}{2S}\dfrac{\sum^{S}_{s=1}[\tilde{f}(y_i,x_i,u^{s}_{i},\theta)-\hat{f}(y_i|x_i,u_{iS},\theta)]^2}{\hat{f}(y_i|x_i,u_{iS},\theta)^2} \right], 
\]
где $\hat{f}(y_i|x_i,u_{iS},\theta)=S^{-1}\sum_{s}\tilde{f}(y_i,x_i,u^{s}_i,\theta)$. Выгоды от сокращения смещения будут меняться от случая к случаю, поскольку смещение может быть больше, чем предполагалось.

\subsection{Пример ненаблюдаеммой гетерогенности}

Предположим, что $y_i{\sim } \mathcal{N}[\theta_i,1]$ и значения скалярного параметра $\theta_i$ заданы выражением $\theta_i=\theta+u_i$, где $u_i$ обозначает ненаблюдаемую гетерогенность, и известно распределение $u_i$. Условная плотность $y$ при заданном значении $u$, определяется как 

\begin{equation}
f(y|u,\theta)=\dfrac{1}{\sqrt{2\pi}}\exp\left\lbrace -(y-\theta-u)^{2}/2\right\rbrace. 
\end{equation}

Тем не менее, выводы по $\theta$ должны быть основаны на плотности безусловного распределения $y$, что требует  интегрирования по $u$. Предположим, что $u$ имеет скошенное распределение с ненулевым средним, не зависящим от неизвестных параметров, и плотность распределения задана выражением

\begin{equation}
g(u)=e^{-u}\exp(-e^{-u}),
\end{equation}

Оценка методом максимального правдоподобия невозможна, поскольку для плотности безусловного распределения $f(y|\theta)$, которая равна $\int{f(y|\theta,u)}g(u)du$, не существует выражения в аналитическом виде. Вместо ММП используется симуляционная оценка максимального правдоподобия, со вспомогательной оценкой из выражения (12.16) так, что $\hat{\theta}_{MSL}$ максимизирует

\begin{equation}
\ln\hat{L}_{N}(\theta)=\dfrac{1}{N}\sum^{N}_{i=1}\ln\left(\dfrac{1}{S}\sum^{S}_{s=1}\dfrac{1}{\sqrt{2\pi}}\exp\lbrace{-(y_i-\theta-u^{s}_i)^{2}/2}\rbrace \right), 
\end{equation}

где значения $u^{s}_{i}, s=1,\ldots ,S$ имеют  распределение экстремальных значений с  плотностью $g(u_i)$ из (12.24). Оценка симуляционного максимального правдоподобия  $\hat{\theta}_{MSL}$ является решением условия первого порядка

\begin{equation}
\dfrac{\partial{\ln\hat{L}_{N}(\theta)}}{\partial\theta}=\dfrac{1}{N}\sum^{N}_{i=1}\dfrac{\sum^{S}_{s=1}(y_i-\theta-u^{s}_i)\exp\lbrace{-(y_i-\theta-u^{s}_{i})^{2}/2}\rbrace}{\sum^{S}_{s=1}\lbrace{-(y_i-\theta-u^{s}_{i})^{2}/2}\rbrace} = 0,
\end{equation}

Не смотря на то, что параметр $\theta$ не может быть выражен аналитически, возможно использовать стандартные итерационные методы для расчета $\hat{\theta}_{MSL}$.

Для того, чтобы оценки симуляционного ММП были состоятельны, необходимо чтобы $S \rightarrow \infty$, помимо того, объем выборки должен стремиться к бесконечности, $N \rightarrow \infty$, это означает, что потенциально потребуется большой объем вычислений. Как обычно, оценка симуляционного ММП распределена асимптотически нормально с асимптотической дисперсией, наиболее легко рассчитываемой BHHH методом (12.21), равна 

\begin{equation}
\hat{\V}[\hat{\theta}_{MSL}]=\left(\sum^{N}_{i=1}\left[\dfrac{\sum^{S}_{s=1}(y_i-\hat{\theta}_{MSL}-u^{s}_i)\exp\lbrace{-(y_i-\hat{\theta}-u^{s}_i)^{2}/2} \rbrace} {\sum^{S}_{s=1}\lbrace{-(y_i-\hat{\theta}_{MSL}-u^{s}_i)^{2}/2} \rbrace} \right]^{2}  \right)^{-1}.
\end{equation}

Эта оценка является полностью эффективной.

\begin{table}[h]
\begin{center}
\caption{\label{tab:mslexample} Оценивание с помощью MSL: пример}
\begin{tabular}{lccccc}
\hline 
\hline
{\bf{Число симуляций}} & $S = 1$ & $S = 10$ & $S = 100$ & $S = 1,000$ & $S = 10,000$ \\ 
\hline
$MSL$-оценка параметра $\hat{\theta}$ & 1.0416 & 1.0594 & 1.175 & 1.185 & 1.1828 \\ 
Стандартная ошибка & (.0968) & (.1093) & (.1453) & (.1448) & (.0091) \\ 
$\ln\hat{L}(\hat{\theta})$ & -136.31 & -174.38 & -190.44 & -192.43 & -192.35 \\ 
\hline 
\hline
\end{tabular} 
\end{center}
\end{table}

Для иллюстрации рассмотрим выборку $\lbrace{y_1,\ldots ,y_{100}}\rbrace$ размером $N=100$, сгенерированную на основе моделей (12.23) и (12.24) с параметром $\theta=1$. В таблице 12.2 даны оценки с учетом роста количества симуляций $S$. Для малых значений $S$  оценка симуляционного ММП несостоятельна. При $S=10,000$  оценка $\hat{\theta}_{MSL}$ стабилизируется, несмотря на то, что оценка средней квадратической ошибки достаточно резко изменяется. Симуляционная функция правдоподобия падает с ростом $S$, но потом стабилизируются.     Падение ожидаемо, поскольку вспомогательная оценка дает несмещенные оценки для $f(y|\theta)$, но со смещением вверх для $\ln{f(y|\theta)}$, поскольку согласно неравенству Йенсена $\ln{\E[\hat{f}(y|\theta)]}>\E[\ln{\hat{f}(y|\theta)}]$, так как функция натурального логарифма глобально вогнута; см. Приложение A (Раздел А.8).

\section{Оценка симуляционного метода моментов}

Симуляционный подход для случаев, когда не существует аналитического выражения целевой функции, может быть применен не только для оценок максимального правдоподобия. Вместе с тем, в некоторых случаях возможно получить состоятельные оценки параметров при небольшом количестве симуляций на каждое наблюдение, несмотря на то, что существует потеря эффективности.

\subsection{Симуляционная М-оценка}

Рассмотрим М-оценку, пусть целевая функция имеет вид

\[
Q_N(\theta)=\dfrac{1}{N}\sum^{N}_{i=1}q(y_i,x_i,\theta).
\]

Метод максимального правдоподобия это частный случай с $q(y,x,\theta)=\ln{f(y|x,\theta)}$.

Предположим, что не существует аналитического выражения для $q(\cdot)$, но возможно получить оценку, используя симуляции. Тогда симуляционная М-оценка минимизирует функцию

\begin{equation}
\hat{Q}_N(\theta)=\dfrac{1}{N}\sum^{N}_{i=1}\hat{q}(y_i,x_i,u_{iS},\theta),
\end{equation}
где по аналогии с разделом 12.4.1, $\hat{q}_i$ оценка для $q_i$, рассчитанная на основе значений вектора $u_{iS}$, состоящего из $S$ симуляций $u^{s}_i, s=1,\ldots ,S$, из соответствующего распределения. Обычно, $\hat{q}(\cdot)=S^{-1}\sum_{s}\tilde{q}(y_i|x_i,\theta,u^{s}_i)$, где $u^{s}_i$ это $s$-тая симуляция.

Симуляционная М-оценка будет состоятельной, если исходная М-оценка состоятельна, а также 

\begin{equation}
\plim\hat{Q}_N(\theta)=\plim{Q_N(\theta)},
\end{equation}
поскольку из Раздела 5.3 следует, что необходимым условием состоятельности исходной М-оценки является максимальность предела целевой функции $\plim{Q_N(\theta)}$ в точке $\theta=\theta_0$. В этом случае первый вероятностный предел берется по всем стохастическим переменным, в том числе по симулированным значениям $u_{iS}$, в то время как второй вероятностный предел не зависит от $u_{iS}$.

Условие (12.29) удовлетворяется, если вспомогательная оценка  такова, что $\hat{q}_i-q_i \stackrel{p}{\rightarrow} 0$ при $S \rightarrow \infty$, поскольку тогда $N^{-1}\sum_{i}\hat{q}_i-N^{-1}\sum_{i}q_{i}\stackrel{p}{\rightarrow} 0$. Это  предположение было сделано Разделе 12.4. Кроме того, симуляционная М-оценка должна иметь такое же предельное распределение как обычный М-оценка, если, также как в Разделе 12.4 $S$ увеличивается с ростом размера выборки, таким образом, что $\sqrt{N}/S \rightarrow 0$. Для выполнения этого условия требуется много симуляций.

\subsection{Сокращение количества симуляций}

Предположим, что вспомогательная оценка $\hat{q}_i$ не только состоятельна, но и несмещенная. Далее, применение закона больших чисел и, для упрощения, исключение из обозначений стохастических переменных кроме симулированных значений, дает равенство $\plim{\hat{Q}_N}(\theta)=\lim{N^{-1}\sum_{i}\E_{u_{iS}}[\hat{q}_i]}=\lim{N^{-1}}\sum_{i}q_i=\plim{Q_{N}(\theta)}$ и условие (12.29) удовлетворено. Таким образом, симуляционная М-оценка  состоятельна, а на одно наблюдения приходится всего лишь одна симуляция $u_i$, при условии $\E_{u_{iS}}[\hat{q}_i]=q_i$.

К сожалению, этот результат трудно применить на практике, поскольку редко возможно найти несмещенную вспомогательную оценку $q_i$. Например, используя метод максимального правдоподобия возможно найти несмещенную вспомогательную оценку для плотности $f_i$, но невозможно найти несмещенную вспомогательную оценку для $\ln{f_i}$. Аналогично, для нелинейного МНК  возможно найти несмещенную вспомогательную оценку для условного среднего, но невозможно найти несмещенную оценку для среднеквадратических ошибок, т.к. она содержит в себе квадрат условного среднего. 

В некоторых случаях, возможно применить полученный результат, чаще это удается в случае, если используется метод моментов или обобщенный метод моментов, а не М-оценка.

\subsection{Симуляционный метод моментов}

Пусть, согласно теории, условие на условный момент задано выражением

\begin{equation}
\E[m(y_i,x_i,\theta_0)]=0,
\end{equation}
где, для упрощения, $m(\cdot)$ скаляр. Допустим, что $w_i$ обозначает инструментальные переменные, и функция $m$ от $x_i$ и $\theta_0$, удовлетворяют условию

\begin{equation}
\E[w_{i}m(y_i,x_i,\theta_0)]=0.
\end{equation}

Оценка параметра, полученная при помощи метода моментов $\hat{\theta}_{MM}$ (см. Главу 6.3.1) минимизирует значение функции

\begin{equation}
Q_{N}(\theta)=\left[\dfrac{1}{N}\sum^{N}_{i=1}w_{i}m(y_i,x_i,\theta)\right]^{'}\left[\dfrac{1}{N}\sum^{N}_{i=1}w_{i}m(y_i,x_i,\theta)\right],
\end{equation}
где для простоты предполагается случай точной идентификации, $\dim[w_i]=\dim[\theta]$. Результаты можно обобщить на более общий случай сверхидентифицируемой модели, тогда обозначения будут более сложными, поскольку необходимо вводить обозначения для взвешивающей матрицы и, кроме того, оценка производится при помощи ОММ.

Оценка параметра методом моментов состоятельна и в пределе имеет нормальное распределение с ковариационной матрицей, значения которой частично зависят от инструментов $w_i$. Рассмотрим пример нелинейной регрессии, где $m(y,x,\theta)=y-\E[y|x]$ --- это остаточный член и условное среднее $\E[y|x]$ является заданной функцией от $x$ и $\theta$. Тогда наилучшим выбором инструментальной переменной будет $w=\partial{\E[y|x]}/{\partial{\theta}}{\mid}_{\theta_0}$ при гомоскедастичности ошибок, поскольку тогда условия первого порядка метода моментов и нелинейного МНК совпадают.

Теперь предположим, что не существует аналитического выражения для $m(y,x,\theta)$. Например, в модели нелинейной регрессии может не быть аналитического выражения для условного среднего. Вместо этого, $m(y,x,\theta)$ может быть выражено интегралом

\begin{equation}
m(y_i,x_i,\theta)=\int{h(y_i,x_i,u_i,\theta)g(u_i)du_i},
\end{equation}
для некоторых функций $h(\cdot)$ и $g(\cdot)$, и интеграл  не имеет аналитического решения. В таком случае невозможно получить оценку методом моментов.

Оценка параметра, полученная симуляционным методом моментов (MSM, method of simulated mometns) $\hat{\theta}_{MSM}$ минимизирует

\begin{equation}
\hat{Q}_{N}(\theta)=\left[\dfrac{1}{N}\sum^{N}_{i=1}w_{i}\hat{m}(y_i,x_i,u_{iS},\theta) \right]^{'}\left[\dfrac{1}{N}\sum^{N}_{i=1}w_{i}\hat{m}(y_i,x_i,u_{iS},\theta) \right],  
\end{equation}
где $\hat{m}(y_i,x_i,u_{iS},\theta)$ несмещенная вспомогательная оценка для $m(y_i,x_i,\theta)$, которая удовлетворяет условию

\begin{equation}
\E[\hat{m}(y_i,x_i,u_{iS},\theta)]=m(y_i,x_i,\theta),
\end{equation}
где $u_{iS}$ означает $S$-тое случайное значение предельной плотности $g(u_i)$ и $S\geq 1$. Примеры $m_i$ и вспомогательных несмещенных оценка $\hat{m}_i$,  даны далее.

\subsection{Распределение оценок симуляционных моментов}

Оценка симуляционных моментов была предложена МакФадденон (1989), который доказал нижеследующие свойства этой оценки.

\begin{proposition}[Распределение оценок симуляционных моментов] (МакФадден (1989): Предположим, что 
\begin{enumerate}
\item У процесса порождающего данные функция $m(y,x,\theta_0)$ имеет нулевое условное среднее как в (12.30) и $w_{i}m(y,x,\theta_0)$ имеет нулевое безусловное среднее, как в (12.31) и выполняются другие предпосылки так, что оценка методом моментов, минимизирующая значение функции (12.32) состоятельна и распределена асимптотически нормально.
\item Функция $m(y,x,\theta_0)$ определена выражением (12.33) и её вспомогательная оценка $\hat{m}(y,x,\theta_0)$ является несмещенной и удовлетворяет условию (12.35).
\end{enumerate}
Тогда, при фиксированном $S$, оценка симуляционного метода моментов, минимизирующая (12.34), состоятельна, распределена асимптотически нормально при $N \rightarrow 0$ и имеет предельное нормальное распределение с
\end{proposition}

\begin{equation}
\sqrt{N}(\hat{\theta}_{MSM}-\theta_0) \stackrel{d}{\rightarrow} \mathcal{N}[0,A^{-1}(\theta_0)B(\theta_0)A^{-1}{(\theta_0)}^{'}],
\end{equation}

где

\begin{equation}
\left. A(\theta_0)=\plim\dfrac{1}{N}\sum^{N}_{i=1}w_{i}\dfrac{\partial{m_{i}(\theta)}}{\partial{\theta}'} \right|_{\theta_0}
\end{equation}

и

\begin{equation}
B(\theta_0)=\plim\dfrac{1}{N}\sum^{N}_{i=1}w_{i}\V[\hat{m}_{i}(\theta_0)]{w_{i}}',
\end{equation}
здесь ковариационная матрица $\V[\cdot]$,  рассчитывается по отношению к условному распределению переменной $y_i$, при заданном значении $x_i$, а также для симулированных значений $u_{iS}$, вычисляемых согласно (12.35).

Перед тем как перейти к выводу данного утверждения отметим следующее. Во-первых, оценка симуляционных моментов имеет особое свойство --- она состоятельна, даже при $S=1$. Во-вторых, для конечного числа $S$ наблюдается потеря эффективности. Ковариационная матрица оценки $\hat{\theta}_{MM}$ аналогична матрице оценки $\hat{\theta}_{MSM}$, за исключением того, что у оценки моментов $\V[\hat{m}_i]$ из (12.38) заменяется на меньшую $\V[m_i]$. В-третьих, потеря эффективности, которая появляется из-за симуляций, исчезает при $S \rightarrow \infty$, так как в этом случае $\V[\hat{m}_i] = \V[m_i]$. В-четвертых, как и для оценивания с помощью метода моментов, метод симуляционных моментов при $S \rightarrow \infty$ может быть неэффективен по сравнению с другими методами оценивания, если инструменты $w$ выбраны плохо.

Состоятельность оценок, полученных методом симуляционных моментов требует выполнения условия (12.29) для $\hat{Q}_N(\theta)$ и $Q_N(\theta)$, заданных выражениями (12.34) и (12.32), соответственно. Согласно закону больших чисел

\[
\plim\dfrac{1}{N}\sum^{N}_{i=1}w_{i}\hat{m}_i=\plim{N^{-1}\sum^{N}_{i=1}w_{i}\E_{u_{iS}}[\hat{m}_i]},
\]
где первый вероятностный предел берется по всем стохастическим переменным, в то время как второй вероятностный предел берется по всем переменным за исключением симулированных значений $u$. В данном случае, $\E_{u_{iS}}[\hat{m}_i]=m_i$, поскольку $\hat{m}_i$ является несмещенной вспомогательной оценкой, поэтому

\[
\plim\dfrac{1}{N}\sum^{N}_{i=1}w_{i}\hat{m}_i=\plim{N^{-1}\sum^{N}_{i=1}w_{i}m_{i}}.
\]
Это, в свою очередь, приводит к тому, что $\plim{\hat{Q}_N(\theta)}=\plim{Q_{N}(\theta)}$. Тогда, оценка $\hat{\theta}_{MSM}$ является состоятельной, если $\theta_0$ является точкой максимума функции  $\plim{Q_N(\theta)}$, что является необходимым условием состоятельности исходного метода моментов.

Для предельного распределения, дифференцирование $\hat{Q}_N(\theta)$ по $\theta$ приводит к

\[
\left(\dfrac{1}{N}\sum^{N}_{i=1}w_{i}\dfrac{\partial\hat{m}_{i}(\theta)}{\partial{\theta}'} \right)'\dfrac{1}{N}\sum^{N}_{i=1}w_{i}\hat{m}_{i}(\hat{\theta})=0.
\]

Первая матрица является квадратной матрицей полного ранга, то есть  $\hat{\theta}_{MSM}$ удовлетворяет условию первого порядка

\[
\dfrac{1}{N}\sum^{N}_{i=1}w_{i}\hat{m}_{i}(\hat{\theta})=0,
\]
где $\hat{m}_{i}(\theta)=\hat{m}_{i}(y_i,x_i,u_{iS},\theta)$. Используя стандартное разложение первого порядка в ряд Тейлора в окрестности точки $\theta_0$, получим

\[
\left. \sum^{N}_{i=1}w_{i}\hat{m}_{i}(\theta_0)+\sum^{N}_{i=1}w_{i}\dfrac{\partial{\hat{m}}_{i}(\theta)}{\partial{\theta}'}\right|_{\theta^{*}}(\hat{\theta}-\theta_0)=0,
\]

и, таким образом,

\[
\left. \sqrt{N}(\hat{\theta}-\theta_0)=-\left(N^{-1}\sum^{N}_{i=1}w_{i}\dfrac{\partial\hat{m}_i(\theta)}{\partial\theta'}\right|_{\theta^{*}}\right)^{-1}N^{-1/2}\sum^{N}_{i=1}w_{i}\hat{m}_i(\theta_0) 
\]
Теперь $\E_{u}[\partial\hat{m}(\theta)/\partial\theta]=\partial{\E_u[\hat{m}(\theta)]}/\partial\theta=\partial{m(\theta)}/\partial\theta$, тогда первая матрица с правой стороны сходится к $A(\theta_0)$, заданной в утверждении 12.2. Второй элемент справа имеет предельное нормальное распределение с нулевым средним и ковариационной матрицей

\[
B(\theta_0)=\plim{\dfrac{1}{N}}\sum^{N}_{i=1}w_{i}\V[\hat{m}_{i}(\theta_0)]w'_{i},
\]
как в утверждении 12.2, где дисперсия $\V[\hat{m}_{i}(\theta_0)]$ считается по отношению к  $u_{iS}$ и условному распределению $y_i$, при заданном значении $x_i$.

Поскольку значения $u_{iS}$ независимы от $y_i$ получим, что

\[
\V_{y,u}[\hat{m}(\theta_0)]=\V_{y}[\E_{u}[\hat{m}(\theta_0)]]+\E_{y}[\V_{u}[\hat{m}(\theta_0)]]=\V_{y}[m(\theta_0)]+\E_{y}[\V_{u}[\hat{m}(\theta_0)]].
\]

Применяя подстановку, получаем детальное выражение для $B(\theta_0)$,  данное в утверждении 12.2.


Симуляция приводит к росту дисперсии оценок симуляционных моментов из-за величины $\E_{y}[\V_{u}[\hat{m}(\theta_0)]]$, стремящейся к нулю при $S \rightarrow \infty$. В частном случае, когда вспомогательная оценка является частотной оценкой, может быть показано, что $\V_{y,u}[\hat{m}(\theta_0)]=(1+1/S)\V_{y}[m(\theta_0)]$, т.е. в данном случае симуляционный метод моментов увеличивает дисперсию MM-оценок в $(1+1/S)$ раз!

\subsection{Выбор между симуляционным методом моментов и симуляционным правдоподобием}

Любой практик хочет взвесить плюсы и минусы симуляционных моментов и симуляционного правдоподобия. При условии, что оценки симуляционных моментов состоятельны для малых $S$, и, учитывая трудность обеспечения достаточно большого значения $S$, чтобы была хорошая аппроксимация функции правдоподобия, почему оценка симуляционного правдоподобия может быть когда-либо более предпочтительной, чем оценка симуляционных моментов?

Во-первых, заметим, что симуляционное правдоподобие в принципе является прямым и простым для реализации способом. При параметрических предположениях, оптимальное взвешивание наблюдений внутренне присуще симуляционному правдоподобию. Метод симуляционных моментов, аналогично ОММ, требует работы с произведениями взвешивающих функций (или инструментальными переменными) и остатков, и эти компоненты могут быть коррелированы. Численная неустойчивость ОММ-оценки (без симуляций) была описана, например, Альтонжи и Сегалом (1996) (см. раздел 6.3.5). Точно так же, Гевеке, Кин, и Рункл (1997) и МакФадден и Рууд (1994) представили доказательства неустойчивости оценки симуляционных моментов. Тем не менее, хотя простота выступает в пользу симуляционного правдоподобия, нельзя забывать про трудности, связанные проведением достаточного количества симуляций.

\subsection{Пример ненаблюдаемой гетерогенности}

Обратимся к примеру, рассмотренному в Разделе 12.4.5. Тогда $y_i{\sim }\mathcal{N}[\theta+u_i,1]$, где $u_i$ имеют плотность $g(u_i)$, заданную выражением (12.24). Поскольку $\E[y_i-\theta-u_i]=0$, можно оценить $\theta$ методом моментов, и оценка  является решением 

\begin{equation}
\dfrac{1}{N}\sum^{N}_{i=1}(y_i-\theta-\E[u_i])=0,
\end{equation}
в результате получим, что $\hat{\theta}_{MM}=\overline{y}-\E[\overline{u}]$. Предположим, что $\E[\overline{u}]$ неизвестно. В таком случае вместо этого можно использовать оценку симуляционных моментов $\hat{\theta}_{MSM}$, которая является решением уравнения

\begin{equation}
\dfrac{1}{N}\sum^{N}_{i=1}\left(y_i-\theta-\dfrac{1}{S}\sum^{S}_{s=1}u^{s}_i \right)=0, 
\end{equation}
где $u^{s}_i$ независимы и имеют распределение экстремальных значений.

Оценивающее уравнение (12.40) может быть решено, и

\begin{equation}
\hat{\theta}_{MSM}=\overline{y}-\overline{\overline{u}},
\end{equation} 
где $\overline{\overline{u}}=(NS)^{-1}\sum_{i}\sum_{s}u^{s}_i$ --- это среднее и по $N$, и по $S$. Однако, в общем случае, может потребоваться итерационный метод для вычисления оценки симуляционного правдоподобия.

Дисперсию  оценки $\hat{\theta}_{MSM}$ легко получить. По построению симуляционные значения $u$ независимы между собой и независимы с начальными данными $y$, таким образом $\V[\hat{\theta}_{MSM}]=\V[\overline{y}]+\V[\overline{\overline{u}}]$. Тогда $\V[\overline{y}]=(\sigma^{2}_u+1)/N$. Поскольку $\overline{\overline{u}}$ среднее из $NS$ симуляций
$u$ и $\V[\overline{\overline{u}}]=\sigma^{2}_{u}/NS$, мы приходим к выводу, что

\begin{equation}
\V[\hat{\theta}_{MSM}]=\V[\overline{y}]+\V[\overline{\overline{u}}] = \dfrac{\sigma^{2}_u+1}{N}+\dfrac{\sigma^{2}_u}{NS}.
\end{equation}
Это выражение можно состоятельно оценить с помощью $\hat{\sigma}^{2}_u=(NS)^{-1}\sum^{N}_{i=1}\sum^{S}_{s=1}(u^{s}_i-\overline{\overline{u}})^2$.

Рассмотрим выборку $\lbrace{y_1,\ldots ,y_{100}}\rbrace$ размера $N=100$, сгенерированную с использованием модели (12.24) при $\theta=1$. В таблице 12.3 даны оценки симуляционных моментов при количестве симуляций стремящемся к бесконечности, $S \rightarrow \infty$. С ростом количества симуляций значение оценки симуляционных моментов сходится к оценке метода моментов, и происходит уменьшение стандартных ошибок.

\begin{table}[h]
\begin{center}
\caption{\label{tab:simulationest} Оценивание с помощью симуляционных моментов: пример}
\begin{tabular}{lccccc}
\hline 
\hline
{\bf{Число симуляций}} & $S = 1$ & $S = 10$ & $S = 100$ & $S = 1,000$ & $S = \infty (MM)$ \\ 
\hline
$MSM$-оценка параметра $\hat{\theta}$ & 1.0073 & 1.1096 & 1.2012 & 1.1887 & 1.1889 \\ 
Стандартная ошибка & (.2471) & (.1657) & (.1681) & (.1676) & (.1684) \\ 
\hline 
\hline
\end{tabular} 
\end{center}
\end{table}


\section{Косвенные оценки}

В этом разделе мы рассмотрим иной симуляционный подход, используемый в случае, когда исследователь хочет применить или оценить относительно простую модель, даже если процесс порождающий данные более сложный или его оценивание вызывает затруднения. Существует несколько вариантов и интерпретаций этого подхода; см. Гурьеру, Монфорт, и Рено (1993), Смит (1993), и Галлант и Таухен (1996). Рассматриваемый подход также получил название подход сопоставления моментов. Описание, представенное в этом разделе во многом совпадает с описанием в работе Гурьеру, Монфорт, и Рено (1993). 

Предположим, что параметрический процесс порождающие данные задан с помощью функции плотности $f(y;\theta),\theta{\e}R^{q}$, параметры которой трудно оценить. Допустим, что возможно специфицировать вспомогательную модель, в которой процесс порождающий данные задан плотностью $f^{a}(y;\beta), \beta{\e}R^{r}$, которую легко оценить с помощью метода квази-максимального (или иногда <<псевдо->>) правдоподобия. 
По причине идентифицируемости, обсуждаемой дальше, предположим, что размерность $\beta$ не меньше, чем размерность $\theta$, т.е. $r\geq q$. К примеру, вспомогательная модель может быть приближением функции правдоподобия исходной модели или может быть точным правдоподобием приближенной модели. 
Для заданной выборки, пусть $\hat{\beta}$ обозначает оценки квази-максимального правдоподобия. Тогда, согласно результатам Раздела 5.7, в общем случае $\hat{\beta}$ является несостоятельной оценкой параметра $\theta$, и, при некоторых предположениях о регулярности, эта оценка сходится по вероятности к так называемому псевдо-истинному значению, которое является функцией от $\theta$. 
Функция, которая связывает параметры вспомогательной модели с параметрами процесса порождающего данные называется связывающей функцией и обозначается $h(\theta)$. Аналитическая форма данной функции может быть известной или неизвестной. Таким образом, не всегда возможно рассчитать $\theta=h^{-1}(\beta)$ или $\hat{\theta}=h^{-1}(\hat{\beta})$.

Метод косвенной оценки может быть использован для улучшения оценок метода квази-максимального правдоподобия и получения оценок с меньшей асимптотическим смещением, чем у  $\hat{\beta}$. Идея состоит в том, чтобы использовать модель  с плотностью $f(y;\theta)$, чтобы случайным образом сгенерировать псевдо-наблюдения $y^{(s)}$ и применить вспомогательную модель с плотностью $f^{a}(y^{(s)};\beta)$, чтобы оценить $\hat{\beta}^{(s)}$, где $s$ обозначает $s$-тую симуляцию. Непрямая оценка является решением уравнения

\begin{equation}
\hat{\theta}=\arg \min_{\theta}(\hat{\beta}^{(s)}-\hat{\beta})'\Omega(\hat{\beta}^{s}-\hat{\beta}),
\end{equation}
где $\Omega$ --- это заданная симметричная положительно определенная матрица. Этот способ оценивания похож на метод минимального расстояния, рассмотренный в Разделе 6.7. То есть, последовательно генерируются псевдо-наблюдения и оценка параметров производится при помощи вспомогательной модели, с использованием псевдо-наблюдений. Итерации продолжается до тех пор, пока квадратичная форма (12.43) не примет минимальное значение. Важный момент состоит в том, что зерно (seed) генерирования псевдо-случайных значений $y^{(s)}$ остается неизменным, и источником изменчивости псевдо-наблюдений  является изменчивость значений $\hat{\beta}^{(s)}$. 

Перед тем как перейти к дальнейшему обсуждению, рассмотрим простой, конкретный пример с нелинейным процессом порождающим данные и линейной вспомогательной моделью. Мотивирован этот пример тем, что вспомогательная модель должна легко оцениваться, а процесс порождающий данные должен легко симулироваться.

Допустим, что процесс порождающий данные имеет вид:

\begin{equation}
y_i=\exp(x'_{i}\gamma)+u_i,
\end{equation}

\[
u_{i} \sim \mathcal{N}[0,\sigma^2].
\]

И вспомогательная модель задана следующим образом: 

\begin{equation}
y_i=x'_{i}\beta+\e_i,
\end{equation}

\[
\e_i\sim \mathcal{N}[0,\sigma^{2}_{\e}].
\]

Заметим, следующую интерпретацию:

$\dfrac{\partial{\E[y|x]}}{\partial{x}}=\beta$ (согласно вспомогательной модели),

$\dfrac{\partial{\ln{\E[y|x]}}}{\partial{x}}=\dfrac{\partial{\E[y|x]}}{\partial{x}}\times \dfrac{1}{\E[y|x]}=\gamma$ (согласно процессу порождающему данные).

Поэтому связывающая функция имеет вид $\gamma{\E[y|x]}=\beta$ или $\gamma=(\E[y|x])^{-1}\beta$. Заметим, что размерность $\beta$ равна размерности $\gamma$, т.е. $\dim[\beta]=\dim[\gamma]$.

При заданных значениях $(x_i,y_i,i=1,\ldots ,N)$, МНК-оценке $\hat{\beta}$, а также псевдо-случайных симуляциях размерности $N$, которые обозначены $u^{(0)}$, создадим $y^{(1)}_i(i=1,\ldots ,N)$ используя

\[
y^{(1)}_i=\exp(x'_{i}\hat{\beta})+u^{(0)}_i
\]
и получим обновлённую оценку $\hat{\beta}^{(1)}=(\sum{x_{i}x'_{i}})^{-1}\sum{x_{i}y^{(1)}_i}$, которая в дальнейшем используется для создания другого множества псевдо-наблюдений. Весь цикл симулирования повторяется, при фиксированных значениях $u^{(0)}$, до тех пор, пока $(\hat{\beta}^{(s)}-\hat{\beta})'\Omega(\hat{\beta}^{(s)}-\hat{\beta})$ не станет константой с ожидаемой точностью. В данном случае целесообразно выбрать $\Omega$ равной единичной матрице или $X'X$, выбор последнего варианта предполагает, что прогнозы из вспомогательной модели являются целью моделирования. Полученная в результате оценка параметра $\gamma$ является косвенной оценкой.

При других предпосылках $\dim{(\beta)}$ будет превышать $\dim{(\theta)}$ так, что  значение $\theta$ может не быть уникальным. Более того, при отсутствии аналитической формы связывающей функции, невозможно восстановить исходное значение параметра $\theta$, даже если размерности равны. В таком случае следует произвести оценку параметров вспомогательной модели, используя наилучшую косвенную оценку.

Для иллюстрации взаимосвязи между косвенной оценкой и оценкой, полученной методом сопоставления моментов, сделаем допущение, что $\Omega=X'X$; тогда $(\hat{\beta}^{(s)}-\hat{\beta})'X'X(\hat{\beta}^{(s)}-\hat{\beta})=(\hat{\beta}^{(s)}X-\hat{\beta}X)'(\hat{\beta}^{(s)}X-\hat{\beta}X)$, это означает, что косвенная оценка <<сопоставляет>> первый момент распределения. Для того, что вторые моменты также совпали, необходимо добавить параметры к вектору $\beta$, к примеру, параметр дисперсии. Таким образом, возможно сделать так, что значения первых начальных моментов будут совпадать.

При выполненных условиях регулярности, косвенный способ оценивания дает состоятельные и асимптотически нормальные оценки. Для дополнительной информации читатель может обратится к ранее упомянутым источникам.

\section{Вспомогательные оценки}

Как и в Разделе 12.3.2 рассмотрим расчет интеграла

\begin{equation}
I=\E[h(x)]=\int{h(x)g(x)dx},
\end{equation}
где для простоты $x$ чаще всего будет скаляром. Как и в Разделе 12.3 $x$ обозначает переменную интегрирования, в то время как в приложении переменная интегрирования обозначена часто за $u$, поскольку $x$ обозначает регрессор. 

Вспомогательная оценка это метод, позволяющий рассчитать $I$. Существует много способов сделать это, помимо прямого интегрирования с помощью Монте-Карло, приведенного в (12.14). В идеальном случае, вспомогательная оценка должна быть несмещенной, поскольку многие результаты Разделов 12.4 и 12.5 получены при использовании несмещенных оценок. Желательной является гладкость оценки, т.к. гладкость означает возможность применения итерационных градиентных методов. Однако, даже в этом идеальном случае время расчетов моделей, интересных с эмпирической точки зрения, может быть фантастически огромным. Далее рассмотрим несколько процедур, специально разработанных для ускорения процесса симулирования путем сокращения, для заданного количества симуляций, дисперсии симуляций по сравнению с базовыми методами, такими как интегрирование с помощью Монте-Карло. Более детально этот вопрос рассмотрен у Гевеке и Кина (2001).

\subsection{Частотные вспомогательный оценки}

Вначале рассмотрим пример частотной вспомогательной оценки, которая может быть применена к ряду дискретных моделей. Рассмотрение этого частного случая хорошо подчеркивает возможные трудности, которые могут возникнуть при симуляциях.

Предположим, что функция $h(x)$ является индикатором и принимает значение 1, если $x \in A$ и 0 в ином случае. Мы хотим подсчитать значение интеграла 
\[
I=\int{{\bf{1}}(x \in A)g(x)dx}.
\]

Применение прямого метода интегрирования с помощью Монте-Карло дает оценку

\[
\hat{I}_{FREQ}=\dfrac{1}{S}\sum^{S}_{s=1}{\bf{1}}(x^{s} \in A),
\]
где $x^s,s=1,\ldots ,S$ являются $S$ симуляциями функции $g(x)$. Этот метод является частотной вспомогательной оценкой поскольку оценка $I$ --- это частота, с которой $S$ симуляций $x^s$ попадают в  множество $A$. 

Основным возможным применением данного способа является мультиномиальная модель дискретного выбора, впервые упомянутая в Разделе 12.2.2. Именно она послужила толчком к исследованию симуляционных методов в экомнометрической литературе.  В модели с тремя альтернативами, вероятность выбора первой альтернативы, $p_1$, задана выражением (12.3) и равна интегралу по положительному ортанту двумерного нормального распределения. Значение частотной оценки $\hat{p}_1$ --- это просто доля симуляций $(u^{s}_1,u^{s}_2)$ двумерного нормального распределения у которых $u^{s}_1\geq 0$ и $u^{s}_2\geq 0$.

Применение частотных оценок имеет ряд ограничений. Во-первых, частотные оценки не являются ни дифференцируемыми, ни непрерывными по параметру $\theta$, который появляется в ${\bf{1}}(x \in A)$ и/или в $g(x)$. Малые изменения в значениях параметра $\theta$ приводят к одному и тому же числу симуляций, попадающих в положительный ортант. По этой причине МакФадден (1989) и Пэйкс и Поллард (1989) предложили более общую асимптотическую теорию, которая охватывает такие негладкие оценки. Однако на практике лучше всего  использовать альтернативные гладкие оценки, которые дифференцируемы по параметрам, так как это позволяет проводить расчёты с помощью градиентных методов.

Во-вторых, данный частотные вспомогательный оценки очень неэффективены, если очень малая часть $x \in A$. Например, для моделей дискретного выбора с $p_1=0.001$, даже если количество симуляций $S$ достигает 10,000 ошибки оценки $\hat{p}_1$ будут очень велики. Аналогичная проблема возникает в более общем случае, когда используется метод Монте-Карло (12.46) с непрерывной функцией $h(x)$, когда вероятность генерирования значений $x$ в области больших $h(x)$ низка.

В-третьих, применение данной вспомогательной оценки может быть затруднено на границах промежутка и, тогда в результате может получится, что $\hat{I}=0$ или $\hat{I}=1$, если согласно предпосылкам модели $0<I<1$ и это условие является необходимым для оценки модели.

\subsection{Сэмплирование по важности}

Сэмплирование по значимости (importance sampling) выражает интеграл (12.46) в следующем виде:

\begin{equation}
I=\int{\left(\dfrac{h(x)g(x)}{p(x)}\right)p(x)dx}
\end{equation}

\[
=\int{w(x)p(x)dx},
\]
где $p(x)$ функция плотности распределения построенная таким образом, что (а) легко сгенерировать случайную выборку из $p(x)$, (b) $p(x)$ имеет носитель совпадающий с  первоначальной областью интегрирования и (c) $w(x)=h(x)g(x)/p(x)$ легко оценить, ее значение ограничено и имеет конечную дисперсию. Далее используем оценку интеграла, рассчитанную при помощи прямого метода Монте-Карло на основе (12.47) вместо (12.46), и получим

\begin{equation}
\hat{I}_{IS}=\dfrac{1}{S}\sum^{S}_{s=1}w(x^{s}),
\end{equation}
где $x^s, s=1,\ldots ,S$ генерируются согласно плотности $p(x)$, а не $g(x)$. Термин сэмплирование по важности используется, потому что $w(x)$ определяет вес или <<важность>> различных точек в выборке. Сэмплирование по важности долгие годы применялось в литературе, посвященной байесовскому моделированию, и было введено в эконометрику Клоэком и Ван Дейком (1978) в качестве метода оценивания апостериорных распределений. Этот вопрос рассмотрен далее в разделе 13.4.

Оценка сэмплирования по важности $\hat{I}_{IS}$ имеет дисперсию $S^{-1}\V_p[w(x)]$ при независимой выборки из $p(x)$.  Дисперсию достигает своего минимума, если $w(x)$ постоянна на всем промежутке интегрирования, так как в этом случае дисперсия $\V_p[w(x)] = 0$. Это можно сделать, если задать $w(x) = \E_g[h(x)]$,  так как в этом случае $p(x) = h(x)g(x)/\E_g[h(x)]$ --- это плотность, интеграл которой равен 1. К сожалению, эта оценка, которая является теоретически идеальной, не доступна на практике, так как $\E_g[h(x)]$ неизвестно. Однако это говорит о потенциальном преимуществе сэмплирования по важности, особенно в случае если $p(x)$ так выбрано, чтобы $w(x)$ было практически неизменным.

Даже если сэмплирование по важности приводит к увеличению дисперсии, у него могут быть другие положительные черты. Оно позволяет получить гладкие оценки, если $w(x)$ является гладкой по оцениваемым параметрам. Более того, сэмплирование по важности полезно, если тяжело генерировать выборку из $g(x)$, что может часто случаться, если $x$ --- это вектор коррелированных случайных переменных.

Для мультиномиальной дискретной пробит-модели широко распространенное сэмплирование по важности --- это GHK (Geweke-Hajivassiliou-McFadden) оценка, предложенная Гевеке (1992), Хаживассилиу и МакФадденон (1994), и Кином (1994). Она рекурсивно усекает многомерную функцию нормального распределения так, чтобы выборка генерировалась только на положительном ортанте. Преимущества этой оценки по сравнению с частотной оценкой состоят в том, что она гладкая, требует гораздо меньший размер выборки для альтернатив, которые могут быть выбраны с малой вероятностью, а также у неё, скорее всего, не будет проблем с граничными значениями.

\subsection{Понижение дисперсии при помощи антитетического ускорения}

В ранее рассмотренных методах предполагалось, что симуляции независимы друг от друга, с использованием методов детально рассматриваемых в Разделе 12.8, при использовании плотности  $g(x)$, или, если используется сэмплирование по важности, то $p(x)$. 

В методах понижения дисперсии используются зависимые симуляции, поскольку это может привести к сокращению дисперсии оценки. Основным примером является создание антитетической выборки, использующей отрицательно коррелированные симуляции. Этот способ рассмотрен в работе Рипли (1987, стр. 129-132), Гевеке (1988) и Хаживассилиу (2000), а также в работе Гевеке (1995), где также рассмотрены и другие способы понижения дисперсии.

Предположим, что необходимо оценить интеграл $I$ в (12.46), где $x$ имеет нулевое среднее и симметричную плотность  $g(x)$. Оценка интеграла с помощью простого метода Монте-Карло,  построенного на выборке из $2S$  независимо и одинаково распределенных значениях с плотностью $g(x)$, равна

\[
\hat{h}_{2S}(x)=\dfrac{1}{2S}\sum^{2S}_{s=1}h(x^s)
\]
и, при  независимости $2S$ симуляций, дисперсия равна 

\[
\V[\hat{h}_{2S}(x)]=\dfrac{1}{2S}\V[h(x)].
\]

При создании антитетической выборки альтернативная оценка, при расчете которой используются только $S$ независимо и одинаково распределенных симуляций,

\begin{equation}
\hat{h}_{A,S}(x)=\dfrac{1}{S}\sum^{S}_{s=1}\dfrac{1}{2}(h(x^s)+h(-x^s)),
\end{equation}
значение выражения равно среднему $h(x)$, оцениваемых в точках $x^s$ и $-x^{s}$. Пара $(x^s,-x^s)$ называется антитетической парой и дает несмещенную оценку $I$, поскольку $x$ симметрично распределен с нулевым средним. Если среднее равно $\mu$, тогда антитетическая пара равна $(x^s,2\mu-x^s)$. Если  количество независимых симуляций  для $x^s$  равно $S$ дисперсия $\hat{h}_{A,S}(x)$ равна 

\[
\V[\hat{h}_{A,S}(x)]=\dfrac{1}{S^2}\sum^{S}_{s=1}\dfrac{1}{4}(\V[h(x^s)]+2\Cov[h(x^s),h(-x^s)]+\V[h(-x^s)])=\dfrac{1}{2S}(\V[h(x)]+\Cov[h(x),h(-x)]).
\]

Таким образом, использование антитетической выборки даст более эффективные оценки, чем использование обычных, независимо и одинаково распределенных выборочных значений, если ковариация отрицательна, тогда значение вариации $\hat{h}_{A,S}(x)$ будет меньше, чем $\hat{h}_{2S}(x)$. Смена знака сгенерированного значения и его повторное использование --- это попытка получить отрицательную корреляцию при моделировании. Отрицательная корреляция будет появляться, если функция задана линейно или если отсутствует сильная нелинейность. Однако, в общем случае, нельзя утверждать, что будет выигрыш в  эффективности. К примеру, если $h(\cdot)$ симметрична относительно нуля, то $\Cov[h(x),h(-x)]=\V[h(x)]$.

Антитетическое сэмплирование может применяться так же для несимметричной плотности $g(x)$. Предположим, что симулированные значения $x$ могут быть получены методом обратного преобразования, рассмотренного далее в Разделе 12.8.2. Тогда возможно получить симулированные значения $u$, к примеру равномерно распределенные с параметрами $[0,1]$, произвести антитетическое преобразование $(1-u)$ и применить метод обратного преобразования для того, чтобы получить симуляции из выбранного распределения, так что $x_1=G^{-1}(u)$ и $x_2=G^{-1}(1-u)$, где $G(\cdot)$ известная  функция распределения $x$. Тогда $(x_1,x_2)$ формируют антитетическую пару и сокращение дисперсии наблюдается, если

\[
\Cov[h(G^{-1}(u)),h(G^{-1}(1-u))]=\Cov[f(u),f(1-u)]<0,
\] 
где $f(u)$ является сложной функцией $h(G^{-1}(u))$. Если $f(\cdot)$ монотонная функция, тогда дисперсия сокращается (Роберт и Казелла, 1999, стр. 112). Однако, свойство монотонности трудно проверить. Кроме того, утверждение о сокращении дисперсии справедливо только для метода обратного преобразования, в то время как на практике применяются другие методы формирования псевдо-случайных чисел (см. Раздел 12.8). Таким образом, трудно заранее понять, можно ли добиться роста эффективности в данной конкретной задаче.


Несмотря на то, что фантастический выигрыш в эффективности проявляется не во всех случаях, существенный выигрыш наблюдается во  многих случаях. Антитетическое сэмплирование также может применяться для ускорения сэмплирования по важности (Даниэлссон и Ричард, 1993)

Антитетическая выборка может использоваться и для многомерных распределений. Рассмотрим двумерную случайную величину $(x,y)$,  плотность которой симметрична относительно точки $(0,0)$. В этом случае сначала меняется знак для всех элементов по-отдельности, а затем попарно. Тогда антитетическая четверка состоит из $((x^s,y^s),(-x^s,y^s),(x^s,-y^s),(-x^s,-y^s))$. Для случайных величин размерности $m$ применяется аналогичная идея.

\subsection{Вычисления с использованием квази-случайной последовательности}

Второй метод сокращения дисперсии предполагает замену псевдо-случайных величин квази-случайными величинами, которые являются систематизированными случайными симуляциями, обеспечивающие лучший охват множества значений. Потенциальное ограничение этого подхода заключаются в том, что случайность требуется для использования закона больших чисел и центральной предельной теоремы, которые оправдывают применение симуляционных методов.

В квази-методе Монте-Карло вместо $S$ псевдо-случайных величин используются неслучайные значения принадлежащие области интегрирования. Основным примером являются последовательности Халтона, их обзор можно найти в книге Пресс и др. (1993) и впервые введенной в эконометрическую литературу Бхатом (2001) и Трейном (2003).

Последовательности Халтона имеют два желательных свойства. Во-первых, эти последовательности построены таким образом, чтобы достаточно равномерно покрывать область значений симулируемого  распределения. При использовании более равномерно распределенных симуляций для каждого отдельного наблюдения, значения симулированных вероятностей меняются меньше, чем значения, рассчитанные на основе случайных симуляций. Эти результаты схожи с детерминированной оценкой интеграла по заданной  сетке. 
Во-вторых, в последовательности Халтона новое наблюдение стремится заполнить пустые места, оставленные предыдущими наблюдениями. По этой причине оцененные вероятности отрицательно коррелируют по наблюдениям. Как и в случае антитетических пар случайных величин отрицательная корреляция уменьшает дисперсию оцениваемой функции. При подходящих условиях регулярности можно показать, что ошибка интегрирования при использовании квази-случайных последовательностей имеет порядок $N^{-1}$ по сравнению с псевдо-случайными последовательностями, где скорость сходимости равна $N^{-1/2}$ (Бхат, 2001).

Последовательность Халтона легко представить с помощью примера. Предположим, что симулируемая функция, зависит от одной случайной переменной. Стартовое значение является простым числом. Последовательность Халтона, в основе которой лежит просто число 2, строится следующим образом. Вначале единичный интервал $(0,1)$ делится на две части. Точка деления $1/2$ становится первым элементом последовательности Халтона. Далее каждая из частей делится на две части. Точки деления, $1/4$ и $3/4$ становятся двумя последующим элементами последовательности. Деление каждой из образовавшихся четырех частей на две части и дальнейшее продолжение процесса дает последовательность $\lbrace1/2,1/4,3/4,1/8,3/8,\ldots \rbrace$. Аналогично, последовательность, построенная на простом числе 3 выглядит как $\lbrace1/3,2/3,1/9,2/9,4/9,\ldots \rbrace$. Последовательности Халтона, строящиеся на составном числе, порождают разбиения единичного отрезка, похожее на разбиение, порождаемое последовательностями Халтона для делителей данного составного числа.

Длина последовательности определяется количеством наблюдений $N$ и количеством симуляций $S$. Принято отбрасывать первые элементы (около 20), поскольку первые элементы коррелированны для последовательностей Халтона, порождаемых разными простыми числами (для примера см. Трейн, 2003). Следовательно, сначала необходимо построить последовательности Халтона длиной $N\times S+20$ и отбросить первые двадцать элементов каждой последовательности. Далее, для каждого элемента каждой последовательности нужно рассчитать обратную функцию  распределения. Итоговые значения являются выборкой Халтона из заданного распределения.

Среди главных преимуществ квази-случайных чисел следует отметить, что квази-случайные числа по построению более равномерно покрывают область значений по сравнению с псевдо-случайными числами. Это можно увидеть на Рисунке 12.1. На этом рисунке, график 2  показывает симуляции двумерного нормального распределения, построенные с использованием последовательности Халтона. Другие графики показывают симуляции псевдо-случайных чисел для такого же распределения. Можно увидеть, что более равномерный охват выборочного пространства достигается в  случае последовательности Халтона.

Для более подробного изучения, рассмотрения примеров симуляционного моделирования с использованием последовательностей Халтона, а также чтобы увидеть  эффективность данного  подхода для одномерного или многомерного случаев, см. Трейн (2003, Глава 9). Метод работает хорошо для мультиномиальной логит-модели с нормально-распределенными случайными параметрами (Раздел 15.7).

\section{Методы генерации случайных величин}

Для рассмотренных ранее симуляций необходимы были способы генерирования случайных величин. В этом разделе описываются методы, используемые для генерирования случайных величин с плотностью, обозначаемой через $g(x)$ или $p(x)$ в Разделе 12.7 и $f(x)$ в этом разделе. Как правило достаточно сгенерировать равномерные или стандартные нормальные случайные величины, поскольку с их помощью можно получить многие другие распределения.

\vspace{2cm}

График 12.1 Сэмплирование по Халтону (график 2) в сравнении с псевдо-случайной выборкой.


Если случайная выборка используются для построения оценки с помощью симуляционных методов, тогда все генерации равномерного или стандартного нормального распределения должны быть сделаны перед вычислением любой оценки для предотвращения <<дрожания>>, проявляющегося в том, что могут не сходиться итерационные методы из-за создания новых симуляции при каждой итерации. Например, если $x\sim \mathcal{N}[\mu,\sigma^2]$ и оценки $\mu$ и $\sigma$ меняются при каждой итерации, тогда сделаем $NS$ начальных симуляций $z\sim \mathcal{N}[0,1]$ и при итерациях пересчитываем значения $x$, $x=\mu+\sigma{z}$, используя исходные симуляции $z$.

В этом разделе мы кратко обсуждаем некоторые стандартные методов генерирования случайных величин. Более углубленное и расширенное рассмотрение представлено во многих монографиях и обзорах, в том числе в работах Брэдли, Фокса и Шрейджа (1983), Дагпунара (1988), Деврой (1986) и Рипли (1987).


Перед тем как перейти к рассмотрению методов, отметим, что термин генератор случайных чисел является оксюмороном. Более точное определение --- псевдо-случайные числа. Главной особенностью этих генераторов является то, что они используют детерминистические способы для создания длинных цепочек чисел, которые имитируют свойства случайной выборки из целевого распределения. Целевое распределение будет зависеть от конкретной ситуации, однако в соответствии с контекстом этой книги равномерное, нормальное, экспоненциальное, гамма, логистическое и пуассоновское распределения являются основными. Последовательность начинается со стартовой точки или зерна (seed). После генерации конечного, но достаточно большого количества значений цикл повторяется вновь. Это означает, что компьютерный алгоритм создаст такие же значения при таком же стартовом значении зерна. Хорошим генератором случайных чисел является тот, который позволяет создать длинную цепочку чисел без циклических повторений и какой-либо внутренней зависимости. Ключевым критерием выбора генераторов является близость свойств симулированного распределения с целевым распределением при умеренных вычислительных затратах.

\subsection{Генератор псевдослучайных равномерно-распределенных чисел}

Псевдо-случайные равномерно распределенные числа создаются при помощи детерминированной последовательности, которая имитирует статистические свойства последовательности равномерно распределенных случайных чисел. Хороший генератор имеет продолжительный период, распределение близкое к равномерному, а сами значения независимы. Важно иметь  хороший равномерный генератор, поскольку псевдо-случайные числа для практически любого распределения могут быть получены путем преобразования равномерно распределенных псевдо-случайных чисел  (Брэдли и др., 1983, стр. 24).

Стандартный генератор описывается уравнением

\[
X_j=(kX_{j-1}+c) \mod{m},
\]
где операция $a \mod b$ означает остаток от деления $a$ на $b$. Это дает последовательность целых чисел от $0$ до $m$, а равномерно распределенная случайная величина получается по формуле $R_j=X_j/m$ (Рипли, 1987,  стр. 20). Значение $X_0$ называется зерном, и необходимо для инициализации генератора.  Получаемая  последовательность является детерминистической, что позволяет её реплицировать, поскольку при одинаковом  значении зерна получаются одни и те же последовательности. Период цикла зависит от $X_0$, $k$ и $c$. Если расчеты производились на основе 32-разрядной целочисленной арифметики, тогда максимальная длина периода равна $2^{31} \simeq 2.1\times 10^9$. Тем не менее, легко можно выбрать неудачные  значения $X_0,k$ и $c$, при которых длина периода ниже чем $2^{31} \simeq 2.1\times 10^9$. Возможные проблемы обсуждаются в книге Пресс и др. (1993). 

\subsection{Неравномерные случайные величины}

Случайные величины, имеющие разное распределение, в том числе нормальное, как правило получены из симуляций равномерно распределенных случайных чисел. К основным методам относятся: (1) обратное преобразование, (2) преобразование, (3) метод принятия или отбрасывания (4) смешивание или комбинирование.

\begin{center}
Обратное преобразование
\end{center}

Обозначим через $F(x)$ функцию распределения непрерывной случайной величины $x$,

\[
F(x)=\Pr[X\leq x]
\]

При заданных значениях равномерно распределенной случайной величины $r$, $0 \leq r \leq 1$, обратное преобразование
\[
x=F^{-1}(r)
\]
дает единственное значение $x$ поскольку $F$ непрерывная и монотонно возрастающая.

Например, функция распределения показательного распределения с единичной интенсивностью равна $1-e^{-x}$. Решая $r=1-e^{-x}$ относительно $x$  получаем $x=-\ln{(1-r)}$. Если случайная выборка из равномерного распределения $[0,1]$ равна 0.64, тогда $x=-\ln{(1-0.64)}=1.0217$. На Рисунке 12.2 изображен график функции распределения $X$ и графически показано как работает метод обратного преобразования. Вначале случайным образом выбирается точка на вертикальной оси на уровне $r$, а затем --- соответствующее ей  значение на горизонтальной оси так.  

Применение этого метода не вызывает затруднений, если $F(\cdot)$ задана явно и $x$ является непрерывной случайной величиной. Даже если нет аналитической формы для $F(\cdot)$ метод иногда возможно применить, хотя количество расчетов при этом увеличивается, поскольку обратные к  стандартным функциям распределения часто реализованы в виде готовых функций в статистических пакетах.

\vspace{2cm}

График 12.2. Метод обратного преобразования для построения выборки из экспоненциального  распределения с единичной интенсивностью. Равномерная величина равная 0.64 (т.е. $F(x) = 1- \exp(-x) = 0.64$), и, следовательно, $x = 1.02$

Метод может применяться для дискретных случайных переменных с функцией распределения, заданной ступенчато. К примеру, если $x$ принимает целые значения, тогда значение $r=0.312$, взятое из равномерного распределения приводит к тому, что $x=j$, где $j$ выбирается так, что $F(j-1)<0.312$ и $F(j)\geq 0.312$.

Общепринятым методом генерации нормально распределенных случайных величин является метод Бокса-Мюллера. В методе Бокса-Мюллера используется способ обратного преобразования, который применяется для пары независимо и равномерно распределенных величин. А именно, если $r_1$ и $r_2$ независимо и равномерно распределены, тогда $x_1=\sqrt{-2\ln{r_1}}\cos(2\pi{r_2})$ и $x_2=\sqrt{-2ln{r_1}}\sin(2\pi{r_2})$ и имеют стандартное нормальное распределение и независимы, $\mathcal{N}(0,1)$.

\begin{center}
Преобразование
\end{center}

Иногда требуемую плотность распределения возможно получить при помощи некоторого преобразования. 

Метод преобразования является действенным способом генерирования распределений, построенных на основе нормального распределения. Среди примеров преобразований отметим возведение в квадрат стандартно нормально распределенной случайной величины для построения случайной величины с центральным хи-квадрат распределением, суммирование $r$ квадратов независимых стандартных нормально распределенных случайных величин, для получения хи-квадрат распределения с $r$ степенями свободы,  а также расчет среднего значения квадрата независимых хи-квадрат распределенных величин для построения случайных величин с $F$-распределением. Метод преобразования применяется не только для распределений связанных с нормальным. 

\begin{center}
Метод принятия-отбрасывания
\end{center}

Предположим, что необходимо сгенерировать величины с плотностью $f(x)$, и это достаточно трудно сделать, однако существует другая плотность $g(x)$, которая покрывает $f(x)$, т.е. выполняется неравенство $f(x)\leq kg(x)$ для всех значений $x$ для некоторой конечной константы $k$. Это отражено на Рисунке 12.3, где тонкая линия изображает огибающую $kg(x)$.

\vspace{2cm}



График 12.3. Метод принятия-отбрасывания генерирует значения с плотностью $g(x)$, где $kg(x)$ покрывает целевую функцию плотности $f(x)$
%Accept-reject method draws from density $g(x)$ where $kg(x)$ envelopes the desired density $f(x)$.

В основе метода принятия-отбрасывания используется плотность $g(x)$ вместо $f(x)$. Значение $x=r$ принимается, если

\[
r\leq \dfrac{f(x)}{kg(x)},
\]
где $r$, значение равномерно распределенной случайной величины. Если условие не выполняется, тогда значение не принимается и тогда необходимо делать новые симуляции до тех пор, пока условие не будет выполняться. Применение метода зависит от простоты получения значений с плотностью $g(x)$, а не с плотностью $f(x)$. В среднем исходное  значение будет принято с вероятностью $1/k$, при больших значениях $k$ необходимо будет сделать много симуляций. 

Для иллюстрации метода допустим, что $Y$ обозначает случайную величину, сгенерированную методом принятия-отбрасывания, $X$ задает случайную величину с плотностью $g(x)$ и $U$ обозначает равномерную случайную величину. Тогда $Y$ имеет функцию распределения

\[
\Pr[Y\leq y]=Pr[X\leq y|U\leq f(x)/kg(x)]
\]

\[
=\dfrac{\Pr[X\leq y, U\leq f(x)/kg(x)]}{\Pr[U\leq f(x)/kg(x)]}
\]

\[
=\dfrac{\int^{y}_{-\infty}\int^{f(x)/kg(x)}_{0}dug(x)dx}{\int^{\infty}_{-\infty}\int^{f(x)/kg(x)}_{0}dug(x)dx}
\]

\[
=\dfrac{\int^{y}_{-\infty}[f(x)/kg(x)]g(x)dx}{\int^{\infty}_{-\infty}[f(x)/kg(x)]g(x)dx}
\]

\[
=\dfrac{\int^{y}_{-\infty}[f(x)/k]dx}{\int^{\infty}_{-\infty}[f(x)/k]dx}
\]

\[
\int^{y}_{-\infty}f(x)dx,
\]
которая соответствует плотности $f(x)$, как и ожидалось.

\begin{center}
Комбинирование
\end{center}



Иногда плотность $f(x)$ может быть выражена через смесь распределений или составное распределение

\[
f(x)=\int{g(x|\e)h(\e)d\e}.
\]
Тогда значения $f(x)$ могут быть получены в два шага, сначала необходимо сгенирировать $\e$ из плотности $h(\e)$, а затем значения $x$ из условной плотности распределения $g(x|\e)$.

В качестве примера рассмотрим получение значений отрицательного биномиального распределения со средним $\lambda$ и дисперсией $\lambda(1+\alpha\lambda)$, где оба параметра, $\lambda$ и $\alpha$ заданные константы. Можно использовать тот факт, что отрицательное биномиальное распределение может быть рассмотрено как смесь распределений Пуассона и гамма (см. Главу 20). Вначале получим значения $\e$ из гамма-распределения со средним $I$ и дисперсией $\alpha$ через преобразование экспоненциального распределения. Далее рассчитываются значения $\lambda\e$ при известных, из предыдущего пункта, значениях $\e$.

Если $h(\e)$ дискретно заданная величина с вероятностями значений $p_j$ отличными от нуля в $C$ точках, $j=1,\ldots ,C$, тогда интегрирование заменяется суммированием. Таким образом,

\[
f(x)=\sum^{C}_{j=1}p_{j}g(x|\e=\e_{j}).
\]
При этом, для получения выборки размера $S$ с плотностью $f(x)$, сгенерируем $Sp_j$ значений из условной плотности $g(x|\e=\e_j)$ и построим выборку размером $S$ путем объединения выборок.

\begin{center}
Некоторые стандартные генераторы
\end{center}

Таблица в Приложении B описывает процесс генерации псевдо-случайных чисел для некоторых стандартных непрерывных и дискретных случаев. В основе лежит предположение, что $r,r_1,r_2,\ldots $ независимо равномерно распределенные случайные значения $R,R_1,R_2,\ldots $ с параметрами $[0,1]$. Следует отметить, что существуют другие методы создания случайных переменных, перечислим один или два из них.

\subsection{Многомерное распределение}

Получение значений многомерного распределения сложнее, чем одномерного. Например, методы обратного преобразования и преобразования могут быть неприменимы. Для многомерных распределений возможно использовать метод смеси и комбинирования распределений, поскольку среди многомерных распределений часто встречаются смеси распределений.

Среди общепринятых методов можно выделить алгоритм Гиббса и иные алгоритмы Монте-Карло по схеме марковской цепи. Эти методы будут рассмотрены позже в Разделе 13.5, поскольку они широко применимы при байесовском анализе, в котором используются сложные многомерные распределения. Как будет объяснено далее, при использовании алгоритма Гиббса может возникнуть корреляция между выборочными значениями, что может привести к понижению эффективности симуляционного моделирования.

Ограничимся рассмотрением многомерного нормального распределения. Значения могут быть получены путем преобразования значений, имеющих одномерное стандартное нормальное распределение. Предположим, что необходимо создать $q$-мерное нормальное распределение, так что $x\sim \mathcal{N}(0,\Sigma)$. Это возможно с помощью преобразования, использующего разложение Холецкого для  положительно определенная матрица $\Sigma$ 

\[
\Sigma=LL', 
\]
где $L$ нижнетреугольная матрица. Например, для $q=2$ разложение Холецкого можно представить как

\[
\begin{bmatrix} \sigma_{11} & \sigma_{12} \\ \sigma_{21} & \sigma_{22} \end{bmatrix} = 
\begin{bmatrix} l_{11} & 0 \\ l_{21} & l_{22} \end{bmatrix}
\begin{bmatrix} l_{11} & l_{21} \\ 0 & l_{22} \end{bmatrix},
\]
которое дает три уравнения $l^{2}_{11}=\sigma_{11},l_{11}l_{21}=\sigma_{12}$ и $l^{2}_{21}+l^{2}_{22}=\sigma_{22}$, в результатом решения получим значения $l_{11},l_{21}$ и $l_{22}$. Для заданного $q$-мерного вектора $\e$, элементы которого стандартно нормально распределены, легко проверить, что из $\e \sim \mathcal{N}(0,I)$ следует $x=L\e$, а линейная комбинация нормально распределенных величин имеет распределение с параметрами $\mathcal{N}(0,\sum)$. В частности, $\E[L\e]=0$ и $\V[L\e]=\E[L\e\e'L']=LL'=\sum$. В основе этого метода лежит свойство, что линейная комбинация нормально распределенных величин является нормально распределенной величиной. Это свойство не выполняется для величин, имеющих распределение отличное от нормального.

\section{Библиографические заметки}

Пресс и другие (1993) приводят хорошее введение в численное и Монте-Карло интегрирование и указывают дальнейшие ссылки, включая некоторые, которые есть в этой главе.

Литература по эконометрике, посвященная оцениванию с помощью симуляций, уделяет особое внимание пробит-модели. Однако методы имеют более широкое применение и могут быть легко и успешно применены в других моделях, которые легче оценить, чем  мультиномиальный пробит. Лерман и Мански (1981) использовали симуляционные частоты для оценивания вероятностей и выяснили, что необходима большая выборка. МакФадден (1989) ввел метод симуляционных моментов и показал, что этот метод является состоятельным и асимптотически нормальным. Пэйкс и Поллард (1989) привели довольно подробное описание асимптотической теории и для метода симуляционных моментов, и для симуляционного правдоподобия. Написанная доступным языком работа Стерна (1997) --- отличная отправная точка. Гурьеру и Монфорт (1996) приводят подробное описание основных методов. Другую литературу на данную тему хорошо изучать в контексте моделей, которые будут рассмотрены с последующих главах. В частности, Хаживассилиу и Рууд (1994) уделяют особое внимание усеченным моделям с нормальным распределением, включая мультиномиальный пробит. Трейн (2003) рассматривает модели дискретного выбора, включая логит со случайными параметрами. 


\begin{center}
Упражнения
\end{center}


\begin{enumerate}
\item [$12 --- 1$] Для оценки интеграла $I=\int{t(x)g(x)dx}$ методом Монте-Карло используется сумма $\hat{I}=N^{-1}\sum{t(x_i)g(x_i)/p(x_i)}$, где $x_i$ значения, сгенерированные с помощью сэмплирования по важности с плотностью $p(x)$. Покажите, что предел по вероятности $\plim \hat{I}=I$.

\item [$12 --- 2$]  Для $f(\theta)=|\sum|^{-1/2}[1+\dfrac{1}{\nu}(\theta-\mu)^{'}\sum^{-1}(\theta-\mu)]^{-(\nu+d)/2}$, рассмотрим интеграл размерности $d$ $\int_{R^{d}}f(\theta)d\theta$. Подынтегральная функция является основой многомерного $t$-распределения, таким образом правильным ответом является обратное к нормирующей постоянной. 
\begin{enumerate}
\item Оцените этот интеграл как среднее по методу Монте-Карло $S^{-1}\sum^{S}_{s=1}f(\theta^{(s)})/h(\theta^{(s)})$, $\theta^{(s)}\sim h(\theta)$, где плотность важности $h(\theta)$ является многомерным $t$-распределением с тем же центром и масштабирующим множителем как и функция $f(\theta)$, но отличается количеством степеней свободы.
\item Оцените стабильность полученного среднего меняя количество степеней свободы $h(\theta)$. Также оцените стабильность среднего увеличив несоответствие между $f(\theta)$ и $h(\theta)$, меняя центр и коэффициент масштаба $h(\theta)$.
\end{enumerate}

\item [$12 --- 3$] Для оценки метода симуляционных моментов, рассмотренной в Разделе 12.5.3 предположите, что вспомогательная оценка является частотной.
\begin{enumerate}
\item Покажите, что $\V_{y,u}[\hat{m}(\theta_0)]=(1+1/S)\V_{y}[m(\theta_0)]$.
\item Далее покажите, что  использования частотной вспомогательной оценки приводит к увеличению дисперсии оценки методом моментом в $(1+(1/S))$ раз.
\item Насколько будет высока потеря эффективности для стандартных ошибок, если $S=10$?
\end{enumerate}

\item [$12 --- 4$]  Обратимся к примеру, рассмотренному в Разделе 12.5.6, рассчитайте оценку $\hat{\alpha}$, которая является решением уравнения $\sum^{N}_{i=1}[y_i-\dfrac{1}{S}\sum^{S}_{s=1}(\alpha+u^{S}_i)]=0$. Получите аналитическое выражение для $\alpha$ оценки и ее дисперсии.

\item [$12 --- 5$] 
\begin{enumerate}
\item Напишите алгоритм для построения псевдо-случайной выборки из трехмерного  нормального распределения $\mathcal{N}[0,\sum]$ c $\sigma_{jj}=1, j=1,2.3$ и ковариациями $\sigma_{12}=\sigma_{13}=\sigma_{23}=0.5$. Постройте выборку из 1,000 реализаций и сравните  оценки среднего значения и дисперсий с теоретическими.
\item Повторите пункт (а), но вместо трехмерного нормального распределения используйте $t$-распределение Стьюдента с пятью степенями свободы.
\end{enumerate}

\item [$12 --- 6$] Напишите функцию для генерирования одномерного усеченного нормального распределения $\tau\mathcal{N}_{[a,b]}[\mu,\sigma^2]$ с применением метода обратного преобразования, рассмотренного в Разделе 12.8.2. Здесь $[a,b]$ нижняя и верхняя точки усечения. Используйте значения $\mu=1$, $\sigma^2=4$ и $a=3$, $b=4$.

\item [$12 --- 7$] Рассмотрим модель бинарной логит регрессии (см. Раздел 14.3).
\begin{enumerate}
\item Запишите логарифмическую функцию правдоподобия.
\item Предположим, что свободный член задан случайно и его значение получается из соответствующего распределения с конечным средним и дисперсией. Как вы можете обосновать  использование ненаблюдаемой гетерогенности в этом случае? Если логит-модель составлена на основе случайной  модели полезности с экстремальным распределением ошибок, как случайный характер свободного члена может повлиять на интерпретацию или расчет результатов? [см. Ревелт и Трейн, 1998.]
\item Предложите подходящее распределение для свободного члена; перепишите функцию правдоподобия с учетом ненаблюдаемой гетерогенности. Далее запишите условную функцию правдоподобия при фиксированной ненаблюдаемой гетерогенности.
\item Опишите пошагово процедуру оценивания модели с использованием симуляционного  правдоподобия. Детально объясните как рассчитать матрицу дисперсий неизвестных параметров. Как вы будете определять количество необходимых симуляций?
\item Рассмотрите возможность применения метода симуляционных моментов как альтернативу оценки симуляционного правдоподобия для логит-модели со случайными параметрами. Запишите условия для моментов при заданной ненаблюдаемой гетерогенности. Далее опишите процедуру расчета оценки симуляционных моментов для данной модели.
\end{enumerate}


\item [$12 --- 8$]
Некоторые статистические пакеты позволяют построить как пуассоновское, так и гамма- распределение псевдослучайных чисел. Также известно, что отрицательное биномиальное распределение можно построить как смесь значений пуассоновского и гамма-распределений (см. Раздел 20.4).
\begin{enumerate}
\item Запишите процедуру получения чисел, имеющих отрицательное биномиальное распределение методом смеси. 
\item Используйте выбранный вами метод для построения выборки, состоящей из 10,000 значений, которые имеют пуассоновское распределение со средним 0.25.
\item Постройте соответствующую выборку из гамма распределения со средним 1 и дисперсией $\alpha$, где $\alpha$ задана так, чтобы получить величины с отрицательным биномиальным распределением и дисперсией 0.3125.
\end{enumerate}
\end{enumerate}




\chapter {Байесовские методы}


\section{Введение}

Эта глава является введением в байесовскую эконометрику. Интерес к анализу байесовских регрессий фантастически вырос с момента публикации трудов Зеллнера (1971) и Лимера (1978). Широкое распространение получило использование байесовского подхода при рутинном анализе данных.  Оно в значительной степени было обусловлено развитием компьютерной техники и разработкой специализированного программного обеспечения. Ввиду столь широкого распространения байесовского  подхода в одной главе невозможно рассмотреть все стороны вопроса. Цель настоящей главы --- дать общее представление об основных идеях и приложениях байесовской эконометрики. Не смотря на скромность поставленных целей, некоторые разделы содержат много технических деталей. 

В отличие от метода правдоподобия или частотного или классического подхода к оценке вероятностей, рассмотренного ранее, для применения байесовского подхода требуется определение априорных вероятностей для неизвестных параметров модели, помимо спецификации самой модели. Это требование, как с теоретической, так и с практической сторон является причиной затруднений для многих исследователей. Существенность определения априорного мнения дало повод многим считать байесовский подход субъективным. Однако, как будет обсуждаться в этой главе, на больших выборках значимость априорного мнения незначительна, можно указать довольно слабые априорные ограничения на параметры, и существуют специальные методы, позволяющие  определить чувствительность выводов к априорным предпосылкам. Из этого можно заключить, что субъективность определения априорных предпосылок преувеличена.

Байесовские подходы имеют большой потенциал использования в прикладной микроэконометрике, в частности, для оценивания параметров сложных моделей, для которых зачастую отсутствует аналитическая форма функции правдоподобия. В главе 12 обсуждаются методы симуляционного моделирования, которые могут применяться в подобной ситуации. Однако, применение методов  симуляицонного моделирования, в особенности метода максимального правдоподобия, может вызвать затруднения, поскольку для максимизации функции требуется достаточно большое количество симуляций, растущее с увеличением размера выборки. Даже при современных возможностях вычислительной техники анализ больших выборок и моделей высокой размерности могут потребовать значительных расчетов. Байесовские методы, напротив, не требуют применения методов максимизации. Эти подходы оценивания могут заменить метод максимального правдоподобия и полученные оценки во многих случаях будут отличной, если не идеальной, заменой. В действительности, выбор байесовских методов не требует смены философии, эти методы могут быть использованы из прагматических соображений.

Вышеприведенные аргументы не означают, что байесовские методы не имеют глубокого теоретического обоснования. Обоснования есть. Следует отметить, по меньшей мере, три особенности. Во-первых, байесовские методы позволяют оценить апостериорное распределение исследуемых параметров целиком, оставив на усмотрение исследователя выбор момента или квантили распределения для представления в отчёте, как правило, этот выбор осуществляется согласно цели исследования. Не нужно формировать отдельные оценки для среднего, медианы, квантилей и т.д., поскольку в апостериорном распределении они есть сразу все! Во-вторых, байесовский подход, учитывающий характеристики исходных данных, дает точные результаты для конечной выборки, тем самым снимая  необходимость корректировок или уточнений. Распределение полученных оценок сходится к нормальному на больших выборках, где влияние априорных предпосылок снижается. В-третьих, байесовские методы позволяют естественным способом выбрать модель.

В разделе 13.2 описаны  основные идеи и положения байесовского анализа, а также основные свойства байесовских оценок. Основные идеи проиллюстрированы в разделе 13.3 с помощью относительно просто интерпретируемых регрессионных моделей. В общем случае для апостериорного распределения не существует аналитического решения. В разделе 13.4 представлены методы Монте-Карло, а именно метод сэмплирования по важности, который применяется для получения численных оценок моментов апостериорного распределения. В разделе 13.5 обсуждается использование марковских цепей для реализации метода Монте-Карло, а именно алгоритм Гиббса и алгоритм Метрополиса-Хастингса, используемые для генерирования случайных значений из сложного апостериорного распределения. Примеры этих способов рассмотрены в разделе 13.6. Дополнительные темы о пополнении данных и байесовского выбора между моделями представлены в разделах 13.7 и 13.8.

\section{Байесовский подход}

В байесовском подходе неопределенность относительно значения параметра $\theta$ моделируется в явном виде путём введением плотности $\pi(\theta)$ априорного распределения, названного так, поскольку спецификация априорного распределения не учитывает имеющиеся данные. Априорное распределение основано на субъективных предположениях, выраженных в терминах вероятности. Спецификация распределения подробно рассмотрена в разделе 13.2.4. Например, предположим, что $\theta$ эластичность дохода, причем на основе экономической модели или  предыдущих исследований допустим, что $\theta$ лежит в промежутке от 0.8 до 1.2 с вероятностью 0.95. Тогда разумным априорным распределением $\theta$ может быть $\theta{\sim}N[1,0.1^{2}]$.

Другой компонент байесовского оценивания ---  функция совместной плотности выборки или функция правдоподобия $f(y|\theta)$. Если система состоит только из одного уравнения, тогда $y$ это вектор $N{\times}1$. Для простоты обозначений зависимость плотности от независимых переменных явно не прописывается. Экзогенные переменные введены в рассмотрение в разделе 13.3, тогда плотность $f(y|\theta)$ заменяется на $f(y|X,\theta)$ и байесовский анализ проводится при фиксированных независимых переменных. Также следует отметить, что в этой главе обозначение $f(\cdot)$ используется для совместной плотности для всех наблюдений, а не плотности $i$-го наблюдения.

Если данных нет, тогда единственное что мы имеем --- это априорное мнение. После того как данные получены, при классическом подходе делается оценка параметра $\theta$ методом максимального правдоподобия. В байесовском подходе анализ правдоподобия выборки объединяется с анализом доступной априорной информации, даже если такая информация представлена в виде распределения вероятностей. В целом подход может рассматриваться как пересмотр априорных предпосылок с помощью полученных данных (функции правдоподобия). В частности, распределение неизвестного параметра $\theta$ может быть получено комбинированием априорной информации и метода максимального правдоподобия. Полученное распределение называется апостериорным, так как оно отражает представления о исследуемом параметре апостериори, то есть после анализа данных.

\subsection{Теорема Байеса}

Основным результатом, с помощью которого формируется апостериорное распределение, является теорема Байеса, согласно которой

\begin{equation}
f(\theta|y)=\dfrac{f(y|\theta)\pi{(\theta)}}{f(y)},
\end{equation}

где $f(y)$   частная плотность распределения вероятностей $y$, определяемая следующим уравнением

\begin{equation}
f(y)=\int R(\theta)f(y|\theta)\pi{(\theta)}d(\theta),
\end{equation}

где $R(\theta)$ обозначает носитель функции $\pi(\theta)$. Этот результат получается из выражения для условной вероятности события $A$ при условии $B$

\[
\Pr[A|B]=\dfrac{\Pr[A{\cap}B]}{\Pr[B]}
\]

\[
\dfrac{\Pr[B|A]\Pr[A]}{\Pr[B]},
\]

где второе равенство следует из $\Pr[B|A]=\Pr[A{\cap}B]/\Pr[A]$.

Поскольку знаменатель $f(y)$ в (13.1) не зависит от $\theta$, можно представить $p(\theta|y)$ как произведение плотности  и априорного распределения; тогда

\begin{equation}
p(\theta|y){\propto}L(y|\theta)\pi(\theta).
\end{equation}

Полученное соотношение позволяет опустить несущественные константы в формуле апостериорной вероятности, тем самым позволяя упростить вывод и представление. Опущенные константы  могут быть восстановлены позже, это будет показано в разделе 13.2.2. Когда функция плотности записывается без нормализирующих констант, ее называют ядром плотности. 

Во многих случаях использование формул (13.1) и (13.3) не позволяет получить аналитическое выражение для плотности апостериорного распределения. Однако, аналитическое выражение для плотности необязательно и  далее будут рассмотрены примеры применения симуляционного моделирования для расчета приближенного значения апостериорной плотности. Этот  метод позволяет использовать байесовский анализ практически для любых приложений параметрической микроэконометрики.

Обычно для обозначения апостериорной плотности используются специальные обозначения, $f(\theta|y)$ заменяется на $p(\theta|y)$. При этом обычная совместная плотность распределения $f(y|\theta)$ обозначается как функция максимального правдоподобия $L(y|\theta)$. Далее будем записывать апостериорную плотность как

\begin{equation}
p(\theta|y){\propto}L(y|\theta)\pi(\theta).
\end{equation}

Такое представление плотности является одним из основных для байесовского подхода и подчеркивает значимое отличие между частотным и байесовским подходами. Согласно частотному подходу, истинное значение параметра является постоянным, при этом оценки значения параметра рассматриваются как случайные величины. В  байесовском подходе, напротив, параметры также считаются случайными величинами.

\subsection{Пример использования теоремы Байеса}

Предположим, что $y{\sim}N[\theta,\sigma^{2}]$, где дисперсия $\sigma^{2}$ известна, а скалярный параметр $\theta$ неизвестен. Для случайной выборки $y_1,\ldots ,y_N$, совместная плотность $y$ определяется выражением

\[
L(y|\theta)=\Pi^{N}_{i=1}(2\pi\sigma^{2})^{-1/2}\exp\left\lbrace -(y_i-\theta)^{2}/2\sigma^{2}\right\rbrace 
\]

\[
=(2\pi\sigma^{2})^{-N/2}\exp\left\lbrace-\sum^{N}_{i=1}(y_i-\theta)^{2}/2\sigma^{2}\right\rbrace
\]

\[
\propto{\exp}\left\lbrace-\frac{N}{2\sigma^{2}}(\overline{y}-\theta)^{2}\right\rbrace,
\]

где $\overline{y}=N^{-1}\sum_{i}y_i$, и $\sum_{i}(y_i-\theta)^{2}=\sum_{i}(y_i-\overline{y}+\overline{y}-\theta)^{2}=\sum_{i}(y_i-\overline{y})^{2}+\sum_{i}(\overline{y}-\theta)^{2}$. Множители, не содержащие параметр $\theta$, входят в константу и отбрасываются. В рамках частотного подхода мы максимизируем  логарифмическую функцию правдоподобия по параметру $\theta$, в результате определяется  оценка параметра $\hat{\theta}=\overline{y}$.

В байесовском подходе дополнительно описывается априорное распределение параметра $\theta$. Для построения аналитических выкладок удобно использовать нормальное априорное распределение $\theta{\sim}N[\mu,\tau^{2}]$, где значения математического ожидания $\mu$ и дисперсии $\tau^{2}$ полагаются известными. Большие значения дисперсии $\tau^{2}$ свидетельствуют о значительной априорной неопределенности. Тогда априорную плотность можно записать как

\[
\pi(\theta)=(2\pi\tau^{2})^{-1/2}\exp\left\lbrace-(\theta-\mu)^{2}/2\tau^{2}\right\lbrace
\]

\[
\propto{\exp}\left\lbrace-(\theta-\mu)^{2}/2\tau^{2}\right\rbrace,
\]

где $(2\pi\tau^{2})^{-1/2}$ не зависит от $\theta$ и включается в коэффициент пропорциональности. Используя (13.4), получим апостериорную плотность, равную

\[
p(\theta|y)=\frac{L(y|\theta)\pi(\theta)}{\int^{\infty}_{-\infty}L(y|\theta)\pi(\theta)d\theta}, -\infty<\theta<\infty.
\]

Знаменатель гарантирует, что апостериорное распределение корректно (т.е., интеграл от плотности равен единице). Для большинства задач знаменатель можно опустить, в этом случае мы работаем с $p(\theta|y)\propto{L(y|\theta)\pi(\theta)}$. Числитель (13.5) может быть записан, как

\[
L(y|\theta)\pi(\theta)
\]

\[
=(2\pi{\sigma}^{2})^{-N/2}\exp{\left\lbrace-\frac{N}{2{\sigma}^{2}}{(\overline{y}-\theta)}^{2}\right\rbrace}{(2\pi{\tau}^{2})}^{-1/2}\exp\left\lbrace-\dfrac{(\theta-\mu)^{2}}{2\tau^{2}}\right\rbrace
\]

\[
=(2\pi)^{(N+1)/2}{(\sigma)^{2}}^{-N/2}(\tau^{2})^{-1/2}\exp\left\lbrace-\dfrac{1}{2\sigma^{2}}\sum^{N}_{i=1}(y_i-\theta)^{2}-\dfrac{(\theta-\mu)^{2}}{2\tau^{2}}\right\rbrace.
\]

Поскольку 

\[
\sum^{N}_{i=1}(y_i-\theta)^{2}=\sum^{N}_{i=1}(y_i-\overline{y})^{2}+N(\overline{y}-\theta)^{2},
\]

а константа интегрирования в (13.5) и другие константы не зависящие от $\theta$ могут быть включены в коэффициент пропорциональности, получим

\begin{equation}
p(\theta|y)\propto{\exp}\left\lbrace-\dfrac{N}{2\sigma^{2}}(\theta-\overline{y})^{2}\right\rbrace{\exp}\left\lbrace-\dfrac{1}{2}\dfrac{(\theta-\mu)^{2}}{\tau^{2}}\right\rbrace
\end{equation}

\[
\propto\lbrace-\dfrac{1}{2}\left[\dfrac{(\theta-\mu)^{2}}{\tau^{2}}+\dfrac{(\overline{y}-\theta)^{2}}{N^{-1}\sigma^{2}}\right] \rbrace
\]

\begin{equation}
\propto{\exp}\lbrace-\dfrac{1}{2}\left[\dfrac{(\theta-\mu_{1})^{2}}{\tau^{2}_{1}}\right]\rbrace.
\end{equation}

Последнее выражение равно ядру $N[\mu_1,\tau^{2}_1]$ распределения, где 

\begin{equation}
\mu_1=\tau^{2}_{1}(N\overline{y}/\sigma^{2}+\mu/\tau^{2}),
\end{equation}

\[
\tau^{2}_1=(N/\sigma^{2}+1/\tau^{2})^{-1}.
\]

Последнее соотношение в (13.7) получено дополнением до полного квадрата, при этом использовалось тождество верное для произвольных скаляров $z$, $y$, $a_1$, $a_2$, $c_1$ и $c_2$:

\[
c_{1}(z-a_1)^{2}+c_{2}(z-a_2)^{2}=(c_1+c_2)\left(z-\left( \dfrac{c_{1}a_1+c_{2}a_2}{(c_1+c_2)^{2}}\right)\right)^{2}+\dfrac{c_{1}c_{2}}{(c_1+c_2)}(a_{1}-a_2)^{2}, 
\]

где $z=\theta, a_1=\mu, a_2=\overline{y}, c_1=1/\tau^2$, и $c_2=1/(N^{-1}\sigma^{2}+\tau^{2})$. Слагаемые, которые не зависят от $\theta$, отбрасываются.

В итоге получим следующие результаты:

Данные: $y|\theta{\sim}N[\theta,\sigma^{2}]$, где значение $\sigma^{2}$ известно.

Априорное распределение: $\theta{\sim}N[\mu,\tau^2]$, $\mu$, $\tau^2$ определено.

Апостериорное распределение: $\theta|y{\sim}N[\mu_1,\tau^{2}_{1}]$, $\mu_1$, $\tau^{2}_1$ задаются соотношениями (13.8).

Среднее значение апостериорного распределения, $\mu_1$, равно взвешенной сумме среднего априорного распределения $\mu$ и выборочного среднего $\overline{y}$, где веса учитывают точность правдоподобия через $\sigma^{2}/N$ и априорного распределения через $\tau^{2}$. Применение байесовского подхода позволяет определить общую дисперсию на основе  параметра точности, значение которого обратно дисперсии. В вышеприведенных выражениях значение параметра точности апостериорного распределения $\tau^{-2}_1$ равно сумме значений параметра точности для выборочного среднего,$\overline{y}$, $N/\sigma^{2}$ и априорного параметра точности $1/\tau^{2}$, таким образом точность будет возрастать при объединении информации из выборки и априорной информации. 

Если априорная информация является неточной, т.е. значение $1/\tau^{2}$ мало, тогда вес среднего априорного распределения также будет мал по отношению к выборочной информации; и априорное распределение играет незначительную роль при определении апостериорного распределения. Аналогично, при большом размере выборки, выборочная информация играет главную роль, поскольку рост размера выборки приводит к росту $N/\sigma^{2}$ относительно $1/\tau^{2}$. Апостериорное распределение сходится к известному асимптотическому нормальному распределению, если не считать, что байесовский подход приводит к трактовке $\theta{sim^{a}N[\overline{y},\sigma^{2}/N]}$, а не $\overline{y}{sim}^{a}N[\theta,\sigma^{2}/N]$.


Рисунок 13.1


Приведем конкретный пример, предположим, что $\sigma^{2}=100$, параметры априорного распределения $\mu=5$ и $\tau^{2}=3$, размер выборки равен $N=50$ и среднее значение $\overline{y}=10$. Тогда параметры функции правдоподобия --- $N[10,2]$, априорного распределения --- $N[5,3]$ и согласно (13.7) и (13.8) апостериорного распределения --- $N[8,12]$. Графики функций плотности изображены на Рисунке 13.1. Среднее значение апостериорного распределения лежит между средним априорного распределения и выборочного среднего, в то время как дисперсия апостериорного  меньше, чем у априорного распределения и функции правдоподобия. 

\subsection{Сравнение байесовского и небайесовского подходов}

Полезно рассмотреть сходства и различия частотного и байесовского подходов.

В параметрическом частотном подходе все статистические выводы основываются на функции правдоподобия. При выполнении соответствующих условий регулярности оценка максимального правдоподобия состоятельна и асимптотически нормальна. Теория выборочного оценивания предоставляет основу для построения вероятностных утверждений об оцениваемых величинах, их  функциях, или условных прогнозах. Априорная информация о параметрах может быть учтена посредством использования ограниченной оценки правдоподобия.

В байесовском анализе, результаты которого представлены в Таблице 13.1, процесс порождающий данные и сами данные  объединены с априорным распределением параметров. Спецификация априорного распределения детально рассмотрена в Разделе 13.2.4. Априорное распределение включает сформулированную в терминах вероятностей информацию доступную до получения исходных данных. Также априорное  распределение может быть основано на <<полученной информации>>. Априорная информация и данные объединяются с помощью теоремы Байеса. 

В результате мы получим апостериорное распределение параметров $\theta$, которое можно рассматривать как преобразованную функцию максимального правдоподобия. Иначе, при наличии данных, при помощи апостериорного распределения возможно отобразить <<исправленное априорное распределение>>. Если выборка малого размера, и, возможно, относительно неинформативна, апостериорное распределение может выглядеть как априорное, однако при больших размерах выборки апостериорное распределение будет отражать особенности данных.


\begin{tabular}{p{6cm}p{8cm}}
Компонента & Формула \\ 
\hline
Выборочная модель & $(y_1,\ldots ,y_N)$ независимы с плотностью $f(y|\theta)$ \\  
Совместная плотность/ функция правдоподобия & $\pi(\theta)$, $L(y|\theta)$; $\theta{\epsilon}\Theta$ \\ 
Априорное распределение & $\pi(\theta), \theta{\epsilon}\Theta$ \\ 
Апостериорное распределение & $p(\theta|y):\begin{cases} 
=f(y|\theta)\pi(\theta)/\int f(y|\theta)\pi(\theta)\,d\theta \\
\propto f(y|\theta)\pi(\theta) \\
\propto L(y|\theta)\pi(\theta)
\end{cases}$ \\ 
Апостериорная плотность распределения$\rightarrow$ апостериорные результаты$\rightarrow$ & оценка параметров, вероятностные утверждения, прогнозирование, сравнение моделей
\end{tabular} 



\subsection{Спецификация априорного распределения}

Байесовский анализ требует спецификации процесса порождающего данные $f(y|\theta)$ и априорной информации $\pi(\theta)$. Процесс порождающий данные, как правило, определяется аналогично случаю полностью параметрического анализа на основе функции правдоподобия. Для анализа бинарных данных применяются логит и пробит-модели, для дискретных данных --- пуассоновская или отрицательная биномиальная модель и т.д.

Принципиальный отличие байесовского анализа, по сравнению с классическим подходом, заключается в необходимости построения априорного распределения. Результаты могут меняться в зависимости от выбора априорного распределения, поскольку разные априорные распределения дают разные апостериорные распределения за исключением случаев, когда объем выборки достаточен для обеспечения преобладания  выборочной информации.

Один из подходов определения исходных предпосылок состоит в выборе такого априорного распределения, которое будет незначительно влиять на формирование апостериорного распределения, тогда результаты будут зависеть в основном от выборочных данных. При наличии достоверной априорной информации предпочтителен  альтернативный подход, согласно которому априорное распределение будет отражать имеющуюся информацию. Ранее применение обоих подходов, в особенности последнего, ограничивалось проблемой  получения апостериорного распределения, на данный момент эта проблема частично решена благодаря развитию вычислительных алгоритмов. Популярный комбинированный подход построен на использовании иерархически выстроенной априорной информации, где неопределенность параметров выражена в терминах функций вероятности, которые, в свою очередь, зависят от других неизвестных параметров. 

\subsubsection*{Неинформативные предпосылки об априорном распределении}

Неинформативные априорные предположения это те, которые не оказывают существенного влияния на итоговое апостериорное распределение.

Простой способ построить априорное распределение, отражающее скудость априорного знание --- использовать равномерное априорное распределение с $\pi(\theta)=c$ для всех значений параметра $\theta$, где $c>0$ постоянно, в этом случае веса для всевозможных значений $\theta$ будут равны между собой.

Одно из ограничений использования равномерного априорного распределения возникает в задачах, где на значение параметров $\theta$ не накладываются ограничения, в этом случае использование равномерного априорного распределения дает неправильные значения плотности, поскольку $\pi(\theta)d\theta=\infty$. При этом, итоговое апостериорное распределение возможно также будет неверным, тем не менее в отдельных случаях получаемое апостериорное распределение является собственным.

Другим недостатком равномерного априорного распределения является неинвариантность к замене параметров. Например, для скалярного параметра $\theta>0$ альтернативной параметризацией плотности $y$ будет запись через параметр $\gamma=\ln {\theta}$, поскольку $-\infty<\gamma<\infty$. Если априорное распределение $\theta$ является равномерным, тогда $\pi(\theta)=c$, однако соответствующее априорное распределение $\pi^{*}(\gamma)$ для $\gamma$ не является равномерным, поскольку $\pi^{*}(\gamma)=\pi(\theta)|d\theta/d\gamma|=ce^{\gamma}$. Таким образом, неинформативное априорное распределение для одной параметризации, может нести информацию в другой параметризации.

Равномерное априорное распределение может быть выражена распределением с очень высокой дисперсией. Например, предположим, что скаляр $\theta$ имеет априорное распределение $N[\mu,\tau^{2}]$, где $\tau^{2}$ очень велико. Тогда для значений $\theta$ правдоподобных при имеющихся данных, значение априорной плотности будет равно $\pi(\theta){\simeq}1/(2\pi\tau^{2})$, т.е. примерно константе, поскольку $\exp[-(\theta-\mu)/2\tau^{2}]{\simeq}1$. Следует отметить, что этот подход, называемый неопределенной, размытой или плоской априорной информацией, имеет такой же недостаток, как равномерное априорное распределение, а именно неинвариантность к замене параметров.

Вместо этого подхода широкое применение среди неинформативных априорных предположений получила априорная вероятность Джеффри (Jeffreys' prior), определяемая из соотношения
\begin{equation}
\pi(\theta)\propto|I(\theta)|^{1/2},
\end{equation}

где для вектора $\theta$, $|I(\theta)|$ является определителем информационной матрицы $I(\theta)=-E[\partial^{2}L/\partial\theta\partial\theta']$ с $L=\ln {L(y|\theta)}$. Априорная вероятность Джеффри, названная в честь пионера байесовского подхода Гарольда Джеффри, обладает свойством инвариантности к замене параметров или к преобразованию параметров модели, так что в независимости от выбранной параметризации модели будет сформирована одна и та же априорная информация.

С целью проверки правила Джеффри рассмотрим простой случай со скалярным параметром. Допустим, что задано преобразование $\gamma=h(\theta), \partial{L}/\partial\gamma=\partial{L}/\partial\theta{\times}\partial\theta/\partial\gamma$ и

\[
\dfrac{\partial^{2}L}{\partial\gamma^{2}}=\dfrac{\partial^{2}L}{\partial\theta^{2}}\left( \dfrac{\partial\theta}{\partial\gamma}\right)^{2}+\dfrac{\partial{L}}{\partial\theta}\dfrac{\partial^{2}\theta}{\partial\gamma^{2}}
\]

Взяв математическое ожидание логарифма функции правдоподобия используя функцию плотности выборки, и заметив, что $\E[\partial{L}/\partial\theta]=0$, в соответствии со свойством скор-функции, имеем

\[
I(\gamma)=I(\theta)\left(\dfrac{\partial\theta}{\partial\gamma}\right)^{2}.
\]

Следовательно,

\[
|I(\gamma)|^{1/2}=|I(\theta)|^{1/2}\mid\dfrac{\partial\theta}{\partial\gamma}\mid.
\]

В общем случае априорное плотность $\pi(\theta)$ для $\theta$ приводит к априорной плотности $\pi^{*}(\gamma)=\pi(\theta){\times}|d\theta/d\gamma|$. Применяя (13.9), получим $\pi^{*}(\gamma)\propto|I(\theta)|^{1/2}{\times}|d\theta/d\gamma|$ и это  равно $|I(\gamma)|^{1/2}$, что и требовалось доказать.

В качестве примера предположим $y{\sim}N[\mu,\sigma^{2}]$  рассмотрим три случая. В первом случае предположим, что $\mu$ неизвестно, а значение $\sigma^{2}$ известно, тогда мера информативности для $\mu$ равна $I(\mu)=N/\sigma^{2}$, и значение априорная вероятность Джеффри $|I(\gamma)|^{1/2}{\propto}c$ равно константе, поскольку в данном случае $\sigma^{2}$ известна. Заметим, что полученное значение не является собственным. Второй случай, когда $\sigma^{2}$ неизвестно, а значение $\mu$ дано, информационная мера для $\sigma^{2}$ равна $I(\sigma^{2})=N/(2\sigma^{4})$ и априорная вероятность Джеффри $|I(\sigma^{2})|^{1/2}\propto\sigma^{2}$. Третья ситуация, когда оба параметра $\mu$ и $\sigma^{2}$ неизвестны, информационная матрица $|I{(\mu,\sigma^{2})}|=(N/\sigma^{2})(N/2\sigma^{4})=N^{2}/2\sigma^{6}$. Тогда, совместная плотность по правилу Джеффри равна $\pi(\mu,\sigma^{2})\propto\sigma^{-3}$. Заметим, что полученное значение отлично от того, которое можно  получить при применении правила Джеффри  отдельно для $\mu$ и $\sigma^{2}$, поскольку $\pi(\mu)\propto{c}$ и $\pi(\sigma^{2})\propto\sigma^{-2}$ дает $\pi(\mu)\pi(\sigma^{2})\propto\sigma^{-2}$. 

Правило Джеффри  можно рассматривать как метод получения априорного распределения, когда отсутствуют другие способы. Однако, в литературе до сих пор нет согласия, дает ли правило Джеффри неинформативные априорные предположения и в каком смысле неинформативные. Вместе с тем, как следует из рассмотренного выше примера, априорная вероятности Джеффри может быть несобственной, и ее использование может дать несобственное апостериорное распределение.

\subsubsection*{Сопряженное априорное распределение}

Когда указывается собственное априорное распределение, которое может быть как информативным, так и размытым, удобно выбрать функциональную форму априорного распределения так, чтобы аналитическое выражение для апостериорной плотности с учетом данных было представимо в простой форме, как, например (13.7).

Как правило, аналитическое решение наиболее часто существует, если выборка и априорное распределение образуют натуральную сопряженную пару, что означает принадлежность выборки, априорного и апостериорного распределений к одному классу. В этом случае априорное распределение называется сопряженным априорным распределением. В Разделе 13.2.2 рассматривается пример, где из нормальности распределения данных и нормальности априорного распределения для среднего следует нормальность апостериорного распределения.

Семейство экспоненциальных распределений является в принципе единственным, для которого существует естественное сопряженное распределение. Однопараметрическая функция из семейства экспоненциального распределения для одного наблюдения может быть записана как

\begin{equation}
f(y|\theta)=\exp\lbrace{a\theta+b(y)+c(\theta)u(y)}\rbrace,
\end{equation}

\[
{\propto}\exp\lbrace{a\theta+c(\theta)u(y)}\rbrace,
\]

где разные функции $a(\cdot), c(\cdot)$ и $u(\cdot)$ позволяют получить разные плотности в семействе распределений, и $b(\cdot)$ нормализующая константа. Например, при $c(\theta)=\mu/\sigma^{2}, a(\theta)=-\mu^{2}/2\sigma^{2}$, и $u(y)=y$ получим ядро нормального распределения $N[\mu,\sigma^{2}]$ (для известного параметра $\sigma^{2}$). Отметим, что $u(y)=y$ означает принадлежность к семейству линейных экспоненциальных распределений, более подробно рассмотренных в Разделе 5.7.3. В общем случае, если $\theta$ вектор, тогда $c(\theta)u(y)$ заменяется на $c(\theta)'u(y)$, где, как правило, $u(\cdot)$ имеет такую же размерность, как $\theta$.

Для случайной выборки размера $N$, распределение которой относится к экспоненициальному семейству, плотность будет равна

\begin{equation}
L(y|\theta){\propto}\exp\lbrace{Na(\theta)+c(\theta)t(y)}\rbrace,
\end{equation}

где $t(y)=\sum_{i}u(y_i)$. Рассмотрим следующее априорное распределение для $\theta$:

\begin{equation}
\pi(\theta|\beta,\alpha){\propto}\exp\lbrace{\beta{a}(\theta)+\alpha{c}(\theta)}\rbrace,
\end{equation}

где $\alpha$ и $\beta$  --- параметры априорного распределения и функции $a(\cdot)$ и $c(\cdot)$ те же, что и в (13.10). Полученная плотность относится к экспоненциальному семейству для $\theta$ в случае постоянного $\alpha$.

\begin{tabular}{ccc}
Распределение & Выборочная плотность & Сопряженное априорное \\ 
\hline 
Нормальное & $N[\theta,\sigma^2]$& $\theta{\sim}N[\mu,\tau^{2}]$\\ 
Нормальное & $N[\mu,1/\theta^2]$& $\theta{\sim}G[\alpha,\beta]$\\ 
Биномиальное & $B[N,\theta]$& $\theta{\sim}Beta[\alpha,\beta]$\\ 
Пуассона & $P[\theta]$& $\theta{\sim}G[\alpha,\beta]$\\ 
Гамма& $G[\nu,\theta]$& $\theta{\sim}G[\alpha,\beta]$\\ 
Мультиномиальное & $MN[\theta_1,\ldots ,\theta_k]$& $\theta_1,\ldots ,\theta_k{\sim}Dirichlet[\alpha_1,\ldots ,\alpha_k]$\\ 
\hline 
\end{tabular} 

После применения теоремы Байеса и упрощения имеем
\begin{equation}
p (\theta|y){\propto}L(y|\theta)\pi(\theta|\beta,\alpha)
\end{equation}

\[
{\propto}\exp \lbrace(\beta+N)a(\theta)+(\alpha+t(y))c(\theta)\rbrace,
\]

что подтверждает, что апостериорное распределение имеет то же ядро, что и исходное априорное распределение в (13.12). Сравнение апостериорной плотности с выборочной показывает, что априорное распределение рассматривается как источник дополнительных $\beta$ наблюдений $y_p$, к примеру, $t(y_p)=\alpha$.

В таблице 13.2 представлены некоторые виды стандартных сопряженных семейств, соответствующие им плотности даны в Приложении B. Гамма-распределение включает экспоненциальное и хи-квадрат распределения как частные случаи. Отрицательное биномиальное, равномерное распределения и распределение Парето также имеют сопряженные априорные распределения.

Преимущество сопряженного априорного распределения заключается в простоте расчетов и аналитических выводов. Тем не менее, использование сопряженных априорных распределений накладывает дополнительные ограничения, которые в настоящее время, с развитием вычислительных средств, становятся всё менее оправданными.

Другим преимуществом построения апостериорного распределения, относящегося к тому же классу, что и априорное, состоит в том что первое может использоваться как новое (учитывающее специфику данных) априорное распределение для дальнейшего анализа. Если априорное распределение интерпретировать как <<полученную информацию>>, апостериорное распределение одного исследования может использоваться как априорное в следующем.


\subsubsection*{Иерархические априорные распределения}

Иерархические априорные распределения возникают в случае, когда предполагается, что  параметры априорного распределения в свою очередь распределены по некоторому закону. Параметры подобных <<распределений над распределениями>>, называются гиперпараметрами. 

Пусть данные имеют совместную плотность распределения $L(y|\theta)$ как в разделе 13.2.1, но теперь априорное распределение $\theta$ зависит от параметра $\tau$, значение которого случайно. Таким образом, априорное распределение $\theta$ определяется как $\pi(\theta|\tau)$, где, в свою очередь, параметры $\tau$ имеют априорное распределение $\pi(\tau)$. Совместное априорное распределение выражается как $\pi(\theta,\tau)=\pi(\theta|\tau)\pi(\tau)$ и, согласно байесовскому правилу, совместное апостериорное распределение может быть  представлено как

\begin{equation}
p(\theta|y){\propto}L(y|\theta)\pi(\theta|\beta,\alpha)
\end{equation}

\[
{\propto}\exp\lbrace{(\beta+N)a(\theta)+(\alpha+t(y))c(\theta)}\rbrace,
\]

\[
p(\theta,\tau|y){\propto}L(y|\theta)\pi(\theta|\tau)\pi(\tau).
\]

Как правило, в литературе основное внимание сосредоточено на построении частной апостериорной плотности $\theta$, которая рассчитывается через интегрирование совместной апостериорной плотности по $\tau$. Параметры априорного распределения $\pi(\tau)$ называются гипермараметрами. Как вариант, эти параметры могут быть параметрами для следующего шага, следовательно выражение для совместной плотности можно записать $\pi(\theta|\tau)\pi(\tau|\phi)\pi(\phi)$ и так далее. Новые вычислительные методы байесовского анализа, в частности, алгоритм Гиббса, хорошо подходят для  иерархического априорного распределения поскольку оно имеет рекурсивную структуру.

Иерархическое априорное распределение можно рассматривать как аналог классических моделей со случайными коэффициентами. Например, для независимых одинаково распределенных дискретных данных можно предположить, что $y_i{\sim}P(\theta_i)$, где параметр распределения Пуассона является случайным. Целесообраpно предположить, что $\theta_i$ имеет распределение, сопряженное с пуассоновским, т.е. гамма-распределение, $\theta_i{\sim}G[\alpha,\beta]$. В классическом подходе $\alpha$ и $\beta$ оцениваются методом максимального правдоподобия. В неиерархическом байесовском методе значения параметров $\alpha$ и $\beta$ заранее определены и используются для построения апостериорного распределения $\theta_i$. В иерархической байесовской модели задается априорное распределение параметров $\alpha$ и $\beta$, например, сопряженное гамма-распределение, и на первом шаге строится совместное апостериорное распределение для $\theta_i$, $\alpha$ и $\beta$ для дальнейшего построения частного апостериорного распределения $\theta_i$.

Иерархическое априорное распределение естественным образом появляется при рассмотрении иерархических моделей, также известных как многоуровневые. Многоуровневые модели широко используются в классических методах, но с использованием специального программного обеспечения (Брик и Рауденбуш, 1992, 2002). Одной из первых по иерархическим моделям была написана работа Линдли и Смита (1972). Авторы рассматривают иерархические модели регрессии с помощью байесовского подхода. Потребность в иерархическом моделировании появляется, если данные разбиты на страты, группы или слои и параметры зависимости могут варьироваться по группам. Например, массив данных состоит из оценок учащихся разных школ. Для моделирования оценок за тест могут потребоваться индивидуальные характеристики, априори отличающиеся для каждого ученика, характеристики класса, зависящие от принадлежности к классу и школьные характеристики, меняющиеся от школы к школе. Поскольку эти данные предполагают кластеризацию наблюдений, иерархическое моделирование также рассмотрено в главе 24. Кроме того, многоуровневые модели и модели панельных данных со случайным эффектом имеют много общего.


Например, предположим, что все данные разбиты н $J$ групп и для каждой группы среднее значение $y$ уникально. Для индивида $i$ из группы $j$ предположим, что $y_{ij}{\sim}N[\theta_j,\sigma^2]$, где для простоты дисперсия $\sigma^2$ известна. Тогда среднее значение для $j$-ой группы равно $\overline{y}_j{\sim}N[\theta_j,\sigma^{2}/N_j]$, где $N_j$ обозначает количество индивидов в группе; предполагается, что наблюдения не зависят друг от друга. В иерархической модели определяется априорное распределение $\theta_i$, например, $\theta_{j}{\sim}N[\mu, \tau^2]$, где для параметров более высокого порядка $\mu$ и $\tau^2$ также задано априорное распределение.


\subsubsection*{Анализ чувствительности}


Частотный подход  допускает оценку модели при разном количестве априорных ограничений, накладываемых на модель. Например, можно оценить модель с одним или несколькими ограничениями, и далее можно сравнить результаты, чтобы определить чувствительность оценок к изменению априорных ограничений.


Аналогичный подход используется в байесовском анализе. Не обязательно считать, что априорное распределение является верным, можно провести анализ чувствительности, прослеживая изменения апостериорного распределения в зависимости от изменений в априорном. Кроме того, можно следить за тем как меняется апостериорное распределение, меняя предположения о процессе порождающем данные. 

\subsection{Апостериорное распределение: плотность и иные характеристики}

Байесовский анализ построен на апостериорном распределении. Для удобства, как правило, используют только общую статистику, например, моменты апостериорного распределения, квантили или частные распределения составляющих компонент вектора параметров $\theta$. Однако, апостериорное распределение также используется для построения прогнозов и вероятностных утверждений, о чем подробно пойдет речь в этом разделе, и для сравнения моделей, которое обсуждается в разделе 13.8.

Квантили некоторых уровней занимают особое место в байесовском анализе.

\subsubsection*{Частное апостериорное распределение}

В общем случае $\theta$ является многомерным вектором, т.е. $\theta'=(\theta_1,\ldots ,\theta_q)$ и предметом исследования может быть апостериорное распределения отдельных компонент $\theta$. Плотность частного апостериорного распределения $k$-того параметра $\theta_k$ рассчитывается путем интегрирования совместной плотности апостериорного распределения остальных $(q-1)$ компонента вектора $\theta$. Частная функция плотности обозначается через $p(\theta_k|y)$ и рассчитывается $(q-1)$-кратным интегрированием.

\begin{equation}
p(\theta|k|y)=\int{p(\theta_1,\ldots ,\theta_p|y)d\theta_1..d\theta_{k-1}d\theta_{k+1}..d\theta_q}
\end{equation}

\[
=\int{p(\theta|y)d\theta_{-k}},
\]

где в компактной записи во второй строке через $\theta_{-k}$ обозначены все компоненты вектора $\theta$, кроме $\theta_k$. Частное апостериорное распределение, как правило, асиметрично и не обязательно унимодально, в то время как асимптотическое нормальное распределение для классических оценок симметрично и унимодально. В связи с этим полезным является графический анализ апостериорного распределения, особенно в тех случаях, когда оно существенно отличается от симметричного унимодального.

\subsubsection*{Моменты апостериорного распределения}

Классическое представление результатов оценивания регрессии включает значения оценок параметров и стандартные ошибки. При байесовском оценивании также могут быть рассчитаны среднее или медиана и стандартное отклонение частного апостериорного распределения для каждого параметра.

\subsubsection*{Точечные оценки}

В рамках классического подхода задача состоит в получении хорошей точечной оценки неизвестного  параметра $\theta_0$, от которого зависит процесс порождающий данные $f(y|\theta_0)$. В байесовском анализе, в отличие от классического, задача состоит в нахождении закона распределения $\theta$, который определяется как  $\theta_0$, так и априорными предположениями о  $\theta_0$.

Таким образом, в байесовском подходе точечной оценке уделяется намного меньше внимания. Для удобства среднее или медиана апостериорного распределения принимаются за точечные оценки. Оптимальная точечная оценка параметра может быть рассчитана на основе функции потерь (см. Раздел 13.2.7).

\subsubsection*{Апостериорные доверительные интервалы}

Найденное апостериорное распределение может быть использовано для построения вероятностных утверждений аналогично частотному подходу. В частности, по методу Байеса могут быть построены доверительные интервалы и области. 

Для $k$-го параметра $100(1-\alpha)$-процентным апостериорным доверительным интервалом $R(\theta_k)$  является любой интервал, в который значения $\theta_k$ попадают  с апостериорной вероятностью $\alpha$, т.е.

\begin{equation}
1-\alpha=\Pr[\theta_k{\epsilon}R(\theta_k)|y]=\int_{R(\theta_k)p(\theta_k|y)d\theta}.
\end{equation}

Существует множество областей, которые соответствуют этой вероятности. Наиболее простой апостериорный доверительный интервал --- между квантилями $\alpha/2$ и $1-{\alpha}/2$, т.е. между 2.5 и 97.5 перцентилями. Более сложным является построение интервала наивысшей плотности апостериорного распределения (highest probability density, HPD), который удовлетворяет (13.15) и   дополнительному условию, согласно которому ни в одной из точек интервала $R(\theta)$ значение плотности распределения не ниже, чем ни в одной из точек за его пределами. Этот интервал необязательно должен быть непрерывныным, если апостериорное распределение полимодально, и в общем случаем он не совпадает с простым интервалом, исключение составляет случай симметричнго унимодального апостериорного распределения.

Построенные интервалы могут быть представлены как области. Например, $100(\alpha)$-процентная область наивысшей плотности апостериорного распределения $R(theta)$ задается соотношением 

\begin{equation}
1-\alpha=\Pr[\theta{\epsilon}R(\theta)|y]=\int R(\theta) p(\theta|y)d\theta.
\end{equation}

Преимущество байесовского подхода заключается в том, что апостериорные интервалы намного проще интерпретировать, чем доверительные интервалы в частотном анализе. Если границы $95$-ти процентного апостериорного интервала для $\theta_k$ равны $(1,4)$, можно утверждать, что значение $\theta_k$ лежит между 1 и 4 с вероятностью 0.95. В случае частотного анализа для $95$-ти процентного доверительного интервала для $\theta_k$ с границами $(1,4)$ можно лишь утверждать, что в случае построения множества доверительных интервалов для различных выборок $95$ процентов этих доверительных интервалов будут включать истинные значения $\theta_k$.

\subsubsection*{Тестирование гипотез}

Тестированию гипотез уделяется незначительное внимание в теме байесовских методов. Как отмечалось при рассмотрении точечных оценок, расчет истинного значения $\theta_0$ не является основной целью байесовских методов. Основной интерес составляет определении области значений, которые может принять параметр $\theta$ при имеющихся наблюдениях и заданном априорном распределении. Для сравнения моделей см. Раздел 13.8.

\subsubsection*{Условная плотность апостериорного распределения}

Условная плотность апостериорного распределения параметра $\theta_k$, при заданных значениях $\theta_j$, может быть рассчитана на основе сведений о совместном и частном апостериорных распределениях из соотношения

\begin{equation}
p(\theta_k|\theta_j,\theta_j \in \theta_{-k},y)=\dfrac{p(\theta_k,\theta_j|y)}{p(\theta_j|y)}.
\end{equation}

Особое внимание уделяется множеству $q$ условных плотностей распределения $p(\theta_k|\theta_{-k}), k=1,\ldots ,q$, также известных как множество полных условных распределений. Эти распределения играют важную роль для современных вычислительных методов, применяемых для расчета параметров совместного апостериорного распределения, которое будет обсуждаться далее.

Определения частной и условной апостериорной плотностей в (13.15) и (13.17) могут быть расширены для блоков параметров. 

\subsubsection*{Предельная функция правдоподобия}

Частная вероятность или частная функция правдоподобия равна знаменателю в формуле Байеса и определяется по следующей формуле: 

\begin{equation}
f(y)=\int{L(y|\theta)}\pi(\theta)d\theta.
\end{equation}

Значение этого выражения равно условному математическому ожиданию функции правдоподобия, $\E[L(y|\theta)]$ по априорной плотности. Частная функция правдоподобия лежит в основе байесовских выводов (см Раздел 13.8), поскольку эта функция включает информацию о носителе априорного распределения.

\subsubsection*{Апостериорная плотность для прогнозов}

Рассмотрим точечный прогноз вне выборки $y^p$, плотность распределения которого равна $f(y^p|\theta)$, где значение $\theta$ неизвестно. Значение предиктивной апостериорной плотности $y^p$ определяется взвешиванием плотности распределения с помощью апостериорного  распределения $\theta$

\begin{equation}
f^{p}(y^p)=\int{f(y^p|\theta)p(\theta|y)d\theta}.
\end{equation}

Если  функция правдоподобия, как регрессионная модель, содержит регрессоры, то они также учитываются при расчете условных плотностей. 

\subsection{Поведение апостериорной плотности на больших выборках}

С ростом выборки влияние априорного распределения на апостериорное, даже при условии информативности первого, снижается, как это показано в примере 13.2.2. Этот факт позволяет утверждать, что асимптотически информация выборки доминирует над априорной информацией или, другими словами, вес априорного распределения  стремится к нулю с ростом выборки.

Поскольку обращаться с апостериорным распределением трудно, существенным является вопрос его асимптотической аппроксимации, поскольку такая аппроксимация может использоваться вместо истинного апостериорного распределения конечной выборки. Приближенные значения достаточно легко получить, поскольку апостериорное распределение асимптотически сходится к функции правдоподобия. Для более подробного изучения см. Гельман и др. (1995).

Для простоты предположим, что наблюдения независимы и одинаково распределены. Тогда логарифм апостериорного распределения представим в виде

\begin{equation}
\sum^{N}_{i=1}\ln {p(\theta|y_i)}=\ln \pi(\theta)+\sum^{N}_{i=1}\ln {f(y_i,\theta)}.
\end{equation}

При таком представлении очевидно, что для больших выборок апостериорное распределение в большей степени определяется вкладом функции правдоподобия, поскольку влияние априорного распределения на апостериорное фиксировано, в то время как влияние выборочного распределения на апостериорное растет с ростом размера выборки $N$. 

Предположим, что апостериорная функция $p(\theta|y)$ унимодальна и асимптотически симметрична. Рассмотрим асимптотические свойства моды апостериорного распределения, $\hat{\theta}$, которая является как локальным, так и глобальным максимумом функции плотности апостериорного распределения.

Для изучения состоятельности оценки $\hat{\theta}$ заметим, что значение апостериорной моды сходится к ММП оценке при $N{\rightarrow}\infty$, поскольку  второе слагаемое в (13.20) доминирует. Таким образом, мода апостериорного распределения состоятельна, если состоятельна ММП-оценка. Тогда, $\hat{\theta}{\rightarrow}^{p}\theta_0$ если  процесс порождающий данные $y$ имеет плотность $f(y|\theta_0)$ и выполняются стандартные условия регулярности для ММП-оценок. 

Для того, чтобы получить асимптотическое распределение $\hat{\theta}$, рассмотрим разложение в ряд Тейлора второго порядка логарифмической апостериорной плотности в окрестности моды $\hat{\theta}$. Тогда

\begin{equation}
\ln {p(\theta|y){\simeq}}\ln {p(\hat{\theta}|y)}+\dfrac{1}{2}(\theta-\hat{\theta})'\left[\dfrac{\partial^{2}\ln {p(\theta|y)}}{\partial\theta\partial\theta'}{\mid}_{\theta=\hat{\theta}}\right](\theta-\hat{\theta}), 
\end{equation}

Так как для моды апостериорного распределения выполняется условие $\partial{\ln }p(\theta|y)/\partial\theta=0$ и асимптотически  производные третьего и высшего порядков по $\theta$ игнорируются, выражение может быть упрощено. Обозначим через

\[
I(\hat(\theta))=-\dfrac{\partial^{2}\ln {p(\theta|y)}}{\partial\theta\partial\theta'}|_{\theta=\hat{\theta}}
\]

полученную информацию, основанную на апостериорной плотности $\ln {p(\theta|y)}$ в точке моды апостериорного распределения. Тогда потенцирование (13.21) даст

\[
p(\theta|y){\propto}\exp\left(-\dfrac{1}{2}(\theta-\hat{\theta})'I(\hat{\theta})(\theta-\hat{\theta})\right), 
\]

и это выражение задает ядро многомерного нормального распределения со средним $\hat{\theta}$ и ковариационной матрицей $I(\hat{\theta})^{-1}$. Из  этого следует, что апостериорное распределение имеет вид

\[
\theta|y \overset{a}{\sim} N[\hat{\theta},I(\hat{\theta})^{-1}].
\]

С ростом размера выборки $N$ составляющая правдоподобия становится доминирующей в апостериорном распределении и влияние априорного распределения становится незначимым. В таком случае, мода оценки $\hat{\theta}$ может быть заменена ММП-оценкой, которая является модой правдоподобия. В результате получим так называемую байесовскую центральную предельную теорему (Гамерман, 1997). Асимптотически, результаты частотного и байесовского выводов будут основаны на одном и тоже предельном многомерном нормальном распределении, и, таким образом, между ними не должно возникать существенного несоответствия. 

В литературе этот результат был обозначен как теорема Бернштейна-фон Мизеса; для более подробного изучения трех компонент этой теоремы, см. Трейн (2003, глава 12). Эти компоненты включают (1) вывод о том, что апостериорное среднее сходится по вероятности к  оценке максимального правдоподобия, (2) онон имеет нормальное распределение и (3) предельное распределение апостериорного среднего совпадает с распределением  оценки максимального правдоподобия. Эти результаты неявно фигурируют в байесовской центральной предельной теореме. Эта теорема является ключевой при использовании принципов правдоподобия для оценивания и формулирования выводов. Вся сила байесовской центральной предельной теоремы в полном объеме будет видна после обсуждения применения численных методов для аппроксимации апостериорного распределение. 

Следует ли из приведенных аргументов, что применение байесовского метода и метода максимального  правдоподобия приводит к существенно аналогичным результатам? Определяется ли выбор между двумя подходами удобством расчетов? Точные ответы на поставленные вопросы отсутствуют. Однако в литературе приводится ряд примеров, в которых не только показано, что эти методы дают схожие результаты, но также, что байесовские методы как правило более эффективны с точки зрения трудоемкости вычислений.

\subsection{Принятие решений в байесовском подходе}

При данном апостериорном распределении $p(\theta|y)$, какая из точечных оценок $\theta$ должна быть выбрана? Этот вопрос рассмотрен в Разделе 4.2 с точки зрения выбора наилучшего прогнозного значения $y$, например, с помощью  квадратичной функции потерь. В данном разделе рассматривается наилучшая оценка параметра $\theta$, с точки зрения, например, с помощью квадратичной функции потерь.

Допустим $L(\theta,\hat{\theta})$ --- заданная функция потерь, где $\hat{\theta}$ оценка неизвестного параметра $\theta$. Величина потери неизвестна, поскольку зависит от неизвестного параметра $\theta$. Тем не менее, возможно найти ожидаемое значение потери, поскольку при байесовском анализе, в отличии от классического подхода, известно распределение $\theta$. Оптимальная оценка, $\hat{\theta}_{OPT}$, это оценка параметра $\hat{\theta}$, которая минимизирует величину ожидаемых апостериорных потерь

\begin{equation}
\underset{\hat{\theta}}{\min}E[L(\theta,\hat{\theta})]=\underset{\hat{\theta}}{\min}\int{L}(\theta,\hat{\theta})p(\theta|y)d\theta,
\end{equation}

Потери, которые зависят от различных значений $(\theta,\hat{\theta})$ взвешиваются по апостериорной вероятности  $p(\theta|y)$.

Можно показать, что апостериорное среднее является оптимальной оценкой для квадратичной функции потерь, $L(\theta,\hat{\theta})=(\theta-\hat{\theta})'(\theta-\hat{\theta})$. Если использовать абсолютное значение ошибки, $L(\theta,\hat{\theta})=|\theta-\hat{\theta}|$, то оптимальной оценкой будет медиана апостериорного распределения. Поскольку апостериорное распределение известно эти оптимальные оценки можно рассчитать аналитически или численно.

При выполнении соответствующих предпосылок, можно показать, что минимизация ожидаемой апостериорной ошибки  эквивалентна минимизации ожидаемого апостериорного риска. Функция риска усредняет возможные потери по гипотетической выборке $y$ из генеральной совокупности, т.е. 

\[
R(\theta,\hat{\theta})=\int{L}(\theta,\hat{\theta})f(y|\theta)dy.
\]

Для того, чтобы избежать путаницы с обозначением для функции потерь и функции правдоподобия, в данном и последующем блоках уравнений, $f(y|\theta)$ используется для обозначения функции правдоподобия $L((y|\theta)$. Ожидаемый апостериорный риск  усредняет риск для различных параметров $\theta{\epsilon}\Theta$ путем присваивания каждому значению веса в соответствии со значением плотности апостериорного распределения, и 

\begin{equation}
E[R(\theta,\hat{\theta})]=\int_{\Theta}\left\lbrace\int{L}(\theta,\hat{\theta})f(y|\theta)dy \right\rbrace{p}(\theta|y)d\theta
\end{equation}

\[
=\int{ \left\lbrace\int_{\Theta}L(\theta,\hat{\theta})p(\theta|y)d\theta\right\rbrace}f(y|\theta)dy
\]

\[
=\int{E[L(\theta,\hat{\theta})]f(y|\theta)dy},
\]

где в первом выражении границы внешнего интеграла равны области определения параметра $\theta$, во втором равенстве порядок интегрирования меняется и в третьей строке записано итоговое выражение. Для данных преобразований предполагаются выполненными соответствующие  ограничения на $L(\theta,\hat{\theta})$ и $p(\theta,y)$. Например, $p(\theta|y)$ должна быть собственной функцией плотности и функция потерь должна быть интегрируема. Таким образом, значение ожидаемого риска будет ограничено и её минимизация --- это корректная операция.

Мы получили широко известный и важный результат, что байесовская оценка является допустимой в том смысле, что она минимизирует  ожидаемый риск для некоторой функции потерь.

\section{Байесовский анализ линейной регрессии}

Поскольку анализ линейных регрессий --- знакомая тема, он является полезным началом перехода к классу нелинейных моделей. Предположим, что данные порождены стандартной линейной регрессионной моделью

\[
y=X\beta+u,
\]

где $X$ обозначает $N \times K$ матрицу полного ранга слабо экзогенных регрессоров. Кроме того, делается предпосылка о независимости, гомоскедастичности и нормальном распределении ошибок, $u \sim N[0,\sigma^{2}I_{N}]$. Таким образом, выборочная условная плотность равна $y|X, \beta, \sigma^{2} \sim N[X\beta,\sigma^{2}I_{N}]$. Мы следуем изложению Зеллнера (1971).

Далее рассмотрим  неинформативное и информативное априорное распределение. В обоих случаях после длинных выкладок возможно получить аналитическое выражение для апостериорного распределения. Для информативного  априорного распределения моменты апостериорного распределения являются взвешенной функцией от выборочного и априорного среднего.

В последующих разделах представлено описание методов для более сложных моделей, тем не менее, анализ этих методов упрощается, если результаты, полученные в этом разделе, могут быть частично применены к некоторым компонентам модели.

\subsection{Неинформативное априорное распределение}

В качестве неинформативного априорного распределения мы используем априорную вероятность Джеффри. Из Раздела 13.2.4 следует, что для $y{\sim}N[\mu,\sigma^{2}]$ априорная плотность для $\mu$ (при заданном значении $\sigma^{2}$) постоянна, в то время, как априорная плотность для $\sigma^{2}$ (при заданном значении $\mu$) пропорциональна $\sigma^{2}$. При рассмотрении регрессии постоянной плотность будет для $\beta_j, j=1,\ldots ,K$, т.е. $\pi(\beta_j)\propto{c}$ и априорная плотность для $\sigma^{2}$ равна $\pi(\sigma^{2})\propto{1/\sigma^{2}}$. Априорная информация о $\beta_j$ рассматривает значения параметра как равновероятные, в тоже время считается, что малые значения $\sigma^{2}$ более вероятны, чем большие. Предполагая независимость $\beta$ и $\sigma^{2}$ совместная априорная плотность равна

\[
\pi{(\beta,\sigma^{2})\propto{1/\sigma^{2}}}.
\]

Функция правдоподобия равна

\begin{equation}
L(\beta,\sigma^{2}|y,X)=(2\pi\sigma^{2})^{-N/2}\exp\left\lbrace-\dfrac{1}{2\sigma^{2}}(y-X\beta)'(y-X\beta)\right\rbrace
\end{equation}

\[
{\propto}(\sigma^{2})^{-N/2}\exp\left(-\dfrac{1}{2\sigma^{2}}\lbrace{\hat{u}'\hat{u}+(\beta-\hat{\beta})'X'X(\beta-\hat{\beta})}\rbrace\right) 
\]

\[
{\propto}(\sigma^{2})^{-N/2}\exp\left(-\dfrac{1}{2\sigma^{2}}(N-K)s^2+(\beta-\hat{\beta})'X'X(\beta-\hat{\beta}))\right),
\]

где $\hat{\beta}=(X'X)^{-1}X'y$ и $\hat{u}=y-X\hat{\beta}$; для перехода во второй строке используется равенство $y-X\beta=\hat{u}-X(\beta-\hat{\beta})$ и $X'\hat{u}=0$; и для перехода в третьей строке --- $s^2=\hat{u}'\hat{u}/(N-K)$.

Совмещая функцию  правдоподобия, (13.25), и априорную функцию плотности, получим апостериорную плотность

\begin{multline}
p(\beta,\sigma^{2}|y,X) \\
\propto\left(\frac{1}{\sigma^2}\right)^{N/2}\exp\left(-\dfrac{1}{2\sigma^{2}}\lbrace(N-K)s^2+(\beta-\hat{\beta})'X'X(\beta-\hat{\beta}))\rbrace\right)\dfrac{1}{\sigma^2} \\
\propto\left(\frac{1}{\sigma^2}\right)^{N/2+1}\exp\left(-\dfrac{1}{2\sigma^{2}}\lbrace(N-K)s^2+(\beta-\hat{\beta})'X'X(\beta-\hat{\beta}))\rbrace\right) \\
\propto \left(\frac{1}{\sigma^2}\right)^{K/2}\exp\left(-\dfrac{1}{2}(\beta-\hat{\beta})'(\sigma^{2}(X'X)^{-1})^{-1}(\beta-\hat{\beta})\right)  \\ 
\times \left(\dfrac{1}{\sigma^2}\right)^{(N-K)/2+1}\exp\left(-\dfrac{(N-K)s^2}{2\sigma^2}\right)  .
\end{multline}

Условное апостериорное распределение $p(\beta|\sigma^2,y,X)$ параметра $\beta$, при заданных значениях $\sigma^2$ и данных $y,X$, является $K$-мерным нормальным со средним $\hat{\beta}$ и дисперсией $\sigma^{2}(X'X)^{-1}$, поскольку $\beta$ присутствует только в первой строке последнего выражения. Условная апостериорная плотность распределения $\sigma^2$ при заданном значении $\beta$ сложнее рассчитать, поскольку $\sigma^2$ присутствует в обеих строках.

Частная апостериорная плотность распределения $\beta$, значения которой получены интегрированием по $\sigma^2$, имеет большее значение для апостериорных выводов о  $\beta$. Проинтегрируем вторую строчку (13.26), используя замену переменной $z=1/\sigma^2$ и равенство $\int^{\infty}_{0}z^{c}\exp(-az)dz=\Gamma(c+1)/a^{c+1}$ верное для произвольных констант $a>0, c>-1$, где $c=N/2+1$ и $a=\lbrace\cdot\rbrace$ --- длинное выражение в фигурных скобках. В результате получим ядро функции частного апостериорного распределения

\begin{equation}
p(\beta|y,X){\propto}\lbrace(N-K)s^2+(\beta-\hat{\beta})'X'X(\beta-\hat{\beta}))\rbrace^{-N/2}
\end{equation}

\[
{\propto}\lbrace{1+(\beta-\hat{\beta})'(s^{2}(N-K)(X'X)^{-1})^{-1}(\beta-\hat{\beta})}\rbrace^{-(N-K+K)/2},
\]
где из Раздела 13.3.5 следует, что это выражение является ядром многомерного t-распределения Стьюдента с центром в точке $\hat{\beta}$ и $N-K$ степенями свободы и ковариационной матрицей $s^2(X'X)^{-1}$ домноженной на $(N-K)/(N-K-2)$. Таким образом,

\begin{equation}
\beta{\sim}t_{k}(\hat{\beta},s^2(X'X)^{-1}).
\end{equation}

Отдельный элемент вектора $\beta$ имеет одномерное t-распределение Стьюдента.

Частная функция апостериорной плотности для $\sigma^2$ получается легче. Интегрируем последнее выражение в (13.26) по $\beta$, заметим, что $\beta$ присутствует только в первой строке. В результате получаем ядро нормальной плотности распределения $N[\hat{\beta},\sigma^2(X'X)^{-1}]$ и интеграл плотности равен 1. Следовательно, функция для частного апостериорного распределения $\sigma^2$ равна

\begin{equation}
p(\sigma^2|y,X){\propto}(\sigma^2)^{-(N-K+1)/2}\exp\left(-\dfrac{(N-K)s^2}{2\sigma^2}\right). 
\end{equation}

Это выражение является ядром для обратного к квадратному корню из гамма распределения. Иными словами, это плотность случайной величины, обратной к квадратному корню из гамма-распределенной случайной величины с $(N-K)$ степенями свободы. Этот результат идентичен результату применения частотного подхода к построению распределения $\hat{\beta}$.

Для нормальной линейной регрессии, применение байесовского анализа с неинформативным  априорным распределением на конечных выборках дает качественно схожие выводы со стандартным частотным анализом. Условное распределение параметра $\beta$ по $\sigma^2$ равно $N[\hat{\beta},\sigma^{2}(X'X)^{-1}]$, а безусловное распределение $\beta$ есть многомерное $t$-распределение. 

Тем не менее, интерпретация результатов довольно отличается, поскольку здесь мы видим распределение  неизвестного параметра $\beta$ со средним значением $\hat{\beta}$, а не оценки $\hat{beta}$ с неизвестным средним $\beta$. Например, $95$-ти процентный интервал наивысшей плотности для $\beta_j$ равен $\hat{\beta}_j{\pm}t_{.025,N-K}{\times}se[\hat{\beta_j}]$, где $se[\hat{\beta_j}]=(s^{2}(X'X)^{jj})^{1/2}$. Из Раздела 13.2.5 следует, что $\beta_j$ принадлежит интервалу наивысшей плотности с апостериорной вероятностью 0.95.

\subsection{Информативное априорное распределение}

Наиболее понятным байесовский анализ модели нормальной линейной регрессии с информативным априорным распределением будет, когда используется предпосылка о независимости сопряженных распределений $\beta$ и $\sigma$. Из Раздела 13.2.4 сопряженное распределение для $\beta$ является нормальным, а для $1/\sigma^2$ -- гамма. Мы получаем плотность нормального-гамма распределения

\begin{equation}
\pi(\beta,1/\sigma^2)=\pi_{N}(\beta|1/\sigma^2)\pi_{\gamma}(1/\sigma^2),
\end{equation}

где $\pi_{N}(\beta|1/\sigma^2)$ --- это плотность нормального распределения $N[\beta_0,\sigma^2{\Omega}^{-1}_0]$, с известными значениями $\beta_0$ и $\Omega_0$ и ядром 

\begin{equation}
\pi_{N}(\beta|1/\sigma^2){\propto}\sigma^{-K}\exp[-\dfrac{(\beta-\beta_0)'{\Omega_0}(\beta-\beta_0)}{2\sigma^2}],
\end{equation}

и $\pi_{\gamma}(1/\sigma^2)$ имеет плотность $G[\nu_0,s^2_0]$, где $\nu_0$ и $s^2_0$ известны и 

\begin{equation}
\pi_{\gamma}(1/\sigma^2)=\sigma^{-(\nu_0+1)}\exp\left[-\dfrac{\nu_{0}s^2_0}{2\sigma^2}\right].
\end{equation}

Отметим, что априорная плотность распределения для параметра  $\beta$ зависит от плотности параметра масштаба $\sigma$. Это имеет смысл, поскольку $\sigma$ отражает масштаб измерения $y$ и, таким образом, должно влиять на значение $\beta$. При заданных априорной плотности и  функции правдоподобия в (13.25), апостериорная плотность имеет нормальный-гамма вид. После математических преобразований получим, что:

\begin{multline}
p(\beta,1/\sigma \mid y, X) \propto (\sigma^2)^{-N/2} 
\exp\left[-\frac{s^2(N-K)}{2\sigma^2}\right]
\exp\left[-\frac{(\beta-\hat{\beta})'X'X(\beta-\hat{\beta})}{2\sigma^2}\right] \\
\times (\sigma^2)^{-K/2} 
\exp\left[-\frac{(\beta-\beta_0)'\Omega_0(\beta-\beta_0)}{2\sigma^2}\right] \\
\times (\sigma^2)^{-\nu_0/2-1} 
\exp\left[-\frac{\nu_0 s_0^2}{2\sigma^2}\right]\\
\propto (\sigma^2)^{(\nu_0/2+N)/2-1-K/2} 
\exp\left[-\frac{s_1^2}{2\sigma^2}\right]\\
\times \exp\left[-\frac{(\beta-\bar{\beta})'\Omega_1(\beta-\bar{\beta})}{2\sigma^2}\right]
\end{multline}
где $\bar{\beta}$ и $\Omega_1^{-1}$ обозначают апостериорное среднее и дисперсию $\beta$, и $s_1^2$ обозначает апостериорное среднее $\sigma^2$:

\begin{multline}
\bar{\beta}=(\Omega_0+X'X)^{-1}(\Omega_0\beta_0+X'X\hat{\beta}) \\
\Omega_1=(\Omega_0+X'X) \\
s_1^2=s_0^2+\hat{u}'\hat{u}+(\beta-\bar{\beta})'[\Omega_0^{-1}+(X'X)^{-1}](\beta-\bar{\beta})
\end{multline}

Апостериорное среднее $\bar{\beta}$ получается путем использование матричного варианта дополнения до полного квадрата. А именно, для векторов $\beta$, $\bar{\beta}$, $\beta_0$ и $\hat{\beta}$ размера $K\times 1$ и симметричных квадратных матриц $A$ и $B$ размера $K\times K$ можно показать, что

\begin{multline}
(\beta-\beta_0)'A(\beta-\beta_0)+(\beta-\hat{\beta})'B(\beta-\hat{\beta}) \\
= (\beta-\bar{\beta})'(A+B)(\beta-\bar{\beta})+(\beta_0-\bar{\beta})'AB(A+B)^{-1}(\beta_0-\bar{\beta})
\end{multline}
где $\bar{\beta}=(A+B)^{-1}(A\beta_0+B\hat{\beta})$.

Совместная частная плотность апостериорного распределения $\beta$ и $\sigma^2$ имеет такую же нормальную-гамма форму как и априорное распределение. 

Условное апостериорное распределение $\beta$ при заданном значении $\sigma^2$ имеет среднее $\overline{\beta}$, средневзвешенное матричное априорного среднего $\beta_0$ и выборочного среднего $\hat{\beta}$.

В общем случае, применение сопряженного априорного распределения математически эквивалентно увеличению количества данных за счет выборки с аналогичным распределением. В этом случае, априорная плотность нормального гамма распределения эквивалентна дополнительной выборке с оценкой параметра $\beta_0$, матрицей $X'X$ равной $\Omega_0$, количеством степеней свободы $\nu_0$ и суммой квадратов ошибок $\nu_{0}s^2_0$. Поскольку значение матрицы $\Omega_0$ фиксировано, $\Omega_0/N{\rightarrow}0$ при $N{\rightarrow}\infty$, в то же время $X'X/N$ сходится к матрице констант. Таким образом, $\overline{\beta} \to\hat{\beta}$, подтверждает что при большой выборке ММП-оценка равна апостериорному среднему. Значение апостериорной дисперсии $\Omega^{-1}_1$ пропорционально $(\Omega_0+X'X)^{-1}$. Более подробно данный вопрос изложен у Лимера (1978). 

Частная апостериорная плотность $\beta$ может быть получена интегрированием совместной плотности по $\sigma^2$. Это даст нам выражение

\begin{equation}
p(\beta|y,X){\propto}\left[s^2_1+(\beta-\overline{\beta})'(\Omega_0+X'X)(\beta-\overline{\beta})\right]^{-(\nu_1+K/2)};
\end{equation}

таким образом частная апостериорная плотность является многомерным $t$-распределением Стьюдента с центром точке $\overline{\beta}$, а не в $\hat{\beta}$ как в случае с неинформативным априорным распределением.

Поскольку сопряженная априорная плотность эквивалентна наличию дополнительной выборки, использование выборочной и априорной информации происходит симметрично, не смотря на то, что информация, полученная из двух источников, может противоречить друг другу. 
Таким образом, за удобство использования сопряженного распределения нужно платить. 
В случае, если априорная и выборочная информация противоречат друг другу, можно предположить, что апостериорное распределение будет бимодальным с модальными значениями равным выборочному и априорному средним, соответственно. 
Одна из возможностей реализовать эту особенность --- предположить, что $\beta$ имеет априорное многомерное $t$-распределение Стьюдента независимое от $1/\sigma^2$ и $1/\sigma^2$ имеет априорное гамма распределение независимое от $X\beta$. Эта априорное распределение называется <<априорным распределение Дики>> (Лимер, 1978, стр. 79). При данной предпосылке частное  апостериорное распределение получается путем перемножения двух плотностей многомерного $t$-распределения Стьюдента; это произведение может также быть выражено как смесь двух  $t$-распределений. Это распределение может обладать свойством бимодальности. Лимер (1978) приводит более подробный анализ этой ситуации.

\subsection{Смешенное оценивание}

В данном разделе рассмотрено как можно представить байесовский анализ линейной регрессии через частотный подход к вероятности. 

В частотном анализе часто учитывают априорную информацию с помощью ограничений в виде равенств. Такую априорную информацию можно считать предельным случаем байесовского анализа, если значения параметров дисперсии устремить к бесконечности. В частотном анализе также может быть использована недетерминистическая априорная информация с помощью  смешенного оценивания. Применение этого анализа не требует сложной математики и содержит интуитивное понимание того, как байесовский подход может объединять в себе априорную и выборочную информацию. 

Рассмотрим нормальную линейную регрессионную модель. Предположим, что в соответствии с априорной информацией регрессионный параметр имеет нормальное распределение, $\beta \sim N[0,\sigma^2_{\nu}I_K]$, обобщение на случай ненулевого математического ожидания не связано с существенными трудностями. Таким образом, априорная информация может быть записана следующим образом

\[
\beta=0+v,
\]

где $v$ --- вектор ошибок размера  $K \times 1$  и $v \sim N[0,\sigma^2_{\nu}I_{K}]$. Внесем это дополнительную априорную информацию в выборку $y=X\beta+u$, и запишем расширенную регрессионную модель

\[
\begin{bmatrix}
y\\0
\end{bmatrix}
=
\begin{bmatrix}
X\\I_{K}
\end{bmatrix}
\beta+
\begin{bmatrix}
u\\{-v}
\end{bmatrix}
\]

Замена параметров даст следующие результаты:

\[
\begin{bmatrix}
y\\0
\end{bmatrix}
=
\begin{bmatrix}
X\\\dfrac{\sigma}{\sigma_\nu}I_{K}
\end{bmatrix}
\beta+
\begin{bmatrix}
u\\-\dfrac{\sigma}{\sigma_\nu}{v}
\end{bmatrix}
=
\begin{bmatrix}
X\\{\lambda}I_{K}
\end{bmatrix}
\beta+
\begin{bmatrix}
u\\{v^{*}}
\end{bmatrix}
\]

где $\lambda=\sigma/\sigma_{\nu}$ и  замена $v^{*}=-\lambda{v}$ была сделана, чтобы все ошибки имели одинаковую дисперсию $\sigma^2$.

Метод оценки, основанный на расширенном наборе данных, называется расшеренным или смешанным МНК. При фиксированном $\lambda$ оценка смешенным МНК равна

\begin{equation}
\hat{\beta}_{\lambda}=[X'X+\lambda^{2}I_{K}]^{-1}X'y
\end{equation}

\[
=[X'X(I_{K}+\lambda^{2}(X'X)^{-1})]^{-1}X'y
\]

\[
=[I_{K}+\lambda^{2}(X'X)^{-1}]^{-1}(X'X)^{-1}X'y
\]

\[
=A_{\lambda}\hat{\beta},
\]

где $A_{\lambda}=[I_{K}+\lambda^{2}(X'X)^{-1}]^{-1}$ и $\hat{\beta}=(X'X)^{-1}X'y$ неограниченная МНК-оценка.

Этот подход, который впервые был введен Хёрлем и Кеннардом (1970) для устранения проблемы мультиколлинеарности при оценке на малых выборках получил название ридж-регрессии (ridge regression). Этот подход относится к задачам регуляризации, в которых дополнительно накладывается условие регулярности. Оценка называется сжимающей оценкой (shrinkage estimator), так как она приближается к априорному среднему, в данном случае к нулевому вектору. Данный подход оправдан, например, для конечных выборок с сильно коррелированными данными, где $t$-статистики близки к нулю, что затрудняет возможность отличать переменные, чьи коэффициенты действительно близки к нулю, от переменных, чьи коэффициенты кажутся близкими к нулю. В пределе сближение оценки коэффициента с нулем означает выкидывание регрессора.

Следует отметить некоторые особенности $\hat{\beta}_{\lambda}$: (1) при фиксированном $\lambda$ оценка $\hat{\beta}_{\lambda}$ является средним апостериорного распределения $\beta$ (2) оценка является матричным взвешенным нулевого вектора $0$ и вектора $\hat{\beta}$ (3) вычисления практически не изменятся, если мы будем стягивать оценку к ненулевому $\beta$, скажем к $\beta_0$. Тогда, результирующая оценка будет матричным взвешенное векторов $\beta_0$ и $\hat{\beta}$.

Симметричная взвешенная матрица $A_{\lambda}=[I_{K}+(\lambda^{2}/N)(N^{-1}X'X)^{-1}]\rightarrow{I_K}$ при $N\rightarrow{\infty}$ поскольку $\lambda^{2}/N{\rightarrow}0$. Поэтому,

\[
\hat{\beta}_{\lambda}{\rightarrow}\hat{\beta}, \text{ при } N\rightarrow\infty
\]

и влияние  априорного распределения на апостериорное среднее становится незначительным при увеличении размера выборки.

Аналогично, условная апостериорная дисперсия параметра $\hat{\beta}_\lambda$ равна

\[
V[\hat{\beta}_{\lambda}]=A_{\lambda}V[\hat{\beta}]A_{\lambda}
\]

\[
=\sigma^2 A_{\lambda}(X'X)^{-1}A_{\lambda},
\]

и при $N \rightarrow \infty$ оказывается, что $V[\hat{\beta}_{\lambda}] \rightarrow \sigma^{2}(X'X)$.

Для конечных выборок, при фиксированных  $\lambda$ и $\sigma^2$, условное апостериорное распределение $\hat{\beta}_{\lambda}$ может быть записано в виде

\begin{equation}
\hat{\beta}_{\lambda}|\lambda,\sigma^2\sim
N[A_{\lambda}\hat{\beta},\sigma^2 A_{\lambda}(X'X)^{-1}A'_{\lambda}].
\end{equation}

Частное апостериорное распределение $\hat{\beta}_{\lambda}$ можно получить с помощью интегрирования по $\lambda$ и $\sigma^2$. 
Предполагая, что $\lambda$ задано, а также, что $\sigma^2$ имеет неинформативное априорное распределение, можно проинтегрировать по $\sigma^2$ как показано в Разделе 13.3.1. 
Этот интеграл можно посчитать в явном виде и в результате получим, что частное  апостериорное распределение параметра $\hat{\beta}_{\lambda}$ является многомерным $t$-распределением Стьюдента. 
Кроме того, возможно задать априорное распределение $\lambda$, например, выбрать в качестве априорного  гамма распределение, поскольку $\lambda>0$ и, далее, перейти к интегрированию. Однако, $\lambda$ входит в  условное апостериорное распределение в сложном виде и аналитически посчитать интеграл невозможно. Следовательно, необходимо воспользоваться численным интегрированием. Численное интегрирование позволяет получить байесовскую трактовку модели.

\subsection{Иерархическая априорная информация}

Рассмотрим трех-уровневую линейную регрессионную модель, которая является иерархической по регрессионным параметрам, но не по параметрам дисперсии. 

На первом уровне линейная регрессия имеет вид $y=X_{1}\beta_{1}+u$, где индекс 1 добавляется для того, чтобы отличить параметры и регрессоры первого и второго уровней. Параметры $\beta_1$ случайны и зависят как от параметров, так и от данных, т.е. $\beta_1=X_{2}\beta_{2}+v$. Например, первый уровень модели описывает результаты отдельных студентов на экзамене, а второй уровень модели описывает характеристики школ. Предполагается, что ошибки нормально распределены. Также предполагается, что параметры второго уровня $\beta_2$ неизвестны и задано их  априорное распределение. Априорное распределение также задано для $\sigma^2$ на первом уровне модели.

Предполагая нормальное распределение ошибок и используя сопряженное априорное распределение, получим следующую модель

\begin{equation}
y|X_{1},\beta_1,\sigma^{2}_1 \sim N[X_{1}\beta_1,\sigma^2_{1}I_N],
\end{equation}

\begin{equation}
\beta_1|X_2,\beta_2,\Sigma_2 \sim N[X_2\beta_2,\Sigma_2],
\end{equation}

\begin{equation}
\beta_2 \sim N[\beta^{*},\Sigma^{*}],
\end{equation}

\begin{equation}
\sigma^{-2}_1|\nu^{*},\sigma^{*2} \sim G[\nu^{*}/2,\nu^{*}\sigma^{*2}/2],
\end{equation}

где векторы имеют следующие размеры: $X_1$ --- $N{\times}K$, $X_2$ --- $K\times M$, $\beta_1$ --- $K{\times}1$, $\beta_2$ --- $M{\times}1$, $\Sigma_2$ --- $K{\times}K$, $\beta^{*}$ --- $M{\times}1$ и $\Sigma^{*}$ --- $M{\times}M$. Во второй строке выписана априорное распределение для  регрессионного параметра $\beta_1$, и в третьей строке выписано априорное распределение второго уровня или априорное распределение для априорного распределения $\beta_2$. При этом предполагается, что $\Sigma_2$ известна. Параметры $(\beta^*,\Sigma^*)$ часто называют гиперпараметрами. Что касается параметров дисперсии, в четвертой строке записано априорное распределение для параметра дисперсии $\sigma^2_1$, где $\nu^{*}$ и $\sigma^{*2}$ известны. Нововведением является условие (13.40).

Отметим, что возможно сократить количество уровней и анализировать двухуровневую модель. В частности, мы можем записать двухуровневую модель с информативным априорным распределением одним из следующих способов:

\[
y|X_1,\beta_1,\sigma^2_1{\sim}N[X_1\beta_1,\sigma^2_1I_N],
\]

\[
\beta_1|X_2,\Sigma_2{\sim}N[X_2\beta^*,\Sigma_2+X_2\Sigma^{*}X'_2]
\]

или

\[
y|x_1,X_2,\beta_2,\Sigma_2,\sigma^2_1{\sim}N[X_1X_2\beta_2,\sigma^2_1I_N+X_1\Sigma_2X'_1],
\]

\[
\beta_2{\sim}N[\beta^*,\Sigma^*].
\]

Если $\sigma^2_1$ известно, данная постановка задачи соответствует условно сопряженному с нормальным априорному распределением. Используя полученные ранее результаты мы можем получить  выражения для средних значений апостериорного распределения $\beta_1$ и $\beta_2$, которые соответственно равны матричному средневзвешенному  $\beta^*$ и $\hat{\beta}_1$ или $\beta^*$ и $\hat{\beta}_2$.

Предпосылка о нормальности используется только для примера. Широкое применение на практике получили иерархические модели для обобщенных линейных моделей, в которых используется экспоненциальное семейство распределений (Альберт, 1988).

В иерархических моделях иногда невозможно получить полное апостериорное вероятностное распределение параметров первого уровня, к примеру, $\beta_1$ в аналитическом виде. К счастью, современные симуляционные методы,  представленные в следующем разделе, прекрасно подходят к моделям с иерархической структурой.

Другой подход, являющийся приложением эмпирического байесовского метода, включает оценку параметров априорного распределения верхнего уровня модели, и похоже на метод максимального правдоподобия. В этом подходе, к примеру, не предполагается, что $\Sigma_2$ и $\Sigma^*$ известные матрицы.

\subsection{Многомерное $t$-распределение и распределение Уишарта}

В байесовском анализе за основу можно брать разные распределения. Далее рассмотрено применение байесовского анализа для оценки линейной регрессии на базе двух многомерных распределений при выполнении предпосылки о нормальности.

Многомерное $t$-распределение это многомерный вариант одномерного $t$-распределения Стьюдента. Это распределение аналогично многомерному нормальному распределению, за исключением того, что хвосты $t$-распределения могут быть значительно тяжелее. Для байесовского анализа хвосты распределения более тяжелые, поскольку предельное апостериорное распределение сопряжено с нормальным апостериорным распределением (см. Раздел 13.3.2) или может использоваться непосредственно для $\beta$ в случае, если хвосты распределения тяжелее нормального больше, чем ожидалось. Для $q \times 1$ $t$ случайных величин, имеющих многомерное $t$-распределение Стьюдента, $\nu$ степеней свободы, математическое ожидание $\mu$ и параметр дисперсий $\Sigma$, совместная плотность распределения имеет вид

\[
f_t(t|\nu,\mu,\Sigma)=
\frac{\Gamma((\nu+1)/2)}{\Gamma(\nu/2)(\pi\nu)^{1/2}|\Sigma|^{1/2}} 
\times \left\lbrace 1+\frac{1}{\nu}(t-\mu)'\Sigma^{-1}(t-\mu)\right\rbrace^{-(\nu+q)/2},
\]

где $\Gamma(\cdot)$ это гамма-функция. Распределения симметрично с модой $\mu$, средним $\mu$, если $\nu>1$ и дисперсия равна $[\nu/(\nu-2)]\Sigma$, если $\nu>2$. Хвосты распределения могут быть намного тяжелее, чем у нормального распределения (например, дисперсия равна $3\Sigma$, если $\nu=3$) и нормальное распределение получается при $\nu{\rightarrow}\infty$. Если $z{\sim}N[0,1]$ и $s{\sim}\chi^2(\nu)$, тогда $t=\mu+\Sigma^{\-1/2}z/\sqrt{s/\nu}$ имеет описанное здесь многомерное $t$-распределение. Этот факт помогает получить псевдо-случайную выборку простым способом.

Распределение Уишарта --- многомерный аналог одномерного хи-квадрат распределения, или, в более общем виде, аналог гамма-распределения. В байесовском анализе распределение Уишарта используется в качестве сопряженного априорного распределения для обратной ковариационной матрицы многомерного нормального распределения. Распределение положительно-определенной случайной матрицы $W$ размера  $q\times q$  называется распределением Уишарта со степенями свободы  $\nu{\geq}q$ и матрицей масштаба $S$, если функция плотности равна

\begin{multline}
f_{W}(W|\nu,S)=2^{{\nu}q/2}\pi^{q(q-1)/4}\Pi^{q}_{j=1}\Gamma\left(\dfrac{\nu+1-j}{2}\right) \\
\times|S|^{-{\nu}/2}|W|^{(\nu-q-1)/2}\exp(-\tr(S^{-1}W)/2), 
\end{multline}

где $\Gamma(\cdot)$ --- это гамма функция и $\tr(\cdot)$ обозначает след матрицы. Это распределение имеет математическое ожидание $\nu S$. Выборочная ковариационная матрица для независимых  нормально распределенных векторов имеет  распределение Уишарта. В общем случае, при фиксированном  $\nu(q)$ и независимых векторах размера $q{\times}1$ с $x_j{\sim}N[0,S], j=1,\ldots ,\nu$, сумма $\sum^{\nu}_{j=1}x_{j}x'_j$ будет иметь распределение Уишарта. Если матрица $W^{-1}$  распределена по Уишарту и плотность её распределения равна $f_{W}(W^{-1}|\nu,S)$, тогда $W$ имеет обратное распределение Уишарта  с плотностью

\[
f_{IW}(W|\nu,S)
\]

\[
=2^{{\nu}q/2}\pi^{q(q-1)/4}\Pi^{q}_{j=1}\Gamma\left(\dfrac{\nu+1-j}{2}\right)|S|^{\nu/2}|W|^{-(\nu+q+1)/2}\exp(-\tr(S^{-1}W)/2). 
\]

\subsection{Монте-Карло интегрирование}

Во многих ситуациях невозможно аналитически записать апостериорное распределение параметров. В таких случаях необходимо использовать численные методы для оценки либо всех параметров апостериорного распределения, либо некоторых ключевых характеристик этого распределения, к примеру, апостериорного среднего.

В этом разделе будет рассмотрен расчет основных характеристик апостериорного распределения, без получения самого апостериорного распределения полностью. Для этого возможно использование методов из Главы 12 и с меньшим количеством расчетов, поскольку интеграл необходимо рассчитать всего один раз, а не для каждого отдельного наблюдения на каждой итерации. В следующем Разделе рассмотрены методы генерирования выборки из апостериорного распределения.

\subsection{Сэмплирование по важности}

Предположим, что задача состоит в оценке апостериорной  моментной функции $\E[m(\theta|y)]$, где математическое ожидание рассчитывается по функции плотности $p(\theta|y)$. Необходимо рассчитать значение выражения

\begin{equation}
\E[m(\theta)]=\int_{R(\theta)}m(\theta)p(\theta|y)d\theta.
\end{equation}

Например, апостериорное среднее $k$-го параметра  равно $\E[\theta_k]=\int{\theta_{k}}p(\theta|y)d\theta$. Другие примеры включают  расчет апостериорного стандартного отклонения,  апостериорной частной плотности, апостериорных доверительных интервалов, а также апостериорного математического ожидания заданной функции от параметров.


Из Главы 12 следует, что прямая Монте-Карло оценка интеграла $\E[m(\theta)]$ равна $\hat{E}[m(\theta)]=S^{-1}\sum_{s}m(\theta^s)$, где $\theta^s, s=1,\ldots ,S$ --- это $S$ значений параметра $\theta$, сгенерированных согласно  апостериорному распределению $p(\theta|y)$. Однако эту оценку невозможно получить в текущей байесовской постановке задачи, если не существует аналитического решения для плотности апостериорного распределения (13.1), поскольку при отсутствии аналитической записи плотности невозможно сгенерировать выборку из $p(\theta|y)$. Поэтому мы будем использовать сэмплирование по важности, упомянутое в Разделе 12.7.2. Интеграл в (13.42) может быть записан также в виде:

\begin{equation}
E[m(\theta)]=\int_{R(\theta)}\left(\dfrac{m(\theta)p(\theta|y)}{g(\theta)}\right)g(\theta)d\theta, 
\end{equation}

где $g(\theta)>0$ известная функция плотности, с таким же носителем как и плотность $p(\theta|y)$, выборку из которой легко генерировать. Оценка интеграла методом Монте-Карло соответственно равна

\[
\hat{E}[m(\theta)]=\dfrac{1}{S}\sum^S_{s=1}\dfrac{m(\theta^s)p(\theta^s|y)}{g(\theta^s)},
\]

где $\theta_s, s=1,\ldots ,S$ --- это $S$ значений $\theta$ сгенерированных вспомогательной плотности $g(\theta)$, а не из целевой плотности $p(\theta|y)$. Следует отметить, что требование о совпадении областей определения $p(\theta|y)$ и $g(\theta)$ может вызвать затруднение, если $p(\theta|y)$ зависит от дополнительных параметров или если известна функциональная форма условной плотности, но неизвестна форма частной апостериорной плотности.

При работе с апостериорной плотностью также необходимо разобраться с  константой интегрирования в знаменателе формулы (13.1). Предположим, что $p^{ker}(\theta|y)$ обозначает ядро апостериорной плотности распределения, т.е. $p^{ker}(\theta|y)=L(y|\theta)\pi(\theta)$ или кратно значению этого выражения. Для простоты обозначений далее мы опустим зависимость от $y$. Тогда апостериорная плотность будет равна

\[
p(\theta)=\dfrac{p^{ker}(\theta)}{\int{p^{ker}(\theta)d\theta}},
\]

и, соответственно, математическое ожидание равно

\[
E[m(\theta)]=\int{m(\theta)\left(\dfrac{p^{ker}(\theta)}{\int{p^{ker}(\theta)d\theta}}\right)d\theta}
\]

\[
=\frac{\int m(\theta)p^{ker}(\theta)d\theta}{{\int}p^{ker}(\theta)d\theta}
\]

\[
=\dfrac{{\int}(m(\theta)p^{ker}(\theta)/g(\theta))g(\theta)d\theta}{{\int}(p^{ker}(\theta)/g(\theta))g(\theta)d\theta}.
\]

С помощью сэмплирования по важности мы получаем оценку для $\E[m(\theta)]$ равную

\begin{equation}
\hat{E}[m(\theta)]=\dfrac{\dfrac{1}{S}\sum^{S}_{s=1}m(\theta)p^{ker}(\theta^s)/g(\theta^s)}{\dfrac{1}{S}\sum^{S}_{s=1} p^{ker}(\theta^s)/g(\theta^s)},
\end{equation}

где $\theta^s, s=1,\ldots ,S$ --- это $S$ значений $\theta$, сгенерированных из вспомогательной плотности сэмплирования по важности $g(\theta)$.

Этот метод впервые предложили Клоэк и Ван Дейк (1978). Гевеке (1989) доказал состоятельность и асимптотическую нормальность оценок при выполнении некоторых условий регулярности. 
Эти условия включают следующие предпосылки: $g(\theta)>0$ на носителе $R(\theta)$ плотности $p(\theta)$; $\E[m(\theta)]<\infty$, что означает существование апостериорных моментов; и ${\int}p(\theta|y)d\theta=1$, т.е. апостериорная плотность является собственной. Как ранее отмечалось, обычно мы работаем с ядром  плотности, равным $p^{ker}(\theta|y)=L(y|\theta)\pi(\theta)$, которое может не давать единицу при интегрировании. 
Априорное распределение $\pi()\theta$ может быть несобственным, но для равенства $p(\theta|y)d\theta=1$ требуется выполнение условия ${\int}\pi(\theta)d\theta<\infty$.

Сэмплирование по важности просто в применении, но на практике возникают нюансы, рассмотренные  у Гевеке (1989). 
Основное требование заключается в том, что у плотности $g(\theta)$ должны быть более толстые хвосты, чем у $p(\theta|y)$, чтобы веса сэмплирования по важности $w(\theta)=p(\theta|y)/g(\theta)$ оставались ограниченными. 
Из-за асимптотической нормальности логарифма апостериорной плотности, можно выбрать в качестве $g(\theta)$  многомерное $t$-распределение, со средним равным моде апостериорного распределения, с ковариационной матрицей пропорциональной обращенной матрице Гессе для логарифма апостериорной плотности. При этом количество степеней свободы должно быть достаточно мало для обеспечения тяжести хвостов распределения $g(\theta)$. Также Гевеке (1989) ввел меру относительной вычислительной эффективности, которая равна отношению количества симуляций, необходимых для достижения заданного уровня точности оценивания $\hat{E}[m(\theta)]$, при использовании $g(\theta)$, к количеству симуляций, которое бы потребовалось, если бы можно было делать симуляции непосредственно из плотности  $p(\theta|y)$. Из Главы 12 следует что для интегралов высокой размерности необходимо большее количество симуляций, чтобы достичь хорошей аппроксимации. Также можно использовать методы ускорения симуляций, рассмотренные в Главе 12, такие как антитетическое сэмплирование.

Сэмплирование по важности использует сгенерированные значения $\theta^s$ из плотности $g(\theta)$ с равной вероятностью. Более эффективным было бы взвешивать сгенерированные значения согласно близости  $g(\theta^s)$ к $p(\theta|y)$. Это можно достичь с помощью повторного сэмплирования по важности (см. Гельман и др., 1995).

Также, сэмплирование по важности может быть использовано для расчета различных характеристик апостериорного распределения, которые рассмотрены в Разделе 13.2.5. Среди этих характеристик  можно выделить такие, как оценки квантилей и перцентилей апостериорного распределения, что позволяет построить $95\%$ апостериорный интервал и нарисовать график плотности для $\theta_k$.

\section{Алгоритм Монте-Карло по схеме марковской цепи}

Современная идея байесовского анализа заключается в том, чтобы вместо оценки основных характеристик апостериорного распределения (см. предыдущий раздел), построить большую выборку из апостериорного распределения. Далее с помощью описательных статистик для построенной выборки апостериорного распределения можно будет получить информацию о моментных характеристиках  оценок, а также о других характеристиках, к примеру, о частном распределении параметров или их функций. Например, имея выборку из $S$ значений из апостериорного распределения, можно рассчитать $\E[\theta_k]$ через $s^{-1}\sum_{s}\theta^{s}_k$.

Задача состоит в том, чтобы сгенерировать значения из апостериорного распределения, когда отсутствует аналитическая форма записи для апостериорной плотности. Если существует распределение, подходящее для расчета апостериорных моменты с помощью сэмплирования по важности, тогда оно подходит и для генерирования выборки из апостериорного распределения с помощью алгоритма принятия-отбрасывания, который был рассмотрен в Разделе 12.8. Однако, применение этого метода может быть неэффективно, поскольку большой процент значений может быть отброшен.

Вместо этого генерируются последовательные значения так, что при большой длине последовательности распределение  генерируемых значений сходится к стационарному, совпадающему с целевой  апостериорной плотностью $p(\theta|y)$. Этот подход называется алгоритмом Монте-Карло по схеме марковской цепи (Markov Chain Monte Carlo, MCMC), поскольку предполагает сочетание симулирования (Монте-Карло) и последовательности, являющейся Марковской цепью. После того, как последовательность сойдется, выборка из $S$ симуляций может быть использована для расчета характеристик апостериорного распределения, например, $\E[\theta]$ можно оценить с помощью $\hat{E}[\theta_k]=S^{-1}\sum_{s}\theta^{s}_k$. Полученные симуляции имеют положительную корреляцию, следовательно, точность оценки снижается, поскольку дисперсия будет больше обычного, т.е. больше, чем $(S-1)^{-1}\sum_{s}(\theta^{s}_k-\hat{E}[\theta_k])^2$.

Генерируемая последовательность является цепью Маркова. Для построения широко применяются два алгоритма, алгоритм Гиббса и алгоритм Метрополиса-Хастингса, первый практически является частным случаем последнего, см. Хастингс (1970). Великолепное детальное описание приводят Гельман и др. (1995), Гамерман (1997) и Роберт и Казелла (1999). Мы приводим только краткую схему. 

\subsection{Марковские цепи}

Перед тем как перейти к рассмотрению алгоритма Гиббса и алгоритма Метрополиса-Хастингса необходимо дать основные определения и идеи, которые используются в литературе по MCMC. Нижеприведенные определения даны в контексте дискретной модели. Эти определения могут также распространятся и на непрерывные модели, которые относятся к случаям, когда апостериорное распределение непрерывно по параметрам.

Марковская цепь определяется как последовательность случайных величин $x_n (n=0,1,2,\ldots )$, где $x_n$ принимает значения в конечном множестве $A$ с ядром перехода $K(\cdot)$, которое определяет вероятность того, что $x_n$ будет равно определенному значению, при заданном предыдущем значении $x_{n-j}$. Марковские цепи определяется свойством

\begin{equation}
\Pr[x_{n+1}=x|x_n,x_{n-1},\ldots ,x_0]=\Pr[x_{n+1}=x|x_n],
\end{equation} 

это означает, что распределение $x_{n+1}$ полностью зависит только от предыдущего значения $x_n$. Ядро перехода представляет собой матрицу перехода $T$ с элементами

\begin{equation}
t_{xy}=\Pr[x_{n+1}=y|x_n=x],
\end{equation}

что равно вероятности перехода из $x$ в $y$. Для марковской цепи с конечным числом состояний множество $A$ значений (или состояний)  $x_n$ конечно и состоит из $m$ элементов, тогда $T$ можно записать как



\[ T=
\begin{bmatrix}
t_{11} & \dots & t_{1m} \\
& \vdots & \ddots & \vdots\\
t_{m1} & \dots & t_{mm}
\end{bmatrix}
\]

где $\sum^{m}_{j=1}t_{ij}=1, i=1,\ldots ,m$.

Далее рассмотрим переход из $x$ в $y$ за $n$ шагов. Вероятность указанного перехода $T^n$ равна произведению $n$-раз матриц $T$. Элементы в строках матрицы $T^n$ --- это вероятности   $m$ состояний на $n$-том шаге, и  $j$-ая строка $t^{(n)}_j=(t^{(n)}_{j1},\ldots ,t^{(n)}_{jm})$ представляет собой вероятности перехода из состояния $j$ в другие состояния на шаге $n$. Если начальное распределение вероятностей отдельных состояний $t^{(0)}$, тогда распределение на $n$-ом шаге имеет вид $t^{(n)}=t^{(0)}T^n=t^{(n-1)}T$. Таким образом, распределение вероятностей на $n$-том шаге определяется только начальным распределением и матрицей перехода.

В контексте симуляций, основное внимание сосредоточено на асимптотическом <<поведении>> цепи Маркова при $n{\rightarrow}\infty$. Говорят, что цепь имеет стационарное распределение или инвариантное распределение при вероятности перехода $t_{xy}$, если 

\begin{equation}
\sum_{x \in A}t_{x}t_{xy}=t_y \text{ для } \forall y \in A,
\end{equation}

где $t_{xy}$ --- вероятность перехода от $x$ к $y$. Домножение на матрицу перехода не приводит к изменениям стационарных вероятностей. Существование и единственность стационарного распределения является важным вопросом.

Если стационарное распределение существует и, если $lim_{n{\rightarrow}\infty}t^{(0)}T^{n}_{x,y}=t^*$, тогда цепь будет асимптотически сходится к $t^*$ независимо от начального распределения. В таком случае $t^*$ является предельным распределением. Несмотря на то, что в этой главе стационарное распределение определено для марковских цепей с конечным числом состояний, MCMC методы могут быть также использованы и для бесконечного количества состояний; см. Гилкс, Ричардсон и Шпигельхальтер (1996, стр. 60-61).

Состояние $y$ может быть возвратным (рекуррентным) или невозвратным (транзиентным). Рекуррентное состояние будет повторно посещено с вероятностью 1, а транзиентное состояние имеет положительную вероятность никогда больше не быть посещенным. 

В рамках байесовского подхода цель заключается в генерировании  значений из апостериорного  распределения, $p(\theta)$. При использовании цепей Маркова для генерирования этих значений, начальные значения вектора параметров, $\theta^{(0)}$, которые аналогичны распределению состояний, часто генерируются с помощью переходного ядра.  Используя подходящий метод для генерирования псевдо-случайных чисел, получим новый вектор значений $\theta^{(1)}$ с помощью ядра перехода, оцененного в точке $\theta^{(0)}$, т.е. $K(\theta^{(0)})$. На $n$-том шаге значения генерируются с помощью переходного ядра $K(\theta^{(n-1)})$. Используемая марковская цепь такова, что при $n{\rightarrow}\infty$ апостериорную плотность  стремится к $p(\theta)$. Когда сходимость к предельному распределению произошла, все последующие значения генерируются из этого распределения, хотя они и являются коррелированными.

Эти идея являются интуитивном объяснением  MCMC алгоритмов, с помощью которых возможно восстановить байесовское апостериорное распределение для многих разных, в том числе многомерных, моделей, к примеру, линейной иерархической модели, рассмотренной в разделе 13.3.4. Предполагая, что задано ядро перехода $K(\theta^{n-1},\cdot)$ с помощью которого можно генерировать значения $\theta$ и внутри которого заложено предельное распределение, целевое апостериорное распределение может быть получено в смысле сколь угодно близкого приближения.

Текущее описание имеет общий характер. На практике, не существует единого выбора ядра перехода и возможно построение разных цепей Маркова. Некоторые варианты могут быть более удачными, чем другие с точки зрения сходимости к предельному распределению. Если сходимость происходит слишком медленно и необходимо производить много расчетов, возможно имеет смысл построить другую цепь Маркова. Также стоит отметить, что нужны критерии сходимости, чтобы определить насколько близко распределение к целевому распределению на $n$-том шаге. 

\subsection{Алгоритм Гиббса}

Вначале рассмотрим алгоритм Гиббса, подход, который относится к MCMC классу, может быть легко описан и применим.

Допустим, что вектор $\theta=[\theta_{1} \theta_2]^{\prime}$ имеет апостериорную плотность $p(\theta)=p(\theta_{1},\theta_{2})$, где для простоты обозначений мы опускаем зависимость от $y$. Если известны все условные плотности, то генерируя  значений цепи поочередно то из $p(\theta_{1}|\theta_{2})$, то  из $p(\theta_{2}|\theta_{1})$,  мы получим в пределе значения из распределения  $p(\theta_{1},\theta_2)$. Для расчета требуется знание обеих плотностей, $p(\theta_{1}{|}\theta_{2})$ и $p(\theta_{2}|\theta_{1})$, данное условие не всегда выполнено на практике.

\subsubsection*{Пример}

В качестве простой иллюстрации рассмотрим двумерные нормально распределенные данные  с равномерной априорной плотностью для среднего и известной ковариационной матрицей. Допустим, что $y=(y_1,y_2){\sim}N[\theta,\Sigma]$, где $\theta=[\theta_{1},\theta_{2}]'$ и диагональные элементы матрицы $\Sigma$ равны 1, а внедиагональные элементы --- $\rho$. Тогда с учетом заданной априорной плотности  $\theta$ возможно показать что апостериорное распределение является двумерным нормальным распределением, $\theta|y{\sim}N[\overline{y},N^{-1}\Sigma]$. Поскольку условные плотности известны


\[
\theta_1|\theta_2, y{\sim}N[(\overline{y}_1+\rho(\theta_2-\overline{y}_2)),(1-\rho^2)/N],
\]

\[
\theta_2|\theta_1, y{\sim}N[(\overline{y}_2+\rho(\theta_1+\overline{y}_1)),(1-\rho^2)/N],
\]

мы можем делать итерации, генерируя значения из условного нормального распределения, используя обновленные значения $\theta_1$ и $\theta_2$. Если цепь достаточно длинная, тогда распределение будет сходиться к двумерному нормальному. В этом примере легко напрямую сгенерировать значения из совместного распределения $\theta|y$, используя разложение Холецкого, которое дано в Разделе 12.8, однако в других примерах возможно генерирование  значений только из условного, а не совместного распределения.

\subsubsection*{Алгоритм Гиббса}

В более общем случае, рассмотрим $q$-мерное целевое распределение $p(\theta)$, где обозначение зависимости от данных опускается. Предположим, что вектор параметров $\theta$ разбит на $d$ блоков. К примеру, в линейной регрессионной модели $\theta'=[\beta {\sigma}^{2}]'$. Допустим, что $\theta_k$ обозначает $k$-ый блок и $\theta_{-k}$ все компоненты вектора $\theta$, которые не входят в $\theta_k$. Предположим, что условное распределение, $p(\theta_{k}|{\theta}_{-k}), k=1,\ldots ,d$ известно. Тогда последовательное сэмплирование может быть организовано согласно следующей процедуре:

1. Возьмем начальные значения $\theta$, $\theta^{(0)}=(\theta^{(0)}_1,\ldots ,\theta^{(0)}_d)$.

2. Следующая итерация состоит в пошаговом изменении всех компонент вектора $\theta$, что дает в результате вектор $\theta^{(1)}=(\theta^{(1)}_1,\ldots ,\theta^{(1)}_d)$, значения которого сгенерированы последовательно с помощью $d$ генерирований из $d$ условных распределений:

\[
p(\theta^{(1)}_1|\theta^{(0)}_2,\ldots ,\theta^{(0)}_d)
\]

\[
p(\theta^{(1)}_2|\theta^{(1)}_1,\theta^{(0)}_3\ldots ,\theta^{(0)}_d)
\]
\[
\vdots
\]
\[
p(\theta^{(1)}_d|\theta^{1}_1,\theta^{(1)}_2,\ldots ,\theta^{(1)}_{d-1}).
\]

3. Возвращаемся на шаг 1, в качестве стартового вектора $\theta$ берём $\theta^{(1)}$ и делаем шаг 2 для того, чтобы построить новый вектор $\theta^{(2)}$. Повторяем шаги 1 и 2 до тех пор, пока не будет достигнута сходимость.

В работе Гилкса др. (1996, стр.7) представлено краткое доказательство того, что стационарное распределение данной цепи является апостериорным. После того как сходимость будет достигнута, значения формируются из целевого совместного апостериорного распределения. Геман и Геман (1984) показали, что стохастическая последовательность ${\theta^{n}}$ является марковской цепью с нужным стационарным распределением. Гельфанд и Смит (1990) показали, что, при выполнении некоторых условий, цепь сходится к стационарному апостериорному распределения при количестве циклов формирования значений из полного множества условных распределений стремящемся к бесконечности. Также см. Таннер и Вонг (1987). По достижении сходимости, можно сгенерировать большую выборку и использовать её для подсчета выборочных аналогов апостериорных моментов частного или совместного распределений.

В вышеупомянутых работах не оговаривается необходимое количество циклов, которое требуется произвести, чтобы достичь сходимости, поскольку это количество может варьироваться в зависимости от модели. Важно убедиться в том, что было произведено необходимое количество циклов для достижения сходимости цепи. Существует множество тестов для проверки  сходимости. Поскольку оценки апостериорных моментов должны быть основаны на значениях, сформированных из апостериорного распределения, как правило первые полученные значения цепи отбрасываются, эти значения называются периодом прожига (burn-in phase).

Алгоритмы последовательной симуляции могут быть изменены таким образом, что каждое значение будет зависеть не только от последнего сформированного значения, но также от более ранних значений. Основное требование заключается в том, что вероятность улучшения текущей аппроксимации апостериорного распределения должна быть положительна и желательно высока. Привлекательность более ограничивающего марковского подхода заключается в более простых доказательствах сходимости  распределения к целевому апостериорному распределению.

Для байесовского анализа алгоритм Гиббса является полезным инструментом, когда для совместного апостериорного распределения нет аналитической формы записи, а для всех условных распределений она есть. Во многих приложениях в значительной степени используются изобретательность и знание о сопряженных плотностях и других байесовских результатах, многие  из которых были получены до распространения симуляционных методов. Эти результаты нужны для того, чтобы специфицировать априорные плотности распределения для которых возможно рассчитать полные условные распределения. 

Далее рассмотрим два примера применения MCMC методов.

\subsubsection*{Пример линейной регрессии}

В Разделе 13.3.2 мы проанализировали апостериорное распределение нормальной линейной гомоскедастичной регрессионной модели при нормальной и гамма сопряженных априорных плотностях. Ранее было показано, что $\beta$ при заданном $\sigma^{-2}$ имеет многомерное нормальное условное апостериорное распределение и условное распределение $\sigma^{-2}$ равно гамма-распределению. Хотя можно получить совместное апостериорное  распределение  в явной форме (см. (13.32)), гораздо проще применить алгоритм Гиббса для формирования большой выборки из совместного апостериорного распределения. Цепь получается путем последовательного генерирования  значений, полученных из нормального условного распределения при фиксированном  параметре точности $\sigma^{-2}$ и из гамма распределения при фиксированном $\beta$.

Структура алгоритма аналогична структуре, которая будет использована далее, в Разделе 13.6 для более сложного случая, когда вместо одного уравнения берется система из двух внешне не связанных уравнений регрессии.

Во многих случаях естественно работать с блоками параметров. Например, в случае множественной линейной регрессионной модели многих уравнений с недиагональной одновременной ковариационной матрицей, параметры условного среднего $(\beta_1,\beta_2,\ldots )$ формируют один блок  и $\Sigma$ формирует второй. Условные распределения можно записать в следующей форме $\beta_1,\beta_2,\ldots |data,\Sigma$ и $\Sigma|data,\beta_1,\beta_2,\ldots $. Чиб и Гринберг (1996, стр. 418-419) рассмотрели применение алгоритма Гиббса для этого случая.

\subsubsection*{Пример иерархического априорного распределения}

Схема Гиббса широко применяется для анализа моделей иерархического априорного распределения. С учетом структуры линейной иерархической модели, которая задана уравнениями (13.39)-(13.41), легко построить марковскую цепь на базе полного множества условных распределений. Аналогичный общий подход может быть расширена для случая нелинейной иерархической априорной модели, при этом может потребоваться осуществление дополнительных шагов если нелинейность появляется в сочетании с  латентной переменной (Альберт, 1988).

\subsection{Алгоритм Метрополиса}

Алгоритм Гиббса --- наиболее известный MCMC алгоритм. Алгоритм Гиббса имеет ограничения в применения, ввиду необходимости создания выборки из  условного распределения, которое может быть неизвестно. Возможно расширить применение MCMC методов с помощью алгоритма Метрополиса и алгоритма Метрополиса-Хастингса. В своей работе Чиб и Гринберг (1995) дают введение и рекомендуют  другие источники. Ниже дано краткое описание алгоритмов.

Согласно алгоритму Метрополиса строится последовательность $\left\lbrace\theta^{(n)}, n=1,2,\ldots \right\rbrace$, распределение значений которой сходится к целевому апостериорному распределению и предполагается, что значение этого распределения можно посчитать с точностью до нормализующей константы.

Для простоты обозначений вновь опустим зависимость функции плотности $p(\theta|y)$ от $y$. Алгоритм включает следующие шаги:

1. Формируем стартовые значение $\theta^{(0)}$ из первичной аппроксимации апостериорного распределения, при этом должно выполняться условие $p(\theta^{(0)})>0$. Например, можно сгенерировать значения из многомерного $t$-распределения с центром равным моде частного апостериорного распределения.


2. Положим $n=1$. Сгенерировать предложение $\theta^*$ из симметричного вспомогательного распределения $J(\theta^{(1)}\mid \theta^{(0)})$, которое для любой пары $(\theta^a,\theta^b)$ обладает свойством $J(\theta^{a}\mid \theta^{b})=J_1(\theta^b\mid \theta^a)$. Примером может служить $\theta^{(1)}\mid \theta^{(0)} \sim N[\theta^{(0)},V]$ при фиксированной матрице $V$. Симметричность распределения предложения упрощает вычисления, но некритична.

3. Рассчитать отношение плотностей $r=p(\theta^*)/p(\theta^{(0)})$

4. Положить $\theta^{(1)}$ равной 

\[
\theta^{(1)}= \begin{cases}
\theta^* \text{ с вероятностью } \min(r,1) \\
\theta^{(0)} \text{ с вероятностью } (1-\min(r,1))
\end{cases}
\]


Это означает, что $\theta^{(1)}$ --- смесь распределений $\theta^*$ и  $\theta^{(0)}$.

5. Вернуться к шагу 2, увеличить счетчик шагов и повторить последующие шаги.

6. После большого количества шагов провести тесты на сходимость. Если тесты не отвергают сходимость, можно считать, что найдено апостериорное распределение.



Этот алгоритм может быть рассмотрен как итерационный метод максимизации $p(\theta)$. Если  $\theta^{*}$ увеличивает $p(\theta)$, тогда обязательно $\theta^{(n)}=\theta^{*}$, если же  $\theta$ уменьшает значение $p(\theta)$, тогда $\theta^{(n)}=\theta^*$ с вероятностью $r<1$. 

Этот алгоритм --- аналог подхода принятия-отказа (см. Раздел 12.8), на этот раз не существует ограничения, что функция плотности предложения домноженная на константу должна накрывать апостериорную функцию плотности.

Алгоритм Метрополиса позволяет создать марковскую цепь, которая имеет свойства обратимости, неприводимости и рекуррентности по Харрису, что гарантирует сходимость к стационарному распределению. Гельман и др. (1995) показали, что стационарное распределение есть ожидаемое апостериорное распределение $p(\theta)$. Допустим, что $\theta_a$ и $\theta_b$ две точки такие, что $p(\theta_b){\geq}p(\theta_a)$. Если $\theta^{(n-1)}=\theta_a$ и $\theta^{*}=\theta_b$, тогда с уверенностью можно сказать, что $\theta^{(n)}=\theta_b$ и $\Pr[\theta^{(n)}=\theta_b,\theta^{(n-1)}=\theta_a]=J_{n}(\theta_b|\theta_a)p(\theta_a)$. Если поменять порядок и задать, что $\theta^{(n-1)}=\theta_b$ и $\theta^*=\theta_a$, тогда $\theta^{(n)}=\theta_a$ с вероятностью $r=p(\theta_a)/p(\theta_b)$ и $\Pr[\theta^{(n)}=\theta_{a}, \theta^{(n-1)}=\theta_b]=J_{n}(\theta_a|\theta_b)p(\theta_b)[p(\theta_a)/p(\theta_b)]=J_{n}(\theta_a|\theta_b)p(\theta_a)=J_{n}(\theta_b|\theta_a)p(\theta_a)$, предполагая, что распределение предложения симметрично. Таким образом, можно сделать вывод, что частные плотности  $\theta^{(n)}$ и $\theta^{(n-1)}$ одинаковы, поскольку их совместное распределение симметрично, таким образом,  $p(\theta)$ является симметричным стационарным распределением марковской цепи.



\subsection{Алгоритм Метрополиса-Хастингса}

Результаты применения алгоритма Метрополиса могут меняться в зависимости от выбранного начального приближения  и выбора распределения предложения. Возможной проблемой является медлительность алгоритма Метрополиса, которая может возникнуть, если замена текущих значений параметров новыми не происходит с нужной частотой, что приводит к замедлению движения цепи. Выполнение алгоритма может быть ускорено путем использования несимметричных предлагающих распределений.

Процедура алгоритма Метрополиса-Хастингса аналогична процедуре алгоритма Метрополиса, за исключением того, что на шаге 2 предлагающее распределение не обязательно должно быть симметричным и на шаге 3 вероятность принятия решений $r$ для всех значений $n$ становится равной 

\[
r_n=\dfrac{p(\theta^{*})/J_{n}(\theta^{*}|\theta^{(n-1)})}{p(\theta^{(n-1)})/J{n}(\theta^{(n-1)}|\theta^*)}=\dfrac{p(\theta^*)J_{n}(\theta^{(n-1)}|\theta^*)}{p(\theta^{(n-1)})J_{n}(\theta^*|\theta^{(n-1)})}.
\]

Оставшиеся шаги выполняются в соответствии с этим изменением. Отметим, что если в $p(\cdot)$ или в $J(\cdot)$ есть нормализующая константа, то в $r_n$ нормализующие константы сокращаются. Таким образом, значение обеих плотностей, апостериорной и предлагающей, могут быть рассчитаны до  константы пропорциональности. См. Хастингс (1970).

\subsection{Примеры алгоритма Метрополиса-Хастингса}

Отличные друг от друга предлагающие распределения приводят к различным алгоритмам с разной степенью эффективности, в том смысле, что отличается количество симуляций, необходимых для генерирования значений из апостериорного распределения. Далее рассмотрим несколько примеров. Следует отметить, что существуют разные рекомендации  для выбора предлагающего распределения, за исключением рекомендации применять алгоритм Гиббса, когда это возможно.  

Алгоритм Гиббса можно рассматривать как частный случай алгоритма Метрополиса-Хастингса. Если вектор $\theta$ разбит на $d$ блоков, тогда на $n$-том шаге алгоритма будет сделано $d$ шагов Метрополиса. Предлагающее распределение --- это  условное распределение, данное  в Разделе 13.5.2. Можно показать, что вероятность принятия здесь всегда равна единице. % здесь есть английский синоним, но в русском его нет

Возможно использовать несколько стратегий, применяя различные переходные ядра  для разных подмножеств параметров. Например, шаг алгоритма Метрополиса-Хастингса может применяться совместно со схемой Гиббса, которую удобно использовать, если возможно непосредственно генерировать значения из соответствующего условного распределения.

В алгоритме Метрополиса-Хастингса с независимыми предложениями все значения предлагающего распределения генерируются  из фиксированной плотности $g(\theta)$, в таком случае вероятность принять упрощается и становится равной выражению $r_n=w(\theta^{*})/w(\theta^{(n-1)})$, где веса значимости равны $w(\theta)=p(\theta)/g(\theta)$. Алгоритм Метрополиса-Хастингса со случайным блужданием генерирует предложения по правилу $\theta^*=\theta^{(n-1)}+\varepsilon$, где $\varepsilon$ генерируется согласно плотности $g(\varepsilon)$.

Гельман и др. (1995, стр. 334) рассматривают симулирование $q$-мерного нормального распределения с ковариационной матрицей $\Sigma$. Для алгоритма Метрополиса со предлагающим  распределением $\theta^{*}|\theta^{(n-1)}{\sim}N[\theta^{(n-1)},c^{2}\Sigma]$ при выборе $c{\simeq}2.4/\sqrt{q}$ дает наибольшую эффективность по сравнению с прямым генерированием из $q$-мерного нормального распределения. Эффективность достигает примерно $0.3$, по сравнению с алгоритмом Гиббса, когда эффективность равна $1/q$, если $\Sigma=\sigma^{2}I_q$.

\section{Пример MCMC: алгоритм Гиббса для внешне не связанных уравнений}

Проиллюстрируем применение алгоритма Гиббса, чтобы проанализировать модель внешне не связанных уравнений. Этот пример  немного более сложный, чем применение к регрессии с одним уравнением, поскольку учитывается, что ошибки уравнений взаимосвязаны между собой.

Рассмотрим пример модели с двумя уравнениями для $i$-го наблюдения

\[
y_{1i}=x'_{1i}\beta_1+\epsilon_{1i},
\]

\[
y_{2i}=x'_{2i}\beta_2+\epsilon_{2i},
\]

где $(\varepsilon_1,\varepsilon_2)$ имеют двумерное нормальное распределение со средним значением ноль и ковариационной матрицей 

\[
\Sigma=
\begin{bmatrix} \sigma_{11} \sigma_{12}\\ \sigma_{21} \sigma_{22} \end{bmatrix}.
\]

Комбинируя эти два уравнения, получим, что для $i$-го наблюдения 

\[
y_{i}=x'_{i}\beta+\epsilon_{i},
\]

где $\varepsilon{\sim}N[0,\Sigma]$. В итоге получим следующее выражение для процесса порождающего данные

\[
y_{i}|x_{i},\beta,\Sigma,\sim N[x'_{i}\beta,\Sigma]
\]

и основная цель состоит в  оценке апостериорного среднего регрессионного параметра $\beta$ и ковариационной матрицы $\Sigma$, при заданных значениях $y,X$.

Рассмотрим независимые априорные распределения, где

\[
\beta \sim N[\beta_{0},B^{-1}_0],
\]

\[
\Sigma^{-1} \sim Wishart[\nu_{0},D_0],
\]

$B_0$ определяет точность, значение которой обратно к ковариационной матрице априорного распределения и обратное распределение Уишарта, описанное в Разделе 13.3.5, является обобщением обратного гамма-распределения. Альтернативный способ, который не рассматривается здесь, заключается в том, чтобы использовать зависимые априорные распределения, по аналогии с тем, как это было сделано в Разделе 13.3.2, что даст $\beta|\Sigma  \sim N[\beta_0,w_{0}\Sigma]$ для заданного $w_0$.

Проведя математические преобразования, получим следующие условные апостериорные распределения

\[
\beta|\Sigma,y,X{\sim}N\left[C_{0}\left(B_{0}\beta_{0}+\sum^{N}_{i=1}x'_{i}\Sigma^{-1}y\right)_{i},C_0\right], 
\]

\[
\Sigma^{-1}|\beta,y,X{\sim} Wishart \left[\nu_{0}+N,\left(D^{-1}_0+\sum^{N}_{i=1}u'_{i}u_{i}\right)^{-1}\right], 
\]

где $C_0=(B_{0}+\sum^{N}_{i=1}x'_{i}\Sigma^{-1}x_i)^{-1}$ и $u_{i}=y_{i}-x'_{i}\beta$. Возможно использовать алгоритм Гиббса, поскольку известны условные апостериорные распределения, а  сэмплирование из них не представляет трудностей.

Таблица 13.3. Алгоритм Гиббса: пример внешне не связанных уравнений

\begin{tabular}{p{4cm}p{2cm}p{2cm}p{2cm}p{2cm}p{2cm}}
\hline 
Параметр априорного распределения, $\tau$ & $\tau$=10 & $\tau=1$ & $\tau=1/10$ & $\tau=10$ & $\tau=10$ \\ 
Размер выборки, $N$ & 1,000 & 1,000 & 1,000 & 1,000 & 10,000 \\ 
Репликации алгоритма Гиббса & 50,000 & 50,000 & 50,000 & 100,000 & 100,000 \\ 
\hline 
$\beta_{11}$ (своб-ый член ур-ия 1) &  0.971 (0.0310) & 1.013 (0.0312) & 0.983 (0.0316) & 1.020 (0.0324) & 1.010 (0.0100) \\ 
$\beta_{12}$ (коэф-нт наклона ур-ия 1) & 1.026 (0.0265) & 0.9835 (0.0271) & 1.006 (.0265) & 1.006 (.0268) & 1.015 (0.0086) \\ 
$\beta_{21}$ (своб-ый член ур-ия 2) & 1.016 (0.0309) & 0.972 (0.0325) & 0.993 (0.0322) & 1.017 (0.0326) & 0.991 (0.0100) \\ 
$\beta_{22}$ (коэф-нт наклона ур-ия 2) & 0.983 (0.0256) & 0.992 (0.0285) & 0.979 (0.0272) & 1.005 (0.0277) & 1.007 (0.0085) \\ 
$\sigma_{11}$ (дисперсия ошибок ур-ия 1) & 0.960 (0.0429) & 0.969 (0.0434) & 1.012 (0.0453) & 1.043 (0.0466) & 1.010 (0.0143) \\ 
$\sigma_{12}$ (ковариация ошибок) & -0.499 (0.0340) & -0.507 (0.0358) & -0.519 (0.0368) & -0.576 (0.0379) & -0.515 (0.0113) \\ 
$\sigma_{22}$ (дисперсия ошибок ур-ия 2) & 0.950 (0.425) & 1.066 (0.0476) & 1.049 (0.0467) & 1.062 (0.0472) & 1.002 (0.0141) \\ 
\hline 
\end{tabular} 

Для проверки сходимости, например, можно рассчитать выборочные среднее и стандартное отклонение итоговых значений и посмотреть меняются ли  или остаются неизменными. В случае, если изменения незначительны, к примеру, менее чем 0,1 при 10,000 симуляций, тогда делается вывод о сходимости. Также возможно рассмотреть несколько цепей Маркова. Симуляции всегда будут коррелированы между собой, важно лишь, какова скорость сходимости автокорреляционной функции к нулю. Иногда высокая автокорреляция является внутренним свойством алгоритма. Тогда можно взять каждое десятое или сотое значение для уменьшения автокорреляции. 

Для проверки сходимости распределения алгоритма Гиббса к стационарному апостериорному распределению мы рассчитали  20-ть коэффициентов автокорреляции для симуляций апостериорного распределения после большого количества значений. Наличие автокорреляции в коэффициентах после прожига может свидетельствовать об отсутствии сходимости в целевом распределении. Когда количество симуляций мало, к примеру, 1,000, коэффициент автокорреляции иногда может превышать 0,06. Однако, если количество симуляций более 50,000, практически отсутствует автокорреляция до 20-го порядка и с увеличением количества симуляций автокорреляция уменьшается. В большинстве случаев, оценка коэффициентов корреляции не превышает 0.005. Можно легко проверить, что для N=1,000  априорный параметр $\tau$ практически не оказывает влияния на апостериорное распределение.


\subsection{Пополнение данных}

В некоторых случаях алгоритм Гиббса может использоваться для более широкого ряда моделей путем введения вспомогательных переменных. В частности, он подходит для моделей с латентными переменными, например, модели дискретного выбора,  модели урезанных и цензурированных данных и модели  смеси распределений, рассматриваемых в последних главах.

В скалярном случае значения латентной зависимой переменной $y^{*}$ ненаблюдаемы; наблюдаемы только значения $y=g(y^{*})$ для  некоторой функции $g$. % здесь опечатка в английском тексте
 Например, для логит и пробит моделей (см. Главу 14) известен только знак $y^{*}$, т.е. $y=1$, если $y^{*}>0$ и $y=0$, если $y^{*}{\leq}0$. 

Байесовский подход к анализу моделей скрытых переменных и, в частности, использование алгоритма Гиббса, хорошо дополняется заменой скрытых переменных на их оцененные  значения. 
Этот шаг доступен, если возможно записать прогнозную плотность скрытых переменных через наблюдаемые переменные. Процедура добавления предполагаемых значений как если бы эти значения были известны, называется пополнением данных. В качестве примера, в разделе 10.3.7 был рассмотрен EM алгоритм. В своей работе Таннер и Вонг (1987) продемонстрировали, что иногда генерировать выборку из апостериорного распределения затруднительно, в то время как после пополнения данных можно использовать алгоритм Гиббса.

Допустим, что апостериорную плотность распределения можно выразить через наблюдаемую переменную $y$ и ненаблюдаемую переменную $y^{*}$,

\begin{equation}
p(\theta|y)=\int_{y^{*}}p(\theta|y,y^{*})f(y^{*}|y)dy^{*},
\end{equation}

где интеграл в правой части может быть проинтерпретирован как усреднение по $y^{*}$. 

Аналогично с EM алгоритмом, метод пополнения данных предполагает повторение цикла из двух шагов: I --- imputation, пополнение данных, P --- posterior, уточнение апостериорного распределение. 


На первом шаге генерируются значения из плотности условного распределения $y^{*}$. При этом происходит интегрирование по $\phi$, которое появляется в распределении,  связывающем $y^{*}$ и $y$. Прогнозное распределение равно 

\begin{equation}
f(y^*|y)=\int_{\phi}f(y^{*}|y,\phi)f(\phi|y)\,d\phi.
\end{equation}

Имея сгенерированное значение согласно $p(\theta|y)$ мы можем сгенерировать значение из $f(y^{*},y)$, повторяя оба шага $m$ раз мы получим  вектор $y_{i}, i=1,\ldots,m$. Это завершает I-шаг.

В результате пополнения данных на $I$-ом шаге, $P$-шаг реализуется путем обновления текущего приближения к апостериорному распределению; а именно:

\begin{equation}
\text{Updated } p(\theta|y)=\frac{1}{m}\sum_{i=1}^{m}p(\theta|y,y_i^*).
\end{equation}

Далее, алгоритм возвращается на $I$-ый шаг.

При $m=1$, процедура состоит в оценке интеграла  (13.49) по алгоритму Гиббса. Применение метода пополнения данных для устранения проблемы пропущенных наблюдений подробно рассмотрено в главе 26.

\section {Байесовский выбор моделей}

В 7-ой и 8-ой главах были рассмотрены  вопросы тестирования гипотез, спецификации модели и сравнения моделей с  помощью частотного подхода к понятию вероятности. Этот раздел посвящен   изучению байесовских факторов ---  основного инструмента, используемого в байесовском анализе для оценивания степени уверенности в верности  гипотезы,  альтернативы тестирования нулевой гипотезы. Также байесовские  факторы могут  быть критерием выбора одной из моделей вне зависимости от того, являются ли они вложенными или нет. В эконометрической литературе, одним из первых исследователей, кто рассматривал байесовские факторы как критерий выбора модели, стал Зеллнер (1971, 1978). Изложение материала  построено на обзорной статье Касс и Рафтери (1995).

Обозначим вектор данных буквой $y$  и с помощью $H_1$ и $H_2$ --- две гипотезы, возможно невложенные, априорные вероятности равны $\Pr[H_1]$ и $\Pr[H_2]$, соответственно. Процессы Вероятности получения данных определенных значений обозначим через $\Pr[y|H_1]$ и $\Pr[y|H_2]$. % здесь в английском оригинале написана чушь :)
В соответствии с информацией из выборки  априорные вероятности преобразуются в апостериорные. Согласно теореме Байеса :

\begin{equation}
\Pr[H_k|y]=\dfrac{\Pr[y|H_k]\Pr[H_k]}{\Pr[y|H_1]\Pr[H_1]+\Pr[y|H_2]\Pr[H_2]}, k=1,2,
\end{equation}

и апостериорное отношение шансов равно

\begin{equation}
\dfrac{\Pr[H_{1}|y]}{\Pr[H_2|y]}=\dfrac{\Pr[y|H_1]\Pr[H_1]}{\Pr[y|H_2]\Pr[H_2]}{\equiv}B_{12}\dfrac{\Pr[H_1]}{\Pr[H_2]},
\end{equation}
 
где $B_{12}=\Pr[y|H_1]/\Pr[y|H_2]$ --- это байесовский фактор. Гипотеза 1 предпочитается, если апостериорное отношение шансов больше 1. Правая часть выражения (13.53) это апостериорное отношение шансов, которое равно произведению байесовского фактора и априорного отношения шансов. Если априорные вероятности равны между собой, т.е. $\Pr[H_1]=\Pr[H_2]$, тогда байесовский фактор равен отношению шансов в пользу $H_1$. При рассмотрении нескольких гипотез байесовский фактор рассчитывается для всех пар гипотез. Байесовский фактор определен даже если гипотезы не являются вложенными. 

Значение байесовского фактора рассчитывается как отношение фукнций правдоподобия. Значение фактора зависит от неизвестных параметров, обозначенных за $\theta_1$ и $\theta_2$, которые исчезают при усреднении или интегрировании  по априорной вероятности так, что

\begin{equation}
\Pr[y|H_k]=\int{\Pr[y|\theta_k,H_k]\pi(\theta_k|H_k)d\theta}, k=1,2,
\end{equation}

Согласно выводам раздела 13.2.5 значение выражения (13.54) равно частной или прогнозной вероятности данных при заданном априорном распределении.

При нахождении значения выражения могут возникнуть трудности из-за зависимости подынтегрального выражения от всех констант функции правдоподобия. Значения констант можно опустить при расчете апостериорной вероятности, но для расчета байесовских факторов значения факторов важны. Может потребоваться численный расчет подынтегрального выражения в (13.54) при отсутствии решения в явном виде, например, с помощью сэмплирования по важности. Подробный обзор литературы по расчету байесовского фактора представлен в работе Касс и Рафтери (1995). Вместе с тем, существуют асимптотические приближения для расчета байесовского фактора, которые легко рассчитываются пакетами, решающими задачу максимального правдоподобия.

Таблица 13.4. Интерпретация байесовских факторов 

\begin{tabular}{p{4cm}p{3cm}p{3cm}}
\hline 
Байесовский фактор $B_{12}$ & $2\ln B_{12}$ & Свидетельства в пользу $H_1$ \\
от 1 до 3 & от 0 до 2 & слабые \\
от 3 до 20 & от 2 до 6 & умеренные \\
от 20 до 150 & от 6 до 10 & сильные \\
больше 150 & больше 10 & очень сильные 
\end{tabular}


Проста интерпретация байесовского фактора через аргументы в пользу гипотезы $H_1$. «Байесовский фактор отражает в численном виде свидетельства, содержащиеся в данных, в пользу той или иной теории, выраженной с помощью статистической модели» (Касс и Рафтери, 1995, стр. 777). В частотном подходе к вероятности часто используют удвоенное значение логарифмического отношения правдоподобия. По аналогии, в байесовском подходе часто приводят удвоенный логарифм  байесовского фактора. В своей работе Рафтери и Касс приводят деление степени уверенности в гипотезе $H_1$ на несколько групп,  см.  таблицу 13.4.

Предположим, что поставлена задача сравнить две вложенные модели. Обозначим за $H_0$ модель с ограничениями и за $H_1$ модель без ограничений. Как было показано ранее для парного сравнения двух моделей через показатель апостериорного отношения шансов требуется расчет байесовского фактора. Для нулевой гипотезы байесовский фактор рассчитывается как:

\[
B_{01}=\frac{m(y|H_0)}{m(y|H_1)},
\]

где $m(y|H_j)$ частная функция правдоподобия соответствующая гипотезе $H_j$. Если модели $H_0$ и $H_1$ вложены, для расчета байесовских факторов можно использовать подход отношения  плотностей Сэвиджа-Дики (см. Вердинелли и Вассерман, 1995).

Важный вклад в методику расчета байесовского фактора сделал Чиб (1995), предложив существенно более простой метод, применимый не зависимо от того, являются ли модели вложенными. В основе подхода Чиба (1995) лежат две, связанные между собой идеи. Первая состоит в записи предельной плотности $m(y)$ для модели $H_k$ как отношения:

\[
m(y)=\dfrac{f(y|\theta)\pi(\theta)}{\pi(\theta|y)},
\]

где числитель равен произведению правдоподобия, включая константу, и априорной вероятности, а знаменатель --- апостериорная плотность $\theta$.
%%%%%%%%%% here!
Данное выражение получается в результате перегруппировки величин в (13.1), где $f(y)$ (ранее $\Pr[y|H_k]$) обозначена $m(y)$. Смысл его всего лишь в том частная плотность --- это  нормализующая константа. Во-вторых, успешное применение алгоритма MCMC, позволит получить оценку апостериорной плотности $\pi(\tilde{\theta}|y)$ в точке $\tilde{\theta}$. Следовательно,

\begin{equation}
\ln \hat{m}y=\ln {f(y|\tilde{\theta})}+\ln {\pi(\tilde{\theta})}-\ln {\pi(\tilde{\theta}|y)}
\end{equation}

Таким образом, если известны значения выражений в правой части уравнения, частную плотность можно рассчитать используя алгоритм Гиббса. Этот подход расширили Чиб и Желязков (2001), а именно, вместо алгоритма Гиббса авторы использовали алгоритм Метрополиса-Хастингса.

В сложных и сильно-параметризованных моделях, расчет байесовского фактора может вызвать затруднения. Вместе с тем,  критерий Шварца, также известный как байесовский информационный критерий (BIC) (см. раздел 8.5), может дать грубое приближение логарифма байесовского фактора. Формула для расчета BIC: $BIC=-2\ln {L}(\hat{\theta}_{ML})+\ln {N_q}$. BIC легко рассчитать, если известно значение логарифма функции правдоподобия.

Из выражения (13.52) следует, что отношение априорных вероятностей играет важную роль при выборе модели. Как правило, исследователь может самостоятельно выбрать эти вероятности. Данный вопрос  рассматривается в литературе, посвященной чувствительности байесовского фактора к выбору априорных вероятностей.

\section{Практические соображения}

В современной литературе по байесовским методам основное внимание уделяется марковским цепям. Построение марковских цепей требует большого количества вычислений, что требует хорошего программного обеспечения. На момент написания данной работы, последняя версия пакета BUGS (Bayesian inference Using Gibbs Sampling) --- WinBUGS -- была признана лучшей, особенно для иерархических моделей и решения проблем пропущенных данных. Последнюю версию BUGS можно найти на сайте. Подробная информация о программных пакетах, предназначенных для работы с байесовскими методами, содержится в работе Гамерманf (1997, Раздел 5.6).

В настоящее время актуальным остается вопрос длины марковской цепи. Разработаны способы проверки сходимости цепей, однако эти способы не универсальны. В своей работе Каппе и Роберт (2000) обозначили основные проблемы использования критериев сходимости, в том числе правила остановки. Важную роль здесь играет сложность условного распределения. Для скалярных параметров сходимость удобно анализировать графически, хотя есть  и формальные тесты (Гевеке, 1992). Вместе с тем,  Гельман и Рубин (1992) предлагают  использовать несколько цепей, где начальное значение для каждой цепи своё, что позволит проверить сходимость к одному и тому же апостериорному распределению. Зеллнер и Мин (1995) предложили несколько критериев сходимости, которые применимы, если апостериорное распределение может быть представлено явно.

\section{Библиографические заметки}

В настоящее время существует много книг, посвященных применению современных методов рассчета к байесовскому анализу, в том числе работы Гамермана (1997), а также Гельмана и др. (1995). Достаточно доступными являются книги Джилла (2002), Купа (2003) и Ланкастера (2004). Куп рассматривал применение байесовских методов ко многим стандартным нелинейным моделям и панельным данным. Также до сих пор актуальны результаты более ранних работ Зеллнера (1971) и Лимера (1978).

13.2 Работа Байеса (1764) доступно изложена в статье Стиглера (1961). Вначале Байес излагает некоторые свойства вероятности, например, $\Pr[A|B]=\Pr[A\cap B]/\Pr[B]$. Полученный результат Байес использует для определения апостериорной вероятности $\Pr[a<\theta<b |y]$, где $a$ и  $b$ --- границы, $y$ --- количество успешных испытаний по схеме Бернулли и $\theta$ искомая вероятность каждого успешного испытания. Байес использовал равномерное распределение, следовательно для апостериорной вероятности справедливо $f(\theta|y){\propto}f(y|\theta)$. Пример, рассмотренный Байесом, довольно трудный, он не смог аккуратно посчитать апостериорную вероятность, для этого требовалось неполное гамма-распределение, которое было изучено только в 20-м веке. Более популярным оказался подход Лапласа и других, получивший название метода обратной вероятности,  для которого также справедливо $f(\theta|y){\propto}f(y|\theta)$. Метод Байеса и метод обратной вероятности были заменены  методом максимального правдоподобия, разработанным Фишером (1992), чья работа прямо критиковала байесовский подход и метод обратной вероятности. 

Условия регулярности для сходимости к нормальному распределению апостериорной  оценки  изложены в работе Хейда и Джонстона (1979). Теорема Бернштейна-фон Мизеса  менее формально  но понятно изложена  в статье Трейна (2003).

13.3

Байесовский анализ линейных регрессий подробно рассмотрен у Зеллнера (1971) и Лимера (1978).

13.4

Интегрирование методом Монте-Карло хорошо представлено у Гевеке (1989) и Гевеке и Кина (2001).

13.5

Вводное описание алгоритма Гиббса приводят Казелла и Джордж (1992). Многочисленные статьи Чиба и совторов, а также Гевеке и соавторов, охватывают многие темы из микроэконометрики. Чиб и Гринберг (1996, раздел 3) приводят приложения MCMC, в частности к внешне не связанным регрессионным моделям, тобит и пробит моделям. Вместе с тем, авторы показали, что количество расчетов можно сократить, если соединить алгоритм Гиббса и метод пополнения данных. 
Пополнение данных необходимо для того, чтобы описать латентные переменные, которые используются для решения проблемы скрытых переменных в цензурированных моделях и моделях дискретного выбора. Чиб (2001), учитывая все новые достижения, провел детальный анализ применения MCMC для актуальных линейных и нелинейных моделей. Гевеке и Кин (2000) сосредоточили основное внимание на методах интегрирования как байесовских, так и небайесовских.


\subsubsection*{Упражнения}

21-1 Покажите, что если $\beta|\lambda{\sim}N[\mu,\lambda^{-1}\Sigma]$ и $\lambda{\sim}Gamma[\alpha/2,\alpha/2]$, то безусловное распределение $\beta$ является многомерным $t$-распределением с параметрами $(\mu,\Sigma,\alpha)$.

21-2 (Адаптировано из работы Чиба, 1992). Рассмотрим цензурированную регрессию или тобит-модель (см. Раздел 16.3), где $y^{*}=x'\beta+\varepsilon, \varepsilon$ независимо и одинаково распределены ${N[0,\sigma^{2}]}$, и значения $y$ наблюдаемо, если $y^{*}>0$ и не определено, если $y^{*}{\leq}0$. Допустим, что количество цензурированных наблюдений $y$ равно $N_0$ и $y_0$ их обозначает. Введем латентную переменную $z$, которая соответствует цензурированным наблюдениям, т.е. $z_{i}<0$ для $i$-го цензурированного наблюдения. Для генерирования латентных переменных $z_i$ может использоваться метод пополнения данных. Латентные переменные здесь представляют собой множество независимых случайных величин и имеют усеченное нормальное распределение на интервале $(-\infty,0)$ и плотность распределения  равна $\phi(z_i|y_i,\beta,\sigma^2)/(1-\Phi(x'_{j}\beta/\sigma))$, $-\infty<z_i<0$, где $\phi$ и $\Phi$ являются функцией плотности и функцией распределения  нормальной величины. Допустим, что $\beta$ имеет нормальное априорное распределение, а $\sigma^{-2}$ гамма априорное распределение.

(a) Покажите, что возможно явно получить условные распределения для $z_i,\beta$ и $\sigma^{-2}$.

(b) Используя результаты части (a) кратко опишите алгоритм Гиббса для симуляции значений $z_i, \beta$ и $\sigma^{-2}$.

(c) Покажите как могут быть получены разумные стартовые значения $\beta$ и $\sigma^{-2}$. 




\part{Модели пространственных данных}

Часть 4, состоящая из глав с 14 по 20, охватывает базовые нелинейные модели ограниченных зависимых переменных для пространственных данных,  определяемые диапазоном значений,  принимаемых зависимой переменной. Затронутые темы включают модели для бинарных, мультиномиальных, счетных данных и данных о длительности состояний. Также рассматриваются сложности, связанные с цензурированием, усечением и самоотбором выборки (sample selection). Основную методологическую базу для моделей в Части 4 составляют метод наименьших квадратов и метод максимального правдоподобия.

Главы 14 и 15 посвящены моделям для бинарных и мультиномиальных данных,  которые обычно используются при анализе ситуаций дискретных исходов и дискретного выбора. Метод максимального правдоподобия здесь является ведущим. Различные способы параметризации условной вероятности в моделях такого типа приводят к применению различных моделей. Особенно прочно установилось использование логит и пробит-моделей. Современная литература сфокусирована на менее жестких методах моделирования с более гибкими функциональными формами для условной вероятности и учитывающих индивидуальную ненаблюдаемую неоднородность. Эти задачи побуждают к применению полупараметрических  и симуляционных  методов оценивания.

Цензурирование,  усечение и самоотбор при построении выборки приводят к  нескольким важным классам моделей. Эти модели анализируются в Главе 16. Давно существующая тобит-модель занимает здесь центральное  место,  но её состоятельное оценивание и статистические выводы опираются на сильные  предположения о распределении переменных. Также мы исследуем более новые полупараметрические методы,  которые основываются на более слабых допущениях.

Главы  17 --- 19 посвящены моделям длительности, которые фокусируются на факторах определяющих  продолжительность того или иного состояния,  таких как длительность периода безработицы,  или на моделировании функции риска при переходе из некоторого начального состояния в другое. Анализ охватывает как модели с дискретным,  так и с непрерывным временем,  а также использование как параметрических,  так и полупараметрических методов,  включая стандартные модели как,  например,  экспоненциальное распределение,  распределение Вейбулла или модель пропорциональных рисков. Глава 18 содержит формулировку и интерпретацию широкого круга моделей,  включающего исследования ненаблюдаемой неоднородности. Относительная важность влияния состояния и ненаблюдаемой неоднородности на среднюю длительность является главным вопросом,  решение которого ставит вопрос об альтернативных подходах к моделированию. В главе 19 описаны модели с несколькими типами событий с использованием подхода конкурирующих рисков и модели множественных состояний.

В главе 20 приведен анализ счетных событий,  типа данных,  который обычно используется для экономики здравоохранения. Существует много общего между моделями счетных данных и моделями продолжительности,  поскольку их общим основанием является анализ случайных процессов. Анализируются широкоиспользуемые распределение Пуассона и отрицательное биномиальное распределение. Также рассмотрены  такие важные классы моделей как \hl{модель с двойным барьером,  модель с раздутым нулем,  модель латентных признаков,  модели с эндогенным регрессором},  каждая из которых описывает одну из черт случайного процесса.



\chapter{Модели бинарного выбора}

\section{Введение}

\textbf{Модель дискретного исхода или качественной реакции} --- это модель зависимой переменной, которая показывает, в какой из $m$ взаимно исключающих классов значений попадет интересующий нас исход. Часто отсутствует естественная упорядоченность классов значений. Например,  классификация может быть проведена по роду деятельности наемных работников.

В этой главе рассматривается простейший случай \textbf{бинарного выбора}, для которого возможны только два исхода. Например, является ли индивид занятым или безработным или приобретет ли покупатель товар или нет. Ситуации бинарного выбора просты для моделирования, и модели обычно оцениваются методом максимального правдоподобия,  поскольку распределение значений обязательно является распределением Бернулли. Если вероятность одного исхода равна $p$, то вероятность другого исхода будет равна $\left(1-p\right)$. В регрессионных моделях вероятность $p$ будем изменяться между индивидами в зависимости от значений регрессоров. Две стандартные модели бинарного выбора, логит и пробит-модели,  устанавливают разные функциональные формы для этой вероятности как функции аргумента. Качественная разница между этими двумя подходами заключается примерно в том же,  в чём и при использовании различных функциональных форм условного среднего в линейной регрессии.

В Разделе 14.2 приведен пример бинарных данных. Раздел 14.3 представляет обобщение статистических результатов для стандартных моделей,  включая логит и пробит-модели. В Разделе 14.4 модели бинарного выбора представлены как возникающие из лежащих в их основе скрытых переменных. Эта формулировка полезна,  потому что позволяет легко перейти к моделям множественного выбора (см. Главу 15) и моделям с цензурированными значениями и самоотбором выборки  (см. Главу 16). В Разделе 14.5 подробно описывается необходимые модификации стандартных методов оценки,  когда в выборку включено слишком большое число исходов из некоторой группы исходов. Проблемы агрегации рассматриваются в разделе 14.6. Полупараметрические методы для моделей бинарного выбора, которые снимают ряд ограничений на параметры модели вероятности $p$,  представлены в Разделе 14.7.

\section{Пример бинарной зависимой переменной: выбор способа рыбалки}

В этом разделе моделируется выбор между рыбалкой с судна и рыбалкой с пристани. Зависимая переменная задается следующим образом

\[
y_i=
\begin{cases}
1, \text{если выбрана рыбалка с судна,} \\ 
0, \text{если выбрана рыбалка с пристани,}
\end{cases}
\] 
где значения $1$ и $0$ выбраны для простоты. Единственной объясняющей переменной является $x_i=\ln  relp_i= \ln  \left(relp_i\right) $ где $relp$ обозначает отношение цены рыбалки с судна ($price_{charter}$) к цене рыбалки с пристани $(price_{pier})$. Таким образом, 

\[
x_i=\ln  relp_i=\ln  (price_{charter, i}/price_{pier, i}).
\] 
Цены на рыбалку с судна и на рыбалку с пристани варьируются среди индивидов под влиянием разнообразных факторов,  например,  разницы в доступности. Ожидается,  что вероятность выбора рыбалки с судна снижается,  если ее относительная цена растет.

Рассмотрим данные,  приведенные в таблице 14.1. Выборка из 630 индивидов --- это подмножество набора данных,  детально описанного в разделе 15.2,  в котором рассматриваются 4 различных способа рыбалки,  а также принимаются во внимание дополнительные регрессоры. Рыбалку с судна предпочли 71.7\% участников выборки. Для людей,  выбравших рыбалку с судна,  она была в среднем дешевле,  чем рыбалка с пристани,  а именно 75\$ против 121\$. Для людей,  выбравших рыбалку с пристани,  верной оказывалась обратная ситуация. Таким образом,  как оказалось,  цена производит ожидаемый эффект.

\begin{center}
\begin{table}[h]
\caption{\label{tab:fishres} Выбор способа рыбалки: результаты}
\begin{tabular}{lccc} 
\hline 
\hline
 & \multicolumn{2}{c}{\textbf{Среднее по подмножествам выборки}} & \\ 
\hline 
 & ${\mathbf y}{\mathbf =}{\mathbf 1}$ & ${\mathbf y}{\mathbf =}{\mathbf 0}$ & \textbf{Итог по всем}\\ 
\textbf{Переменная} & \textbf{Рыбалка с лодки} & \textbf{Рыбалка с пристани} & ${\mathbf y}$ \\
\hline 
Цена рыбалки с судна & 75 & 110 & 85 \\  
$(price_{charter}(\$))$ & & & \\
Цена рыбалки с пристани & 121 & 31 & 95 \\
$(price_{pier}(\$))$ & & & \\
$\ln relp$  & -0.264 & 1.643 & 0.275 \\
Доля выбравших & 0.717 & 0.283 & 1.000 \\ 
Количество наблюдений & 452 & 178 & 630 \\ 
\hline
\hline 
\end{tabular}
\end{table}
\end{center}


\textbf{Линейная регрессия} зависимой переменной $y_i$ от независимой переменной $x_i$ игнорирует дискретность зависимой переменной и не ограничивает прогнозируемую вероятность интервалом от 0 до 1.

\textbf{логит-модель} является более подходящей (см. параграф 14.3.4) и определяет вероятность по формуле

\[
p_i=\Pr\left[y_i=1\left|x_i\right.\right]=\frac{\exp\left(\beta_1+\beta_2x_i\right)}{1+\exp\left(\beta_1+\beta_2x_i\right)}.
\] 
Эта модель несомненно обеспечивает $0<p_i<1$. Метод максимального правдоподобия (см. параграф 14.3.3) дает оценки параметров,  приведенные в первом столбце таблицы 14.2. Предполагаемый предельный эффект в логит-модели определяется по формуле

\[\frac{dp_i}{dx_i}=\frac{\exp\left(\beta_1+\beta_2x_i\right)}{{\left(1+\exp\left(\beta_1+\beta_2x_i\right)\right)}^2}\beta_2.\] 

\begin{table}[h]
\caption{\label{tab:fishlogit} Выбор способа рыбалки: расчеты для логит и пробит-моделей}
\begin{minipage}{17cm}
\begin{tabular}{lccc} 
\hline 
\hline
\textbf{Регрессор}\footnote{Зависимая переменная $y=1$,  если выбрана рыбалка с судна,  $y=0$,  если выбрана рыбалка с пристани. Регрессор $x=\ln relp$ --- это натуральный логарифм отношения цены рыбалки с лодки к цене рыбалки с пристани. Значимость коэффициентов моделей,  полученных методом максимального правдоподобия для логит и пробит-моделей и с помощью МНК для линейной регрессии,  оцениваются $t$-статистикой в скобках.} & \textbf{логит-модель} & \textbf{пробит-модель} & \textbf{Линейная регрессия} \\ 
\hline 
Константа & 2.053 & 1.194 & 0.784 \\ 
 & (12.15) & (13.34) & (65.58) \\
Коэффициент у $\ln relp$ & $-1.823$ & $-1.056$ & $-0.243$ \\  
 & $(-12.61)$ & $(-13.87)$ & $(-28.15)$ \\
$-\ln L$ & $-206.83$ & $-204.41$ & --- \\ 
Псевдо R-квадрат & 0.449 & 0.455 & 0.463 \\ 
\hline
\hline 
\end{tabular}
\end{minipage}
\end{table}
Поскольку $\widehat{\beta}_{2,LOGIT} < 0$ следовательно,  как и ожидалось,  $dp_i/dx_i<0$. Фактическая величина предельного эффекта изменяется вместе со значением $x_i$ (см. Параграф 14.3.2). Приближением предельного эффекта для логит-модели,  в отличие от других моделей,  является $dp_i/dx_i \simeq  \overline{y}\left(1-\overline{y}\right) \widehat{\beta}_2=-0.370$. Линейная регрессия напротив дает прямую оценку,  равную -0,243.

В качестве альтернативной используется \textbf{пробит-модель} (см. подраздел 14.3.5),  которая определяет

\[p_i=\Pr\left[y_i=1\left|x_i\right.\right]=\Phi \left(\beta_1+\beta_2x_i\right), \] 
где $\Phi (\cdot )$ --- это функция стандартного нормального распределения. 

Таким образом,  $p_i=\int^{\beta_1+\beta_2x_i}_{-\infty} \left(2\pi \right)^{-1/2} e^{-z^2/2}dz$. Коэффициенты,  полученные методом максимального правдоподобия,  приведены во втором столбце таблицы 14.2 и заметно отличаются от коэффициентов для логит-модели. Поскольку оценки были получены с использованием различных спецификаций моделей, эти коэффициенты не сопоставимы. Эта ситуация аналогична невозможности сравнивать коэффициенты моделей с условным средним $x'\beta $ и $\exp \left(x'\beta \right)$. Для пробит-модели справедливо $dp_i/dx_i =\phi\left(\beta_1+\beta_2x_i\right)\beta_2$,  где $\phi (\cdot )$ --- функция плотности стандартного нормального распределения. Тогда,  снова получаем,  что $dp_i /dx_i<0$,  пока $\widehat{\beta}_{2, PROBIT} < 0.$

Несмотря на то,  что коэффициент при независимой переменной неизбежно различается среди моделей,  из таблицы 14.2 видно,  что t-статистика везде примерно одинакова и достигает очень большой величины. Значение логарифмической функции правдоподобия для пробит-модели,  2,42,  выше этого же значения для логит-модели. Это свидетельствует в пользу пробит-модели,  поскольку обе модели используют одинаковое количество параметров. Для множества других примеров разница между значениями $\ln  L$ среди моделей невелика. Прогнозируемые с помощью трех моделей значения вероятности представлены на графике как функции от $x$ на рисунке 14.1. Для линейной регрессии предполагается,  что функция $\Pr  \left[y_i=1\left|x_i\right.\right]\ =\beta_1+\beta_2 x_i$ линейна по $x_i$,  тогда как нелинейные функции для логит и пробит-моделей в принципе эквивалентны.

\section{Логит и пробит-модели}

Далее будет приведена более формальная теория для этих моделей. Проблема бинарного выбора будет показана как прямое продолжение примера с подбрасыванием монеты из вводных статистических курсов, только теперь вероятность успеха  зависит от регрессора. Два обычно используемых способа параметризации приводят к логит и пробит-моделям. Обоснование таких способов параметризации с применением скрытых переменных будет приведено в разделе 14.4.

Оценка вероятности по рассмотренным моделям

\textbf{Красивый график со страницы 466}

Рисунок 14.1. Рыбалка с лодки: предсказанная вероятность с использованием логит,  пробит-моделей и линейной регрессии,  когда в качестве единственного регрессора используется натуральный логарифм отношения цен. Действительные значения исходов,  1 и 0,  также отражены на графике для наглядности. Приведены данные для 620 случаев.

\subsection{Общая модель бинарного выбора}

Для случаев бинарного выбора зависимая переменная $y$ принимает одно из двух возможных значений. Предположим,  что

\[ 
y= \begin{cases}
1 \text{с вероятностью} \hspace{0.3cm} p,  \\ 
0 \text{с вероятностью} \hspace{0.3cm} 1-p. 
\end{cases}
\] 
Общность модели никак не ограничивается выбором значений для исходов,  равных 1 и 0,  поскольку моделируется только переменная $p$,  обозначающая вероятность наступления конкретного исхода. Во вводном курсе статистики эта модель описывает результат подбрасывания монеты. Выпадение орла означает,  что $y=1$,  и происходит с вероятностью $p$.

Регрессионная модель строится путем параметризации вероятности $p$ в зависимости от вектора независимой переменной $x$ и вектора-столбца параметров $\beta$ размера $K \times 1$. Обычно используется модель с одноиндексной формой,  для которой \textbf{условная вероятность} полагается

\begin{equation} 
\label{GrindEQ__14_1_} 
p_i\equiv \Pr  \left[y_i=1\left|x_i\right.\right] =F\left(x'_i\beta \right),  
\end{equation} 
где $F(\cdot )$ --- выбранная функциональная форма модели. Чтобы обеспечить выполнение условия $0\le p\le 1$ естественно определить $F(\cdot )$ как некоторую функцию  распределения.

В таблице 14.3 представлены наиболее часто используемые модели бинарного выбора. \textbf{логит-модель} возникает,  если $F(\cdot )$ --- функция логистического распределения. \textbf{пробит-модель} возникает,  если $F\left(\cdot \right)$ ---функция стандартного нормального распределения. Обратите внимание,  что,  поскольку $F\left(\cdot \right)$ является функцией распределения,  то она используется только для моделирования значений параметра $p$,  и никак не указывает на функцию распределения переменной $y$. Реже используемая \textbf{двойная логарифмическая модель} возникает,  если для $F\left(\cdot \right)$ используется функция распределения экстремальных значений. Эта функция отличается от остальных,  поскольку является асимметричной относительно нуля,  и используется,  если один из исходов является более редким. В \textbf{модели линейной регрессии} не используется функция распределения,  вместо этого полагает $p_i=x'_i\beta $.
 
\begin{table}[h]
\begin{center}
\caption{\label{tab:binarychoice} Проблема бинарного выбора: наиболее часто используемые модели}
\begin{tabular}{lcc} 
\hline 
\hline
\textbf{Модель} & \textbf{Вероятность }$(p=\Pr[y=1\left|x\right.])$ & \textbf{Предельный эффект}$(\partial p/\partial x_j)$ \\ 
\hline
логит-модель & $\Lambda \left(x'\beta \right)=\frac{e^{x'\beta}}{1+e^{x'\beta}}$\newline  & $\Lambda \left(x'\beta \right)\left[1-\Lambda \left(x'\beta \right)\right]\beta_j$ \\
пробит-модель & $\Phi \left(x'\beta \right)=\int^{x'\beta }_{-\infty }{\phi (z)dz}$ & $\phi (x'\beta )\beta_j$ \\
Двойная логарифмическая & $C\left(x'\beta \right)=1- \exp(-\exp(x'\beta))$ & $ \exp(- \exp  (x'\beta))\exp  (x'\beta)\beta_j$ \\ 
модель & & \\
Линейная регрессия & $x'\beta $ & $\beta_j$ \\ 
\hline 
\hline
\end{tabular}
\end{center}
\end{table}

 

\subsection{Предельные эффекты}

Зачастую интерес представляет \textbf{предельный эффект} от изменения независимой переменной $x$ на условную вероятность наступления $y=1$. Для общей модели вероятности \eqref{GrindEQ__14_1_} и изменения значения $j$-го регрессора,  предельный эффект рассчитывается по формуле

\begin{equation} 
\label{GrindEQ__14_2_} 
\frac{\partial \Pr[y_i=1|x_i]}{\partial x_{ij}}=F'\left(x'_i\beta \right)\beta_j,  
\end{equation} 
где $F'\left(z\right)=\partial F(z)/\partial z$. Предельные эффекты изменяются с  изменением $x_i$, как и для любой нелинейной модели, а также зависят от выбора формы $F(\cdot )$. В последнем столбце таблицы 14.3 приведены формулы предельных эффектов для моделей,  используемых для ситуаций бинарного выбора.

Предельные эффекты для нелинейных моделей обсуждаются в параграфе 5.2.4. Для каждой конкретной модели существует несколько методов расчета среднего предельного эффекта. Лучше всего использовать формулу $N^{-1}\sum_i F'(x'_i\widehat\beta) \widehat\beta_j$,  то есть среднее предельных эффектов по выборке. Некоторые программы вместо этого  рассчитывают предельный эффект для среднего арифметического каждого регрессора, $F'( \overline{x}' \widehat\beta) \widehat\beta_j$. Еще один способ расчета предельного эффекта --- оценивание в точке  $\overline{y}$, выборочного среднего $y$, т.е. $F(x'\beta)=\overline{y}$ и $F'(x'\beta)=F'(F^{-1}(\overline{y}))$. Такой подход особенно удобен для логит-модели,  поскольку тогда предельный эффект будет оценен как $\overline{y}(1-\overline{y}){\widehat\beta}_j$. Дальнейшие рассуждения для конкретных моделей приведены в параграфах 14.3.4 --- 14.3.7.

Многие исследования вместо предельных эффектов  приводят только коэффициенты регрессии. Стандартные модели бинарного выбора --- это одноиндексные модели,  следовательно,  отношение коэффициентов двух разных регрессоров равно отношению их предельных эффектов.Знак коэффициента  модели совпадает со знаком предельного эффекта, т.к. $F'(\cdot)>0$. Зная коэффициенты модели можно оценить сверху  значение предельных эффектов. Для логит-модели --- это $\partial p/ \partial x_j \le 0.25\widehat\beta_j$, т.к. $\Lambda (x'\beta) (1-\Lambda (x'\beta)) \le 0.25$, с максимальным значением при $\Lambda (x'\beta)=0.5$ и $x'\beta=0$. Для пробит-модели --- $\partial p/\partial x_j \le 0.4 \widehat\beta_j$, т.к. $\phi(x'\beta)\le 1/\sqrt{2\pi} \simeq 0.4$,  с максимальным значением при $\Phi (x'\beta)=0.5$ и $x'\beta =0$

\subsection{Метод максимального правдоподобия}

Оценивание производится по выборке $(y_i,  x_i)$, $i=1, \ldots,  N$,  в которой предполагается независимость по $i$. Здесь приведены результаты для вероятности $p_i$,  определяемой по формуле \eqref{GrindEQ__14_1_},  с уточнением сначала для логит, а затем --- для пробит-модели.

\subsubsection*{Метод максимального правдоподобия для общей модели бинарного выбора}

Исход имеет распределение Бернулли или биномиальное распределение с одним испытанием. Очень удобно использовать следующую  компактную форму  запись для \textbf{функции вероятностей}  $y_i$:

\begin{equation} 
\label{GrindEQ__14_3_}
 f(y_i|x_i)=p^{y_i}_i(1-p_i)^{1-y_i}, y_i=0, 1,  
\end{equation} 
где $p_i=F(x'_i\beta )$. Данная формула даёт вероятности $p_i$ и $(1-p_i)$,  т.к.  $f(1)=p^1(1-p)^0=p$ и $f(0)=p^0(1-p)^1=1-p$.

Из формулы \eqref{GrindEQ__14_3_} следует формула для логарифма вероятности $\ln f(y_i)=y_i \ln p_i +(1-y_i) \ln (1-p_i)$. С учетом независимости по $i$ и модели \eqref{GrindEQ__14_1_} для $p_i$ логарифмическая функция правдоподобия принимает вид

\begin{equation} 
\label{GrindEQ__14_4_}
 \mathcal{L}_N (\beta)=\sum^N_{i=1} \{ y_i \ln F(x'_i\beta)+ (1-y_i) \ln (1-F(x'_i\beta)) \}.
 \end{equation} 

Продифференцировав по $\beta$,  получаем,  что \textbf{оценка} параметра $\widehat\beta_{ML}$,  полученная \textbf{методом максимального правдоподобия}, является решением уравнения

\[
\sum^N_{i=1} \{ \frac{y_i}{F_i} F'_i x_i - \frac{1-y_i}{1-F_i} F'_ix_i \} =0, 
\] 
где $F_i=F(x'_i\beta)$, $F'_i=F'(x'_i\beta)$,  а $F'(z)=\partial F(z)/\partial z$. Приведя выражение в скобках к общему знаменателю $F_i(1-F_i)$ и упростив результат, получим условие первого порядка максимизации функции правдоподобия

\begin{equation} 
\label{GrindEQ__14_5_} 
\sum^N_{i=1} \frac{y_i-F(x'_i\beta)}{F(x'_i\beta)(1-F(x'_i\beta))}F'(x'_i\beta)x_i=0. 
\end{equation} 
Явного решения этого уравнения в общем случае не существует.  Однако метод Ньютона-Рафсона очень быстро сходится к $\widehat\beta_{MLE}$, по крайней мере для логит и пробит-моделей,  поскольку для них функция правдоподобия вогнута на все области определения.

\subsubsection*{Состоятельность метода максимального правдоподобия}

Оценка, полученная методом максимального правдоподобия, \textbf{состоятельна},  если функция условного распределения $y$ при данном $x$ задана верно. Поскольку в данном случае используется распределение Бернулли,  то единственная возможная ошибка может быть в неверном задании формулы для вероятности. Таким образом,  метод максимального правдоподобия состоятелен,  если $p_i \equiv F(x'_i\beta)$,  и не состоятелен в противном случае.

Более формально,  заметим,  что для бинарных данных $\E[y]=1\times p+0\times (1-p)=p$. Поэтому из уравнения \eqref{GrindEQ__14_1_} следует,  что 

\begin{equation} 
\label{GrindEQ__14_6_} 
\E[y_i|x_i]=F(x'_i\beta). 
\end{equation} 

Это в свою очередь означает,  что ожидаемое значение левой части уравнения \eqref{GrindEQ__14_5_}  равно нулю, что есть основное  условие состоятельности. Данное особое условие состоятельности выполнено для распределений экспоненциального семейства, при условии правильно специфицированного среднего, см. параграф 5.7.3.  Распределение Бернулли принадлежит к экспоненциальному семейству.

\subsubsection*{Распределение ММП оценки}

В случае правильной спецификации  распределения,  $\widehat\beta_{ML} \overset{a}{\sim} \mathcal{N}[\beta,  \left(-\E[\partial^2 \mathcal{L}_N /\partial \beta \partial \beta' ]\right)^{-1}]$ (см. параграф 5.6.4). Продифференцировав выражение \eqref{GrindEQ__14_4_} по $\beta'$ и взяв с минусом ожидаемое значение,  получаем ожидаемую \textbf{асимптотическую ковариационная матрицу}

\begin{equation} 
\label{GrindEQ__14_7_} 
\hat{\V} \left[ \widehat\beta_{ML} \right] = \left( \sum^N_{i=1} \frac{1}{F(x'_i \widehat\beta) (1-F(x'_i \widehat\beta))} F'(x'_i\widehat\beta)^2 x_i x'_i\right)^{-1},  
\end{equation} 
выражение упрощается, поскольку $\E[y_i-F(x'_i\beta)]=0$. Ковариационная матрица имеет простую форму $(\sum_i \hat{w}_i x_i x'_i)^{-1}$,  где веса $\hat{w}_i$ приведены в выражении \eqref{GrindEQ__14_7_}.

Поскольку для состоятельности метода достаточно только правильной спецификации условного среднего или функции вероятности,  будет естественным рассмотреть метод квази-максимального правдоподобия (см. раздел 5.7) и строить выводы на сэндвич-форме ковариационной матрицы $A^{-1}BA^{-1}$, а не на матрице $-A^{-1}$,  используемой в выражении \eqref{GrindEQ__14_7_}. Здесь

\begin{equation} 
\label{GrindEQ__14_8_} 
\V [y_i| x_i]= F(x'_i\beta)(1-F(x'_i\beta)),  
\end{equation} 
т.к. $\V[y]=(1-p)^2 \times p + (0-p)^2 \times (1-p) = p(1-p)$. После некоторых алгебраических операций получаем,  что $A=-B$, и,  следовательно,  $A^{-1}BA^{-1}=-A^{-1}$,  при независимости по $i$. Единственный случай,  когда равенство \eqref{GrindEQ__14_8_} не соблюдается,  это $p\ne F(x'\beta)$. В этом случае метод максимального правдоподобия будет страдать от более фундаментальной проблемы несостоятельности.

Модели бинарного выбора необычны тем,  что нет никаких преимуществ от использования сэндвич формы матрицы,  если данные независимы по $i$. Единственная причина, из-за которой нужно использовать робастную  форму ковариационной матрицы, --- это наличие корреляции между наблюдениями в результате кластеризации. Тогда робастная оценка должна  быть устойчива к кластеризации (см. раздел 24.5), а не к неправильной спецификацией условной дисперсии.

\subsection{логит-модель}

\textbf{логит-модель} или \textbf{логистическая регрессионная модель} задает

\begin{equation} 
\label{GrindEQ__14_9_} 
p=\Lambda \left(x'\beta \right)=\frac{e^{x'\beta }}{1+e^{x'\beta }},  
\end{equation} 
где $\Lambda (\cdot)$ --- логистическая функция распределения (см. раздел 14.4.1 для более подробной информации) с $\Lambda \left(z\right)={e^z}/{\left(1+e^z\right)={1}/{(1+e^{-z})}}.$

Условие первого порядка максимизации функции правдоподобия для логит-модели упрощается до

\begin{equation} 
\label{GrindEQ__14_10_} 
\sum^N_{i=1} (y_i-\Lambda (x'_i\beta))x_i=0,  
\end{equation} 
поскольку $\Lambda'(z)=\Lambda (z)[1-\Lambda (z)]$.
Тогда простые остатки $y_i-\Lambda (x'_i\beta)$ ортогональны  вектору независимых переменных,  как и в модели линейной регрессии. Такое простое условие возникает поскольку $\Lambda (\cdot)$ --- каноническая функция связи для распределения Бернулли (см. параграф 5.7.4).

Если среди независимых переменных $x_i$ присутствует константа,  то \eqref{GrindEQ__14_10_} приводит к тому, что $\sum_i(y_i- \Lambda (x'_i\widehat\beta))=0$. Таким образом,  сумма остатков в логит-модели равна нулю. Это подразумевает,  что средняя предсказанная вероятность внутри выборки $N^{-1}\sum_i \Lambda (x'_i\widehat\beta)$ обязательно равняется выборочной доле $\overline{y}$. 

\textbf{Предельный эффект} для логит-модели может быть достаточно легко получен, когда известны коэффициенты,  поскольку $\partial p_i/\partial x_{ij}=p_i\left(1-p_i\right)\beta_j$, где $p_i=\Lambda_i=\Lambda (x'_i\beta)$. 
Подставив $p_i=\overline{y}$,  получаем грубую оценку предельного эффекта --- $(1-\overline{y})\widehat\beta_j$. 
Для $0.3<p_i<0.7$, например,  значение $\partial p_i/\partial x_{ij}$ лежит между $0.21\beta_j$ и $0.25\beta_j$. В случае данных,  для которых $p_i\simeq 0,0$, то есть когда большинство исходов равно нулю,  $\partial p_i/\partial x_{ij} = p_i \beta_j$. Тогда $\beta_j$ пропорционально влияет на вероятность того,  что $y_i=1$ при изменяющемся $x_{ij}$.

В литературе по статистике более распространена интерпретация коэффициентов в терминах предельных эффектов на отношение шансов (odds ratio), а не на вероятность. Для логит-модели 

\[
p=
\exp(x'\beta )/(1+\exp(x'\beta ))
\] 

\begin{equation} 
\label{GrindEQ__14_11_} 
\Longrightarrow \frac{p}{1-p}=\exp(x'\beta ) 
\end{equation} 

\[
\Longrightarrow \ln  \frac{p}{1-p}=x'\beta.
\] 
Здесь $1/\left(1-p\right)$ определяет отношение вероятности того,  что $y=1$ к вероятности того,  что $y=0$ и называется \textbf{отношением шансов} или  \textbf{относительным риском}. Например,  рассмотрим испытание фармацевтического средства,  для которого $y=1$ означает количество выживших,  а $y=0$ означает количество смертельных случаев. Независимые переменные включают дозировку препарата. Значение отношения шансов,  равное 2,  означает,  что вероятность  выживания превосходит вероятность смертельного случая в два раза. Для логит-модели \textbf{логарифм отношения шансов} линеен по независимой переменной.

В статистическом анализе и статистических пакетах используется второе равенство из \eqref{GrindEQ__14_11_}. Предположим,  $j$-ый регрессор увеличивается на одну единицу. Тогда $\exp(x'\beta )$ увеличивается до $\exp  (x'\beta +\beta_j)=\exp(x'\beta) \times \exp(\beta_j)$. Из \eqref{GrindEQ__14_11_} следует,  что отношение шансов возрастает в $\exp (\beta_j)$ раз. Таким образом,  коэффициент при независимой переменной в логит-модели равный,  например,  0,1 означает,  что увеличение независимой переменной на одну единицу увеличит величину отношения шансов в $\exp(0.1) \simeq 1.105$ раз. По-другому можно сказать, что отношение шансов увеличивается на долю 0.105, или что \textbf{относительная} вероятность выживания увеличивается на 10.5\%. Такая интерпретация логит-модели широко используется в приложениях к биостатистике.

Для экономистов более естественно интерпретировать второе или третье равенство из \eqref{GrindEQ__14_11_},  подразумевая,  что $\beta_j$ --- это \textbf{полуэластичность}. Тогда,  применив производную,  будем интерпретировать коэффициент при независимой переменной в логит-модели  равный 0,1,  таким образом:  увеличение регрессора на одну единицу увеличивает отношение шансов на долю 0,1. Это в точности совпадает с интерпретацией,  используемой в статистике для очень малых $\beta_j$, т.к. $\exp (\beta_j)-1 \simeq \beta_j$.

\subsection{пробит-модель}

В \textbf{пробит-модели} условная вероятность задается как 

\begin{equation} 
\label{GrindEQ__14_12_} 
p=\Phi (x'\beta) =\int^{x'\beta}_{-\infty} \phi(z)dz,
\end{equation} 
где $\Phi (\cdot)$ --- функция распределения стандартной нормальной величины с производной $\phi (z)=(1/\sqrt{2\pi})\exp (-z^2/2)$, которая является функцией плотности стандартного нормального распределения.

Условие первого порядка максимизации функции правдоподобия для пробит-модели следующее 

\[
\sum^N_{i=1} w_i (y_i-x'_i\beta)x_i=0, 
\] 
где, в отличие от логит-модели,  веса $w_i=\phi (x'_i\beta)/[\Phi (x'_i\beta)(1-\Phi (x'_i\beta))]$ варьируются в зависимости от наблюдений. Предельные эффекты в пробит-модели --- $\partial p_i/ \partial x_{ij}=\phi (x'_i\beta)\beta_j=\phi (\Phi^{-1}(p_i))\beta_j$, где $p_i=\Phi (x'_i\beta)$. 
Дальнейших упрощений как в логит-модели не последует, тем не менее, $\partial p_i/ \partial x_{ij} \le 0.40\beta_j$, поскольку $\phi(z) \le \phi (0)=1/\sqrt{2\pi}$. % здесь в английском оригинале ошибочно написано \phi(0.5)

пробит-модель не так проста как логит-модель. Несмотря на это,  она широко используется,  так как является естественной моделью,  если отправной точкой является  регрессионная модель с латентной нормально распределенной переменной (см. раздел 14.4).

\subsection{Линейная регрессионная модель}

Простой альтернативой для логит и пробит-моделей является \textbf{линейная регрессионная модель }зависимости переменной $y$ от переменной $x$. Эта модель имеет очевидный недостаток,  поскольку возможно получить значения предсказываемой вероятности $x'_i\widehat\beta$, которые будут отрицательны или превышать единицу.

Тем не менее,  МНК оценка полезна как исследовательский инструмент. На практике она обеспечивает разумную прямую оценку средних по выборке предельных эффектов изменения $x$ на вероятность события $y=1$, даже если является плохой моделью для оценки индивидуальных вероятностей. На практике эта оценка позволяет хорошо определять, какие из переменных являются статистически значимыми. Во многих случаях получается,  что $0<x'_i\widehat\beta<1$ для всех наблюдений, в этом случае использование линейной регрессии более разумно.

Если используется МНК оценка,  то оценки стандартных ошибок должны быть скорректированы на  \textbf{гетероскедастичность}. Линейная регрессия оправдана,  если вероятность $p_i=x'_i\beta$. Тогда $y_i|x_i$ имеет среднее $x'_i\beta $ и гетероскедастичную дисперсию $x'_i\beta (1-x'_i\beta)$,  которая изменяется вместе с $x_i$.

В теории возможно использование более эффективного метода максимального правдоподобия,  если $p_i=x'_i\beta$. Из \eqref{GrindEQ__14_5_} условие первого порядка максимизации функции правдоподобия --- $\sum_i x_i(y_i-x'_i\beta )/[x'_i\beta (1-x'_i\beta)]=0$. Эта оценка может быть численно неустойчивой,  поскольку большое значение придается наблюдениям со значением $x'_i\beta$,  близким к 0 или 1. Более того,  повышение эффективности по сравнению с МНК часто незначительны.

И, хотя, МНК-оценки с гетероскедастичными стандартными ошибками могут служить полезным инструментом анализа данных,  для окончательного анализа данных гораздо лучше использовать оценки,  полученные методом максимального правдоподобия,  в логит и пробит-моделях.

\subsection{Выбор бинарной модели}

Какую из моделей лучше выбрать -- логит или пробит-модель? Этот вопрос исследуется в настоящем параграфе.

\subsubsection*{Теоретические соображения}

Теоретически ответ на этот вопрос зависит от процесса, порождающего данные,  который неизвестен. В отличие от других приложений метода максимального правдоподобия здесь отсутствует проблема определения типа распределения --- единственно возможным распределением для переменной со значениями (0, 1) является распределение Бернулли. Основная сложность заключается в определении функциональной формы модели для параметров распределения. Если в процессе, порождающем данные, вероятность наступления события $p=\Lambda (x'\beta )$,  то следует использовать логит-модель,  а оценки,  основанные на других моделях,  таких как пробит-модель,  являются потенциально несостоятельными. Аналогичное качественное заключение следует сделать,  если в процессе, порождающем данные, вероятность наступления события $p=\Phi (x'\beta )$. В этом случае следует использовать пробит-модель. Очень маловероятно,  что $p=x'\beta $,  поскольку тогда вероятность  $p$ может принимать значения за пределами интервала от 0 до 1.

Теоретические последствия ошибки при определении спецификации модели,  однако,  не так велики как следующая трудность. Допустим, что регрессоры распределены таким образом,  что среднее каждого регрессора,  при фиксированной линейной комбинацией $x'\beta $,  линейно по $x'\beta $.  В этом случае  может быть показано,  что  выбор неправильной функции $F$ одинаково повлияет на все оценки параметра при независимых переменных.  Значит отношение параметров при независимых переменных будет постоянным для всех моделей, см. работу Рууда (1983). Этому условию удовлетворяет семейство сферических распределений,  включая и многомерное нормальное распределение.

логит-модель имеет относительно более простую форму условий первого порядка и асимптотического распределения. Популяризатор логит-модели Берксон  (1951)  считает это одной из причин, почему логит-модель предпочтительнее  пробит-модели. Среди  общих линейных моделей,  которые широко используются в биостатистике,  логит-модель является естественной моделью,  поскольку она соответствует использованию канонической связи для биномиального распределения. Также преимуществом логит-модели выступает интерпретация коэффициентов модели в терминах логарифма отношения шансов.

Еще одной мотивацией для логит-модели является \textbf{дискриминантный анализ}. С точки зрения дискриминантного анализа обе переменные $y$ и $x$ являются случайными. Переменная $x$ наблюдаемая,  а переменная $y$ --- ненаблюдаемая. При известном $x$,  необходимо определить принимает $y$ значение,  равное нулю или единице. Классическим примером является определение типа  гоминида ($y=0$ или $1$),  к которому принадлежит череп в зависимости от различных параметров черепа $(x)$. Если условное распределение характеристик $x$ для данного $y$ является многомерным нормальным,  то апостериорная  вероятность для $y$ при заданном $x$ будет аналогична вероятности в логит-модели. Для дополнительной информации см. работы Амэмии (1981,  стр. 1507 --- 1510) и Маддалы (1983,  стр.  17 --- 21).

С другой стороны, пробит-модель обосновывается нормальным распределением скрытой переменной. (см. раздел 14.4),  и естественным образом обобщается до тобит-модели (см. главу 16). Поэтому многие экономисты предпочитают использовать пробит-модель.

\subsubsection*{Эмпирические соображения}

С эмпирической точки зрения,  как логит,  так и пробит-модель могут быть использованы. Часто существует небольшая разница между вероятностями,  спрогнозированными с помощью логит и пробит-модели. Наибольшая разница получается для остатков,  для которых вероятности близки к 0 или 1. Разница получается намного меньше,  если наибольший интерес устремлен к средним по выборке предельным эффектам,  чем к индивидуальным показателям.

Общепринято считаемой корректной, естественной метрикой для сравнения моделей для вероятности $p_i$ является значение логарифмической функции правдоподобия,  т.к. логит и пробит-модели имеют одинаковое количество параметров. Таким образом,  для каждой модели можно посчитать 

\[
\mathcal{L}_n (\widehat\beta)=\sum_i \{y_i \ln \hat{p}_i+(1-y_i) \ln (1-\hat{p}_i) \}, 
\] 
где $\hat{p}_i=\Lambda (x'_i \widehat\beta_{Logit})$ или $\hat{p}_i=\Phi (x'_i\widehat\beta_{Probit})$. Часто значения логарифмической функции правдоподобия очень похожи для двух  моделей,  что снова говорит о незначительном преимуществе  одной модели по сравнению с другой. Для более подробного знакомства с формальными тестами для  невложенных моделей см. работу Песарана и Песарана (1995) и раздел 8.5.

Различные модели дают довольно различные оценки параметров регрессии $\widehat\beta$. Тем не менее,  это всего лишь следствие использования разных формул для вероятностей. Более осмысленным является сравнение предельных эффектов в разных моделях,  поскольку эти величины имеют одинаковый масштаб в рассматриваемых моделях. Из раздела 14.2.3 получаем что  $\partial p/\partial x_j \le 0.25 \widehat\beta_j$ для логит-модели,  $\partial p/\partial x_j\le 0.4\widehat\beta_j$ для пробит-модели,  и $\partial p/\partial x_j \le \widehat\beta_j$ для линейной регрессии. Отсюда следует эмпирическое правило 

\begin{equation} 
\label{GrindEQ__14_13_} 
\widehat\beta_{Logit} \simeq 4 \widehat\beta_{OLS},  
\end{equation} 

\[
\widehat\beta_{Probit} \simeq 2.5 \widehat\beta_{OLS}, 
\] 

\[
\widehat\beta_{Logit} \simeq 1.6 \widehat\beta_{Probit}.
\] 
В своей работе Амэмия (1981,  стр. 1488) продемонстрировал,  что эти неравенства для параметров наклона хорошо работают,  если $0.1 \le p \le 0.9$. Большие отличия моделей наблюдаются в хвостах распределения. Альтернативный метод для логит-модели,  основанный на данном далее \eqref{GrindEQ__14_18_} приводит к $\widehat\beta_{Logit} \simeq (\pi/\sqrt{3}) \widehat\beta_{Probit}$.

\subsubsection*{Эндогенные регрессоры}

Логит и пробит-модели могут быть обобщены на случай многих затруднений,  которые обычно возникают в микроэкономическом анализе. В частности,  эндогенные регрессоры могут быть учтены с помощью  методов,  аналогичных представленным для цензурированных данных в параграфе 16.8.2,  и для панельных данных в главе 23.

Из-за таких сложностей проще работать с линейными вероятностными моделями, так как стандартную линейную модель  можно применять при условии, что стандартные ошибки имеют поправку на гетероскедастичность. Даже если логит и пробит-модели используются в конечном счете, линейная модель может быть использована для того, чтобы провести интерпретационный анализ.

\subsection{Определение адекватности модели}

Методы диагностики и выбора нелинейных моделей были представлены в разделе 8.7. Здесь мы рассмотрим применение этих методов к моделям бинарного выбора. Не существует единственного универсального критерия  качества модели,  поэтому статистические пакеты предлагают несколько критериев,  подробно описанных в работах Амэмии  (1981)  и Маддалы (1983).

\subsection*{Псевдо-$R^2$}

Стандартным критерием качества линейной регрессионной модели является $R^2$. Обобщение этого критерия на класс нелинейных моделей называется \textbf{псевдо-}$R^2$. Этот показатель может быть получен несколькими способами.

Предпочтительным является критерий относительного прироста, обозначенный в параграфе 8.7.1 как $R^2_{RG}$. Значение этого критерия не всегда можно посчитать,  но это не касается моделей бинарного выбора,  поскольку $Q_{max}$,  максимально возможное значение логарифмической функции максимального правдоподобия,  равно нулю. Для получения данного результата, заметим, что наилучшее соответствие модели данным наблюдается, если $y^*$  предсказывает значение $y=1$ с вероятностью $p=1$ и значение $y=0$ с вероятностью $1-p=0$. В таком случае $f(y^*)=1, $ а $\ln f(y^*)=0$. Тогда $R^2_{RG}=1-(0-Q_{fit})/(0-Q_0)=1-Q_{fit}/Q_0$. Мы получаем критерий $R^2$ для моделей бинарного выбора,  предложенный МакФадденом  (1974): 

\begin{equation} 
\label{GrindEQ__14_14_} 
R^2_{Binary} = 1 - \frac{\mathcal{L}_N (\widehat\beta)}{\mathcal{L}_N (\overline{y})} = 1-\frac{\sum_i \{y_i \ln  \hat{p}_i+(1-y_i) \ln (1-\hat{p}_i) \}}{N[\overline{y} \ln \overline{y} + (1-\overline{y}) \ln  (1-\overline{y})]},  
\end{equation} 
где $\hat{p}_i=F(x'_i\widehat\beta)$ и $\overline{y}=N^{-1}\sum_i y_i.$

Другие $R^2$-подобные критерии,  часто специфичные для бинарных данных,  приведены в работах Амэмии  (1981)  и Маддалы  (1983). Одним из простых критериев является квадрат выборочной  корреляции между $y_i$ и $\hat{p}_i$. Один из этих дополнительных критериев был предложен в работе МакФаддена. Именно он,  а не $R^2$ по формуле \eqref{GrindEQ__14_14_}, часто приводится во многих исследованиях.

\subsubsection*{Предсказанные исходы}

Качество линейной регрессионной модели часто оценивается сравнением рассчитанных и реальных значений. Для бинарных данных спрогнозированная величина $\hat{y}$ должна быть бинарной,  поскольку исходная величина $y$ бинарна. Критерий $\sum_i (y_i - \hat{y}_i)^2$ равен количеству ошибочных прогнозов,  которые возникают,  если пара $(y,  \hat{y})$ принимает  значения $(1, 0)$ или $(0, 1)$. Естественным  правилом прогнозирования, будет правило присваивающее  $\hat{y}=1$,  если $\hat{p} = F(x'_i \widehat\beta) > 0.5$. Этот подход имеет недостаток,  поскольку,  если большинство исходов в выборке $y=1$,  то часто $\sum_i (y_i-\hat{y}_i)^2 = n(1-\overline{y})$. Тогда вполне вероятно,  что  $\hat{p} > 0.5$ и,  следовательно,  $\hat{y}=1$ для всех наблюдений. Эти же проблемы возникают,  если большинство исходов в выборке $y=0$.

Более того,  можно рассмотреть диапазон пороговых значений. Полагая $\hat{y}=1$  когда $\hat{p}>c$,  получим кривую \textbf{операционной характеристики приемника} (ROC-кривая),  которая отображает зависимость доли правильно классифицированных значений $y=1$ от доли неправильно классифицированных значений $y=0$ с изменением $c$. Для $c=1$ все прогнозные значения будут равны 0. Таким образом,  все значения  $y=1$ определены некорректно,  а все значения $y=0$ определены корректно. ROC-кривая принимает значение $(0, 0)$. Аналогично для $c=0$ ROC-кривая принимает значение $(1, 1).$

Если у модели нет предсказательной силы, то кривая ROC имеет вид прямой линии. Чем более выпукла кривая ROC, чем больше площади под ней, тем большей прогнозной силой обладает модель.

\subsubsection*{Прогнозируемые вероятности}

Поскольку бинарные данные имеют простое дискретное распределение,  можно, например, сравнить среднюю по выборке предсказанную вероятность  $y=1$, т.е. $N^{-1}\sum_i \hat{p}_i$,  где $\hat{p}_i = F(x'_i \widehat\beta)$,  с выборочной долей $\overline{y}$. Тем не менее,  этот подход непригоден для логит-модели с константой,  поскольку равенство $N^{-1}\sum_i \hat{p}_i=\overline{y}$ всегда выполняется, т.к.  условие первого порядка  ММП приводят к тому, что $\sum_i [y_i-\Lambda (x'_i\widehat\beta)]=0.$ Аналогичный результат верен и для модели линейной регрессии. Для пробит-модели результат будет несколько отличаться,  но на практике будет достаточно близким.

Тем не менее,  такой подход может быть использован для прогнозов по подвыборкам,  и,  быть  основой для построения хи-квадрат критерия качества модели,  приведенного в параграфе 8.2.6.

\section{Модели скрытых переменных}

\textbf{Скрытая переменная} --- это переменная,  которая наблюдается лишь частично. Скрытые переменные могут быть введены в модели бинарного выбора двумя различными способами. В первом случае скрытая переменная --- это индекс ненаблюдаемой склонности интересующего нас события к тому, что оно произойдет. Во втором случае скрытая переменная --- это разница в получаемой полезности,  если происходит желаемое событие. Этот случай предполагает,  что бинарный исход является результатом индивидуального выбора. Последний метод проясняет необходимость провести различие между регрессорами,  которые варьируются в зависимости от альтернатив для конкретного индивида,  и регрессорами,  такими как социально-экономические показатели,  которые постоянны для  конкретного индивида независимо от альтернатив.

Следует подчеркнуть,  что бинарные исходы имеют распределение Бернулли,  как  в разделе 14.3. Модели скрытых переменных просто дают обоснование для конкретной функциональной формы для параметра распределения Бернулли.

Модели скрытых переменных  обеспечивают обобщение моделей для случаев мультиномиальных или цензурированных исходов (подробнее см. в главах 15 и 16). Они также обеспечивают базу для Байесовского анализа с использованием метода пополнения данных (data augmentation), см. раздел 13.7. Байесовский анализ для случая данных бинарного и мультиномиального исходов кратко рассмотрен в параграфах 15.7.2 и 15.8.2.

\subsection{Индексная модель}

В формулировке с \textbf{индексной функцией} мы моделируем ненаблюдаемую непрерывную случайную переменной $y^*$,  при этом мы можем наблюдать бинарную переменную $y$,  которая принимает значение 1 или 0 в зависимости от того,  пересекает ли переменная $y^*$ заданный порог или нет. Различные предположения в отношении распределения переменной $y^*$ приводят к различным моделям бинарного выбора.

Пусть переменная $y^*$ является скрытой (или ненаблюдаемой),  например,  желание трудоустроиться при моделировании предложения труда. Естественной регрессионной моделью для переменной $y^*$ будет \textbf{индексная функциональная модель} 

\begin{equation} 
\label{GrindEQ__14_15_} y^*=x'\beta + u. 
\end{equation} 
Тем не менее,  эта модель не может быть оценена,  поскольку $y^*$ не наблюдаема. Вместо этого,  мы наблюдаем 

\begin{equation} 
\label{GrindEQ__14_16_} 
y = \begin{cases}
1  \ \text{если } y^*>0,  \\ 
0  \ \text{если } y^*\le 0,  
\end{cases} 
\end{equation}
где ноль в качестве граничного значения объясняется следующей нормализацией.

Из \eqref{GrindEQ__14_16_} следует, что

\begin{equation} 
\label{GrindEQ__14_17_} 
\Pr[y=1|x] = \Pr[y^* > 0] = \Pr[x'\beta + u > 0] = \Pr[-u < x'\beta] = F(x'\beta),  
\end{equation} 
где $F$ --- функция распределения $-u$,  которая равна функции распределения $u$, если плотность симметрична относительно нуля.

Индексная модель,  следовательно,  аргументирует использование функциональной формы $F(\cdot)$ в выражении \eqref{GrindEQ__14_1_}.

\subsubsection*{Пробит и логит-модели}

пробит-модель возникает,  если ошибка $u$ имеет нормальное стандартное распределение,  поскольку тогда из \eqref{GrindEQ__14_17_} следует, что  $\Pr[-u < x'\beta] = \Phi(x'\beta)$, где $\Phi(\cdot)$ --- функция стандартного нормального распределения.

Теперь рассмотрим \textbf{логистическое распределение}. В стандартной форме функция логистического распределения имеет следующий вид 

\begin{equation} 
\label{GrindEQ__14_18_} 
\Lambda(u) = e^u /(1+e^u), \, -\infty < u < \infty. 
\end{equation} 
Функция плотности $\Lambda'(u) = e^u/(1 + e^u)^2$ симметрична относительно 0. Для случайной переменной,  распределенной логистически,  среднее значение  равно 0,  а дисперсия --- $\pi^2/3\simeq {1,814}^2$.

логит-модель применяется,  если ошибка $u$ имеет логистическое распределение,  поскольку тогда из \eqref{GrindEQ__14_17_} следует $\Pr[-u < x'\beta] = \Lambda (x'\beta)$. Обратите внимание,  что $\beta$ рассчитывается по-разному в этих моделях,  поскольку разный вид имеет функция $\V[u].$ 

\subsubsection*{Соображения по поводу идентификации модели}

\textbf{Идентификация} одноиндексной модели требует наложения ограничения на дисперсию $u$, поскольку одноиндексная модель может оценить $\beta$ только с точностью до домножения на константу. Мы наблюдаем знаем только тот факт, принимает ли переменная $y^*$ значение больше нуля или нет,  или,  что эквивалентно,  выполнено ли неравенство $x'\beta +u>0$ или нет. Однако,  это эквивалентно неравенству $x'\beta^++u^+>0$, где $\beta^+=a\beta $ и $u^+=au$ для любых $a>0.$ Задавая условие на  дисперсию ошибок ($u$ или $u^+$) мы получаем уникальные  $\beta$. Дисперсия ошибки постулируется равной единицы для пробит-модели и $\pi^2/3$ для логит-модели.

Пороговое значение для индексной модели не обязательно должно быть равно нулю. В более общем случае, если  $y=1$, когда $y^*>z'\delta $,  то \eqref{GrindEQ__14_17_} принимает вид $\Pr[y=1] =F(x'\beta -z'\delta)$. Тогда $\delta $ может быть идентифицировано тогда и только тогда,  когда все компоненты $z$ и $x$ различаются. В частности,  если и $x$,  и $z$ содержат константу,  $\delta$ не может быть идентифицировано. В этом случае мы нормируем пороговое значение, приравнивая его к нулю. Обратите внимание,  что математическое ожидание ошибки также необходимо нормировать. Для логит и пробит-моделей это значение установлено равным нулю. 

\subsubsection*{Обсуждение}

Индексная модель подразумевает непосредственную интерпретацию $\beta$ как изменения в скрытой переменной $y^*$,  когда переменная $x$ изменяется на единицу. Несмотря на то,  что переменная $y^*$ ненаблюдаема,  такая интерпретация имеет смысл при фиксированной  дисперсии  ошибки $u$. Например,  параметр при независимой переменной в пробит-модели,  равный 0.5,  означает,  что изменение регрессора на одну единицу,  приведет к изменению стандартного отклонения $y^*$ на 0.5,  поскольку значение дисперсии $y^*$ в этой модели равно 1.

\textbf{Обобщения} индексного подхода включают в себя модели упорядоченного дискретного выбора (см. раздел 15.9) и модели для цензурирования и самоотбора выборки (см. главу 16). 

\subsection{Модели случайной полезности}

В моделях случайной полезности потребитель выбирает между альтернативами 0 и 1 в зависимости от того,  какая из них принесет ему большее удовлетворение или полезность. Дискретная переменная $y$ принимает значение 1,  если альтернатива 1 принесет большую полезность,  либо принимает значение 0,  если альтернатива 0 принесет большую полезность.

В \textbf{аддитивной модели случайной полезности} (additive random utility model, ARUM) полезности альтернатив 0 и 1 задаются следующим образом 

\begin{equation} 
\label{GrindEQ__14_19_} 
U_0 = V_0 + \e_0,  
\end{equation} 

\[
U_1 = V_1 + \e_1, 
\] 
где $V_0$ и $V_1$ являются детерминированными компонентами полезности,  а $\e_0$  и $\e_1$ --- случайными компонентами полезности. В качестве простого примера можно привести $V_0 = x'\beta_0$  и $V_1 = x'\beta_1$. Из раздела 14.4.3 следует, что тогда только величина $(\beta_1-\beta_0)$ является идентифицируемой.

Выбор делается в пользу альтернативы,  полезность которой больше. Допустим,  $y=1$ наблюдается,  если $U_1>U_0$. Благодаря наличию в полезности случайной компоненты,  это событие имеет вероятность  

\begin{equation}
\label{GrindEQ__14_20_} 
\begin{split}
\Pr[y=1] = \Pr[U_1 > U_0] \\
&= \Pr[V_1+\e_1 > V_0 + \e_0] \\
&= \Pr[\e_0 - \e_1 < V_1 - V_0] \\
&= F(V_1 - V_0),
\end{split}
\end{equation}
где $F$ функция распределения величины $(\e_0 - \e_1)$. Отсюда следует,  что $\Pr[y=1] = F(x'\beta)$, если $V_1 - V_0 = x'\beta$.

Применение модели ARUM требует нормирования, поскольку если $U_1 > U_0$, то $aU_1 > aU_0$. Это обычно обеспечивается определением дисперсии величины $\e_0 - \e_1$ или дисперсий величин $\e_0$ и $\e_1$.

Различные определения распределений величин $\e_0$ и $\e_1$ приводят к  различным $F(\cdot)$,  и,  следовательно,  различным моделям дискретного выбора. Модели случайной полезности особенно полезны при определении моделей неупорядоченного множественного выбора.

\subsubsection*{Пробит и логит-модели}

Вполне ясно,  что ошибки $\e_0$ и $\e_1$ из \eqref{GrindEQ__14_19_} имеют нормальное распределение. Тогда величина $(\e_0 - \e_1)$ также имеет нормальное распределение. Нормирование дисперсии величины $(\e_0 - \e_1)$ к единице приводит к пробит-модели,  поскольку тогда $F(\cdot)$ из выражения \eqref{GrindEQ__14_20_} --- функция стандартного нормального распределения.

Теперь рассмотрим \textbf{распределение экстремальных значений первого типа} или \textbf{ логарифмическое распределение Вейбулла}. Тогда плотность распределения переменной $\e$ задается как  

\begin{equation} 
\label{GrindEQ__14_21_} 
f(\e) = e^{-\e} \exp (-e^{-\e}), -\infty < \e <\infty,   
\end{equation} 
а функция распределения --- $F(\e) = \exp (-e^{-\e})$. Распределения экстремальных значений, редко используемые в эконометрике,  получаются как предел  при $N\to \infty $ распределения максимума из $N$ одинаково распределенных случайных величин. Распределение экстремальных величин первого типа -- это частный случай,  для которого коэффициент асимметрии положителен на $(-\infty, \infty)$ и основная масса значений находится между $-2$ и 5. Медиана этого распределения равна $-\ln (-\ln(0.5)) \simeq 0.36651$, среднее --- $\Gamma'(1)\simeq 0.57722$,  где $\Gamma'(x)$ обозначает производную гамма-функции,  и дисперсия $\pi^6/6 \simeq 1.28255^2$. Это распределение хорошо приближается лог-нормальным распределением.

логит-модель возникает,  если величины $\e_0$ и $\e_1$ полагаются независимыми и имеют распределение экстремальных величин первого типа. Тогда можно показать,  что разница $(\e_0 - \e_1)$ распределена логистически (см. работу Джонсона и Котца, 1970),  тогда $F(\cdot)$ --- функция логистического распределения.

Альтернативный вывод этого результата,  полученный при непосредственном применении распределения экстремальных величин,  приведен в разделе 14.8. Этот вывод указывает на сложность получения аналитического решения для вероятностей,  когда модель ARUM обобщается на случай выбора между тремя и более альтернативами (см. раздел 15.5). Современные вычислительные методы позволяют оценивать такие модели даже при отсутствии явных аналитических решений.

\subsection{Регрессоры,  изменяющиеся в зависимости от выбранной альтернативы}

В большинстве приложений моделей бинарного выбора некоторые регрессоры варьируются в зависимости от индивидов,  но не обязательно,  чтобы регрессоры варьировались в зависимости от альтернатив.

Одна крайность заключается в том,  что регрессоры не изменяются в зависимости от альтернатив. Например,  в моделях предложения труда (решение работать) социально-экономические факторы такие,  как доход и пол не изменяются в зависимости от альтернатив. Возможный регрессор,  такой как ставка заработной платы,  не изменяется в зависимости от выбора альтернативы работать или не работать,  но этот регрессор обычно не включается в модель,  поскольку он наблюдаем только среди тех,  кто решил работать.

Другая крайность состоит в том,  что все регрессоры могут меняться в зависимости от альтернативы. Например,  в модели выбора способа транспортировки регрессорами могут выступать затраты времени и денег в двух моделях транспортировки.

Общая гибридная модель ARUM определяет детерминированные компоненты полезности в выражении \eqref{GrindEQ__14_19_} как  

\begin{equation} 
\label{GrindEQ__14_22_} 
V_{ij} = z'_{ij} \alpha_j + w'_i\gamma_j, \hspace{0.3cm} j=0, 1,  
\end{equation} 
где $z_{ij}$ --- независимые переменные,  которые принимают разные значения в зависимости от выбранной альтернативы,  тогда как $w_i$ --- индивидуальные характеристики,  которые не изменяются в зависимости от сделанного выбора. Тогда из \eqref{GrindEQ__14_20_} получаем 

\[
\Pr[y_i=1] = F(z'_{i1} \alpha_1 - z'_{i0} \alpha_0 + w'_i(\gamma_1 - \gamma_0)).
\] 
Для \textbf{регрессоров,  постоянных для альтернатив},  только разница параметров $(\gamma_1 - \gamma_0)$ может быть определена. Для \textbf{регрессоров,  изменяющихся в зависимости от альтернатив}, которые изменяются также в зависимости от индивида,  коэффициенты могут изменяться в зависимости от альтернатив,  но обычно их считают, что $\alpha_1 = \alpha_0 = \alpha$. Например,  ожидается,  что потери в полезности при увеличении транспортных расходов на одну единицу скорее всего будут одинаковыми независимо от способа транспортировки. Таким образом, модель ARUM приводит к выражению

\begin{equation} 
\label{GrindEQ__14_23_} 
\Pr[y_i = 1] = F((z_{i1} - z_{i0})'\alpha + w'_i(\gamma_1 - \gamma_0)),  
\end{equation} 
которое является начальной моделью бинарного выбора \eqref{GrindEQ__14_1_},  где регрессорами являются: регрессоры $w$, не зависящие от альтернатив, и разницы регрессорами $z$, зависящих от альтернатив, $(z_{i1}-z_{i0})$. 

\section{ Выборки с самоотбором}

\textbf{Самоотбор выборки} возникает всякий раз,  когда данные в выборку попадают в зависимости от значений объясняемой  переменной $y$,  а не абсолютно случайно,  или частично детерминировано значениями,  которые принимает независимая переменная $x.$

В модели дискретных данных часто возникает самоотбор выборки,  поскольку при проведении опросов  часто сознательно отбирают случаи,  которые в реальности случаются редко. Например,  автобусом может пользоваться малое количество людей, а в выборке может оказаться  излишнее количество пассажиров автобусов. В медицинской литературе аналогичная проблема возникает при использовании \textbf{метода случай-контроль},  когда,  например,  бинарный анализ данных может быть основан на полной выборке тех,  у кого случился сердечный приступ,  и подвыборке тех людей,  у которых наблюдались аналогичные симптомы,  но не случился сердечный приступ. Стандартный термин --- отбор при построении выборки (choice based sampling) --- может ввести в некоторое заблуждение,  поскольку выборка строится не на основе индивидуального выбора испытуемых.

Чтобы увидеть несостоятельность стандартных методов бинарного выбора,  рассмотрим логит-модель с одной независимой переменной. Тогда $\Lambda (x'_i \beta) = \Lambda (\beta)$,  а условие первого порядка максимизации функции правдоподобия логит-модели примет вид $N^{-1}\sum_i (y_i-\Lambda (\beta)) = 0. $ Таким образом,  $\widehat\beta = \ln (\overline{y} / (1-\overline{y}))$. Состоятельность оценки $\widehat\beta$ определенно требует,  чтобы выборка была случайной,  поскольку,  например, излишнее предпочтение исхода $y=1$ приводит к неправильной оценке $\overline{y}$, и,  следовательно,  $\widehat\beta.$

Методы достижения состоятельности оценок модели для эндогенной выборки,  возникающей, например, при отборе при построении выборки,  подробно изложены в разделе 24.4. Анализ является довольно прост,  если известна степень завышения или занижения доли исходов в выборке. Допустим,  $Q_1$ обозначает долю населения,  для которой $y=1$,  и $H_1 = \overline{y}$ обозначает долю выборки,  для которой $y=1.$ По аналогии определим $Q_0 = 1-Q_1$ и $H_0 = 1-H_1.$ Тогда состоятельное оценивание возможно при использовании \textbf{взвешенного метода максимального правдоподобия},  предложенного Мански и Лерманом  (1977). Для моделей бинарного выбора максимизируется взвешенная логарифмическая функция правдоподобия 

\[
\mathcal{L}^W_N (\beta) = \sum^N_{i=1} \left\{ \left(\frac{Q_1}{H_1}\right)y_i \ln F(x'_i\beta) + \left(\frac{Q_0}{H_0}\right) (1-y_i) \ln (1-F(x'_i\beta)) \right\}.
\] 
Например,  если доля исхода $y=1$ в выборке завышена,  тогда $Q_1/H_1 < 1$,  и излишние наблюдения с исходом $y=1$ взвешиваются с меньшим весом. Такая оценка может быть легко получена при использовании программ для моделей бинарного выбора,  которые позволяют взвешивать наблюдения. Тогда наблюдениям с исходом $y=1$ будет добавлен вес $Q_1/H_1$,  а  наблюдениям с исходом $y=0$, будет добавлен вес $Q_0/H_0.$

Подробный обзор   методов максимального правдоподобия в случаях отбора при построении выборки для бинарных и множественных данных,  включая методы,  когда неизвестны $Q_1$ и $Q_0$,  приведен в работе Амэмии (1985,  раздел 9.5). Взвешенный метод максимального правдоподобия не является эффективным,  однако прост в использовании,  и потеря в эффективности может быть незначительной. В работах Мански и МакФаддена (1981а) предлагается более эффективный метод (см. работы Амэмии и Вуонга,  1987). Косслетт (1981a, b) предлагает дальнейшее улучшение,  которое является полностью эффективным,  но непрактичным для реализации. В работах Имбенса  (1992)  и Ланкастера и Имбенса  (1996)  предлагается оценка обобщенным методом моментов как альтернативный метод,  который возможно реализовать и который является полностью эффективным. Кинг и Ценг  (2001)  приводят обзор для бинарной логит-модели. Дополнительно они предлагают поправки для малых выборок, которые,  помимо учета завышения выборочной доли исхода, полезны  в случае малой вероятности исхода. Для более подробной информации обратитесь к разделу 24.4.

Литература по эпидемиологии сосредотачивается на логит-модели для исследований с помощью метода случай-контроль. Этот метод приписывают Прентису и Пайку  (1979). Можно обратиться к работе Бреслоу  (1996),  особенно к её разделу 4.3,  где обсуждается связь между эконометрикой и литературой по эпидемиологии.

\section{ Группировка и агрегирование данных}

В некоторых случаях могут быть доступны только сгруппированные или агрегированные данные,  несмотря на это индивидуальное поведение лучше всего моделируется с помощью моделей бинарного выбора. Группировка данных не создает проблем,  когда она основана на уникальных значениях регрессоров,  а на каждое уникальное значение регрессора приходится много наблюдений.  Начнем с простого примера,  прежде чем перейти к более реалистичному.

\subsection{Метод Берксона минимизации хи-квадрат}

Предположим,  независимая переменная $x_i$, $i=1, \ldots, N$ принимает только $T$ уникальных значений,  где $T\ll N$. Тогда для каждого значения регрессора имеются множественные наблюдения $y$. Такая группировка называется \textbf{множеством наблюдений на одну точку}. Такая выборка может возникнуть,  в частности,  для экспериментальных данных,  когда $x$ имеет небольшую размерность и принимает лишь несколько значений. Пусть $x_t$,  $t=1, \ldots,  T$ --- это  $T$ уникальных значений. $N_t$ --- число наблюдений за $y_t$ для $t$-ого уникального значения $x$,  тогда $\sum^T_{t=1} N_t=N$, а $\overline{p}_t$ является долей наблюдений $y_i=1$ когда $x_i=x_t$. Обратите внимание,  что индекс $t$ используется,  чтобы обозначить группировку,  и необязательно обозначает время.

Для каждого $i$,  для которого $x_i=x_t$,  формула вероятности Бернулли принимает вид  

\begin{equation} 
\label{GrindEQ__14_24_} 
p_t= \Pr[y_i = 1| x_i = x_t] = F(x'_t\beta),  
\end{equation} 
как и раньше. Обратив выражение \eqref{GrindEQ__14_24_},  получим

\[
F^{-1}(p_t) = x'_t\beta.
\] 
Сейчас величина $p_t$ неизвестна,  но может быть оценена с помощью $\overline{p}_t$. Поэтому Берксон предложил построить регрессию $F^{-1}(\overline{p}_t)$ на $x_t$. Таким образом,  мы будем оценивать с помощью МНК преобразованную модель  

\begin{equation} 
\label{GrindEQ__14_25_} 
F^{-1}(\overline{p}_t)  = x'_t\beta + v_t, \ \ t=1, \dots,  \ T. 
\end{equation} 
Ошибка $v_t = F^{-1}(\overline{p}_t) - F^{-1}(p_t)$ гетероскедастична.  Дисперсия будет снижаться при увеличении $N_t$,  поскольку ${\overline{p}}_t$ точнее оценивает $p_t$,  а также будет зависеть от формы $F(\cdot )$. Используя метод разложения в ряд Тейлора (см. работы Амэмии (1981,  стр. 1498) или Маддалы (1983,  стр. 31)),  получим состоятельную оценку дисперсии ошибки $v_t$ 

\begin{equation} 
\label{GrindEQ__14_26_} 
\overline{\sigma}^2_t = \frac{\overline{p}_t (1 - \overline{p}_t) }{N_t [F'(F^{-1}(\overline{p}_t))]^2}. 
\end{equation} 
\textbf{Оценка} $\widehat{\mathbf{\beta}}_{\mathbf{MC}}$, \textbf{полученная методом Берксона}, минимизирует взвешенную сумму остатков $\sum^T_{t=1} (F^{-1} (\overline{p}_t) - x'_t\beta) / \overline{\sigma}^2_t$ по $\beta$. Метод легко реализовать с помощью МНК,  построив регрессию $F^{-1}(\overline{p}_t)/
\overline{\sigma}_t$ на $x_t/\overline{\sigma}_t.$

Этот метод прост в использовании,  поскольку требует только использование МНК. Несмотря на это,  он полностью эффективен,  поскольку,  можно показать,  что оценка имеет тоже асимптотическое распределение,  что и оценка,  полученная методом максимального правдоподобия,  при котором каждое наблюдение берется в отдельности,  а не группируется вместе с наблюдениями с тем же значением $x_t$. Для логит-модели этот метод особенно прост,  поскольку $F^{-1}(\overline{p}_t) = \ln (\overline{p}_t / (1 - \overline{p}_t))$ и $\overline{\sigma}^2_t = 1/[N_t \overline{p}_t (1 - \overline{p}_t)].$

Преимуществом метода Берксона является простота расчетов,  хотя растущая  производительность компьютеров делает этот пункт спорным. Для сгруппированных экономических данных редко бывает ситуация,  когда есть несколько наблюдений  на одно уникальное значение регрессоров,  за исключением случаев,  когда регрессоры является совокупностью нескольких индикаторов. Метод Берксона помогает лучше понять агрегирование данных. Далее мы рассмотрим эту тему.

\subsection{Моделирование агрегированных данных}

Примером \textbf{агрегирования данных} в эконометрике может служить моделирование зависимости доли занятых людей или доли тех,  кто пользуется автобусными перевозками в разных регионах, в зависимости от  средних характеристик людей по регионам.

Приведем конкретный пример. Обозначим $\overline{p}_t$ уровень безработицы в регионе $t$,  а $\overline{x}_t$ --- средний уровень образования в регионе $t$. Одной из возможных моделей может быть линейная регрессия $\overline{p}_t$ на $\overline{x}_t$. Поскольку $0 < \overline{p}_t < 1, $ многие исследования,  вместо этого преобразовывают  $\overline{p}_t$ в неограниченную зависимую переменную и оценивают модель

\begin{equation} 
\label{GrindEQ__14_27_} 
\ln \left(\frac{\overline{p}_t}{1 - \overline{p}_t}\right) =\overline{x}'_t \beta +u_t,  
\end{equation} 
где $u_t$ -- ошибка.

Эта модель очень похожа на модель в методе Берксона для логит-модели,  когда $F^{-1}(\overline{p}_t) = \ln  (\overline{p}_t/(1-\overline{p}_t))$. Однако,  это не потому,  что метод Берксона приемлем, только если все регрессоры в $t$-ой ячейке принимают одинаковое значение. Здесь,  наоборот,  регрессоры могут принимать различные значения, у разных людей в регионе $t$ может быть разный уровень образования.

Чтобы увидеть последствия агрегирования,  когда имеет место \textbf{неоднородность регрессора в одной ячейке},  предположим,  что модель поведения индивида является индексной моделью (см. подраздел 14.4.1),  для которой 

\[
y^*_i=x'_i\beta +u_i, 
\] 

\[
u_i \sim \mathcal{N}[0, 1].
\] 
Будем считать,  что ошибка имеет нормальное распределение,  как в  пробит,  а не в логит-модели,  поскольку тогда возможно получение аналитических результатов. 

Смоделируем неоднородность как

\[
x_i \sim \mathcal{N}[\mu_t, \Sigma_t], 
\] 
для индивидов в ячейке $t$. Это действительно допускает вариацию между ячеек,  а сложность заключается в том,  что $\Sigma_t \ne 0$, т.е. присутствует неоднородность регрессора в одной ячейке. Тогда для региона $t$ при фиксированных $\beta$, $\mu_t$ и $\Sigma_t$, 

\begin{multline*}
\Pr[y_i = 1] = \Pr[x'_i\beta + u_i > 0] \\
=\Pr\left[\frac{x'_i\beta + u_i - \mu'_t\beta}{\sqrt{1+\beta'\Sigma_t\beta}} > \frac{-\mu'_t\beta}{\sqrt{1+\beta'\Sigma_t\beta}}\right] \\
=\Phi \left(\frac{\mu'_t\beta}{\sqrt{1+\beta'\Sigma_t\beta}}\right),
\end{multline*} 
где мы использовали полученное из предпосылок распределение суммы $x'_i \beta + u_i \sim \mathcal{N}[\mu'_t\beta,  (1+\beta'\Sigma_t \beta)]$. Чтобы величина имела стандартное нормальное распределение мы вычли среднее и разделили на стандартное отклонение.

Используя те же аргументы,  которые привели к выражению \eqref{GrindEQ__14_25_} от \eqref{GrindEQ__14_24_},   параметр $\beta$ индивидуального бинарного выбора может быть состоятельно оценен с помощью нелинейного МНК в регрессии

\begin{equation} 
\label{GrindEQ__14_28_} 
\Phi^{-1}(\overline{y}_t) = \frac{\overline{x}'_t\beta}{\sqrt{1+\beta'S_t \beta}} + w_t,  
\end{equation} 
где $\overline{y}_t$ и $\overline{x}_t$ --- средние значения,  а $S_t$ --- выборочная дисперсия $x_i$ в ячейке $t$. Метод минимизации хи-квадрат Берксона,  строит регрессию $\Phi^{-1}(\overline{y}_t)$ на $\overline{x}_t$ и даёт состоятельные оценки для $\beta$ только если $\Sigma_t=0.$

\subsection{Обсуждение}

Проблема агрегированных данных намного сложнее в нелинейных моделях. Если изначально модель индивидуального выбора была линейной $y_i = x'_i\beta + u_i$ с $x_i \sim \mathcal{N}\left[\mu_t, \Sigma_t\right]$ в $t$-ой ячейке,  тогда соответствующая линейная регрессия $\overline{y}_t$ на $\overline{x}_t$ даст состоятельную оценку $\beta $. Для нелинейных моделей аналогичная агрегация приводит к несостоятельности оценки параметра индивидуального выбора,  если не будет предпринята корректировка,  как в выражении \eqref{GrindEQ__14_28_}. Более того,  пример,  приведенный в параграфе 14.6.2 из работы МакФаддена и Рейда  (1975) ,  нетипичен,  поскольку в нем агрегирование  приводит к явным результатам, что нетипично для нелинейных моделей. Данный пример очень подробно рассмотрен в работе Камерона  (1990),  который изучал его в более широком контексте \textbf{агрегирования данных в нелинейных моделях}.

Активно проблема агрегирования данных дискретного выбора,  обычно множественного,  обсуждается в литературе по маркетингу в теме \textbf{рыночной доли брендов}. В работе Олленбая и Росса  (1991)  представлен пример,  когда смещение агрегированной логит-модели оказывается небольшим. Что более важно,  современные вычислительные алгоритмы позволяют оценить параметры индивидуального выбора в случае агрегированных данных,  даже если агрегирование приводит к неаналитическому решению. С примером можно ознакомиться в работе Бэрри  (1994)  и Нэво  (2001),  в которой оцениваются модели,  качественно  аналогичные  логит-модели со случайными параметрами,  описанной в разделе 15.7.

И наконец,  обратите внимание,  что для многих случаев агрегированных показателей таких,  как уровень безработицы по региону,  исследователя не интересуют параметры индивидуального выбора. Единственной задачей становится,  построение разумной модели для зависимой переменной $\overline{p}_t$,  которая принимает значения между нулем и единицей. Тогда линейная регрессия \eqref{GrindEQ__14_27_} может подходить. Ошибка $u_t$ в выражении \eqref{GrindEQ__14_27_} больше не будет иметь дисперсию,  приведенную в \eqref{GrindEQ__14_26_}. Эта ошибка будет гетероскедастичной,  поэтому статистические выводы должны быть основаны на робастной к гетероскедастичности стандартной ошибке Уайта.

\section{Полупараметрические методы}

Модель бинарного выбора,  возможно,  является главным примером полупараметрической регрессии. Большинство эконометрических исследований предполагают одноиндексную форму $F(x'_i\beta )$,  когда функциональная форма $F$ не определена. Целью исследования является нахождение состоятельной оценки параметра $\beta$,  в идеальном случаем $\sqrt{N}$-состоятельной и асимптотически нормальной,  в то время как $F(\cdot )$ рассматривается как мешающая функция. Здесь может применяться одноиндексная полупараметрическая модель из параграфа 9.7.4. Дополнительные методы оценки интерпретируют модель индексной функции для ситуаций бинарного исхода. 
Кроме того,  возможно использование полупараметрического метода максимального правдоподобия,  который достигает границы эффективности полупараметрических методов. При этом не приходится прибегать к  дополнительным предположениям,  поскольку ясно,  что применяется распределение Бернулли,  и только $F(x'_i\beta )$ неизвестна.

\subsection{Полупараметрические методы оценки условного среднего}

В общей задаче оценивания зависимая переменная $y_i$ принимает значения,  равные 0 или 1,  с условным средним

\[
\E[y_i |x_i] = m(x_i), 
\] 
где $m(\cdot )$ --- неизвестна. Обратите внимание,  что $m(x_i)$ также равно условной вероятности наступления события $y_i=1$.

Непараметрические регрессионные методы,  описанные в разделах 9.4 --- 9.6,  могут быть использованы,  несмотря на бинарную природу зависимой переменной. Это хорошо видно на рисунке 14.1,  на графике рассеивания бинарной переменной $y$, зависимой от скалярной переменной $x$. Естественно рассмотреть ядерную регрессии $y$ на $x$. Предсказанные  значения будут лежать между 0 и 1,  за исключением особых случаев таких,  как использование ядерных функций более высокого порядка,  когда предсказанные значения могут быть отрицательными.

Для многих микроэкономических задач переменная $x$ имеет слишком много измерений,  чтобы непараметрические методы работали хорошо (проклятие размерности). Полупараметрические регрессионные модели,  которые частично определяют $m(\cdot)$,  приведены в разделе 9.7. Аддитивные модели довольно популярны для решения статистических задач. В эконометрике,  наоборот,  используется одноиндексная модель,  поскольку популярным начальным пунктом является индексная модель из параграфа 14.4.1. Мы получаем \textbf{одноиндексную модель} со скрытой переменной $y^*=x'\beta +u.$ Тогда

\[
\E[y_i| x_i] = F(x'_i\beta), 
\] 
где мы используем обозначения из этой главы, в частности $F(\cdot)$ вместо $g(\cdot)$,  чтобы обозначить неизвестную функцию. 

Из параграфа 9.7.4,  параметр $\beta$ может быть определен только с точностью до сдвига и масштабирования. Это также видно из параграфа 14.4.1,  где ошибка $u$ в индексной модели была нормирована,  чтобы ее среднее равнялось нулю (сдвиг),  а дисперсия была фиксирована (масштабирование). Теперь ограничения на ошибку $u$ не накладываются,  поэтому параметр $\beta$ не может быть полностью идентифицирован,  однако отношение коэффициентов при независимых переменных идентифицируемо. Для  более подробного анализа проблемы идентификации в моделях бинарного выбора можно ознакомиться с работой Мански (1988b).

$\sqrt{N}$-состоятельная асимптотически нормальная оценка $\beta$ может быть получена путем оценки средней производной  или полупараметрическим МНК (см. параграф 9.7.4). Тем не менее,  чаще используются альтернативные модели,  специфические для ситуаций бинарного выбора.

\subsection{Оценивание по методу максимального счета}

Полупараметрические методы для бинарного выбора часто основываются на индексной модели $y^*=x'\beta + u$. В этом случае удобнее записать модель следующим образом 

\[
y_i = 1(x'_i\beta + u_i > 0), 
\] 
где $1(A)=1$,  если событие $A$ происходит.

Мански  (1975)  отмечает,  что прогнозным значением $y_i$ является $1(x'_i\beta >0)$. Если считать, что $u_i=0$,  поскольку $u_i$ неизвестно,  то количество  верных прогнозов будет равно

\begin{equation} 
\label{GrindEQ__14_29_} 
S_N(\beta)=\sum^N_{i=1} \{y_i 1(x'_i\beta > 0) + (1-y_i) 1(x'_i\beta \le 0)\},  
\end{equation} 
поскольку верные прогнозы случаются, если $y_i=1$ и $1(x'_i\beta >0)$ или $y_i=0$ и $1(x'_i\beta \le 0).$ Оценка $\widehat\beta_{MS}$, полученная по \textbf{методу максимального счета} Мански,  максимизирует функцию $S_N(\beta)$. Это нестандартная задача,  поскольку $1(x'_i\beta > 0)$ является недиффиренцируемой по $\beta.$ Мански (1975,  1985) доказал состоятельность в предположении что $F(0) = 0.5$ или,  что эквивалентно,  $\mathrm{Median}[u_i|x_i]=0.$ В дальнейшем было доказано,  что $N^{1/3}(\widehat\beta_{MS} - \beta)$ имеет ненормальное предельное распределение,  поэтому статистические выводы можно делать,  используя бутстрэп (см. работу Мански и Томпсона (1986)).

Метод Мански можно рассмотреть как метод абсолютных отклонений. Из параграфа 4.6.2,  в методе абсолютных отклонений минимизируется сумма разниц между $y_i$ и медианой $\mathrm{Median}[y_i|x_i]$. Этот менее знакомый метод аналогичен МНК,  в котором минимизируется сумма квадратов разниц между $y_i$ и $\E[y_i|x_i]$. % в английском оригинале ошибка
Чтобы использовать метод абсолютных отклонений,  необходимо получить $\mathrm{Median}[y_i|x_i]$. Если $\mathrm{Median}[u_i|x_i] = 0$,  то $\mathrm{Median}[y^*_i|x_i] = x_i'\beta$, тогда $\mathrm{Median}[y_i|x_i] = 1(x'_i\beta > 0).$ Тогда метод абсолютных отклонений в модели бинарного выбора минимизирует функцию 

\begin{equation} 
\label{GrindEQ__14_30_} 
Q_N(\beta) = \sum^N_{i=1} |y_i-1(x'_i\beta > 0)|. 
\end{equation} 
Из упражнения 14.4 $Q_N(\beta) = N - S_N(\beta)$, поэтому оценка по методу максимального счета эквивалентна оценке,  полученной методом абсолютных отклонений. Чтобы ознакомиться с другими способами интерпретации метода максимального счета как метода наименьших отклонений, обратитесь к работе Мански (1985,  стр. 320).

Целевая функция $S_N(\beta)$ по методу максимального счета, приведенная в выражении \eqref{GrindEQ__14_29_},  недифференцируема. Ее можно переписать в виде

\[
S_N(\beta) = \sum^N_{i=1} (2y_i-1) 1(x'_i\beta > 0) + N - \sum^N_{i=1} y_i,
\] 
(см. упражнение 14.4). Вторая сумма в выражении может быть проигнорирована, поскольку не содержит $\beta$.

Метод, использующий дифференцируемую целевую функцию, называется \textbf{сглаженным методом максимального счета} Хоровица, в нём максимизируется  функция

\[
Q^S_N(\beta) = \sum^N_{i=1} (2y_i-1) K (x'_i\beta/h_N),
\] 
где $K(x'\beta/h_N)$ --- сглаженная версия $1(x'\beta > 0).$ Поскольку $1(x'\beta > 0)$ принимает значение,  равное нулю,  при отрицательных значениях $x'\beta $,  и равное единице,  при положительных значениях $x'\beta $, естественно выбрать в качестве $K(\cdot)$ функцию распределения,  для которой выполняется $K(0)=0.5$,  и небольшое значение $h_N$. Сглаживание облегчает  расчет оценки,  но анализ осложнён необходимостью сходимости $h_N \to 0$ с подходящей скоростью при  $N \to \infty $. Оценка асимптотически нормальна, состоятельна со скоростью сходимости близкой к $\sqrt{N}$. Для дополнительной информации ознакомьтесь с работой Хоровица,  который предлагает алгоритм бутстрэпа,  снижающий вероятность ошибки первого рода в тестах на малых выборках.

Метод абсолютных отклонений может быть применен к цензурированным регрессионным моделям (см. параграф 16.9.2). 

\subsection{Метод максимальной ранговой корреляции}

Начнем с одноиндексной модели с $\E[y_i| x_i] = F(x'_i\beta)$. Если $F(x'_i\beta)$ монотонно возрастает по $x'_i\beta$,  то $\E[y_i| x_i] > \E[y_j| x_j]$ при $x'_i\beta > x'_j\beta $. 
Таким образом,  вполне вероятно,  хотя и не гарантированно,  что наблюдаемая величина $y_i>y_j$,  когда $x'_i\beta > x'_j\beta $. Это наводит на мысль, что имеет смысл подобрать такое $\beta$,  при котором с большой частотой $y_i > y_j$, если $x'_i\beta > x'_j\beta $.

\textbf{Метод максимальной ранговой корреляции} (Maximum Rank Correlation, MRC) Хана (1987) выбирает $\beta$,  которое максимизирует функцию

\[
Q^{MRC}_N(\beta) = \sum^N_{i=1} \sum^N_{\begin{array}{c} j=1 \\ j < i \end{array}} 
1(y_i > y_j) 1(x'_i\beta > x'_j\beta) + 1(y_i < y_j) 1(x'_i\beta < x'_j\beta).
\] 
В этой сумме $ij$-ый элемент равняется единице,  если $y_i>y_j$,  когда $x'_i\beta > x'_j\beta $,  или $y_i < y_j$,  когда $x'_i\beta < x'_j\beta $.  Слагаемое равняется нулю,  если наоборот имеет место $y_i < y_j$,  когда $x'_i\beta > x'_j\beta $,  или $y_i > y_j$,  когда $x'_i\beta < x'_j\beta $. Метод называется методом максимальной ранговой корреляции,  потому что $Q^{MRC}_N(\beta)$ пропорционально коэффициенту ранговой корреляции Кендалла между $y_i$ и $x'_i\beta $.

Оценка, полученная этим методом, является $\sqrt{N}$-состоятельной и асимптотически нормальной (см. работу Шермана, 1993). 

\subsection{Полупараметрический метод максимального правдоподобия}

Для моделей бинарного  функция правдоподобия обязательно имеет вид \eqref{GrindEQ__14_4_} при условии  независимости наблюдений. Единственная сложность заключается в том,  что форма $F\left(\cdot \right)$ неизвестна. Кляйн и Спэйди (1993) предложили \textbf{полупараметрический метод макисмального правдоподобия},  который максимизирует функцию 

\[
\mathcal{L}_N(\beta) = \sum^N_{i=1} \left\{y_i \ln \hat{F}(x'_i\beta) + (1-y_i) \ln (1-\hat{F}(x'_i\beta)) \right\},
\] 
где $\hat{F}(x'_i\beta)$ -- непараметрическая оценка $F(x'_i\beta)$.

Этот метода в плане идеи аналогичен взвешенному полупараметрическому МНК Ишимуры (1993),  разобранному в параграфе 9.7.4. Поэтому при вычислении этих оценок возникают сходные вопросы. Точно также алгоритм поочередно  вычисляет то $\widehat\beta$ при заданном $\hat{F}$, то $\hat{F}$ при заданном $\widehat\beta.$ Исходя из условия первого порядка максимизации функции правдоподобия \eqref{GrindEQ__14_5_} оценка полупараметрическим методом правдоподобия может быть вычислена как решение уравнения 

\[
\sum^N_{i=1} \frac{\widehat{F'}(x'_i\beta)}{\hat{F}(x'_i\beta)(1-\hat{F}(x'_i\beta))} (y_i-\hat{F}(x'_i\beta))x_i=0, 
\] 
которое аналогично выражению для взвешенного полупараметрического МНК с весами $w_i=\hat{F}'_i/[\hat{F}_i (1-\hat{F}_i)].$

Привлекательность метода Кляйна и Спэйди заключается в том,  что он полностью эффективен в том смысле,  что он достигает границы эффективности полупараметрических методов. Тем не менее,  процесс вычисления достаточно сложен. Более подробно проблемы вычисления оценки методом Ишимуры рассматривается в параграфе 9.7.4,  а также в работах Кляйна и Спэйди (1993) и Пагана и Уллаха (1999, стр. 283 --- 285).  

\subsection{Cравнение полупараметрических методов}

Внимание эконометристов обычно сосредоточено на одноиндексных моделях,  и даже несмотря на это существует масса полупараметрических методов для оценки моделей бинарного выбора. Ни один из этих методов на практике не оказывается существенно проще других для использования. Целевые функции могут иметь множество оптимальных точек и могут не быть гладкими. К примеру,  Хоровиц  (1992)  использует алгоритм имитации отжига для сглаженного метода максимального счета,  а Дорси и Майер  (1995)  применяют генетические алгоритмы для метода максимального счета.

Интерпретация коэффициентов также сложна. Например,  метод максимального счета,  примененный к данным по рыбалке,  приводит к оценке свободного члена 0.776 и оценке наклона $-0.631$ (с бутстрэп стандартной ошибкой 0.103),  но эти коэффициенты невозможно напрямую сравнить с коэффициентами из таблицы 14.2. Действительно,  поскольку оценки параметов при независимых переменных определены с точностью до масштаба,  полупараметрические оценки являются полезными,  если в регрессию включено несколько регрессоров. В этом случае оценки  коэффициентов можно сравнивать с оценкой коэффициента, выбранного за  базу сравнения.

Методы максимального счета и максимизации ранговой корреляции нетипичны среди полупараметрических методов тем,  что не требуют сглаживания параметров,  таких как ширина окна сглаживания,  что является их привлекательной чертой. Результаты таких методов $\sqrt{N}$-состоятельны.

В недавней работе Бланделла и Пауэлла  (2004)  был предложен полупараметрический метод оценки с \textbf{эндогенными регрессорами}.

\section{От распределения экстремальных значений к логит-модели}

При переходе к логит-модели от модели ARUM в разделе 14.4.2 использовалось утверждение,  что разница $(\e_0 - \e_1)$ случайных величин имеющих распределение экстремальных значений имеет логистическое распределение. Для полноты изложения приведем доказательство,  основанное на распределении $\e_0$ и  $\e_1.$

Переписав вторую строку выражения \eqref{GrindEQ__14_20_},  получим 

\begin{equation}
\label{GrindEQ__14_31_}
\begin{split}
\Pr[y=1] = \Pr[\e_0 < \e_1 + V_1 - V_0] \\
&= \int^{\infty}_{-\infty} \int^{\e_1 + V_1 - V_0}_{-\infty} f(\e_0, \e_1) d\e_0 d\e_1 \\
&= \int^{\infty}_{-\infty} f(\e_1) \left\{ \int^{\e_1 + V_1 - V_0}_{-\infty} f(\e_0) d\e_0 \right\} d\e_1, 
\end{split}
\end{equation} 
в последней строке предполагается,  что величины $\e_0$ и $\e_1$ --- независимы. Пусть $f(\e_0)$ --- плотность распределения экстремальных значений первого типа,  тогда из \eqref{GrindEQ__14_31_} получим 

\begin{equation} 
\label{GrindEQ__14_32_} 
\begin{split}
\Pr[y=1] = \int^{\infty}_{-\infty} f(\e_1) \left\{\int^{\e_1 + V_1 - V_0}_{-\infty} e^{-\e_0} \exp (e^{-\e_0)} d\e_0 \right\} d\e_1 \\
& = \int^{\infty}_{-\infty} f(\e_1) [\exp (-e^{-\e_0})]^{\e_1 + V_1 - V_0}_{-\infty} d\e_1 \\
& = \int^{\infty}_{-\infty} f(\e_1) \exp (-e^{-(\e_1 + V_1 - V_0)}) d\e_1.
\end{split}
\end{equation} 

Используя плотность распределения экстремальных значений для $\e_1$,  из \eqref{GrindEQ__14_32_} получим 

\begin{equation} 
\label{GrindEQ__14_33_} 
\begin{split}
\Pr[y=1] = \int^{\infty}_{-\infty} e^{-\e_1} \exp (-e^{-\e_1}) \exp (-e^{-(\e_1 + V_1 - V_0)}) d\e_1 \\
& = \int^{\infty}_{-\infty} e^{-\e_1} \left\{ \exp (-e^{-\e_1} - e^{-(\e_1 + V_1 - V_0)} \right\} d\e_1 \\
& = \int^{\infty}_{-\infty} e^{-\e_1} \left\{ \exp (-e^{-\e_1} - e^{-\e_1} e^{-(V_1 - V_0}) \right\}  d\e_1 \\
& = \int^{\infty}_{-\infty} e^{-\e_1} \exp  \left\{ -e^{-\e_1} (1+e^{-(V_1 - V_0)}) \right\} d\e_1.
\end{split}
\end{equation} 
Из $\int^{\infty}_{-\infty} ae^{-\e} \exp (-ae^{-\e}) d\e = 1$ следует,  что $\int^{\infty}_{-\infty} e^{-\e} \exp (-ae^{-\e}) d\e = 1/a.$ Используя это и $a = 1+e^{-(V_1 - V_0)}$,  из выражения \eqref{GrindEQ__14_33_} получим 

\begin{equation} 
\label{GrindEQ__14_34_}
\begin{split} 
\Pr[y=1] = (1+e^{-(V_1 - V_0)})^{-1} \\
& = e^{V_1}/(e^{V_0} + e^{V_1}) \\
& = e^{V_1 - V_0}/(1 + e^{V_1 - V_0}). 
\end{split}
\end{equation} 

Используя равенство  $V_1 - V_0 = x'\beta $ получаем логит-модель. 

\section{Практические соображения}

Логит и пробит-модели встроены в большинство статистических пакетов. Основной вопрос состоит в выборе модели. На практике разница между предельными эффектами,  предсказанными этими моделями,  получается небольшой,  за исключением случаев, когда большинство исходов либо равно нулю,  либо равно единице.

Полупараметрические методы обычно требуют написания дополнительного кода на языках программирования,  таких как GAUSS,  хотя на языке Lindep осуществлены методы Мански,  Кляйна и Спэди.

\section{Библиографические заметки}

Логит и пробит-модели являются часто используемыми и относительно простыми нелинейными регрессионными моделями,  которые изложены во многих стандартных учебниках,  например, у Грина (2003). Обзоры Амэмии (1981)  и МакФаддена  (1984)  содержат все основные результаты. Маддала  (1983)  и Амэмия приводят много деталей. Книги Трейна (1986) и Бен-Акивы и Лермана (1985) хороши для практического применения. Эти источники охватывают как модели бинарного,  так и множественного выбора.

\begin{enumerate}
\item [$14.3$] Блисс  (1934)  предложил пробит преобразование для графика кривой \hl{доза-смертность}. Берксон  (1951)  популяризировал использование простейшей логит-модели.
\item [$14.4$] Модели скрытых переменных особенно популярны в литературе по \hl{физиометрии}.
\item [$14.5$] В работе Амэмии (1985,  раздел 9.5) приведено блестящий обзор вопросов самоотбора выборки в моделях бинарного выбора (см. также раздел 24.4). 
\item [$14.6$] Камерон  (1990)  изучал проблему агрегирования данных в моделях бинарного выбора и обобщил результаты исследований Келияна  (1980)  и Стокера  (1984)  о возможности оценивания индивидуальных параметров  в нелинейных моделях,   используя агрегированные данные.
\item [$14.7$] Метод максимального счета Мански  (1975)  --- это один из основных примеров ранних исследований полупараметрических регрессий. Полупараметрические методы для моделей бинарного выбора описаны в работах М.-Дж. Ли  (1996),  Хоровица  (1997)  и Пагана и Уллаха  (1999). Особенно много методов описывается в последней работе.
\end{enumerate}

\section*{Упражнения}

\begin{enumerate}
\item [$14 - 1$] Рассмотрим скрытую перемененную заданную уравнение $y^*_i = x'_i\beta +\e_i$,  где $\e_i \sim \mathcal{N} [0, 1].$ Предположим, что мы наблюдаем только $y_i=1$, если $y^*_i < U_i$ и $y_i = 0$,  если $y^*_i\ge U_i$,  где верхняя граница $U_i$ --- известная для каждого индивида константа,  которая может изменяться в зависимости от индивида.
\begin{enumerate}
\item  Найдите $\Pr [y_i=1 | x_i]$. Подсказка: Обратите внимание,  что предлагаемый случай отличается от стандартного,  во-первых,  потому что присутствует $U_i$, а во-вторых,  неравенства изменило знак, теперь $y_i = 1$,  если $y^*_i < U_i$.
\item  Опишите подробные шаги получения состоятельной оценки параметра $\beta$.
\item  Допустим,  вы оценили параметры модели и обнаружили,  что оценка параметра для третьего регрессора $x_{3i}$ принимает значение $\widehat\beta_3 = 0.2.$ Приведите содержательную интерпретацию оценки $\widehat\beta_3.$
\end{enumerate}

\item [$14 - 2$] Рассмотрим логит-модель $\Pr[y=1|x_1, x_2] = \Lambda (\beta_0 + \beta_1 x_{1i} + \beta_2x_{2i})$, где $\Lambda (\rm{z}) = e^z/(1+e^z)x.$
\begin{enumerate}
\item  Выпишите производную функции правдоподобия и информационную матрицу в развернутом виде.
\item Используйте их чтобы провести тест Вальда и тест множителей Лагранжа для гипотезы $H_0: \beta_2 = 0$.
\item  Объясните, как реализовать тесты вычислительными методами
\item  В каком смысле логит-модель по своей сути гетероскедастична?
\end{enumerate}

\item [$14 - 3$] Предположим,  мы используем индексную форму для модели дискретного выбора,  но скрытая переменная принимает строго положительные значения. Поэтому мы используем предположение о том,  что скрытая переменная $y^*$ имеет экспоненциальную функцию плотности с параметром $\gamma $,  тогда плотность распределения $f(y^*)$ принимает вид $f(y^*) = \gamma^{-1} \exp (-y^*/\gamma)$,  с параметром $\gamma = \exp (x'\beta).$ Мы наблюдаем $y = 1$,  если $y^* > z'\alpha$,  и $y = 0,$ если $y^* \le z'\alpha$.
\begin{enumerate}
\item  Постройте логарифмическую функцию правдоподобия для наблюдаемых данных.
\item Какой эффект произведет изменение на одну единицу переменной $x_{ji}$ на вероятность $\Pr[y_i = 1]?$
\item  Предположим,  что $y = 1$,  если $y^* > \exp (z'\alpha)$ и $x = z.$ Существуют ли какие-либо сложности с идентификацией $\alpha$ или $\beta$? Объясните свой ответ.
\end{enumerate}

\item [$14 - 4$] Рассмотрим оценивание по методу максимального счета с целевыми функциями $S_N(\beta)$ \eqref{GrindEQ__14_29_} и $Q_N(\beta)$ \eqref{GrindEQ__14_30_}.
\begin{enumerate}
\item  Покажите,  что $S_N(\beta) = \sum_i [1(y_i = 1) \times 1(x'_i\beta > 0) + 1(y_i = 0) \times 1(x'_i\beta \le 0)].$
\item  Покажите,  что $Q_N(\beta) = \sum_i [1(y_i = 1) \times 1(x'_i\beta \le 0) + 1(y_i = 0) \times 1(x'_i\beta > 0)].$
\item  Используя $1(y_i = 1) = 1 - 1(y_i = 0)$, покажите, что $Q_N(\beta) = N - S_N(\beta).$
\item  Используя $1(x'_i\beta \le 0) = 1 - 1(x'_i\beta > 0)$, покажите, что выражение \eqref{GrindEQ__14_29_} может быть записано как $S_N(\beta) = \sum_i (2y_i - 1) 1(x'_i\beta > 0) + N - \sum_i y_i.$
\end{enumerate}

\item [$14 - 5$] Воспользуйтесь данными о расходах на здравоохранение из раздела 16.6. Рассмотрим пробит регрессию зависимой переменной DMED, индикатора положительности расходов на здравоохранение,  для простоты только от одной независимой переменной NDISEASE,  количества хронических заболеваний.
\begin{enumerate}
\item  Оцените коэффициент при независимой переменной методом наименьших квадратов.
\item  Найдите пробит оценку коэффициента при независимой переменной.
\item  Используя результаты из пункта (b),  определите предельный эффект от роста количества хронических заболеваний двумя путями: средний по выборке и оцененный с помощью среднего по выборке значения переменной NDISEASE.
\item  Найдите логит оценку коэффициента при независимой переменной.
\item  Используя результаты из пункта (d),  определите предельный эффект от роста количества хронических заболеваний тремя путями: средний по выборке,  оцененный с помощью среднего по выборке переменной NDISEASE,  оцененный при $\Lambda (x'\beta) = \overline{y}.$
\item  Для логит-модели рассчитайте во сколько раз изменится отношения шансов,  когда изменяется переменная NDISEASE.
\end{enumerate}

\item [$14 - 6$] Продолжим анализ ситуации из упражнения 14.5.
\begin{enumerate}
\item  Сравните три модели бинарного выбора на основе статистической значимости независимой переменной NDISEASE.
\item  Сравните три модели бинарного выбора на основе рассчитанного предельного эффекта.
\item  Сравните три модели бинарного выбора на основе спрогнозированных вероятностей.
\item  Сравните логит и  пробит-модели на основе логарифмической функции правдоподобия.
\end{enumerate}
\end{enumerate}




\chapter{Мультиномиальные модели}

\section{Введение}

В предыдущей главе рассматривались модели дискретного выбора, зависимая переменная в которых принимает только одно из двух возможных значений. В настоящей главе мы будем рассматривать случаи с несколькими возможными значениями, обычно взаимно исключающими. К примерам таких ситуаций можно отнести выбор способа совершения ежедневной поездки на работу (на автобусе, на машине или пешком), различные типы медицинского страхования (раздельная оплата услуг, управляемое медицинское обслуживание или отсутствие страхования), различные статусы экономически активного населения (занятый на полный рабочий день, на неполный рабочий день или безработный), выбор места отдыха, выбор профессии или выбор товара.

Статистическая модель в определенном смысле довольно проста --- данные имеют мультиномиальное распределение, также как бинарные данные должны иметь биномиальное или распределение Бернулли. Оценка параметров модели наиболее часто производится методом максимального правдоподобия, поскольку очевидно, что данные распределены мультиномиально. Однако, при возникновении некоторых осложнений, вместо него используется метод моментов.

Различные мультиномиальные модели возникают вследствие использования различных функциональных форм для определения вероятности наступления события, имеющего мультиномиальное распределение, аналогично разнице между пробит и логит-моделями в случае бинарных данных. Разграничение также проводится между моделями, для которых регрессоры варьируются в зависимости от альтернатив для конкретного индивида, и моделями, в которых регрессоры не зависят альтернатив. Например, при рассмотрении проблемы выбора способа транспортировки некоторые независимые переменные, такие как время в пути и затраты, будут зависеть от сделанного выбора, тогда как другие, например, возраст индивида, не зависят от выбранной альтернативы.

Простейшая мультиномиальная модель, условная или мультиномиальная логит-модель, достаточна проста в использовании, однако, имеет слишком много ограничений на практике, особенно, если данные о множественных исходах получены путем индивидуального выбора. Для неупорядоченных исходов могут быть использованы менее жесткие модели на базе модели случайной полезности. 
В этой модели выбор делается в пользу альтернативы, которая принесет большую полезность, а полезность каждой альтернативы оценивается как сумма детерминированной и случайной составляющей. Различные определения случайной составляющей приводят к  различным функциональным формам для вероятностей выбора, а, следовательно, к различным мультиномиальным моделям. Дополнительные модели возникают в приложениях, если какие-то предположения могут быть выдвинуты по поводу процесса принятия решения, такие как естественная упорядоченность альтернатив или последовательность принятия решений. На практике используется широкий круг мультиномиальных моделей.

В разделе 15.2 представлен пример мультиномиальных данных, на котором будут продемонстрированы обсуждаемые в настоящей главе идеи. Общие выводы для мультиномиальных моделей приведены в разделе 15.3. Условная и мультиномиальная логит-модель рассматриваются в разделе 15.4. Аддитивная модель случайной полезности представлена в разделе 15.5. Вложенная логит-модель, логит-модель со случайными параметрамии, мультиномиальная пробит-модель являются предметом рассмотрения в разделах 15.6-15.8. Модели с упорядоченными исходами и модели с последовательными решениями подробно рассматриваются в разделе 15.9. Многомерные модели с более чем одной зависимой переменной представлены в разделе 15.10. Полупараметрические методы оценки кратко рассматриваются в разделе 15.11.

\section{Пример: выбор способа рыбалки}

В настоящем разделе проиллюстрирована мультиномиальная логит-модель, простейшая мультиномиальная модель с неупорядоченными исходами. Ее варианты представлены в разделе 15.4., где рассматриваются регрессоры, изменяющиеся между альтернативами. Акцент делается на интерпретацию оцениваемой модели. Предельный эффект от изменения значения независимой переменной является гораздо более сложным, чем простое влияние на условное среднее. Вместо этого, для мультиномиальных данных, мы изучаем предельный эффект на вероятность каждого исхода. Отсюда сумма предельных эффектов равна нулю, поскольку сумма вероятностей равна единице.

Рассмотрим проблему выбора способа рыбалки. Зависимая переменная $y$ принимает значение 1, 2, 3 или 4, в зависимости от того, какая из взаимно исключающих альтернатив, способов рыбалки -- рыбалка с пляжа, с пристани, с собственной или взятой на прокат лодки, соответственно, будет выбрана. Неупорядоченная мультиномиальная модель, такая как мультиномиальная логит-модель, здесь применима, поскольку отсутствует четкое упорядочивание значений зависимой переменной. В качестве независимых переменных примем индивидуальный доход, который не зависит от выбранного способа рыбалки, а также цену и коэффициент вылова, которые зависят от выбранного способа рыбалки и индивида.

Пример выборки в 1182 человека взят из исследования, проведенного Томсоном и Круком (1991)  и проанализированного Херриджесом и Клингом (1999). Описание данных приведено в таблице 15.1, содержащей средние по подвыборкам людей, которые предпочли каждый из способов, а также средние значения для регрессоров по выборке в целом. 

\subsection{Условная логит-модель: регрессоры, значения которых зависят от альтернатив}

Сначала рассмотрим роль цены и коэффициента вылова, регрессоров, значения которых зависят от альтернатив. Исключением здесь является цены на рыбалку с пляжа и на рыбалку с пристани, которые совпадают.

Просмотрев таблицу 15.1, наблюдаем, что люди стремятся рыбачить там, где это дешевле всего делать. Например, для людей, выбравших рыбалку с пляжа, средняя цена составила \$36 в сравнении со средними ценами в \$36, \$98 и \$125 для других способов. В более общем смысле, для людей, выбиравших рыбалку с пляжа или с пристани, эти способы в среднем оказывались намного дешевле, чем рыбалка с лодки, а для людей, выбиравших рыбалку с лодки, эти способы оказывались намного дешевле, чем рыбалка с пляжа или пристани. Зависимость между выбором способа рыбалки и коэффициентом вылова менее ясна, тем не менее, очевидно, наибольший коэффициент вылова у рыбалки с лодки, взятой в аренду.

\textbf{Таблица 15.1. }Множественный выбор способа рыбалки: результаты

\begin{tabular}{|p{1.3in}|p{0.5in}|p{0.5in}|p{0.7in}|p{0.8in}|p{0.7in}|} \hline 
\textbf{Независимые переменные} & \multicolumn{4}{|p{2.5in}|}{\textbf{Средние по подвыборкам}} & \textbf{Средние по всей выборке по всем }${\mathbf y}$  \\ \hline 
 & ${\mathbf y}{\mathbf =}{\mathbf 1}$\textbf{\newline Рыбалка с пляжа} & ${\mathbf y}{\mathbf =}{\mathbf 2}$\textbf{\newline Рыбалка с пристани} & ${\mathbf y}{\mathbf =}{\mathbf 3}$\textbf{\newline Рыбалка с собственной лодки} & ${\mathbf y}{\mathbf =}{\mathbf 4}$\textbf{\newline Рыбалка с арендованной лодки} &   \\ \hline 
Доход (в \$1000 в месяц) & 4,052 & 3,387 & 4,654 & 3,881 & 4,099 \\ \hline 
Цена рыбалки с пляжа (\$) & 36 & 31 & 138 & 121 & 103 \\ \hline 
Цена рыбалки с пристани (\$) & 36 & 31 & 138 & 121 & 103 \\ \hline 
Цена рыбалки с собственной лодки (\$) & 98 & 82 & 42 & 45 & 55 \\ \hline 
Цена рыбалки с арендованной лодки (\$) & 125 & 110 & 71 & 75 & 84 \\ \hline 
Коэффициент вылова при рыбалке с пляжа & 0,28 & 0,26 & 0,21 & 0,25 & 0,24 \\ \hline 
Коэффициент вылова при рыбалке с пристани & 0,22 & 0,20 & 0,13 & 0,16 & 0,16 \\ \hline 
Коэффициент вылова при рыбалке с собственной лодки & 0,16 & 0,15 & 0,18 & 0,18 & 0,17 \\ \hline 
Коэффициент вылова при рыбалке с арендованной лодки & 0,52 & 0,50 & 0,65 & 0,69 & 0,63 \\ \hline 
Вероятность наступления события & 0,113 & 0,151 & 0,354 & 0,382 & 1.000 \\ \hline 
Количество наблюдений & 134 & 178 & 418 & 452 & 1182 \\ \hline 
\end{tabular}



Для регрессоров, значения которых зависят от альтернатив, таких как цена и коэффициент вылова, мультиномиальная логит-модель называется условной логит-моделью (см. параграф 15.4.1). Вероятность того, что $i$-ый индивид выберет $j$-ый способ рыбалки определяется по формуле

\[p_{ij}={\Pr  \left[y_i=j\right]\ }=
\frac{{\rm exp} ({\beta }_PP_{ij}+{\beta }_CC_{ij})}{\sum^4_{k=1}{{\rm exp} ({\beta }_PP_{ik}+{\beta }_CC_{ik})}},\ \ \ \ j=1,\dots ,4,\] 

где $P$ обозначает цену, $C$ -- коэффициент вылова, индекс $i$ обозначает $i$-го индивида, а коэффициенты $j$ или $k$ -- альтернативу. Эта модель является очевидным обобщением бинарной логит-модели и дает значения вероятностей, которые лежат между 0 и 1, а в сумме дают единицу. В других мультиномиальных моделях используются другие функциональные формы для определения $p_{ij}$.

Оценки коэффициентов приведены в колонке CL таблицы 15.2. Для условной логит-модели, тем не менее, в отличие от некоторых других мультиномиальных моделей, знаки коэффициентов непосредственно интерпретируются. Предвосхищая выводы из раздела 15.4.3., поскольку ${\beta }_P< 0$,  возрастание цены на одну альтернативу приводит к снижению вероятности выбора этой альтернативы и возрастанию вероятности выбора других альтернатив. Аналогично, поскольку ${\beta }_C>0$, возрастание коэффициента вылова одной альтернативы приводит к росту вероятности ее выбора и к снижению вероятности выбора других альтернатив.


%%% here 

Стандартным показателем влияния регрессора является $N^{-1}\sum^N_{i=1}{{\partial p_{ij}}/{\partial x_{ikr}}}$, среднее предельное изменение вероятности выбора альтернативы $j$, когда $r$-ый регрессор увеличивается на одну единицу для альтернативы $k$ и остается неизменным для остальных альтернатив. Для условной логит-модели этот показатель оценивается как $N^{-1}\sum^N_{i=1}{{\hat{p}}_{ij}\left({\delta }_{ijk}-{\hat{p}}_{ik}\right){\widehat{\beta }}_r}$ (см. (15.18)), где $\widehat{\beta }$ --- оценка параметра $\beta $, а ${\hat{p}}_{ij},\ j=1,\dots ,m,$ --- прогнозируемые вероятности исходов.

Средние изменение вероятности для четырех способов рыбалки для изменения двух регрессоров, цены и коэффициента вылова, приведены в таблице 15.3. В таблице приведены предельный эффект на вероятность выбора при изменении цены на 100 единиц (или на \$100) и предельный эффект при изменении  коэффициента вылова на одну единицу. Например, рост цены на рыбалку с пляжа на \$100 приводит к снижению вероятности выбора этого способа на 0,272 и увеличению вероятности выбора на 0,119, 0,080 и 0,068 соответственно рыбалки с пристани, с собственной лодки и с арендованной лодки. Обратите внимание, что, как и ожидалось, сумма изменений вероятностей равна нулю.

\textbf{Таблица 15.2. }Выбор способа рыбалки: оценка мультиномиальной логит-модели*

\begin{tabular}{|p{0.7in}|p{0.7in}|p{1.7in}|p{0.5in}|p{0.5in}|p{0.5in}|} \hline 
\textbf{Регрессор} & \textbf{Тип} & \textbf{Коэффициент} & \multicolumn{3}{|p{1.4in}|}{\textbf{Модель}} \\ \hline 
  &   &   & \textbf{CL} & \textbf{MNL} & \textbf{Mixed} \\ \hline 
Цена (P) & Зависимый & ${\beta }_P$ & -0,021 & - & -0,025 \\ \hline 
Коэффициент вылова (C) & Зависимый & ${\beta }_{CR}$ & 0,953 & - & 0,358 \\ \hline 
Константа  & Независимый & ${\alpha }_1$: рыбалка с пляжа & - & 0,0 & 0,0 \\ \hline 
 &  & ${\alpha }_2$: рыбалка с пристани & - & 0,814 & 0,778 \\ \hline 
 &  & ${\alpha }_3$: рыбалка с собственной лодки & - & 0,739 & 0,527 \\ \hline 
 &  & ${\alpha }_4$: рыбалка с арендованной лодки & - & 1,341 & 1,694 \\ \hline 
Доход (I) & Независимый & ${\beta }_{I1}:$рыбалка с пляжа & - & 0,0 & 0,0 \\ \hline 
 &  & ${\beta }_{I2}:$рыбалка с пристани & - & -0,143 & -0,128 \\ \hline 
 &  & ${\beta }_{I3}:$рыбалка с собственной лодки & - & 0,092 & 0,089 \\ \hline 
 &  & ${\beta }_{I4}:$рыбалка с арендованной лодки & - & -0,032 & -0,033 \\ \hline 
$-lnL$ &  &  & -1311 & -1477 & -1215 \\ \hline 
Псевдо-$R^2$ &  &  & 0,162 & 0,099 & 0,258 \\ \hline 
\end{tabular}

*Регрессор может быть двух типов: зависимый от исхода или независимый от исхода. Исходы: (1) рыбалка с пляжа, (2) с пристани, (3) с собственной лодки, (4) с арендованной лодки. Оценки параметров методом максимального правдоподобия приведены для условной логит-модели (столбец CL, conditional logit), для мультиномиальной логит-модели (столбец MNL, multinomial logit) и для смешанной логит-модели (столбец Mixed). Мультиномиальная и смешанная логит-модели нормализованы по базовой категории, в качестве которой выбрана рыбалка с пляжа. Все оценки, кроме ${\beta }_{I4}$, статистически значимы на уровне значимости 5\%.

Расчет этих предельных эффектов и вероятностей требует дополнительных вычислений после оценки модели. В  условной логит-модели используется простая формула ${\overline{p}}_j\left({\delta }_{jk}-{\overline{p}}_k\right){\widehat{\beta }}_r$, где ${\overline{p}}_j$ --- средняя по выборке вероятность. Для эффекта от изменения цены на рыбалку с пляжа на вероятность выбора этого способа получим $100\times 0,113\left(1-0,113\right)\times \left(-0,021\right)=-0,21$ по сравнению со средним по выборке значением -0,272 в таблице. Это приближение становится менее разумным при приближении значений вероятностей к 0 или 1.

Результаты в таблице 15.3 являются соответствуют идеи, что наибольшая взаимозаменяемость наблюдается между рыбалкой с пристани и с пляжа, а также между рыбалкой с собственной и с арендованной лодок. А именно рост цены или снижение коэффициента вылова для рыбалки с пристани приводит к росту предпочтительности рыбалки с пляжа, и наоборот. Аналогичные выводы сохраняются и для соотношения между рыбалкой с арендованной и с собственной лодки.

\textbf{Таблица 15.3. }Выбор способа рыбалки: предельные эффекты для условной логит-модели*

\begin{tabular}{|p{0.7in}|p{0.4in}|p{0.5in}|p{0.5in}|p{0.6in}|p{0.4in}|p{0.5in}|p{0.5in}|p{0.6in}|} \hline 
 & \multicolumn{4}{|p{2.0in}|}{\textbf{Изменение цены на \$100}} & \multicolumn{4}{|p{2.0in}|}{\textbf{Изменений коэффициента вылова на единицу}} \\ \hline 
 & \textbf{с пляжа} & \textbf{с пристани} & \textbf{с собст-венной лодки} & \textbf{с арендо-\newline ванной лодки} & \textbf{с пляжа} & \textbf{с пристани} & \textbf{с собст-венной лодки} & \textbf{с арендо-ванной лодки} \\ \hline 
Изменение Pr[с пляжа] & -0,272 & 0.119 & 0,085 & 0,068 & 0,126 & -0,055 & -0,040 & -0,032 \\ \hline 
Изменение Pr[с пристани] & 0.119 & -0,263 & 0,080 & 0,064 & -0,055 & 0,122 & -0,037 & -0,030 \\ \hline 
Изменение Pr[с собственной лодки] & 0,080 & 0,080 & -0,391 & 0,225 & -0,040 & -0,037 & 0,182 & -0,105 \\ \hline 
Изменение Pr[с арендованной лодки] & 0,068 & 0,064 & 0,225 & -0,357 & -0,032 & -0,030 & -0,105 & 0,166 \\ \hline 
\end{tabular}

* Средняя предельное изменение вероятности выбора каждой из альтернатив, когда значение регрессора изменяется для одной из альтернатив и остается неизменным для остальных.

Эти изменения в вероятности выбора рассчитаны для очень больших изменений значений регрессоров. В приведенном примере средняя цена по выборке составила \$86, а средний коэффициент вылова 0,30. В некоторых случаях вместо предельных эффектов рассчитываются эластичности. Однако эластичности для вероятностей выбора следует использовать очень аккуратно, поскольку вероятности ограничены значениями от 0 до 1. Изменение в прогнозируемой вероятности с 0,01 до 0,02 приведет к значению эластичности приблизительно в 50 раз больше, чем изменение прогнозируемой вероятности с 0,50 до 0,51. 

\subsection{Мультиномиальная логит-модель: независимые от альтернатив регрессоры}

Теперь рассмотрим влияние дохода, рассчитанного как месячный доход в тысячах долларов. Из таблицы 15.1 видно, что если доход растет, то выбор способа рыбалки постепенно смещается с рыбалки с пристани, для которой средний месячный доход человека составляет \$3387, к рыбалке с арендуемой лодки, затем к рыбалке с пляжа и, наконец, к рыбалке с собственной лодки, для которой среднемесячный доход равняется \$4654.

Поскольку доход не зависит от выбранной альтернативы, подходящей является мультиномиальная логит-модель (приведенная в параграфе 15.4.1). Эта модель позволяет коэффициентам при независимых переменных изменяться в зависимости от альтернативы, а вероятность определяется как

\[p_{ij}={\Pr  \left[y_i=j\right]\ }=
\frac{{\rm exp}({\alpha }_j+{\beta }_{Ij}I_i)}
{\sum^N_{k=1}{{\rm exp}({\alpha }_k+{\beta }_{Ik}I_i)}},
\ \ \ \ \ \ \ j=1,\dots ,4,\] 

где $I$ обозначает доход. Нормализация параметров является необходимой, поскольку сумма вероятностей равна единице. Обычно выбирают ${\alpha }_1=0$ и ${\beta }_{I1}=0.$

Оценки параметров приведены в столбце MNL таблицы 15.2. Интерпретация коэффициентов осложняется по сравнению с условной логит-моделью. В частности, для мультиномиальных логит-моделей положительное значение параметра регрессии вовсе не означает, что увеличение значения регрессора приведет к увеличению вероятности выбор некоторой альтернативы. Вместо этого, интерпретация мультиномиальной логит-модели возможна относительно базовой категориальной группы, в нашем примере это рыбалка с пляжа, поскольку коэффициенты для этого исхода были нормированы к нулю. В сравнении с рыбалкой с пляжа более высокий доход ведет к снижению относительной вероятности выбора рыбалки с пристани (поскольку ${\beta }_{I2}=-0,143<0)$ или с арендованной лодки (поскольку ${\beta }_{I4}=-0,032$) и увеличению относительной вероятности выбора рыбалки с собственной лодки (поскольку ${\beta }_{I3}=0,092$).

Величина реакции на изменение дохода может быть измерена с помощью показателя $N^{-1}\sum^N_{i=1}{{\partial p_{ij}}/{\partial I_i}}$, предельного эффекта, усредненного по индивидам. Для мультиномиальных логит-моделей этот показатель оценивается с помощью $N^{-1}\sum^N_{i=1}{{\hat{p}}_{ij}(\widehat{{\beta }_j}-{\widehat{\overline{\beta }}}_i)}$ (см. (15.19)), где $\widehat{{\beta }_j}$ -- оценка параметра ${\beta }_j$, ${\overline{\beta }}_i=\sum^m_{l=1}{p_{il}{\beta }_l}$ средняя, взвешенная с помощью вероятностей, ${\beta }_l$, а ${\hat{p}}_{ij},\ j=1,\dots ,m,$ - прогнозируемые вероятности. Для четырех исходов увеличение месячного дохода на \$1000 приведет к изменению на 0,000, -0,021, 0,033 и -0,012 в вероятностях выбора соответственно рыбалки с пляжа, с пристани, с собственной лодки и с арендованной лодки. Это показывает незначительные изменения вероятности рыбалки с пляжа, отказ от рыбалки с пристани или арендуемой лодки, и рост предпочтительности рыбаки с собственной лодки. Поскольку среднемесячный доход составляет \$4100, изменения вероятности существенны.

Тем не менее, доход сам по себе не лучший показатель выбора способа рыбалки. В последних строках таблицы 15.2 видно, что значение логарифмической функции правдоподобия и псевдо-$R^2$ для мультиномиальной логит-модели значительно меньше значений для условной логит-модели. Дополнильные расчеты показывают, что для всех индивидов из выборки вероятности, прогнозируемые с помощью мультиномиальной логит, изменяются от 0,095 до 0,115 для рыбалки с пляжа, от 0,036 до 0,234 для рыбалки с пристани, от 0,240 до 0,626 для рыбалки с собственной лодки и от 0,244 до 0,416 для рыбалки с арендованной лодки. Поскольку в модель включена константа, средние значения для этих прогнозируемых вероятностей для каждого исхода равны средним вероятностям по выборке. Этот вывод для мультиномиальной логит-модели --- следствие из \eqref{GrindEQ__15_16_}.

\subsection{Смешанная логит-модель}

Более богатой получается модель, которая комбинирует две ранее описанные модели. Это  достигается при использовании так называемой смешанной логит-модели (см. параграф 15.4.1), для которой

\[{\Pr  \left[y_i=j\right]\ }=
\frac{{\rm exp}({\beta }_PP_{ij}+{\beta }_CC_{ij}+{\alpha }_j+{\beta }_{Ij}I_i)}
{\sum^4_{k=1}{{\rm exp}({\beta }_PP_{ik}+{\beta }_CC_{ik}+{\alpha }_k+{\beta }_{Ij}I_k)}}.\] 

Эта модель, не спутайте с моделью из раздела 15.7, которую также называют смешанной моделью, может быть рассмотрена как условная логит-модель, для которой

\[{\Pr  \left[y_i=j\right]\ }=
\frac{{\rm exp}({\beta }_PP_{ij}+{\beta }_CC_{ij}+\sum^4_{l=1}{({\alpha }_ld_{ijl}+{\beta }_{Il}dI_{ijl})})}
{\sum^4_{k=1}{{\rm exp}({\beta }_PP_{ik}+{\beta }_CC_{ik}+\sum^4_{l=1}{({\alpha }_ld_{ijl}+{\beta }_{Il}dI_{ijl})})}},\] 

где $d_{ijl}$ --- фиктивная переменная равная единице, если $j=l$, и нулю в обратном случае, а $dI_{ijl}=d_{ijl}I_i$ равняется доходу, если $j=l$, и нулю в обратном случае. В этом случае, строим регрессию $y_i$ от восьми регрессоров: $P_{ij},\ C_{ij},\ d_{ij2},\ d_{ij3},d_{ij4},dI_{ij2},\ dI_{ij3}$ и $dI_{ij4}.$ Поскольку ${\alpha }_1=1\ $ и ${\beta }_{l1}=0$, независимыми переменными $d_{ij1}$ и $dI_{ij1}$ опущены. 
Обратите внимание, что если мы оценим эту условную логит-модель взяв регрессорами только лишь с $d_{ijl}$ и $dI_{ijl}$, оценки условной логит-модели совпадут с оценками мультиномиальной логит-модели, приведенными ранее. Мультиномиальная логит-модель всегда может быть оценена как условная логит-модель (см. параграф 15.3.4).

Хотя смешанная логит-модель дает более содержательную оценку, чем условная логит-модель, у условной логит-модели есть следующее преимущество. Если дополнительная альтернатива будет предложена к выбору, тогда можно предсказать вероятность ее выбора, поскольку параметры условной логит-модели не изменяются в зависимости от альтернатив.

Результаты для смешанной модели приведены в последней колонке таблицы 15.2. По сравнению с первыми двумя моделями коэффициенты модели изменились не значительно, за исключением существенного изменения оценки параметра для коэффициента вылова. Это изменение обусловлено включением зависимых от альтернатив фиктивных переменных, а не переменной дохода. Смешанная логит-модель гораздо предпочтительней других моделей, поскольку имеет гораздо более высокое значение логарифмической функции правдоподобия и лучшие результаты формальных статистических тестов.

\section{ Общие выводы}

Результаты, полученные в настоящем разделе, относятся ко всем мультиномиальным моделям. Внимание в настоящей главе обращено к различным спецификациям мультиномиальной модели, используемым на практике.

\subsection{Мультиномиальные модели}

Пусть есть $m$ альтернатив, а зависимая переменная $y$ определена так, что принимает значение $j$, если выбрана $j$-ая альтернатива, $j=1,\dots ,m.$ (Некоторые авторы вместо такой постановки полагают $m+1$ альтернатив, где $j=0,\ 1,\dots ,m.$) Определим вероятность того, что выбрана альтернатива $j$

\begin{equation} 
\label{GrindEQ__15_1_} 
p_j={\Pr  \left[y=j\right]\ },\ \ \ \ \ \ \ \ \ j=1,\dots ,m. 
\end{equation} 

Введем $m$ бинарных переменных для каждого наблюдения $y$,

\begin{equation} \label{GrindEQ__15_2_} 
y_j=\left\{ \begin{array}{c}
1,\text{ если } y=j, \\ 
0,\text{ если } y\ne j. 
\end{array}
\right. 
\end{equation} 

Таким образом, $y_j$ равняется единице, если альтернатива $j$ -- наблюдаемый исход, а остальные исходы $y_k$ равняются нулю, то есть для каждого наблюдения $y$ ровно один исход из $y_1,\ y_2,\ \dots ,\ y_m$ будет иметь ненулевое значение. Тогда \textbf{функция плотности мультиномиального распределения } может быть записана как

\begin{equation} \label{GrindEQ__15_3_} f\left(y\right)=p^{y_1}_1\times \dots \times p^{y_m}_m=\prod^m_{j=1}{p^{y_j}_j}. \end{equation} 

Для регрессионных моделей введем индекс $i$ для $i$-го индивида и независимую переменную $x_i$. Запишем модель вероятности того, что индивид $i$ выберет $j$-ую альтернативу,

\begin{equation} \label{GrindEQ__15_4_} p_{ij}={\Pr  \left[y_i=j\right]\ }=F_j\left(x_i,\beta \right),\ \ \ \ \ j=1,\dots ,m,\ \ \ \ \ i=1,\dots ,N. \end{equation} 

Функциональная форма $F_j$ должна быть такой, что вероятности попадают в промежуток от 0 до 1, а сумма по $j$ равняется единице. Различные функциональные формы $F_j$ соответствуют определённым моделям, в частности мультиномиальной логит-модели, вложенной логит-модели, мультиномиальной пробит-модели, модели с упорядоченными исходами, модели с последовательными решениями и модели многомерного выбора. Эти модели описываются в последующих разделах.

\subsection{Оценка методом максимального правдоподобия}

Плотность мультиномиального распределения для одного наблюдения приведена в \eqref{GrindEQ__15_3_}. Функция правдоподобия для выборки из $N$ независимых наблюдений тогда принимает вид $L_N=\prod^N_{i=1}{\prod^m_{j=1}{p^{y_{ij}}_{ij}}},$ где индекс $i$ обозначает $i$-го из $N$ индивидов, а индекс $j$ -- $j$-ую из $m$ альтернатив. \textbf{Логарифмическая функция правдоподобия} принимает вид

\begin{equation} \label{GrindEQ__15_5_} {\mathcal L}={\ln  L_N=\sum^N_{i=1}{\sum^m_{j=1}{y_{ij}{\ln  p_{ij}\ }}}\ }, 
\end{equation} 

где $p_{ij}=F_j(x_i,\beta )$ -- функция параметра $\beta $ и независимой переменной, определенная в выражении \eqref{GrindEQ__15_4_}. В более общем случае, количество альтернатив может варьироваться в зависимости от индивида, тогда вместо $m$ альтернатив появится $m_i$ альтернатив.

Оценка параметра $\widehat{\beta }$ будет решением  условий первого порядка максимизации функции правдоподобия

\begin{equation} 
\label{GrindEQ__15_6_} 
\frac{\partial {\mathcal L}}{\partial\beta }=\sum^N_{i=1}{\sum^m_{j=1}{\frac{y_{ij}}{p_{ij}}}}\frac{\partial p_{ij}}{\partial\beta }=0,\  
\end{equation} 

которое обычно нелинейно по $\beta $. Распределение переменной $y_i$ с необходимостью является мультиномиальным, поэтому правильная спецификация процесса порождающего данные означает правильную функциональную форму $F_j(x_i,\beta )$ для вероятностей $p_{ij}$. Это обеспечивает состоятельность оценки, поскольку тогда $\E\left[y_{ij}\right]=p_{ij},$ и, взяв математическое ожидание из \eqref{GrindEQ__15_6_}, получаем выражение $\E\left[{\partial {\mathcal L}}/{\partial \beta }\right]=\sum^N_{i=1}{\sum^m_{j=1}{{\partial p_{ij}}/{\partial \beta }}},$ которое равняется нулю, поскольку $\sum^m_{j=0}{p_{ij}}=1.$

Применяется стандартная асимптотическая теория, а ковариационная матрица равняется обращенной и взятой с минусом информационной матрице. Продифференцировав двойную сумму из выражения \eqref{GrindEQ__15_6_} по $\beta '$ и используя $\E\left[y_{ij}\right]=p_{ij}$, получим следующее упрощенное выражение

\begin{equation} 
\label{GrindEQ__15_7_} 
\widehat{\beta }\overset{a}{\sim }{\mathcal N}
\left[{\beta }_0,
{\left({\left.\sum^N_{i=1}{\sum^m_{j=1}{\frac{1}{p_{ij}}\frac{\partial p_{ij}}{\partial \beta }
\frac{\partial p_{ij}}{\partial {\beta }'}-
\frac{{\partial }^2p_{ij}}{\partial \beta \partial{\beta }'}}}\right|}_{{\beta }_0}\right)}^{-1}
\right]. 
\end{equation} 

Если наблюдения независимы по $i$, тогда нет необходимости использовать более общую форму ковариационной матрицы, поскольку данные однозначно имеют мультиномиальное распределение и равенство информационной матрицы будет сохранено.

Как уже отмечалось, различным моделям соответствуют различные формы $F_j(x_i,\beta )$ для $p_{ij}$, а, следовательно, различные выражения для \eqref{GrindEQ__15_6_} и \eqref{GrindEQ__15_7_}.

Оценивание методом максимального правдоподобия для выборок, основанных на отборе, например, в которых увеличено количество редко наблюдаемых исходов, рассматривается в разделах 14.5. и 24.4.

\subsection{ Оценка методом моментов}

Для простых случаев пространственных данных стандартным методом оценки является метод максимального правдоподобия.

Однако когда возникают осложнения, такие как эндогенность или корреляция между наблюдениями  возрастает, будет более удобно использовать \textbf{метод моментов} для оценки модели. Предполагая, что вероятности определены верно, можем рассмотреть любой метод оценивания с оценивающим уравнением

\begin{equation} 
\label{GrindEQ__15_8_} 
\sum^N_{i=1}{\sum^m_{j=1}{\left(y_{ij}-p_{ij}\right)z_i=0,}} 
\end{equation} 

где $z_i$ --- вектор той же размерности, что и $\beta $, независимый от $y_{ij},$ например, $z_i={\partial p_{ij}}/{\partial \beta }.$ Эта оценка будет состоятельной, если функциональная форма для $p_{ij}$ задана правильно, поскольку тогда $\E\left[y_{ij}\right]=p_{ij}$, а математической ожидание левой часть выражения \eqref{GrindEQ__15_8_}  равно нулю. Эффективность метода будет зависеть от выбора $z_i$, и в наиболее общем случае может быть использован обобщенный метод моментов. Выражение для оценки \eqref{GrindEQ__15_8_} также является основой для симуляционного метода моментов для мультиномиальной пробит-модели (см. параграф 15.8.2).

\subsection{Зависимые от выбора альтернативы регрессоры}

Мультиномиальные модели различаются не только по принципу выбора функции $F_j(\cdot )$ в выражении \eqref{GrindEQ__15_4_}, но также и по тому, как  независимые переменные и параметры изменяются в зависимости от выбранных альтернатив.

Одна крайность представляет собой случай, когда все регрессоры являются \textbf{зависимыми от альтернатив.} Это означает, что все независимые переменные принимают разные значения при выборе различных альтернатив. Пусть $x_{ij}$ обозначает значение регрессора для индивида $i$ и альтернативы $j$, а также пусть $x_i={\left[x'_{i1}x'_{i2}\ \dots \ \ x'_{im}\right]}'.$ Тогда \eqref{GrindEQ__15_4_} обычно принимает форму

\[F_j\left(x_i,\beta \right)=F_j\left(x'_{i1}\beta ,\dots ,x'_{im}\beta \right),\] 

где параметр $\beta $ постоянен для всех альтернатив. Примером может служить условная логит-модель, описанная далее в \eqref{GrindEQ__15_10_}.

Другая крайность представляет собой случай \textbf{независимости регрессоров от альтернатив}, т.е. $x_i$ принимает постоянные значения для всех альтернатив. Примером могут послужить социально-экономические характеристики индивида в модели выбора способа транспортировки. Тогда выражение \eqref{GrindEQ__15_4_} обычно принимает форму

\[F_j\left(x_i,\beta \right)=F_j\left(x'_i\beta ,\dots ,x'_i\beta \right),\] 

где параметр ${\beta }_j$ изменяется в зависимости от альтернативы, и $\beta ={\left[{\beta }'_1{\beta }'_2\dots {\beta }'_m\right]}'.$ Идентификация параметров требует нормализации, такой как ${\beta }_1=0.$ Примером может послужить мультиномиальная  логит-модель, описанная далее в \eqref{GrindEQ__15_11_}.

Разграничение регрессоров на зависимые и независимые от альтернатив имеет практическое значение, поскольку стандартные обозначения и компьютерные программы для мультиномиальных моделей работают исключительно либо с теми, либо с другими. На практике, конечно, какие-то регрессоры могут быть зависимыми от альтернатив, а какие-то независимыми. В таком случае лучшим решением будет использование программы, созданной для зависимых от альтернатив регрессоров, поскольку тогда возможно перейти от постоянных для альтернатив регрессоров к формату зависимых от альтернатив регрессоров. Допустим $x_i$ --- вектор размера $K\times 1$. Тогда определим $x_{ij}$ как вектор размера $Km\times 1$, все элементы которого нули, кроме $j$-го блока, принимающего значение $x_i$. Тогда 

\[x_{ij}={\left[0'\dots 0' \, x'_i \, 0'\dots 0'\right]}',\] 

а также определим $\beta ={\left[0'{\beta }'_2\dots {\beta }'_m\right]}'$, где ${\beta }_1=0$ для нормализации. Тогда $x'_i{\beta }_j=x'_{ij}\beta .$ Независимые переменные по существу включены как взаимодействия с зависимыми от альтернатив фиктивными переменными. Пример приведен в параграфе 15.2.3. Также возможно перейти от зависимых от альтернатив регрессоров к формату независимых от альтернатив регрессоров, но тогда необходимо ввести $(m-1)$ выражений-ограничений на параметры модели для каждого зависимого от альтернативы регрессора.

\subsection{Данные о выявленных и заявленных предпочтениях}

Мультиномиальные данные, используемые в микроэкономических исследованиях, обычно возникают при изучении проблемы индивидуального потребительского выбора. Данные о выборе потребителя могут быть \textbf{данными о выявленных предпочтениях, } основанными на действительных решениях и выборе, или \textbf{данными о заявленных предпочтениях, }которые являются результатом ответа на вопросы о гипотетических ситуациях. Примеров выявленных предпочтений будет действительный выбор некоторой профессии. Примером заявленных предпочтений будут результаты маркетингового исследования по поводу выбора топливосберегающего транспортного средства, в котором респондентам предложено выбрать между гипотетическими транспортными средствами, которые различаются по своим характеристикам, таким как потребление топлива, комплектация и цена.

Данные о выявленных предпочтениях часто содержат очень мало или не содержат вообще данных об альтернативах, которые не были выбраны. Например, мы можем знать цену выбранного отдельным потребителем продукта, но не будем знать цены на альтернативные продукты. Привлекательность данных о заявленных предпочтениях для мультиномиальных моделей заключается в том, что доступны данные по ключевым переменными, таким как цена, для всех возможных альтернатив. Это в частности является большим преимуществом, если необходимо спрогнозировать вероятность выбора или доли рынка новой альтернативы на основе ее характеристик, поскольку все параметры могут быть постоянными для альтернатив, а значения регрессоров изменяться в зависимости от альтернатив.

Существую споры по поводу использования данных о заявленных предпочтениях, поскольку результаты могут зависеть от формулировки вопросов. Более того, люди могут преувеличивать или преуменьшать свою готовность платить в поддержку конкретной политики. Например, некоторые могут преувеличивать свои запросы, чтобы поддержать политику бережного отношения к окружающей среде. 

\textbf{Данные о сканированных штрих кодах} в магазине являются особенно привлекательными, потому что они дают доступ к данным о выявленных предпочтениях, тогда как данные о ценах на все альтернативы также доступны.

\subsection{Оценивание и выбор модели}

Параметры регрессии в мультиномиальных моделях могут трудно поддаваться интерпретации. Несмотря на это, полезно рассмотреть предельный эффект (или эластичности) от изменения значения регрессора на вероятность исхода. Формулы для условной и мультиномиальной логит-моделей приведены в параграфе 15.4.3 и используются в разделе 15.2.

Некоторые методы оценивания качества модели приведены в работах Амэмии (1981) и Маддалы (1983). Меры похожие на $R^2$, т.е. основанные на аналоге квадратов остатков, работают в этом случае плохо. Сравнение прогнозируемых вероятностей с реальными исходами имеет ограниченное применение, поскольку в мультиномиальных моделях с константой обязательно средняя прогнозируемая вероятность равняется той же средней выборочной вероятности для каждой альтернативы. Было бы полезно взглянуть на диапазон вероятностей для каждой альтернативы внутри выборки. Чем шире диапазон, тем более содержательный характер имеет модель. Для более подробной информации обратитесь к обсуждению этого вопроса в параграфе 14.3.7 для случая бинарных данных.

Обычно оценка параметров мультиномиальных моделей происходит методом максимального правдоподобия. Тогда, при условии, что модели являются вложенными, может быть использован стандартный тест отношения правдоподобия. Если модели являются невложенными, тогда возможно использовать варианты информационного критерия Акаике, основанного на соответствующей логарифмической функции правдоподобия с корректировкой на число степеней свободы, равное количеству параметров в модели (см. параграф 8.5.1).

Полезный показатель псевдо-$R^2$, предложенный МакФадденом (1973), имеет вид

\begin{equation} \label{GrindEQ__15_9_} 
R^2=1-{{\ln  L_{fit}\ }}/{{\ln  L_0\ },} 
\end{equation} 

где ${\ln  L_{fit}\ }$ обозначает изучаемую модель, а $L_0$ обозначает модель только с константой, которая оценивает вероятность выбора каждой альтернативы  средним по выборке. Для любой мультиномиальной модели теоретическое максимальное значение логарифмической функции правдоподобия равно нулю. Такая ситуация возникает, если $p_{ij}=1$, когда $y_{ij}=1$ и $p_{ij}=0$, в других случаях для $i$ и $j$. Таким образом, можно также предложить показатель $R^2$ равный

\[R^2=\frac{{\ln  L_{fit}\ }-{\ln  L_0\ }}{{\ln  L_{max}\ }-{\ln  L_0\ }}.\] 

Этот показатель можно интерпретировать как долю от максимального потенциального прироста логарифмической функции правдоподобия, полученную при использовании соответствующей модели (см. параграф 8.7.1).

\section{Мультиномиальная логит-модель}

Простейшей мультиномиальной моделью является мультиномиальная логит-модель, предложенная Люсом (1959). Варианты этой модели, которые обычно используются, различаются по принципу зависимости или независимости значений регрессоров от альтернатив. Многие из вопросов, рассмотренных в настоящем разделе, относятся и к другим моделям, представленным в последующих разделах.

\subsection{Условная, мультиномиальная и смешанная логит-модели}

В случае зависимости значений регрессоров от выбираемых альтернатив (см. параграф 15.3.4) используется \textbf{условная логит-модель}. Согласно этой модели

\begin{equation} \label{GrindEQ__15_10_} p_{ij}=\frac{e^{x'_{ij}\beta }}{\sum^m_{l=1}{e^{x'_{il}\beta }}},\ \ \ \ \ \ \ \ j=1,\dots ,m. \end{equation} 

Поскольку ${\exp  \left(x'_{il}\beta \right)\ }>0$, вероятности попадают в промежуток от нуля до единицы, а их сумма по $j$ равна единице. Действительно, стоит один раз осмыслить формулу \eqref{GrindEQ__15_10_}, кажется, что она является наиболее простым определением, которое гарантирует требуемые вероятностные свойства. Поскольку $\sum^m_{i=1}{p_{ij}}=1$, эквивалентную модель можно получить, определив $x_{ij}$ как отклонение регрессора от значения регрессора для первой альтернативы, и скажем, установив $x_{i1}=0.$

Если, наоборот, регрессоры постоянны для всех альтернатив, то используется \textbf{мультиномиальная логит-модель}. Согласно этой модели

\begin{equation} \label{GrindEQ__15_11_} p_{ij}=\frac{e^{x'_i{\beta }_j}}{\sum^m_{l=1}{e^{x'_i{\beta }_l}}},\ \ \ \ \ \ \ \ j=1,\dots ,m. \end{equation} 

Поскольку $\sum^m_{i=1}{p_{ij}}=1,$ необходимо наложить ограничение,  чтобы обеспечить идентификацию модели, в качестве которого обычно выступает ${\beta }_1=0.$

Две обозначенные модели могут быть скомбинированы в модель, которую некоторые авторы называют \textbf{смешанной логит-моделью, }для которой

\begin{equation} \label{GrindEQ__15_12_} p_{ij}=\frac{e^{x'_{ij}\beta +w'_i{\gamma }_j}}{\sum^m_{l=1}{e^{x'_{il}\beta +w'_i{\gamma }_l}}},\ \ \ \ \ \ \ \ j=1,\dots ,m, \end{equation} 

где $x_{ij}$ изменяются в зависимости от альтернатив, а $w_i$ не изменяются в зависимости от альтернатив. Как рассматривалось в параграфах 15.2.3 и 15.3.4., смешанная и мультиномиальная логит-модели могут быть представлены как условная логит-модель. Обратите внимание, что термин смешанная логит-модель также иногда используется для совершенно другой модели, которая подробно рассмотрена в разделе 15.7.

Всем этим моделям можно дать общее название мультиномиальных логит-моделей, однако, мы будем следовать стандартной договоренности о различении мультиномиальной и условной логит-моделей.

Очевидным обобщением мультиномиальной логит-модели является

\begin{equation} \label{GrindEQ__15_13_} p_{ij}=\frac{V_{ij}}{\sum^m_{l=1}{V_{il}}},\ \ \ \ \ \ \ \ j=1,\dots ,m, \end{equation} 

где $V_{ij}>0$ может быть довольно произвольной функцией регрессора $x_i$ и параметра $\beta $. Такая модель называется \textbf{универсальной логит-моделью.} И хотя, из этой модели можно вывести обширный класс моделей, она редко используется в эконометрике, поскольку не возникает естественным образом из теории выбора.

\subsection{ММП для смешанной и множественной логит-моделей}

Здесь будут представлены ключевые формулы для условной и мультиномиальной логит-моделей. Полные доказательства приведены в разделе 15.12.

Для условной логит-модели, для которой $p_{ij}$ определяется по формуле \eqref{GrindEQ__15_10_}, ${\partial p_{ij}}/{\partial \beta }=p_{i??}\left(x_{ij}-{\overline{x}}_i\right),$  где ${\overline{x}}_i=\sum^m_{l=1}{p_{il}x_{il}}$ -- среднее значение регрессоров, взвешенное с помощью вероятностей (см. параграф 15.12.1). Условие первого порядка максимизации функции правдоподобия для условной логит-модели, представленное в выражении \eqref{GrindEQ__15_6_} для общего $p_{ij}$, упрощается непосредственно до

\begin{equation} \label{GrindEQ__15_14_} \sum^N_{i=1}{\sum^m_{j=1}{y_{ij}\left(x_{ij}-{\overline{x}}_i\right)=0.}} \end{equation} 

Продифференцировав это выражение по $\beta '$, используя $\E\left[y_{ij}\right]=p_{ij}$ и произведя некоторые алгебраические преобразования, получаем

\begin{equation} \label{GrindEQ__15_15_} {\widehat{\beta }}_{CL}\overset{a}{\sim }
{\mathcal N}\left[\beta ,\ {\left(\sum^N_{i=1}{\sum^m_{j=1}{p_{ij}}}\left(x_{ij}-{\overline{x}}_i\right)\left(x_{ij}-{\overline{x}}_i\right)'\right)}^{-1}\right]. \end{equation} 

Для мультиномиальной модели $p_{ij}$ определяется по формуле \eqref{GrindEQ__15_11_}, а в параграфе 15.12.2 показано, что ${\partial p_{ij}}/{\partial {\beta }_k}=p_{ij}\left({\delta }_{ijk}-p_{ik}\right)x_i,$ где ${\delta }_{ijk}$ переменная-индикатор, равная единице, если $j=k$, и равная нулю, если $j\ne k.$ Тогда, после некоторых алгебраических преобразований, получаем условие первого порядка максимизации функции правдоподобия для мультиномиальной логит-модели

\begin{equation} \label{GrindEQ__15_16_} \frac{\partial {\mathcal L}}{\partial {\beta }_k}=\sum^N_{i=1}{\left(y_{ik}-p_{ik}\right)x_i=0,\ \ \ \ \ \ \ \ \ k=1,\dots ,m.} \end{equation} 

В обычном случае${\widehat{\beta }}_{CL}\overset{a}{\sim }
{\mathcal N}\left[\beta ,\ {\left(E\left[{{\partial }^2{\mathcal L}}/{\partial\beta \partial{\beta }'}\right]\right)}^{-1}\right],$ для которой после некоторых алгебраических преобразований получаем $j$-ый блок информационной матрицы

\begin{equation} \label{GrindEQ__15_17_} E\left[\frac{{\partial }^2{\mathcal L}}{\partial {\beta }_j\partial {\beta }'_k}\right]=\sum^N_{i=1}{p_{ij}\left({\delta }_{ijk}-p_{ik}\right)x_ix'_i},\ \ \ \ \ \ \ \ \ j=1,\dots ,m,\ \ k=1,\dots ,m. \end{equation} 

\subsection{Интерпретация параметров регрессии}

Необходимо проявлять осторожность при интерпретации параметров любой нелинейной модели. Это, в частности, относится и к мультиномиальным моделям, для которых, например, совершенно необязательно соответствие между знаком коэффициента и вероятностью. В настоящем параграфе будут представлены утверждения, используемые в примере из раздела 15.2.

\subsubsection*{Предельные эффекты и эластичности}

Сосредоточимся на рассмотрении \textbf{предельных эффектов} на вероятности выбора при изменении значения регрессора для заданного индивида. \textbf{Эластичности } могут быть рассчитаны путем умножения предельного эффекта на значение соответствующего регрессора и деления на вероятность. Обычно эти показатели далее усредняются по индивидам для получения среднего предельного эффекта или среднего значения эластичности.

Для условной логит-модели определим предельный эффект как изменение $j$-ой вероятности на единицу изменения значения регрессора для $k$-ой альтернативы. Например, какой эффект произведет на вероятность выбора различных способов транспортировки увеличение времени путешествия на автобусе на одну минуту, если время путешествия другими видами транспорта останется неизменным? Из параграфа 15.12.1

\begin{equation} \label{GrindEQ__15_18_} \frac{\partial p_{ij}}{\partial x_{ik}}=p_{ij}\left({\delta }_{ijk}-p_{ik}\right)\beta , \end{equation} 

где ${\delta }_{ijk}$ определено после \eqref{GrindEQ__15_15_}. Отсюда следует, что, если коэффициенты регрессии положительны, тогда увеличение соответствующего  регрессора для $k$-ой альтернативы увеличивает вероятность выбора $k$-ой альтернативы и снижает вероятность выбора других альтернатив.

Для мультиномиальной модели, наоборот, определим предельный эффект на $j$-ую вероятность от изменения на единицу значения регрессора, который принимает одинаковые значения для всех альтернатив. Например, какой эффект на вероятность принятия решения работать производит увеличение возраста на один год? Из параграфа 15.22.2

\begin{equation} \label{GrindEQ__15_19_} \frac{\partial p_{ij}}{\partial x_i}=p_{ij}\left({\beta }_j-{\overline{\beta }}_i\right), \end{equation} 

где ${\overline{\beta }}_i=\sum_l{p_{il}}{\beta }_l$ -- среднее значение ${\beta }_l$, взвешенное с помощью вероятностей. Отсюда следует, что знак эффекта необязательно будет совпадать со знаком ${\beta }_j$, за исключением случая, когда ${\beta }_j>{\beta }_k$ для всех $k\ne j,$. Необязательно есть какой-то смысл в проверке равенства нулю некоторого коэффициента. Как и для других нелинейных моделей мы можем рассчитать среднее значение эффекта по формуле $N^{-1}\sum_i{{\partial p_{ij}}/{\partial x_i}}=\sum_i{p_{ij}\left({\beta }_j-{\overline{\beta }}_i\right),}$ или мы можем применить неаналитические методы и сравнить изменения в среднем значении вероятности в зависимости от изменения независимой переменной.

Сравнение с базовой категорией

Коэффициенты условной и мультиномиальной логит-моделей могут быть более точно интерпретированы также как и коэффициенты логит-модели, в терминах относительного риска (см. параграф 14.3.4). Это  возможно, поскольку эти модели могут быть представлены как логит-модели бинарного выбора.

Для мультиномиальной логит-модели сравнение проводится с базовой категорией, которой является нормализованная альтернатива с коэффициентом равным нулю. Чтобы увидеть это, отметим, что вероятности в мультиномиальной логит-модели \eqref{GrindEQ__15_11_} приводят к тому, что условная вероятность наблюдения альтернативы $j$, если  наблюдается альтернатива $j$ или $k$, принимает вид

\[{\Pr  [y=j | y=j\text{ или } k]\ }=\frac{p_j}{p_j+p_k}\] 

\begin{equation} \label{GrindEQ__15_20_} =\frac{e^{x'{\beta }_j}}{e^{x'{\beta }_j}+e^{x'{\beta }_k}} \end{equation} 

\[=\frac{e^{x'({\beta }_j-{\beta }_k)}}{1+e^{x'({\beta }_j-{\beta }_k)}},\] 

то есть вид логит-модели с коэффициентами$({\beta }_j-{\beta }_k)$. Вторая строка уравнения получается после некоторых упрощений. Предположим, что нормализации подверглась альтернатива 1, тогда ${\beta }_1=0.$

Тогда

\[{\Pr  \left[y_i=j\mathrel{\left|\vphantom{y_i=j y_i=j\text{ или } k}\right.\kern-\nulldelimiterspace}y_i=j\text{ или } k\right]\ }=\frac{e^{x'_i{\beta }_j}}{1+e^{x'_i{\beta }_j}},\] 

а параметр ${\beta }_j$ может быть интерпретирован по аналогии с коэффициентами логит-модели бинарного выбора между альтернативами $j$ и 1. Аналогично бинарной логит-модели \textbf{относительные шансы} выбора альтернативы $j$, а не альтернативы 1, составит

\[\frac{{\rm Pr}[y_i=j]}{{\rm Pr}[y_i=1]}=e^{x'_i{\beta }_j},\] 

а, следовательно, $e^{{\beta }_{jr}}$ позволяет оценить пропорциональное изменение значения отношения шансов, когда $x_{ir}$ изменяется на единицу. Такая интерпретация будет зависеть от того, какая альтернатива была выбрана для нормализации и имеет коэффициент, равный нулю. Для такой интерпретации, чтобы она была действительно полезной, необходимо определить естественную \textbf{базовую категорию}. Например, если изучаются способы поездки на работу, альтернативные поездке на автомобиле, тогда следует нормализовать коэффициенты этой альтернативы и приравнять их нулю. 

Аналогичный подход может быть также применен и к условной логит-модели, для которой 

\begin{equation} \label{GrindEQ__15_21_} {\Pr  \left[y_i=j\mathrel{\left|\vphantom{y_i=j y_i=j\text{ или } k}\right.\kern-\nulldelimiterspace}y_i=j\text{ или } k\right]\ }=\frac{e^{(x_{ij}-x_{ik})'\beta }}{1+e^{(x_{ij}-x_{ik})'\beta }}, \end{equation} 

а нормализация осуществляется по отношению к значению регрессора базовой категории.

\subsection{Независимость от посторонних альтернатив}

Ограниченность условной и мультиномиальной логит-моделей состоит в том, что различие между $m$ альтернативами сводится к серии попарных сравнений, которые не зависят от характеристик прочих альтернатив, кроме пары, находящейся в рассмотрении. Это очевидно из \eqref{GrindEQ__15_20_} и \eqref{GrindEQ__15_21_}, которые показывают, что мультиномиальная логит-модель сводится к логит-модели бинарного выбора между любой из пар. Условная вероятность не зависит от других альтернатив. 

Примером может послужить ситуация, когда условная вероятность выбора поездки на работу на машине, а не  на красном автобусе, в мультиномиальной или условной логит-модели, будет независима от наличия возможности ездить на синем автобусе. Тем не менее, на практике мы будем ожидать, что включение возможности поездки на синем автобусе, которая во всех аспектах совпадает с поездкой на красном автобусе, за исключением цвета автобуса, не будет оказывать большого влияния на предпочтение поездки на машине и сократит вдвое использование красного автобуса, приводя тем самым к увеличению относительной условной вероятности использования машины, предполагая наличие возможности ездить на машине или на красном автобусе.

Эта слабость мультиномиальной логит-модели известна в литературе как парадокс красного и синего автобуса или, более формальным языком, как предпосылка \textbf{независимости от посторонних альтернатив}. Она может быть проверена при помощи теста Хаусмана (см. работу Хаусмана и МакФаддена, 1984). Например, мы можем сравнить оценки коэффициентов для вероятности выбора поездки на красном автобусе в модели выбора с тремя альтернативами -- поездкой на автомобиле, красном или синем автобусе, с поездкой на машине в качестве базовой категории, с оценками коэффициентов для вероятности выбора поездки на красном автобусе для модели бинарного выбора между поездкой на машине или красном автобусе, снова с поездкой на машине в качестве базовой категории.

Большинство литературы по эконометрике обращено к альтернативным моделям с неупорядоченными исходами, в которых данный недостаток отсутствует. Эти модели представлены в разделах 15.6-15.8.

\section{Аддитивные модели случайной полезности}

Мультиномиальные модели с неупорядоченными исходами, которые являются более общими, чем мультиномиальная и условная логит-модели, могут быть получены с использованием общей структуры аддитивных моделей случайной полезности, представленной в настоящем разделе. В последующих разделах будут описаны основные примеры таких моделей.

\subsection{Модель ARUM}

\textbf{Аддитивная модель случайной полезности} (ARUM, additive random utility model) была введена в параграфе 14.4.2 для случая проблемы бинарного выбора. В общей мультиномиальной модели с $m$ альтернатив полезность $j$-ой альтернативы будет задана как

\begin{equation} \label{GrindEQ__15_22_} U_j=V_j+{\varepsilon }_j,\ \ \ \ \ \ \ \ j=1,2,\dots ,m, \end{equation} 

где $V_j$ обозначает детерминированный компонент полезности, а ${\varepsilon }_j$ --- случайный компонент полезности. Обычно для $i$-го индивида $V_{ij}=x'_{ij}\beta $ или $V_{ij}=x'_i{\beta }_j,$ тем не менее, структурный анализ может указать на явную или неявную функцию полезности, используемую в теории потребительского спроса. Для простоты обозначения в дальнейшем мы избавимся от индивидуального индекса $i$.

Выбранной становится альтернатива с наибольшей полезностью, то есть

\begin{equation} \label{GrindEQ__15_23_} {\Pr  \left[y=j\right]\ }={\Pr  \left[U_j\ge U_k,\text{ для всех } k\ne j\right]\ } \end{equation} 

\[={\Pr  \left[U_k-U_j\le 0,\text{ для всех } k\ne j\right]\ }\] 

\[={\Pr  \left[{\varepsilon }_k-{\varepsilon }_j\le V_j-V_k,\text{ для всех } k\ne j\right]\ }\] 

\[={\Pr  \left[{\widetilde{\varepsilon }}_{kj}\le {-\tilde{V}}_{kj},\text{ для всех } k\ne j\right],\ }\] 

где тильда и второй индекс $j$ обозначают разницу по отношению к базовой альтернативе $j$.

% здесь :)
Различные мультиномиальные модели могут быть построены в зависимости от различных предположений о совместном распределении случайных ошибок. Эти модели являются корректными статистически, с суммой вероятностей по всем альтернативам, равной единице. Кроме того, они согласуются со стандартной экономической теорией принятия решений.

Например, рассмотрим выражение ${\rm Pr}[y=1]$ для модели выбора с тремя альтернативами. Используя последнее равенство из выражения \eqref{GrindEQ__15_23_} и определив ${\widetilde{\varepsilon }}_{31}={\varepsilon }_3-{\varepsilon }_1$ и ${\widetilde{\varepsilon }}_{21}={\varepsilon }_2-{\varepsilon }_1$, получим

\begin{equation} \label{GrindEQ__15_24_} {\Pr  \left[y=1\right]\ }={\Pr  \left[{\widetilde{\varepsilon }}_{21}\le -{\tilde{V}}_{21},\ \ {\widetilde{\varepsilon }}_{31}\le -{\tilde{V}}_{31}\right]\ } \end{equation} 

\[=\int^{-{\tilde{V}}_{31}}_{-\infty }{\int^{-{\tilde{V}}_{21}}_{-\infty }{f({\widetilde{\varepsilon }}_{21},{\widetilde{\varepsilon }}_{31})d{\widetilde{\varepsilon }}_{21}d{\widetilde{\varepsilon }}_{31}}},\] 

то есть двумерный интеграл, который обычно не имеет аналитического решения. Более общая модель выбора с $m$ альтернативами предполагает $(m-1)$-мерный интеграл, для которого может быть, а может и не быть получено аналитическое решение для ${\Pr  \left[y=j\right]\ }.$

В общем случае все ошибки ${\varepsilon }_1,{\varepsilon }_2,\ \dots ,{\varepsilon }_m$ могут коррелировать между собой. Необходимо наложить ряд \textbf{ковариационных ограничений}, поскольку модель \textbf{идентифицируется} только с точностью до $(m-1)$  разниц между ошибками (см. последнее равенство в выражении (15.23)), и дополнительно одна дисперсия должна быть специфицирована, поскольку $U_j$ определяется только с точностью до масштаба.

\subsection{Различные  мультиномиальные модели с неупорядоченными исходами}

Различные мультиномиальные модели с неупорядоченными исходами возникают в зависимости от предположений о совместном распределении ошибок ${\varepsilon }_1,{\varepsilon }_2,\ \dots ,{\varepsilon }_m$. Анализ получается наиболее простым, если предположения об ошибках приводят к получению аналитического решения для вероятностей выбора. Тем не менее, во многих приложениях подобные допущения принимают слишком ограничительный характер.

Вычислительно интенсивные методы, приведенные в главе 12 позволяют оценить модель, даже если отсутствует аналитическое решение для вероятностей выбора. В параграфах 15.7.2 и 15.8.2 приведены примеры использования этих методов в случае мультиномиальных данных.

Ошибки, имеющие распределение экстремальных значений первого типа

Сначала допустим, что ${\varepsilon }_j$ -- независимые одинаково распределенные случайные величины, имеющие распределение экстремальных значений первого типа с плотностью распределения

\begin{equation} \label{GrindEQ__15_25_} f\left({\varepsilon }_j\right)=e^{-{\varepsilon }_j}{\exp  \left(-e^{-{\varepsilon }_j}\right)\ },\ \ \ \ \ \ \ \ j=1,2,\dots ,m. \end{equation} 

Свойства этой функции плотности были описаны в параграфе 14.4.2, где было показано что из нее следует логит-модели для случая бинарных исходов.

Для случая множественных исходов, для моделирования которых используется ARUM со случайными ошибками, имеющими распределение экстремальных значений первого типа, может быть показано, что \eqref{GrindEQ__15_23_} дает

\begin{equation} \label{GrindEQ__15_26_} {\Pr  \left[y=j\right]\ }=\frac{e^{V_j}}{e^{V_1}+e^{V_2}+\dots +e^{V_m}}. \end{equation} 

Это условная логит-модель, когда $V_j=x'_j\beta $, или мультиномиальная логит-модель, когда $V_j=x'{\beta }_j.$ Результат может быть получен путем либо интегрирования и упрощения аналогично случаю бинарных данных (см. раздел 14.8), либо как частный случай результата вложенной логит-модели, полученный в разделе 15.6. Таким образом, условная и мультиномиальная логит-модели могут быть получены из модели ARUM.

Допущение о том, что ошибки ${\varepsilon }_j$ независимы по альтернативам $j$, является слишком ограничительным, поскольку вполне вероятно может быть нарушено, если две альтернативы похожи. Например, предположим, что альтернативы 1 и 2 похожи. Низкое значение ${\varepsilon }_1$ (т.е. большое по модулю и отрицательное) приведет к переоценке полезности альтернативы 1. Тогда, также можно ожидать переоценки полезности альтернативы 2, поэтому ${\varepsilon }_2$ также принимает низкое значение. Поскольку низкие значения для ${\varepsilon }_1$ и ${\varepsilon }_2$, как правило, случаются одновременно, что аналогично и для высоких значений, ошибки должны коррелировать. Это другой способ рассмотрения проблемы красного и синего автобуса, а также проявление невыполнения допущения логит-моделей о независимости от посторонних альтернатив.

Обобщенная модель на базе распределения экстремальных значений и вложенная логит-модель (см. раздел 15.6) смягчают допущение о том, что ошибки, имеющие распределение экстремальных значений независимы по альтернативам. Ошибки сгруппированы так, что наблюдается независимость между группами ошибок, но допускается наличие корреляции между ошибками внутри одной группы. Тогда становится доступным получение аналитического решения для вероятностей выбора. И хотя эти модели намного богаче, чем мультиномиальная логит-модель, являющейся частным случаем  отсутствия корреляции ошибок внутри группы, во многих приложениях группировка ошибок происходит в некотором смысле произвольно.

В логит-модели со случайными параметрами (см. раздел 15.7) в мультиномиальную логит-модель вводится дополнительная хаотичность, которая вызывает корреляцию между полезностями разных альтернатив. Такая модель является примером обобщенной модели случайной полезности (см. параграф 15.7.3).

Ошибки, имеющие нормальное распределение

Мультиномиальная пробит-модель (см. раздел 15.8) возникает, если полагается, что ошибки ${\varepsilon }_1,\dots ,{\varepsilon }_m$ имеют совместное нормальное распределение. Такое предположение является более естественным, чем предположение о распределении экстремальных значений первого типа для ошибок. Это позволяет допустить очень богатую структуру корреляции между ошибками за счет необходимости применения численных или симуляционных методов, с помощью которых находится значение $(m-1)$-мерного нормального интеграла.

\subsection{Согласованность с моделями случайной полезности}

Довольно легко предъявить аналитическое выражение для вероятности выбора, которая лежит между нулем и единицей, а их сумма по альтернативам равна единице. Более или менее общим примером является универсальная логит-модель в выражении \eqref{GrindEQ__15_13_}. В эконометрической литературе особое значение придается  мультиномиальным моделям, которые согласовываются с максимизацией функции случайной полезности. Этот случай аналогичен ограничению анализа функции спроса, которая согласуется с теорией потребительского выбора. 

Пусть $V=\left(V_1,\dots ,V_m\right).$ Из работы Борша-Супана (1987, стр. 19), набор вероятностей выбора $p_j\left(V\right),\ =1,\dots ,m,$ совместим с максимизацией в ARUM, если выполняется

\begin{enumerate}
\item  $p_j\left(V\right)\ge 0,\ \sum^m_{j=1}{p_j\left(V\right)=1}$, и $p_j(V)=p_j(V+\alpha )$для всех $\alpha \in R$;

\item  ${\partial p_j(V)}/{\partial V_k}={\partial p_k(V)}/{\partial V_j}$; и

\item  ${{\partial }^{\left(m-1\right)}p_j(V)}/{\partial V_1\dots \left[\partial V_i\right]\dots \partial V_m\ge 0,}$ где квадратные скобки обозначают пропускаемый элемент.
\end{enumerate}

Эти условия, сформулированные Уильямсом (1977), Дэйли и Закари (1979) и МакФадденом (1981), обеспечивают соответственно (1) хорошее поведение вероятностей и инвариантность к смещению; (2) интегрируемость $p_j$ аналогично условию Слуцкого; (3) функция распределения ошибок в соответствующей ARUM модели имеет собственную (неотрицательную) функцию плотности.

\subsection{Анализ благосостояния}

Основным преимуществом использования мультиномиальных моделей, к которым относится модель случайной полезности, является то, что они позволяют произвести анализ благосостояния. Тогда эффект от изменения одной или более детерминант выбора, таких как цена или время в пути в ситуации выбора способа транспортировки, может быть оценен в денежном эквиваленте.

В стандартном \textbf{анализе благосостояния} используется компенсирующая или эквивалентная вариация дохода. Детерминированная компонента полезности из выражения \eqref{GrindEQ__15_22_} задается как косвенная функция полезности

\begin{equation} \label{GrindEQ__15_27_} V_j=V\left(I-p_j,x_j\right), \end{equation} 

где $I$ обозначает доход, $p_j$ -- цена $j$-ой альтернативы, а $x_j$ --- характеристики $j$-ой альтернативы. Для простоты обозначения неизвестный параметр регрессии $\beta $ опущен. Тогда полезность альтернативы $j$ будет определяться по формуле

\begin{equation} \label{GrindEQ__15_28_} U_j=U\left(I-p_j,x_j,{\varepsilon }_j\right)=V\left(I-p_j,x_j\right)+{\varepsilon }_j. \end{equation} 

Предположим, что значение некоторой характеристики изменилось с $x'_j$ на $x''_j$. Тогда \textbf{компенсирующая вариация дохода } $CV$ --- это изменение в доходе, необходимое для поддержания полезности на начальном уровне. Таким образом, наивысший уровень полезности, достижимый при доходе $I$ и значении характеристики $x'_j$, должен быть равен наивысшему уровню полезности, достижимой при уровне дохода $(I-CV)$ и значении характеристики $x''_j$. Тогда компенсирующая вариация $CV$ неявным образом определяется как решение уравнения 

\begin{equation} \label{GrindEQ__15_29_} {\mathop{\max }_{j=1,\dots ,m} U\left(I-p_j,x_j,{\varepsilon }_j\right)=\ }{\mathop{\max }_{j=1,\dots ,m} U\left(I-CV-p_j,x''_j,{\varepsilon }_j\right).\ } \end{equation} 

В качестве примера рассмотрим модель выбора с двумя альтернативами, для которой $U_j=I+x_j+{\varepsilon }_j,\ j=1,2,$ а скалярная величина $x_j$ изменяет свое значение с $x'_j$ на $x''_j$. Тогда возможны четыре ситуации. Если альтернатива 1 выбирается как до, так и после изменения характеристики, тогда $CV=\left(x''_1-x'_1\right),$ поскольку $U''_1=I-CV+x''_1+{\varepsilon }_1=I+x'_1+{\varepsilon }_1=U'_1.$ Аналогично, если альтернатива 2 выбирается до и после изменения характеристики, тогда $CV=\left(x''_2-x'_2\right).$ Если происходит переключение с альтернативы 1 к альтернативе 2, тогда $U''_2=U'_1$ означает $I-CV+x''_2+{\varepsilon }_2=I+x'_1+{\varepsilon }_1,$ отсюда $CV=x''_2-x'_1+{\varepsilon }_2-{\varepsilon }_1.$ Аналогично, если происходит переключение с альтернативы 2 к альтернативе 1, тогда $CV=x''_1-x'_2+{\varepsilon }_1-{\varepsilon }_2.$ В более общем смысле, для $m$ альтернатив компенсирующая вариация определяется как $CV_{jk}=V''_k-V'_j+{\varepsilon }_k-{\varepsilon }_j$, если изменение значения $x$ приводит к переключению с альтернативы $j$ к альтернативе $k.$

Компенсирующая вариация дохода зависит от наблюдаемых переменных $(I,\ p_j,$ и $x_j)$,  параметров, который можно оценить, и от ненаблюдаемых ошибок ${\varepsilon }_j.$ Ненаблюдаемые ошибки устраняются путем расчета ожидаемого значения компенсирующей вариации $\E\left[CV\right],$ которое включает интегрирование по ${\varepsilon }_j$. Из предыдущего примера должно быть очевидным, что такое интегрирование может быть достаточно сложным. В работе Дагсвика и Карлстрёма (2004) приведены достаточно общие результаты по этому поводу, которые будут обсуждаться далее в параграфе 15.6.5.

Для некоторых моделей не существует аналитического решения для $\E\left[CV\right].$ Тогда необходимо найти интеграл функции $CV$, определенной по формуле \eqref{GrindEQ__15_19_}, по ${\varepsilon }_j$, используя численные методы. Как описано в параграфе 12.3.2 значение этого интеграла может быть найдено следующим образом:

\begin{enumerate}
\item  На итерации $s$ сгенерировать ${\varepsilon }^s$ из распределения $\varepsilon =\left({\varepsilon }_1,\dots ,{\varepsilon }_m\right).$

\item  Найти $CV^s$ из уравнения $\underset{j=1,\dots ,m}{\max } 
U\left(I-p_j,x_j,{\varepsilon }_j\right)=
\underset{j=1,\dots ,m}{\max } U\left(I-CV^s-p_j,x''_j,{\varepsilon }_j\right).\ $

\item  Повторить шаги $1$ и 2 $S$ раз.

\item  Оценить $E[CV]$ как $S^{-1}\sum^S_{t=1}{CV^t}.$
\end{enumerate}

Этот метод позволяет получить $\E\left[CV\right]$ для каждого индивида в выборке. Усреднение, возможно со взвешиванием, дает оценку по генеральной совокупности. Приложение моделей с ошибками, имеющими распределение экстремальных значений первого типа обсуждаются в параграфе 15.6.5.

\section{Вложенная логит-модель}

Вложенная логит-модель является наиболее легко поддающимся аналитической трактове обобщением мультиномиальных моделей. Она является идеальной моделью, когда присутствует явная структура вложения, однако не все случаи множественного выбора имеют очевидную структуру вложения. 

\subsection{Модель с ошибками, имеющими обобщенное распределение экстремальных значений}

МакФадден (1978) предложил достаточно общий класс моделей, основанных на предположении, что совместным распределением ошибок является \textbf{обобщенное распределение экстремальных значений (ОРЭЗ) } с совместной функцией распределения 

\begin{equation} 
\label{GrindEQ__15_30_} 
F\left({\varepsilon }_1,{\varepsilon }_2,\dots ,{\varepsilon }_m\right)={\exp  \left[-G\left(e^{-{\varepsilon }_1},e^{-{\varepsilon }_2},\dots ,e^{-{\varepsilon }_m}\right)\right]\ }, 
\end{equation} 

где функция $G(Y_1,Y_2,\dots ,Y_m)$ выбрана так, что она удовлетворяет набору допущений, таких как неотрицательность, однородность первой степени, имеет смешанные частные производные, которые являются непрерывными и неположительными для четного и неотрицательными для нечетного порядка, а ${\mathop{\lim }_{Y_{j\to \infty }} G\left(Y_1,Y_2,\dots ,Y_m\right)=\infty .\ }$ Эти предположения обеспечивают то, что совместное распределение и финальные частные распределения будут корректно определены, а сумма вероятностей составит единицу. 

Если ошибки имеют ОРЭЗ, тогда явные значения для вероятностей в модели случайной полезности \eqref{GrindEQ__15_22_} могут быть получены из

\begin{equation} \label{GrindEQ__15_31_} p_j={\Pr  \left[y=j\right]\ }=e^{V_j}\frac{G_j(e^{-V_1},e^{-V_2},\dots ,e^{-V_m})}{G(e^{-V_1},e^{-V_2},\dots ,e^{-V_m})}, \end{equation} 

где $G_j\left(Y_1,Y_2,\dots ,Y_m\right)={\partial G(Y_1,Y_2,\dots ,Y_m)}/{\partial }Y_j$ (см. работу МакФаддена, 1978, стр. 81).

Широкий спектр моделей может быть получен при различных спецификациях формы $G\left(Y_1,Y_2,\dots ,Y_m\right).$ Мультиномиальная логит-модель возникает, если $G\left(Y_1,Y_2,\dots ,Y_m\right)=\sum^m_{k=1}{Y_k}$, следовательно она является моделью с ошибками, имеющими ОРЭЗ. Другой широко используемой моделью этого класса является вложенная логит-модель.

\subsection{Вложенная логит-модель}

Вложенная логит-модель разбивает возможные решения на несколько групп. Простым примером является задача выбора, когда сначала люди делают выбор между двухгодичным и четырехгодичным образованием, а затем внутри каждой из групп решают получать образование в государственном или частном учебном заведении. Эту ситуацию можно изобразить следующим образом:


Колледж

2 года               4 года

Частный Государственный Частный Государственный



Ошибки в модели случайной полезности могут быть коррелированы для индивидов, сделавших выбор внутри двухгодичной группы  и четырехгодичной группы, однако не может быть корреляции между ошибками этих двух групп.

Более общими словами, пусть на верхнем уровне существуют $J$ ветвей, между которыми необходимо сделать выбор. Пусть $j$-ая ветвь имеет $K_j$ ответвлений, пронумерованных как $j1,\dots ,jk,\dots ,jK_j.$ Тогда полезность для альтернативы $j$-ой из $J$ ветвей и $k$-ой из $K_j$ ответвлений будет

\begin{equation} \label{GrindEQ__15_32_} U_{jk}=V_{jk}+{\varepsilon }_{jk},\ \ k=1,2,\dots ,K_j,\ \ j=1,2,\dots ,J, \end{equation} 

где для модели с $m$ вариантами выбора $K_1+\dots +K_J=m.$ Эта модель может быть проиллюстрирована следующим образом:

Корень



Ветвь 1  \dots  Ветвь $j$ \dots  Ветвь$J$



Ответвление 1\dots Ответвление $K_1$   \dots      Ответвление $k$      \dots      Ответвление 1\dots Ответвление $K_J$

$V_{11}+{\varepsilon }_{11}$\dots $V_{1K_1}+{\varepsilon }_{1K_1}$\dots $V_{jk}+{\varepsilon }_{jk}$\dots $V_{J1}+{\varepsilon }_{J1}$\dots $V_{JK_J}+{\varepsilon }_{JK_J}$

Также в эту модель может быть добавлен и третий ряд ответвлений, и т.д. Для простоты обозначения мы будем рассматривать двухуровневую модель.

Для любой модели с вложенными вероятностями $p_{jk}$, совместная вероятность попасть на ветвь $j$ и на ответвление $k$ может быть разложена на произведение $p_j$, вероятность выбора ветви $j$, на $p_{j|k}$, вероятность выбора ответвления $k$ при условии выбора ветви $j$. Тогда

\[p_{jk}=p_j\times p_{j|k}.\] 

Вложенная логит-модель МакФаддена (1978) возникает, когда ошибка ${\varepsilon }_{jk}$ имеет совместную функцию распределения ОРЭЗ

\begin{equation} \label{GrindEQ__15_33_} F\left(\varepsilon \right)={\rm exp}{\rm [-}{\rm G}\left(e^{-{\varepsilon }_{11}},\dots ,e^{-{\varepsilon }_{1K_1}};;\dots ;;e^{-{\varepsilon }_{J1}},\dots ,e^{-{\varepsilon }_{JK_j}}\right)] \end{equation} 

для следующей конкретной формы функции $G\left(\cdot \right)$

\begin{equation} \label{GrindEQ__15_34_} G\left(Y\right)=G\left(Y_{11},\dots ,Y_{1K_1},\dots ,Y_{JK_j}\right)=\sum^J_{j=1}{{\left(\sum^{K_j}_{k=1}{Y^{{1}/{{\rho }_j}}_{jk}}\right)}^{{\rho }_j}.} \end{equation} 

Параметр ${\rho }_j$ является функцией коэффициента корреляции между ${\varepsilon }_{jk}$ и ${\varepsilon }_{jl}$, но не равен этому коэффициенту. По сути, можно показать, что ${\rho }_j$ равняется $\sqrt{1-Cor[{\varepsilon }_{jk},{\varepsilon }_{jl}]},$ т.е. параметр ${\rho }_j$ находится в обратной зависимости к коэффициенту корреляции и ественными являются значения $0\le {\rho }_j\le 1.$ Допущение о ${\rho }_j=1$ соответствует независимости между ${\varepsilon }_{jk}$ и ${\varepsilon }_{jl}$ и приводит к  мультиномиальной логит-модели. Назовем ${\rho }_j$\textbf{параметрами масштаба, } поскольку они масштабируют параметры в моделях, рассмотренных далее.

Обозначения варьируются у разных авторов. МакФадден (1978) и Маддала (1983), в отличие от нас, определяют эту функцию распределения в терминах ${\sigma }_j=1-{\rho }_j$, который называют \textbf{параметром расхождения. } Другие используют ${\mu }_j={1}/{{\rho }_j}$. Многие авторы нумеруют альтернативы $ij$ для $n$-го индивида, в то время как в настоящем пособии для альтернатив используются индексы $jk$ и коэффициент $i$ обозначает $i$-го индивида.

Переменная исхода $y_{jk}$ равна единице, если выбрана альтернатива $jk$, и нулю в противном случае. Тогда из выражения \eqref{GrindEQ__15_32_}, $p_{jk}={\Pr  \left[y_{jk}=1\right]\ }={\Pr  \left[U_{jk}\ge U_{lm},\text{ для всех } l,m\right]\ }.$ Аналитические решения для вероятностей $p_{jk}$, как функций $V_{jk}$ и ${\rho }_j$, получены в параграфе 15.12.3. Эти решения затем рассчитываются для детерминированной составляющей функции полезности 

\begin{equation} \label{GrindEQ__15_35_} V_{jk}=z'_j\alpha +x'_{jk}{\beta }_j,\ \ k=1,\dots ,K_j,\ \ j=1,\dots ,J, \end{equation} 

где $z_j$ изменяется только в зависимости от выбранной ветви, а $x_{jk}$ изменяется в зависимости как выбранной ветки, так и выбранного ответвления этой ветви. Параметры $\alpha $ и ${\beta }_j$ называются \textbf{параметрами регрессии.}

Модель, заданная уравнениями \eqref{GrindEQ__15_32_}-\eqref{GrindEQ__15_35_} дает \textbf{вложенную логит-модель}

\begin{equation} 
\label{GrindEQ__15_36_} 
p_{jk}=
p_j\times p_{j|k}=
\frac{{\rm exp}(z'_j\alpha +{\rho }_jI_j)}{\sum^J_{m=1}{{\rm exp}(z'_m\alpha +{\rho }_mI_m)}}
\times 
\frac{{\rm exp}(x'_{jk}{\beta }_j/{\rho }_j)}{\sum^{K_j}_{l=1}{{\rm exp}(x'_{jl}{\beta }_j/{\rho }_j)}}. 
\end{equation} 
% hi
(см.  15.12.3), где

\begin{equation} \label{GrindEQ__15_37_} I_j={\ln  \left(\sum^{K_j}_{l=1}{{\rm exp}(x'_{jl}{\beta }_j/{\rho }_j)}\right)\ } \end{equation} 

называется \textbf{включающей величиной }или \textbf{лог-суммой. } Одним из привлекательных свойств вложенной логит-модели является то, что вероятности $p_{ij}$ и $p_{j|i}$ естественным образом имеют вид как в условной логит-модели.

Описанные выше результаты получены для случая зависимости регрессоров от выбираемой альтернативы. Расчеты могут быть адаптированы для случая независимых от альтернатив регрессоров $V_{jk}=z'{\alpha }_j+x'{\beta }_{jk},$ с нормализацией одного из параметров ${\beta }_{jk}$. Алгебраически все, что необходимо, это разделение детерминированной составляющей $V_{jk}=A_j+B_{jk},$ где $A_j$ относится к некоторой ветви, а $B_{jk}$ относится как к конкретной ветви, так и к ее некоторому ответвлению.

\subsection{Оценка вложенной логит-модели}

Для $i$-го наблюдения мы видим $K_1+\dots +K_J$ исходов $y_{ijk}$, где $y_{ijk}=1$, если выбрана альтернатива $jk$, и $y_{ijk}=0$ в противном случае. Тогда $p_{ijk}=p_{ij|k}\times p_{ij}$, а плотность  для наблюдений $y_i=(y_{i11},\dots ,y_{iJK_j})$ может быть компактно записана как

\[f\left(y_i\right)=\prod^J_{j=1}{\prod^{K_j}_{k=1}{{\left[p_{ik|j}\times p_{ij}\right]}^{y_{ijk}}=\prod^J_{j=1}{\left(p^{y_{ij}}_{ij}\prod^{K_j}_{k = 1}{p^{y_{ijk}}_{ik|j}}\right)}},}\] 

где $y_{ij}=\sum^{K_j}_{l=1}{y_{ijl}}$ равно единице, если выбрана ветвь $j$, и нулю в противном случае.

Плотность распределения для выборки имеет вид $\prod^N_{i=1}{f\left(y_i\right).}$\textbf{Оценка методом максимального правдоподобия с полной информацией (ММППИ)} максимизирует функцию

\begin{equation} \label{GrindEQ__15_38_} {\ln  L\ }=\sum^N_{i=1}{\sum^J_{j=1}{y_{ij}{\ln  p_{ij}\ }}+\sum^N_{i=1}{\sum^J_{j=1}{\sum^{K_j}_{k=1}{y_{ijk}{\ln  p_{ik|j}\ }}}}}, \end{equation} 

относительно параметров $\alpha $, ${\beta }_j$ и ${\rho }_j.$

Альтернативой является менее эффективный \textbf{метод последовательного оценивания} или оценка методом максимального правдоподобия с ограниченной информацией (ММПОИ), который использует разделение $p_{jk}$ на произведение $p_{k|j}$ и $p_j$. Первый этап оценивания основывается на втором слагаемом из правой части выражения \eqref{GrindEQ__15_38_}, который согласно \eqref{GrindEQ__15_36_} является условной логит-моделью с оцениваемым параметром ${\beta }_j/{\rho }_j.$ Второй этап получения оценки основывается на первом слагаемом из правой части выражения \eqref{GrindEQ__15_38_}, которое согласно \eqref{GrindEQ__15_36_} является условной логит-моделью с добавленным регрессором ${\hat{I}}_{ij}$, оценкой включающей величины из \eqref{GrindEQ__15_37_}, которая может быть рассчитана с помощью оценки параметра, полученной на первом этапе. Параметры $\widehat{\alpha }$ и ${\widehat{\rho }}_j$ получены непосредственно на втором шаге, а ${\widehat{\beta }}_j$ равняется произведению ${\hat{p}}_j$ на оценку $\widehat{{\beta }_j/{\rho }_j}$, полученную на первом шаге.

Этот метод последовательного оценивания менее эффективен, чем ММППИ. На втором шаге обычные стандартные ошибки условной логит-модели занижают реальные стандартные ошибки метода последовательного оценивания, поскольку они не допускают наличия погрешности в оценке включающей величины. МакФадден (1981) приводит формулу для корректного расчета стандартных ошибок. Также может быть применен метод бутстрэпа. Метод последовательного получения оценки был предложен очень давно, когда еще даже оценки условной логит-модели было трудно получить. Сейчас относительно просто запрограммировать функцию правдоподобия, поэтому лучшим вариантом является использование ММППИ. Метод последовательного оценивания полезен для получения начальных значений параметров, поскольку функция правдоподобия в ММППИ не является вогнутой на всей области определения.

Для примера применим вложенную логит-модель к данным из раздела 15.2. Вложенная структура содержит рыбалку с побережья или с лодки на верхнем уровне и рыбалку с пляжа или с пристани (для рыбалки с побережья) или с частной или арендованной лодки (для рыбалки с лодки) на нижнем уровне. Регрессорами $x_{jk}$ в выражении \eqref{GrindEQ__15_36_}, которые будут варьироваться на нижнем уровне, выступают цена $(P)$ и коэффициент вылова $\left(C\right).$ Регрессором $z_j$ на верхнем уровне, который принимает значение в зависимости от выбора рыбалки с побережья или с лодки, выступает переменный индикатор $d$ равный единице, если выбрана рыбалка с побережья, и $d\times I$, доход умноженный на индикатор выбора рыбалки с побережья. Оценка условной логит-модели (с ${\rho }_1={\rho }_2=1$) дает  ${\ln  L\ }=-1252$, значение, которое, как ожидалось, меньше значения логарифмической функции правдоподобия для аналогичной, но менее подверженной ограничениям модели, приведенной в последнем столбце таблицы 15.2. Оценка ММППИ соответствующей вложенной логит-модели, для которой ${\rho }_1$ и ${\rho }_2$ теперь могут варьироваться, дает более высокое значение логарифмической функции правдоподобия и приводит к отказу от ограниченной условной логит-модели, используя тест отношения правдоподобия $\chi^2(2)$.

\subsection{Обсуждение}

Основным ограничением вложенной логит-модели является то, что не каждая проблема выбора может  быть естественным образом представлена в виде вложенной структуры. При необходимости можно выбрать вложенную схему, используя тест отношения правдоподобия или информационный критерий Акаике. Однако итоговая схема не всегда соответствует априорным ожиданиям.

Другим практическим вопросом является то, что состоятельность вложенной логит-модели с  ARUM требует выполнения трех условий из параграфа 15.5.2. 
Третье из этих условий выполняется глобально, если $0\le {\rho }_j\le 1$, и, в случае структуры с более чем двумя уровнями вложенности, дополнительно требуется, чтобы $\rho $ на верхних уровнях вложенной структуры не превосходил значений $\rho $ на ее нижних уровнях. 
На практике возможно получение оценки параметра ${\rho }_j$, которая выходит за пределы единичного интервала. Вложенную модель и тогда можно использовать, поскольку она приводит к вероятностям выбора лежащим в диапазоне от 0 до 1, однако модель уже не вытекает из ARUM. Борш-Супан и другие исследователи рассматривали условие локальной идентификации, при выполнении которого вложенная логит-модель может быть согласованной с ARUM, даже если ${\rho }_j$ лежит за пределами интервала от нуля до единицы. Также полезным может быть проведение  поиска на сетке по ${\rho }_j$, чтобы ограничить ${\rho }_j$ единичным интервалом и посчитать вызванное тем самым уменьшение значения логарифмической функции правдоподобия, если таковое имеет место.

Вложенная логит-модель, описанная в \eqref{GrindEQ__15_36_} и \eqref{GrindEQ__15_37_}, была предложена МакФадденом (1978), который получил ее как модель с ошибками с ОРЭЗ. \textbf{Ранний вариант } вложенной логит-модели был аналогичен \eqref{GrindEQ__15_36_}-\eqref{GrindEQ__15_37_}, за исключением того, что ${\rm exp}(x'_{jl}{\beta }_j/{\rho }_j)$ было заменено на ${\rm exp}(x'_{jl}{\beta }_j)$. Это было альтернативным способом вывода модели, как естественного расширения условной логит-модели, поскольку она является специальным случаем \eqref{GrindEQ__15_36_}-\eqref{GrindEQ__15_37_}, когда ${\rho }_j=1.$ Для более подробной информации ознакомьтесь с работами МакФаддена (1978, стр. 79), Маддалы (1983, стр. 70) и Грина (2003, стр. 726).

Важно отметить, что два варианта вложенной логит-модели различаются, если значения ${\rho }_j$ различаются для разных альтернатив (см. работы Коппельмана (1998), Вена (1998) и Трейна (2003, стр. 88)). В некоторых ранних исследованиях оценки, полученные методом последовательного оценивания, существенно отличались от оценок, полученных ММППИ, ставя под сомнение робастность вложенной логит-модели. Впрочем, в некоторых из этих исследований различные методы оценивания были применены к разным вариантам вложенной логит-модели. Более того, даже сегодня в различных статистических пакетах оцениваются разные варианты вложенной логит-модели.

Вложенная логит-модель может быть обобщена для случая альтернатив более высокого уровня (или вложенности). Например, в работе Голдберга (1995) рассматривается пятиуровневая структура: (1) купить автомобиль; (2) купить новый автомобиль, если выбрана альтернатива (1); (3) автомобиль какого из девяти классов был приобретен, если выбрана альтернатива (2); (4) иностранного или отечественного производства; (5) модель автомобиля. Дополнительной привлекательной стороной вложенной логит-модели в случае многоуровневой структуры вложенности является то, что для оценки параметров достаточно рассмотреть фиксированное или случайно выбранное подмножество альтернатив (см. работу МакФаддена, 1978).

\subsection{Анализ благосостояния}

Анализ благосостояния для моделей ARUM был представлен в параграфе 15.5.4. В общем случае не существует аналитического решения для $\E\left[CV\right]$, ожидаемого значения компенсирующей вариации дохода.

Примечательно, что для моделей, основанных на ОРЭЗ стандартной ошибки, линейных по доходу, $V\left(I-p_j,\ x_j\right)=\alpha \left(I-P_j\right)+f(x_j)$, МакФадден (1995) и другие исследователи в более ранних работах показали, что существует явное решение для ожидаемого значения компенсирующей вариации

\[\E\left[CV\right]=\frac{1}{\alpha }\left({\ln  G\left(e^{V''_1},\dots ,e^{V''_m}\right)\ }-{\ln  G\left(e^{V'_1},\dots ,e^{V'_m}\right)\ }\right),\] 

где функция $G(\cdot )$ для ОРЭЗ определена в выражении \eqref{GrindEQ__15_34_}, а $V'_j$ и $V''_j$ --- значения детерминированной компоненты полезности до и после изменения.

Для моделей, основанных на  ОРЭЗ стандартной ошибки, с нелинейной зависимостью компенсирующей вариации от дохода, тем не менее, не существует явного решения. Одним из подходов является симуляционный метод, описанный в параграфе 15.5.4. Для мультиномиальной логит-модели применить этот метод достаточно просто, поскольку просто получить ошибки, имеющие распределение экстремальных значений, используя метод преобразования, описанный в параграфе 12.8.2, --- сгенерируйте $u$, имеющее равномерное распределение на (0, 1), и  далее возьмите $\varepsilon =-{\ln  (-{\ln  (u)\ })\ }$. Для более общей логит-модели, однако, сложно сгенерировать ОРЭЗ ошибки, даже в простом случае двумерного распределения экстремальных значений. МакФадден (1995) предложил применять метод Монте-Карло по схеме марковской цепи совместно с алгоритмом Метрополиса-Гастингса (см. раздел 13.5). Герриджес и Клинг (1999) приводят блестящее описание этого метода с приложением к вложенной логит-модели для данных о выборе способа рыбалки из раздела 15.2, используя различные неявные функции полезности, включая и транслогарифмическую.

Не так давно Дагсвик и Карлстрём (2004) показали, что, хотя, нет явного решения для $\E\left[CV\right]$для моделей с ОРЭЗ ошибкой, если присутствует нелинейная зависимость компенсирующей вариации от дохода, то возможно аналитическими методами свести $\E\left[CV\right]$ к одномерному интегралу. Нахождение значения этого интеграла методом Гаусса -- более простая процедура, чем применение вышеупомянутого симуляционного метода.

\section{Логит-модель со случайными параметрами}

Логит-модель со случайными параметрами позволяет легко обобщить мультиномиальную или условную логит-модель так, чтобы допустить корреляцию полезностей разных альтернатив между собой. Эта модель, возможно, является главным эконометрическим примером моделей со случайными параметрами для пространственных данных в микроэкономике.

\subsection{Логит-модель со случайными параметрами}

\textbf{Логит-модель со случайными параметрами }  определяет полезность $j$-ой альтернативы для $i$-го индивида как

\begin{equation} \label{GrindEQ__15_39_} U_{ij}=x'_{ij}{\beta }_i+{\varepsilon }_{ij},\ \ j=1,2,\dots ,m. \end{equation} 

где ${\varepsilon }_{ij}$ -- независимые случайные величины, имеющие распределение экстремальных значений, как и в условной логит-модели, однако, дополнительно допускается случайность параметра ${\beta }_i$. Наиболее распространённым предположением является то, что 

\begin{equation} \label{GrindEQ__15_40_} {\beta }_i\sim {\mathcal N}\left[\beta ,\ {\Sigma }_{\beta }\right]. \end{equation} 

Также возможно использовать логнормальное, а не нормальное распределение для этих параметров, если их знаки известны априори. Эта модель также называется \textbf{смешанной логит-моделью, }если заимствовать терминологию для панельных моделей со случайными параметрами. Представив мультиномиальную логит-модель как условную, получим результат, относящийся и к мультиномиальной логит-модели со случайными параметрами.

Модель можно перезаписать как

\[U_{ij}=x'_{ij}\beta +v_{ij,}\] 

\[v_{ij}=x'_{ij}\beta +{\varepsilon }_{ij},\] 

где $u_i\sim {\mathcal N}\left[0,\ {\Sigma }_{\beta }\right].$ Тогда $Cov\left[v_{ij},\ v_{ik}\right]=x'_{ij}{\Sigma }_{\beta }x_{ik}$, $j\ne k$, то есть введение случайных параметров привлекательно тем, что позволяет допустить корреляцию между альтернативами.

Во многих приложениях этой модели ковариационная матрица ${\Sigma }_{\beta }$ определяется как диагональная, а также дополнительно предполагается  равенство нулю некоторых диагональных элементов. Тогда количество элементов ковариационной матрицы, которые должны быть оценены, равно количеству компонент ${\beta }_i$, которые определены как случайные.

В качестве примера рассмотрим смешанную условную логит-модель со скалярным регрессором и параметрами $\beta $ и ${\sigma }^2_{\beta }.$ Допустим, что оценками параметров являются $\widehat{\beta }=2,0$ со стандартной ошибкой 0,5 и ${\widehat{\sigma }}^2_{\beta }=1,0$ со стандартной ошибкой 0,2. Тогда нулевая гипотеза о том, что параметр имеет постоянное значение, то есть ${\sigma }^2_{\beta }=0$, решительно отвергается, поскольку $t=1,0/0,2=5,0.$ Эффект, который оказывает на ${\rm Pr}[y_i=j]$ увеличение $x_{ij}$, различен между индивидами и имеет положительное значение для примерно 97,5\% выборки, поскольку по оценкам ${\beta }_i\sim {\mathcal N}[2,0;1,0]$. Чтобы ознакомиться с примерами, которые делают акцент на интерпретации оценок коэффициентов, обратитесь к работе Ревелта и Трейна (1998).

В литературе по теории отраслевых организаций рассматривается \textbf{агрегирование} по потребителям в моделях аналогичных логит-модели со случайными параметрами, чтобы оценить параметры спроса, используя \textbf{данные по рынку в целом.} См., например, Берри (1994), Нево (2001) и также Алленби и Росси (1991).

\subsection{Оценка логит-модели со случайными параметрами}

В модели линейной регрессии со случайными параметрами оценивание методом наименьших квадратов дает оценку среднего значения параметра $\beta $, которая является состоятельной, хотя и неэффективной. Однако, оценки в нелинейных моделях, при получении которых не был учтен случайный характер параметров, будут несостоятельными. Таким образом, стандартный ММП, примененный для условной логит-модели, даст несостоятельные оценки параметров, если процесс порождающий данные описывается \eqref{GrindEQ__15_39_} и \eqref{GrindEQ__15_40_}. Вместо этого, оценка ММП должна явно учитывать стохастичность ${\beta }_i$.

Если ${\beta }_i$ известно, тогда единственным источником случайности становится ${\varepsilon }_{ij}$, и получается условная логит-модель с вероятностью $p_{ij}={e^{x'_{ij}{\beta }_i}}/{\sum^m_{l=1}{e^{x'_{il}{\beta }_i}}}$. Поскольку фактически ${\beta }_i$ является случайной величиной, эту случайность необходимо учесть. Отсюда получаем

\begin{equation} \label{GrindEQ__15_41_} 
p_{ij}={\Pr  \left[y_i=j\right]\ }=
\int{\frac{e^{x'_{ij}{\beta }_i}}{\sum^m_{l=1}{e^{x'_{il}{\beta }_i}}}\phi({\beta }_i|\beta ,\ {\Sigma }_{\beta })\,d{\beta }_i}, 
\end{equation} 

где интеграл является многомерным, а $\phi({\beta }_i|\beta ,\ {\Sigma }_{\beta })$ обозначает плотность распределения многомерного нормального распределения для параметра ${\beta }_i$ со средним $\beta $ и дисперсией ${\Sigma }_{\beta }.$

Оценка ММП максимизирует функцию ${\ln  L_N=\sum^N_{i=1}{\sum^m_{j=1}{y_{ij}{\ln  p_{ij}\ }}}\ }$ по $\beta $ и ${\Sigma }_{\beta }.$ Сложность заключается в том, что интеграл, количество измерений которого определяется количеством случайных компонент ${\beta }_i$  с неравной нулю дисперсией, не имеет аналитического решения. Следовательно, его необходимо оценить с помощью симуляционных методов.

Одним из подходов является прямое симуляционное приближение $p_{ij}$  (см. параграф 12.4.1). Согласно этому подходу интеграл в выражении \eqref{GrindEQ__15_41_} замещается средним по $S$ оценкам подынтегральной функции, полученным для некоторых случайных ${\beta }_i$, имеющих распределение ${\mathcal N}[\beta , \Sigma_{\beta }]$. Тогда \textbf{оценка симуляционного  правдоподобия } максимизирует функцию

\begin{equation} \label{GrindEQ__15_42_} {\ln  {\hat{L}}_N(\beta ,\ {\Sigma }_{\beta })\ }=\sum^N_{i=1}{\sum^m_{j=1}{y_{ij}{\ln  \left[\frac{1}{S}\sum^S_{s=1}{\frac{e^{x'_{ij}{\beta }^{\left(s\right)}_i}}{\sum^m_{l=1}{e^{x'_{il}{\beta }^{\left(s\right)}_i}}}}\right]\ }}}, \end{equation} 

где ${\beta }^{\left(s\right)}_i,\ s=1,\dots ,S$ -- случайные величины, сгенерированные согласно  плотности $\phi\left({\beta }_i;\ \beta ,\ {\Sigma }_{\beta }\right).$ Поскольку $\beta $ и ${\Sigma }_{\beta }$ являются неизвестными, сумма из \eqref{GrindEQ__15_42_} предполагает итерационную процедуру получения оценок, в которой ${\beta }^{(r)}$ и ${\Sigma }^{(r)}_{\beta }$ оцениваются на $r$-ой итерации. Состоятельность оценок требует, чтобы  $S\to \infty $, также как и $N\to \infty $ и $\sqrt{N}/S\to \infty $ (см. параграф 12.4.3). Для ускорения расчетов  используют ряды Гальтона (см. параграф 12.7.4) и альтернативные симуляционные методы.

Альтернативные методы оценки используют Байесовские методы с относительно плоским априорным распределением. Трейн (2001, 2003) определяет иерархичное априорное распределение с $\beta \sim {\mathcal N}[{\beta }^*,\ {\Omega }^*]$, где ${\Omega }^*$предполагается очень большим, а ${\Sigma }_{\beta }$ имеет обратное распределение Уишарта со $K={\dim  \left[\beta \right]\ }$ степенями свободы и параметром масштаба $I_K$. Вместо того, чтоб использовать апостериорное распределениетолько для $\beta $ и ${\Sigma }_{\beta }$ вычислительно быстрее дополнительно включить параметр ${\beta }_i,\ i=1,\dots ,N.$ Тогда (1) условным апостериорным распределением для$\beta |{\Sigma }_{\beta },\ {\beta }_i$ будет нормальное, (2) условнымапостериорным распределением для ${\Sigma }_{\beta }|\beta ,\ {\beta }_i$ будет обратное распределение Уишарта, и (3) условным апостериорнымраспределениемдля${\beta }_i|{\Sigma }_{\beta }$будет $\beta $, который пропорционален подынтегральному выражению из \eqref{GrindEQ__15_41_}. Учитывая эти выводы, оценка условного апостериорного распределения может быть получена с использованием вариаций семплирования Гиббса (см. параграф 13.5.2) с осложнениями, которые приводят к необходимости использования одной итерации алгоритма Метрополиса-Гастингса для третьего апостериорного распределения (см. параграф 13.5.4), поскольку полный набор условных выражений не доступен. На практике на расчеты по этому методу затрачивается время, аналогичное ИММП, и, при условии относительно плоского априорного распределения, будут получены оценки параметров модели и стандартные ошибки, которые отклоняются в общем случае не более чем на 10\% от значений оценок ИММП.

\subsection{Обобщенная модель случайной полезности}

Желательно использовать более гибкие модели, чем мультиномиальная логит-модель. В связи с этим в настоящее время наблюдается большой интерес к логит-модели со случайными параметрами. МакФадден и Трейн (2004) показали, что любая модель случайной полезности может быть сколь угодно хорошо приближена смешанной логит-моделью, хотя такой результат требует аккуратного выбор регрессоров и  распределения смеси.

Нет причин ограничивать применение моделей со случайными параметрами только к мультиномиальной логит-модели. Например, она может быть расширена до вложенной логит-модели. Более того, в модель могут быть включены дополнительные источники случайности, в особенности скрытые классы и скрытые переменные.

Начнем описание таких расширений с модели ARUM \eqref{GrindEQ__15_22_}. В этой модели полезность, получаемая индивидом $i$ от $j$-ой альтернативы задается как $U_{ij}=V_{ij}\left(x_i,\ \beta \right)+{\varepsilon }_{ij}$, где $x_i$ обозначает наблюдаемые данные, $\beta $ обозначает неизвестный параметр, а ${\varepsilon }_{ij}$ обозначает ошибки, независимые по $i$, но возможно коррелирующие по $j$. Предположение о том, что ${\varepsilon }_{ij}$ имеет распределение \eqref{GrindEQ__15_23_} приводит к аналитическому решению для вероятностей выбора обозначаемых

\[p_{ij}=F_j\left(V_i\left(x_i,\beta \right),\ {\Theta }_{\varepsilon }\right),\] 

где $V_i\left(x_i,\beta \right)=[V_{i1}\left(x_i,\beta \right),\dots ,V_{im}\left(x_i,\beta \right)]$, а ${\Theta }_{\varepsilon }$ обозначает неизвестные параметры распределения ${\varepsilon }_i=\left({\varepsilon }_{i1},\dots ,{\varepsilon }_{im}\right).$  Такое аналитическое решение может быть получено, если ${\varepsilon }_i$ имеет ОРЭЗ, в частности можно получить мультиномиальную и вложенную логит-моделям.

Более общая модель включает дополнительную случайность в модель, описанную выше. Во-первых, ранее детерминированная часть полезности принимает вид $V_{ij}=V_{ij}\left(x_i,\ {\xi }_i,\ \beta \right)$. Тогда, допустив, что ${\varepsilon }_i$ такие, что условные по ${\xi }_i$ вероятности можно записать в явном виде, получаем для безусловных вероятностей

\begin{equation} \label{GrindEQ__15_43_} p_{ij}=\int{F_j\left(V_i\left(x_i,\ {\xi }_i,\ \beta \right),{\theta }_{\varepsilon }\right)f\left({\xi }_i\mathrel{\left|\vphantom{{\xi }_i {\theta }_{\xi }}\right.\kern-\nulldelimiterspace}{\theta }_{\xi }\right)d{\xi }_i}, \end{equation} 

где $f\left(\xi \mid \theta_{\xi}\right)$ обозначает плотность распределения $\xi $. Логит-модель со случайными параметрами является примером, для которого $V_{ij}=x'_{ij}\beta +x'_{ij}{\xi }_i$, где ${\xi }_i$ имеет распределение ${\mathcal N}[0,\ \Sigma ]$, и модель можно описать с помощью случайных параметров. 
Тем не менее, ${\xi }_i$ также может быть введена в модель как дополнительный элемент случайности или как скрытая переменная. Во-вторых, можно предположить, что индивиды принадлежат к одному из $C$ скрытых классов. См. модель дюрации в разделе 18.5 и  работу Свайта (2003) для примеров применения ОРЭЗ в случае скрытых классов или моделей смеси. Если $\beta $ и ${\theta }_{\varepsilon }$ изменяются в зависимости от класса, то \eqref{GrindEQ__15_43_}  принимает безусловный вид

\begin{equation} \label{GrindEQ__15_44_} p_{ij}=\sum^C_{c=1}{\left[\int{F_j\left(V_j\left(x_i,\ {\xi }_i,\ {\beta }^c\right),{\theta }^c_{\varepsilon }\right)f\left({\xi }_i\mathrel{\left|\vphantom{{\xi }_i {\theta }_{\xi }}\right.\kern-\nulldelimiterspace}{\theta }_{\xi }\right)d{\xi }_i}\right]}{\pi }_c, \end{equation} 

где ${\pi }_c$ обозначает вероятность принадлежности индивида к $c$-му классу, и обычно $C=2$ или $C=3$. Тогда оценка симуляционного ММП максимизирует функцию

\[{\ln  {\hat{L}}_N(\beta ,\ {\Sigma }_{\beta })=\sum^N_{i=1}{\sum^m_{i=1}{y_{ij}{\ln  \left[\frac{1}{S}\sum^S_{s=1}{\sum^C_{c=1}{F_j(V_i(}x_i,{\xi }^s_i,{\beta }^c),\ {\theta }^c_{\varepsilon }){\pi }_c}\right],\ }}}\ }\] 

где ${\xi }^s_i$ обозначает $s$-ое сгенерированное значение из $f\left({\xi }^s_i \mid {\theta }_{\xi }\right)$. Камакура и Уэдел (2004) оценивают мультиномиальную модель конечной смеси с помощью Байесовских методов.



Рисунок 15.1. Обобщенная модель случайной полезности

Explanatory variables --- Объясняющие переменные
Disturbances --- Случайные ошибки
Indicators --- Индикаторы
Latent classes --- Скрытые классы
Latent variables --- Скрытые переменные
Utilities --- Полезности
Revealed preference indicator --- Индикатор выявленного предпочтения
Observable variable --- Наблюдаемая переменная
Unobservable variable --- Ненаблюдаемая переменная
Structural relationship --- Структурные связи
Disturbances --- Случайные ошибки
Measurement relationship --- Измерение
Stated preference indicators --- Индикаторы заявленных предпочтений



Уолкер и Бен-Акива назвали такую модель \textbf{обобщенной моделью случайной полезности}. Они ссылаются на многие статьи с подобными расширениями, рассматривают использование \textbf{данных о задаявленных предпочтениях}, чтобы дополнить данные о выявленных предпочтениях, а также приводят много эмпирических примеров. На рисунке 15.1, взятом из работы Уолкера и Бен-Акивы, обобщены различные расширения модели случайной полезности.

Литература по мультиномиальному моделированию активно исследует сильно параметризованные модели, которые включают модели со случайными параметрами, скрытыми переменными и скрытые параметры, а также комбинируют данные из более чем одного источника. Эти методы применимы к любому типу пространственных данных, а не только к случаям дискретных исходов.

\section{Мультиномиальная пробит-модель}

Альтернативным очевидным способом включения корреляции между альтернативами в ненаблюдаемой компоненте является применение нормально распределенных ошибок. Однако оценка ММП затрудняется, поскольку в наиболее общем случае необходимо взять $(m-1)$-мерный интеграл.

\subsection{Мультиномиальная пробит-модель}

\textbf{Мультиномиальная пробит-модель } --- это мультиномиальная модель с $m$ альтернативами для выбора, где полезность $j$-ой альтернативы задается как

\begin{equation} \label{GrindEQ__15_45_} U_j=V_j+{\varepsilon }_j,\ \ j=1,2,\dots ,m, \end{equation} 

а ошибки имеют совместное нормальное распределение с параметрами

\begin{equation} \label{GrindEQ__15_46_} \varepsilon \sim {\mathcal N}\left[0,\ \Sigma \right], \end{equation} 

где $\varepsilon =[{\varepsilon }_1,\dots ,{\varepsilon }_m]'$ --- $m\times 1$ вектор. Как правило, $V_j=x'_j\beta $ или $V_j=x'{\beta }_j$.

Различные мультиномиальные пробит-модели возникают при использовании различных спецификаций ковариационной матрицы $\Sigma $. Для некоторых из внедиагональных элементов задаются ненулевые значения, чтобы разрешить корреляцию между ошибками, однако необходимо наложить некоторые ограничения на матрицу $\Sigma $. Обратите внимание, что, если между ошибками нет корреляции, то мультиномиальная пробит-модель по-прежнему не имеет аналитического решения для вероятностей, и тогда проще предположить, что ошибки имеют  распределение экстремальных значений и использовать условную или мультиномиальную логит-модель.

Ограничения на матрицу $\Sigma $ необходимы, чтобы обеспечить \textbf{идентифицируемость}. Из \eqref{GrindEQ__15_23_} очевидно, что для любой модели ARUM выбор определяется разницей в полезностях или ошибках. Тогда рассмотрим разницу $U_j-U_1$ между полезностью альтернативы $j$ и полезностью  альтернативы 1, которая выбрана в качестве базовой альтернативы. Банч (1991)  продемонстрировал, что все кроме одного параметры ковариационной матрицы ошибок ${\varepsilon }_j-{\varepsilon }_1$ идентифицируемы (см. обсуждение в конце параграфа 15.5.1). Одним из способов достижения идентификации является нормализация, скажем, ${\varepsilon }_1=0$, и фиксирование одного из элементов ковариационной матрицы. Например, если $m=2$, пусть ${\varepsilon }_1=0$, тогда ${\sigma }_{11}=0$ и ${\sigma }_{12}=0$, и дополнительно ограничим ${\sigma }_{22}=1$. Тогда ${\varepsilon }_2-{\varepsilon }_1={\varepsilon }_2\sim {\mathcal N}[0,1]$, что приводит к пробит-модели бинарного выбора.

Дополнительные ограничения на матрицу $\Sigma $ или параметр $\beta $ могут понадобиться для успешного приложения модели для конкретного практического случая. Кин (1992) продемонстрировал, что даже если были сделаны предположения о ковариации ошибок, чтобы обеспечить идентификацию матрицы, на практике параметры мультиномиальной пробит-модели могут быть очень неточно оценены в моделях с регрессорами, которые не зависят от альтернатив. В таком случае необходимо наложить дополнительные ограничения на мультиномиальную пробит-модель. Такая неточность оценки качественно аналогична сильной мультиколлинеарности независимых переменных в линейной регрессии. Кин обнаружил, что ограничение исключения на регрессоры (с одним исключением для каждого индекса полезности) работает хорошо.  В качестве альтернативы, которая используется гораздо чаще, дополнительные ограничения могут налагаться на параметры ковариации.

Популярной экономичной моделью для ошибок является \textbf{факторная модель}

\[{\varepsilon }_j=v_j+\sum^L_{l=1}{c_{jl}{\xi }_l,\ \ j=1,2,\dots ,m,}\] 

где $v_j$ и ${\xi }_1,\dots ,{\xi }_L$ -- это  одинаково распределенные случайные величины, имеющие стандартное нормальное распределение, а $c_{jl}$ -- это веса, которые называются \textbf{коэффициентами нагрузки}, и их необходимо оценить. Эта модель позволяет значительно снизить количество параметров ковариации с $m(m+1)/2$ до $L$ и работать уже с $(L+1)$-мерным интегралом. Численные методы, обычно численное интегрирование, могут быть использованы при малых значениях $L$, в то время как необходимо использовать симуляционные методы при больших значениях $L$. Для панельных данных модель случайных эффектов (см. параграф 21.2.1) может быть рассмотрена как факторная модель с ошибкой $u_{it}={\alpha }_i+\epsilon_{it}$. Факторная модель может быть особенно удобной в случае использования пробит-модели для панельных данных.

\subsection{Оценка мультиномиальной пробит-модели}

Предпочтительно оценивать параметры регрессии и дисперсии ошибок с помощью ММП с логарифмической функцией правдоподобия, приведенной в параграфе 15.3.2. Сложность заключается в том, что не существует аналитической формы выражения для вероятностей выбора.

Для пробит-модели с тремя альтернативами

\[p_1={\Pr  \left[y=1\right]\ }=\int^{-{\tilde{V}}_{31}}_{-\infty }{\int^{{\tilde{V}}_{32}}_{-\infty }{f\left({\widetilde{\varepsilon }}_{21},{\widetilde{\varepsilon }}_{31}\right)d{\widetilde{\varepsilon }}_{21}d{\widetilde{\varepsilon }}_{31}}}\] 

(см. выражение (15.24)), где $f\left({\widetilde{\varepsilon }}_{21,}{\widetilde{\varepsilon }}_{31}\right)$ имеет двумерное нормальное распределение с двумя свободными параметрами ковариации, а ${\tilde{V}}_{21}$ и ${\tilde{V}}_{31}$ зависят от регрессоров и параметра $\beta $. Значение этого двумерного интеграла может быть быстро найдено численными методами. В более общем случае, однако, для модели с $m$ альтернативами требуется вычислить $(m-1)$-мерный интеграл. Трехмерный интеграл является пределом возможностей численных методов, ограничивая применение численных методов интегрирования мультиномиальной пробит-моделью с четырьмя альтернативами.

Для моделей большей размерности альтернативой является использование симуляционных методов. Для простоты обратимся к мультиномиальной модели с тремя альтернативами. Одной из возможностей выступает использование частотных симуляций, когда $p_1$ апроксимируется долей пар $\left({\widetilde{\varepsilon }}_{21,}{\widetilde{\varepsilon }}_{31}\right)$, которые меньше $(-{\tilde{V}}_{21},-{\tilde{V}}_{31})$. Как описано в параграфе 12.7.1 этот метод не является гладким и может быть очень неэффективен (см. параграф 12.7.2). Более того, при текущей постановке задачи, при использовании этого метода возможно получение граничных значений для вероятности ${\hat{p}}_1=0\text{ или } 1$. В общем случае лучше использовать метод сэмплирования по важности, подробно описанный в параграфе 12.7.2. В случае интегрирования многомерного интеграла по методу Монте Карло очень популярной разновидностью метода сэмплирования по важности является GHK-симулятор, названный в честь Гевеке (1992), Хадживасилу и МакФаддена (1994) и Кина (1994). Благодаря этому методу многомерная функция плотности нормального распределения усекается. В сопоставлении с частотной симуляцией этот метод является гладким, требует гораздо меньше испытаний для альтернатив с низкой вероятностью выбора и вряд ли будет иметь проблемы с граничными значениями. В работе Трейна (2003) приведено подробное описание этого метода.

В предшествующем обсуждении рассматривалась оценка вероятностей выбора в мультиномиальной пробит-модели, при известных параметре $\beta $ и матрице $\Sigma $. Фактически же $\beta $ и $\Sigma $ требуется оценить. Оценка, полученная \textbf{методом симуляционного правдоподобия} максимизирует функцию

\[{\ln  {\hat{L}}_N(\beta ,\Sigma )=\sum^N_{i=1}{\sum^m_{j=1}{y_{ij}{\ln  {\hat{p}}_{ij}\ }}}\ },\] 

где ${\hat{p}}_{ij}$ получена с использованием GHK-симулятора. Для состоятельности оценки требуется, чтобы количество испытаний  $S\to \infty $, также как и $N\to \infty $. Этот метод очень требователен к ресурсам. На $r$-ом этапе итерационного процесса (см. главу 10) получаются оценки ${\widehat{\beta }}^{(r)}$ и ${\widehat{\Sigma }}^{(r)}$, и требуется произвести перерасчет ${\hat{p}}_{ij}$, что требует $S$ испытаний для каждого из $N$ индивидов.

Альтернативной процедурой получения оценок выступает \textbf{симуляционный метод моментов} (см. раздел 12.5). Из выражения \eqref{GrindEQ__15_8_} состоятельная оценка, полученная методом моментов, является решением $\sum^N_{i=1}{\sum^m_{j=1}{\left(y_{ij}-p_{ij}\right)z_i=0}}$, где, например, $z_i=x_i$. Соответствующие оценки $\beta $ и $\Sigma $, полученные симуляционным методом моментов, являются решением уравнений

\[\sum^N_{i=1}{\sum^m_{j=1}{\left(y_{ij}-{\hat{p}}_{ij}\right)z_i=0}},\] 

где ${\hat{p}}_{ij}$ получена с помощью несмещенной оценки. Тогда $\left(y_{ij}-{\hat{p}}_{ij}\right)z_i$ является несмещенным для $\left(y_{ij}-p_{ij}\right)z_i$. Таким образом, получение состоятельной оценки возможно даже, если $S=1$. Этот момент позволяет значительно сократить расчеты. Тем не менее, присутствует потеря эффективности для малых $S$, и даже для больших $S$ симуляционный метод моментов менее эффективен, чем симуляционный ММП, поскольку для этого примера метод моментов менее эффективен, чем метод максимального правдоподобия. Реже используемым методом, который сопоставим по эффективности с симуляционным ММП, является симуляционный метод скоринга (method of simulated scores) (см. работы Хадживасилу и МакФаддена, 1998).

В альтернативных методах оценки используется Байесовские алгоритмы. В отличие от логит-модели со случайными параметрами, для этих методов отсутствует аналитическое решение для вероятностей, которые необходимо получить из полезностей. Скрытые полезности $U_i=(U_{1i},\dots ,U_{ji})$ вводятся как вспомогательные переменные, а также используется подход пополнения данных (см. раздел 13.7). Обозначив $U=(U_1,\dots ,U_N)$ и $y=(y_1,\dots ,y_N)$, получаем цикл сэмплирования Гиббса с шагами (1) условное апостериорное распределение для $\beta |y,\ U,\ \Sigma $, (2) условное апостериорное распределение для $\Sigma |y,\beta ,U$, (3) апостериорное распределением для $U_i|y,\ \beta ,\ \Sigma $. Альберт и Чиб (1993) детально рассматривают как мультиномиальные модели с неупорядоченными исходами, так и с упорядоченными исходами. МакКуллох и Росси (1994) подробно описывают приложения мультиномиальной пробит-модели. Чиб (2001) обсуждает сложности наложения ограничений на матрицу $\Sigma $, необходимые для идентификации (см. параграф 15.8.1).

\subsection{Обсуждение}

Как для мультиномиальной пробит-модели, так и для логит-модели со случайными параметрами отсутствует аналитическое решение для $p_{ij}$. Тем не менее, для логит-модели со случайными параметрами по крайней мере существует аналитическое решение для вероятностей при фиксированных ${\beta }_i$, и единственной проблемой будет интегрирование по ${\beta }_i$. Для мультиномиальной пробит-модели, которая предшествует логит-модели со случайными параметрами, такие условные вероятности не могут быть получены, и получение приближенного значения $p_{ij}$ осложняется, особенно, если значение $p_{ij}$ близко к нулю или единице. По-видимому, будет проще воспользоваться вложенной логит-моделью, логит-моделью со случайными параметрами или моделями смеси, чем мультиномиальной пробит-моделью.

\section{Упорядоченные, последовательные и ранжированные исходы}

В настоящем разделе будут представлены модели более структурированные, чем модели с неупорядоченными исходами, такие как модели с естественным упорядочиванием альтернатив или последовательностью принятия решений. Анализ таких моделей прост, поскольку применяются хорошо исследованные модели, с различными спецификациями для вероятности $p_{ij}$, а оценка производится ММП, основанном на \eqref{GrindEQ__15_4_}.

\subsection{Мультиномиальные модели с упорядоченными исходами}

Допустим, что альтернативы упорядочены естественным образом. Например, состояние собственного здоровья может быть оценено как прекрасное, хорошее, плохое или ужасное. Для таких данных может быть использована одна из моделей неупорядоченных исходов, однако, более экономной и рациональной будет модель, в которой учитывается это упорядочивание.

Отправной точкой будет индексная модель с одной скрытой переменной 

\begin{equation} \label{GrindEQ__15_47_} y^*_i=x'_i\beta +u_i, \end{equation} 

где $x$ здесь не включает константу, в отличие от параграфа 14.4.1. По мере того как $y^*$ пересекает серию возрастающих неизвестных порогов, мы переходим от одной альтернативы к другой. Например, для очень маленьких значений $y^*$ состояние здоровья оценивается как ужасное, для $y^*>{\alpha }_1$ состояние здоровья повышается до плохого, при $y^*>{\alpha }_2$ оно повышается дальше до хорошего и т.д.

В общем случае для \textbf{модели с } ${\mathbf m}$ \textbf{ упорядоченными альтернативами}определим

\begin{equation} \label{GrindEQ__15_48_} y_i=j,\ \text{ если } {\alpha }_{j-1}<y^*_i\le {\alpha }_j, \end{equation} 

где ${\alpha }_0=-\infty $ и $a_m=\infty $. Тогда

\[{\Pr  \left[y_i=j\right]\ }={\Pr  \left[{\alpha }_{j-1}<y^*_i\le {\alpha }_j\right]\ }\] 

\[={\Pr  \left[{\alpha }_{j-1}<x'_i\beta +u_i\le {\alpha }_j\right]\ }\] 

\begin{equation} \label{GrindEQ__15_49_} ={\Pr  \left[{\alpha }_{j-1}-x'_i\beta <u_i\le {\alpha }_j-x'_i\beta \right]\ } \end{equation} 

\[=F\left({\alpha }_j-x'_i\beta \right)-F\left({\alpha }_{j-1}-x'_i\beta \right),\] 

где $F$ --- функция распределения $u_i$. Параметры регрессии $\beta $ и $(m-1)$ значений порогов ${\alpha }_1,\dots ,{\alpha }_{m-1}$ получаются путем максимизации логарифмической функции правдоподобия \eqref{GrindEQ__15_5_} с $p_{ij}$, определенными в выражении (15.49). Для \textbf{логит-модели с упорядоченными исходами} $u$ имеет логистическое распределение с функцией $F\left(z\right)={e^z}/{(1+e^z)}$. Для \textbf{пробит-модели с упорядоченными исходами} $u$ имеет стандартное нормальное распределение, а $F(\cdot )$ -- функция распределения стандартной нормальной величины. Пусть $K$ обозначает количество независимых переменных за исключением константы, тогда модель с $m$ упорядоченными альтернативами будет иметь $K+m-1$ параметр, в то время как мультиномиальная логит-модель $(m-1)(K+1)$ параметр.

Знак параметра регрессии $\beta $ может быть непосредственно интерпретирован как индикатор того, будет ли скрытая переменная $y^*$ возрастать вместе с независимой переменной. Предельные эффекты на вероятности имеют вид

\[\frac{\partial {\rm Pr}[y_i=j]}{\partial x_i}=\left\{F'\left({\alpha }_{j-1}-x'_i\beta \right)-F'\left({\alpha }_j-x'_i\beta \right)\right\}\beta ,\] 

где $F'$ обозначает производную функцию $F$. Знак выражения в скобках может быть положительным или отрицательным.

Эта модель также может применяться для анализа счетных данных, для которых характерно наличие всего нескольких значений. Кэмерон и Триведи (1992) применили пробит-модель с упорядоченными исходами для моделирования количества медицинских консультаций. Хаусман, Ло и МакКинли (1992) применяли пробит-модель с упорядоченными исходами к данным об изменении количества, которое может быть и отрицательным, и дополнительно учитывали гетероскедастичность ошибок $u_i$.

\subsection{Мультиномиальные модели с последовательным выбором}

В некоторых ситуациях решения принимаются  последовательно. Например, некто может принимать решение о том, продолжать ли образование после школы или нет. Если принято решение не продолжать образование, тогда $y=1$. Если $y\ne 1$, то принимается решение о том, получать ли среднее специальное $\left(y=2\right)$ или высшее образование $\left(y=3\right)$. Учитывая эту последовательность, вероятности выбора могут быть просто получены. Например, смоделируем первую ситуацию выбора с помощью пробит-модели, а затем вторую ситуацию, если она имеет место, с помощью пробит-модели. Тогда ${\Pr  \left[y=1\right]\ }=\Phi (x'_1{\beta }_1)$ и ${\Pr  \left[y=2\mathrel{\left|\vphantom{y=2 y\ne 1}\right.\kern-\nulldelimiterspace}y\ne 1\right]\ }=\Phi (x'_2{\beta }_2)$. Безусловная вероятность имеет вид

\[{\Pr  \left[y=2\right]\ }={\Pr  \left[y=2\mathrel{\left|\vphantom{y=2 y\ne 1}\right.\kern-\nulldelimiterspace}y\ne 1\right]\ }\times {\Pr  \left[y\ne 1\right]\ }=\Phi \left(x'_2{\beta }_2\right)\left(1-\Phi \left(x'_1{\beta }_1\right)\right).\] 

Параметры ${\beta }_1$ и ${\beta }_2$ могут быть найдены путем максимизации логарифмической функции правдоподобия \eqref{GrindEQ__15_5_}, где $p_{1i}=\Phi \left(x'_{1i}{\beta }_1\right),\ p_{2i}$ дано в предыдущем выражении, а $p_{31}=1-p_{1i}-p_{21}$.

Этот подход базируется на правильном определении последовательности принятия решений. Лучше  для описанной ситуации  может подходить вложенная логит-модель с тремя альтернативами, в которой ошибки в полезностях от среднего специального и высшего образования коррелируют между собой и являются независимыми по отношению к ошибке в значении полезности от решения отказаться от образования. Эти модели можно сравнить, используя методы, основанные на  функции правдоподобия и описанные в разделе 8.5. 

\subsection{Модели для ранжированных данных}

В рассмотренных моделях предполагалось, что альтернативы являются взаимно исключающими, и только одна из них может быть выбрана. В более общем смысле, альтернативы могут мыть ранжированы, особенно в случае данных о заявленных предпочтениях. Например, могут быть известна наилучшая и вторая лучшая альтернативы. 

\textbf{Логит-модель с ранжированными исходами} может быть очень просто оценена (см. работу Беггса, Карделла и Хаусмана, 1981). Рассмотрим условную логит-модель с четырьмя альтернативами, когда наилучшей является альтернатива 2, а во второй лучшей --- альтернатива 3. Альтернатива 2 выбрана из всех четырех альтернатив, а затем альтернатива 3 выбрана из оставшихся альтернатив 1, 3 и 4. Совместная вероятность выбора тогда имеет вид 

\[\frac{e^{x'_{i2}\beta }}{e^{x'_{i1}\beta }+e^{x'_{i2}\beta }+e^{x'_{i3}\beta }+e^{x'_{i4}\beta }}\times \frac{e^{x'_{i3}\beta }}{e^{x'_{i1}\beta }+e^{x'_{i3}\beta }+e^{x'_{i4}\beta }}.\] 

Оценка производится методом максимального правдоподобия, аналогичный вид имеют и  остальные 11 совместных вероятностей.

Для мультиномиальной пробит-модели не существует подобной спецификации. Хаджиивасилу и Рууд  (1994) в своей работе описали получение совместных вероятностей симуляционными методами. Они использовали \textbf{пробит-модель с ранжированными исходами}, чтобы проиллюстрировать разнообразие симуляционных методов оценивания.

\section{Ситуации многомерного дискретного выбора}

Описанные выше модели, кроме моделей с ранжированными исходами, являются моделями для одной дискретной зависимой переменной, которая принимает одно из $m$ взаимно исключающих значений. В настоящем разделе мы рассмотрим модели, в которых присутствуют более одного дискретного исхода. Логарифмическая функция правдоподобия аналогична \eqref{GrindEQ__15_5_} для мультиномиальных моделей, а разным моделям соответствуют разные функциональные формы для вероятностей выбора. Эти вероятности могут учитывать корреляцию между различными выборами и их одновременность.

\subsection{ Двумерная дискретная переменная}

Для простоты рассмотрим двумерные дискретные данные $(y_{1i},y_{2i})$. Например, в совместной модели предложения труда и фертильности зависимая переменная $(y_{1i},y_{2i})$ для индивида $i$ может принимать значения $y_{1i}=2$, если индивид работает и $y_{1i}=1$, если не работает, а также $y_{2i}=2$, если имеет детей, и $y_{2i}=1$, если не имеет детей.

В более общем смысле, $y_1$ может принимать значения $1,\dots ,m_1$, а $y_2$ может принимать значения $1,\dots ,m_2$. Для индивида $i$ определим

\begin{equation} \label{GrindEQ__15_50_} p_{ijk}={\Pr  \left[y_{1i}=j,\ y_{2i}=k\right]\ },\ \ j=1,\dots ,m_1,\ \ k=1,\dots ,m_2. \end{equation} 

Обратите внимание, что $p_{ijk}$ определяет вероятности взаимно исключающих событий, и $\sum_j{\sum_k{p_{ijk}=1}}$. Определим $m_1\times m_2$ соответствующих бинарных индикаторов равных $y_{jk}=1$, если $(y_1=j,\ y_2=k)$ и $y_{jk}=0$ в обратном случае. Тогда совместная функция плотности для $i$-го наблюдения будет иметь вид

\[f\left(y_{1i},\ y_{2i}\right)=\prod^{m_1}_{k=1}{\prod^{m_2}_{j=1}{p^{y_{ijk}}_{ijk}}}.\] 

Тогда логарифмическая функция правдоподобия имеет вид $\sum^N_{i=1}{\sum^{m_1}_{k=1}{\sum^{m_2}_{j=1}{y_{ijk}{\ln  p_{ijk}\ }}}}$, а оценка производится методом максимального правдоподобия, как в параграфе 15.4.2.

Основной разницей между многомерными и мультиномиальными моделями является разница в спецификации функциональной формы для вероятностей.

В простейшем случае две дискретные зависимые переменные независимы между собой, и $p_{ijk}={\Pr  \left[y_{1i}=j\right]\ }\times {\rm Pr}[y_{2i}=k]$. Тогда $y_1$ и $y_2$ могут быть смоделированы с помощью независимых мультиномиальных моделей.

Если, наоборот, две переменные рассматриваются как взаимосвязанные, простым решением будет применение мультиномиальной логит-модели для вероятностей $p_{ijk}$. Тогда двумерный исход $(y_1,y_2)$ естественным образом рассматривается как $m_1\times m_2$ одномерный исход. Например, в описанном примере с предложением труда и фертильностью одним из четырех исходов тогда будет наличие работы и детей.

В следующем параграфе рассмотрим модели для этих двух случаев.

\subsection{Двумерная пробит-модель}

Двумерная пробит-модель --- это совместная модель для двух бинарных исходов, которая обобщает индексную функциональную модель (см. параграф 14.4.1) с одной скрытой переменной на случай двух скрытых переменных, которые могут коррелировать между собой.

Зададим ненаблюдаемые скрытые переменные

\begin{equation} \label{GrindEQ__15_51_} y^*_1=x'_1{\beta }_1+{\varepsilon }_1, \end{equation} 

\[y^*_2=x'_2{\beta }_2+{\varepsilon }_2,\] 

где ${\varepsilon }_1$ и ${\varepsilon }_2$ имеют совместное нормальное распределение со средним равным нулю, дисперсией равной единице, и корреляцией равной $\rho $. Тогда \textbf{двумерная пробит-модель } определяет наблюдаемые исходы как

\[y_1=\left\{ \begin{array}{c}
2,\text{ если } y^*_1>0, \\ 
1,\text{ если } y^*_1\le 0, \end{array}
\right.\] 

\[y_2=\left\{ \begin{array}{c}
2,\text{ если } y^*_2>0, \\ 
1,\text{ если } y^*_2\le 0, \end{array}
\right.\] 

где используются обозначения (2, 1), вместо (1, 0), чтобы соответствовать обозначениям, принятым в настоящей главе. Эта модель распадается на две отдельные пробит-модели для $y_1$ и $y_2$, когда корреляция между ошибками $\rho =0$.

Если $\rho \ne 0$, то не существует аналитического решения для вероятностей выбора. Например,

\[{p}_{22}={\Pr  \left[y_1=2,\ y_2=2\right]\ }={\Pr  \left[y^*_1>0,\ y^*_2>0\right]\ }\] 

\[={\Pr  \left[-{\varepsilon }_1<x'_1{\beta }_1,\ -{\varepsilon }_2<x'_2{\beta }_2\right]\ }\] 

\[={\Pr  \left[{\varepsilon }_1<x'_1{\beta }_1,\ {\varepsilon }_2<x'_2{\beta }_2\right]\ }\] 

\[=\int^{x'_1{\beta }_1}_{-\infty }{\int^{x'_2{\beta }_2}_{-\infty }{ \phi \left(z_1,z_2,\rho \right)dz_1dz_2}}\] 

\[=\Phi \left(x'_1{\beta }_1,x'_2{\beta }_2,\rho \right),\] 

где $\phi \left(z_1,z_2,\rho \right)$ и $\Phi \left(z_1,z_2,\rho \right)$, соответственно, функция плотности и функция распределения стандартного нормального распределения $(z_1,z_2)$ со средним равным нулю, единичной дисперсией, корреляцией $\rho $. Здесь четвертое равенство имеет место для двумерного нормального распределения со средним, равным нулю.

Применение алгебраических преобразований для вероятностей других возможных исходов дает 

\[p_{jk}={\Pr  \left[y_1=j,\ y_2=k\right]\ }=\Phi \left(q_1x'_1\beta ,q_2x'_2\beta ,\ \rho \right),\] 

где $q_l=1$, если $y_l=2$, и $q_l=-1$, если $y_l=1$ для $l=1,2$. Это является основой для оценки ММП, подробно описанной у Грина (2003), который также рассматривает расчет предельных эффектов.

Оценивание модели требует расчета двумерного интеграла, которое возможно при применении численных методов. Обобщение до многомерной пробит-модели очевидно, хотя и приведет к сложностям в вычислениях из-за более высоких порядков интеграла. Если все исходы упорядочены, то эта модель может быть обобщена до \textbf{двумерной пробит-модели с упорядоченными исходами}.

Также можно рассмотреть систему одновременных уравнений пробит-модели, которая обобщает \eqref{GrindEQ__15_51_}, допуская эндогенность переменных в правой части уравнений. Например, первое уравнение для $y^*_1$ может включать $y^*_2$ и/или $y_2$ в качестве регрессора, и аналогично $y^*_2$, с введением некоторых ограничений, чтобы обеспечить идентифицируемость модели. Эта модель аналогична системе уравнений тобит-модели, описанной в параграфе 16.8.2.

\section{ Оценка полупараметрическими методами}

В некоторых исследованиях полупараметрические методы оценки распространяются на модели неупорядоченных мультиномиальных данных. Абэ (1999) оценивает условную логит-модель заменяя $x'_{ij}\beta $ в выражении \eqref{GrindEQ__15_10_} на аддитивную составляющую $\sum_p{{\beta }_pf_p(x_{ijp})}$, где $p$ обозначает $p$-ый компонент $x_{ij}$, а функция $f_p(\cdot )$ оценивается по данным. Л.-Ф. Ли (1995) распространил применение метода оценки Кляйна и Спэди (1993) с моделей бинарных исходов на модели мультиномиальных исходов. Полупараметрические методы для мульти-индексных моделей также могут применяться к мультиномиальным моделям с неупорядоченными исходами. Сложность заключается в обеспечении попадания значения прогнозируемых вероятностей в промежуток от нуля до единицы и равенства их суммы единице.

Модели с упорядоченными исходами хорошо поддаются полупараметрическому анализу, поскольку содержат индекс $x'\beta $, который пересекает некоторое количество пороговых значений. Например, в работе Кляйна и Шермана (2002) представлен метод оценки, который является $\sqrt{N}$-состоятельным и асимптотически нормальным как для регрессии, так и для пороговых значений по местонахождению и масштабу, при условии что ошибки независимы от регрессоров. 

\section{Вывод формул для мультиномиальной, условной и вложенной логит-моделей}

Рассмотрим условную и мультиномиальную логит-модели. Мы получим первую и вторую производную логарифмической функции правдоподобия и выражения для эффектов от изменения в значениях регрессоров на вероятности выбора. Затем вложенная логит-модель будет выведена из модели с ошибками ОРЭЗ.

\subsection{Условная логит-модель}

Вероятность в условной логит-модели имеет вид $p_{ij}={e^{x'_{ij}\beta }}/{\sum_l{e^{x'_{il}\beta }}}$. Взяв производную получим

\[\frac{\partial p_{ij}}{\partial \beta }=\frac{e^{x'_{ij}\beta }}{\sum_l{e^{x'_i{\beta }_l}}}x_{ij}-\frac{e^{x'_{ij}\beta }}{{\left(\sum_l{e^{x'_{il}\beta }}\right)}^2}\sum_l{e^{x'_{il}\beta }x_{il}}=p_{ij}x_{ij}-p_{ij}\sum_l{p_{il}x_{il}=}p_{ij}x_{ij}-p_{ij}{\overline{x}}_{i}=p_{ij}\left(x_{ij}-{\overline{x}}_i\right),\] 

где ${\overline{x}}_i=\sum_l{p_{il}x_{il}}$. Тогда 

\[\frac{\partial {\mathcal L}}{\partial \beta }=\sum_i{\sum_j{\frac{y_{ij}}{p_{ij}}}\frac{\partial p_{ij}}{\partial \beta }=\sum_i{\sum_j{\frac{y_{ij}}{p_{ij}}p_{ij}(x_{ij}-{\overline{x}}_i)=\sum_i{\sum_j{y_{ij}\left(x_{ij}-{\overline{x}}_i\right).}}}}}\] 

Отсюда следует, что 

\[\frac{{\partial }^2{\mathcal L}}{\partial \beta \partial \beta' }=-\sum_i{\sum_j{y_{ij}\frac{\partial {\overline{x}}_i}{\partial {\beta }'}}}\] 

\[=-\sum_i{\sum_j{y_{ij}\frac{\partial \sum_l{p_{il}x_{il}}}{\partial {\beta }'}}}\] 

\[=-\sum_i{\sum_j{y_{ij}\sum_l{p_{il}\left(x_{il}-{\overline{x}}_i\right)x'_{il}}}}\] 

\[=\sum_i{\sum_j{p_{ij}\left(x_{ij}-{\overline{x}}_i\right)x'_{ij}}}\] 

\[=\sum_i{\sum_j{p_{ij}\left(x_{ij}-{\overline{x}}_i\right)\left(x_{ij}-{\overline{x}}_i\right)'}},\] 

т.е. \eqref{GrindEQ__15_15_}. В предпоследнем равенстве используется тот факт, что $y_{ij}$ равняется единице ровно для одной альтернативы и нулю в других ситуациях так, что $\sum_j{y_{ij}\sum_l{a_{il}}}=\sum_j{\sum_l{y_{ij}a_{il}}}=\sum_j{a_{ij}}$, а в последнем равенстве используется равенство $\sum_j{p_{ij}\left(x_{ij}-{\overline{x}}_i\right){\overline{x}}'_i}=\sum_j{\left(p_{ij}x_{ij}-{p_{ij}\overline{x}}_i\right){\overline{x}}'_i}=\sum_j{({\overline{x}}_i-{p_{ij}\overline{x}}_i)}{\overline{x}}'_i = 0$, поскольку $\sum_j{p_{ij}=1}$.

Теперь рассмотрим эффект от изменения значения независимой переменной. Для условной логит-модели

\[\frac{\partial p_{ij}}{\partial x_{ij}}=\frac{e^{x'_{ij}\beta }}{\sum_l{e^{x'_{il}\beta }}}\beta -\frac{e^{x'_{ij}\beta }}{{\left(\sum_l{e^{x'_{il}\beta }}\right)}^2}e^{x'_{ij}\beta }\beta =p_{ij}\left(1-p_{ij}\right)\beta ,\] 

в то время как для $j\ne k$

\[\frac{\partial p_{ij}}{\partial x_{ik}}=-\frac{e^{x'_{ij}\beta }}{{\left(\sum_l{e^{x'_{il}\beta }}\right)}^2}e^{x'_{ik}\beta }\beta =-p_{ij}p_{ik}\beta .\] 

Скомбинировав эти результаты, получим \eqref{GrindEQ__15_18_}.

\subsection{Мультиномиальная логит-модель}

Вероятность в мультиномиальной логит-модели определяется как $p_{ij}={e^{x'_i{\beta }_j}}/{\sum_l{e^{x'_i{\beta }_l}}}$. Получим частные производные

\[\frac{\partial p_{ij}}{\partial {\beta }_j}=\frac{e^{x'_i{\beta }_j}}{\sum_l{e^{x'_i{\beta }_l}}}x_i-\frac{e^{x'_i{\beta }_j}}{{\left(\sum_l{e^{x'_i{\beta }_l}}\right)}^2}e^{x'_i{\beta }_j}x_i=p_{ij}x_i-p_{ij}p_{ij}x_i,\] 

а для $k\ne j$

\[\frac{\partial p_{ij}}{\partial {\beta }_k}=-\frac{e^{x'_i{\beta }_j}}{{\left(\sum_l{e^{x'_i{\beta }_l}}\right)}^2}e^{x'_i{\beta }_k}x_i=-p_{ij}p_{ik}x_i.\] 

Совмещая эти результаты, получим

\[\frac{\partial p_{ij}}{\partial {\beta }_k}={\delta }_{ijk}p_{ij}x_i-p_{ij}p_{ik}x_i=p_{ij}\left({\delta }_{ijk}-p_{ik}\right)x_i,\] 

где переменная-индикатор ${\delta }_{ijk}=1$, если $j=k$, и

\[\frac{\partial {\mathcal L}}{\partial {\beta }_k}=\sum_i{\sum_j{\frac{y_{ij}}{p_{ij}}\frac{\partial p_{ij}}{\partial {\beta }_k}}}\] 

\[=\sum_i{\sum_j{\frac{y_{ij}}{p_{ij}}\left({{\delta }_{ijk}p}_{ij}-p_{ij}p_{ik}x_i\right)}}\] 

\[=\sum_i{\left[\sum_j{y_{ij}{\delta }_{ijk}-y_{ij}p_{ik}}\right]x_i}\] 

\[=\sum_i{[y_{ik}-p_{ik}]}x_i,\] 

как утверждается в \eqref{GrindEQ__15_16_}, где в последней строке используется определение ${\delta }_{ijk}$ и равенство $\sum_j{y_{ij}=1}$. Выражение для второй производной имеет вид

\[\frac{{\partial }^2{\mathcal L}}{\partial {\beta }_j\partial {\beta }'_k}=-\sum_i{\sum_j{\frac{\partial p_{ij}}{\partial {\beta }'_k}x_i=-\sum_i{\sum_j{p_{ij}\left({\delta }_{ijk}-p_{ik}\right)x_ix'_i}}}},\] 

откуда получаем \eqref{GrindEQ__15_17_}.

Когда изменяется значение регрессора

\[\frac{\partial p_{ij}}{\partial x_i}=\frac{e^{x'_i{\beta }_j}}{\sum_l{e^{x'_i{\beta }_l}}}{\beta }_j-\frac{e^{x'_i{\beta }_j}}{{\left(\sum_l{e^{x'_{il}\beta }}\right)}^2}\sum_l{e^{x'_i{\beta }_l}}{\beta }_l=p_{ij}{\beta }_j-p_{ij}\sum_l{p_{il}{\beta }_l}=p_{ij}\left({\beta }_j-{\overline{\beta }}_i\right),\] 

где ${\overline{\beta }}_i=\sum_l{p_{il}{\beta }_l}$, как утверждается  в выражении \eqref{GrindEQ__15_19_}.

\subsection{Вложенная логит-модель}

Рассмотрим двухуровневую модель с ошибками ОРЭЗ, заданную в \eqref{GrindEQ__15_32_} и \eqref{GrindEQ__15_33_} с функцией

\[G\left(Y\right)=G\left(Y_{11},\dots ,Y_{1K_1},\dots ,Y_{J1},\dots ,Y_{JK_J}\right)=\sum^J_{j=1}{a_j{\left(\sum^{K_j}_{k=1}{Y^{{1}/{{\rho }_j}}_{jk}}\right)}^{{\rho }_j}\ ,}\] 

которая является обобщением \eqref{GrindEQ__15_34_} благодаря коэффициентам $a_j$. Общий результат модели с ОРЭЗ \eqref{GrindEQ__15_31_} принимает вид ${\Pr  \left[y_{jk}=1\right]\ }={Y_{jk}G_{jk}}/{G(Y)}$, где $G_{jk}$ является производной от $G(Y)$ по $Y_{jk}$, а $Y_{jk}=e^{V_{jk}}$.

Мы получаем 

\[G_{jk}=\frac{\partial G\left(Y\right)}{\partial Y_{jk}}={a_j\left(\sum^{K_l}_{l=1}{Y^{{1}/{{\rho }_j}}_{jl}}\right)}^{{\rho }_j-1}\times Y^{\left({1}/{{\rho }_j}\right)-1}_{jk},\] 

что дает

\[Y_{jk}G_{jk}=a_j{\left(\sum^{K_l}_{l=1}{Y^{{1}/{{\rho }_j}}_{jl}}\right)}^{{\rho }_j}Y^{{1}/{{\rho }_j}}_{jk}.\] 

Тогда 

\[p_{jk}\equiv \frac{Y_{jk}G_{jk}}{G\left(Y\right)}=\frac{a_j{\left(\sum^{K_l}_{l=1}{Y^{{1}/{{\rho }_j}}_{jl}}\right)}^{{\rho }_j-1}Y^{{1}/{{\rho }_j}}_{jk}}{{\sum^J_{m=1}{a_m\left(\sum^{K_l}_{l=1}{Y^{{1}/{{\rho }_m}}_{ml}}\right)}}^{p_m}}.\] 

Вероятность выбора ветви $j$ после упрощений принимает вид

\[p_j\equiv \sum^{K_j}_{k=1}{p_{jk}}=\frac{a_j{\left(\sum^{K_l}_{l=1}{Y^{{1}/{{\rho }_j}}_{jl}}\right)}^{{\rho }_j}}{{\sum^J_{m=1}{a_m\left(\sum^{K_l}_{l=1}{Y^{{1}/{{\rho }_m}}_{ml}}\right)}}^{p_m}}.\] 

Условная вероятность выбора ответвления $k$ при условии выбора ветви $j$ принимает вид

\[p_{k|j}\equiv \frac{p_{jk}}{p_j}=\frac{Y^{{1}/{{\rho }_j}}_{jk}}{\sum^{K_l}_{l=1}{Y^{{1}/{{\rho }_j}}_{jl}}}.\] 

Этот результат приведен в работе Маддалы (1983, стр. 72).

Необходимо подсчитать  это выражение в точке $Y_{jk}={\rm exp}(V_{jk})$. Предположим, 

\[V_{jk}=z'_j\alpha +x'_{jk}{\beta }_j.\] 

Тогда, произведя некоторые алгебраические преобразования, получим

\[{\left(e^{V_{jk}}\right)}^{{1}/{{\rho }_j}}={\rm exp}({z'_j\alpha }/{{\rho }_j}){\rm exp}(x'_{jk}{\beta }_j/p_j),\] 

\[\sum^{K_l}_{l=1}{{\left(e^{V_{jl}}\right)}^{{1}/{{\rho }_j}}}={\rm exp}({z'_j\alpha }/{{\rho }_j}){\rm exp}(I_j),\] 

\[{\left(\sum^{K_l}_{l=1}{{\left(e^{V_{jl}}\right)}^{{1}/{{\rho }_j}}}\right)}^{{\rho }_j}={\exp  \left(z'_j\alpha +p_jI_j\right)\ },\] 

где

\[I_j={\ln  \left(\sum^{K_l}_{l=1}{{\rm exp}(x'_{jk}{\beta }_j/p_j)}\right)\ }.\] 

Отсюда следует, что вероятность выбора ветви $j$ принимает вид

\[p_j=\frac{a_j{\left(\sum^{K_l}_{l=1}{{\left(e^{V_{jl}}\right)}^{{1}/{{\rho }_j}}}\right)}^{p_j}}{\sum^J_{m=1}{{\left(\sum^{K_l}_{l=1}{{\left(e^{V_{ml}}\right)}^{{1}/{{\rho }_m}}}\right)}^{p_m}}}\] 

\[=\frac{a_j{\rm exp}(z'_j\alpha +p_jI_j)}{\sum^J_{m=1}{a_m({\rm exp}(z'_m\alpha +p_mI_m))}},\] 

как и заявлено в \eqref{GrindEQ__15_36_} для первого случая. Обратите внимание, что скалярный параметр $a_j$ можно включить в $z_j$ как специфичную для конкретной ветви фиктивную переменную, поскольку $a_j{\exp  \left(z'_j\alpha +p_jI_j\right)\ }={\rm exp}({\ln  a_j+\ }z'_j\alpha +p_jI_j)$. Без ограничения общности, поэтому, выберем $a_j=1$.

Вероятность выбора ответвления $k$ ветви $j$ имеет вид

\[p_{k|j}=\frac{{\left(e^{V_{jk}}\right)}^{{1}/{{\rho }_j}}}{\sum^{K_l}_{l=1}{{\left(e^{V_{jl}}\right)}^{{1}/{{\rho }_j}}}}=\frac{{\rm exp}({z'_j\alpha }/{{\rho }_j}){\rm exp}(x'_{jk}{\beta }_j/p_j)}{\sum^{K_l}_{l=1}{{\rm exp}({z'_j\alpha }/{{\rho }_j}){\rm exp}(x'_{jl}{\beta }_j/p_j)}}=\frac{{\rm exp}(x'_{jk}{\beta }_j/p_j)}{\sum^{K_l}_{l=1}{{\rm exp}(x'_{jl}{\beta }_j/p_j)}},\] 

как утверждается для второго случая в выражении \eqref{GrindEQ__15_36_}.

\section{Практические соображения}

Мультиномиальная логит-модель адекватна для описания данных или оценки предельных эффектов на вероятности, однако считается плохой, если требуется более структурная интерпретация параметров из-за предположения о независимости от посторонних альтернатив. Оценить мультиномиальную логит-модель можно во многих пакетах.

Вложенная логит-модель может быть оценена в пакете STATA и с помощью дополнения NLOGIT для пакета LIMDEP, а также ее достаточно легко реализовать на языке программирования, таком как GAUSS. Очевидно, что следует использовать эту модель, если имеется очевидная вложенная структура, однако, обычно такой ясной структуры не наблюдается.

Логит-модель со случайными параметрами необходимо специально кодировать на таком языке программирования, как, например, GAUSS, а также требуется использование симуляционных методов оценивания, описанных в главе 12. Кен Трейн приводит код для этой модели на своем веб-сайте \url{elsa.berkeley.edu/~train}.

Мультиномиальную пробит-модель еще сложнее оценить для случая более четырех альтернатив, и она на практике довольно редко используется успешно. Поэтому в настоящее время предпочитают использовать логит-модель со случайными параметрами.

\section{Библиографические заметки}

\textbf{15.3.} Хорошими основными источниками для освоения мультиномиальных моделей являются работы Амэмии (1981, 1985), Маддалы (1983) и Грина (2003). В работах Бен-Акивы и Лермана (1985), Трейна (1986) и Борш-Супана (1987) приведен и множество примеров, и обзор теории. В работе Трейна (2003) можно найти великолепное изложение мультиномиальных моделей с неупорядоченными исходами, оценка которых производится симуляционными методами.

\textbf{15.5.} В основополагающей статье МакФаддена (1981) описывается применение моделей дискретного выбора на продвинутом уровне, основной акцент делается на подход, использующий модель случайной полезности. Для более подробной информации об анализе благосостояния ознакомьтесь с работами Смолла и Розена (1981), Трейна (2003, стр. 59-61) и Дагсвика и Карлстрёма (2004).

\textbf{15.6.} В работе Борша-Супана дается блестящее описание и применение вложенной логит-модели.

\textbf{15.7.} Логит-модель со случайными параметрами и другие недавние достижения в этой области описаны в работе Трейна (2003). Ревелт и Трейн (1998) --- это ранний пример использования этой модели.

\textbf{15.8.} Болдак (1999) оценивает с помощью симуляционного правдоподобия мультиномиальную пробит-модель с девятью альтернативами.

\textbf{Упражнения}

\textbf{15-1. } Рассмотрите модель со скрытой переменной, заданной как $y^*=x'\beta +\varepsilon $, с $\varepsilon \sim {\mathcal N}[0, 1]$. Допустим, что мы наблюдаем только $y=2$, если $y^*<\alpha $, $y=1$, если $\alpha \le y^*<U$, и $y=0$, если $y^*\ge U$, где верхняя граница $U$ --- известная для каждого индивида константа и может варьироваться между индивидами, однако $\alpha $ не известно.

\begin{enumerate}
\item  Найдите условные вероятности наступления $y=0$, $y=1$ и $y=2$.

\item  Приведите подробное описание способа получения состоятельных оценок $\beta $ и $\alpha $.
\end{enumerate}

\textbf{15-2. } Используя 50\% подвыборку данных из примера с выбором способа рыбалки из раздела 15.2:

\begin{enumerate}
\item  Оцените условную логит-модель из параграфа 15.2.1.

\item  Прокомментируйте статистическую значимость полученных оценок параметров.

\item  Каков эффект от роста цен на выбор различных способов рыбалки?
\end{enumerate}

\textbf{15-3. } Используя 50\% подвыборку данных из примера с выбором способа рыбалки из раздела 15.2:

\begin{enumerate}
\item  Оцените мультиномиальную логит-модель из параграфа 15.2.2.

\item  Прокомментируйте статистическую значимость полученных оценок параметров.

\item  Каков эффект от роста дохода на выбор различных способов рыбалки?
\end{enumerate}

\textbf{15-4. }Используйте 50\% подвыборку данных из примера с выбором способа рыбалки из раздела 15.2. Допустим, ситуация была сведена к модели с тремя альтернативами, которые были упорядочены так, что $y=0$, если выбрана рыбалка с пристани или пляжа, $y=1$, если выбрана рыбалка с собственной лодки, и $y=2$, если выбрана рыбалка с арендованной лодки.

\begin{enumerate}
\item  Оцените логит-модель с упорядоченными исходами, когда в качестве единственной независимой переменной выступает доход.

\item  Дайте интерпретацию полученным оценкам коэффициентов.

\item  Сравните качество подгонки этой модели с качеством подгонки мультиномиальной модели с тремя альтернативами с доходом в качестве регрессора.
\end{enumerate}





\chapter {Тобит-модели и модели выбора}

\section{Введение}

В этой главе рассматриваются две связанные между собой темы: модели с частично наблюдаемой зависимой переменной и модели с полностью наблюдаемой зависимой переменной для совокупности отобранных данных. К выше обозначенным моделям относятся модели с ограниченной зависимой переменной, модели со скрытой переменной, обобщенные тобит-модели и модели самоотбора. 

Все эти модели имеют общую характеристику, а именно даже в самом простом случае, когда условное среднее генеральной совокупности линейно по параметрам, МНК-оценки несостоятельны, поскольку выборка нерепрезентативна. Альтернативные способы оценки, в которых используются сильные предпосылки о распределении, необходимы для получения состоятельных оценок параметров.

К основным причинам неполноты данных относят усечение и цензурирование. При усечении как зависимая, так и независимая переменные наблюдаемы частично. Например, в качестве зависимой переменной используется доход, и в выборку входят индивиды только с низким уровнем доходов. В то же время, для цензурированых данных пропуски в данных могут быть только для зависимой переменной. Например, люди всех уровней дохода могут быть включены в выборку, но в целях конфиденциальности может быть установлено пороговое значение дохода, скажем 10 000 долл., выше которого, значение дохода принимается равным 10 000 долл. Усечение влечет за собой большую потерю информации, чем цензурирование. Основным примером усечения и цензурирования является тобит-модель, названная в честь Тобина (1958). Тобин изучал линейные регрессионные модели, опираясь на предпосылку, что ошибки нормально распределены. Проблемы характерные для оценки усеченных и цензурированных моделей рассмотрены в последующих главах, например, для цензурированных данных по длительности (глава 17). В целом, усечение и цензурирование -- это примеры пропущенных данных, см. главу 27.

Применения методов оценки первого поколения требует строгих предпосылок о распределении. Малейшие отклонения от изначальных предпосылок, например, нарушение предпосылки о гомоскедастичности ошибок, могут привести к несостоятельным оценкам параметров. По данной причине в этой главе рассматриваются полупараметрические регрессионные методы. Полупараметрические методы успешно применяются для простых моделей, например для данных цензурированных сверху. Вместе с тем, в настоящее время, отсутствует общепринятый способ оценивания для случаев, когда происходит самоотбор по ненаблюдаемым переменным.

В разделе 16.2 дана общая теория цензурированных и усеченных нелинейных регрессий; подробное описание тобит-модели дано в разделе 16.3. Альтернативная модель оценки цензурированных данных -- двухчастная модель -- вводится в разделе 16.4. В разделе 16.5 представлена модели самоотбора выборки. В разделе 6.6. на примере оценки затрат на здравоохранение сравниваются двухчастная модель Хекмана и модели самоотбора выборки. В разделе 16.7 рассматривается модель Роя. В разделе 16.8 рассматриваются полные структурные модели с угловыми решениями, полученные путем максимизации полезности или через обобщение системы одновременных уравнений для случая самоотбора выборки. В разделе 16.9 дается анализ полупараметрических методов.


\section{Эконометрические модели с цензурированными и усеченными данными}

Рассмотрим общие методы оценки полностью параметрических моделей с цензурированными или усеченными данными. Эти методы могут быть также использованы для оценки моделей, представленных в последующих главах, например, для счетной модели и модель длительности. Использование на практике тобит-модели для анализа цензурированных и усеченных данных продемонстрировано в разделах 16.2.1 и 16.3.

\subsection{Пример цензурированной и усеченной модели}

Пусть $y^* $ частично наблюдаемая переменная. При усечении снизу $y^* $ наблюдаемы, если $y^* $ выше порогового значения. Допустим, что пороговое значение равно нулю. Тогда, $y=y^* $, если $y^* >0$. Поскольку в выборке отсутствуют отрицательные значения, среднее усеченных данных будет выше среднего $y^* $. Для цензурированных снизу данных $y^* $ ненаблюдаемо полностью при $y^* <0$, но известно, что выполняется неравенство $y^* <0$, при этом значение $y^* $ приравнивается для простоты к нулю. В связи с этим, среднее цензурированных данных будет превышать среднее $y^* $. Очевидно, что среднее значение усеченных или цензурированных выборок не может использоваться без корректировки для оценки среднего исходной модели.

В этой главе рассматриваются аналогичные вопросы при оценки регрессионных моделей. Казалось бы, усечение или цензурирование приводит к изменению константы, не оказывая влияния на коэффициент наклона, но это не так. Например, если в исходной модели $\E[y^* |x]=x'\beta$, для усеченной или цензурированой модели $\E[y|x]$ нелинейно зависит от $x$ и $\beta$, следовательно, оценки МНК параметра $\beta$, а также оценка предельного эффекта последнего будут несостоятельны.

В качестве примера проанализируем модель предложения труда на симулированных данных. Предположим что зависимость ожидаемого количества отработанных часов в год, $y^* $ от почасовой заработная плата, $w$, описывается логарифмической функцией, и процесс порождающий данные представлен тремя уравнениями:

\begin{equation}
\begin{array}{l}
y^* =-2500+1000 \ln w +\varepsilon\\
\varepsilon\sim N[0,1000^2],\\
\ln w \sim N[2.75,0.60^2].
\end{array}
\end{equation}

% Рисунок figure 16.1
Рисунок 16.1
Tobit: Censored and Truncated Means --- Тобит: усеченные и цензурированные средние

Different Conditional Means --- Различные условные средние
Natural Logarithm of Wage --- Логарифм зарплаты
Actual Latent Variable --- Значение скрытой переменной
Truncated Mean --- Усеченное среднее
Censored Mean --- Цензурированное среднее
Uncensored Mean --- Нецензурированное среднее

Рисунок 16.1. Тобит регрессия количества часов работы от логарифма зарплаты: нецензурированное условное среднее (внизу), цензурированное условное среднее (посередине) и усеченное условное среднее (сверху) при усечении/цензурировании часов работы ниже нуля. Данные порождены простой линейной моделью.

Эта тобит-модель подробно рассмотрена в разделе 16.3. Согласно модели, эластичность заработной платы равна $1000/y^* $, следовательно, при полной занятости (2,000 часов) эластичность равна $0,5$. При увеличении заработной платы на $1\%$, количество часов работы увеличивается в среднем на $10$.

На рисунке 16.1 изображен график рассеивания в осях $y$ и $\ln{ w}$ для сгенерированной выборки из двухсот наблюдений. График безусловного среднего $y^* $ равного $-2500+1000 \ln w $ -- нижняя прямая линия.

При цензурировании в нуле отрицательные значения $y^* $ обнуляются, поскольку отрицательное значение ожидаемого количества рабочих часов, означает что индивиды предпочитают отдых. Как показывает практика, примерно $35\%$ индивидов предпочитают отдых. Замена отрицательных значений $y^* $ нулевыми, приводит к завышению среднего значения низкой заработной платы и практически не влияет на среднее высокой заработной платы, поскольку незначительный процент $y^* $ принимает нулевые значения. На средней кривой графика изображено итоговое значение цензурированного среднего, рассчитанное по формуле (16.23).


При усечении в нуле $35\%$ данных с отрицательными $y^*$ отбрасываются. Следовательно, для усеченных данных среднее выше, чем для цензурированных данных, поскольку отрицательные значения не учитываются. График усеченного среднего отображает верхняя линия на рисунке 16.1.

Очевидно, что среднее цензурированных и усеченных данных нелинейно по $x$, даже если линейно по $x$ среднее генеральной совокупности. Оценка цензурированной или усеченной регрессии МНК даст несостоятельные оценки коэффициентов наклона. Из графика видно, что график линейно аппроксимированных усеченного и цензурированного средних имеет более пологий наклон, чем график неусеченного среднего. Таким образом, среднее цензурированных или усеченных данных должно рассчитываться по специальным формулам. К сожалению, как мы увидим, эти формулы основаны на сильных предположениях о законе распределения.

\subsection{Механизмы цензурирования и усечения}

Как правило, в регрессионном анализе $y$ обозначает наблюдаемое значение зависимой переменной.В отличие от классического подхода, предполагается, что $y$ --- это частично наблюдаемая скрытая переменная $y^* $ и зависимость можно описать следующим уравнением:

\[
y=g(y^* ),
\]

где $g(\cdot )$ обозначает некоторую функцию. Варианты функциональной зависимости рассмотрены ниже. 


\subsubsection*{Цензурирование}


При цензурировании мы полностью наблюдаем регрессоры $x$, полностью наблюдаем $y^* $ из определенного подмножества значений $y^*$, а частично наблюдаем $y^*$ для оставшихся значений $y^*$. При цензурировании снизу (или слева):

\begin{equation}
y=
\begin{cases}
y^* , \text{ если } y^* >L\\
L, \text{ если }y^*  \leq L.
\end{cases}
\end{equation}

Например, всех потребителей можно разбить на две группы: к первой группе относятся индивиды с ненулевыми затратами на товары длительного пользования, т.е. $y^* >0$, а ко второй группе с нулевыми затратами, т.е. $y^*  \leq 0$. При цензурировании сверху (или справа):

\begin{equation}
y=
	\begin{cases}
	y^* ,	 \text{ если }y^* <U\\
	U,		\text{ если }y^*  \geq U.
	\end{cases}
\end{equation}

Например, может быть определена верхняя граница для данных о доходе, к примеру, $U=100,000$. Такое цензурирование в литературе по методам дюрации, получило название цензурирование первого типа (см. раздел 17.4.1). 


\subsubsection*{Усечение}

Усечение приводит к потере информации, поскольку теряются все данные за порогом теряются. При усечении снизу мы наблюдаем только:

\begin{equation}
y=y^* , \text{ если } y^* >L
\end{equation}

К примеру, выборка может включать данные только по индивидам, которые купили товары длительного пользования ($L=0$). При усечении сверху

\begin{equation}
y=y^* , \text{ если } y^* <U.
\end{equation}

Например, в выборку могут быть включены только данные по индивидам с низким уровнем дохода.

\subsubsection*{Интервальные данные}

Интервальные данные записываются с помощью интервалов. Как правило, именно в этой форме записываются данные исследования для обеспечения анонимности персональных данных. Например, доход может лежать в промежутке от 10,000 долл. до 100,000 долл. Для таких данных может быть установлено несколько границ цензурирования, тогда наблюдаемой переменной $y$ является интервал, в котором может находиться ненаблюдаемое значение $y^* $. 

\subsection{Цензурироваанная и усеченная оценка с помощью ММП}

Проблему цензурирования и усечения легко преодолеть, если исследователь использует полностью параметрический подход. Это может хорошо сработать для интервальных или ограниченных сверху данных, например, где целесообразно предположить логнормальное распределение дохода или использовать отрицательную биномиальную модель для количества визитов к доктору.

Если условное по параметрам распределение $y^* $ специфицировано, тогда возможно рассчитать эффективные и состоятельные ММП-оценки, учитывая цензурированное или усеченное распределение $y$. Предположим, что $f^{*}(y^* |x)$ и $F^{*}(y^* |x)$ обозначает функцию плотности условного распределения (или условную вероятность) и функцию распределения скрытой переменной $y^* $. Тогда всегда можно найти функции $f(y|x)$ и $F(y|x)$, соответствующие условной функции плотности и функции распределения наблюдаемой зависимой переменной $y$, поскольку $y=g(y^* )$ --- это преобразование $y^* $.

Ограниченность полностью параметрического подхода обусловлена сильными предпосылками о распределении. Например, ММП-оценка линейной регрессионной модели остается состоятельной даже при нарушении предпосылки о нормальности распределения ошибок, тогда как ММП-оценка цензурированной регрессии будет несостоятельной (см. раздел 16.3.2). Более гибкие модели и полупараметрические методы будут рассмотрены в последующих разделах. 


\subsubsection*{Цензурированный метод максимального правдоподобия}


Цензурирование и усечение оказывают влияние на условное среднее и условную плотность. В первую очередь рассмотрим влияние на плотность.


В качестве примера рассмотрим ММП оценку при цензурировании снизу. Для $y>L$ плотность $y$ принимает такое же значение как плотность $y^* $, следовательно, $f(y|x)=f^{*}(y|x)$. Для нижней границы $y=L$ плотность вырождена, и мы полагаем её равной вероятности наблюдения $y^* <L$ или $F^{*}(L|x)$. Таким образом, при цензурировании снизу 

\[
f(y|x)=
\begin{cases}
f^{*}(y|x),& \text{если $y>L$},\\
F^{*}(L|x),& \text{если $y=L$}
\end{cases}
\]


Ранее было отмечено, что при $y^* <L$ не обязательно брать $y=L$. Даже если все значения $y$ ненаблюдаемы, при $y^*  \leq L$, значение плотности равно $F^{*}(L|x)$. По аналогии с бинарными моделями, введем индикаторную переменную 

\begin{equation}
d=
\begin{cases}
1, & \text{если $y>L$}, \\
0, & \text{если $y=L$}.
\end{cases}
\end{equation}


Тогда условную плотность цензурированных снизу данных можно записать:

\begin{equation}
f(y|x)=f^{*}(y|x)^{d}F^{*}(L|x)^{1-d}.
\end{equation}



Например, для $N$ наблюдений цензурированная ММП-оценка максимизирует значение выражения:

\begin{equation}
\ln{ L_N{\theta}}=\sum_{i=1}^{N} \lbrace{d_i}\ln{ f^{*}(y_i|x_i,\theta)-d_i)\ln{ F^{*}}(L_i|x_i,\theta)\rbrace} ,
\end{equation}


где $\theta$ является параметрами распределения для $y^* $. В общем случае, при цензурировании нижняя граница может быть для каждого индивида своя, хотя, как правило, $L_i=L$. ММП-оценка цензурированной регрессии состоятельна и асимптотически нормально распределена, при условии, что правильно специфицирована исходная плотность до цензурирования, $f^{*}(y^* |x,\theta)$.

При цензурировании сверху функция максимального правдоподобия имеет вид (16.8), за исключением того, что $d=1$, если $y<U$ и $d=0$, в ином случае, а вместо $F^{*}(L|x,\theta)$ используется $1-F^{*}(U|x,\theta)$. Хорошим примером являются справа усеченные данные по длительности. 


\subsubsection*{Усеченная ММП-оценка}


При усечении снизу в точке $L$ условная плотность распределения $y$ равна (зависимость от $x$ опущена для простоты)


\[
f(y)= f^{*}(y|y>L)=f^{*}(y)/\Pr[ y|y>L] =f^{*}/[1-F^{*}(L)].
\]

Тогда, усеченная ММП-оценка максимизирует значение функции


\begin{equation}
\ln L_N(\theta) =\sum_{i=1}^N \lbrace \ln{ f^{*}(y_i,x_i,\theta)-\ln[1-F^{*}(L_i|x_i,\theta)]}\rbrace.
\end{equation}

При усечении сверху, функция максимального правдоподобия примет вид (16.9), где $F^{*}(L|x,\theta)$ заменяется на $F^{*}(U|x,\theta)$.


Если не учитывать усечение или цензурирование оценки параметров будут несостоятельны. Например, если не учитывать усечения, тогда будет максимизироваться значение функции $\sum_{i} \ln{  f^{*}(y_i,x_i,\theta)}$, что является неверным, поскольку отбрасывается второе слагаемое (16.9). Состоятельность цензурированных и усеченных ММП-оценок обеспечивается правильной спецификацией $f(\cdot )$, что, в свою очередь, требует правильной спецификации плотности скрытой переменной $f^{*}(\cdot )$. Даже если функция плотности $f^{*}(\cdot )$ принадлежит экспоненциальному семействе (см. раздел 5.7.3), не только среднее, но и сама плотность должны быть правильно специфицированы.


\subsubsection*{Оценка ММП для интервальных данных}


Пусть значения скрытой переменной $y^* $ принадлежат одному из $(J+1)$ взаимоисключающих интервалов $(-\infty,a_1], (a_1,a_2],\ldots ,(a_J,\infty)$, где значения $a_1,a_2,\ldots ,a_J$ известны. Поскольку

\[
\Pr[ a_j<y^* <a_{j+1}]=\Pr[ y^*  \leq a_{j+1}]-\Pr[ y^*  \leq a_j] =
F^{*}(a_{j+1})-F^{*}(a_j),
\]

ММП-оценка интервальных данных максимизирует значение функции 

\begin{equation}
\ln{ L_N(\theta)}=\sum_{i=1}^N\sum_{j=0}^J d_{ij} \ln[F^{*}(a_{j+1}|x_i,\theta)-F^{*}(a_j|x_i,\theta)].
\end{equation}


где $d_{ij}$, $j=0,\ldots ,J$, индикатор равный единице, если $y_{ij}{\epsilon}(a_j,a_{j+1}]$ и нулю иначе. Модель аналогична логит- и пробит-регрессии (см. раздел 15.9.1)., за исключением того, что границы интервалов $a_1,\ldots ,a_J$ известны


\subsection{Пример пуассоновской усеченной и цензурированной регрессии}


Предположим, что $y^* $ распределено по Пуассоновскому закону, т.е. $f^{*}(y)=e^{-\mu}\mu^{y}/y!$ и $\ln{ f^{*}(y)}=-\mu+y\ln{ \mu}-\ln{ y!}$, где среднее значение $\mu=\exp (x'\beta)$.

Предположим, что задача состоит в моделировании количества визитов в больницу, но данные известны только для тех, кто посещал врача. Следовательно, данные усечены снизу и $y=y^* $ только для $y^* >0$. Тогда $F^{*}(0)=\Pr[ y^*  \leq 0]=\Pr[ y^* =0]=e^{-\mu}$ и оценка $\beta$ максимизирует значение выражения

\[
\ln{ L_N(\beta)}=\sum_{i=1}^N{\lbrace-\exp (x_i'\beta)+y_ix_i'\beta-\ln{ y!}-\ln[1-\exp (-\exp (x_i'\beta))]\rbrace}
\]


Теперь предположим, что данные цензурированы выше границы равной 10, тогда $y=y^* $ если $y^* <0$ и $y=10$, если $y^* {\geq}10$. Тогда $\Pr[ y^* {\geq}10]=1-\Pr[ y^* <10]=1-\sum_{k=0}^9 f^{*}(k)$. Из формулы (16.8) следует, что цензурированная ММП оценка $\beta$ максимизирует значение функции


\[
\ln{L_N(\beta)}=\sum_{i=1}^N \left\lbrace d_i[-\exp (x_i'\beta)+y_ix_i'\beta-\ln{ y_i}!]+(1-d_i)\ln{ \left[\sum_{k=0}^9 e^{-\exp (x_i'\beta)}(\exp (x_i'\beta))^{k}/k!\right]}\right\rbrace.
\]


В обоих случаях условия первого порядка значительно усложняются из-за усечения и цензурирования. Вместе с тем, максимизация исходной функции плотности, т.е. при отсутствии усечения и цензурирования, даст несостоятельные оценки параметров.

\subsection{Условное среднее в цензурированных и усеченных регрессиях}

Цензурирование или усечение данных приводит к изменению условного среднего.

Например, рассмотрим усеченное в нуле пуассоновское распределение. Усеченная вероятность равна $f^{*}(y)/[1-F^{*}(0)], y=1,2,\ldots $, тогда усеченное среднее $\sum^{\infty}_{k=1}kf^{*}(k)/[1-F^{*}(0)]=\sum^{\infty}_{k=0}=\sum^{\infty}_{k=0}kf^{*}(k)/[1-F^{*}(0)]=\mu/[1-e^{-\mu}]$. Следовательно, условное математическое ожидание примет значение:


\[
\E[y|x]=\exp (x'\beta)/[1-\exp (-\exp (x'\beta))],
\]

а не $\exp (x'\beta)$, что справедливо для полных данных.

Это значение $\E[y|x]$ может использоваться для получения оценок с помощью нелинейного МНК. Тем не менее, преимущество нелинейного МНК по сравнению с ММП невелико, т.к. нелинейный МНК использует сильные предпосылки о распределении, которые в целом настолько же сильные, как и предпосылки, выполнение которых необходимо для получения более эффективных ММП-оценок.

\section{Тобит-модель}

Усеченные и цензурированные данные чаще всего встречаются в эконометрическом анализе линейных регрессионных моделей с нормально-распределенными ошибками, когда наблюдаются только положительные значения зависимой переменной. Эти модели получили название тобит-модели, в честь Тобина (1958), исследователя, который применил модель с выше обозначенными характеристиками для оценки затрат индивидов на товары длительного пользования. На практике модель слишком ограничена. Тем не менее, существует необходимость в детальном изучении модели Тобина, поскольку тобит-модель является основой для построения моделей более общего класса, которые будут рассмотрены в последующих разделах этой главы.

\subsection{Тобит-модель}

Цензурированная регрессионная модель с нормально распределенными данными, или тобит-модель, принадлежит классу моделей с цензурированием снизу в нуле, где латентная переменная линейно зависит от параметров, ошибки аддитивны, нормально распределены и гомоскедастичны. Таким образом, 

\begin{equation}
 y^* =x'\beta+\varepsilon,
\end{equation}
где ошибки распределены нормально, с параметрами $0$ и $\sigma^2$:

\begin{equation}
\varepsilon{\sim}N[0,\sigma^2],
\end{equation}

и $\sigma^2$ константа. Следовательно, распределение скрытой переменной имеет вид: $y^*  \sim N[x'\beta,\sigma^2]$. Наблюдаемые значения $y$ определены в выражении (16.2) с $L=0$, поэтому 

\begin{equation}
y=
\begin{cases}
y^* , &\text{если } y^* >0, \\
-, & \text{если } y^*  \leq 0,
\end{cases}
\end{equation}

где $-$ означает, что $y$ пропущено. При $y^*\leq 0$ может не наблюдаться конкретное значение $y$, однако, в некоторых случаях, например для затрат на товары длительного пользования, мы наблюдаем $y=0$. 

Уравнения (16.11)-(16.13) задают базовую тобит-модель, рассмотренную Тобином (1958). В более общем виде, тобит-модель для латентных переменных начинается с (16.11) и (16.12) и может включать различные способы цензурирования: сверху, одновременно сверху и снизу (тобит-модель с двумя порогами) и интервальное цензурирование, -- общий вид модели может меняться. В этом разделе все цензурирование удовлетворяет условию (16.13). Модели последующих разделов иногда называют обобщенными тобит-моделями.

Приравнивание $L$ к нулю является не только естественным, но и необходимым условием для линейной модели с константой и постоянным пороговым значением $L$. Тогда, переменная $y$ наблюдаема, при $y^* >L$, что эквивалентно выражению $\beta_1+x'_2\beta_2+\varepsilon$ или $(\beta_1-L)+x'\beta_2+\varepsilon>0$. То есть идентифицируемо только значение $\beta_1 - L$. В общем случае, модель с латентными переменными $y^* =x'\beta$ с переменным порогом цензурирования $L=x'\gamma$, эквивалентна модели $y^* =x'(\beta-\gamma)+\varepsilon$ с фиксированным значением порога $L=0$. Эти выводы являются следствием цензурирования в линейной модели с аддитивной ошибкой и не применимы к нелинейным моделям, например, для пуассоновского распределения.

Используя выражение (16.7) для плотности цензурированного распределения, где $f^{*}(y) \sim N[x'\beta, \sigma^2]$ получаем:

\[
F^{*}(0)=\Pr[ y^* \leq0] =\Pr[ x'\beta+\varepsilon \leq 0] =\Phi(-x'\beta/\sigma) 
=1-\Phi(x'\beta\sigma)
\]

где $\Phi(\cdot )$ функция стандартного нормального распределения и для получения последнего равенства используется свойство симметрии стандартного нормального распределения. Тогда, плотность цензурированного распределения равна:

\begin{equation}
f(y)=
\left[\dfrac{1}{\sqrt{2\pi\sigma^{2}}}\exp \lbrace-\dfrac{1}{2\sigma^{2}}(y-x'\beta)^{2}\rbrace\right]^d
\left[1-\Phi\left(\dfrac{x'\beta}{\sigma}\right)\right]^{1-d},
\end{equation}

где значение бинарного индикатора $d$ определено в (16.6) при $L=0$. Оценка параметра тобит-модели методом максимального правдоподобия $\hat{\theta}=(\hat{\beta}',\hat{\sigma}^2)'$ является максимумом цензурированной функции (16.8). Используя (16.4), получим смешение плотностей дискретного и непрерывного распределения: 

\begin{multline}
\ln L_N(\beta,\sigma^2)=\sum_{i=1}^N
\left\lbrace d_i
\left(-\frac{1}{2}\ln 2\pi-\frac{1}{2}\ln{ \sigma^2-\frac{1}{2\sigma^2}(y_i-x_i')^2}\right)
+\right.\\
\left.
(1-d_i)\ln \left(1-\Phi
\left( \frac{x_i'\beta}{\sigma}\right)
\right)
\right\rbrace,
\end{multline}

и условие первого порядка:

\begin{equation}
\begin{array}{l}
\dfrac{\partial \ln L_N}{\partial\beta}=\sum_{i=1}^N\dfrac{1}{\sigma^2}\left(d_i(y_i-x_i'\beta)-(1-d_i)\dfrac{\sigma\phi_i}{(1-\Phi_i)}\right)x_i=0 \\
\dfrac{\partial \ln L_N}{\partial\sigma^2}=\sum_{i=1}^N\sum\left\lbrace d_i \left(-\dfrac{1}{2\sigma^2}+\dfrac{(y_i-x_i')^2}{2\sigma^4}\right)+(1-d_i)\dfrac{\phi_i x_i'\beta}{(1-\Phi_i)2\sigma^3}\right\rbrace=0
\end{array}
\end{equation}


мы используем тот факт, что $\partial\Phi(z)/\partial{z}$, где $\phi(\cdot )$ функция плотности нормальной стандартной случайной величины и определяем $\phi_i=\phi(x_i'\beta/\sigma)$ и $\Phi_i=\Phi(x_i'\beta/\sigma)$.


Как обычно, $\hat{\theta}$ состоятельна, если плотность условного распределения корректно определена, т.е. модель задана уравнением (16.11) и (16.12) и принцип цензурирования отражен в (16.13). Оценка максимального правдоподобия имеет нормальное распределение, ковариационную матрицу для нормального распределения можно найти в работах Маддала (1983, стр. 155) и Амэмия (1985, стр.373).

Тобин (1958) предложил оценивать тобит-модель методом максимального правдоподобия и обосновал возможность применения ММП оценки. В работе Амэмия (1973) представлено формальное доказательство применимости стандартной теории для плотности смешенного, дискретно-непрерывного, цензурированного распределения. В приложении к классической работе Амэмия подробно рассмотрена асимптотическая теория для экстремумов оценок, представленных в разделе 5.3. 

Если данные усечены снизу от нуля, а не цензурированы, тогда ММП-оценка тобит-регрессии $\hat{\theta}=(\hat{\beta}',\hat{\sigma}^2)$ является максимумом для усеченной функции правдоподобия


\begin{equation}
\ln{ L_N(\beta,\sigma^2)}=\sum_{i=1}^N{\sum\left\lbrace -\dfrac{1}{2}\ln{ \sigma^2}-\dfrac{1}{2}{\ln{ 2\pi}}-\dfrac{1}{2\sigma^2}(y_i-x_i'\beta)^2-\ln{ \Phi(x_i'\beta/\sigma)}\right\rbrace }
\end{equation}

где для $y^* $ используется выражение (16.9) и параметры распределения $y^* $ определены в (16.11) и (16.12).

\subsection{Несостоятельность тобит оценок метода максимального правдоподобия}

Главным недостатком тобит-моделей является жесткость предпосылок о распределении ошибок. При нарушении предпосылки о гомоскедастичности ошибок или о нормальном распределении оценки максимального правдоподобия будут несостоятельны.

Это утверждение следует из условий первого порядка (16.16) максимального правдоподобия, заданные сложной функцией, зависящей от переменных $d_i, y_i, \phi_i$ и $\Phi_i$. Первое уравнение в (16.16) удовлетворяет условию $\E[\partial{lnL_N}/\partial\beta]=0$, которое являются необходимыми условием состоятельности оценок (см. Раздел 5.3.7), если

\[
\begin{array}{l}
\E[d_i]=\Phi_i, \\
\E[d_{i}y_{i}]=\Phi_{i}x'_{i}\beta+\sigma\phi_i.
\end{array}
\]

Можно показать, что эти моментные условия выполняются, если спецификация модели задана выражениями (16.11) и (16.12) и принцип цензурирования соответствует (16.13). Тем не менее, эти условия скорее всего не будут выполняться, если спецификация модели определена иначе, чем в (16.12) и (16.13), поскольку не будет выполняться условие гомоскедастичности или предпосылка о нормальности распределения. Например, если ошибки гетероскедастичны оценка будет несостоятельна, поскольку $\E[d_i]=\Phi(x'_{i}\beta/\sigma_{i}){\neq}\Phi_{i}$, кроме случая $\sigma^2_i=\sigma^{2}$.

Тем не менее, при гетероскедастичности нормально распределенных ошибок возможно получение состоятельных оценок, если задать модель для дисперсии $\sigma^2_i=\exp (z'_i\gamma)$. При цензурировании данных ниже нуля в функции максимального правдоподобия $lnL_{N}(\beta,\gamma)$, определенной в (16.15), значение дисперсии $\sigma^2$ заменяется на $\exp (z'_i\gamma)$. Обязательными условиями состоятельности оценки являются нормально распределенные ошибки и верно заданная функциональная форма для гетероскедастичности.


Ясно, что выполнение предпосылок о распределении более важно в цензурированных и усеченных моделях, даже если это распределение устойчиво к неправильной спецификации когда цензурирование или усечение отсутствуют. Тесты на спецификацию тобит-модели представлены в разделе 16.3.7. Для многих цензурированных данных тобит регрессии не применимы. Модели более общей формы рассмотрены далее.


\subsection{Цензурированное и усеченное среднее в линейной регрессии}


Для цензурированных и усеченных линейных регрессий (16.11) условное математическое ожидание наблюдаемой переменной $y$ отлично от $x'\beta$, а условная дисперсия отлична от $\sigma^2$ даже если ошибки $\varepsilon$ гомоскедастичны. Распределение $y$ также не является нормальным, несмотря на нормальное распределение ошибок. В этом разделе приведены общие результаты для линейной регрессии, в последующих разделах (16.3.4-16.3.7) рассмотрен частный случай, когда ошибки нормально распределены. Результаты учитывают цензурирование и усечение и являются основой для методов, не входящих в группу методов максимального правдоподобия.

Вначале рассмотрим усеченное среднее. Влияние усечения данных можно предположить интуитивно. Усечение данных слева исключает малые значения, следовательно, среднее значение должно увеличиться, а усечение справа, наоборот, предполагает уменьшение среднего. Поскольку усечение предполагает сокращение диапазона значений, дисперсия также должна сократиться.

При усечении слева, значения $y$ наблюдаемы, если $y^* >0$. Опуская обозначение $x$ в зависимости математического ожидания, получаем

\begin{multline}
\E[y]=\E[y^* |y^* >0]=\\
=\E[x'\beta+\varepsilon|x'\beta+\varepsilon>0]\\
=\E[x'\beta|x'\beta+\varepsilon>0]+\E[\varepsilon|x'\beta+\varepsilon>0]\\
=x'\beta+\E[\varepsilon|\varepsilon>-x'\beta],
\end{multline}

где второе равенство использует (16.11), а в последнем равенстве используется свойство независимости $\varepsilon$ от $x$. Как и ожидалось, усеченное среднее превышает значение $x'\beta$, поскольку $\E[\varepsilon|\varepsilon>c]$ превосходит $\E[\varepsilon]$ для любого значения константы $c$.

Для данных, цензурированных слева от нуля, положим, что вместо $y^*  \leq 0$ мы наблюдаем $y=0$. Цензурированное среднее рассчитывается путем взятия сначала условного среднего $y$ при фиксированном индикаторе $d$, определенном в (16.6) с $L=0$, а потом взятием безусловного среднего. Снова опуская зависимость от $x$, получаем, что среднее значение слева цензурированных данных равно:


\begin{multline}
\E[y]=E_{d}[E_{y|d}[y|d]] \\
=\Pr[ d=0]{\times}\E[y|d=0]+\Pr[ d=1]{\times}\E[y|d=1]\\
=0{\times}\Pr[ y^*  \leq 0]+\Pr[ y^* >0]{\times}\E[\varepsilon|\varepsilon>-x'\beta],\\
=\Pr[ y^* >0]{\times}\E[y^* |y^* >0],
\end{multline}

где $\Pr[ y^* >0]=1-\Pr[ y^*  \leq 0]=\Pr[ \varepsilon>-x'\beta]$ равно единица минус вероятность цензурирования и $\E[y^* |y^* >0]$ усеченное среднее, ранее выведенное в (16.18).

Таким образом, при цензурировании или усечении данных с пороговым значением ноль условное среднее будет равно:

для скрытой переменной:

\[\E[y|x^{*}]=x'\beta\]

для модели, усеченной слева (в нуле):

\begin{equation}
\E[y|x,y>0]=x'\beta+\E[\varepsilon|\varepsilon>-x'\beta],
\end{equation}

для модели, цензурированной слева (в нуле): 

\[
\E[y|x]=\Pr[ {\varepsilon}>-x'\beta]\lbrace{x'\beta+\E[\varepsilon|\varepsilon>{-x}'\beta]}\rbrace.
\]

Очевидно, что даже если условное среднее начальной переменной линейно, то при цензурировании или усечении данных линейность исчезает, что приводит к несостоятельности МНК-оценок.

Для решения проблемы несостоятельности можно использовать параметрический метод, накладывая ограничения на распределение $\varepsilon$. Затем можно получить выражения для $\E[\varepsilon|\varepsilon>x'\beta]$ и $\Pr[ \varepsilon>x'\beta]$ и, следовательно, можно рассчитать цензурированное или усеченное среднее. Параметрический метод оценки с нормально распределенными ошибками представлен в следующем разделе.

Рисунок 16.2.

Inverse Миллс Ratio as Cutoff Varies --- Обратное отношение Миллса и пороговое значение

inverse Миллс ratio --- обратное отношение Миллса
N[0,1] cdf --- N[0,1] функция распределения
N[0,1] pdf --- N[0,1] функция плотности

Inverse Миллс, pdf and cdf --- Обратное отношение Миллса, функции плотности и распределения
Cutoff point c --- Пороговое значение c

Рисунок 16.2. Изменение обратного отношения Миллса в зависимости от значения порога цензурирования или усечения c. На рисунке также изображены функции распределения и плотности для стандартной нормальное величины.



Цель второго подхода заключается в поиске решений, для которых отсутствует необходимость принятия параметрических предпосылок. Этот подход будет рассмотрен в последующих разделах, но следует отметить, что выражение для усеченного среднего это одноиндексная модель с подправочным членом, убывающим по $x'\beta$, поскольку $\E[\varepsilon|\varepsilon>-x'\beta]$ монотонно убывающая функция от $x'\beta$.

\subsection{Цензурированное и усеченное среднее в тобит-модели}

Для тобит-модели остатки регрессии $\varepsilon$ нормально распределены. При дальнейшем рассмотрении будем использовать вывод из раздеда 16.10.1:

Утверждение 16.1 (Усеченные моменты для стандартного нормального распределения): Предположим, что $z \sim N[0,1]$. Тогда моменты усеченных слева значений $z$ равны:


(i) $\E[z|z>c]=\phi(c)/[1-\Phi(c)]$, и $\E[z|z>-c]=\phi(c)/\Phi(c)$,\\
(ii) $\E[z^{2}|z>c]=1+c\phi(c)/[1-\Phi(c)]$, и \\
(iii) $\V[z|z>c]=1+c\phi(c)/[1-\Phi(c)]-\phi(c)^{2}/[1-\Phi(c)]^{2}$ \\

Результат (i) Утверждения 16.1 изображен на рисунке (16.2). Рассмотрим усечение нормально распределенной величины $z \sim N[0,1]$ ниже некоторого значения $c$, где $c$ принадлежит промежутку от $-2$ до $2$. Нижняя кривая --- это функция плотности $\phi(c)$ для нормальной стандартной случайной величины. Средняя линия изображает график функции распределения $\Phi(c)$ в точке $c$ и задает вероятность усечения при усечении в точке $c$. Значение этой вероятности приближенно равно $0.023$ в точке $c=-2$ и $0.977$ при $c=2$. Верхняя линия изображает график усеченного среднего $\E[z|z>c]=\phi(c)/[1-\Phi(c)]$. Как и предполагалось, оно близко к  $\E[z]=0$  при $c=-2$, т.к. усечение невелико и $\E[z|z>c]>c$. Неожиданным оказывается, что $\phi(c)/[1-\Phi(c)]$ приближенно линейна. Значение моментов при усечении сверху можно рассчитать по формуле $\E[z|z<c]=-\E[-z|-z>-c]=-\phi(c)/\Phi(c)$. 

Используя (16.18), получаем, что усеченное среднее случайной ошибки равно:

\begin{multline}
\E[\varepsilon|\varepsilon>{-x}'\beta]={\sigma}\E\left[\dfrac{\varepsilon}{\sigma}|\dfrac{\varepsilon}{\sigma}>\dfrac{{-x}'\beta}{\sigma}\right] \\
=\sigma\phi(-\dfrac{x'\beta}{\sigma})/[1-\Phi(-\dfrac{x'\beta}{\sigma})] \\
=\sigma\phi(\dfrac{x'\beta}{\sigma})/[\Phi(\dfrac{x'\beta}{\sigma})] \\
=\sigma\lambda(\dfrac{x'\beta}{\sigma}),
\end{multline}

где второе равенство следует из утверждения 16.1, для получения третьего равенства использовалось свойство симметрии функции $\phi(z)$ относительно нуля и мы определяем 

\begin{equation}
\lambda(z)=\dfrac{\phi(z)}{\Phi(z)}.
\end{equation}

При определении $\lambda$ мы руководствовались определениями и терминологией, используемых в работе Амэмия (1985), параметр $\lambda(\cdot )$ получил название обратное отношение Миллса. В своей работе Джонсон и Котц (1970, p.278) отметили, что фактически Миллс составил таблицу отношений $(1-\Phi(z))/\phi(z)$ и обратное значение этого отношения $\phi(z)/[1-\Phi(z)]=\phi(z)/\Phi(-z)$, которое задает функцию риска нормального распределения. Ряд авторов записывают (16.21) в другом виде, а именно: $\E[\varepsilon|\varepsilon>-x'\beta]=\sigma\lambda^{*}(-x'\beta/\sigma)$, где $\lambda^{*}(z)=\phi(z)/\Phi(-z)$ обратное отношение Миллса.

Вместе с тем, $\Pr[ \varepsilon>-x'\beta]=\Pr[ -\varepsilon<x'\beta]=\Pr[ -\varepsilon/\sigma<x'\beta/\sigma]=\Phi(x'\beta/\sigma)$. Тогда условное среднее в (16.20) равно:

для скрытой переменной: 

\begin{equation}
\E[y^* |x]=x'\beta
\end{equation},

для моделей c усеченными слева данными (в нуле): 
\[
\E[y|x,y>0]=x'\beta+\sigma\lambda(x'\beta/\sigma)
\]

для моделей с цензурированными слева данными (в нуле): 
\[
\E[y|x]=\Phi(x'\beta/\sigma)x'\beta+\sigma\phi(x'\beta/\sigma)
\]


Дисперсия рассчитывается по аналогичным формулам (см. упражнение 16.1). Пусть $w=x'\beta/\sigma$, тогда 

для скрытой переменной:
\begin{equation}
\V[y^* |x]=\sigma^2
\end{equation}

для моделей с усеченными слева данными (в нуле): 
\[
\V[y|x,y>0]=\sigma^2[1-w\lambda{w}-\lambda(w)^2]
\]

для моделей с цензурированными слева данными (в нуле):
\[
\V[y|x,y>0]={\sigma}^2\Phi(w)\left\lbrace w^2+w\lambda(w)+1-\Phi(w)[w+\lambda(w)]\right\rbrace^2
\]

Очевидно, что усечение и цензурирование приводят к гетероскедастичности ошибок и при усечении $\V[y|x<\sigma^2]$, следовательно, усечение уменьшает изменчивость значений.

Необходимым условием достижения выше обозначенных результатов является выполнение предпосылки о нормальном распределении ошибок. В работе Маддалы (1983, стр. 369) представлены результаты аналогичные утверждению 16.1 для лог-нормального, логистического, экспоненциального, гамма распределений и распределения Лапласа.

\subsection{Предельные эффекты в тобит-модели}

Под предельным эффектом понимают влияние изменений значений регрессоров на условное среднее зависимой переменной. Значение предельного эффекта может меняться в зависимости от интересующего объекта, это может быть среднее значение скрытой переменной, $x'\beta$,  или среднее цензурированных или усеченных данных, определенное в (16.23).

Дифференцируя каждое выражение по $x$, получим следующие результаты:


Скрытая переменная: 
\begin{equation}
\partial \E[y^* |x]/{\partial{x}}=\beta
\end{equation}

модель с данными усеченными слева (от нуля): 
\[
\partial{E}[y,y>0|x]/{\partial}x={1-w\lbrace\lambda}(w)-\lambda(w)^2\rbrace\beta
\]


модель с данными цензурированными слева (от нуля): 
\[
\partial \E[y|x]/\partial x=\Phi(w)\beta
\]



где $w=x'\beta/\sigma$, $\partial\Phi(z)/\partial{z}=\phi(z)$ и $\partial\phi(z)/\partial{z}=-z\phi{(z)}$. Простое выражение для среднего цензурированных данных получается после некоторых преобразований. Можно разложить эффект на два: для $y=0$ и для $y>0$ (см. МакДоналд и Моффитт, 1980).

В некоторых случаях, цензурирование или усечение вызваны особенностями сбора данных, и значение усеченного и цензурированного среднего не представляют интереса сами по себе, а интересно лишь значение $\partial \E[y^* |x ]\partial x=\beta$. Например, при цензурированных сверху данных о зарплате мы хотим рассчитать эффект образования именно на нецензурированное среднее.


В других случаях, усечение и цензурирование имеют поведенческую интерпретацию. Например, при моделировании рабочего времени рассчитываются три предельных эффекта (16.25), а именно: влияние изменения регрессоров на (1) желаемое количество часов работы, (2) фактическое количество отработанных часов для работающих и (3) фактическое количество отработанных часов работающими и неработающими. Для расчета (1) требуется значение оценки $\beta$, для расчета (2) и (3) коэффициенты наклона МНК-регрессии, хотя и являются несостоятельными для $\beta$, могут дать разумную примерную  оценку предельного эффекта, поскольку среднее цензурированных и усеченных данных примерно линейно по $x$.

\subsection{Альтернативные способы оценки тобит-модели}

Следует отметить, что нелинейный метод наименьших квадратов (НМНК) так же, как и метод максимального правдоподобия может дать состоятельную оценку параметров при корректном определении среднего усеченных и цензурированных данных. Далее рассмотрим МНК-оценки и нелинейные МНК-оценки.

\subsubsection*{Нелинейный МНК}

На основе полученных результатов (16.23) с помощью нелинейного МНК могут быть рассчитаны состоятельные оценки параметров тобит-модели. Например, для усеченных данных минимизируется значение функции

\[
S_{N}(\beta,\sigma^2)=\sum^N_{i=1}\left(y_{i}-{x'}_{i}{\beta}-{\sigma}{\lambda}({x'}_{i}{\beta}/{\sigma})\right)^2
\]

по параметрам $\beta$ и ${\sigma}^2$, и далее статистические выводы производятся с учетом  гетероскедастичности (16.24). Аналогичные вычисление могут быть сделаны для цензурированных данных. Аналогичный способ может применяться к цензурированным данным.

На практике нелинейный МНК не используется. Неотъемлемым условием состоятельности оценок является корректная спецификация усеченного среднего, при этом, согласно (16.21), корректная спецификация возможна при выполнении предпосылки о нормальном распределении и гомоскедастичности случайных ошибок. Поэтому оценка $S_N$ может быть рассчитана методом максимального правдоподобия, основанном на строгих предпосылках о распределении, что позволяет получить эффективные оценки. Кроме того, на практике оценка НМНК может быть неточной. Из графика 16.2 можно сделать вывод, что $\lambda(x'\beta/\sigma)$ приближенно линейна по $x'\beta/\sigma$, что приводить к мультиколлинеарности, поскольку $x$ также является регрессором. В разделе 16.5 будут представлены модели, в которых используется поправочный коэффициент похожий на $\sigma\lambda(x'\beta/\sigma)$ в (16.23), но не зависящий от $x$.


\subsubsection*{Двухшаговая процедура Хексмана}

Из (16.23) следует, что среднее усеченных (в нуле) данных  равно

\begin{equation}
\E[y|x]=x'\beta+\sigma\lambda(x'\beta/\sigma).
\end{equation}

Вместо НМНК, если данные цензурированы, значение можно оценить по следующей двухшаговой процедуре. На первом шаге, оцениваем пробит-регрессию $d$ по $x$, где бинарная переменная $d$ равна 1, если $y>0$, в результате получим состоятельные оценки параметра $\hat{\alpha}$, где $\alpha=\beta/\sigma$. На втором шаге, для расчета состоятельных оценок $\beta$ и $\sigma$, оцениваем регрессию по усеченными данными $y$ на $x$ и $\lambda(x'\hat{\alpha})$ с помощью МНК.

Описание и применение процедуры Хекмана (1976, 1979) для моделей с самоотбором выборки приведены  в разделе 16.5.4. 
Вывод формулы для оценки стандартной ошибки $\hat{\beta}$, которая учитывает наличие регрессора $\lambda(x'\hat{\alpha})$ и  гетероскедастичности, вызванной усечением, содержится в разделе 16.10.2.

\subsubsection*{МНК оценка тобит-модели}

МНК оценки коэффициента $\beta$ при усечении и цензурировании несостоятельны. Это вызвано тем, что цензурированное и усеченное среднее в формуле (16.23) не равняются $x'\beta$, что нарушает существенное условие состоятельности МНК оценок.

Для цензурированных данных МНК дает линейную аппроксимацию к нелинейной цензурированной зависимости среднего. Из рисунка 16.1 и (16.25) ясно, что эта кривая более пологая, чем линия регрессии для нецензурированных данных, наклон которой равен истинному угловому коэффициенту. Голдбергер (1981) аналитически доказал, что если $y$ и $x$ имеют совместное нормальное распределение, а цензурирование производится в нуле, то МНК оценки наклона сходятся к истинному угловому коэффициенту, домноженному на $p$, где $p$ --- доля наблюдений с положительными значениями $y$. Жесткие предпосылки данного доказательства были ослаблены в работе Рууд (1986). На практике данное соотношение даёт хорошую меру несостоятельности МНК оценок в тех случаях, когда следует использовать тобит-модель.

Усеченная кривая зависимости среднего также более полога, чем неусеченная. Голдбергер (1981) получил аналогичный случаю цензурированных данных результат. Если $y$ и $x$ имеют совместное нормальное распределение, а усечение производится в нуле, то МНК оценки сходятся к истинному угловому коэффициенту, домноженному на некоторую подправочную константу. Эта подправка имеет громоздкое выражение, лежит от нуля до единицы и одинакова для всех коэффициентов модели. Таким образом, МНК занижает абсолютную величину истинных коэффициентов наклона. 


\subsection{Тесты на спецификацию для тобит-моделей}


Учитывая жесткие предпосылки тобит-модели, хорошей практикой является тестирование предпосылок о распределении. Существует 4 стратегии.

Согласно первому подходу, модель дополняют параметрами и проводят тест Вальда, LR или LM тест. Наиболее простым вариантом является использование LM теста, поскольку требуется оценка только тобит-модели. LM тест против альтернативной гипотезы о гетероскедастичности вида $\sigma_i^2=\exp (x_i'\alpha)$ очень прост. Используя форму внешнего произведения градиентов (см. раздел 7.3.5) находится домноженный на $N$ коэффициент $R^2$ во вспомогательной регрессии $1$ на $s_{1i}$ и $s_2i$. Здесь $f_i=f(y_i|x_i,\beta,\alpha)$ --- плотность, определенная в (16.14), где $\sigma$ заменяется на $\exp (x'\alpha)$, а $s_{1i}=\partial \ln f_i/\partial\beta$ и $s_{2i}=\partial \ln f_{i}/\partial \alpha$.  Тильда обозначает подстановку в  цензурированну тобит-модель зануленных компонент $\alpha$, кроме свободного члена. Аналогичные тест на нормальность гораздо труднее, т.к. не существует общепринятого распределения, обобщающего нормальное.


Во втором подходе проводятся тесты условных моментов (см. раздел 8.2), которые не требуют определения альтернативной модели. В частности, согласно условиям первого порядка (16.16) для цензурированной тобит-регрессии, имеет смысл строить тесты на условные моменты основываясь на обобщенных остатках: 

\[
e_i=d_i\dfrac{y_i-x'_i\beta}{\sigma^2}-(1-d_i)\dfrac{\phi_i}{\sigma(1-\Phi_i)}.
\]

При правильной спецификации тобит-модели $\E[e_i|x_i]=0$, поскольку условия регулярности приводят к тому, что  $\E[\partial{\ln{ f(y_i)}}/\partial\beta]=0$. Тогда можно проверить с помощью М-теста нулевую гипотезу $H_0:\E[ez]=0$ против альтернативной $H_a:\E[ez]{\neq}0$ с помощью $N^{-1}\sum_{i=1}^N{\hat{e}_iz_i}$, где $\hat{e}_i=e_i$ посчитанному в точке  ММП-оценки тобит-модели  $(\hat{\beta};\hat{\sigma}^2)$. 
Из раздела 8.2.2 следует, что тестовую статистику можно посчитать помножив $N$ на $R^2$ вспомогательной регрессии -- единицы на $\hat{e}_iz_i$, $\hat{s}_{1i}$ и $s_{2i}$, где  $f_{i}=f(y_{i}|x_{i},\beta,\sigma^2)$ --- плотность, определенная в (16.14), а $s_{1i}=\partial{ln}f_{i}/\partial\beta$ и $s_{2i}=\partial{ln}f_{i}/\partial\sigma^2$ из (16.16) рассчитаны в точке $(\hat{\beta},\hat{\sigma}^2)$. Переменные $z_i$ и $x_i$ могут не совпадать, тогда тест можно интерпретировать как тест на пропущенные переменные. Также существуют моментные тесты, использующие моменты более высокого порядка. Более подробный анализ представлен у Чешера и Айриш (1987) и Пагана и Велла (1989). 

Третий подход состоит в адаптации диагностических методов и методов тестирования, разработанных для данных по продолжительности жизни цензурированных справа, к нормально распределенным данным цензурированным слева.


Согласно четвертому подходу, вместо ММП-оценивания $\beta$ тобит-модели используется полупараметрические методы, представленные в разделе 16.9, которые являются состоятельными при  более слабых предпосылках о распределении.

Для более глубокого изучения спецификации тестов тобит-моделей можно обратиться к работе Пагана и Велла, в которой рассмотрена теория и практические примеры, а также к работе Меленберга и Ван Суета (1996), где автор более подробно остановился на рассмотрении практических вопросов. Обе работы посвящены вопросу  тестов на спецификацию как для тобит-модели, так и для более общей модели с самоотбором выборки (см. раздел 16.5).


\section{Двухчастная модель}

В ранее рассмотренных моделях с цензурированными данными механизм цензурирования был неразрывно связан с процессом, порождающим значения зависимой переменной. В более общем случае, способ цензурирования и процесс, порождающий зависимую переменную могут быть разделены. Например, для расчета ежегодных затрат индивида на госпитализацию одна модель может быть использована для ответа на вопрос, будет ли госпитализация, вторая --- для моделирования последующих затрат на госпитализацию, если она состоялась. Необходимость использования двух моделей может быть вызвана тем, что некоторые значения получаются слишком часто или, наоборот, слишком редко, чем это характерно для простейшей модели. Например, частота появления нуля может быть выше, чем это характерно для пуассоновского распределения. Возможность генерации нулевых и ненулевых значений при помощи разных законов распределения повышает гибкость. Следует отметить, что двухчастная модель --- это частный случай модели смеси.

Существует два подхода к обобщению, а именно: рассмотренная в этом разделе двухчастная модель описывает механизм цензурирования и процесс, порождающий значения зависимой переменной, для наблюдаемых исходов. Модель с самоотбором выборки, рассмотренная в следующем разделе, задает совместное распределение для индикатора цензурирования и зависимой переменной, из чего можно получить условный закон распределения зависимой переменной для наблюдаемых исходов.  Сравнение этих подходов дано в разделе 16.5.7.

\subsection{Двухчастная модель}

Пусть индивид, данные о котором наблюдаемы называется \textbf{участником} изучаемого процесса. Бинарная переменная $d$ равна 1 для участников и 0 для неучастников. Предположим, что для участников наблюдаемое значение $y>0$, а для тех, кто не обучается $y=0$. Для неучастников, наблюдается только значение $\Pr[ d=0]$. Условная плотность $y$ для участников при $y>0$ задана выражением $f(y|d=1)$, для некоторой плотности $f(\cdot )$. Двухчастная модель для $y$ определена следующей системой:

\begin{equation}
f(y|x)=
\begin{cases}
\Pr[d=0|x], \, \text{ если } y=0, \\
\Pr[d=1|x]f(y|d=1,x) \, \text{ если } y>0.
\end{cases}  
\end{equation}

Эта модель была предложена в работе Крэгга (1971) как обобщение тобит-модели. Действительно, тобит-модель можно определить как частный случай (16.27). Смоделировать индикатор участия $d$ можно с помощью логит или пробит-модели. Скрытая переменная $d$ равна 1, если $I=x'\beta+\varepsilon$ больше нуля и тогда модель может быть рассмотрена как модель преодоления порогов, поскольку пересечение порогового значения будет означать участие. Для того, чтобы обеспечить положительные значения $y$ для участников, плотность $f(y|d=1,x)$ должна быть плотностью положительной случайной величины, например, подойдет лог-нормальное распределение, также можно взять плотность нормально распределенных значений, усеченную левее нуля.

Для простоты, одинаковые регрессоры фигурируют в формуле как на первом, так и на втором шаге, однако это  не обязательно, а иногда даже желательно. Метод максимального правдоподобия легко реализовать, поскольку он позволяет отдельно оценивать модель для учатсия, где используется полный массив данных и оценку параметров плотности $f(y|d=1,x)$ для положительных значений, $y>0$.


\subsection{Пример двухчастной модели}


Дуан и др. (1983) приводят хороший пример применения двухчастной модели для прогнозирования расходов на лечение, используя базу данных Эксперимента по страхованию здоровья корпорации RAND (Rand Health Insurance Experiment). Авторы скомпоновали пробит-модель, которая предсказывает, тратил ли индивид средства на лечение в течении года, где $\Pr[ d=1|x]=\Phi(x'_1+\beta_{1})$, и модель затрат на лечение с лог-нормальным распределением ненулевых затрат, заданную уравнением $\ln{ y|d}=1, x \sim N[x'_2\beta_{2},\sigma^{2}_{2}]$. Тогда ожидаемое значение затрат на лечение по всей выборки равно

\begin{equation}
\E[y|x]=\Phi(x'_1\beta_1)\exp [\sigma^{2}_2/2+x'_2\beta_2],
\end{equation}

где второй член выражения получается в силу того, что из $\ln{ y} \sim N[\mu,\sigma^2]$ следует, что $\E[y]=\exp (\mu+\sigma^{2}/2)$. Более подробно вопрос преобразований в этой модели рассмотривает в своей работе Маллахай (1998).

Двухчастная модель часто используется для моделирования счетных данных. Например, при моделировании количества визитов к доктору одна модель позволяет определить посещал ли пациент врача, а вторая модель рассчитывает количество повторных визитов только для индивидов, которые имеют хотя бы по одному визиту к врачу. Тогда $\Pr[ d=1]$ задается как  вероятность того, что переменная с пуассоновским или отрицательным биномиальным распределением окажется больше нуля, в то время как вероятность $f(y|d=1)$ определяется как вероятность пуассоновского или отрицательного биномиального  распределения, усеченная снизу. В литературе по счетным данным, эта модель, предложенная в работе Маллахай (1986), получила название модель преодоления порогов. Она подробно рассматривается в разделе 20.4.5. 

Двухчастная модель с непрерывными данными используется, например, для оценки модели затрат с большим количеством нулевых данных (идея предложена Крэггом). В следующем разделе рассмотрен альтернативный метод --- модели с самоотбором выборки.


\section{Модели с самоотбором выборки}


Самоотбор наблюдений может возникнуть в разных ситуациях. В настоящее время разработано много моделей с самоотбором выборки. Прежде чем приводить примеры практического применения, рассмотрим теоретическую сторону вопроса. К самым известным моделям с самоотбором выборки относят бинарную модель выбора Хекмана (1979) и модель Роя (см. раздел 16.7).


\subsection{Модели с самоотбором выборки}


Исследования по наблюдаемым данным  редко основаны на чисто случайных выборках. Как правило используется экзогенно заданная выборка (см. раздел 3.2.4) и, следовательно, могут быть применены стандартные методы. В случае если выборка, намеренно или случайно, сформирована на базе  значений зависимой переменной, оценки параметров будут несостоятельными, только если не будут применены коррекционные меры. Такие выборки называются выборками с самоотбором наблюдений.


Существует много моделей для  выборок с самоотбором, поскольку существует много механизмов самоотбора. Вместе с тем, можно не знать, имел ли самоотбор наблюдений место. Например, рассмотрим интерпретацию среднего балла за тестирование по успеваемости, такого, как Экзамен на определение академически способностей (Scholastic Aptitude Test), когда участие добровольно. Понижение среднего балла может быть следствием ухудшения знаний студентов. Понижение также может быть обусловлено тем, что большее количество студентов стали участвовать в тесте и появились участники с более низким баллом.

Самоотбор выборки может быть результатом решения конкретных индивидов принимать или не принимать участие в изучаемой деятельности. Самоотбор выборки также может быть результатом смещения выборки, например, в выборку может попасть существенно больше участвующих в изучаемой деятельности. В экстремальном случае в выборку попадают только участники деятельности. В любом случаем проблемы получаются сходные, и данное множество моделей называют моделями с самоотбором выборки.


В этой главе рассматриваются только три модели отбора наблюдений. Самая простая это тобит-модель, см. раздел 16.3. Базовая широко используемая модель, получившая название двумерной модели самоотбора (bivariate sample selection model), рассмотрена в этом разделе. Модель двумерного самоотбора обобщает тобит-модель путем введения скрытой переменной цензурирования, не совпадающей со скрытой переменной, определяющей наблюдаемые значения зависимой переменной. В разделе 16.7 рассмотрена другая популярная модель --- модель Роя. В модели Роя выбор одного из двух возможных значений зависимой переменной определяется цензурирующей переменной. Перечисленные модели по терминологии Амэмия (1985, стр. 384), называются тобит-моделями 1,2 и 5.


Состоятельность оценки требует сильных предпосылок о законах распределения даже в случае полупараметрических методов. Исследования на экспериментальных данных являются альтернативой моделям самоотбора наблюдений, поскольку случайный характер выборки может решить проблему самоотбора. Однако, использование экспериментальных данных может быть затруднено высокими издержками на его проведение и рядом других причин. В главе 25 подробно рассмотрено оценивание эффектов воздействий, при этом к наблюдаемым данным применяется экспериментальный подход.


\subsection{Модель двумерного самоотбора выборки, тобит-2}


Пусть $y^*_{2}$ обозначает интересующий нас результат. В стандартной тобит-модели с усеченными данными результат известен только при $y^* _{2}>0$. В моделях общего класса вводится дополнительная латентная переменная $y^* _1$ и результат $y^* _2$ известен, только при $y^* _{1}>0$. Например, $y^* _1$ --- это решение, работать или не работать, а $y^* _2$ определяет количество рабочих часов, при этом $y^* _1{\ neq}y^* _2$, поскольку существуют фиксированные издержки, например, ежедневные затраты на проезд, которые более значимы для решения работать или нет, чем количество желаемых рабочих часов.

Модель двумерного  выбора включает уравнение участия,

\begin{equation}
y_1=
\begin{cases}
1, \text{ если } y_1^{*}>0, \\
0, \text{ если } y_1^{*} \leq 0
\end{cases}
\end{equation}

и уравнение наблюдаемой переменной:

\begin{equation}
y_2=
\begin{cases}
y_2^{*}, \text{ если } y_1^{*}>0, \\
-, \text{ если }y_1^{*} \leq 0
\end{cases}
\end{equation}

Согласно этой модели $y_2$ наблюдается, только при $y^* >0$. При $y^* _{1} \leq 0$ величина $y_2$ не обязана иметь смысл. В стандартной модели определяется линейная зависимость для латентной переменной и аддитивными ошибками:

\begin{equation}
\begin{array}{l}
y_1^{*}=x'_{1}\beta_1+\varepsilon_1, \\
y_2^{*}=x'_{2}\beta_2+\varepsilon_{2},
\end{array}
\end{equation}

при наличии корреляции между $\varepsilon_1$ и $\varepsilon_2$ появляются проблемы при оценке $\beta_2$. Тобит-модель является частным случаем этой модели, когда $y^* _1=y^* _2$.

Для этой модели отсутствует общепринятое название. Хекман (1979) использовал эту модель для демонстрации результатов оценки регрессии при самоотборе наблюдений. 
По своей спецификации модель эквивалентна тобит-модели со стохастическим пороговым значением (Нельсон, 1977). 
Предположим, что значение $y^* _2$ наблюдаемо, если $y^* _{2}>L^{*}$, где $y^* _2$ определено в (16.31) и пороговое значение задается выражением $L^{*}=z'\gamma+u$ вместо $L^{*}=0$, как в разделе 16.3. Тогда, равнозначно, что $y^* _2$ наблюдаема, если $y^* _1>0$ и $y^* _{1}=y^* _{2}-L^{*}=(x'_2\beta_{2}-z'\gamma)+(\varepsilon_2-\nu)=x'_{1}\beta_{1}+\varepsilon_1$ и где $x_1$ объединяет $x_2$ и $z$, и $\beta_1$ и $\varepsilon_1$ определяются естественным образом. Амэмия (1985, стр. 384) назвал эту модель тобит-модель второго типа. Вулдридж (2002, стр. 506) предложил название модель с пробит-уравнением выбора. В других работах можно встретить названия обобщенная тобит-модель или модель самоотбора выборки.

Естественным способом оценки является ММП метод, при условии, что коррелированные ошибки имеют совместное нормальное распределение и гомоскедастичны:

\begin{equation}
\begin{bmatrix}
\varepsilon_1\\ \varepsilon_2
\end{bmatrix}
\sim
N
\left[
\begin{bmatrix}
0 \\ 0
\end{bmatrix},
\begin{bmatrix}
1 &\sigma_{12} \\ \sigma_{12}&\sigma_{22}^2
\end{bmatrix}
\right]
\end{equation}

Как и для пробит-модели в разделе 14.4.1, нормируем дисперсию $\sigma^{2}_1=1$, поскольку известен только знак $y^* _1$. 

Из (16.29) и (16.30) следует, что для  $y^* >0$ вероятность значения $y^* _2$ равна вероятности того, что $y^* >0$ умноженной на вероятность $y^* _2$ условную по $y^* >0$. Следовательно, для положительных значений $y_2$ значение плотности равно $f^{*}(y^* _2|y^* _1>0){\times}\Pr[ y^* _1>0]$. Для события $y^* _{1} \leq 0$ мы знаем только лишь, что оно произошло и вероятность равна $\Pr[ y_{1}^{*} \leq 0]$. Следовательно, функция правдоподобия для модели двумерного самоотбора имеет вид:

\begin{equation}
L=\Pi_{i=1}^n{\lbrace \Pr[ y_{1i}^{*} \leq 0]\rbrace}^{1-y_i}{\lbrace f(y_{2i}|y_{1i}^{*}>0){\times}\Pr[ y_{1i}^{*}>0]\rbrace}^{y_{1i}}
\end{equation}

где первый множитель --- это дискретная составляющая при $y^* _{1i} \leq 0$, поскольку тогда $y_{1i}=0$, и второй множитель --- это непрерывная составляющая, для $y^* _{1i}>0$. Эта функция правдоподобия может использоваться не только для линейных регрессий с нормально распределенными ошибками, но и при оценке более широкого класса моделей.

Для линейной регрессии с совместно  нормально распределенными ошибками двумерная плотность  $f^{*}(y^* _1,y^(*)_2)$ является нормальной, поэтому условная функция плотности второго аргумента является одномерной нормальной и с ней легко работать. Подробный анализ можно найти у Амэмия (1985, стр. 385-387), в том числе точную форму функции правдоподобия. 

Изначально эта модель использовалась для анализа предложения труда, где $y^* _1$ ненаблюдаемое желание работать или способность к труду, а $y_2$ фактически отработанное количество часов. Модель больше подходит к оценке предложения труда, поскольку в отличие от тобит-модели не требует искусственной переменной <<желаемых>> часов работы. При анализе предложения труда возникала трудность связанная с неизвестной зарплатой для неработающих индивидов.
Эту трудность можно преодолеть путем добавления уравнения для предлагаемой заработной платы и, далее, подставить данное значение в исходное уравнение, хотя модель при этом перестает быть моделью двумерного самоотбора. Хороший пример оценки предложения труда рассмотрен в работе Мроз (1987).

\subsection{Условное среднее в модели двумерного самоотбора}

В этом разделе рассмотрим условное среднее усеченных данных модели двумерного самоотбора. Значение среднего не равно $x'_{2}\beta_2$, а МНК регрессия $y_2$ на $x_2$ дает несостоятельные оценки параметров. Тем не менее, выражение для условного среднего может быть использовано в качестве мотивации для использования альтернативного метода оценивания, рассматриваемого в последующем разделе и использующего более слабые предпосылки о распределении чем для ММП. 

Рассмотрим усеченное среднее в модели двумерного самоотбора выборки, где используются только положительные значения $y_2$. Общий вид модели:

\begin{equation}
\E[y_2|x,y_1^{\ast}>0]=\E[x_2^{\prime}\beta_2+\varepsilon_2|x'\beta_1+\varepsilon_1>0] = x_2'\beta_2+\E[\varepsilon_2 | \varepsilon_1>x_1'\beta_1]],
\end{equation}

где $x$ объединяет $x_1$ и $x_2$. Если случайные ошибки не зависят друг от друга, тогда последнее слагаемое упрощается и принимает вид: $\E[\varepsilon_2]=0$ и МНК-регрессия $y_2$ на $x_2$ даст состоятельные оценки $\beta_2$. Однако, если существует зависимость между  ошибками, тогда усеченное среднее не равно $x'\beta_2$, и необходимо учитывать механизм отбора. 

Для расчета $\E[\varepsilon_2|\varepsilon_1>-{x'}_1\beta_1]$, где $\varepsilon_1$ и $\varepsilon_2$ коррелированы Хекман (1979) отметил, что если ошибки $(\varepsilon)1,\varepsilon_2)$ в (16.31) имеют совместное нормальное распределение как в (16.32), тогда для уравнения (16.36) справедливо, что 

\begin{equation}
\varepsilon_2=\sigma_{12}\varepsilon_1+\xi,
\end{equation} 

где случайная переменная $\xi$ независима от $\varepsilon_1$. Для получения этого результата заметим, что из совместного нормального распределения

\[
\begin{bmatrix}
z_1\\z_2
\end{bmatrix}
\sim
N
\left[
\begin{bmatrix}
{\mu}_1\\{\mu}_2
\end{bmatrix},
\begin{bmatrix}
\Sigma_{11}&\Sigma_{12} \\ \Sigma_{21}&\Sigma_{22}
\end{bmatrix}
\right],
\]

следует нормальность условного распределения

\[
z_2|z_1 \sim N[\mu_2+\Sigma_{21}{\Sigma^{-1}}_{11}(z_{1}-\mu_1),\Sigma_{22}-\Sigma_{21}{\Sigma^{-1}}_{11}\Sigma_{12}],
\]

следовательно

\begin{equation}
z_2=\mu_2+\Sigma_{21}{\Sigma^{-1}}_{11}(z_1-\mu_1)+\xi,
\end{equation}

где $\xi \sim N[0,\Sigma_{22}-\Sigma_{21}{\Sigma^{-1}}\Sigma_{12}]$ независима от $z_1$. Для  совместной плотности распределения (16.32) известно, что $\mu_1=\mu_2=0$ и ${\sigma^{2}}_1=1$, следовательно, выражение (16.36) приводит к  (16.35).

Подставляя (16.35) в формулу для усеченного среднего, (16.34), получим:

\[
\E[y_2|x,y^* _1>0]=x'_2\beta_2+\E[(\sigma_{12}\varepsilon_1+\psi)|\varepsilon_1>-x^{-1}\beta_1]=x'_{2}\beta_2+\sigma_{12}\E[\varepsilon_1|\varepsilon_1>-x'_1\beta_1],
\]
 
при этом используется свойство независимости $\psi$ и $\varepsilon_1$. Составляющая самоотбора  аналогична простой тобит-модели и используя выражение для $\E[z|z>{-c}]$ из Утверждения 16.1 получаем:


\begin{equation}
\E[y_2|x,{y^* }_1>0]={x'}_2\beta_2+\sigma_{12}\lambda({x'}_1\beta_1),
\end{equation}

где $\lambda(z)=\phi(z)/\Phi(z)$ и мы воспользовались тем, что ${\sigma}_1=1$. Вместе с тем, результат 16.1 (iii) позволяет выразить дисперсию усеченных данных

\begin{equation}
\V[y_2|x,{y^* }_1]={\sigma^2}_2-{\sigma^2}_{12}\lambda({x'}_1\beta_1)({x'}_1\beta_1+\lambda({x'}_1\beta_1)).
\end{equation}

В предыдущем анализе не указывалось значение $y_2$ при ${y^* } \leq 0$. В некоторых случаях, при ${y^* }<0$ значение $y_2$ может быть равно нулю. В таком случае осмысленно рассматривать среднее значение усеченных данных. 


Рассчитаем условное среднее $y_2$ при фиксированных $y^* _1$ и $y^* _2$, а затем возьмем безусловное:

\begin{multline}
\E[y_2|x]=E_{{y^* }_1}[\E[y_2|x,{y^* }_1]] \\
=\Pr[ {y^* }_1 \leq 0|x]{\times}0+\Pr[ {y^* }_1>0|x]{\times}\E[{y^* }_2|x,{y^* }_2>0] \\
=0+\Phi({x'}_1\beta_1)\left\lbrace{x'}_2\beta_2+\sigma_{12}\lambda\left({x'}_1\beta_1\right)\right\rbrace \\
=\Phi({x'}_1\beta_1){x'}_2\beta_2+\sigma_{12}\phi({x'}_1\beta_1),
\end{multline}

где третья строка получается на основе соотношения (16.37) и последняя строка на основе $\lambda(z)=\phi(z)/{\Phi}(z)$. Можно показать, что дисперсия цензурированных данных гетероскедастична.


\subsection{Двухшаговая оценка Хекмана}

МНК-оценки регрессии $y_2$ на $x_2$ при использовании только наблюдаемых и положительных значений $y_2$ дает несостоятельные оценки $\beta$, за исключением случая некоррелированных ошибок, т.е. $\sigma_{12}=0$. Это следует из формулы среднего усеченных данных (16.37), где  присутствует дополнительный <<регрессор>> $\lambda(x'_1\beta_1)$. 

Иногда двухшаговую оценку Хекмана называют хекит-оценкой (Heckit estimator), где в МНК регрессию добавляется пропущенная переменная $\lambda(x'_1\beta_1)$. Таким образом, используя положительные значения $y_2$ с помощью МНК оценивается  регрессия:

\begin{equation}
y_{2i}=x'_{2i}\beta_2+\sigma_{12}\lambda(x'_{1i}\hat{\beta}_1)+\nu_i,
\end{equation}

где $\nu$ случайная ошибка, $\hat{\beta}_1$ оценка, полученная на первом шаге пробит- регрессии $y_1$ на $x_1$, поскольку $\Pr[ y^* _1>0]=\Phi({x'}_1\beta_1)$ и $\lambda(x'_1\hat{\beta}_1)=\phi(x'_1\hat{\beta}_1)/{\Phi}(x'_1\hat{\beta}_1)$ оценка обратного отношения Миллса. Данная модель не дает прямой оценки дисперсии $\sigma^2_2$, но из формулы усеченной дисперсии (16.38) можно получить оценку $\hat{\sigma}_2^2=N^{-1}\sum_i[\hat{v}_i^2+\hat{\sigma}_{12}^2\hat{\lambda}_i(x_1'\hat{\beta}_1+\hat{\lambda}_i)]$, где $\hat{v}_i$ МНК остатки  регрессии (16.40) и $\hat{\lambda}_i=\lambda(x_{1i}'\hat{\beta}_1)$. Корреляция между ошибками в (16.32) может быть оценена по формуле $\hat{\rho}=\hat{\sigma}_{12}/\hat{\sigma}_2.$

Тестирование гипотезы $\sigma_{12}=0$ или $\rho=0$ есть проверка наличия корреляции ошибок и необходимости подправки вызванной самоотбором наблюдений. Одним из таких тестов является тест Вальда, основу которого составляет обратное отношение Миллса, $\hat{\sigma}_{12}$. 

Следует отметить, что как классические МНК стандартные ошибки коэффициентов, так и робастные к гетероскедастичности стандартные ошибки коэффициентов для регрессии (16.40) некорректны. Для правильного расчета стандартных ошибок на втором этапе оценивания нужно учесть два фактора. Во-первых, даже если $\beta_1$ известно ошибки в (16.40) гетероскедастичны, что следует из (16.38). Во-вторых, вместо $\beta_1$ используется оценка параметра, эта трудность рассмотрена в разделе 6.6, а применение к обычной тобит-модели --- в разделе 16.10.2. Корректные формулы для расчёте стандартных ошибок даны в работе Хекмана (1979); см. также Грин (1981). Вывод формулы для случая простой тобит-регрессии рассмотрен в разделе 16.10.2. Реализация формулы может вызвать трудности, поэтому лучше воспользоваться готовыми возможностями статистического пакета или использовать бутстрэп.

Полученная оценка $\beta_2$ состоятельна . Несмотря на потерю эффективности по сравнению с ММП-оценкой при предпосылки о совместном нормальном распределении ошибок, метод популярен по ряду причин: 1) прост в применении; 2) применим к моделям самоотбора выборки, в частности, к моделям из раздела 16.7; 3); 3) используются более слабые предпосылки о распределении, чем  совместное нормальное распределения случайных ошибок $\varepsilon_1$ и $\varepsilon_2$; и 4) предпосылки о распределении можно ослабить еще больше и воспользоваться полупараметрическими методами, см. раздел 16.9.

Предпосылка, которая всегда должна выполняться --- это (16.35), в общем виде может быть записана:

\begin{equation}
\varepsilon_2=\delta\varepsilon_1+\xi,
\end{equation}

где $\xi$ независима от $\varepsilon_1$. Эта модель довольно осмысленна. Согласно этой модели, например, случайная ошибка в уравнении расходов на товары длительного пользования кратна ошибке в уравнении принятия решения о покупке с учетом некоторого шума. Фактически уравнение (16.41) --- это линейная регрессия для случайных ошибок.
Учитывая предположение (16.41) условное среднее (16.34) можно записать как:

\begin{equation}
\E[y_2|y_1^{*}>0]=x_2'\beta_2+\delta \E[\varepsilon_1|\varepsilon_1>-x_1^{*}\beta_1].
\end{equation}

Если $\varepsilon_1$ имеет стандартное нормальное распределение, тогда мы получаем (16.37), основание для МНК регрессии (16.40).

В общем случае, уравнение (16.42) можно оценить методом Хекмана, если ошибки $\varepsilon_1$ имеют распределение, отличное от нормального; см. например Олсен (1980). Вместе с тем, можно полупараметрические методы, при этом  не накладываются ограничения на функциональную форму $\E[\varepsilon_1|\varepsilon_1>-x_1'\beta_1]$ (см. раздел 16.9). 

\subsection{Идентификация}

Теоретически, модель двумерного самоотбора с нормально распределенными ошибками может быть идентифицирована без дополнительных ограничений на регрессоры. В частности, одни и те же регрессоры могут использоваться для обеих переменных $y_1^{*}$ и $y_2^{*}$. 

Вместе с тем, если в модели с нормально распределенными ошибками в обоих уравнениях используются ровно одни и те же регрессоры, модель близка к  неидентифицируемой. Если $x_1=x_2$, тогда, используя (16.37) и вывод из раздела 16.3.2 о том, что обратное отношение Миллса, $\lambda(\cdot )$, можно приближенно описать как линейное по параметрам, $\E[y_2|y_1^{*}]{\simeq}x_2'\beta_2+a+bx_2'\beta_1$. Идентичность регрессоров приводит к мультиколлинеарности. Анализу мультиколлинеарности посвящено много работ среди которых Навата (1993), Навата и Нагасэ (1996) и Льюнг и Ю (1996). Мультиколлинеарность можно обнаружить используя параметр обусловленности, приведенный в разделе 10.4.2, где как следует из (16.40) регрессорами являются $x_2$ и $\lambda(x'_1\hat{\beta}_1)$. Проблема оказывается менее ярко выраженной, если велика изменчивость $x'_1\hat{\beta}_1$, т.е. если пробит-модель может хорошо различать участников и неучастников.

Полупараметрическая версия двухшаговой процедуры Хекмана (см. раздел 16.9.3) требует дополнительного исключающего ограничения. То есть идентификация в модели двумерного самоотбора достигается за счет предположений о функциональной форме.

Таким образом на практике может потребоваться чтобы в модели двумерного самоотбора один из регрессоров уравнения участия ($y_1^*$) не включен в уравнения наблюдаемой зависимой переменной ($y_2^*$). Например, фиксированные издержки, не зависящие от количества часов работы будут влиять на решение работать или нет, но не будут влиять на количество рабочих часов. Это исключение регрессоров может быть довольно сильным ограничением в некоторых приложениях, см. раздел 16.6, т.к. зачастую трудно сформулировать осмысленные ограничения.

\subsection{Предельные эффекты}

Предельные эффект в модели двумерного самоотбора зависят от того, рассматриваем ли мы среднее скрытой переменной, усеченное среднее из (16.37) или цензурированное среднее (там где оно имеет смысл).

Удобно обозначить за $x$ объединение регрессоров $x_1$ и $x_2$, и переобозначить $x_1'\beta_1$ как $x_1'\gamma_1$ и $x_2'\beta_2$ как $x_2'\gamma_2$. Например, усеченное среднее записывается как $\E[y_2|x]=x'\gamma_2+\sigma_{12}\lambda(x'\gamma_1)$. Заметим что в $\gamma_1$ и $\gamma_2$ будут нулевые элементы если $x_1\neq x_2$. Дифференцируя по $x$  получаем предельные эффекты:

Для нецензурированного среднего:
\begin{equation}
\partial \E[y^*_2|x] /\partial x=\gamma_2
\end{equation}

Для усеченного среднего (в нуле):
\[
\partial \E[y_2|x,y_1=1] /\partial x=\gamma_2-\sigma_{12}\lambda(x'\gamma_1)(x'\gamma_1+\lambda(x'\gamma_1))
\]

Для цензурированного среднего (в нуле):
\[
\partial \E[y_2|x] /\partial x=\gamma_1 \phi(x'\gamma_1)x'\gamma_2+
\Phi(x'\gamma_1)x'\gamma_2-\sigma_{12}x'\gamma_1 \phi(x'\gamma_1)\gamma_1
\]

где $\lambda(z)=\phi(z)/\Phi(z)$ и мы пользуемся тем, что $\partial \phi(z)/\partial z=-z\phi(z)$ и $\partial \lambda(z)/\partial z=-z\phi(z)/\Phi(z)-\phi(z)^2/\Phi(z)^2=-\lambda(z)(z+\lambda(z))$. Интерпретация этих производных аналогична обсуждавшейся ранее в Разделе 16.5.3. Как было отмечено, анализ цензурированного среднего корректен если $y_2$ принимает нулевое значение при $y_1=0$. В приложениях рассматриваемых далее, например для случая логнормальных расходов на здоровье, цензурированное среднее не используется.

\subsection{Самоотбор по наблюдаемым и ненаблюдаемым переменным}

Существует много ситуаций, которые могут быть проанализированы с помощью двухшагового процесса принятия решений, сначала принимается решение об участии, а затем принимается решение об уровне участия. Эти решения тесно связаны между собой и зависят от общих факторов. Естественной моделью для таких ситуация является модель двумерного самоотбора, (16.29)-(16.31).

После включения регрессоров ошибки ($\varepsilon_1$ и $\varepsilon_2$) в двух уравнениях могут оказаться некоррелированы. Например, в модели госпитализации, после учета индивидуальных характеристик может оказаться, что отсутствует корреляция между ошибкой в уравнении отвечающем за госпитализацию и ошибкой в уравнении, определяющем длину госпитализации. В этом случае анализ упрощается, т.к. самоотбор происходит только на основе наблюдаемых данных, например, (16.37) упрощается до $\sigma_{12}=0$. Два явления могут моделироваться независимо друг от друга и может быть использована более простая двухчастная модель из раздела 16.4.

В других случая ошибки могут быть коррелированы даже после введения регрессоров. Например, в модели предложения труда, ненаблюдаемые факторы способствующие тому, что индивид принимает решение работать, могут также способствовать тому, что он принимает решение работать дольше, чем предсказывается наблюдаемыми регрессорами. Можно протестировать наличие данной корреляции между ошибками. Если корреляция есть, то на сцену выходят методы данной главы. Требуются довольно сильные предположения о распределении даже для двухшаговой процедуры Хекмана.

Исследование Дуана и др. (1983), изложенное в 16.4.2, критиковали за использование двухчастной модели, более ограничивающей, чем модель с самоотбором выборки. Работа вызвала широкую дискуссию, список релевантных статей можно найти в работе Льюнга и Ю (1986), которые подчеркивают важную роль корреляции обратного отношения Миллса и регрессоров.

Модели с самоотбором выборки, такие как модель с двумерным самоотбором, могут рассматриваться как модели, в которых самоотбор возможен как по наблюдаемым, так и по ненаблюдаемым регрессорам. Часто говорят про модели с самоотбором по ненаблюдаемым переменным, опуская неявно самоотбор по наблюдаемым переменным. Эта глава как раз адресована анализу этих моделей.

Если же самоотбор происходит только по наблюдаемым переменным, анализ существенно упрощается. В качестве примера можно рассмотреть двухчастную модель из этой главы. Глава 25 об эффектах воздействия сосредоточена на самоотборе по наблюдаемым переменным (см. обсуждение в разделе 25.3.3). 

\section{Модель самоотбора выборки: оценка затрат на здоровье}

Для иллюстрации метода используем данные Эксперимента по страхованию здоровья корпорации RAND (Rand Health Insurance Experiment). Данные взяты из статьи Деба и Триведи (2002), авторы использовали модель счетных данных для анализа посещений амбулаторных больных к лечащему врачу и другим специалистам. В разделе 20.3 обобщаются данные, а в разделе 20.7 представлены выводы по некоторым стандартным счетным моделям.


В этом разделе модель строится для ежегодных затрат на здоровье. Регрессоры определены в Таблице 20.4 и могут быть разбиты на несколько групп: медицинское страхование (LC, IDP, LPI и FMDE), социально-экономические характеристики (LINC, LFAM, AGE, FEMALE, CHILD, FEMCHILD, BLACK и EDUCDEC) и индикаторы состояния здоровья (PHYSLIM, NDISEASE, HLTHG, HLTHF и HLTHP). В главе 20 анализ строится на данных за 4 года, в этом примере данные взяты только для второго года, что дает 5574 наблюдения и описательные статистики похожи на результаты в Таблице 20.4, но в точности не совпадают.

Зависимая переменная $y$ обозначает ежегодные затраты индивида на здоровье. При построении эконометрической модели необходимо учитывать два фактора: 1) затраты на здоровье равны нулю для $23.2\%$ наблюдений и 2) распределение $y$ скошенно вправо, среднее значение равно $221 $долл. при медиане $53 $долл. Логарифмирование затрат позволяет понизить скошенность, для логарифма  среднее равно $4.07$, а медиана $3.96$; статистика скошенности сокращается от 24.0 до 0.3. Куртозис равен $3.29$, что примерно равно нормальному значению, $3$. 

Далее основное внимание будет сосредоточено на моделировании $\ln{ y}$, где $y$ положительная величина. Затраты на здоровье можно также оценивать используя двухчастная модель, см. раздел 16.4.2. и модель двумерного самоотбора, см. раздел 16.5.2., где $y_1$ в (16.29) индикатор положительного значения затрат, а $y_2$ в (16.30) равен $\ln{ y}$. Следует отметить, что при $y_1=0$ $y_2$ не имеет смысла, поскольку значение $\ln{ 0}$ неопределено. Двухчастная модель  --- это частный случай модели двумерного самоотбора, когда $\sigma_{12}=0$, (16.32).



\begin{table}[h!]
\caption{\label{tab:16.1} Затраты на здоровье: результаты оценок двухчастной модели и модели самоотбора выборки}
\begin{center}
\begin{tabular}{lcccccc}
\hline
\hline
& \multicolumn{2}{c}{Двухчастная} & \multicolumn{2}{c}{Двухшаговая, самоотбор} & \multicolumn{2}{c}{ММП, самоотбор} \\
Модель & DMED & LNMED & DMED & LNMED & DMED & LNMED \\
\hline
LC & -0.119 (-4.41) & -0.016 (-0.52) & -0.119 (-4.41) &   -0.028 (-0.70)  & -0.107 (-4.03) & -0.076 (2.25) \\
IDP & -0.128 (-2.45) & -0.079 (-1.28) & -0.128 (-2.45) & -0.028 (-0.70) &  -0.109 (-2.13) & -0.150 (-2.26) \\
LPI & 0.028 (3.19)  &  0.003 (0.28)  &  0.028 (3.19)  & 0.005 (0.47)  & 0.029 (3.42)  & 0.015 (1.42) \\
FMDE & 0.008 (0.47)  &  -0.031 (-1.69) &   0.008 (0.47)  & -0.030 (-1.62) &  0.001 (0.05) & -0.024 (1.21) \\
PHYSLIM & 0.273 (3.67)  &  0.262 (3.81)  &  0.273 (3.67)  &  0.281 (3.50) & 0.285  (3.94) & 0.355 (4.70) \\
NDISEASE  & 0.022 (6.25)  &  0.020 (5.78)  &  0.022 (6.25)  & 0.022 (4.29) & 0.021  (6.03) & 0.029 (7.54) \\
HLTHG &  0.039 (0.88)  &  0.144 (2.97)  &  0.039 (0.88)  &  0.147 (3.01) & 0.058  (1.35) & 0.156 (2.99) \\
HLTHF & 0.192 (2.29)  &  0.364 (4.13)  &  0.192 (2.29)   &   0.382 (3.98) & 0.224  (2.75) & 0.445 (4.66) \\
HLTHP & 0.640 (3.01)  &  0.787 (4.63)  &  0.640 (3.01)   &   0.833 (4.22) & 0.798  (3.90) & 0.999 (5.32) \\
$\rho$ & & 0.000 & & 0.168 & & 0.736 \\
$\sigma_2$ & &  & & 1.401 & & 1.570 \\
$\sigma_{12}=\rho \sigma_2$ & & 0.000 & & 0.236 (0.47) & & 1.155 (16.43) \\
$-\ln L$ & \multicolumn{2}{c}{10184.1} & & & \multicolumn{2}{c}{10170.1} \\
\hline
\hline
\end{tabular}
\end{center}
\end{table}

В скобках приведены $t$-статистики. Регрессоры также включают социо-экономические характеристики. DMED --- это индикатор положительности расходов на здоровье. LNMED --- это логарифм расходов, если они положительны. В случае двухшаговой модели $t$-статистики второго шага основаны на ошибках, учитывающих получение оценок обратного отношения Миллса на первом шаге.


В таблице 16.1 показаны оценки коэффициентов для группы факторов медицинского страхования и индикаторов состояния здоровья. Оценки коэффициентов группы социально-экономических характеристик сокращены для краткости изложения.

Для начала сравним результаты двухчастной модели и модели двумерного самоотбора с двухшаговой процедурой оценки. Оценки DMED идентичны, поскольку используется пробит-модель с одними и теми же регрессорами. Вместе с тем, оценки LMED отличаются, поскольку в модели двумерного самоотбора на втором шаге в МНК регрессию добавляется предсказанное значение обратного отношения Миллса. Этот дополнительный регрессор статистически незначим $(t=0.47)$ и его абсолютное значение мало, а $\hat{\rho}$ близко к нулю и равно $0.168$. В результате две модели дают аналогичные оценки коэффициентов для LNMED.


Как было отмечено в разделе 16.4.4 двухшаговый метод может не дать хороших результатов, если обратное отношение Миллса сильно коррелировано с другими регрессорами. В рассматриваемом примере результаты корректны, поскольку прогнозное значение вероятности может меняться в диапазоне от $0.15$ до $0.99$, а индекс обусловленности (см. раздел 10.4.4) для регрессоров на втором шаге	увеличивается лишь в два раза после добавления обратного отношения Миллса. Несмотря на то, что для улучшения качества оценок, как правило, накладываются ограничения исключения на параметры, для рассматриваемого примера трудно установить какие именно факторы  из DMED уравнения следует исключить из уравнения LNMED.

ММП-оценка модели двумерного самоотбора выборки, значительно отличается от двух предыдущих, как для DMED, так и для LNMED уравнения. Случайные ошибки скрытых переменных DMED и LNMED высоко коррелированы, оценка корреляции  равна  $\hat{\rho}=0.736$ и коэффициент корреляции статистически значим $(t=16.43)$. 
Большая разница между оценками $\sigma_{12}$ (или $\rho$), полученными двухшаговой процедурой и ММП-спасобом, может означать неадекватность модели двумерного самоотбора. Отвергание нулевой гипотезы теста Хаусмана (см. раздел 8.4) согласно которой оценки параметров имеют одинаковый предел по вероятности, может проинтерпретировать как отвержение гипотезы о совместном нормальном распределении ошибок, необходимой при переходе от двухшаговой процедуры к оцениванию ММП. 
Также возникает более фундаментальный вопрос об адекватности предпосылки (16.41) и предпосылки о независимых и нормально распределенных ошибках $\varepsilon_1$. Неустойчивость довольно часто присуща моделям двумерного самоотбора, особенно, если одни и те же параметры используются в обеих частях модели. Кроме того, в модели затрат на здоровье предпосылка о нормальном распределении остатков может не выполняться из-за вероятности больших выбросов. Несмотря на то, что скошенность LNMED близка к нулю, а куртозис близок к $3$, стандартные тесты на гетероскедатичность, скошенность и куртозис отвергают нулевую гипотезу о нормальном распределении LNMED, с P-значением равным 0.000.


\section{Модель Роя}


В модели двумерного самоотбора отдельное значение зависимой переменной может быть не наблюдаемо. Таким образом,  $y_2$ наблюдаема, если $y_1=1$ и возможно полностью не наблюдаемо при $y_1=0$. В этом разделе рассмотрим модель, в которой $y_2$ наблюдаема для всех индивидов, но только в одном из двух возможных состояний. Эта важная модель подчеркивает роль фактов, опровергающих гипотезу, а также связана с оценкой эффективности программ, подробно рассмотренной в Главе 25. 

\subsection{Модель Роя}

В одной из самых популярных работ Роя, 1951 анализируется распределение зарплат в зависимости от вида деятельности (внимание уделено и среднему и дисперсии), индивиды отличаются навыками, а также индивиды выбирают род деятельности самостоятельно. Изложение являлось довольно неформальным, и не использовался математический аппарат. Вместе с тем, предполагалось, что производительность индивида, принадлежащего к определенной профессии, имеет лог-нормальное распределение при отсутствии возможности выбора. Вместе с тем, формальная модель не оценивалась. В 1970-е г. ряд авторов, независимо друг от друга, разработали аналогичные модели, оценка которых строилась на пространственных данных, а самоотбор происходи и по наблюдаемым и по ненаблюдаемым переменным. Эти модели получили название модели Роя. 

Определим типовую форму модели Роя. Значение скрытой переменной $y^* _1$ определяет значение наблюдаемой переменной, $y^*_2$ или $y^*_3$. В частности, мы наблюдаем знак $y_1^{*}$

\begin{equation}
y_1=
\begin{cases}
	1, & \text{ если } y_1^{*}>0, \\
	0, & \text{ если } y_1^{*}\leq0,
\end{cases}
\end{equation}

и наблюдаем ровно одно из значений $y_2^{2}$ и $y_2^{3}$:

\begin{equation}
y=
\begin{cases}
	y_2^{2}, & \text{если $y_1^{*}>0$,} \\
	y_2^{3}, & \text{если $y_1^{*}\leq0$,}
\end{cases}
\end{equation}

Как правило определяют линейную модель с аддитивной ошибкой для скрытых переменных,

\begin{equation}
y_1^{*}=x_1^{*}\beta_1=\varepsilon_1,
\end{equation}

\[
y_2^{*}=x_2^{*}\beta_2=\varepsilon_2,
\]

\[
y_3^{*}=x_2^{*}\beta_3=\varepsilon_3.
\]

Модель с аддитивным эффектом имеет вид $x_3'\beta_3=x_2'\beta_2+\alpha$. Самая простая параметрическая модель с коррелированными случайными ошибками получается, если ошибки имеют совместное нормальное распределение с параметрами:

\begin{equation}
\begin{bmatrix}
\varepsilon_1 \\ \varepsilon_2 \\ \varepsilon_3
\end{bmatrix}
 \sim N
\left[
	\begin{bmatrix}
	0 \\ 0 \\ 0
	\end{bmatrix},
	\begin{bmatrix}
	1 & \sigma_{12} & \sigma_{13} \\
	\sigma{12} & \sigma_2^{2} & \sigma_{23} \\
	\sigma_{13} & \sigma_{23} & \sigma_3^2
	\end{bmatrix}
\right]
\end{equation}

где значение дисперсии $\sigma^2_1$ нормируется к единице, поскольку известен только знак $y_1^{*}$.

Вид функции максимального правдоподобия похож на модель двумерного самоотбора (раздел 16.5), за исключением того, что $y_3^{*}$ наблюдается при $y_1^{*} \leq 0$, следовательно в (16.33) выражение  $\Pr[ y_{1i}^{*} \leq 0]$ заменяется на $f(y_{3i}|y_{1i}^* \leq 0){\times}\Pr[ y_{1i}^{*} \leq 0].$

Как правило, модель с аддитивными ошибками оценивают используют двухшаговый метод Хекмана, который применяется для оценки математического ожидания усеченных данных:

\begin{multline}
\E[y|x,y_1^* > 0]=x_2'\beta_2+\sigma_{12}\lambda(x_1'\beta_1), \\
\E[y|x,y_1^*\leq 0]=x_3'\beta_3-\sigma_{13}\lambda(-x_1'\beta_1)
\end{multline}

где $\lambda(z)=\phi(z)/\Phi(z)$ и $\sigma_1^{2}=1$. На первом шаге, при оценивании пробит модели для $y_1^{*}>0$, рассчитываются оценки $\beta_1$ и $\lambda(x_1'\hat{\beta}_1)$. Далее рассчитываются $\beta_2, \sigma_{12}$ и $(\beta_3,\sigma_{13})$ с помощью двух отдельных МНК регрессий. Оценки дисперсий $\sigma_2^2$ и $\sigma_3^2$ можно вычислить используя значения квадратов остатков регрессий, по аналогии с техникой в модели двумерного самоотбора, применяемой после (16.40). Маддалла (1983, стр. 225) детально изучил особенности этой модели, назвав её моделью переключающейся регрессии с эндогенным переключением. В работе Амэмия (1985, стр.399) модель обозначена как тобит-модель пятого типа.

\subsection{Вариации модели Роя}

Многие модели принадлежат классу моделей Роя. В работе Маддалла (1983, глава 9) дается много ссылок на модели самоотбора. Также см. Амэмия (1985, глава 10). Далее мы рассмотрим несколько показательных примеров.

К частному случаю модели Роя можно отнести модель двумерного самоотбора, когда обнуляется $y_3$ и модель строится только для среднего усеченных данных $\E[y_2^{*}|y_1^{*}>0]$. Более показательным примером может быть модель двумерного самоотбора, где $y=0$ при $y_1^{*} \leq 0$, например, модель предложения труда, когда $y=y_2^{*}$ или $y=0$, следовательно, $y_3^{*}=0$.

В исследовании Л.-Ф. Ли (1978), $y_2^{*}$ и $y_3^{*}$ обозначают заработную плату по профсоюзным ставкам и не по профсоюзным ставкам, соответственно, и $y_1^{*}$ стремление вступить в профсоюз. Следовательно, появляется дополнительное уравнение

\[
y_1^{*}=y_2^{*}-y_3^{*}+z'\gamma+\zeta,
\]

где $z'\gamma+\zeta$ отражает затраты на членство в профсоюзе и эта составляющая очень по духу близка к работе  Роя (1951). Заменяя $y_2^{*}$ и $y_3^{*}$, получим приведенную форму для $y_1^{*}$:

\[
y_1^{*}=(x_2'\beta_2-x_3'\beta_3+z'\gamma)+(\varepsilon_2-\varepsilon_3+\zeta).
\]

Эта модель идентична  ранее полученной модели,  корректирующим членом $\lambda(x_1'\hat{\beta}_1)$ получаемым путем построения пробит-модели $y_1$ на $x_1$  на первом шаге, где $x_1$ обозначает единственные регрессоры в $x_2$, $x_3$ и $z$.

Если константа может принимать два значения, отстоящих например, на $\alpha$, тогда  модель Роя можно записать с помощью двух латентных переменных: 

\[
y_1^{*}=x_1'\beta_1+\varepsilon_1,
\]

\[
y^* =x'\beta+{\alpha}y_1+\varepsilon,
\]

где $y=y^* $ всегда наблюдается, а бинарная переменная $y_1$ равна единице, если $y_1^{*}{\geq}0$ и нулю, в ином случае. Эта модель может быть отнесена к моделям с эндогенной дамми переменной $(y_1)$. Модель может быть оценена с применением двухшаговой процедуры Хекмана к $\E[y^*|x]$. Кроме того, для оценки может использоваться метод инструментальных переменных, при наличии инструментов для $y_1$. При этом требуется наличие регрессора, которые не определяет уровень результата, но определяет какой из результатов был выбран.

Рассмотренные модели Роя похожи на модели, рассмотренные в литературе по оценке эффективности воздействия. Существует два варианта исхода, $y_2^{*}$ и $y_3^{*}$, но мы можем наблюдать только один из них. Подход, обозначенный в этой главе, основан на более сильных предпосылках о характере распределения ненаблюдаемых переменных. В главе 25 представлены альтернативные методы, особое внимание стоит обратить на раздел 25.3, в котором проводится сравнение подходов.

\section{Структурные модели}

Особенность моделей  самоотбора выборки заключается в том, что исход частично зависит от решения об участии, которое в свою очередь зависит от ожидаемого исхода. При этом, решение об участии и желаемый исход определяются одновременно. В предыдущих главах для описания этой взаимозависимости использовалась приведенная форма уравнения участия. В частности см. изложение работы Ли (1978) в разделе 16.7.2. Использование приведенной формы --- часто используемый прием, хотя и менее эффективен, чем использование полностью структурной версии модели. 

В этом разделе подробно анализируются структурные экономические модели, в основе которых лежит максимизация полезности, и структурные статистические модели, позволяющие обобщить системы одновременных уравнений чтобы учесть влияние цензурирования и усечения, включая модели для бинарных исходов.

\subsection{Структурные модели, построенные на принципе максимизации полезности}

Изначально, структурные модели использовали для анализа предложения женского труда. Модель из учебника по экономике включает описывает индивидов, максимизирующих полезность от потребления благ и времени досуга, при условии бюджетного ограничения и временного ограничения, состоящего в том, что всё доступное время делится между досугом и работой.
Для решения во внутренней области предельная норма замещения (MRS) работы досугом равна ставке заработной платы. Однако, угловое решение, когда женщины решают не работать, может возникнуть, если MRS превышает ставку заработной платы. Гронау (1973) и Хекман (1974) представили эконометрические модели, согласующиеся с моделями максимизации полезности. Модели Гронау и Хекман похожи на тобит-модели; они учитывают тот факт, что для неработающих женщин размер предлагаемой зарплаты неизвестен. В последующих вариантах добавляются фиксированные издержки работы, что приводит к  моделям самоотбора выборки, а также используются панельные данные, что приводит к панельной тобит-модели. Исследование этих моделей проводили Киллингсворс и Хекман (1986), а также Бланделл и Маккарди (2001), а практическое применение продемонстрировал Мроз (1987).


Для иллюстрации структурного подхода приведем следующий пример. Дубин и МакФадден (1984) построили модель потребления электроэнергии домашними хозяйствами (ДХ) (электричества или природного газа) и выбора электрического прибора (например, электрической или газовой печи) как взаимосвязанных решений, принимаемых исходя из функции полезности. Так, для $j$-го из $m$ бытовых приборов косвенная функция полезности отдельного домашнего хозяйства имеет вид:

\begin{equation}
V_j=\lbrace\alpha_1/\beta+\alpha_1{p_1}+\alpha_2{p_2}+w'\gamma+\beta(y-r_j)+\eta\rbrace{e}^{-{\beta}p_i}+\varepsilon_j,
\end{equation}

где $p_1$ и $p_2$ цены на электроэнергию и газ, $y$ доход ДХ и $r_j$ усредненные ежегодные затраты на энергию $j$-го прибора:

\[
r_j=p_1q_{1j}+p_2q_{2j}+\rho{c}_j,
\]

где $q_{1j}$ и $q_{2j}$ количество потребляемого газа и электроэнергии прибором $j$, $c_j$ затраты на прибор и $\rho$ ставка дисконтирования. Отличия в предпочтениях ДХ определяется наблюдаемой переменной $w$, ненаблюдаемой переменной $\eta$ и случайной ошибкой $\varepsilon_j$ для каждого прибора; $\varepsilon_j$ независимы между собой, но коррелированы с $\eta$. Вместе с тем, существует <<фактор вкуса>> $\alpha_{0j}$.


Спрос на энергию для $j$-го прибора равен $-(\partial{V_j}/\partial{p_1})(\partial{V_j}/\partial{y})$, используя тождество Роя, мы получаем


\[
x_1-q_{1j}=\alpha_{0j}+\alpha_{1}p_1+\alpha_2{p_2}+w'\gamma+\beta(y-r_j)+\eta.
\]

Чтобы подчеркнуть, что выбор прибора $j$ задан эндогенно, рассмотрим $m$ индикаторных переменных $\delta_{jk},k=1,\ldots ,m$, где

\[
\delta_{jk}=
	\begin{cases}
	1, & \text{если }k=j\\
	0, & \text{если }k{\neq}j
	\end{cases}
\]

Тогда спрос на электроэнергию $x_1$ при заданном приборе $j$  определяется следующим выражением:

\begin{equation}
x_1-q_{1j}=\sum_{k=1}^m 
\sum\alpha_{0k}\delta_{jk}+\alpha_{1}p_1+\alpha_{2}p_2+w'\gamma+\beta\left(y-\sum_{k=1}^{m}{r_j\delta_{jk}}\right)+\eta.
\end{equation}

Даже если модель (16.50) линейна по параметрам, МНК-оценки будут несостоятельны из-за эндогенного характера $\delta_{jk}$. Дубин и МакФадден (1984) рассмотрели два варианта алгоритма оценки.

Подход инструментальных переменных позволяет оценить (16.50), используя в качестве инструментов для $\delta_{jk}$ и $r_j\delta_{jk},k=1,\ldots ,m$ переменные $\hat{p}_k$ и $r_{j}\hat{p}_k$, соответственно, где $\hat{p}_k$ предсказанные значения вероятности выбора заданного прибора. $V_j$ обозначает косвенную функцию полезности, которая состоит из детерминистической функции $U_i$ и стохастического компонента. Такая запись функции соответствует представлению $U_i$ в разделе 15.5.1 с помощью модели полезности с аддитивными случайными ошибками (additive random utility model, ARUM). Согласно аналогичному подходу, 

\[
p_k=\Pr[ V_k>V_l,l{\neq}k,l=1,\ldots ,m]
\]

\[
=\Pr[ \varepsilon_l-\varepsilon_k<\lbrace(\alpha_{0k}-\alpha_{0l})-\beta(r_k-r_l)\rbrace{e^{-\beta{p_1}}},\text{ для всех } l{\neq}k]
\]

\[
=\dfrac{\exp [(\alpha_{0k}-\beta{r_k})e^{-\beta{p_1}}\pi/\lambda\sqrt{3}]}{\sum_{l=1}^{m} \exp [(\alpha_{0l}-\beta{r_l})e^{-\beta{p_1}}\pi/\lambda\sqrt{3}]},
\]


при предпосылке, что $\varepsilon_j, j=1,\ldots ,m$ независимо и имеют одинаковое распределение экстремальных значений второго типа с функцией распределения $F(\varepsilon)=\exp (-\exp (-\gamma-\varepsilon\pi/\lambda\sqrt{3}))$, где $\gamma{\approx}0.5772$ константа Эйлера. В данном примере среднее значение $\varepsilon_j$ равно нулю, а дисперсия равна $\lambda^2/2$, что отличается от способа описания распределения экстремальных значений второго рода, используемого в Главе 14 и 15. Оценка нелинейной мультомиальной логит-модели дает предсказанные значения вероятности $\hat{p}_k$.

В альтернативном методе самоотбора выборки $\E[\eta|j-\text{ый прибор выбран}]{\neq}0$ и для расчета математического ожидания используются предпосылки о распределении $\eta$ и $\varepsilon_1,\ldots ,\varepsilon_m$. В частности, предположим, что $\eta|\varepsilon_1,\ldots ,\varepsilon_m$ независимы и одинаково распределены со средним $\sqrt{2}\sigma/\lambda\sum_{k=1}^{m}R_{k}\varepsilon_k$ и дисперсией $\sigma^{2}(1-\sum_{k=1}^m R_k^2)$, где $\sum_{k=1}^m R_k=0$ и $\sum_{k=1}^m R_k^2<1$ и распределение $\varepsilon_k$ было определено ранее. После математических преобразований, см. в работу Дубина и МакФаддена, получим значение 

\[
\E[\eta|j-\text{ый прибор выбран}]{\neq}0=\sum_{k \neq j}^m
(\sigma\sqrt{6}R_k/\pi)[\dfrac{p_k\ln{ p_k}}{1-p_k}+\ln{ p}].
\]

Тогда приходим к оцениванию с помощью МНК: 

\[
x_1-q_{1j}=\sum_{k=1}^m a_{0k}\delta_{jk}+\alpha_1p_1+\alpha_2 p_2 +w'\gamma +
\beta \left(y-\sum_{k=1}^m r_j\delta_{jk}\right)+
\sum_{k\neq j}^m \gamma_k 
\left[ \frac{\hat{p}_k \ln \hat{p}_k}{1-\hat{p}_k} +\ln \hat{p}_k \right] +\xi
\]

где $\hat{p}_k$ предсказанное значение вероятности $p_k$ и $\xi$ ошибка с асимптотическим средним равным нулю.


Дубин и МакФадден оценили эти модели, используя наблюдения по $3,249$ ДХ для двух вариантов благ: электрический нагрев воды и обогрев помещения и газовый. 

Ханеманн (1984) аналогичным образом моделировал уровень потребления брендовых товаров,  индивиды покупают товар только одной марки из множества, а Кэмерон (1988) моделировал спрос на медицинское обслуживание в зависимости от выбора конкретного медицинского полиса из нескольких возможных.

Некоторая изворотливость может потребоваться, чтобы сконструировать модель, приводящую к аналитическому выражению для вероятности выбора и для величины спроса при условии выбора услуги, как в примере Дубина и МакФаддена. Вычислительные методы детально рассмотрены в 12-ой и 13-ой главах. Эти методы позволяют определить оценки модели, даже когда отсутствует аналитическое решение. Тем не менее, результат всегда зависит от функциональной формы полезности и распределения ненаблюдаемых переменных.

\subsection{Системы одновременных уравнений в тобит- и логит-моделях}

Для иллюстрации вопросов, возникающих при обобщении систем одновременных уравнений, рассмотренных в разделе 2.4, возьмем модель самоотбора выборки с двумя латентными переменными и зададим уравнение для латентной переменной. Общий вид такой модели:

\begin{equation}
\begin{array}{l}
y_1^{*}=\alpha_1y_2^{*}+\gamma_1y_1+\delta_1y_2+x'_1\beta_1+\varepsilon_1,\\
y_2^{*}=\alpha_2y_1^{*}+\gamma_2y_1+\delta_2y_2+x_2'\beta_2+\varepsilon_2,
\end{array}
\end{equation}

где $y_1^{*}$ и $y_2^{*}$ частично наблюдаемы, но они определяют наблюдаемые значения $y_1$ и $y_2$, и ошибки имеют совместное нормальное распределение. Например, может быть известно значение индикатора, $y_1=1$, если $y_1^{*}>0$ и значение переменной $y_2=y_2^{*}$, если $y_1^{*}>0$. Теоретически и латентная переменная, или  наблюдаемая зависимая или обе  могут выступать в качестве регрессоров, хотя для идентификации модели необходимы ограничения, описанные ниже.

\subsubsection*{Эндогенные скрытые переменные}


Самое простое ограничение: только латентные переменные могут быть регрессорами в (16.51). Тогда,

\begin{equation}
\begin{array}{l}
y_1^{*}=\alpha_{1}y_2^{*}+x_1'\beta_1+\varepsilon_1,\\
y_2^{*}=\alpha_{2}y_1^{*}+x_2'\beta_2+\varepsilon_2.
\end{array}
\end{equation}

Примером является модель двумерного самоотбора с эндогенными скрытыми переменными (16.31), она получается при добавлении  дополнительного условия $\alpha_2=0$ и записана в приведенной а не в структурной форме по $y_1^*$. Модель (16.52) можно легко оценить, поскольку приведенная форма $y_1^{*}$ и $y_2^{*}$ легко получаемся способом, который используется для обычной системы одновременных уравнений. Параметры приведенной форме можно легко оценить, используя пробит или тобит-модель, в зависимости от способа расчета $y_1$ и $y_2$ при заданных значениях $y_2^{*}$ и $y_1^{*}$. Оценки параметров структурной формы (16.52) могут быть рассчитаны через замену регрессоров $y_2^{*}$ и $y_1^{*}$ на прогнозы из приведенной формы $\hat{y_2^{*}}$ и $\hat{y_1^{*}}$. 

Модели типа (16.52) получили название систем одновременных тобит-моделей. Если зависимые переменные $y_1$ и $y_2$ бинарны, тогда (16.52) --- двумерная пробит-модель. Способы оценки рассмотрены в работах Нельсон и Olson (1978), Амэмия (1979), а также в статье Ли, Маддала и Троста (1980) и общий подход представил Л.-Ф. Ли (1981). Стандартные ошибки рассчитываются с использованием двухшаговой процедуры для М-оценок. Однако, намного проще рассчитать стандартные ошибки методом бутстрэпа, см. раздел 11.2. Идентификация требует наложения ограничений на систему уравнений (16.51) по аналогии с линейной системой уравнения.

\subsubsection*{Эндогенные регрессоры}

Популярным вариантом модели (16.52) является тобит-модель с полностью наблюдаемыми эндогенными переменными. Тогда  $y_2^{*}$ полностью наблюдаема, т.е. $y_2=y_2^{*}$, в то время как $y_1=y_1^{*}$, если $y_1^{*}>0$ и $y_1=0$, в иных случаях. Модель можно записать:

\begin{equation}
\begin{array}{l}
y_1^{*}=\alpha_1{y_2}+x_1'\beta_1+\varepsilon_1 \\
y_2=x'\pi+\nu 
\end{array}
\end{equation}

где первое уравнение ---  структурное, а второе --- приведенная форма для эндогенного регрессора $y_2$. Вновь отметим, что в этом примере $y_2$ непрерывна. Для совместно нормально распределенных ошибок $\varepsilon_1=\gamma\nu+\xi$, где $\xi$ независимые нормально распределенные ошибки (см. раздел 5.1),  $y_1^{*}=\alpha_1{y_2}+x_1'\beta_1+\gamma\hat{\nu}+e_1$.

В двухшаговой процедуре оценки сначала рассчитываются остатки  $\hat{v}=y_2-x'\hat{\pi}$ 
в регрессии МНК $y_2$ на $x$, а затем оценивается тобит-модель

\[
y_1^*=\alpha_1 y_2 + x'_1\beta_1+\gamma \hat{v}+e_1,
\]



где $e_1$ нормально распределены. Для тестирование на эндогенность $y_2$ можно использовать тест Вальда с нулевой гипотезой $\gamma=0$, а стандарнтные ошибки взять из тобит-модели. Данный тест является расширением вспомогательной регрессии, реализуемой в тесте эндогенности Хаусмана в линейной модели (см. Раздел 8.4.3). Если нулевая гипотеза отвергается, то на упомянутом втором шаге тобит-регрессии получаются состоятельные оценки $\alpha_1$ и $\alpha_2$, но стандартные ошибки нужно скорректировать из-за дополнительного регрессора $\hat{v}$ на первом шаге. См. работу Смита и Бланделла (1986) для подробностей по тобит-модели и Риверса и Вуонга (1988), где изложена аналогичная процедура с оценкой пробит-модели на втором шаге.


\subsubsection*{Эндогенные цензурированные или бинарные переменные}

Анализ усложняется, если цензурированные или бинарные эндогенные переменные $y_1$ или $y_2$ выступают в роли регрессоров в (16.51). Хекман (1978) проанализировал следующую модель:

\begin{equation}
\begin{array}{l}
y_1^{*}=\gamma_1{y_1}+\delta_1{y_2^{*}}+x_1'\beta_1+\varepsilon_1, \\
y_2^{*}=\alpha_2{y_1^{*}}+\gamma_2{y_1}+x_2'\beta_2+\varepsilon_2,
\end{array}
\end{equation}

где наблюдается $y_1=1$, если $y_1^{*}>0$, и $y_1=0$, если $y_1^{*}{\geq}0$ и $y_2=y_2^{*}$ всегда наблюдаема. Модель усложняется тем, что $y_1$ выступает в роли регрессора. В приведенной форме для $y_1^{*}$ регрессорами могут быть только $x_1$ или $x_2$. Следовательно, должно выполняться условие согласованности, $\delta_1\gamma_2+\gamma_1=0$. Если это условие выполняется, тогда приведенная форма примет вид:

\[
\begin{array}{l}
y_1^{*}=x'\pi_1+\nu_1,\\
y_2=\gamma_2{y_1}+x'\pi_2+\nu_2.
\end{array}
\]

Это частный случай модели Роя, где участие ($y=1$) приводит только к сдвигу зависимой переменной (через $\gamma_2$). В целом, модели с цензурированными или усеченными эндогенными переменными трудно оценивать. Например, см. Бланделл и Смит (1989).

\subsubsection*{Пример}

Брукс, Кэмерон и Картер (1998) применили систему одновременных тобит-моделей для объяснения результатов голосования представителей конгресса в поддержку производителей сахара. Голос конгрессмэна (за или против), денежную поддержку от производителей сахара и денежную поддержку от производителей сахарозаменителей обозначим  $y_1, y_2$ и $y_3$, соответственно; $y_1$ -- бинарная переменная, а $y_2$ и $y_3$ цензурированы в нуле. Авторы составили систему одновременных уравнений для $y_1^{*}$, $y_2^{*}$ и $y_3^{*}$, таким образом, структурная модель имеет простейший вид (16.52).

Насколько приемлема данная спецификация? В рассматриваемом примере $y_2^{*}$ и $y_3^{*}$ должны зависеть от скрытой переменной $y_1^{*}$, поскольку фактический результат голосования  будет известен позже финансирования. С $y_1^{*}$ сложнее, т.к. данная латентная переменная зависит от фактических значений вкладов $y_2$ и $y_3$, а не от латентных денежных вкладов. Однако, если данная ситуация рассматривается как повторяемая игра, тогда можно использовать скрытые переменные $y_2^{*}$ и $y_3^{*}$. Очевидно, что обоснованность данного предположения будет зависеть от конкретного случая. Идентификация параметров модели обеспечивается исключением некоторых экзогенных регрессоров. Оценки параметров будут состоятельны, если ошибки действительно имеют совместное нормальное распределение.

\subsection{Полупараметрическое оценивание}

Предыдущие методы решали проблему частично пропущенных данных с помощью предположения об их законе распределения. При этом получалась либо функция правдоподобия, либо функция цензурированного, усеченного среднего или среднего с учетом самоотбора.


Оценки уязвимы даже к незначительным погрешностям в предпосылках о распределении случайных ошибок. Например, оценки, полученные методом максимального правдоподобия или с помощью двухшаговой процедуры Хекмана в стандартной тобит-модели несостоятельны, если нарушается либо предпосылка о нормальности, либо предпосылка о гомоскедастичности. См. работу Паарш (1982) и ссылки в ней.

Много исследований было посвящено развитию полупараметрических методов, состоятельных при менее строгих предпосылках. Перед изложением полупараметрических методов отметим, что в качестве альтернативы можно продолжать использовать полностью параметрические методы, в основе которых лежат менее строгие предпосылки о распределении.

\subsection{Гибкие параметрические модели}

Для начала возьмем классическую тобит-модель $y_i^{*}=x_i'\beta+\varepsilon_i$. Возможны два варианта ослабления предпосылки $\varepsilon_i \sim N[0,\sigma_i^2]$. Во-первых, можно допускать гетероскедастичность, $\sigma_i^2=\exp (z_i'\gamma)$, тогда нужно оценить оба параметра, $\beta$ и $\gamma$. Во-вторых, можно использовать менее строгое распределение. Например, полиномиальное разложение нормального распределения (см. раздел 9.7.7).

Для модели двумерного самоотбора выборки может использоваться аналогичный подход, т.е. можно предположить более широкий класс для совместного  распределении случайных ошибок $(\varepsilon_1,\varepsilon_2)$. Ли (1983) предложил использовать некие преобразованные $(\varepsilon_1^{*},\varepsilon_2^{*})$ вместо $(\varepsilon_1,\varepsilon_2)$ для которых предпосылка о двумерном нормальном распределении может быть более реалистична. 

Можно также использовать байесовские методы. Чиб (1992) рассматривал цензурированную тобит-модель. Автор рассматривает скрытую переменную $y^* $ как вспомогательную и использовал метод пополнения данных (см. раздел 13.7). При сэмплировании по Гиббсу последовательно генерируются  (1) условное апостериорное для $\beta|y,y^* ,\sigma^2$, (2) условное апостериорное для $\sigma^2|y,y^* ,\beta$ и (3) апостериорное для $y^* |y,\beta,\sigma^2$.

Гибкий параметрический подход хорошо подходит для анализа нелинейных цензурированных, усеченных регрессий и нелинейных регрессий самоотбора для счетных данных, данных по длительности и смешенных данных, поскольку в этом случае трудно найти полупараметрический метод.


\subsection{Полупараметрические методы оценки цензурированных регрессий}


Рассмотрим линейную модель для скрытой переменной $y_i^{*}=x_i'\beta+\varepsilon_i$, цензурированной слева  в нуле, т.е. $y_i=y_i^{*}$, если $y_i^{*}>0$ и $y_i=0$, если $y_i^{*}{\geq}0$. Как правило полупараметрическая модель имеет вид:

\begin{equation}
y_i=\max(x_i'\beta+\varepsilon_i,0).
\end{equation}

Это тобит-модель (16.11)-(16.13), за исключением того, что не указан закон распределения $\varepsilon$. С небольшими изменениями эту модель можно использовать для данных цензурированных слева но не в нуле, или для данных цензурированных справа. Например, если $y=\min(x'\beta+\varepsilon,U)$, тогда $U-y=\max(U-x'\beta_\varepsilon,0)$. Цель состоит в получении состоятельных оценок без предположения о законе распределения $\varepsilon_i$. Методы оценки называются полупараметрическими, поскольку неусеченное среднее $x_\beta'$ параметризировано, а распределение случайных ошибок --- нет. Представленные ниже методы отличаются по предпосылкам о распределении $\varepsilon$. 

Согласно (16.8) ММП оценивание возможно, если известна функция  распределения $y^* $ и, следовательно, $\varepsilon$. Функция распределения $\varepsilon$ может быть состоятельно оценена с использованием оценки Каплана-Майера, приведенной в главе 17  для цензурированных справа данных по длительности. Также, непараметрическая оценка распределения $\varepsilon$ может быть получена с помощью  разложения в ряд Галланта и Нычки (1987); см. раздел 9.7.7. Эти полупараметрические методы максимального правдоподобия редко используются.

В литературе наиболее часто встречается расчёт оценки через условные моменты. Из (16.20) следует, что формула условного цензурированного среднего $\E[y|x]$ соответствует одноиндексной модели с $\E[y|x]=g(x'\beta)$, где функция $g(\cdot )$ неизвестна, если распределение $\varepsilon$ не указано. Модели для одноиндексных моделей  (см. раздел 9.7.4) можно применять, однако, как сказано в разделе, $\beta$ может быть оценена только с точностью до сдвига и масштаба. 

В более популярном подходе рассматриваются условные цензурированные моменты, которые менее изменчивы при цензурировании. Пауэлл (1984) предложил использовать условную медиану. Главная предпосылка состоит в том, что медиана $\varepsilon|x$ равна нулю, следовательно, условная медиана $y|x$ равна условному среднему $x'\beta$. Интуицию метода Пауэлла легче всего понять, предположив, что $y$ независимы и одинаково распределены. Если меньше половины значений выборки цензурировано, то меньше половины выборки принимают нулевые значения, значит больше половины выборки положительны, тогда выборочная медиана цензурированных значений является состоятельной оценкой медианы по совокупности. 
Пауэлл (1984) применил эту идею к регрессионному анализу, действуя по такой же логике для тех наблюдений, у которых меньше половины значений $\varepsilon|x$ цензурированы, где $\varepsilon=y-x'\beta$ и для расчета используются оцененные значения $\beta$.  Тогда регрессионный анализ производится по аналогии с оценкой медианы по методу наименьших абсолютных отклонений (см. раздел 4.6). В результате получается цензурированная оценка метода наименьших абсолютных отклонений (censored least absolute deviation, CLAD) минимизирующая:

\begin{equation} 
Q_N(\beta)=N^{-1}\sum_{i=1}^N|y_i-\max(x_i'\beta,0)|.
\end{equation}

Необходимым условием состоятельности оценок является равенство медианы $\varepsilon|x$ нулю. При выполнении этого условия оценки будут состоятельны, даже если  ошибки условно гетероскедастичны. Оценка $\beta$ является $\sqrt{N}$ состоятельной и асимптотически нормальной. Эффективность оценки может возрасти, если взвешивать слагаемые в сумме с помощью $f(0|x_i)$, значений условной плотности $\varepsilon_i|x_i$ в нуле. Цензурированный метод наименьших абсолютных отклонений может также использоваться для расчета условных квантилей.

Вместо медианы можно использовать симметрично усеченное среднее, значение которого также не меняется при цензурировании. Пусть $\varepsilon|x$ симметрично распределено. Тогда, для наблюдений с положительным средним (т.е. $x'\beta>0$) $y|x$ имеет симметричное распределение на интервале $(0,2X'\beta)$. Следовательно, с равной вероятностью $x'\beta+\varepsilon+<0$ и $y=0$ или $x'\beta+\varepsilon>2x'\beta$ и все данные искусственно приравниваются к $2x'\beta$ для сохранения симметрии относительно $x'\beta$. В результате:

\begin{equation}
\E[1(x'\beta>0)[\min(y,2x'\beta)-x'\beta]x]=0,
\end{equation}

где $1(x'\beta)>0$ ограничение, согласно которому используются только наблюдения с положительным средним и зависимая переменная или $y=0$, или $0<y<2x'\beta$, или $2x'\beta$ если $y>2x'\beta$. Моментная оценка  по формуле (16.57) не дает единственного решения для $\beta$. Пауэлл (1986b) предложил симметричный цензурированный метод наименьших квадратов (symmetric censored least squares, SCLS):

\begin{equation}
Q_N(\beta)=N^{-1}\lbrace[y_i-\max(y_i/2,x'\beta)]^{2}+1(y_i>2x_i'\beta)[y_i^2/4-\max(0,x_i'\beta)]^2\rbrace.
\end{equation}

Можно показать что выражение (16.58) является выборочным аналогом условий первого порядка для моментов (16.57). Чей и Оноре (1998) описывает цензурированный МНК графически. Также они приводят графическое описание оценивания с помощью попарных разностей  Оноре и Пауэлла (1994). 

Меленберг и Ван Сует (1996), Чей и Оноре (1998) и Чей и Пауэлл (2001) продемонстрировали практические примеры для некоторых из этих методов. Паган и Улла (1999) приводят дополнительныме методы и излагают теориюы.

Рассмотрим пример использования цензурированного метода наименьших абсолютных отклонений для оценки данных тобит-модели с нормальными ошибками. Оценка параметра наклона, при истинном  значении 1000, получилась равной 956 (стандартная ошибка равна 117) при использовании ММП и равной 838 (стандартная ошибка 165) по методу цензурированных наименьших абсолютных отклонений. Как и предполагалось, устойчивость цензурированного метода наименьших абсолютных отклонений к ненормальности ошибок достигается за счет снижения  эффективности оценок.

\subsection{Полупараметрическая оценка для моделей самоотбора}

Полупараметрическая оценка моделей самоотбора выборки вызывает больше сложностей. В качестве примера, рассмотрим часто встречающуюся модель --- двумерную модель самоотбора выборки ---, определение которой было дано в разделе 16.5.2, однако здесь опускается предположение о совместном нормальном распределении ошибок $\epsilon_1,\epsilon_2$.


Возможно найти полупараметрическую ММП-оценку. В частности, Галлант и Нычка (1987) в явном виде рассматривали  двумерную  модель самоотбора выборки как подходящую для использования их метода с использованием разложения в ряд, см. раздел 9.7.7.

Вместе с тем, в литературе в качестве отправной точки, как правило, используют выражение для усеченного условного среднего, которое, согласно (16.34), равно: 

\begin{equation}
\E[y_{2i}|x_i,y_{1i}^{*}>0]=x_{2i}^{*}\beta_2+\E[\varepsilon_2|\varepsilon_1>-x_{1i}'\beta_1]\\
=x_{2i}'\beta_2+g(x_{1i}'\beta_1),
\end{equation}

по аналогии с (16.41), предполагается, что распределение $\varepsilon_{2i}|x_i,\varepsilon_{1i}$ зависит только от $x_{1i}$. Поскольку распределение $(\varepsilon_1,\varepsilon_2)$ не указано,  функция $g(\cdot )$ неизвестна, следовательно требуется применение полупараметрических методов. Поскольку возможно, что $g(x_1'\beta_1)=x_1'\beta_1$ идентификация модели с неспецифицированной функцией $g()$ требует исключающего ограничения, а именно, хотя бы один из регрессоров в $x_1$ не должен входить в $x_2$. Чем менее коррелированы $x_1'\beta_1$ и $x_2$ тем лучше разделяются $\beta_2$ и $g()$. Модель (16.59) является частично линейной и может быть оценена с помощью методов из раздела 9.7.3. Популярными методами являются метод Робинсона (1998а), оценка с помощью взятия разностей, или разложение функции $g(x_1'\beta_1)$ в ряд. Поскольку $\beta_1$ неизвестно, используется регрессия $y_{2i}$ на $x_{2i}'\beta_2+g(x_1'\beta_1)$, где $\beta_1$ может быть получено путем построения регрессии $y_{1i}$ на $x_{1i}$ с использованием одной из полупараметрических оценок раздела 14.7. Эти методы дают состоятельную оценку для $\beta_2$. Чтобы в дополнение к коэффициенту наклона в зависимости $y_2$ состоятельно оценить константу можно обратить внимание на работу Эндрюса и Шафгенса (1998). 

Ньюи, Пауэлл и Уолкер (1990) применили данный подход к исследованию предложения труда женщин. Модель для индикатора участия была оценена разными методами, а уравнения для $y_2$ было оценено методом Робинсона (1988а). Меленберг и Ван Суент моделировали расходы на путешествия используя разнообразные полупараметрические методы и для двумерных моделей самоотбора и для цензурированных регрессий. Широкий класс моделей рассматривают Дас, Ньюи и Велла (2003).

Мански (1989) изучал идентификацию в двумерной модели самоотбора при довольно минимальных предположениях и приводит границы для среднего и предельных эффектов при заданных значениях регрессоров и условии участия.



\section{Вывод тобит-модели}

\subsection{Моменты стандартного нормального распределения для усеченных данных}

Пусть $z \sim N[0,1]$, плотность распределения равна $\phi(z)=(1/\sqrt{2\pi})\exp (-z^2/2)$ и функция распределения обозначена $\Phi(z)$. Поскольку $\Pr[ z>c]=1-\Phi(c)$, условная плотность $z|z>c$ равна $\phi(z)/(1-\Phi(c))$. Следовательно,


\begin{multline}
\E[z|z>c]=\int_c^{\infty}z(\phi(z)/[1-\Phi(c)])\,dz\\
=\int_c^{\infty} z (1/\sqrt{2\pi})\exp (-z^2/2) \,dz /[1-\Phi(c)]\\
=\int_c^{\infty}
\dfrac{\partial}{\partial{z}}
\left(-(1/\sqrt{2\pi})\exp (-z^2/2)\right)
dz/
[1-\Phi(c)]\,dz\\
=\left[ -(1/\sqrt{2\pi})\exp (-z^2/2) \right]_c^{\infty}/ [1-\Phi(c)]\\
=\phi(c)/[1-\Phi(c)].
\end{multline}

Аналогичным образом, 

\begin{multline}
\E[z^2|z>c]=\int_c^{\infty}z^2(\phi(z)/[1-\Phi(c)])\,dz \\
=\int_c^{\infty} z{\times}z \times (1/\sqrt{2\pi}\exp (-z^2/2)/[1-\Phi(c)]\,dz \\
=\int_c^{\infty}z{\times}\dfrac{\partial}{\partial{z}}\left(-(1/\sqrt{2\pi})\exp (-z^2/2)\right)dz/[1-\Phi(c)]\,dz \\
=\left[z{\times}-(1/\sqrt{2\pi})\exp (-z^2/2)\right]_c^{\infty}/[1-\Phi(c)] \\
-\int_c^{\infty}z{\times}\dfrac{\partial}{\partial{z}}(z)\left(-(1/\sqrt{2\pi})\exp (-z^2/2)\right)dz/[1-\Phi(c)]\,dz\\
=c\phi(c)/[1-\Phi(c)]+(1-\Phi(c))/[1-\Phi(c)] \\
=c\phi(c)/[1-\Phi(c)]+1.
\end{multline}

После некоторых математических преобразований, 


\begin{align}
V[z|z>c]=\E[z^2|z>c]-(\E[z|z>c])^2 \\
=1+c\phi(c)/[1-\Phi(c)]-\phi(c)^2/[1-\Phi(c)]^{2}.
\end{align}



\subsection{Асимптотика двухшаговой процедуры Хекмана к тобит-модели}

Получение  асимптотической ковариационной матрицы для метода Хекмана осложнено зависимостью  от оценок параметров, полученных на первом шаге. Существует несколько способов расчета асимптотической ковариационной матрицы, ряд из которых рассмотрен в работе Амэмия (1985, стр.369-370). В этом разделе в основное расчета лежат общие методы для ковариационной матрицы при использовании двухшаговой М-оценки, ранее рассмотренной в разделе 6.6. В качестве примера приведем самый простой случай для тобит-модели (см. раздел 16.3.6). Метод Хекмана можно применить к двухшаговой оценке моделей двумерного самоотбора (см. раздел 16.5.4) и системе одновременных уравнений тобит-модели. Наиболее простым способ является парный бутстрэп (см. раздел 11.2).

Оценим параметр $\gamma=[\beta'\sigma]$ в (16.26) из модели для положительных значений $y_i$:

\[
y_i=x'_{i}\beta+\sigma\lambda(x'_{i}\alpha)+\eta_i
\]

\[
=w_i(\alpha)'\gamma+\eta_i,
\]

где $w_{i}(\alpha)={[{x'}_i \,\,\, \lambda_i({x'}_i\alpha)]}'$ и $\eta_i=y_i-{x'}_i\beta-\sigma\lambda({x'}_i\alpha)$ гетероскедастичны с дисперсией $\sigma^2_{\eta i}$, которая определена в (16.24). На первом шаге находим оценку $\hat{\alpha}$ неизвестного параметра $\alpha$ ММП для пробит-модели. Двухшаговая процедура Хекмана может быть записана двумя уравнениями:

\begin{equation}
\sum_{i=1}^N(y_i-\Phi({x'}_i\alpha))
\frac{\phi^2({x'}_i\alpha)}{\Phi({x'}_i\alpha)(1-\Phi({x'}_i\alpha))}x_{i}=0
\end{equation}


\[
-\sum_{i=1}^N d_iw_i(\alpha)(y_i-w_i(\alpha)'\gamma) =0,
\]

где первое уравнение есть условие первого порядка в пробит-модели для оценки $\alpha$ и второе уравнение есть условие первого порядка МНК для оценки  $\gamma$ при положительных значениях $y_{i}(d=1)$.

В общем виде эти уравнения можно записать: $\sum_{i=1}^Nh(x_i,\theta)=0$, где $\theta=(\alpha',\gamma')'$. При обычном разложении в ряд Тейлора до первой степени $\hat{\gamma}-\gamma\overset{d}{\to} N[0,G^{-1}_0S_0(G^{-1}_0)']$, где $G_0=\lim N^{-1}\E[\sum_{i=1}^N\partial h(x_i,\theta)/\partial\theta]$ и $S_0=\lim N^{-1}\E[\sum_{i=1}^N  h(x_i,\theta)h(x_i,\theta)]'$. Нас интересуют компоненты относящиеся к $\gamma$. Выражение можно упростить, поскольку матрица $\partial h(x_i,\theta)/\partial\theta$ оказывается блочно-треугольной, поскольку $\gamma$ отсутствует в первом блоке уравнений. Разделение дает следующий результат:

\[
V[\hat{\theta}_2]=
G_22^{-1}\left\lbrace S_22+G_21[G_11^{-1}S_11G_11^{-1}]G_21'-G_{21}G_{11}'S_{12}-S_{21}G_{11}^{-1}G_{21}'\right\rbrace{G_{22}^{-1}},
\]

где матрицы определены в разделе 6.6.

Применительно к нашей задаче, сначала рассмотрим составляющие в  $G_0$. Тогда
\[
G_{11}=\lim\dfrac{1}{N}\sum_{i=1}^N \dfrac{\phi^2(x_i'\alpha)}{\Phi(x_i'\alpha)(1-\Phi(x_i'\alpha))}x_ix_i'
\]

\[
G_{21}=\lim\dfrac{1}{N}\sum_{i=1}^N d_iw_i\dfrac{\partial\lambda(x'_{i}\alpha)}{\partial\alpha},
\]

\[
G_{22}=\lim\dfrac{1}{N}\sum_{i=1}^N \E[d_{i}w_iw'_i].
\]

Для выражения $G_11$ используется факт того, что $G^{-1}_{11}$ равно дисперсии ММП оценки пробит-модели. Выражение $G_{21}$ получается при использовании 

\[
\E\left[\dfrac{{\partial}h_{2i}}{\partial\theta'_{i}}\right]=\E \left[-\dfrac{\partial d_{i}w_{i}(\alpha)(y_{i}-w_{i}(\alpha)')\gamma}{\partial{\alpha}}\right]
\]

\[
=\E\left[w_{i}\dfrac{{\partial}d_{i}w_{i}(\alpha)}{\partial\alpha'}\right]
\]

\[
=\E\left[d_{i}w_{i}\dfrac{\partial\lambda(x'_{i}\alpha)}{\partial\alpha}\right].
\]

Выражение для $G_{22}$ получено с использованием

\[
\dfrac{\partial h_{2i}}{\partial\theta'_{2}}=\dfrac{{\partial}d_{i}w_{i}(\alpha)(y_i - w_{i}(\alpha)'\gamma)}{\partial\gamma}=d_{i}w_{i}w'_i
\]

Переходя к $S_0$, получим:

\[
S_{11}=G^{-1}_{11},
\]

\[
S_{21}=0
\]

\[
S_{22}=\lim\dfrac{1}{N}\sum^{N}_{i=1}\E[d_{i}(y_i - w_{i}(\alpha)'\gamma)^2]
\]

Для получения выражения $S_{11}$ применяем равенство информационной матрицы. Взяв математическое ожидание и сделав несколько манипуляций, получим, что $S_{21}=0$ и $S_{22}$ равно $V[\eta_i]$.

Объединив эти результаты, можно рассчитать двухшаговую оценку Хекмана $\hat{\gamma} \overset{a}{\sim} N(\gamma,V_{\gamma})$, где

\begin{equation}
\hat{V}_{\gamma}=(\hat{W}'\hat{W})^{-1}(\hat{W}'\Sigma_{\hat{\eta}}\hat{W}+\hat{W}'\hat{D}\hat{V}_{\alpha}\hat{D}\hat{W})(\hat{W}'\hat{W})^{-1},
\end{equation}

и, где, $\hat{W}'\hat{W}=\sum^{N}_{i=1}d_{i}\hat{w}_{i}\hat{w}'_{i},\hat{D}=\mathrm{Diag}[\partial\lambda(x'_{i}\alpha)/\partial\alpha|_{\hat{\alpha}}]$, $\hat{V}_{\alpha}$ ковариационная матрица для ММП оценки пробит-модели на первом шаге, а $\Sigma_{\hat{\eta}}$ --- диагональная матрица, в которой значения на диагонали равны $\hat{\sigma}^{2}_{\eta_i}$. Значение $\hat{\sigma}^{2}_{\eta_i}$. Можно легко рассчитать, если программа предусматривает расчет матриц. Сложнее всего рассчитать $\sigma^{2}_{\eta_i}=V[\eta_i]$, данное в (16.24). При возникновении трудностей в расчете $\sigma^{2}_{\eta_i}=\V[\eta_i]$, возможно использовать подход Уайта (1980) и использовать $\hat{\sigma}^{2}_{i}=(y_{i}-x'_{i}\hat{\beta}+\hat{\sigma}\lambda_{i}(x'_{i}\hat{\alpha}))^2$

\section{Практические соображения}

В большинстве статистических пакетов предусмотрена ММП оценка тобит-модели при выполнении предпосылки о нормальности распределения. Двухчастную модель легко оценить, поскольку можно по-отдельности оценить каждую часть. Двумерная модель самоотбора может быть оценена по двухшаговой процедуре Хекмана, используя пробит и МНК. Тем не менее, стандартные ошибки трудно рассчитать через пробит и МНК из-за двухшаговой процедуры и намного проще использовать пакет, где двухшаговая процедура Хекмана уже реализована. Как правило, использование полупараметрических методов требует написания специальной программы в языке программирования, например, можно использовать GAUSS. В некоторых статистических пакетах предусмотрена возможность ММП оценки цензурированных или усеченных данных, например модели Пуассона или отрицательного биномиального распределения для счетных данных.

Цензурирование и усечение не составляют проблемы, если закон распределения переменных верно специфицирован. Например, цензурированные сверху данные могут быть легко использованы для оценки параметров, если логнормальное распределение хорошо согласовывается с данными. Также может использоваться цензурированный метод наименьших абсолютных отклонений, предпосылки о распределении в котором менее строгие. 

Больше проблем возникает с оценкой моделей самоотбора выборки. Большинство моделей этого типа полагаются на строгие предпосылки о распределении. Полупараметрические методы сталкиваются с требованием к идентифицируемости модели, состоящем в том, что переменная, которая определяет участие, не должна определять значение зависимой переменной. Более перспективный метод, часто используемый в литературе по эффектам воздействия, заключается в рассмотрении только случаев, когда самоотбор происходит только  на основе  наблюдаемой переменной.

\section{Библиографические заметки}

Существует много литературы по моделям самоотбора выборки. Подробный список источников дают  в своих работах Маддала (1983) и Гурьеру (2000), более краткий список приводят Амэмия (1984,1985) и Грин (2003). 

16.3 Тобин (1958) предложил и применил тобит-модель к анализу расходов. Амэмия (1973) формально обосновал состоятельность оценок и их нормальное асимптотическое распределение. Хекман (1974) применил тобит-модель к анализу предложения женского труда с последующим детальным анализом результатов.

16.4. Многие исследования Эксперимента по страхованию здоровья корпорации RAND (Rand Health Insurance Experiment), например, работа Дуана и др. (1983) могут послужить примером двухчастной модели.

16.5 Хекман (1976, 1979) рассмотрел двухшаговый метод оценки двумерной модели самоотбора, который лежит в основе многих полупараметрических способов оценки. Мроз (1987) применяет процедуру к анализу предложения труда женщин и показал роль принятия предположения об экзогенном характере заработной платы.

16.7. Существует много интерпретаций идей Роя (1951), также как много вариантов тобит-модели. Л.-Ф. Ли (1978) оценил разницу между оплатой труда работников, состоящих и не состоящих в профсоюзе.

16.8. Работа Дубина и МакФаддена (1984) является одним из основных примеров структурного микроэконометрического анализа, построенного на полной спецификации функции полезности и распределения ненаблюдаемых значений.

16.9 Полупараметрический метод оценки моделей двумерного самоотбора детально рассмотрен в книгах М.-Дж. Ли (1996), Пагана и Улла (1999), обзорах  Велла (1998) и Л.-Ф. Ли (2001). Вместе с тем, Чей и Оноре (1998), а также Чей и Пауэлл (2001) приводят примеры использования моделей цензурированных данных, а Меленберг и Ван Сует (1996) примеры использования двумерной  модели самоотбора. 


\section{Упражнения}

\begin{enumerate}
\item [$16 - 1$] В упражнении рассматривается как влияет степень усечения на тобит-модель.
\begin{enumerate}
\item Сгенерируйте 2000 значений скрытой переменную $y^* =k+3x+u$, где $u \sim N[0,3]$ и $x \sim U[0,1]$. Выберите такое $k$, чтобы примерно $30\%$ значений $y^* $ были бы отрицательны.

\item Сгенерируйте усеченную или цензурированную подвыборку, отбрасывая наблюдения для которых $y^* <0$

\item Оцените модель методом наименьших квадратов для 2000 наблюдений, предполагая, что скрытая переменная наблюдаема. Сравните полученные результаты с теорией, учитывая, что была сделана одна репликация.

\item  Оцените модели МНК используя усеченную снизу подвыборку, т.е. $y>0$.

\item Найдите оценки параметров, используя метод максимального правдоподобия для усеченных данных. Прокомментируйте результаты, учитывая теоретические свойства ММП. Сравните полученные результаты с выводами в (c) и (d)

\item  Повторите шаги (a)-(f) для других $k$ чтобы цензурировалось $20\%, 40\%$ и $50\%$ наблюдений. Сравните результаты с выводами при $30\%$ цензурируемых наблюдений. На основе полученных результатов, сделайте вывод о влиянии цензурирования высокого порядка на значения оценок параметров. Подкрепите ваши аргументы теорией.
\end{enumerate}

\item [$16 - 2$] Рассмотрите модель скрытой переменной $y_i^{*}=x_i'\beta+\varepsilon_i$, $\varepsilon_i \sim N[0,\sigma^2]$. Предположим, что $y_i^{*}$ цензурирована сверху, т.е. мы наблюдаем $y_i=y_i^{*}$, если $y_i^{*}<U_i$ и $y_i=U_i$, если $y_i^{*}\geq U_i$, верхний предел $U_i$ известная константа для каждого индивида и может меняться от индивида к индивиду.
\begin{enumerate}
\item  Запишите логарифмическую функцию максимального правдоподобия. (Подсказка: Функция имеет вид отличный от стандартной функции максимального правдоподобия, поскольку добавляется  $U_i$ и равенства соответствуют случаю $y_i=y_i^{*}$, если $y_i^{*}<U_i$)

\item  Запишите выражения для усеченного среднего $\E[y_i | x_i,y_i<U_i]$. (Подсказка: Для $z \sim N[0,1]$ мы имеем $\E[z|z>c]=\phi(c)/[1-\Phi(c)]$; $\E[z|z<c]=-\E[-z|-z>-c]$ и $-z \sim N[0,1].$)

\item Запишите двухшаговую процедуру Хексмана для оценки модели из пункта (a).

\item  Запишите выражение для усеченного среднего $\E[y_i|x_i]$. (Подсказка: Ответ к пункту (b) существенно поможет)
\end{enumerate}
\item [$16 - 3$] Это упражнение посвящено последствиям неправильной спецификации тобит-модели. Для анализа возьмите модель из упражнения 16.1.
\begin{enumerate}
\item  Сгенерируйте $y^* $ при гетероскедастичных случайных ошибках, т.е. $u \sim N[0,\sigma^2z]$, где $z>0$ и корреляция между $z$ и $x$ ненулевая, но и не равна единице. Подберите $k$ чтобы $30\%$ всех данных оказались цензурированы. Оцените модель цензурированных данных с нормально распределенными ошибками с помощью ММП и сравните полученные результаты с оценками, когда случайные ошибки гомоскедастичны.

\item  Пусть предпосылка о нормальности не выполняется. Используя процедуру Монте-Карло, оцените модель, если количество наблюдений равно 1000 и процедура повторяется 500 раз. Для каждого повтора сгенерируйте цензурированную выборку так, чтобы распределение случайной ошибки было смесью двух распределений  $N[1,9]$ или $N[0.4,1]$ c вероятностями $0.4$ и $0.6$, соответственно. Оцените модель и сравните полученные результаты с результатами, когда предпосылка о нормальности выполняется.
\end{enumerate}
\item [$16 - 4$] Рассмотрим модель пуассоновской регрессии, где вероятность $y^* $ равна $f^{*}(y^* )=e^{-\mu}\mu^{y}/y^* , y_i^{*}=0,1,2,\ldots $ и наблюдения не зависимы по $i$. Из-за ошибки, связанной с записью данных, значение $y^*$ полностью наблюдаемо при $y^* {\geq}2$. Если $y^* =0$ или $1$ известно только, что $y^*  \leq 1$, в данных такие наблюдения записаны как $y^* =1$. Определим наблюдаемые данные, $y=y^* $ для $y_i^{*}{\geq}2$ и $y=1$ для $y_i^{*}=0$ или $1$.
\begin{enumerate}
\item  Найдите вероятность $f(y)$ для наблюдаемых значений $y$.
\item  Найдите $\E[y]$. Здесь требуются некоторые математические преобразования.


Теперь введем регрессоры с $\E[y^*|x]=\exp(x'\beta)$ и определим индикатор $d=1$ для $y^*\geq 0 $ и $d=0$ для $y^*$ равного $0$ или $1$.

\item  Запишите точную формулу для целевой функции правдоподобия, которая позволяла бы рассчитать состоятельные оценки $\beta$, используя данные $y_i$, $d_i$ и $x_i$.


\item  Запишите точную формулу для целевой функции правдоподобия, которая позволяла бы рассчитать состоятельные оценки $\beta$, используя только данные $d_i$ и $x_i$.

\item  Возможно ли получить состоятельные оценки $\beta$, используя только $d_i$ и $x_i$. Поясните ваш ответ.
\end{enumerate}
\item [$16 - 5$]
Используя 50\% данных RAND по затратам на здоровье за 12 месяцев мы хотим дать ответ на широкий вопрос: какая из моделей наилучшим образом подходит для оценки затрат? 
\begin{enumerate}
\item 	Используя описательные статистики по  затратам, проанализируйте к чему приводит большая пропорция нулевых наблюдений. Приводит ли это к нарушению предпосылки о нормальности? Существует ли преобразование, которое бы позволяло сделать предпосылку о нормальности более применимой?
\item 	Проанализируйте три модели, с одинаковыми независимыми переменными. Регрессоры точно такие же как в  упражнении 20.6 по счетным данных. Среди моделей: (i) тобит-модель; (ii) двухчастная модель (модель преодоления порогов) и (iii) модель самоотбора выборки. Опишите соотношения и взаимосвязи между моделями, предложите способ сравнить и выбрать одну из моделей. Если вы предполагаете, что можно столкнуться с проблемами, касающихся спецификации или оценки параметров, обозначьте их и предложите возможные пути решения. Особое внимание обратите на ограничения.
\item 	Оцените все три модели. В двухчастной модели второе уравнение составлено только для индивидов с ненулевыми затратами. Для оценки модели самоотбора выборки используйте ММП-оценка и двухшаговая процедура Хекмана. Изложите ваши соображения по исключающему ограничению, необходимому для идентификации модели. Является ли самоотбор проблемой в данном случае?
\item Каким образом можно сравнить качество подгонки по трем моделям? Какая модель наилучшим образом согласуется с данными? Какие критерии вы используете для сравнения?
\item 	Оцените влияние двух переменных, log income и log (1 + уровень совместного страхования) на уровень затрат на здоровье. Сравните предельные эффекты изменения этих переменных для тобит-модели и двухчастной модели. Подумайте, как наилучшим образом проинтерпретировать полученные результаты для гетерогенной выборки.
\item 	Дайте краткое пояснение, при выполнении каких предпосылок квантильная регрессия (см. раздел 4.6) может быть альтернативным методом оценки. Какие преимущества и недостатки такой альтернативы?
\end{enumerate}
\end{enumerate}



                %%%% Comments and Terms %%%%
                %%%%%%%%%   "Транзитные данные"
                % \url{http://finbiz.spb.ru/download/4_2008_nivorog.pdf}
                %%%%%%%%%   Ениколопов о Treatment groups
                % \url{http://quantile.ru/06/06-RE.pdf}
                %%%%%%%%%   Ненаблюдаемая гетерогенность
                %\url{http://books.google.de/books?id=B-6hxCay0PkC&pg=PA551&lpg=PA551&dq=%22%D0%BD%D0%B5%D0%BD%D0%B0%D0%B1%D0%BB%D1%8E%D0%B4%D0%B0%D0%B5%D0%BC%D0%B0%D1%8F+%D0%B3%D0%B5%D1%82%D0%B5%D1%80%D0%BE%D0%B3%D0%B5%D0%BD%D0%BD%D0%BE%D1%81%D1%82%D1%8C%22&source=bl&ots=V1NuZfl839&sig=LlpEFlSPrgVIc5kT102FqB32lbA&hl=en&sa=X&ei=uw55UpPYI8bHswaYz4HACQ&redir_esc=y#v=onepage&q=%22%D0%BD%D0%B5%D0%BD%D0%B0%D0%B1%D0%BB%D1%8E%D0%B4%D0%B0%D0%B5%D0%BC%D0%B0%D1%8F%20%D0%B3%D0%B5%D1%82%D0%B5%D1%80%D0%BE%D0%B3%D0%B5%D0%BD%D0%BD%D0%BE%D1%81%D1%82%D1%8C%22&f=false} % \url{http://quantile.ru/01/01-Articles.pdf}
                % \url{http://www.quantile.ru/01/01-SA2.pdf}
                %%%%%%%%%   Модели с преобразованием
                % \url{http://quantile.ru/05/05-Issue.pdf}
                %%%%%%%%%   Набор идентичных данных
                % \url{http://www.proz.com/kudoz/english_to_russian/medical%3A_pharmaceuticals/1501766-tied_data.html}
                %%%% %%%%%%%% %%%%

\chapter{Транзитные данные: Анализ выживаемости}\label{ch:17}
% В отечественной литературе нет единой терминологии в отношении моделей выживаемости. Поэтому многие понятия, которые мы будем употреблять в этой и последующих главах можно встретить под другими названиями. Например, анализ выживаемости также называют анализом дожития. Чтобы избежать путаницы, в скобках мы будем указывать англоязычные термины, и иногда приводить альтернативные варианты перевода на русский.
% \textbf{Идея: в конце книги сделать приложение-табличку с англ.терминами и вариантами перевода?}


\section{Введение}
\label{sec:17.1}

\noindent
Эконометрическими моделями времени жизни \emph{(models of durations)} называются модели, которые позволяют анализировать продолжительность пребывания в каком-либо состоянии до перехода к другому. Например, это может быть безработное состояние, состояние жизни или отсутствие страхового полиса. В биостатистике такую продолжительность называют \textbf{временем жизни}, а момент перехода --- \textbf{смертью}. В операционных исследованиях занимаются изучением времени жизни таких объектов, как лампы, приборы или машины, поэтому предметом анализа будет являться срок службы прибора, а моментом перехода --- \textbf{время отказа} \emph{(failure time)}. В эконометрике \textbf{состояние} определяют как класс исследуемого объекта в определенный момент времени, а \textbf{момент перехода} как перемещение объекта из одного состояния в другое. \textbf{Длительность состояния} \emph{(spell length)} или \textbf{время жизни} \emph{(duration)}, называют временем, которое объект находился в данном состоянии. В качестве типовой задачи регрессионого анализа можно рассматривать оценку влияния размера пособия по безработице на длительность безработного состояния или же вероятность перехода в другое.

Литература, посвященная анализу времени жизни, может путать и сбивать читателя с толку по ряду причин.
Во-первых, объектом анализа может выступать как продолжительность пребывания в некотором состоянии, так и вероятность перехода в другое; при этом может использоваться несколько различных функций распределения.
Во-вторых, результаты зависят не только от выбранной модели, но и от метода построения выборки. Например, для данных по безработице можно построить выборку типа поток \emph{(flow sampling)} (мы знаем момент времени, когда индивид получил статус безработного), запас \emph{(stock sampling)} (мы знаем только статус индивида) или же выборку всего населения, независимо от статуса.
В-третьих, данные зачастую являются цензурированными. Это главная причина, по которой следует моделировать переходы, а не среднюю длительность состояния, как обычно принято в регрессионном анализе, поскольку не требуется вводить строгие предположения относительно распределения ошибок для состоятельности оценок.
% ALT: предпочитается моделировать?
В-четвертых, транзитные данные могут включать не одно, а несколько состояний. Например, индивид может быть безработным, работать на пол-ставки, работать полный рабочий день, или вовсе быть исключен из рабочей силы. Более того, данные по конкретному индивиду могут содержать несколько переходов между этими состояниями.
Наконец, методы анализа выживаемости применяются во многих прикладных разделах статистики, в каждом из которых существуют свои принятые названия и обозначения. Так,
    \textbf{анализ времени жизни} или \textbf{перехода} из одного состояния в другое, также называется \textbf{анализом выживаемости} или \textbf{дожития} \emph{(survival analysis)} (т.е. времени, в течение которого объект оставался живым) в биостатистике,
    \textbf{анализом времени отказа} \emph{(failure time analysis)} (т.е. времени до отказа лампочки, детали и т.д.) в операционных исследованиях,
    \textbf{анализом таблицы выживания} \emph{(life table analysis)} в демографии и актуарной математике (где переход в другое состояние означает смерть) и
    \textbf{анализом рисков} \emph{(hazard analysis)} в страховании.
    В прикладных социальных науках также занимаются изучением рецидивов, продолжительности браков, временем до очередных выборов.

Результаты, представленные в данной главе, основаны на данных \textbf{типа поток} с \textbf{единственным переходом}. В качестве классического примера можно рассматривать моделирование момента смерти, известное из анализа выживаемости, или анализа таблицы выживания. Поскольку это является наиболее популярным примером в статистике, соответствующие методы анализа встроены в большинство статистических пакетов. В начале этой главы мы представим эмпирический пример для того, чтобы подчеркнуть некоторые отличительные черты, характерные для данных о выживаемости.

В силу такой специфичности данных мы начнем анализ с определения базовых концепций и описательных характеристик при отсутствии каких-либо объясняющих переменных. В частности, в разделах \ref{sec:17.3}-\ref{sec:17.5} мы рассмотрим такие основные понятия, как риск, кумулятивный риск и функция выживания. В разделе \ref{sec:17.4} мы определим различные типы цензурирования, необходимые для учета ненаблюдаемых переходов. Например, клинические испытания обычно завершаются до того, как умрет последний объект; значит, момент смерти последнего объекта будет неизвестен, или цензурирован. В разделе \ref{sec:17.5} мы представим непараметрические оценки риска, кумулятивного риска (оценка Нельсон-Аалена) и функции выживания (оценка Каплан-Мейера), которые являются состоятельными оценками при предпосылке о независимом цензурировании.

Сохранив эту предпосылку, мы расширим анализ с помощью введения регрессоров. Оценивание полностью параметрических моделей, в частности, модели Вейбулла, представлено в разделе \ref{sec:17.6}. Роль цензурирования аналогична её роли в модели тобит. Некоторые важные модели времени жизни можно найти в разделе \ref{sec:17.7}. Альтернативный полупараметрический подход, предложенный Коксом (1972), заключается в моделировании функции риска, или вероятности смерти при условии, что она не наступила раньше, при относительно слабых предположениях о распределениях параметров. Модель Кокса, ставшая с тех пор стандартной для анализа выживаемости, представлена в разделе \ref{sec:17.8}. В отличие от большинства пространственных моделей, модели времени жизни допускают изменения регрессоров во времени (например, размера пособия по безработице). Соответствующие модели с регрессорами, меняющимися со временем, описаны в разделе \ref{sec:17.9}. Модели рисков в дискретном случае рассмотрены в разделе \ref{sec:17.10}. Наконец, эмпирический пример в разделе \ref{sec:17.11} является иллюстрацией к описанным методам.

В следующих двух главах мы рассмотрим более сложные аспекты моделирования переходов, редко встречающиеся в учебниках. В частности, речь пойдет о ненаблюдаемой гетерогенности и моделях с множественными состояниями.




\section{Пример: Длительность забастовок}\label{sec:17.2}

\begin{figure}[ht!]\caption{Длительность забастовок: оценка Каплан-Мейера функции выживания. Данные по 566 завершенным забастовкам в США в 1968-76 гг.}\label{fig:17.1}
\centering
%\includegraphics[scale=0.7]{fig.png}
\end{figure}

\noindent
Возьмем данные о длительности забастовок, на которых были основаны % которые были использованы в таких работах, как
такие работы как Кеннан (1985), Джаггиа (1991c) и другие. Интересующая нас переменная --- это длительность забастовок в обрабатывающей промышленности США, измеренная в днях с начала забастовки. Выборка включает 566 завершенных (нецензурированных) наблюдений. Средняя продолжительность забастовки ($dur$) составляет 43.6 дней, медиана --- около 28 дней. Однако 90 дней спустя после начала наблюдения все еще продолжаются 88 забастовок.

Графически информацию о длительности забастовок можно представить в виде эмпирической \textbf{функции выживания}. На вертикальной оси рисунка \ref{fig:17.1} отображена доля забастовок, которые продолжаются по прошествии соответствующего количества дней на горизонтальной оси. Поскольку нас интересует только длительность забастовки, а не дата ее начала, на рисунке отсутствуют даты. Как и ожидалось, функция равна единице в нуле и затем постепенно падает до нуля, предполагая, что все забастовки должны рано или поздно закончиться.

Пусть переменная ($z$) показывает отклонение фактического выпуска от тренда, т.е. является индикатором состояния экономики. Тогда положительные значения $z$ означают, что экономика находится выше тренда, отрицательные --- ниже. Узнать, является ли средняя продолжительность забастовки проциклической (т.е. $\partial (dur) / \partial z > 0$) или же контрциклической (т.е. $\partial (dur) / \partial z < 0$) величиной, можно с помощью простой линейной регрессии переменной $\ln (dur)$ на $z$, оценив условное мат.ожидание $\ln (dur)$.

Однако предположим, что мы также хотим узнать вероятность того, что забастовка, которая длится $t$ дней, закончится в $t+1$ день, или же условную вероятность завершения забастовки как функцию от ее длительности, с учетом влияния $z$, построив биномиальную регрессию зависимой переменной, принимающей значения 0 или 1, на независимую переменную $z$. Тогда анализ выживаемости окажется более прямым и эффективным методом, который также позволяет работать с цензурированными данными. В следующем разделе мы рассмотрим статистические методы, применяемые в анализе выживаемости.




\section{Основные понятия}\label{sec:17.3}

\noindent
Длительность состояния является неотрицательной случайной величиной ($T$), зачастую дискретной в экономических данных. Для определения основных понятий мы будем рассматривать непрерывный случай, с последующим переходом к дискретному.


\subsection{Функции выживания, риска и кумулятивная функция риска}\label{sec:17.3.1}

\noindent
Обозначим \textbf{функцию распределения} величины $T$ как $F(t)$, \textbf{функцию плотности}  --- как $f(t) = dF(t) / dt$. Тогда вероятность того, что время жизни, или длительность состояния, окажется меньше $t$, равна
    \begin{align}
    \label{eq:17.1}
    F(t) &= \Pr[T \le t]\\
    &= \int^{t}_{0} f(s)ds. \notag
    \end{align}

Вероятность того, что время жизни окажется больше или равно $t$, называется \textbf{функцией выживания}, задаваемой следующим образом
    \begin{align}
    \label{eq:17.2}
    S(t) &= \Pr[T > t]\\
    &= 1 - F(t). \notag
    \end{align}

Согласно Кальбфляйш и Прентис (2002), определение функции распределения в (\ref{sec:17.1}) соответствует стандартному. Некоторые авторы (например, Ланкастер (1990)) дают иное определение функции распределения: $F(t) = \Pr[T < t]$, и, следовательно, функции выживания: $S(t) = \Pr[T\ge t]$, поскольку функции риска заданы при условии $T\ge t$, а не при $T>t$.
Такие различия в спецификации будут иметь место в дискретном случае (\ref{sec:17.3.2}), непосредственно в момент перехода к другому состоянию. 

Функция выживания монотонно убывает от единицы до нуля, поскольку функция распределения монотонно возрастает с нуля. Если все индивиды, которые подвержены риску перейти в другое состояние, в итоге перейдут в него (т.е. исследуемое событие рано или поздно произойдет), то $S(\infty) = 0$, в противном же случае $S(\infty) > 0$, и распределение длительностей называется несобственным. Выборочное среднее наступившего события является интегралом от функции выживания: $\int^{\infty}_{0} S(u)du$. Чтобы получить такой результат, необходимо проинтегрировать по частям
    $$\int^{\infty}_{0}uf(u)du = \int^{\infty}_{0}udF(u) = uF(u)|^{\infty}_{0} - \int^{\infty}_{0}F(u)du.$$
Поскольку $F(\infty) = 1$, а $F(0) = 0$, то
    \begin{align}
    \label{eq:17.3}
    \mathrm{E}[T] = \int^{\infty}_{0} (1-F(u))du = \int^{\infty}_{0} S(u)du.
    \end{align}
То есть, средняя длительность равна площади под кривой выживания.

\textbf{Функция риска} % в учебнике МКП она называется интенсивностью отказов или коэффициентом смертности
также является одним из наиболее важных понятий в анализе времени жизни. Буквально, она означает мгновенную вероятность перехода к другому состоянию, при условии что объект дожил до некоторого момента времени $t$
    \begin{align}
    \label{eq:17.4}
    \lambda(t) &= \lim_{\Delta t \rightarrow 0} \frac{\Pr [t \le T \le t+ \Delta t | T \ge t] }{\Delta t}\\
    &= \frac{f(t)}{S(t)}. \notag
    \end{align}

    \begin{table}[!ht]\caption{\textit{Анализ выживаемости: Основные определения}}\label{tab:17.1}
    \begin{center}
\begin{tabular}{lccc}
\hline \hline
Функция             & Символ        & Определение                                                    & Отношение\\
\hline
Плотности           & $f(t)$        &                                                                & $f(t)=dF(t)/dt$\\
Распределения       & $F(t)$        & $\Pr[T \le t]$                                                 & $F(t)=\int_{0}^{t} f(s)ds$ \\
Выживания           & $S(t)$        & $\Pr[T>t]$                                                     & $S(t)=1-F(t)$ \\
Риска               & $\lambda(t)$  & $\lim_{h \rightarrow 0} \frac{\Pr[t \le T < t+h|T \ge t]}{h}$  & $\lambda(t)=f(t)/S(t)$ \\
Кумулятивного риска & $\Lambda(t)$  & $\int_{0}^{t}\lambda(s)ds$                                     & $\Lambda(t)=-\ln S(t)$ \\
\hline \hline
    \end{tabular}
    \end{center}
    \end{table}
Легко проверить, что функция риска равна производной от логарифма функции выживания,
    $$\lambda(t) = - \frac{d \ln(S(t))}{dt}.$$
Риск $\lambda(t)$ описывает распределение длительности $T$. В частности, интегрируя $\lambda(t)$ и зная $S(0)=1$, можно показать, что
    \begin{align}
    \label{eq:17.5}
    S(t) = \exp{ \left( -\int^{t}_{0}\lambda(u)du \right)}.
    \end{align}
В регрессионном анализе выживаемости объектом для исследования является условный коэффициент риска $\lambda(t|\mathbf{x})$, в то время как стандартный подход предполагает анализ условного среднего, $\mathrm{E}(T|\mathbf{x})$. Однако он не подходит для моделирования длительностей, если наблюдения являются цензурированными.

Наконец, еще одной важной характеристикой распределения является \textbf{кумулятивная}, или \textbf{интегральная функция риска}
    \begin{align}
    \label{eq:17.6}
    \Lambda(t) &= \int^{t}_{0} \lambda(s)ds \\
    &= - \ln(S(t)), \notag
    \end{align}
где для последнего преобразования используется уравнение \ref{eq:17.5}. Если $S(\infty)=0$, то $\Lambda(\infty)=\infty$. Кумулятивный риск представляет больший интерес, чем обычная функция риска, в силу более высокой точности оценивания.

Для любого $T$ можно показать, что преобразование $\Lambda(T)$ имеет нормированное экспоненциальное распределение, а $\ln\Lambda(T)$ --- распределение экстремальных значений, что позволяет проводить тесты на спецификацию модели (см. раздел 18.7.2
% UNCOMMENT AT THE END OF CHAPTER 18
%\ref{sec:18.7.2.}
).

Основные функции для неотрицательной с.в. $T$ описаны в Таблице \ref{tab:17.1}.

Иногда используются и другие функции, например, \textbf{преобразование Лапласа} $L(s)=\mathrm{E}[\exp{(-sT)}], s>0$, что является примером производящей функции моментов для с.в. $T$ при условии, что она положительна.


\subsection{Дискретные данные}\label{sec:17.3.2}

\noindent
Очень часто длительность события измеряется интервалом. Например, данные могут содержать информацию о неделе, когда наступил переход, но конкретные день и время могут быть неизвестны. В таких случаях говорят, что моменты перехода являются сгруппированными, и предполагается, что риск внутри интервала остается постоянным. К таким данным применяются модели риска в дискретном времени.

Для начала необходимо определить \textbf{функцию риска в дискретном времени} как вероятность перехода в момент $t_j,j=1,2, \ldots .$, при условии, что объект дожил до некоторого времени $t$:
    \begin{align}
    \label{eq:17.7}
    \lambda_j &= \Pr[T = t_j|T \ge t_j]\\
    &= f^{\mathrm{d}} (t_j)/S^{\mathrm{d}}(t_{j-}), \notag
    \end{align}
где надстрочный индекс d обозначает дискретный случай, и $S^\mathrm{d}(a\_)=\lim_{s\rightarrow a\_}{S^d(t_j)}$, поскольку формально $S^\mathrm{d}(t)$ равна $\Pr{[T>t]}$, а не $\Pr{[T\ge t]}$.
% ALT: adjustment made --- скорректирована?

\textbf{Функцию выживания в дискретном времени} можно рекурсивно вывести из функции риска как
    \begin{align}
    \label{eq:17.8}
    S^{\mathrm{d}} &= \Pr[T \ge t]\\
    &= \prod_{j|t_j \le t} (1 - \lambda_j). \notag
    \end{align}
Например, $\Pr{[T>t_2]}$ равна произведению вероятности, что переход не наступит в момент $t_1$, и вероятности, что он не наступит в $t_2$, при условии что объект дожил до момента $t_2$, то есть $\Pr{[T>t_2]}=(1-\lambda_1)\times(1-\lambda_2)$. Функция $S^\mathrm{d}(t)$ является убывающей ступенчатой функцией со скачками в моментах $t_j,j=1,2, \ldots .$
% ALT: разрывами?
% ENG: step function with jumps

\textbf{Кумулятивная функция риска в дискретном времени} равна
    \begin{align}
    \label{eq:17.9}
    \Lambda^{\mathrm{d}}(t) = \sum_{j|t_j \le t} \lambda_j.
    \end{align}

Используя \ref{eq:17.7}, получим, что вероятность наступления перехода в момент $t_j$ в дискретном случае равна $\lambda_j S^\mathrm{d}(t_j).$

Можно обобщить непрерывный и дискретный случаи. Функция выживания определена \textbf{произведением интегралов}, которое сокращается до регулярного произведения (\ref{eq:17.8}) в дискретном случае и экспоненты от регулярного интеграла (\ref{eq:17.5}) в непрерывном. См. Кальбфляйш и Прентис (2002, с. 10), или Ланкастер (1990, с. 10-12).

Заметим, что дискретные данные могут возникать потому, что сам процесс, порождающий переходы, оказывается дискретным. Однако чаще мы все же встречаем непрерывные процессы с дискретными наблюдениями. Например, мы можем знать неделю или месяц наступления перехода, но не конкретный день или час. Такие данные иногда называют \textbf{сгруппированными}. К дискретному случаю можно перейти следующим образом. Разделим период наблюдения на $k + 1$ интервалов $[a_0,a_1), [a_1,a_2),  \ldots , [a_{k-1},a_{k}), [a_{k},a_{\infty})$. Длительность $T=t_j$ обозначает переход к интервалу $[a_{j-1},a_{j})$, т.е. переход в момент $a_{j-1}$ или позже. Таким образом, дискретные данные часто возникают как результат группировки данных, где переходы моделируются в непрерывном времени, а затем делаются необходимые поправки для сгруппированных данных. Дальнейшее обсуждение представлено в Главе \ref{sec:17.10}.




\section{Цензурирование}\label{sec:17.4}

\noindent
Поскольку некоторые объекты невозможно наблюдать в течение всего времени их жизни, данные о выживаемости оказываются цензурированными. Для таких наблюдений известно лишь то, что время их жизни лежит в определенном интервале. Например, исследование о текущем состоянии безработицы может содержать информацию только о незавершенных событиях, то есть о людях, которые еще не нашли работу, поэтому мы не сможем понять, сколько времени требуется людям для поиска работы. Вместо этого, мы сможем узнать лишь, как долго тот или иной индивид уже является безработным.


\subsection{Механизмы цензурирования}\label{sec:17.4.1}

\noindent
На практики данные могут быть цензурированными справа, слева или в интервале. При \textbf{цензурировании справа}, или цензурировании сверху, мы можем наблюдать время жизни объекта с начального, или нулевого, момента вплоть до момента цензурирования $c$. К этому времени некоторые объекты уже прекратят существование, но некоторые нет, и о них известно лишь то, что момент смерти находится где-то в промежутке $(c,\infty)$. \textbf{Цензурирование слева}, или цензурирование снизу, предполагает, что объекты прекратили существование в промежутке $(0,c)$, но конкретный момент смерти неизвестен. Примером является классическая модель тобит, в которой информация о времени жизни некоторых объектов и времени их цензурирования неизвестна. \textbf{Интервальное цензурирование} возникает в том случае, когда время жизни объектов можно наблюдать только в определенном временном интервале, $[t^*_1,t^*_2)$.

В литературе по анализу выживаемости рассматривается преимущественно цензурирование справа. Однако даже при таком ограничении возможно применение различных видов цензурирования, таких как случайное цензурирование, цензурирование типа I и цензурирование типа II. %???

\textbf{Случайное}, или экзогенное, \textbf{цензурирование} означает, что у каждого объекта в выборке есть независимые друг от друга момент перехода $T^*_i$ и момент цензурирования $C^*_i$. Мы наблюдаем момент перехода $T^*_i$, если он произошел до цензурирования, и момент цензурирования $C^*_i$, если после. При этом известно, было ли время жизни объекта цензурировано или нет. Тогда наблюдения $(t_1,\delta_1), (t_2,\delta_2),  \ldots , (t_N,\delta_N)$ являются реализациями случайных величин
        \begin{align}
        \label{eq:17.10}
        T_i&=\min(T^*_i, C^*_i),\\
        \delta_i&=\mathbf{1}[T^*_i<C^*_i], \notag
        \end{align}
где функция-индикатор $\mathbf{1}[A]$ равна 1, если событие $A$ происходит, и 0, если нет. То есть, $\delta_i$ равна 1, если момент перехода известен, и 0, если нет.\footnote{Здесь и далее: индикатор цензурирования (\textit{censoring indicator}), хотя по смыслу переменная является индикатором \textit{отсутствия} цензурирования}
Случайное цензурирование может возникать в ряде случаев, например, при случайно неудавшейся попытке завершить наблюдение, случайном прекращении участия некоторых объектов в исследовании, или завершении самого исследования. % ???

\textbf{Цензурирование типа I} применяется, когда время жизни превышает определенную фиксированную величину $t_{c_i}$. Например, можно проводить испытание выборки электроламп на надежность в течение 5000 часов. По ходу испытания некоторые лампы погаснут, и, таким образом, мы узнаем их моменты отказа, но некоторые продолжат гореть и по окончании испытания. В таком случае, их срок службы окажется цензурирован справа. Это особый случай экзогенного цензурирования со значением $C^*_i=t_{c_i}$. Классическая модель тобит является примером цензурирования типа I снизу для непрерывной с.в. на интервале $(-\infty,\infty)$.


\subsection{Независимое (неинформативное) цензурирование}\label{sec:17.4.2}

\noindent
Для того чтобы стандартные методы анализа выживаемости были применимы к цензурированным данным, механизм цензурирования должен быть \textbf{независимым (неинформативным)}. Это означает, что параметры распределения $C^*$ не должны содержать какой-либо информации о параметрах распределения длительности $T^*$. Тогда можно принять индикатор цензурирования $\delta$ как экзогенный, и в таком случае сам механизм можно не моделировать.

В цензурированных данных $(t,\delta)$ нецензурированные наблюдения встречаются с вероятностью
    $$\Pr{[T=t,\delta=1]}=\Pr{[T=t|\delta=1]}\times\Pr{[\delta=1]}.$$
Если механизм цензурирования является независимым, то $\Pr{[T=t|\delta=1]}=\Pr{[T=t]}$. Если же цензурирование неинформативно, то можно также избавиться от множителя $\Pr{[\delta=1]}$, поскольку он не содержит информации о параметрах распределения $T$. Аналогично, цензурированные наблюдения встречаются с вероятностью
    $$\Pr{[T=t,\delta=0]}=\Pr{[T\ge t|\delta=0]}\times\Pr{[\delta=0]},$$
где $\Pr{[T\ge t|\delta=0]}=\Pr{[T\ge t]}$, если цензурирование независимо, и можно убрать множитель $\Pr{[\delta=0]}$, если неинформативно. Объединив вышеперечисленное, получим, что интересующая нас плотность равна $\Pr{[T=t]}$, когда $\delta=1$, и $\Pr{[T\ge t]}$, когда $\delta=0$.

Если добавить регрессоры $\mathbf{x}$, может случиться так, что $T^*$ и $C^*$ будут изменяться вместе с одними и теми же переменными. Но даже в таком случае, важно лишь, чтобы параметры $C^*$ были неинформативны по отношению к параметрам $T^*$. Проще говоря, цензурирование не должно происходить потому, что объект подвержен большему или меньшему риску перехода, при условии что $\mathbf{x}$ неизменны.

\textbf{Цензурирование типа II} происходит, если из выборки размера $N$ совершает переход требуемое количество объектов $p$. В таком случае известны лишь первые $p$ наиболее коротких длительностей, а остальные $N-p$ оказываются цензурированы в момент $C^*_i=t_{(p)}$. Например, клиническое испытание может быть завершено после того, как умер $p$-ый пациент.

Случайное цензурирование, цензурирование типа I и II являются примерами независимого цензурирования. Более формальное описание представлено в Кальбфляйш и Прентис (2002, с. 194-196).


\section{Непараметрические модели}\label{sec:17.5}

\noindent
Методы непараметрического оценивания хорошо подходят для описания поведения функций выживания. Как показано в примере с длительностями забастовок, полезно знать безусловное матожидание функции риска или выживания до включения регрессоров в анализ.

В данном разделе мы представим оценки функций выживания, риска и кумулятивного риска в условиях независимого цензурирования. Мы не будем рассматривать оценку плотности распределения из-за трудностей, возникающих при цензурировании; более того, функции риска и выживания проще интерпретировать, чем плотность.

Регрессоры отсутствуют. Для того чтобы выяснить, как ведет себя ключевая переменная в различных условиях эксперимента или при разных уровнях воздействия,
% ENG: treatment regimes or levels of treatment
можно по отдельности построить непараметрические оценки для каждого значения объясняющей переменной и затем сравнить их. Однако в экономике обычно требуются более сложные структурные модели с регрессорами, представленные в разделах \ref{sec:17.6}-\ref{sec:17.10}.

Мы будем работать с дискретными данными, такими как данные таблицы выживания, используя для этого формулы из раздела \ref{sec:17.3.2}.
% Опечатка в Cameron: 17.3.3
Рассмотрим, например, $N_0$ индивидов определенного пола и возраста, которые последовательно наблюдаются в течение нескольких лет. Через год в группе останется $N_1$ индивидов, поскольку $N_1-N_0$ индивидов либо умрут, либо данные о них окажутся утеряны (цензурированы). Через два года в группе останется $N_2-N_1$ индивидов и т.д. Таким образом, на основе таких данных мы сможем построить дискретную функцию выживания без каких-либо предпосылок о распределениях параметров.


\subsection{Непараметрическое оценивание}\label{sec:17.5.1}

\noindent
Очевидно, оценкой функции выживания при отсутствии цензурирования будет единица за вычетом функции распределения. Тогда $\hat{S}(t)$ будет равна количеству объектов длительностью больше $t$, деленной на размер выборки $N$. Это ступенчатая функция со скачками в моментах перехода (отказа), см. рисунок \ref{fig:17.1}.
% ENG: step function with jumps
Альтернативная оценка, которая будет представлена в уравнении \ref{eq:17.13}, также сохраняет состоятельность при отсутствии цензурирования.

Обозначим $t_1<t_2< \ldots <t_j< \ldots <t_k$ как \textbf{дискретные моменты отказа} объектов в выборке размера $N$, $N\ge k$. Пусть $d_j$ равняется количеству длительностей, завершившихся в момент $t_j$. Поскольку данные дискретны, $d_j$ может превышать единицу. Время жизни некоторых объектов невозможно наблюдать полностью. Пусть $m_j$ равняется количеству наблюдений, цензурированных справа в момент $[t_j,t_{j+1})$. Предположим также, что механизм цензурирования является независимым, то есть все, что мы знаем о наблюдении, цензурированном в $[t_j,t_{j+1})$, это только то, что момент отказа для него наступил позже $t_j$. Объект находится под риском отказа, если он до сих пор не был завершен или цензурирован. Пусть $r_j$ равняется количеству наблюдений, находящихся под риском в момент $t_{j-}$, то есть непосредственно перед $t_j$. Тогда $r_j=(d_j+m_j)+ \ldots +(d_k+m_k)=\sum_{l|l\ge j} (d_l+m_l)$. Заметим, что $r_1=N$. Таким образом,
        \begin{align}
        \label{eq:17.11}
        d_j&=\# \textrm{ объектов, завершенных в момент } t_j,\\
        m_j&=\# \textrm{ объектов, цензурированных в момент } [t_j,t_{j+1}), \notag \\
        r_j&=\# \textrm{ объектов под риском в момент } t_{j-}=\sum_{l|l\ge j} (d_l+m_l).\notag
        \end{align}

Очевидно, оценкой функции риска будет количество наблюдений, завершенных в момент $t_j$, деленное на количество наблюдений с риском отказа в момент $t_{j-}$, поскольку в дискретном случае $\lambda_j=\Pr{[T=t_j|T\ge t_j]}$
        \begin{align}
        \label{eq:17.12}
        \hat{\lambda_j}=\frac{d_j}{r_j}.
        \end{align}

Дискретная функция выживания определена в уравнении \ref{eq:17.8}. \textbf{Оценка Каплан-Мейера}, или множительная оценка функции выживания, является выборочным аналогом
        \begin{align}
        \label{eq:17.13}
        \hat{S}(t)=\prod_{j|t_j\le t} (1-\hat{\lambda}_j) = \prod_{j|t_j\le t} \frac{r_j-d_j}{r_j}.
        \end{align}

    \begin{table}[!htbp]\caption{\textit{Оценка коэффициента отказов и функции выживания: пример}${}^a$}\label{tab:17.2}
    \begin{tabularx}{\textwidth}{X X X X c c c}
    \hline \hline
$j$ &$r_j$  &$d_j$  &$m_j$  &$\hat{\lambda}_j=d_j/r_j$ &$\hat{\Lambda}(t_j)$   &$\hat{S}(t_j)$\\
    \hline
1   &80     &6      &4      &$6/80$                    &$6/80$                  &$1-6/80$\\
2   &70     &5      &3      &$5/70$                    &$6/80+5/70$             &$(1-5/70)\times(1-6/80)$\\
3   &62     &2      &1      &$2/62$                    &$\hat{\lambda}_2+2/62$  &$\hat{S}(t_2)\times(1-2/62)$\\
4   &-      &-      &-      &-                                  &               &\\
    \hline \hline
% FOOTNOTE --- CHANGE TO MINIPAGE?
\multicolumn{7}{l}{${}^a$ \scriptsize{В момент $t_j$, $r_j$ равно количеству наблюдений с риском, $d_j$ --- количеству смертей (отказов), $m_j$ --- количество}}\\[-0.15cm]
\multicolumn{7}{l}{\hspace{0.3cm}\scriptsize{пропущенных (цензурированных) наблюдений; $\hat{\lambda}_j$ является оценкой функции риска, $\hat{\Lambda}_j$ --- оценкой функции}}\\[-0.15cm]
\multicolumn{7}{l}{\hspace{0.3cm}\scriptsize{кумулятивного риска, $\hat{S}(t_j)$ --- оценкой функции выживания.}}
    \end{tabularx}
    \end{table}
Это убывающая ступенчатая функция со скачками в каждом моменте отказа. Оценка Каплан-Мейера может быть представлена как непараметрическая оценка ММП (см. Кальбфляйш и Прентис, 2002, стр. 14-16).

При отсутствии цензурирования $\hat{S}(t)$ упрощается до $\hat{S}(t)=r/N$, то есть количеству наблюдений с риском, деленному на размер выборки, что равняется единице за вычетом функции распределения. Чтобы показать это, заметим, что $r_j-d_j=r_{j+1}$, если $m_j=0$, то есть количество наблюдений с риском за вычетом количества смертей в момент $j$ равно количеству наблюдений с риском в момент $j+1$. Тогда можно преобразовать выражение \ref{eq:17.13} как $\hat{S}(t)=\prod_{j|t_j\le t} r_{j+1}/r_j$ и затем упростить его до $r/r_1$, где $r_1=N$.

Дискретная функция кумулятивного риска определена в уравнении \ref{eq:17.9}. \textbf{Оценка Нельсон-Аалена} функции кумулятивного риска, является выборочным аналогом
        \begin{align}
        \label{eq:17.14}
        \hat{\Lambda}(t)=\sum_{j|t_j\le t} \hat{\lambda}_j=\sum_{j|t_j\le t} \frac{d_j}{r_j}.
        \end{align}
С помощью данной оценки можно также получить оценку функции выживания $\tilde{S}(t_j)=\exp{(-\hat{\Lambda}(t))}$, используя равенство $S(t)=\exp{(-\Lambda(t))}$ из непрерывного случая.

В качестве примера, представим, что изначально было 80 наблюдений, из которых 6 завершились в момент $t_1$,  4 были цензурированы в момент $[t_1,t_2)$, 5 завершились в $t_2$, 3 были цензурированы в $[t_2,t_3)$, 2 завершились в $t_3$, 1 было цензурировано в $[t_3,t_4)$ и т.д. Тогда можно найти оценки функций кумулятивного риска и выживания для $t\le t_3$, которые представлены в таблице \ref{tab:17.2}.

\textbf{Набор идентичных данных}
% ENG: Tied Data
возникает, когда в определенный момент времени происходит сразу несколько отказов. Принято считать, что такие данные являются результатом группировки данных, а не самого процесса, порождающего их. Оценка риска $\hat{\lambda}_j=d_j/r_j$ предполагает, что все смерти наступают одновременно в момент $t_j$. На самом деле, смерти могут происходить постепенно в интервале $[t_j,t_{j+1})$, равно как и цензурирование. Тогда $r_j$ будет в среднем переоценивать количество объектов, находящихся под риском в интервале $[t_j,t_{j+1})$. В качестве стандартной поправки в анализе таблицы выживания $\hat{\lambda}_j=d_j/r_j$ заменяют на $d_j/(r_j-m_j/2)$. Аналогичные изменения делаются и для $\hat{S}(t)$, $\hat{\Lambda}(t)$ и т.д. Возможны также и другие поправки.

Основные таблицы и графики оценки Каплан-Мейера встроены в большинство статистических программ для анализа выживаемости. Пример подобных расчетов для данных о забастовках представлен в таблице \ref{tab:17.3}, которая дополняет, таким образом, рисунок \ref{fig:17.1}.

    \begin{table}[!htbp]\caption{\textit{Продолжительность забастовок: оценки Каплан-Мейера}}\label{tab:17.3}
    \begin{center}
\begin{tabular}{ccccc}
    \hline \hline
                &\textbf{Всего}        &\textbf{Количество} &\textbf{Функция}    &\textbf{Стандартная}\\
\textbf{День}   &\textbf{на начало}    &\textbf{отказов}    &\textbf{выживания}  &\textbf{ошибка}\\
    \hline
1               &566                        &10                 &0.9823                     &0.0055\\
2               &556                        &21                 &0.9452                     &0.0096\\
3               &535                        &16                 &0.9170                     &0.0116\\
4               &519                        &17                 &0.8869                     &0.0133\\
5               &502                        &18                 &0.8551                     &0.0148\\
6               &484                        &9                  &0.8392                     &0.0154\\
7               &475                        &12                 &0.8180                     &0.0162\\
8               &463                        &12                 &0.7968                     &0.0169\\
\vdots          &\vdots                     &\vdots             &\vdots                     &\vdots\\
13              &411                        &11                 &0.7067                     &0.0191\\
14              &400                        &11                 &0.6873                     &0.0195\\
    \hline \hline
\end{tabular}
    \end{center}
    \end{table}


\subsection{Доверительные интервалы для непараметрических оценок}\label{sec:17.5.2}

\noindent
Оценка функции риска $\hat{\lambda}_j=\frac{d_j}{r_j}$ не является непрерывной, при этом разрывы становятся тем больше, чем меньше $r_j$ по отношению к $d_j/r_j$. Поэтому перед построением графиков зависимости от времени иногда полезно проводить сглаживание оценок риска с помощью непараметрических методов, рассмотренных в разделе 9.5% \ref{sec:9.5}
.

Функции выживания и кумулятивного риска являются более гладкими, и зависимость от времени принято изображать на графике вместе с доверительными интервалами, которые отражают изменчивость выборки. Существует несколько способов оценки доверительных интервалов. Мы будем использовать формулы, которые применяются при расчетах в STATA.

Для оценки Каплан-Мейера функции выживания обычно используют оценку дисперсии Гринвуда
    $$\widehat{\Var}[\hat{S}(t)]=\hat{S}(t)^2\sum_{j|t_j\le t}\frac{d_j}{r_j(r_j-d_j)}.$$
Полученные доверительные интервалы зачастую рассчитываются на основе $\ln(-\ln\hat{S}(t))$, а не $\hat{S}(t)$, поскольку такое преобразование гарантирует, что они находятся в пределах допустимых значений
% ALT: области определения?
функции выживания, то есть между нулем и единицей. Получаем $100(1-\alpha)\%$ доверительный интервал
        \begin{align}
        \label{eq:17.15}
        S^{\mathrm{d}}(t)\in (\hat{S}(t)\exp{}^{(-z_{\alpha/2} \hat{\sigma}(t))},\hat{S}(t)\exp{}^{(z_{\alpha/2} \hat{\sigma}(t))}),
        \end{align}
где $\sigma(t)$ обозначает стандартное отклонение $\ln(-\ln\hat{S}(t))$, которое рассчитывается по формуле
    $$\hat{\sigma}^2_s(t)=\frac{\sum_{j|t_j\le t} d_j/(r_j(r_j-d_j))}{\left[ \sum_{j|t_j\le t} \ln ((r_j-d_j)/d_j)\right] ^2}.$$

    \begin{table}[!htbp]\caption{\textit{Распределения Вейбулла и экспоненциальное: плотность, функция распределения, функция выживания, риска, кумулятивного риска, среднее и дисперсия}}\label{tab:17.4}
    \begin{center}
\begin{tabular}{lll}
\hline \hline
\textbf{Функция}&\textbf{Экспоненциальное}&\textbf{Вейбулла}\\
\hline
$f(t)$          &$\gamma\exp{(-\gamma t)}$  &$\gamma \alpha t^{\alpha-1}\exp{(-\gamma t^{\alpha})}$\\
$F(t)$          &$1-\exp{(-\gamma t)}$      &$1-\exp{(-\gamma t^{\alpha})}$\\
$S(t)$          &$\exp{(-\gamma t)}$        &$\exp{(-\gamma t^{\alpha})}$\\
$\lambda(t)$    &$\gamma$                   &$\gamma\alpha t^{\alpha-1}$\\
$\Lambda(t)$    &$\gamma t$                 &$\gamma t^{\alpha}$\\
$\E(T)$ &$\gamma^{-1}$              &$\gamma^{-1/\alpha}\Gamma(\alpha^{-1}+1)$\\
$\Var(T)$ &$\gamma^{-2}$              &$\gamma^{-2/\alpha}[\Gamma(2\alpha^{-1}+1)-[\Gamma(\alpha^{-1}+1)]^2]$\\
$\gamma,\alpha$ &$\gamma>0$                 &$\gamma>0,\alpha>0$\\
\hline \hline
\end{tabular}
    \end{center}
    \end{table}

Для оценки кумулятивного риска Нельсон-Аалена одной из оценок дисперсии будет
    $$\widehat{\Var}[\hat{\Lambda}(t)]=\sum_{j|t_j\le t} \frac{d_j}{r^2_j}.$$
С помощью преобразования $\ln\hat{\Lambda}(t)$ получаем $100(1-\alpha)\%$ доверительный интервал для кумулятивного риска
        \begin{align}
        \label{eq:17.16}
        \Lambda(t)\in [\hat{\Lambda}(t)\exp{(-z_{\alpha/2}\hat{\sigma}_{\Lambda}(t))}, \hat{\Lambda}(t)\exp{(z_{\alpha/2}\hat{\sigma}_{\Lambda}(t))}],
        \end{align}
где $\hat{\sigma}_{\Lambda}(t)$ обозначает стандартное отклонение для $\ln\hat{\Lambda}(t)$, которое рассчитывается по формуле
$$\hat{\sigma}^2_{\Lambda}(t)=\widehat{\Var}[\hat{\Lambda}(t)]/[\hat{\Lambda}(t)^2].$$




\section{Параметрические модели регрессии}\label{sec:17.6}

\noindent
В данном разделе мы опишем свойства двух распределений, составляющих основу параметрических моделей. Также мы рассмотрим некоторые стандартные модели регрессии для анализа времени жизни.


\subsection{Экспоненциальное распределение и распределение Вейбулла}\label{sec:17.6.1}

\noindent
Исходной точкой для параметрического анализа является экспоненциальное распределение, поскольку чистый точечный процесс Пуассона генерирует длительности, распределенные экспоненциально, см. Ланкастер (1990 стр. 86). В силу свойства отсутствия памяти \textbf{экспоненциальное распределение длительностей} предполагает постоянный во времени коэффициент риска $\gamma$. Из \ref{eq:17.5} следует, что $S(t)=\exp{(-\int^t_0\gamma d u)}=\exp{(-\gamma t)}$. Плотность равна $f(t)=-S'(t)=\gamma\exp{(-\gamma t)}$, и кумулятивный риск $\Lambda(t)=-\ln S(t)=\gamma t$ линеен по времени.

Применение экспоненциального распределения ограничено наличием единственного параметра, поэтому в эконометрике часто используют обобщение, называемое распределением Вейбулла. В таблице \ref{tab:17.4} представлены моменты, плотность и другие функции для экспоненциального распределения и распределения Вейбулла, где первое является частным случаем второго при $\alpha=1$. Функция $\Gamma(\cdot)$ в таблице \ref{tab:17.5} является гамма функцией.

    \begin{table}[!htbp]\caption{\textit{Стандартные параметрические модели и их функции риска и выживания}${}^a$}\label{tab:17.5}
    \begin{center}
\begin{tabular}{llll}
\hline \hline
\textbf{Параметрическая модель} &\textbf{Функция риска}     &\textbf{Функция выживания}                 &\textbf{Тип}\\
\hline
Экспоненциальное    &$\gamma$                               &$\exp{(-\gamma t)}$                        &PH, AFT\\
Вейбулла            &$\gamma\alpha t^{\alpha-1}$            &$\exp{(-\gamma t^{\alpha})}$               &PH, AFT\\
Вейбулла обобщенное &$\gamma\alpha t^{\alpha-1}S(t)^{-\mu}$ &$[1-\mu\gamma t^{\alpha}]^{1/\mu}$         &PH\\
Гомперца            &$\gamma\exp{(\alpha t)}$               &$\exp{(-(\gamma/\alpha)(e^{\alpha t}-1))}$ &PH\\
Лог-нормальное      &$\frac{\exp{(-(\ln t-\mu)^2/2\sigma^2)}}{t\sigma\sqrt{2\pi[1-\Phi((\ln t -\mu)/\sigma)]}}$ &$1-\Phi((\ln t-\mu)/\sigma)$    &AFT\\
Лог-логистическое   &$\alpha\gamma^{\alpha}t^{\alpha-1}/[(1+(\gamma t)^{\alpha})]$                              &$1/[1+(\gamma t)^{\alpha}]$    &AFT\\
Гамма               &$\frac{\gamma(\gamma t)^{\alpha-1}\exp{[-(\gamma t)]}}{\Gamma(\alpha)[1-I(\alpha,\gamma t)]}$&$1-I(\alpha,\gamma t)$   &AFT\\
\hline \hline
\multicolumn{4}{c}{${}^a$ \scriptsize{Все параметры положительны, за исключением $-\infty<\alpha<\infty$ для модели Гомперца.}}
\end{tabular}
    \end{center}
    \end{table}

В модели Вейбулла функция риска $\lambda(t)=\gamma\alpha t^{\alpha-1}$ монотонно возрастает при $\alpha>1$ и убывает при $\alpha<1$. Это является частным случаем класса моделей пропорциональных рисков (PH),
% ALT: семейства
см. раздел \ref{sec:17.7.1}, где $\lambda(t)$ может быть разложена на две компоненты: базовую, зависящую только от $\lambda_0(t)$ и $t$, и относительную (например, $\gamma$), которая зависит только от набора независимых переменных.

На рисунке \ref{fig:17.2} представлены свойства распределения Вейбулла при $\gamma=0.01$ и $\alpha=1.5$. Функция плотности скошена вправо, что вполне характерно для данных по длительностям. Форма функции выживания одинакова для большого количества распределений, что затрудняет сравнительный визуальный анализ различных оценок функции выживания. Функция риска является возрастающей, поскольку для данного примера $\alpha>1$. При этом для разных параметрических моделей могут быть разные формы функции риска, включая монотонно убывающие и возрастающие, прямые и обратные U-образные формы.

\begin{figure}[ht!]\caption{Распределение Вейбулла: плотность, функции выживания, риска и кумулятивного риска при $\gamma=0.01$ и $\alpha=0.5$}\label{fig:17.2}
\centering
%\includegraphics[scale=0.7]{fig.png}
\end{figure}

На практике оценка функции риска часто оказывается неточной, в особенности в правом хвосте. Оценка кумулятивного риска $\Lambda(t)$ является более точной и, таким образом, позволяет определять некоторые различия между моделями. Еще точнее можно оценить зависимость $\ln\Lambda(t)$ от времени, поскольку в модели Вейбулла $\ln\Lambda(t)=\ln\gamma+\alpha\ln t$ линеен по $\ln t$ с наклоном $\alpha$.


\subsection{Некоторые параметрические модели}\label{sec:17.6.2}

\noindent
Наиболее популярные параметрические модели основаны на распределениях Вейбулла, Гомперца, экспоненциальном, лог-нормальном, лог-логистическом и гамма. Их функции риска и выживания представлены в таблице \ref{tab:17.5}.

Для гамма распределения $\Gamma(\alpha)=\int^{\infty}_{0}e^{-t}t^{\alpha-1}dt$ называется \textbf{гамма функцией}, а $I(\alpha,\gamma t)$ --- \textbf{неполной гамма функцией}, где $I(\alpha,x)=\int^{x}_{0}e^{-t}t^{\alpha-1}dt/\Gamma(\alpha)$, $0<I(\alpha,x)<1$.

Обобщенная модель Вейбулла была предложена в работе Мудхолкара, Сриваставы и Коллиа (1996). За счет добавления параметра $\mu$ функция риска становится более гибкой, что делает модель менее ограничительной. При этом модель Вейбулла является предельным случаем обобщенного варианта при $\mu\rightarrow 0$. Из таблицы \ref{tab:17.5} следует, что
    $$\ln\lambda(t)=\ln(\gamma\alpha)+(\alpha-1)\ln t-\mu\ln S(t).$$
Поскольку $\partial S(t)/\partial t<0$, правая часть уравнения возрастает по $t$ при $\mu>0$ и $\alpha>1$. Если $\alpha\le1$ и $\mu<0$, то функция риска монотонно убывает. Если $\alpha>1$ и $\mu<0$, то риск состоит из двух частей, одна из которых является возрастающей функцией, а другая --- убывающей, что в сумме позволяет кривой риска иметь одномодальную, или U-образную, форму.

Модель Гомперца похожа на вейбулловскую тем, что риск может как возрастать (при $\alpha>0$), так и убывать (при $\alpha<0$), или же быть постоянным, как в экспоненциальной модели (при $\alpha=0$). Распределение Гомперца хорошо подходит для анализа данных смертности и поэтому используется больше в биостатистике, чем в эконометрике.

Для лог-нормального распределения риск имеет обратную U-образную форму и сперва возрастает, а затем убывает по $t$. Аналогично, для лог-логистического распределения при $\alpha>1$. В силу данного свойства эти модели больше подходят для анализа данных по длительностям, чем экспоненциальная модель или модели Вейбулла и Гомперца.

Другие параметрические модели включают распределения Рэлея и Мейкхэма, обратную гауссовскую кусочно-непрерывную модель рисков и обобщенную гамма модель (Лоулесс, 1982), частными случаями которой являются гамма и вейбулловская модели. Детали представлены в Кальбфляйш и Прентис (1980, глава 3) и Ланкастер (1990, глава 3).

В основном, распределения содержат два параметра. % Описаны/описываются двумя параметрами
\textbf{Регрессоры} включены как $\gamma=\exp{(\mathbf{x}'\beta)}$ с константой $\alpha$, или $\mu=\mathbf{x}'\beta$ с константой $\sigma^2$ для лог-нормального распределения.

Основные трудности в параметрическом моделировании связаны с тем, что состоятельность оценок зависит от правильной спецификации модели, выбор которой достаточно широк. Большинство моделей относятся либо к PH моделям (первые четыре в таблице \ref{tab:17.5}), либо к моделям ускоренной жизни (первые две и последние три в таблице \ref{tab:17.5}). Модель Вейбулла принадлежит обоим классам и поэтому часто применяется в экономических приложениях. Другой широко используемой моделью, в частности, для анализа большого количества наблюдений, является кусочно-постоянная модель рисков, которая представляет собой частный случай модели PH.


\subsection{Оценивание ММП}\label{sec:17.6.3}

\noindent
Мы будем изучать полностью параметрический анализ методами максимального правдоподобия и наименьших квадратов при независимом или неинформативном цензурировании, используя при этом формулы для непрерывного случая, поскольку параметрические модели основаны на непрерывных распределениях. Предположим, что регрессоры постоянны во времени, оставив случай с меняющимися регрессорами для рассмотрения в разделе \ref{sec:17.9}.

Пусть $T^*$ обозначает нецензурированные длительности, распределенные условно с плотностью $f(t|\mathbf{x},\theta)$, где $\theta$ является вектором размерности $q\times1$, а $\mathbf{x}$ --- набор регрессоров, которые могут меняться между объектами, но постоянны между состояниями для каждого объекта. При наличии цензурированных наблюдений анализ становится несколько сложнее. Так, длительности незавершенного события $t$ теперь соответствует новая переменная-индикатор цензурирования, которое неинформативно по предположению.

Из раздела \ref{sec:17.4.2} следует, что воздействие аналогично воздействию в модели тобит. Вклад нецензурированных наблюдений в функцию правдоподобия будет составлять $f(t|\mathbf{x},\mathbf{\theta})$. О наблюдениях, цензурированных справа, мы знаем только то, что их время жизни превышает $t$, а значит их вклад будет равен
        \begin{align}
        \Pr[T>t]&=\int^{\infty}_{t}f(u|\mathbf{x},\mathbf{\theta})du\notag\\
                &=1-F(t|\mathbf{x},\theta)=S(t|\mathbf{x},\mathbf{\theta}),\notag
        \end{align}
где $S(\cdot)$ --- функция выживания. Функция плотности для $i$-го наблюдения может быть записана как
    $$f(t_i|\mathbf{x}_i,\mathbf{\theta})^{\delta_i}S(t_i|\mathbf{x}_i,\mathbf{\theta})^{1-\delta_i},$$
где $\delta_i$ --- индикатор цензурирования справа, равный
        $$\delta_i=\begin{cases}
            1 \textrm{ (нет цензурирования)},\\
            0 \textrm{ (цензурирование справа)}.
                   \end{cases}$$
Взяв логарифмы и просуммировав, получим, что  оценка ММП $\hat{\theta}$ максимизирует логарифм правдоподобия
        \begin{align}
        \label{eq:17.17}
        \ln \mathrm{L}(\mathbf{\theta})=\sum^{N}_{i=1}[\delta_i\ln f(t_i|\mathbf{x}_i,\mathbf{\theta})+(1-\delta_i)\ln S(t_i|\mathbf{x}_i,\mathbf{\theta}a)],
        \end{align}
где предполагается, что объекты $i$ независимы. Первое слагаемое относится к завершенным событиям, а второе --- к цензурированным справа. Поскольку $\ln S(t) = \Lambda(t)$, а $\ln f(t) = \ln(\lambda(t)S(t)) = \ln \lambda(t) + \ln S(t)$, логарифм функции правдоподобия можно записать в терминах простой и интегральной функций риска:
        \begin{align}
        \label{eq:17.18}
        \ln \mathrm{L}(\mathbf{\theta})=\sum^{N}_{i=1}[\delta_i\ln\lambda(t_i|\mathbf{x}_i,\mathbf{\theta})+\Lambda(t_i|\mathbf{x}_i,
        \mathbf{\theta})].
        \end{align}
Результат является полезным в случае, если параметрическая модель изначально задана через функцию риска, а не через функцию плотности.

Далее применяется стандартная теория. Оценка ММП распределена как
$$\bm{\hat{\theta}}\thicksim\!\!\!\!\!^{{}^{a}}\hspace{0.1cm}\mathcal{N}\left[\bm{\theta},(-\E[\partial^2\ln\mathrm{L}/\partial\bm{\theta}\partial\bm{\theta}'])^{-1} \right],$$
% в книжке не по центру
если функция плотности специфицирована верно, см. раздел 5.7.3. % \ref{sec:5.7.3}
Если спецификация неверна, то оценка ММП не является состоятельной, независимо от наличия цензурированных наблюдений. Исключение составляет экспоненциальная модель при отсутствии цензурирования, для состоятельности оценок которой требуется лишь правильная спецификация условного матожидания, см. раздел 5.7.3. % \ref{sec:5.7.3}
Однако при наличии цензурирования неверная спецификация приводит к несостоятельности оценок даже в этой модели. Неустойчивость является главным недостатком параметрического подхода, как и в модели тобит.

Метод максимального правдоподобия позволяет анализировать и другие типы цензурированных данных. При \textbf{цензурировании слева} максимальная длительность состояния составляет $t$, и вклад таких наблюдений равен $\Pr[T^*<t]=\int^{t}_{0}f(s|\mathbf{x},\mathbf{\theta})ds=F(t|\mathbf{x},\mathbf{\theta}).$

При \textbf{интервальном цензурировании} длительность наблюдений лежит в интервале $[t_a,t_b)$, и их вклад в функцию правдоподобия составляет $\Pr[t_a\le T^*<t_b]=\int^{t_b}_{t_a}f(s|\mathbf{x},\mathbf{\theta})ds=S(t_a|\mathbf{x},\mathbf{\theta})-S(t_b|\mathbf{x},\mathbf{\theta}).$

В экономике данные по длительностям зачастую цензурированы интервалом. Например, данные по безработицы могут быть сгруппированы по неделям или месяцам при непрерывном распределении, например, распределении Вейбулла. Обычно предполагается, что эффект интервального цензурирования достаточно незначителен, что им можно пренебречь. Например, для индивида, являющегося безработным в течение двух месяцев, но нашедшего работу на третьем месяце, длительность состояния будет составлять ровно три месяца вместо интервала от двух до трех.


\subsection{Компоненты ММП}\label{sec:17.6.4}
% ALT: элементы

\noindent
Учитывая многообразие типов данных, в частности, то, что наблюдения могут быть завершенными, урезанными или цензурированными, для ММП параметрически специфицированной модели требуется правильная запись функции правдоподобия. (Ланкастер (1979) показывает различные варианты выражений правдоподобия, соответствующие трем типам данных по длительностям безработицы). Каждый тип наблюдений несет определенный вклад в функцию правдоподобия так, что итоговая функция представляет собой произведение соответствующих элементов, таких как (см. Кляйна и Мошбергера, 1997, стр. 66):
        \begin{align}
        &\textrm{завершенные длительности: }                                &&f(t),\notag\\
        &\textrm{урезанные слева в момент } t_L \textrm{ } (t\ge t_L):      &&f(t)/S(t_L),\notag\\
        &\textrm{цензурированные слева в момент } t_{C_L}:                  &&1-S(t_{C_L}),\notag\\
        &\textrm{цензурированные справа в момент } t_{C_R}:                 &&S(t_{C_R}),\notag\\
        &\textrm{урезанные справа в момент } t_{C_R} \textrm{ } (t\le t_R): &&f(t_R)/[1-S(t_R)],\notag\\
        &\textrm{цензурированные в интервале в моментах } t_{C_L}, t_{C_R}: &&S(t_{C_L})-S(t_{C_R}).\notag
        \end{align}


\subsection{Пример ММП Вейбулла}\label{sec:17.6.5}

\noindent
Распределение Вейбулла детально представлено в \ref{sec:17.6.1}. Функция риска равна $\lambda(t)=\gamma\alpha t^{\alpha-1}$, где $\alpha>0$ и $\gamma>0$.

Существуют различные способы добавления регрессоров в модель. Обычно предполагают, что $\gamma=\exp{(\mathbf{x}'\be)}$, что гарантирует, что $\gamma>0$, при этом $\alpha$ не зависит от переменных. (Некоторые программы предполагают, что $\gamma=\exp{(-\mathbf{x}'\be)}$, что приводит к противоположным знакам оценок $\be$). Тогда
        \begin{align}
        \ln f(t|\mathbf{x},\be,\alpha)&=\ln[\exp{(\mathbf{x}'\be)}\alpha t^{\alpha-1}\exp{(-\exp{(\mathbf{x}'\be)}t^{\alpha})}]\notag\\
                                        &=\mathbf{x}'\be+\ln\alpha+(\alpha-1)\ln t-\exp{(\mathbf{x}'\be)t^{\alpha}}\notag
        \end{align}
и
        \begin{align}
        \ln S(t|\mathbf{x},\be,\alpha)&=\ln[\exp{(-\exp{(\mathbf{x}'\be)}t^{\alpha})}]\notag\\
                                        &=-\exp{(\mathbf{x}'\be)t^{\alpha}}.\notag
        \end{align}
Функция правдоподобия (\ref{eq:17.17}) принимает вид
        \begin{align}
        \label{eq:17.19}
        \ln \mathrm{L}=\sum_i[\delta_i\{\mathbf{x}_i'\beta+\ln\alpha+(\alpha-1)\ln t_i-\exp{(\mathbf{x}_i'\be)t_i^{\alpha}}\}-(1-\delta_i)\exp{(\mathbf{x}_i'\be)t_i^{\alpha}}].
        \end{align}
Условия первого порядка для $\be$ и $\alpha$ равны
        \begin{align}
        \frac{\partial\ln\mathrm{L}}{\partial\be}&=\sum_i(\delta_i-\exp{(\mathbf{x}_i'\be)t_i^{\alpha}})\mathbf{x}_i=\mathbf{0},\notag\\
        \frac{\partial\ln\mathrm{L}}{\partial\alpha}&=\sum_i\delta_i(1/\alpha+\ln t_i)-\ln t_i\exp{(\mathbf{x}_i'\be)t_i^{\alpha}}=0.\notag
        \end{align}
При этом для состоятельности требуются достаточно строгие предположения. Например, даже при отсутствии цензурирования для $\E[\partial\ln\mathrm{L}/\partial\be]=0$ необходимо, чтобы $\E[T^{\alpha}|\mathbf{x}]=\exp{(\mathbf{x}'\be)}.$


\subsection{Интерпретация оценок модели}\label{sec:17.6.6}

\noindent
Стандартный способ интерпретации коэффициентов в нелинейных моделях регрессии заключается в оценке их влияния на условное матожидание. Если $\gamma=\exp{(\mathbf{x}'\be)}$, то из таблицы \ref{tab:17.4} следует, что матожидание завершенных наблюдений в модели Вейбулла равно $\E[T^*|\mathbf{x}]=\exp{(-\mathbf{x}'\be/\alpha)}\Gamma(\alpha^{-1}+1)=\exp{(-\mathbf{x}'\be/\alpha)}\Gamma(\alpha^{-1})/\alpha.$ Значит, можно найти ожидаемую длительность при разных значениях $\mathbf{x}$. Например, можно предсказать продолжительность (завершенного) состояния безработицы индивида определенного возраста, пола и уровня образования.

Однако параметрические модели регрессии позволяют предсказывать и другие характеристики, кроме матожидания. Например, можно оценить, какую долю в общей продолжительности безработицы занимают длительности тех объектов, которые находились в безработном состоянии дольше определенного времени, или же узнать долю длительностей индивидов определенного социально-экономического статуса. % ALT: суммарной? Язык свернешь!!
При этом модели времени жизни построены не только на анализе независимых переменных, но и на предположении о форме функции риска, поскольку экономическая теория может заранее предсказывать ее поведение.

Несмотря на множество различных вариантов интерпретации, обычно останавливаются % рассматривают?
на вейбулловском коэффициенте риска, $\lambda(t)=\gamma\alpha t^{\alpha-1}$, его изменениях во времени и с регрессорами. Как уже упоминалось в разделе \ref{sec:17.3.2}, коэффициент возрастает при $\alpha>1$ и убывает при $\alpha<1$, поэтому основной интерес представляет тестирование гипотезы, что $\alpha=1$. Изменение регрессоров имеет мультипликативный эффект на функцию риска так, что
        $$d\lambda(t)/d\mathbf{x}=\exp{(\mathbf{x}'\be)}\alpha t^{\alpha-1}\be=\lambda(t)\be.$$
Таким образом, при $\beta_j>0$ рост $x_j$ приводит к повышению риска и снижению ожидаемой длительности. % Кэмерон два раза одно и то же пишет.. (положительный коэффициент $\beta_j$ предполагает рост коэффициента риска с ростом значений регрессоров $\mathbf{x}$.)


\subsection{Оценивание с помощью МНК}\label{sec:17.6.7}

\noindent
Помимо ММП оценивание полностью параметрических моделей можно проводить с помощью МНК, аналогично оцениванию цензурированной модели тобит. Заметим, что метод редко применяется на практике, так как по-прежнему требует корректной спецификации функции плотности, но при этом оценки коэффициентов менее эффективны, чем в ММП.

Рассмотрим экспоненциальную модель времени жизни. Поскольку $\E[T|\mathbf{x}]=1/\gamma=\exp{(-\mathbf{x}'\be)}$, то регрессия НМНК $t_i$ на $\exp{(-\mathbf{x}_i'\be)}$ дает состоятельную оценку $\be$. Экспоненциальная модель времени жизни может быть записана иначе, в виде линейной регрессии $\ln t=\mathbf{x}'\be+u$, где $u$ имеет распределение экстремальных значений (см. раздел \ref{sec:17.7.2}). Тогда $\E[\ln T|\mathbf{x}]=\mathbf{x}'\be-c$, где $c \simeq 0.5722$ --- константа Эйлера. Таким образом, $\be$ является состоятельной оценкой коэффициентов линейной регрессии $\ln t_i$ на $\mathbf{x_i}$. При цензурировании справа требуется найти сами моменты цензурирования, что также возможно для экспоненциальной модели.

Более общие результаты представлены в работе Кифера (1988, стр. 665). Кифер рассматривает модель PH (\ref{eq:17.21}) с $\phi(\mathbf{x}'\be)=\exp{(\mathbf{x}'\be)}$, то есть,
        $$\lambda(t|\mathbf{x})=\lambda_0(t,\alpha)\exp{(\mathbf{x}'\be)}.$$
Интегральную функцию базового риска можно получить следующим образом:
        \begin{align}
        \label{eq:17.20}
        \int^{t}_{0}\lambda(s|\mathbf{x})ds&=\int^{t}_{0}\lambda_0(s,\alpha)\exp(\mathbf{x}'\be)ds,\\
        \Lambda(t|\mathbf{x})&=\Lambda_0(t,\alpha)\exp(\mathbf{x}'\be),\notag\\
        \ln\Lambda(t|\mathbf{x})&=\ln\Lambda_0(t,\alpha)+\mathbf{x}'\be,\notag\\
        -\ln\Lambda_0(t,\alpha)&=\mathbf{x}'\be-\ln\Lambda(t|\mathbf{x})\notag\\
        &=\mathbf{x}'\be+u,\notag
        \end{align}

где ошибка $u=-\ln\Lambda(t|\mathbf{x})$ имеет распределение экстремальных значений.


Результат не зависит от конкретного выбора базового риска, поскольку для определенного $\lambda_0(t,\alpha)$ зависимую переменную $t$ можно преобразовать как $-\ln\Lambda_0(t,\alpha)$, так, чтобы получить линейную модель регрессии с ошибками, распределенными по закону экстремальных значений. % 1 тип?
Для экспоненциальной модели $\ln\Lambda_0(t,\alpha)=\ln t$, для модели Вейбулла $\ln\Lambda_0(t,\alpha)=\alpha\ln t$. Для цензурированных данных типа I нужно найти $\E[\ln\Lambda_0(T,\alpha)|T>t^*]$, используя распределение экстремальных значений, а затем применить двухшаговую процедуру Хекмана. Эти результаты могут быть использованы в качестве основы для простых диагностических тестов, которые будут рассмотрены в следующей главе.





\section{Некоторые важные модели времени жизни}\label{sec:17.7}

\noindent
Пожалуй, наиболее часто применяемой моделью в анализе времени жизни является модель пропорциональных рисков. Однако знакомство с некоторыми ее вариантами, % ALT: модификациями?
а также с моделями ускоренной жизни (AFT), представленными в разделе \ref{sec:17.7.2}, будет полезно.


\subsection{Модель пропорциональных рисков}\label{sec:17.7.1}

\noindent
Как уже упоминалось, в \textbf{модели пропорциональных рисков} (PH) условный коэффициент риска $\lambda(t|\mathbf{x})$ может быть представлен в виде двух независимых функций:
        \begin{align}
        \label{eq:17.21}
        \lambda(t|\mathbf{x})=\lambda_0(t,\mathbf{\bm{\alpha}})\phi(\mathbf{x},\bm{\beta}),
        \end{align}
где $\lambda_0(t,\bm{\alpha})$ называется \textbf{базовым риском} и является функцией только от $t$, а $\phi(\mathbf{x},\bm{\beta})$ является функцией только от регрессоров $\mathbf{x}$. Обычно $\phi(\mathbf{x},\bm{\beta})=\exp(\mathbf{x}'\bm{\beta})$. В литературе часто встречаются \textbf{полиномиальные базовые риски}.

Все функции риска $\lambda(t|\mathbf{x})$ типа \ref{eq:17.21} пропорциональны по отношению к базовому риску с коэффициентом $\phi(\mathbf{x},\bm{\beta})$, который не является функцией от $t$. Популярность моделей PH объясняется тем, что для состоятельности оценок $\be$ не требуется спецификации функциональной формы для $\lambda_0(\cdot)$ (см. раздел \ref{sec:17.8}).

Экспоненциальная модель, а также модели регрессии Вейбулла и Гомперца являются моделями PH, поскольку их риски равны $\exp(\mathbf{x}'\bm{\beta})$, $\exp(\mathbf{x}'\bm{\beta})\alpha t^{\alpha-1}$, $\exp(\mathbf{x}'\bm{\beta})\exp(\alpha t)$, соответственно.

Еще одним примером модели PH, используемой преимущественно в анализе продолжительности безработицы, является \textbf{модель кусочно-постоянных рисков}, которая разбивает $\lambda_0(t,\bm{\alpha})$ на $k$ сегментов так, что
        \begin{align}
        \label{eq:17.22}
        \lambda_0(t,\bm{\alpha})=e^{\bm{\alpha}_j}, \hspace{0.2cm} c_{j-1}\le t\le c_j, \hspace{0.2cm} j=1, \ldots ,k,
        \end{align}
где $c_0=0$, $c_k=\infty$ и другие пороговые значения $c_1, \ldots ,c_{k-1}$ определены, а параметры $\alpha_1, \ldots ,\alpha_k$ подлежат оценке. Параметры возведены в степень, что гарантирует, что $\lambda_0(t,\bm{\alpha})>0.$ Эта модель оценивает больше базовых параметров, чем, например, модель Вейбулла, которая содержит только один базовый риск. Тем не менее, она может быть практична в применении для анализа достаточно больших объемов данных.

Индентифицируемость модели PH при ненаблюдаемой гетерогенности
% неоднородности?
будет рассмотрена в разделе 18.3. % \ref{sec:18.3} # UNCOMMENT AFTER 18 CHAPTER


\subsection{Модель ускоренной жизни}\label{sec:17.7.2}

\noindent
Модель AFT специфицирована в виде
        \begin{align}
        \label{eq:17.23}
        \ln t=\mathbf{x}'\bm{\beta}+u,
        \end{align}
где различные предположения о распределении $u$ приводят к различным моделям AFT. Поскольку $\ln t$ принимает значения в интервале $(-\infty,\infty)$, то $u$ может относиться к любому распределению, непрерывному на $(-\infty,\infty)$.

Название \textbf{ускоренной жизни} объясняется тем, что $t=\exp(\mathbf{x}'\bm{\beta})\nu$, где $\nu=e^u$, имеет коэффициент риска $\lambda(t|\mathbf{x})=\lambda_0(\nu)\exp(\mathbf{x}'\bm{\beta})$, где базовый риск $\lambda_0(\nu)$ не зависит от времени. Произведя замену $\nu=t\exp(-\mathbf{x}'\bm{\beta})$, получим функцию риска
        \begin{align}
            \label{eq:17.24}
            \lambda(t|\mathbf{x})=\lambda_0(t\exp(-\mathbf{x}'\bm{\beta}))\exp(\mathbf{x}'\bm{\beta}).
        \end{align}
При $\exp(-\mathbf{x}'\bm{\beta})>1$ происходит ускорение базового риска $\lambda_0(t)$, а при $\exp(-\mathbf{x}'\bm{\beta})<1$ --- замедление.

Если $u\thicksim\mathcal{N}[0,\sigma^2]$, то модель является лог-нормальной; лог-логистическая модель получается, если $u$ распределена логистически. Гамма модель также может быть представлена как AFT, если предположить, что функция плотности $u$ равна $f(u)=\exp(\alpha u-e^u)/\Gamma(\alpha)$.

Единственными моделями, которые одновременно относятся к классам PH и AFT, являются экспоненциальная и вейбулловская. Модель AFT, в частности, получается, если $u$ равно $\alpha\omega$, где $\omega$ имеет распределение экстремальных значений с функцией плотности $f(\omega)=e^{\omega}\exp(-e^\omega)$.

Модели времени жизни $g(t)=\mathbf{x}'\bm{\beta}+u$ не обязательно должны иметь логлинейную функциональную форму, $g(t)=\ln t$. За счет изменения функциональной формы можно получить целый класс моделей с преобразованием, к которому принадлежит, например, модель регрессии Бокса-Кокса.


\subsection{Гибкие модели рисков}\label{sec:17.7.3} % Адаптивный / приспосабливающийся ?

\noindent
Некоторые модели изначально задаются через функцию риска, а не через функцию плотности. Например, риск может быть квадратичен по отношению к $t$ так, что $\lambda(t)=\mathbf{x}'\bm{\beta}+a_1t+a_2t^2$, что приводит к U-образной форме функции риска. Соответствующий кумулятивный риск равен $\Lambda(t)=(\mathbf{x}'\bm{\beta})t+(a_1/2)t^2+(a_2/3)t^3$. Зная $\lambda(t)$ и $\Lambda(t)$, можно записать логарифм правдоподобия, используя полученные ранее результаты.

Недостаток этого подхода заключается в том, что $\lambda$ и $\Lambda$ могут принимать отрицательные значения и функция риска может оказаться несобственной, поскольку функция плотности не обязательно будет интегрироваться к единице.




\section{Модель пропорциональных рисков Кокса}\label{sec:17.8}

\noindent
Для цензурированных данных с единственным переходом, или двумя возможными состояниями, оценивать полностью параметрические модели относительно легче, однако неверная спецификация любого из параметров модели приводит к несостоятельным оценкам. Эту проблему можно отчасти решить за счет выбора более гибкой параметрической функциональной формы. Но несмотря на аргументированность % ALT: обоснованность
подхода, идентификация и оценка таких функциональных форм часто оказывается затруднена. Примером является обобщенная гамма модель.

Другим решением проблемы является применение полупараметрических методов, которые не требуют спецификации распределений всех параметров и, таким образом, позволяют снизить зависимость оценок от выбора функциональной формы. Модели тобит для цензурированных данных также оказываются недостаточно робастны. Однако методы, применяемые при их оценивании, существенно отличаются от тех, которые используются при анализе выживаемости, где объектом моделирования является коэффициент риска, который не имеет содержательной интерпретации в случае тобит. Подход, рассматриваемый в данном разделе, оказался настолько успешен на практике, что стал стандартным для анализа данных о выживаемости.


\subsection{Модель пропорциональных рисков}\label{sec:17.8.1}

\noindent
В первую очередь, необходимо выбрать функциональную форму для коэффициента риска в модели PH (см. раздел \ref{sec:17.7.1}) таким образом, чтобы условный коэффициент риска $\lambda(t|\mathbf{x},\bm{\beta})$ состоял из двух независимых функций
        \begin{align}
        \label{eq:17.25}
        \lambda(t|\mathbf{x},\bm{\beta})=\lambda_0(t)\phi(\mathbf{x},\bm{\beta}),
        \end{align}
где $\lambda_0(t)$ называется базовым риском и является функцией только от $t$, а функция $\phi(\mathbf{x},\bm{\beta})$ зависит только от регрессоров $\mathbf{x}$. В начале мы будем предполагать, что $\mathbf{x}$ не зависят от времени, но позже опустим эту предпосылку. Модель является полупараметрической, так как в ней отсутствует спецификация функциональной формы для $\lambda_0(t)$, в то время как функциональная форма для $\phi(\mathbf{x},\bm{\beta})$ полностью определена. Чаще всего предполагают, что $\phi(\mathbf{x},\bm{\beta})$ принимает экспоненциальную форму
        \begin{align}
        \label{eq:17.26}
        \phi(\mathbf{x},\bm{\beta})=\exp(\mathbf{x}'\bm{\beta}),
        \end{align}
поскольку она позволяет легко интерпретировать коэффициенты, гарантируя при этом, что $\phi(\mathbf{x},\bm{\beta})>0$. Например, увеличив $j$-ый регрессор $x_j$ на единицу, считая, что все остальные регрессоры неизменны, получим
        \begin{align}
        \label{eq:17.27}
        \lambda(t|\mathbf{x}_{\textrm{new}},\bm{\beta})&=\lambda_0(t)\exp(\mathbf{x}'\bm{\beta}+\bm{\beta}_j)\\
        &=\exp(\beta_j)\lambda(t|\mathbf{x},\bm{\beta}).\notag
        \end{align}
То есть, новый риск равен произведению исходного риска на $\exp(\beta_j)$, а изменение риска --- его произведению на $1-\exp(\beta_j)$. Взяв производную, получим, что изменение риска равно произведению исходного риска на коэффициент $\beta_j$
        \begin{align}
        \label{eq:17.28}
        \partial\lambda(t|\mathbf{x},\bm{\beta})/\partial x_j=\lambda_0(t)\exp(\mathbf{x}'\bm{\beta})\beta_j=\beta_j\lambda(t|\mathbf{x},\bm{\beta}),
        \end{align}
что соответствует предыдущему результату, поскольку $\exp(\beta_j)\simeq1+\beta_j$. Статистические пакеты обычно предоставляют оценки с соответствующими доверительными интервалами для обоих показателей, $\beta_j$ и $\exp(\beta_j)$.

В общем виде изменения регрессоров оказывают мультипликативный эффект на исходный риск
        \begin{align}
        \label{eq:17.29}
        \partial\lambda(t|\mathbf{x},\bm{\beta})/\partial\mathbf{x}&=\lambda_0(t)\partial\phi(\mathbf{x},\bm{\beta})/\partial x_j\\
        &=\lambda(t|\mathbf{x},\bm{\beta})\times[\partial\phi(\mathbf{x},\bm{\beta})/\partial x_j]/\phi(\mathbf{x},\bm{\beta}), \notag
        \end{align}
для оценки которого требуется лишь знание коэффициентов $\bm{\beta}$, но не базового риска $\lambda_0(t)$. Идентификация моделей PH обсуждается в следующей главе, где рассматривается более общий случай, допускающий наличие ненаблюдаемой гетерогенности.


\subsection{Оценивание методом частичного правдоподобия}\label{sec:17.8.2}

Кокс (1972, 1975) предложил модель PH, которая не требует одновременного оценивания коэффициентов $\bm{\beta}$ и базового риска $\lambda_0(t)$. При желании, базовый риск может быть восстановлен после оценивания $\bm{\beta}$. Результаты представлены с учетом независимого цензурирования и наличия идентичных данных.

Аналогично \ref{sec:17.5}, данные состоят из завершенных наблюдений и наблюдений, находящихся под риском. Обозначим $t_1<t_2< \ldots <t_j< \ldots <t_k$ как \textbf{дискретные моменты отказа} объектов в выборке размера $N$, $N\ge k$. \textbf{Множество объектов под риском} $R(t_j)$ включает в себя наблюдения, находящиеся под риском непосредственно перед моментом $t_j$, $D(t_j)$ --- наблюдения, завершенные в момент $t_j$, а $d_j$ равно количеству завершенных наблюдений. Таким образом,
        \begin{align}
        \label{eq:17.30}
        R(t_j) &=\{l:t_l\ge t_j\}           &&=\textrm{множество объектов под риском в момент } t_j,\\
        D(t_j) &=\{l:t_l=t_j\}              &&=\textrm{множество объектов, завершенных в момент } t_j,\notag\\
        d_j    &=\sum_l \mathbf{1}(t_l=t_j) &&=\textrm{количество объектов, завершенных в момент } t_j.\notag
        \end{align}
Множество объектов под риском содержит все незавершенные и нецензурированные на момент $t_j$ наблюдения. Если $d_j>1$, то имеет место набор идентичных данных.

Вероятность того, что определенный объект, находящийся под риском, будет завершен в момент $t_j$ равна условной вероятности отказа объекта $j$, деленной на условную вероятность отказа любого из объектов из множества $R(t_j)$, или сумму вероятностей отказа каждого объекта в множестве $R(t_j)$. То есть,
        \begin{align}
        \Pr\left[T_j=t_j|R(t_j)\right]&=\frac{\Pr\left[T_j=t_j|T_j\ge t_j\right]}{\sum_{l\in R(t_j)}\Pr\left[T_l=t_l|T_l\ge t_j\right]}\notag\\
        &=\frac{\lambda_j(t_j|\mathbf{x}_j,\bm{\beta})}{\sum_{l\in R(t_j)}\lambda_l(t_j|\mathbf{x}_l,\bm{\beta})}\notag\\
        &=\frac{\phi(\mathbf{x}_j,\bm{\beta})}{\sum_{l\in R(t_j)}\phi(\mathbf{x}_l,\bm{\beta})}. \notag
        \end{align}
Поскольку риски пропорциональны, базовый риск в последней строке сокращается. Как следствие, свободный член в модели неидентифицируем, и его отсутствие упрощает оценивание коэффициентов $\bm{\beta}$. Однако формула требует корректировки, если возникают идентичные данные, характерные для сгруппированных длительностей (то есть, если несколько наблюдений завершились в одном интервале группировки). Предположим, например, что известны два наблюдения $j_1$ и $j_2$, идентичных в момент $t_j$, с набором регрессоров $\mathbf{x}_{j1}$ и $\mathbf{x}_{j2}$, соответственно. Если $j_1$ завершается раньше $j_2$, то формула вероятности равна
        $$\phi(\mathbf{x}_{j1},\bm{\beta})/\sum_{l\in R(t_j)}\phi(\mathbf{x}_l,\bm{\beta})+\phi(\mathbf{x}_{j2},\bm{\beta})/\sum_{l\in R_1(t_j)}\phi(\mathbf{x}_l,\bm{\beta}),$$
где множество $R_1(t_j)$ равно $R(t_j)$, за исключением объекта $j_1$. Выражение аналогично, если $j_2$ завершается раньше $j_1$. Вклад в функцию правдоподобия будет равен сумме этих вероятностей. Однако формула правдоподобия становится громоздкой, если в выборке содержится достаточно большой набор идентичных данных, поэтому часто используют приближение Бреслоу и Пето, см. Кокс и Оакс (1984)
        \begin{align}
        \label{eq:17.31}
        \Pr\left[T_j=t_j|j\in R(t_j)\right]\simeq\frac{\prod_{m\in D(t_j)}\phi(\mathbf{x}_m,\bm{\beta})}{\left[\sum_{l\in R(t_j)} \phi(\mathbf{x}_l,\bm{\beta})\right]^{d_j}},
        \end{align}
где $D(t_j)$ является множеством наблюдений, завершенных в момент $t_j$, а $d_j$ обозначает их количество. Приближение верно, если количество отказов ($d_j$) в момент $t_j$ относительно мало по сравнению с количеством объектов под риском.

Произведение вероятностей отказа $\Pr\left[T_j=t_j|j\in R(t_j)\right]$ для первых k наблюдений называется \textbf{частичная функцией  правдоподобия}, предложенной Коксом для оценивания коэффициентов $\bm{\beta}$. То есть,
        \begin{align}
        \label{eq:17.32}
        \mathrm{L}_p(\bm{\beta})=\prod^{k}_{j=1}\frac{\prod_{m\in D(t_j)}\phi(\mathbf{x}_m,\bm{\beta})}{\left[\sum_{l\in R(t_j)} \phi(\mathbf{x}_l,\bm{\beta})\right]^{d_j}}.
        \end{align}
Суть метода заключается в максимизации логарифма частичной функции правдоподобия % Опечатка в Кэмероне?? Минимизация?
        \begin{align}
        \label{eq:17.33}
        \ln\mathrm{L_p}=\sum^{k}_{j=1}\left[\sum_{m\in D(t_j)}\ln\phi(\mathbf{x}_m,\bm{\beta})-d_j\ln\left(\sum_{l\in R(t_j)}\phi(\mathbf{x}_l,\bm{\beta})\right)\right],
        \end{align}
где цензурированные наблюдения содержатся лишь во втором члене функции $\mathrm{L_p}$, поскольку составляют множество объектов под риском до тех пор, пока не оказываются цензурированы.

Пусть $\delta_i = 1$ является индикатором отсутствия цензурирования. Тогда уравнение \ref{eq:17.33} может быть перезаписано в виде
        \begin{align}
        \label{eq:17.34}
        \ln\mathrm{L_p}(\bm{\beta})=\sum^{N}_{i=1}\delta_i\left[\ln\phi(\mathbf{x}_i,\bm{\beta})-\ln\left(\sum_{l\in R(t_i)}\phi(\mathbf{x}_l,\bm{\beta})\right)\right].
        \end{align}

Для стандартной спецификации $\phi(\mathbf{x},\bm{\beta})=\exp(\mathbf{x}'\bm{\beta})$, или $\ln\phi(\mathbf{x},\bm{\beta})=\mathbf{x}'\bm{\beta}$, условия первого порядка выглядят следующим образом
        $$\frac{\partial\ln\mathrm{L_p}(\bm{\beta})}{\bm{\beta}}=\sum^{N}_{i=1}\delta_i[\mathbf{x}_i-\mathbf{x}^*_i(\bm{\beta})]=\mathbf{0},$$
где $\mathbf{x}^*_i (\beta)=\sum_{l\in R(t_i)}\mathbf{x}_l\exp(\mathbf{x}_l'\bm{\beta})/\sum_{l\in R(t_i)}\exp(\mathbf{x}_l'\bm{\beta})$ представляет собой взвешенное среднее по регрессорам $\mathbf{x}_l$ из множества объектов под риском в момент отказа $t_i$.

Данный метод является методом максимального правдоподобия с ограниченной информацией, поскольку оцениваемая функция не содержит информации о базовом риске $\lambda_0(t)$, но при этом не относится к моделям условного или маргинального правдоподобия. В литературе представлено широкое обсуждение того, можно ли считать $\mathrm{L_p}(\bm{\beta})$ функцией правдоподобия. Можно показать (Андерсен и др., 1993), что, хотя $\ln \mathrm{L_p}$ не является настоящей функцией правдоподобия, полученные оценки, при которых $\mathrm{L_p}$ максимально, тем не менее, состоятельны. См. также Кальбфляйш и Прентис (2002, стр. 99-101) и Ланкастер (1990, глава 9).

Используя результаты, представленные в главе 5, % \ref{ch:5} # UNCOMMENT AFTER THE END OF CHAPTER 5
и предположив для упрощения, что $\mathbf{A}(\bm{\beta})=-\mathbf{B}(\bm{\beta})$, получим
        \begin{align}
        \label{eq:17.35}
        \bm{\hat{\beta}}\thicksim\!\!\!\!\!^{{}^{a}}\hspace{0.1cm}\mathcal{N} \left[ \bm{\beta},\left(-\E\left[\frac{\partial^2\ln\mathrm{L_p}(\bm{\beta})}{\partial\bm{\beta}\partial\bm{\beta}'}\right]\right)^{-1} \right].
        \end{align}
Оценка не является эффективной, хотя по сравнению с полностью параметрическими моделями PH, например, моделью Вейбулла, разница в дисперсии оценки небольшая.


\subsection{Функция выживания в модели Кокса}\label{sec:17.8.3}

Большинство исследований ограничивается оцениванием и интерпретацией коэффициентов $\bm{\beta}$, используя для этого формулы \ref{eq:17.28} или \ref{eq:17.29}, в то время как представленные методы позволяют проводить более содержательный анализ. Например, некоторые работы дополнительно исследуют форму функции базового риска. Зная же оценки коэффициентов $\bm{\beta}$, полученные методом частичного правдоподобия в модели PH, можно вывести вывести функцию выживания, оценка которой аналогична оценке Каплан-Мейреа в разделе \ref{sec:17.5.1}.

Функция выживания соответствует функции риска следующим образом: $S(t|\mathbf{x},\bm{\beta})=\exp\left[-\int^{t}_{0}\lambda_0(s)\phi(\mathbf{x},\bm{\beta})ds\right]$. Также мы можем определить $S_0(t)=\exp\left[-\int^{t}_{0}\lambda_0(s)ds\right]$. Тогда
        $$S(t|\mathbf{x},\bm{\beta})=S_0(t)^{\phi(\mathbf{x},\bm{\beta})}.$$

Рассмотрим дискретный случай, где базовый риск равен $1-\alpha_j$ в каждый момент времени $t_j$, $j=1, \ldots ,k$. Оценка $\hat{\alpha}_j$ является решением уравнения, рассмотренного более подробно в следующем разделе, \ref{sec:17.8.4}
        \begin{align}
        \label{eq:17.36}
        \sum^{k}_{l\in D(t_j)}\frac{\phi(\mathbf{x}_l,\bm{\hat{\beta}})}{1-\hat{\alpha}_j^{\phi(\mathbf{x}_l,\bm{\hat{\beta}})}}=\sum_{m\in R(t_j)}\phi(\mathbf{x}_m,\bm{\hat{\beta}}),\hspace{0.3cm}j=1, \ldots ,k,
        \end{align}
где $\bm{\hat{\beta}}$ получена методом частичного правдоподобия, $D(t_j)$ обозначает множество объектов, умерших в $t_j$, а $R(t_j)$ --- множество объектов под риском в момент $t_j$. Из раздела \ref{sec:17.3.2} % Опечатка в Cameron: 17.3.3
мы знаем, что функция выживания в дискретном случае равна произведению мгновенных условных вероятностей дожития, $S_0(t)=\prod_{j|t_j\le t}\alpha_j$. Значит, оценка функции выживания будет равна
        \begin{align}
        \label{eq:17.37}
        \hat{S}_0(t)=\prod_{j|t_j\le t}\hat{\alpha}_j.
        \end{align}
При отсутствии регрессоров $\hat{S}_0(t)$ упрощается до оценки Каплан-Мейера. То есть, полагая $\phi(\mathbf{x}_j,\bm{\beta})=1$, получим коэффициент базового риска $1-\hat{\alpha_j}=d_j/r_j$. При наличии регрессоров, но отсутствии идентичных наблюдений, коэффициент базового риска будет равен $1-\hat{\alpha}_j=\phi(\mathbf{x}_j,\bm{\hat{\beta}})/\sum_{m\in R(t_j)}\phi(\mathbf{x}_j,\bm{\hat{\beta}})$.

Для наблюдений со взвешенными регрессорами $\mathbf{x}=\mathbf{x}^*$ функция выживания оценивается с помощью
        $$\hat{S}(t|\mathbf{x}^*,\bm{\beta})=\hat{S}_0(t)^{\phi(\mathbf{x}^*,\hat{\bm{\beta}})}.$$
Линейные преобразования регрессоров не влияют на оценки $\bm{\beta}$, но влияют на функцию базового риска. Например,
        \begin{align}
        \lambda(t|\mathbf{x},\bm{\beta})&=\lambda_0\exp(\mathbf{x}'\bm{\beta})\notag\\
        &=\lambda_0(t)\exp(\mathbf{\bar{x}}'\bm{\beta})\exp((\mathbf{x}-\mathbf{\bar{x}})'\bm{\beta})\notag\\
        &=\lambda^*_0(t)\exp((\mathbf{x}-\mathbf{\bar{x}})'\bm{\beta}),\notag
        \end{align}
где новый базовый риск равен $\lambda^*_0(t)\exp((\mathbf{x}-\mathbf{\bar{x}})'\bm{\beta})$. % опечатка
То есть, вычитание выборочного среднего приведет к изменению базового риска, и в таком случае нужно быть аккуратным при интерпретации базового риска и функции выживания.

Несмотря на то, что оценка базового риска играет важную роль при сравнении коэффициентов риска между различными группами наблюдений, часто для ее интерпретации целесообразно проводить сглаживание, поскольку из-за разрывов сравнивать коэффициенты может оказаться непросто.


\subsection{Вывод функции выживания}\label{sec:17.8.4}

\noindent
Уравнение \ref{eq:17.36} для оценивания $\alpha_j$ получено согласно Кальбфляйш и Прентис (2002, стр. 114-118).

Вклад объекта с длительностью $t_j$ в функцию правдоподобия равен вероятности дожить до момента $t>t_{j-1}$ за вычетом вероятности дожития до момента $t>t_j$. То есть,
        \begin{align}
        S(t_j|\mathbf{x},\bm{\beta})-S(t_{j+1}|\mathbf{x},\bm{\beta})&=S_0(t_j)^{\phi(\mathbf{x},\bm{\beta})}-S_0(t_{j+1})^{\phi(\mathbf{x},\bm{\beta})}\notag\\
        &=(\alpha^{-1}_jS_0(t_{j+1}))^{\phi(\mathbf{x},\bm{\beta})}-S_0(t_{j+1})^{\phi(\mathbf{x},\bm{\beta})}\notag\\
        &=(\alpha^{-\phi(\mathbf{x},\bm{\beta})}_j-1)S_0(t_{j+1})^{\phi(\mathbf{x},\bm{\beta})}\notag
        \end{align}
где $S_0(t_{j+1})=\prod^{j}_{l=1}\alpha_l=\alpha_jS_0(t_j)$. 

Вклад объектов, цензурированных в момент $t_j$, равен вероятности дожить до $t > t_j$, или $S_0(t_{j+1})^{\phi(\mathbf{x},\bm{\beta})}$. Таким образом, умершие или цензурированные объекты за период $[t_j,t_{j-1})$ входят в вероятность следующим образом $S_0(t_{j+1})^{\phi(\mathbf{x},\bm{\beta})}=\prod^{j}_{l=1}\alpha^{\phi(\mathbf{x},\bm{\beta})}_l$, а для умерших есть ещё дополнительный множитель $\left(\alpha^{-\phi(\mathbf{x},\bm{\beta})}_j-1\right)$. Следовательно, можно записать функцию правдоподобия по всем объектам

        $$\mathrm{L}(\bm{\alpha},\bm{\beta})=\prod^{k}_{j=1}\left[\prod_{l\in D(t_j)}(\alpha^{-\phi(\mathbf{x}_l,\bm{\beta})}_j-1)\prod_{m\in R(t_j)}(\alpha^{-\phi(\mathbf{x}_m,\bm{\beta})}_j)\right].$$
Логарифм правдоподобия равен
        $$\ln\mathrm{L}(\bm{\alpha},\bm{\beta})=\sum^{k}_{j=1}\left[\sum_{l\in D(t_j)}\ln(\alpha^{-\phi(\mathbf{x}_l,\bm{\beta})}_j-1)+\sum_{m\in R(t_j)}-\phi(\mathbf{x}_m,\bm{\beta})\ln\alpha_j \right].$$
Условие первого порядка $\partial\ln\mathrm{L}(\bm{\alpha},\bm{\hat{\beta}})/\partial\alpha_j=0$ --- это и есть выражение \ref{eq:17.36}.
% ALT: Используя условие первого порядка $\partial\ln\mathrm{L}(\bm{\alpha},\bm{\hat{\beta}})/\partial\alpha_j=0$, получим выражение \ref{eq:17.36}.




\section{Регрессоры, меняющиеся со временем}\label{sec:17.9}

\noindent
До сих пор мы рассматривали регрессоры, которые могут меняться между объектами, но постоянны во времени для каждого объекта, например, пол. Такая структура данных типична для пространственных моделей, логит-моделей или моделей тобит. В моделях выживаемости, однако, характеристики объектов могут меняться в пределах одного состояния, в связи с чем соответствующие регрессоры будут варьироваться во времени. Например, в ходе лечения может меняться доза лекарственных препаратов, а в течение периода поиска работы могут поменяться размеры пособия по безработице. Или же индивид может жениться, оставаясь при этом безработным, вследствие чего изменится значение переменной, отвечающей за семейный статус.

При работе с регрессорами, меняющимися со временем, возникает два типа проблем. Во-первых, полная история изменений переменной может предсказывать коэффициент риска, в связи с чем потребуется использовать лаговые переменные в качестве объясняющих. Их же отсутствие приведет к неверной спецификации модели. Во-вторых, такие регрессоры могут испытывать \textbf{эффект обратной связи} и поэтому не будут являться экзогенными, как обычно предполагается в моделях времени жизни. Например, длительность безработного состояния может зависеть от стратегии поиска работы, однако сама стратегия может также меняться в зависимости от количества прошедшего времени. Аналогично, доза лекарственного препарата может определяться состоянием пациента.

Поскольку детерминированные изменения во времени учитывать проще, стандартный анализ справляется с проблемами первого типа при условии, что регрессоры слабо экзогенны. То есть, независимо от того, является ли процесс, определяющий изменчивость во времени, стохастичным или детерминированным, нет необходимости знать параметры этого процесса. Некоторые авторы (например, Кальбфляйш и Прентис, 2002, стр. 196-200) классифицируют такую изменчивость во времени как \textbf{внешнюю} \textit{(external time variation)}. Эндогенные регрессоры же определяют \textbf{внутреннюю} изменчивость \textit{(internal time variation)}.

Простым решением первого рода проблем (особенно, если программное обеспечение не поддерживает оценку регрессоров, меняющихся со временем) будет являться замена изменчивой во времени переменной на ее среднее по всем наблюдениям значение за весь период длительности состояния. Качественное ПО, тем не менее, допускает изменчивость регрессоров.

Рассмотрим некий объект, например, безработного индивида, который находится в таком состоянии в течение времени $T$, а затем совершает переход, то есть, находит работу. Пусть $0<t_1<t_2<t_3<T$, где $t_1$, $t_2$ и $t_3$ являются промежуточными моментами наблюдения. Предположим далее, что существует два объясняющих фактора $x_1$ и $x_2(t)$, первый из которых постоянен во времени, а второй изменчив. Для простоты допустим, что $x_1$ является бинарной переменной, а $x_2$ принимает значения $x_2(t_1)$, $x_2(t_2)$ и $x_3(t_3)$ в соответствующих интервалах $[0,t_1)$, $[t_1,t_2)$ и $[t_2,T)$. Наконец, предположим, что меняющийся со временем регрессор экзогенен, а процесс, определяющий его изменчивость детерминирован. Тогда эти данные можно записать в виде следующей таблицы
    \begin{table}[!htbp]
    \begin{center}
\begin{tabular}{ccccc}
\hline \hline
\textbf{Наблюдение}&\textbf{Длительность}&$\bm{x_1}$&$\bm{x_2(t)}$&\textbf{Индикатор цензурирования}\\
\hline
1   &$t_1$  &1  &$x_2(t_1)$ &0\\
1   &$t_2$  &1  &$x_2(t_2)$ &0\\
1   &$T$    &1  &$x_2(T)$   &1\\
\hline \hline
\end{tabular}
    \end{center}
    \end{table}

Смысл заключается в том, что мы можем разбить одну длительность на три части в соответствии с изменениями регрессоров. Таким образом, значения регрессоров будут равны $(1,x_2(t_1))$ и $(1,x_2(t_2))$ для первых двух частей, и $(1,x_2(T))$ для третьей. Интуитивно, это аналогично тому, как если бы у нас было три наблюдения, два из которых были бы цензурированы ($\delta_i = 0$) и одно завершено ($\delta_i = 1$).

Предположим теперь, что коэффициент риска может также зависеть от предыдущих значений переменной, меняющейся со временем, $x_2(t)$. Тогда структура данных будет выглядеть следующим образом
    \begin{table}[!htbp]
    \begin{center}
\begin{tabular}{cccccc}
\hline \hline
\textbf{Наблюдение}&\textbf{Длительность}&$\bm{x_1}$&$\bm{x_2(t)}$&$x_2(t-1)$&\textbf{Индикатор цензурирования}\\
\hline
1   &$t_1$  &1  &$x_2(t_1)$ &0            &0\\
1   &$t_2$  &1  &$x_2(t_2)$ &$x_2(t_1)$   &0\\
1   &$T$    &1  &$x_2(T)$   &$x_2(t_2)$   &1\\
\hline \hline
\end{tabular}
    \end{center}
    \end{table}

Мы предположили, что до начала изучаемого состояния значение переменной $x_2(t)$ равнялось 0. Заметим также, что в обоих примерах $x_2(t)$ изменяется дискретно.

Несмотря на то, что запись наблюдений, состоящих из нескольких строк, возможна, для больших объемов данных она может оказаться громоздкой и неудобной. В особенности, трудности возникнут, если ПО начнет распознавать строки как отдельные наблюдения. К счастью, статистические пакеты обычно предлагают опцию задать переменные как меняющиеся со временем.
Можно также подогнать ступенчатые или непрерывные функции на основе прошедшего в определенном состоянии количества времени. % не очень понятно это предложение


\subsection{Расширенная модель Кокса}\label{sec:17.9.1}

Представленную в разделе \ref{sec:17.8} модель Кокса с постоянными регрессорами можно легко обобщить на случай регрессоров, меняющихся со временем.

В общем виде функция риска зависит от всей траектории регрессоров $\mathbf{x}(t)$ следующим образом
        $$\lambda(t|\mathbf{x}(t))=\lim_{\Delta t\rightarrow0}\frac{\Pr[t\le T<t+\Delta t|\mathbf{x}(t),T\ge t]}{\Delta t}.$$
В форме PH функция риска зависит лишь от текущих значений $\mathbf{x}(t)$
        $$\lambda(t|\mathbf{x}(t))=\lambda_0(t,\bm{\alpha})\phi(\mathbf{x}(t)),\bm{\beta}).$$

Из раздела \ref{sec:17.8.2} мы знаем, что метод частичного правдоподобия основывается на оценивании регрессоров $\mathbf{x}(t_j)$ для объектов, принадлежащих множеству под риском $R(t_j)$. Значит, в каждый момент отказа $t_j$ необходимо заменить $\mathbf{x}_i$ на $\mathbf{x}_i(t_j)$. Функция правдоподобия в таком случае примет вид
    $$\ln\mathrm{L_p}=\sum^{k}_{j=1}\left[\sum_{m\in D(t_j)}\ln\phi(\mathbf{x}_m(t_j),\bm{\beta})-d_j\ln\left(\sum_{l\in R(t_j)}\phi(\mathbf{x}_l(t_j),\bm{\beta})\right)\right].$$

Заметим, что данные теперь представляют собой более сложную структуру, предполагающую наличие нескольких наблюдений для каждого объекта. В качестве иллюстрации предположим, что время дискретно, длительность наблюдения равна 25 и присутствует только один регрессор $x_1$, который принимает значение 50 в интервале $[0,5]$, 100 в $[6,15]$ и 200 в $[16,25]$. Пусть моменты отказа упорядочены и равны, соответственно, 3, 8, 13,18 и 25. Тогда $x_1(t_1)=50$, $x_1(t_2)=100$, $x_1(t_3)=100$, $x_1(t_4)=200$, и $x_1(t_5)=200$.




\section{Пропорциональные риски в дискретном времени}\label{sec:17.10}

\noindent
Если наблюдаются только агрегированные интервалы (такие как неделя или месяц), в которых происходят отказы, то для анализа подходят модели сгруппированных длительностей.

Простая идея заключается в том, чтобы сформировать панель данных и затем оценить ее с помощью составной \textit{(stacked)} логит или пробит модели для вероятности отказа с индивидуальными константами в каждом периоде. Этот подход продемонстрирован в разделе \ref{sec:17.10.3}. Но сначала мы представим модель PH в дискретном виде, которая была рассмотрена во многих работах, включая Кальбфляйш и Прентис (1980), Фармейер и Тутц (1994), Кифер (1988) и Мейер (1990). Мы будем использовать те же обозначения, что и Блейк, Лунде и Тиммерманн (1999).


\subsection{Пропорциональные риски в дискретном времени}\label{sec:17.10.1}

\noindent
Для сгруппированных данных с моментами группировки $t_a$, $a=1, \ldots ,A,$ функция риска в дискретном времени задается как
        $$\lambda^{\mathrm{d}}(t_a|\mathbf{x})=\Pr[t_{a-1}\le T<t_a|T\ge t_{a-1},\mathbf{x}(t_{a-1})], \hspace{0.3cm} a=1, \ldots ,A.$$
Допускается, что регрессоры могут меняться во времени. Соответствующая функция выживания в дискретном времени равна
        $$S^{\mathrm{d}}(t_a|\mathbf{x})=\Pr[T\ge t_{a-1}|\mathbf{x}]=\prod^{a-1}_{s=1}\left(1-\lambda^{\mathrm{d}}(t_s|\mathbf{x}(t_s))\right).$$
Сначала определим общую зависимость между коэффициентами риска в дискретном и непрерывном времени. В дискретном случае риск отказа равен вероятности того, что отказ произойдет в интервале $[t_{a-1},t_a)$, деленной на вероятность дожить, по крайней мере, до момента $t_{a-1}$, то есть
        \begin{align}
        \label{eq:17.38}
        \lambda^{\mathrm{d}}(t_a|\mathbf{x})=\frac{S(t_{a-1}|\mathbf{x})-S(t_{a}|\mathbf{x})}{S(t_{a-1}|\mathbf{x})},
        \end{align}
где $S(t|\mathbf{x})$ является функцией выживания. Зная, что $S(t|\mathbf{x}) = \exp(-\int^{t}_{0}\lambda(s)ds)$ в непрерывном случае, после некоторых преобразований получим
% ALT: уравнение \ref{eq:17.38} можно перезаписать как
        \begin{align}
        \label{eq:17.39}
        \lambda^{\mathrm{d}}(t_a|\mathbf{x})=1-\exp(-\int^{t_a}_{t_{a-1}}\lambda(s)ds).
        \end{align}
В непрерывном случае дискретный риск можно определить как
        $$\lambda(t)=\lambda_0(t)\exp(\mathbf{x}(t_{a-1})'\bm{\beta})$$
для $t$ из промежутка $[t_{a-1},t_a)$. Заметим, что регрессоры постоянны внутри интервалов, но могут меняться между интервалами, а $\lambda_0(t)$ может изменяться и внутри интервала. Тогда уравнение \ref{eq:17.39} принимает вид
        \begin{align}
        \label{eq:17.40}
        \lambda^{\mathrm{d}}(t_a|\mathbf{x})&=1-\exp(-\exp(\mathbf{x}(t_{a-1})'\bm{\beta})\times\int^{t_a}_{t_{a-1}}\lambda(s)ds)\\
        &=1-\exp(-\lambda_{0a}\exp(\mathbf{x}(t_{a-1})'\bm{\beta}))\notag\\
        &=1-\exp(-\exp(\lambda_{0a}+\mathbf{x}(t_{a-1})'\bm{\beta})),\notag
        \end{align}
где $\lambda_{0a}=\int^{t_a}_{t_{a-1}}\lambda_{0}(s)ds$. Соответствующая функция выживания равна
        \begin{align}
        \label{eq:17.41}
        S^{\mathrm{d}}(t_a|\mathbf{x})=\prod^{a-1}_{s=1}\exp\left(-\exp(\ln\lambda_{0s}+\mathbf{x}(t_{s-1})'\bm{\beta})\right).
        \end{align}

Функция плотности для $i$-го объекта равна произведению функции выживания в каждом периоде, в котором он оставался живым, и функции риска в каждом моменте отказа. Тогда, на основе \ref{eq:17.40} и \ref{eq:17.41} можно записать функцию правдоподобия
        \begin{align}
        \label{eq:17.42}
        \mathrm{L}(\bm{\beta},\lambda_{01}, \ldots ,\lambda_{0A})=&\prod^{N}_{i=1}\left[\prod^{a_i-1}_{s=1}\exp(-\exp(\ln\lambda_{0s}+\mathbf{x}_i(t_{s-1})'\bm{\beta}))\right]\\
        &\times(1-\exp(-\exp(\ln\lambda_{0a_i}+\mathbf{x}_i(t_{a-1})'\bm{\beta}))).\notag
        \end{align}
Для упрощения мы не рассматриваем цензурированные наблюдения, а отказ для $i$-го наблюдения наступает в момент $t_{a_i}$. Предполагается, что в промежутке $[t_{a-1},t_a)$ происходит по крайней мере один отказ.

ММП предполагает максимизацию уравнения \ref{eq:17.42} по отношению к $\bm{\beta}$ и $\lambda_{01}, \ldots ,\lambda_{0A}$. В общем случае ММП отличается от метода частичного правдоподобия, хотя в некоторых случаях они могут быть эквивалентны. Модели с ограниченным числом параметров допускают наличие структуры в параметрах $\lambda_{01}, \ldots ,\lambda_{0A}$, например, полиномиальную зависимость от времени. Еще более ограниченными являются полностью параметрические модели, такие как модель Вейбулла, которая подразумевает, что $\lambda_{0s}=\int^{t_a}_{t_{a-1}}\alpha s^{\alpha-1}ds$.


\subsection{Подход Хана и Хаусмана}\label{sec:17.10.2}

\noindent
Хан и Хаусман предложили относительно простой и удобный подход для оценивания базового риска, который предшествует работе Блейк и др. (1999) и имеет некоторое сходство с работами Мейера (1990) и Сейоши (1992). Этот подход допускает значительную свободу в плане спецификации базового риска $\lambda^{d}_{0}(t)$, но в то же время сохраняет параметрическую форму для объясняющих переменных (например, $\exp(\mathbf{x}'\bm{\beta})$). К преимуществам можно отнести также то, что он в явном виде предназначен для работы с дискретными данными и позволяет проще учитывать такие их особенности, как идентичность и ненаблюдаемую гетерогенность. Идентичные данные однако могут представлять основную проблему при работе с дискретными данными, так как, например, длительность безработного состояния может часто совпадать с окончанием выплат пособия по безработице (которое в США составляет 26 недель).

Начнем с того, что для наблюдения $i$ определим коэффициент риска $\lambda_i(\tau)$ в форме PH, равный условной вероятности того, что объект будет завершен в промежутке $(\tau,\tau+\Delta)$
        $$\lambda_i(\tau)=\lambda_0(\tau)\exp(-\mathbf{x}_i'\bm{\beta}),$$
где $\lambda_0(\tau)$ обозначает базовый риск. Взяв логарифмы и поменяв переменные местами, как уже было показано в \ref{eq:17.20}, получим
        \begin{align}
        \label{eq:17.43}
        \Lambda_0(t)-\mathbf{x}_i'\bm{\beta}=\varepsilon_i
        \end{align}
где $\Lambda_0(t)=\ln\int^{t}_{0}\lambda_0(\tau)d\tau$ обозначает логарифм интегрального риска и $\varepsilon_i=\ln\int^{t}_{0}\lambda_i(\tau)d\tau$. Тогда можно записать вероятность отказа как
        $$\Pr[\textrm{отказа в периоде t}]=\int^{\Lambda_0(t)-\mathbf{x}_i'\bm{\beta}}_{\Lambda_0(t-1)-\mathbf{x}_i'\bm{\beta}}f(\varepsilon)d\varepsilon$$
Пусть $y_{it}=1$, если отказ $i$-го объекта происходит в момент $t$, и $y_{it}=0$ в обратном случае. Тогда функция правдоподобия для $N$ объектов будет равна
        \begin{align}
        \label{eq:17.44}
        \ln\Lambda(\bm{\beta},\Lambda_0(1), \ldots ,\Lambda_0(T))=\sum^{N}_{i=1}\sum^{T}_{t=1}y_{it}\ln\left[\int^{\Lambda_0(t)-\mathbf{x}_i'\bm{\beta}}_{\Lambda_0(t-1)-\mathbf{x}_i'\bm{\beta}}f(\varepsilon)d\varepsilon\right]
        \end{align}
где $\Lambda_0(1), \ldots ,\Lambda_0(T)$ оцениваются одновременно с $\bm{\beta}$. Спецификация функциональной формы при этом не важна.

Интеграл в функции правдоподобия равен разнице функция распределения $[\Lambda_0(t-1)-\mathbf{x}_i'\bm{\beta},\Lambda_0(t)-\mathbf{x}_i'\bm{\beta}]$. Конкретная форма зависит от предположения о функциональной форме функции распределения. Если случайная ошибка $\varepsilon_i$ подчиняется стандартному нормальному распределению, то логарифм правдоподобия преобразуется в порядковую пробит модель; если же предполагается, что ошибка имеет распределение экстремальных значений, то выражение принимает вид порядковой логит модели. В частности, предполагая, что ошибка распределена нормально, интеграл можно записать как
    $$\Pr[\Lambda_0(t)<\mathbf{x}_i'\bm{\beta}+\varepsilon_i\le\Lambda_0(t+1)]=\Phi(\Lambda_0(t+1)-\mathbf{x}_i'\bm{\beta})-\Phi(\Lambda_0(t)-\mathbf{x}_i'\bm{\beta}).$$

По сравнению с методом частичного правдоподобия, который игнорирует базовый риск, подход, предложенный Ханом и Хаусманом (1990), позволяет одновременно оценивать все неизвестные параметры при относительно невысоких издержках расчета. Результаты моделирования по методу Монте-Карло показывают, что подход хорошо справляется с моделированием случайных функций риска, без каких-либо предположений об их функциональной форме.

\subsection{Бинарный выбор в дискретном времени}\label{sec:17.10.3}

\noindent
Альтернативным способом анализа дискретных данных по длительностям являются модели бинарного выбора, поскольку в каждый момент времени возможен лишь один из исходов --- отказ или наступает, или же нет.

В дискретном времени переходы моделируются как
        \begin{align}
        \label{eq:17.45}
        \Pr[t_{a-1}\le T<t_a|T\ge t_{a-1}|\mathbf{x}]=F(\lambda_a+\mathbf{x}'(t_{a-1})\bm{\beta}), \hspace{0.3cm} a=1, \ldots ,A,
        \end{align}
где коэффициенты при объясняющих переменных постоянны во времени, а константа $\lambda_a$, $a=1, \ldots ,A$ может принимать различные значения. В качестве функции $F$ обычно выбирают нормальное или логистическое распределение. В таком случае, коэффициенты $\lambda_a$ и $\bm{\beta}$ можно оценить с помощью составной логит или пробит модели, в которой константа изменяется со временем. Из-за простоты в применении этот метод является довольно привлекательным.

Итоговая функция правдоподобия имеет вид
        $$\mathrm{L}(\bm{\beta},\lambda_1, \ldots ,\lambda_A)=\prod^{N}_{i=1}\left[\prod^{a_i-1}_{s=1}(1-F\left(\lambda_s+\mathbf{x}'_i(t_{s-1})\bm{\beta}\right))\right]\times F\left(\lambda_{a_i}+\mathbf{x}'(t_{a_i-1})\bm{\beta}\right).$$
За исключением выбора функции $F$, выражение аналогично \ref{eq:17.42}. Так как риск \ref{eq:17.40} имеет распределение экстремальных значений с $\ln\lambda_{0a}+\mathbf{x}(t_{a-1})'\bm{\beta}$, то выражение \ref{eq:17.40} воспроизводит скорее сопряженную лог-логистическую модель бинарного выбора % ENG: "Complementary log-log" Термин должен совпадать с термином из главы 14, таблица 14.3 !!!
(см. таблицу 14.3), % \ref{tab:14.3} # UNCOMMENT AFTER CHAPTER 14
чем обычную логит или пробит-модель.


\section{Пример: длительность состояния безработицы}\label{sec:17.11}

\noindent
Настоящее эмпирическое приложение основано на данных из работы МакКолла (1996), % любезно
предоставленных ее автором, Брайаном МакКоллом. База данных получена из Приложений об уволенных работниках (\textit{Displaced Workers Supplements}, DWS) % ALT: смещенных / замещенных работниках
к Текущему обследованию населения \textit{(Current Population Survey)} за 1986, 1988, 1990 и 1992 гг. Будем называть мерой длительности длительность безработицы, хотя формально мы наблюдаем лишь длительность нахождения индивида без работы, независимо от того, находился ли он или она в поиске работы, или же нет.

Для анализа требуется выработать критерий, на основе которого можно будет относить индивидов, устроившихся на новую после увольнения работу, к работающим на полную ставку или по совместительству. Будем называть индивида частично занятым, если он работал менее 35 часов в неделю, предшествующую моменту опроса, и занятым полный рабочий день в обратном случае.

В таблице \ref{tab:17.6} представлены ключевые экономические факторы, объясняющие продолжительность периода отсутствия работы. Из соображений краткости мы не будем рассматривать весь набор переменных, который был использован в статье МакКолла (1996), а возьмем лишь его часть.
    \begin{table}[!htbp]\caption{\textit{Длительность безработицы: описание переменных}}\label{tab:17.6}
    \begin{center}
\begin{tabular}{llc}
\hline \hline
\textbf{Название переменной}&\textbf{Описание переменной}           &\textbf{Среднее}\\
\hline
spell       &периоды безработицы: двухнедельный интервал            &6.248\\
CENSOR1     &1, если занят полный рабочий день                      &0.321\\
CENSOR2     &1, если является частично занятым                      &0.102\\
CENSOR3     &1, если уволился с новой работы:                       &0.172\\
            &тип занятости неизвестен                               &\\
CENSOR4     &1, если все еще безработный                            &0.375\\
UI          &1, если оформил страховой случай                       &0.553\\
RR          &коэффициент «замещения»                                &0.454\\
DR          &коэффициент «безразличия»                              &0.109\\
TENURE      &количество лет на старой работе                        &4.114\\
LOGWAGE     &логарифм недельной зарплаты                            &5.693\\
\hline \hline
\end{tabular}
    \end{center}
    \end{table}

\begin{figure}[ht!]\caption{Длительность безработицы: Оценки Каплан-Мейера функции выживания. Данные за 1986-92 гг. по 3343 наблюдениям, некоторые из которых не были завершены.}\label{fig:17.3}
\centering
%\includegraphics[scale=0.7]{fig.png}
\end{figure}

Длительности безработицы измерялись с двухнедельными интервалами. Переменные CENSOR1, CENSOR2, CENSOR3 и CENSOR4 являются индикаторами типа новой работы (или ее отсутствия). Для анализа, представленного в данной главе, мы будем использовать лишь первую переменную, CENSOR1. Таким образом, будем считать, что наблюдение завершено, если индивид устроился на новую работу на полный рабочий день. Другая фиктивная переменная UI \textit{(unemployment insurance)} равняется единице, если индивид оформил страховой случай \textit{(UI claim)} и получал, как следствие, пособие по безработице. Переменная RR обозначает коэффициент «замещения» \textit{(replacement rate)}, который равен недельному пособию по безработице, деленному на недельную заработную плату, которую индивид получал на старой работе. Переменная DR обозначает коэффициент «безразличия» \textit{(disregard rate)} и равна максимальному потенциальному заработку за неделю, который индивид мог бы получать, работая на полставки без какого-либо сокращения пособия по безработице, деленному на недельную заработную плату, которую индивид получал на старой работе. Смысл остальных переменных понятен из названий.

Начнем с описательного анализа данных по длительностям безработицы. Сперва, на рисунке \ref{fig:17.3} изобразим кривую выживания Каплан-Мейера. Серыми линиями обозначены 95\% доверительные интервалы, рассчитанные по формулам, представленным в разделе \ref{sec:17.5.2}. Как и ожидалось, кривая выживания быстро убывает в начале, но падение замедляется с течением времени.

Из таблицы \ref{tab:17.7} можно увидеть, что вероятность выжить после первого периода равна 0.91, что означает, что примерно 9\% индивидов нашли работу в течение первых двух недель.

Кривые выживания можно также изобразить в зависимости от значений переменной UI, то есть, по отдельности для тех, кто получал пособие по безработице, и тех, кто его не получал (см. рисунок \ref{fig:17.4}). Как и следовало ожидать, вероятность оставаться безработным по истечении соответствующего периода времени выше для индивидов, получавших страховые выплаты.

Оценка кумулятивного риска Нельсон-Аалена, изображенная на рисунке \ref{fig:17.5}, выглядит практически линейной, что говорит о слабой изменчивости коэффициента риска. Если бы он менялся значительными скачками, то кумулятивный риск имел бы нелинейную форму.

    \begin{table}[!htbp]\caption{\textit{Длительность безработицы: Функция выживания Каплан-Мейера и функция кумулятивного риска Нельсон-Аалена}}\label{tab:17.7}
    \begin{center}
\begin{tabular}{lcc}
\hline \hline
\textbf{Период}&\textbf{Функция выживания}&\textbf{Кумулятивный риск}\\
\hline
1   &0.9121 &0.0879\\
2   &0.8541 &0.1514\\
3   &0.8103 &0.2027\\
4   &0.7864 &0.2322\\
5   &0.7376 &0.2943\\
\vdots&\vdots&\vdots\\
12  &0.5974 &0.5005\\
13  &0.5680 &0.5496\\
14  &0.5270 &0.6219\\
\vdots&\vdots&\vdots\\
26  &0.3651 &0.9809\\
27  &0.3098 &1.1325\\
28  &0.3098 &1.1325\\
\hline \hline
\end{tabular}
    \end{center}
    \end{table}

Функции кумулятивного риска в зависимости от переменной UI представлены на рисунке \ref{fig:17.6}. Как и предполагалось, риск растет быстрее для тех, кто не получал страховых выплат.

Мы рассмотрим четыре варианта параметрических моделей регрессии с объясняющими переменными UI, RR, DR, LOGWAGE и RRUI, DRUI, где $RRUI = RR \times UI$, а $DRUI = DR \times UI$. В качестве спецификации выберем экспоненциальную модель, модели Вейбулла, Гомперца и PH Кокса, риск для которых выглядит следующим образом
        $$\lambda(t|\mathbf{x})=\lambda_0(t,\alpha)\phi(\mathbf{x},\bm{\beta})=\lambda_0(t,\alpha)\exp(\mathbf{x}'\bm{\beta}),$$
    \begin{figure}[ht!]\caption{Длительность безработицы: Оценки функции выживания в зависимости от значений переменной UI. Данные те же, что и на рисунке \ref{fig:17.3}.}\label{fig:17.4}
    \centering
%    \includegraphics[scale=0.7]{fig.png}
    \end{figure}
    % % % % %
    \begin{figure}[ht!]\caption{Длительность безработицы: оценка кумулятивного риска Нельсон-Аалена. Данные те же, что и на рисунке \ref{fig:17.3}.}\label{fig:17.5}
    \centering
%    \includegraphics[scale=0.7]{fig.png}
    \end{figure}
где $\lambda_0(t,\alpha)=\textrm{constant}=\exp(a)$ для некоторой константы $a$ в экспоненциальной модели, $\lambda_0(t,\alpha)=\exp(a)\alpha t^{\alpha-1}$ в модели Вейбулла (риски монотонны), $\lambda_0(t,\alpha)=\exp(a)\exp(\gamma t)$ в модели Гомперца, и предположения о форме базового риска $\lambda_0$ отсутствуют в модели PH Кокса. Заметим, что такая формулировка подразумевает модель пропорциональных рисков, которую с тем же успехом можно переформулировать как параметрическую модель, или же модель AFT. В такой форме функция правдоподобия позволяет найти оценки параметров $(\alpha,\bm{\beta})$, которые вместе с соответствующими $t-$статистиками представлены в таблице \ref{tab:17.8}. Здесь же указан логарифм правдоподобия с минусом, где для модели Кокса существует логарифм лишь частичного правдоподобия. Можно увидеть, что качество экспоненциальной модели и модели Гомперца одинаково. Модель Вейбулла соответствует данным лучше остальных. Из таблицы \ref{tab:17.8} можно также заметить, что с течением времени вероятность того, что наблюдение будет завершено, увеличивается (так как $\alpha=1.129>1$).

\begin{figure}[ht!]\caption{Длительность безработицы: оценка кумулятивного риска в зависимости от значений переменной UI. Данные те же, что и на рисунке \ref{fig:17.3}.}\label{fig:17.6}
\centering
%\includegraphics[scale=0.7]{fig.png}
\end{figure}

    \begin{table}[!htbp]\caption{\textit{Длительность безработицы: оценки параметров для четырех параметрических моделей}}\label{tab:17.8}
    \begin{center}
\begin{tabular}{lcccccccc}
\hline \hline
&\multicolumn{2}{c}{\textbf{Экспоненциальное}}&\multicolumn{2}{c}{\textbf{Вейбулла}}&\multicolumn{2}{c}{\textbf{Гомперца}}&\multicolumn{2}{c}{\textbf{Кокса PH}}\\
\cmidrule(r){2-3}\cmidrule(r){4-5}\cmidrule(r){6-7}\cmidrule(r){8-9}
\textbf{Переменная} &коэфф. &t      &коэфф. &t      &коэфф. &t      &коэфф. &t\\
\hline
RR                  &0.472  &0.79   &0.448  &0.70   &0.472  &0.78   &0.522  &0.91\\
DR                  &-0.576 &-0.75  &-0.427 &-0.53  &-0.563 &-0.74  &-0.753 &-1.04\\
UI                  &-1.425 &-5.71  &-1.496 &-5.67  &-1.428 &-5.69  &-1.317 &-5.55\\
RRUI                &0.966  &0.92   &1.105  &1.57   &0.969  &1.58   &0.882  &1.52\\
DRUI                &-0.199 &-0.20  &-0.299 &-0.28  &-0.211 &-0.21  &-0.095 &-0.10\\
LOGWAGE             &0.35   &3.03   &0.37   &2.99   &0.35   &3.03   &0.34   &3.03\\
CONS                &-4.079 &-4.65  &-4.358 &-4.74  &-4.097 &-4.65  &-      &-\\
$\alpha$            &       &       &1.129  &&&&&\\
$-\ln\textrm{L}$    &\multicolumn{2}{c}{2700.7}&\multicolumn{2}{c}{2687.6}&\multicolumn{2}{c}{2700.6}&\multicolumn{2}{c}{-}\\
\hline \hline
\end{tabular}
    \end{center}
    \end{table}

Для всех моделей значимыми являются лишь коэффициенты при переменных UI и LOGWAGE. Коэффициент при UI отрицателен, что указывает на то, что для тех, кто получал пособие по безработице, длительность безработного состояния завершается медленней. При этом различия в оценках коэффициента UI между моделями незначительны: по сравнению с оценками экспоненциальной модели в абсолютных значениях данный коэффициент выше лишь на 5\% и 0.2\% для моделей Гомперца и Вейбулла, соответственно, и на 8\% ниже для модели Кокса. Аналогично, коэффициент при LOGWAGE положителен и различается между моделями несильно.

Заметим, что в то время, как в эконометрической литературе оценки коэффициентов $(\alpha,\bm{\beta})$ принято записывать в метрике AFT, в биостатистике обычно используют метрику, основанную на PH. Отношение риска равно $\lambda(t|\mathbf{x})/\lambda_0(t,\alpha)=\phi(\mathbf{x},\bm{\beta})=\exp(\mathbf{x}'\bm{\beta})$. То есть, для фиктивной переменной $x$, принимающей значения $0/1$, эффект от ее изменения от 0 до 1 может быть измерен как $\exp(\beta)-1$, что отражает изменения по отношению к базовому риску. Большинство статистических пакетов предоставляют пользователю право выбора, в какой метрике результаты будут отображены. Сравнительные преимущества использования обеих метрик можно найти в работе Кливза, Гоулда и Гутьерреса (2002).

Рассмотрим экспоненциальную модель, результаты оценивания которой представлены в таблице \ref{tab:17.9}, где соответствующий коэффициент является экспонентой, возведенной в степень коэффициента из таблицы \ref{tab:17.8}. Можно увидеть, что отношение риска при переменной UI равно 0.241. Это означает, что для индивида, оформившего страховой случай, «риск найти работу» снижается на 76\% по отношению к базовому риску. Для моделей Вейбулла, Гомперца и Кокса аналогичный риск снижается, соответственно, на 78\%, 76\% и 73\%.

Поскольку в данном примере мы предполагали, что данные цензурированы справа, и игнорировали наличие ненаблюдаемой гетерогенности, все три модели представляют качественно схожие результаты. Однако относительно низкое число объясняющих переменных, которые при этом сильно значимы, указывает на то, что значительная часть дисперсии может быть не объяснена (например, по причине ненаблюдаемой гетерогенности). Этот вопрос и будет рассмотрен в следующей главе.


    \begin{table}[!htbp]\caption{\textit{Длительность безработицы: оценки коэффициентов риска для четырех параметрических моделей}}\label{tab:17.9}
    \begin{center}
\begin{tabular}{lcccccccc}
\hline \hline
&\multicolumn{2}{c}{\textbf{Экспоненциальное}}&\multicolumn{2}{c}{\textbf{Вейбулла}}&\multicolumn{2}{c}{\textbf{Гомперца}}&\multicolumn{2}{c}{\textbf{PH Кокса}}\\
\cmidrule(r){2-3}\cmidrule(r){4-5}\cmidrule(r){6-7}\cmidrule(r){8-9}
\textbf{Переменная} &$\bm{\beta}$ &t      &$\bm{\beta}$ &t      &$\bm{\beta}$ &t      &$\bm{\beta}$ &t\\
\hline
RR                  &1.603  &0.63   &1.565  &0.57   &1.604  &0.62   &1.686  &0.71\\
DR                  &0.562  &-1.02  &0.653  &-0.66  &0.570  &-0.99  &0.471  &-1.55\\
UI                  &0.241  &-12.65 &0.224  &-13.12 &0.240  &-12.65 &0.268  &-11.53\\
RRUI                &2.626  &1.01   &2.760  &0.99   &2.635  &1.01   &2.416  &1.01\\
DRUI                &0.819  &-0.22  &0.742  &-0.33  &0.810  &-0.23  &0.909  &-0.10\\
LOGWAGE             &1.420  &2.56   &1.441  &0.08   &1.42   &2.55   &1.40   &2.57\\
$\alpha$            &       &       &1.129  &&&&&\\
$-\ln\textrm{L}$    &\multicolumn{2}{c}{2700.7}&\multicolumn{2}{c}{2687.6}&\multicolumn{2}{c}{2700.6}&\multicolumn{2}{c}{-}\\
\hline \hline
\end{tabular}
    \end{center}
    \end{table}



\section{Практические соображения}\label{sec:17.12}

\noindent
Параметрический анализ выживаемости включен в большинство статистических пакетов, в том числе в виде дополнительных модулей или расширений, которые предоставляют возможность выполнять различные стандартные операции. Например, можно получить непараметрическую оценку Каплан-Мейера функции выживания, включая расчет доверительных интервалов и многочисленные графические приложения. Расширения обычно сопровождаются методическими рекомендациями, зачастую настолько детальными, что их можно использовать как дополнительные справочники. Например, Эллисон (1995) предлагает практические рекомендации по работе в SAS; Кливз и др. (2002) предлагают руководство по работе в STATA. Зачастую в таких справочниках можно найти не только базовые принципы и алгоритмы выполнения определенных команд, но и советы по анализу специфических типов данных, использования альтернативных спецификаций и интерпретации результатов. Поэтому выполнение практических заданий в таких статистических пакетах, как LIMDEP, STATA, SAS S-plus, является удобным способом изучения анализа времени жизни и закрепления знаний. При этом руководства представляют собой идеальные источники информации по стандартным вопросам.




\section{Библиографические заметки}\label{sec:17.13}

\begin{itemize}

    \item[\textbf{17.3-17.7}]
Классической работой по анализу выживаемости, с акцентом на модель Кокса, является работа Кальбфляйша и Прентиса (1980, 2002). Работы Лоулесса (1982), Кокса и Оакса (1984) также представляют собой полезные источники информации; помимо этого, существует значительное количество текстов, посвященных анализу выживаемости. Байесовские методы оценивания можно найти в работах Ибрахима, Чена и Синха (2001). В последнее время все больше исследований основываются на анализе счетных процессов. Данный подход подробно представлен в работе Флеминга и Харрингтона (1991) и Андерсена и др. (1993).

Следует отметить, что перечисленные работы (в особенности две последние) представляют собой довольно перспективные исследования. Ланкастер (1990) предлагает детальное, хотя и техническое, описание анализа выживаемости, уделяя при этом больше внимания общим вопросам, представленным в следующих двух главах. По социальным исследованиям информацию можно найти в работе Эллисона (1984), которая, как и Ланкастер (1990), охватывает гораздо больше, чем базовые модели с единственным переходом. По вопросам практического применения моделей полезно начать с работы Кифера (1988).

    \item[\textbf{17.8}]
Подробное обсуждение метода частичного правдоподобия можно также найти в Ланкастер (1990).

    \item[\textbf{17.10}]
В работах Мейера (1990), Хана и Хаусмана (1990) и Блейка и др. можно найти полезную информацию по дискретным моделям риска с учетом ненаблюдаемой гетерогенности, которая будет рассмотрена в следующей главе.

    \item[\textbf{17.11}]
Экономические приложения представлены в работах Кифера (1988) и Грина (2003). Примеры параметрических моделей времени жизни в приведенной форме можно найти в статьях Ланкастера (1979), Нарендранатана, Никелла и Стерна (1985), Джаггиа (1991c) и Гритца (1993). Хотя анализ выживаемости представлен, по большей мере, в приведенной форме, в последнее время интерес сместился в сторону более сложных структурных моделей (см., например, Ван ден Берг (1990) и Феролл (1997)). По вопросам структурного анализа можно обращаться к работам Ланкастера (1990) и Ван ден Берга (2001). Ван ден Берг также представляет довольно интересное обоснование модели РH с точки зрения экономической теории. В качестве времени жизни допускается использование различных концепций. Например, Тунали и Притчетт (1997) предлагают три варианта: календарное время, возраст и длительности.
\end{itemize}


\section{Упражнения}\label{sec:17.ex}
\begin{itemize}

    \item[\textbf{17--1}]
(Сапра, 1998) Покажите, что модель времени жизни с плотностью распределения Парето первого рода $f(t) = \alpha k^\alpha /t^{\alpha + 1}$, $\alpha > 0$, $t \ge k \ge 0$ является моделью ускоренной жизни, но не является моделью пропорциональных рисков. [Подсказка: покажите, что $t$ может быть представлено в виде линейной регрессии $k = \exp(\mathbf{x}'\bm{\beta})$ с аддитивными гомоскедастичными ошибками.]

    \item[\textbf{17--2}] (Ланкастер, 1979) Для каждого из пунктов выпишите соответствующие выражения функции правдоподобия в терминах функции плотности $f(t|\mathbf{x},\bm{\theta})$ и функции выживания $S(t|\mathbf{x},\bm{\theta})$.
        \item[\textbf{(a)}] Дана выборка независимых завершенных длительностей $t_i$, $i = 1,  \ldots , N$.
        \item[\textbf{(b)}] Выборка получена на основе опроса безработных индивидов, проводившегося в течение $h$ периодов. Между началом и окончанием опроса часть респондентов нашли работу, а часть --- нет. Для тех, кто нашел работу, известен конкретный момент наступления этого события.
        \item[\textbf{(c)}] Структура выборки аналогична пункту (b), за исключением того, что момент наступления события неизвестен.

    \item[\textbf{17--3}]
        \item[\textbf{(a)}] Используя 50\% случайной выборки по данным МакКолл, рассчитайте непараметрическую оценку Каплан-Мейера функции выживания и оценку интегральной функции риска по отдельности для каждого типа цензурирования. Отличается ли функция выживания для наблюдений, завершенных в связи с переходом на полставки и на полную?
        \item[\textbf{(b)}] Игнорируя типы цензурирования, оцените следующие модели риска: (i) экспоненциальная, (ii) Вейбулла, (iii) лог-логистическая, (iv) PH Кокса. Используйте тот же набор регрессоров, который был представлен в этой главе.
        \item[\textbf{(c)}] Сравните (i)--(iii) модели и обсудите, какая из них лучше всего отражает структуру данных. Какие можно сделать выводы относительно формы функции риска для каждой из этих моделей?
\end{itemize}



\chapter{Модели смеси и ненаблюдаемая гетерогенность}

% Тонкости перевода:
%
% 1. heterogeneity term
% я переводил словосочетание как "параметр гетерогенности", что может слегка запутать читателя, поскольку обычно "параметр" относится к некоторому распределению. Можно поменять на "переменную", или "элемент" гетерогенности? но как-то не звучит

% составляющая гетерогенности

\section{Введение}\label{sec:18.1}

\noindent
По вопросам ненаблюдаемой гетерогенности существует значительный объем статистической и эконометрической литературы. Ненаблюдаемая гетерогенность включает в себя те отличия между объектами, которые не могут быть измерены с помощью включенных регрессоров. Как правило, и наблюдаемые, и ненаблюдаемые отличия влияют на продолжительность пребывания в определенном состоянии. Если есть ненаблюдаемая гетерогенность, то коэффициенты риска для объектов могут различаться даже при одинаковых значениях объясняющих переменных. Если же эту гетерогенность проигнорировать, то её влияние можно спутать с составляющей базового риска.

Идею можно понять на следующем примере. Пусть агрегированная функция риска для наблюдений по безработице является убывающей функцией от продолжительности поиска работы. Если бы все объекты были одинаковыми, это означало бы негативную зависимость от длительности, то есть, более низкую вероятность найти работу для индивидов, которые дольше находятся в поисках. Предположим, однако, что объекты неоднородны и делятся, в частности, на два типа: быстрые F (fast) с постоянным коэффициентом риска, равным 0.4, и медленные S (slow) с коэффициентом риска 0.1. Совокупность объектов на 50\% состоит из 1-го типа и на 50\% из 2-го. Из 100 наблюдений типа F 40 переходов происходят в первом периоде, 24 перехода во втором и 14.4 --- в третьем. Для типа S мы наблюдаем, соответственно, 10, 9 и 8.1 переходов в первом, втором и третьем периодах. Тогда общие доли переходов будут равны $(40 + 10) / 200 = 0.25$, $(24 + 9) / 150 = 0.22$ и $(14.4 + 8.1) / 117 = 0.192$. Следовательно, убывающий коэффициент риска является следствием агрегирования двух неоднородных групп объектов с различными, но постоянными коэффициентами риска. Таким образом, правильность выводов зависит от наличия и учета ненаблюдаемой гетерогенности.

В моделях линейной регрессии ненаблюдаемая гетерогенность не считается серьезной проблемой, если она некореллирована с набором объясняющих факторов, поскольку в таком случае условное математическое ожидание неизменно, оценки несмещены, а ненаблюдаемые различия заключены в остатках модели. В моделях времени жизни последствия неучета таких различий однако могут быть более серьезными. Даже если предположить, что ненаблюдаемая гетерогенность некореллирована с регрессорами и не оказывает воздействия на ожидаемую длительность, математическое ожидание условного риска будет смещено. Заметим, что именно коэффициент риска представляет интерес для интерпретации при наличии цензурирования. Например, прежде чем определять направления политики, мы, возможно, захотим понять, как условная вероятность найти работу зависит от длительности поисков.

Ненаблюдаемая гетерогенность лежит в основе решения множества эмпирических загадок и несоответствий. Несмотря на то, что эта проблема представлена в контексте анализа выживаемости, рассматриваемые методы применяют и в других областях эконометрики, поскольку все эконометрические модели так или иначе содержат пропущенные ненаблюдаемые индивидуальные эффекты. Примерами моделей, учитывающих такие эффекты, в других областях являются логит-модели со случайными параметрами (см. раздел 15.7)% \ref{sec:15.7} # UNCOMMENT AFTER 15 CH
, модели с самоотбором выборки (см. раздел 16.4)% \ref{sec:16.4} # UNCOMMENT AFTER 16 CH
, модели конечной смеси для счетных данных (см. раздел 20.4)% \ref{sec:20.4} # UNCOMMENT AFTER 20 CH
, а также модели панельных данных с фиксированными и случайными индивидуальными эффектами (см. разделы 21-23)% \ref{ch:21}-\ref{ch:23}
. % Термины должны совпадать с терминами из глав 15, 16, 20, 21-23 !!!
Все эти модели относятся к проблеме ненаблюдаемой гетерогенности. В биостатистике также используется термин \textbf{уязвимость} \textit{(frailty)}, и соответствующие модели называются моделями с уязвимостью \textit{(frailty models)}. В актуарной математике с помощью (мультипликативной) ненаблюдаемой гетерогенности измеряют пропорциональное увеличение или уменьшение коэффициента риска (интенсивности смертности, \textit{``force of mortality''}) для определенного индивида по отношению к среднему индивиду. Допускается, что индивидуальная гетерогенность может зависеть от времени, но при анализе данных пространственного типа удобнее предполагать ее независимость.

Важно понимать последствия такой неизбежной недоспецификации. В моделях линейной множественной регрессии пропуск значимых переменных может приводить к смещению оценок учтенных переменных. В моделях времени жизни, нелинейных в отличие от предыдущего случая, анализ недоспецификации гораздо сложнее. В частности, существует целый класс моделей, называемый \textbf{моделями смеси} и посвященный проблеме ненаблюдаемой гетерогенности. Суть данной главы заключается в том, чтобы представить методы, позволяющие строить и анализировать модели смеси и последствия неучета гетерогенности.

Тот факт, что гетерогенность и истинную зависимость (от состояния) (\textit{true state dependence}, предшествующий опыт) необходимо различать, берет начало с обсуждения истинного и кажущегося \textbf{заражения} \textit{(contagion)}.\footnote{Убывающая функция риска как следствие агрегирования неоднородных групп относится к гетерогенности, а постоянные коэффициенты риска в этих группах являются истинной зависимостью.} В частности, Нейман был одним из первых, кто заметил, что на основе панельных данных возможно обнаружить эти различия эмпирическим путем. Если же доступны данные лишь пространственного типа, то необходимо формулировать предпосылки о распределении параметров. В последнее время в эмпирических исследованиях все чаще стараются уходить от параметрической спецификации и тестировать корректность таких предпосылок.

В первой части главы (разделы \ref{sec:18.2}--\ref{sec:18.4}) мы рассмотрим модели смеси, основанные на предположении о непрерывном распределении гетерогенности. Модели с дискретной гетерогенностью будут представлены в разделе \ref{sec:18.5}. В следующем разделе \ref{sec:18.6} мы определим взаимосвязь между различными понятиями длительностей, свойственных для данных типа запас и типа поток. Тесты на неправильную спецификацию и пропуск гетерогенности описаны в разделе \ref{sec:18.7}. Наконец, эмпирический пример в разделе \ref{sec:18.8} является иллюстрацией к идеям, представленным в этой главе.




\section{Ненаблюдаемая гетерогенность и дисперсия}
\label{sec:18.2}

\noindent
В данном разделе мы рассмотрим ненаблюдаемую гетерогенность в экспоненциальной и вейбулловской моделях. Спецификация в форме произведения (то есть, мультипликативно) позволяет исключить ненаблюдаемую гетерогенность при интегрировании, не изменяя условное математическое ожидание, но завышая дисперсию и, что наиболее важно, воздействуя на условную функцию риска. Здесь мы также представим известную модель Вейбулла с гамма-распределенной гетерогенностью.


\subsection{Смешанные модели}\label{sec:18.2.1}

\noindent
Рассмотрим наиболее простой случай, экспоненциальную модель времени жизни. При отсутствии гетерогенности распределение завершенных длительностей $t_i$ задается как условное относительно слабо экзогенных регрессоров $\x_i$. Это эквивалентно отсутствию случайной компоненты в условном математическом ожидании: $\E[T|\x]=\exp{(\xb)}$. В моделях смеси мы задаем распределение $(t_i|\x_i,\nu_i)$, где $\nu_i$ обозначает составляющую ненаблюдаемой гетерогенности для наблюдения $i$. Другими словами, мы допускаем, что индивиды могут иметь случайные различия, неучтенные объясняющими факторами. Маргинальное распределение для $t_i$ получается с помощью усреднения по отношению к $\nu_i$.

Необходимо определить конкретную функциональную зависимость $t_i$ от $(\x_i,\nu_i)$. Чаще всего предполагают, что регрессоры представлены в виде экспоненты с \textit{мультипликативной} ошибкой. Например, модель пропорциональных рисков (PH, proportional hazard), обозначенная уравнениями \ref{eq:17.25} 
и \ref{eq:17.26} 
в разделе \ref{sec:17.8} 
с учетом ненаблюдаемой гетерогенности $\nu$ будет выглядеть следующим образом
    $$\la(t|\x,\nu)=\la_0(t)\exp(\xb)\nu, \nu>0.$$
Следовательно, можно найти интегральный базовый риск, который будет равен
    \begin{align}
    \label{eq:18.1}
    \la_0(t)                        &=\la(t|\x,\nu)\exp(-\xb)\nu^{-1},\\
    \int\la_0(u)du                  &=\exp(-\xb)\nu^{-1}\int\la(u|\x,\nu)du, \notag \\
    \ln\left[\int\la_0(u)du\right]  &=-\xb-\ln\nu+\e, \notag
    \end{align}
где предполагается, что $\e=\ln\int\la(x|\x,\nu)du$, и $\nu$ независима от регрессоров и моментов цензурирования. Часто используют нормализацию $\E[\nu]=1$ (по соображениям идентифицируемости). Если $\nu>1$, то коэффициент риска выше среднего, если же $\nu<1$, то ниже. Предпосылка о независимости является строгой, хотя и необязательно реалистична. Предпосылка о мультипликативном характере связи также сделана специально, поскольку она удобна для математических вычислений и гарантирует неотрицательность $t_i$. В рамках стандартного подхода сперва делаются предположения о распределении $\nu_i$, а затем выводится частное распределение для $t_i$.

Спецификация гетерогенности в мультипликативной форме имеет важное следствие. 
% Кэмерон, мы искали второе --- не нашли :)
Очевидно, что дисперсия распределения смеси (условие относительно наблюдаемых переменных) превышает дисперсию распределения всей совокупности (условие относительно наблюдаемых переменных и составляющей гетерогенности). Следовательно, дисперсия будет завышена. Рассмотрим экспоненциальный случай. Произведем замену $\mu_i=\exp(\xib)$ на
    \begin{align}
    \label{eq:18.2}
    \mu_i^*&=\E[t_i|\x_i,\nu_i]\\
            &=\exp(\xib)\nu_i \notag \\
            &=\exp(\xib)\exp(\e_i) \notag \\
            &=\exp(\beta_0+\e_i+\x_{1i}'\be_1), \notag
    \end{align}
где в третьей строке составляющая ненаблюдаемой гетерогенности, $\nu_i$, записана как $\exp(\e_i)$, а в последней строке $\xib$ разделен на свободный член и коэффициент наклона. Результат в последней строке можно интерпретировать как условное среднее со случайным свободным членом $(\beta_0+\e_i)$. Обычно предполагают, что $\nu_i$ независимо одинаково распределены ($iid$) и не зависят от $\x_i$; также иногда определяют конкретную форму распределения.

Предположим, что $\nu_i \sim \mathrm{iid}$ с $\E[\nu_i]=1$ и $\V[\nu_i]=\sigma_{\nu}^{2}$, где условие $\E[\nu_i]=1$ позволяет идентифицировать свободный член. Моменты экспоненциального распределения $t_i$ могут быть записаны как $\E[t_i|\x_i,\nu_i]=\mu_i\nu_i$. Используя теорему о декомпозиции дисперсии из раздела A.8, 
получим
    \begin{align}
    \label{eq:18.3}
    \V[t_i|\x_i]    &=\V_{\nu}[\E_{t|\nu,\x}(t_i|\nu_i,\x_i)] + \E_{\nu}[\V_{t|\nu,\x}(t_i|\nu_i,\x_i)] \\
                    &=\mu_i^{2}\V(\nu_i)+\mu_i^{2}(\V(\nu_i)+1) \notag\\
                    &=\mu_i^{2}[1+2\sigma_{\nu}^{2}] \notag\\
                    &>\mu_i^{2}. \notag
    \end{align}
То есть, ненаблюдаемая гетерогенность завышает безусловную дисперсию.


\subsection{Выбор распределения неоднородности}\label{sec:18.2.2}

\noindent
Чтобы изучить влияние ненаблюдаемой гетерогенности на распределение $t$, необходимо вывести частное распределение $t_i$ за счет исключения составляющей $\nu$ из $S(t|\x,\nu)$ с помощью интегрирования, где параметрическое распределение $\nu$ обычно задано. Но чем необходимо руководствоваться при выборе этого распределения?

Во-первых, необходимо принимать во внимание, что $\nu_i>0$. Следовательно, можно выбрать такое распределение, которое порождает только положительные значения случайной величины. Примерами являются гамма, обратное гауссовское и лог-нормальное распределения.

\textbf{Функция плотности гамма распределения} равна
    \begin{align}
        \label{eq:18.4}
        g(\nu;\de,k) = \frac{\de^k\nu^{k-1}\exp(-\de\nu)}{\Ga(k)}, \nu>0,
    \end{align}
c математическое ожиданием $\E[\nu]=k/\de$ и дисперсией $\V[\nu]=k/\de^2$. Нормализуя $\E[\nu]=1$, получим $k=\de$ и $\V[\nu]=1/\de$. Данное распределение удобно для математических вычислений и встроено в большинство популярных статистических пакетов для анализа времени жизни.

\textbf{Функция плотности обратного гауссовского распределения} равна
    \begin{align}
        \label{eq:18.5}
        g(\nu;\de,\ttt) = \de\pi^{1/2}\exp(2\de\ttt^{1/2})\nu^{-3/2}\exp(-\ttt\nu-\de^{2}/\nu), \nu>0,
    \end{align}
c математическое ожиданием $\E[\nu] = \de\ttt^{1/2}$ и дисперсией $\V[\nu]=\de\ttt^{-3/2}/2$. Нормализуя $\E[\nu]=1$, получим $\ttt=\de^2$ и $\V[\nu]=1/2\ttt$. По сравнению с гамма, обратное гауссовское распределение имеет более тяжелый хвост.

Заметим, что нет гарантии, что мы сможем аналитически найти частное распределение $t$. Например, некоторые смеси, такие как сочетание Вейбулла и гамма распределений, позволяют найти решения в аналитическом виде, в то время как другие --- нет. Наличие аналитического решения удобно лишь для математических вычислений и не является необходимостью. % не представляет интереса само по себе.
К сожалению, экономическая теория редко предсказывает данный аспект моделирования.

Во-вторых, при выборе распределения важно учитывать его универсальность и гибкость. Так, гамма обладает рядом привлекательных свойств, которые делают его довольно гибким по отношению к данным. Однако для распределений с более тяжелым правым хвостом, вероятно, лучше подойдет обратное гауссовское. Оба распределения являются однопараметрическими (после нормализации). Более гибкий двухпараметрический класс распределений представлен в работе Хугаард (1986), для которого гамма и обратное гауссовкое являются частными случаями. Дискретное (непараметрическое) представление, % ALT: дискретный случай
которое будет рассмотрено позже, также допускает % ALT: позволяет, допускает
значительную гибкость при оценивании.


\subsection{Смесь распределений Вейбулла--гамма}\label{sec:18.2.3}

\noindent
В данном разделе мы рассмотрим известную \textbf{смесь распределений Вейбулла и гамма}, для которой смесь экспоненциального и гамма является частным случаем. В силу своей универсальности (а именно, потому, что позволяет моделировать и убывающие, и возрастающие риски) она является наиболее важной \textbf{моделью смешанных пропорциональных рисков} (MPH).

В модели Вейбулла условная функция выживания относительно мультипликативного $\nu$ равна
    \begin{align}
        \label{eq:18.6}
        S(t|\nu) = \exp(-\mu t^\al\nu), \la>0, \al>0,
    \end{align}
где ранее в главе \ref{ch:17} 
на месте $\mu$ находилась $\al$.

Безусловную функцию выживания можно получить как среднюю функцию, взвешенную по плотности распределения неоднородной совокупности $\nu$, $g(\nu)$, или
    \begin{align}
        \label{eq:18.7}
        S(t) = \E_\nu[S(t|\nu)] = \int S(t|\nu)g(\nu)d\nu.
    \end{align}
Тип смеси зависит от выбора распределения $g(\nu)$, которое, при соответствующих изменениях в формуле, может быть как непрерывным, так и дискретным. Интеграл в \ref{eq:18.7} необязательно должен иметь аналитическое решение. Например, для лог-нормального распределения $g(\nu)$ решения нет, а для гамма --- есть. Для удобства математических вычислений мы будем работать с гамма распределением.

При гамма-распределенной гетерогенности безусловная функция выживания записывается как
    \begin{align}
        \label{eq:18.8}
        S(t) &= \int^{\infty}_{0}\exp(-\mu t^{\al}\nu) \frac{\de^k\nu^{k-1}\exp(\de\nu)}{\Ga(k)} d\nu \\
             &= \frac{\de^k}{\Ga(k)} \int^{\infty}_{0}\nu^{k-1}\exp(-\nu(\mu t^{\al} + \de)) d\nu. \notag
    \end{align}
Чтобы получить распределение смеси, необходимо взять интеграл. Пусть $\mu t^{\al} + \de = \beta$, тогда
    $$S(t) = \frac{\de^k}{\Ga(k)} \int^{\infty}_{0} \frac{(\nu\beta)^{k-1}}{\beta^{k-1}} \exp(-\nu\beta) d\nu.$$
Произведя замену $y = \nu\beta$ так, что $d\nu = \beta^{-1}dy$, получим
    \begin{align}
        \label{eq:18.9}
        S(t) &= \frac{\de^k}{\Ga(k)\beta^k} \int^{\infty}_{0} y^{k-1} \exp(-y) dy \notag\\
             &= \frac{\de^k}{\Ga(k)} \frac{\Ga(k)}{(\mu t^\al + \de)^k} \notag \\
             &= \de^k(\mu t^\al + \de)^{-k} \notag \\
             &= [1 + (\mu t^\al / \de)]^{-k},
    \end{align}
где во второй строке мы использовали определение функции распределения $\Ga(k)$ и произвели обратную замену для $\beta$.

Безусловную плотность распределения длительностей можно получить, взяв производную по $t$ и домножив на $-1$, то есть
    \begin{align}
        \label{eq:18.10}
        f(t) = \frac{k}{\de} \mu \al t^{\al-1} [1 + (\mu t^\al /\de)]^{-(k+1)}.
    \end{align}
Следовательно, безусловная функция риска $\la(t) = f(t) / S(t)$ равна
    \begin{align}
        \label{eq:18.11}
        \la(t) = \frac{k}{\de} \mu \al t^{\al-1} [1 + (\mu t^\al /\de)]^{-1}.
    \end{align}

Эти выражения, представленные в общем виде, можно упростить, положив, что $k = \de$, что равносильно нормализации $\E[\nu] = 1$. Тогда выражения для \textit{смеси Вейбулла--гамма} будут выглядеть следующим образом:
    \begin{align}
        \label{eq:18.12}
        S(t) = [1 + (\mu t^\al / \de)]^{-\de},
    \end{align}

    \begin{align}
        \label{eq:18.13}
        f(t) = -\frac{\pa S(t)}{\pa t} = \mu \al t^{\al-1}[1 + (\mu t^{\al} / \de)]^{-(\de + 1)},
    \end{align}

    \begin{align}
        \label{eq:18.14}
        \la(t) = -\frac{\pa \ln S(t)}{\pa t} = \mu \al t^{\al-1}[1 + (\mu t^{\al} / \de)]^{-1},
    \end{align}
где коэффициент риска стремится к коэффициенту риска в модели Вейбулла при дисперсии $1/\de$, стремящейся к нулю.

Хотя модель Вейбулла позволяет моделировать как возрастающие, так и убывающие риски, она также требует предпосылки об условно монотонных рисках на индивидуальном уровне. Тем не менее, из-за своих удобных свойств она получила широкое распространение в эконометрической литературе; см. Ланкастер (1979), Нарендранатан, Никелл и Стерн (1985).

\textbf{Смесь экспоненциального и гамма распределений} получается, если положить $\al = 1$. В этом случае соответствующие выражения равны $S(t) = [1 + (\mu t / \de)]^{-\de}$, $f(t) = \mu [1 + (\mu t / \de)]^{-(\de + 1)}$, и $\la(t) = \mu[1 + (\mu t / \de)]^{-1}$. Данная смесь также называется \textbf{распределением Парето} второго рода и имеет более тяжелые хвосты по сравнению с обычным экспоненциальным распределение из-за различия в дисперсии, $1/\de$. Заметим, что $r$-ый момент существует только для $\de > r$.


\subsection{Интерпретация функции риска в моделях смеси}\label{sec:18.2.4}

\noindent
В экономических приложениях ключевой интерес представляет наличие положительной или отрицательной зависимости от длительности, например, увеличение или снижение вероятности найти работу с продлением периода поисков. Вероятность найти работу может расти из-за роста альтернативных издержек быть безработным. Вероятность найти работу может падать, если рассматривать работника как <<плохой товар>>.
% по смыслу
% (e.g., owing to worker is reservation wage falling)
% (e.g., owing to the worker being viewed as damaged goods)
В случае, если объекты независимо и одинаково распределены ($iid$), истинную зависимость можно установить с помощью непараметрических методов оценивания. Если же объекты распределены неодинаково или зависимы ($non$-$iid$), убывающая функция риска может получиться в результате агрегирования как убывающих, так и постоянных индивидуальных рисков. Понять, какой из случаев на самом деле имеет место, бывает довольно трудно.

Обратимся к интерпретации функции риска в модели смеси экспоненциального и гамма распределений при наличии ненаблюдаемой гетерогенности. Заметим, что даже если индивидуальный риск (то есть, риск при условии $\nu$) постоянен при $\mu$, средний, или агрегированный, риск $\la(t)$ убывает по $t$. То есть, на основе значений агрегированного риска нельзя судить о характере взаимосвязи на индивидуальном уровне. Негативная зависимость наблюдается именно из-за агрегирования объектов, обладающих случайными различиями в функциях риска. Подобная ошибочная интерпретация может возникнуть и в смеси Вейбулла и гамма распределений, где наклон индивидуальной функции риска по-прежнему зависит от $\al$, но агрегированная функция риска подвержена воздействию гетерогенности. Поэтому неучет ненаблюдаемой гетерогенности может приводить к недооценке коэффициентов наклона функции риска. Результат является довольно общим (см. Ланкастер, 1990); подробное описание представлено ранее в работе Саланта (1977).

Результат основан на утверждении (см. например, Ланкастер, 1979; Хекман и Сингер, 1984a), что при наличии ненаблюдаемой гетерогенности оценки могут быть существенно смещены, что указывает на необходимость проведения тестов на наличие ненаблюдаемых эффектов. Рассмотрим эту идею в контексте модели смеси Вейбулла, где $S(t) = \int \exp(-\mu t^{\al}\nu) g(\nu) d\nu$. Агрегированная функция риска равна
    \begin{align}
        \la(t) &= -\int\frac{\pa \ln S(t|\nu)}{\pa t} g(\nu) d\nu \notag\\
               &= \al\mu t^{\al-1} \int \frac{\nu\exp(-\mu t^{\al}\nu)}{S(t|\nu)} g(\nu) d\nu \notag \\
               &= \al\mu t^{\al-1} \E[\nu|T\ge t]. \notag
    \end{align}
Поскольку математическое ожидание $\E[\nu|T\ge t]$ рассчитывается как среднее по $\nu$ среди доживших до момента $t$, со временем оно должно убывать, так как индивиды с высокими значениями $\nu$ покидают состояние быстрее, чем индивиды с низкими значениями $\nu$. Как следствие, наклон агрегированной функции риска меняется. Такой эффект можно воспринимать как форму \textbf{смещения отбора} \textit{(selectivity bias)}, % термин должен совпадать с термином из главы 16.5!!!
см. раздел 16.5. % \ref{sec:16.5} # UNCOMMENT AFTER 16 CH
Формально, среднее по $\nu$ при условии пройденного времени может быть записано как
    $$\E[\nu|T\ge t] = \int\frac{\nu\exp(-\mu t^\al\nu)}{S(t|\nu)} g(\nu) d\nu.$$
Тогда для модели смеси Вейбулла
    \begin{align}
    \label{eq:18.15}
        \frac{\pa\E[\nu|T\ge t]}{\pa t} &= -\al\mu t^{\al-1} \left[ \int\frac{\nu^2\exp(-\mu t^\al\nu)}{S(t|\nu)} g(\nu) d\nu \right] \notag\\
                                        &\hspace{0.43cm} +\al\mu t^{\al-1} \left[ \int\frac{\nu\exp(-\mu t^\al\nu)}{S(t|\nu)} g(\nu) d\nu \right]^2 \notag \\
                                        &= -\al\mu t^{\al-1} \{\E[\nu^2|T\ge t] - (\E[\nu|T\ge t])^2 \} \notag\\
                                        &= -\al\mu t^{\al-1} \V[\nu|T\ge t] \\
                                        &< 0. \notag
    \end{align}
В результате, неучет ненаблюдаемой гетерогенности приводит к тому, что оценка коэффициента риска падает быстрее (или же медленнее растет), чем истинный коэффициент.

Интересно также сопоставить пропорциональный эффект изменения регрессоров на коэффициент риска в модели при наличии и при отсутствии гетерогенности. При отсутствии, логарифм условной функции риска равен
    $$\ln\la(t|\mu) = \ln(\mu t^{\al-1}) + \ln\al,$$
и пропорциональный эффект изменения $x_j$ на $\mu$ составляет
    $$\frac{\pa\ln\la(t|\mu)}{\pa x_j} = \be_j,$$
что является одним из свойств модели пропорциональных рисков.

При наличии гетерогенности
    \begin{align}
        \ln\la(t|\mu) &= \ln(\mu t^{\al-1}) + \ln\al + \ln\E[\nu|T\ge t] \notag \\
                      &= \ln\al + \ln\mu + (\al-1)\ln t + \ln\E[\nu|T\ge t], \notag
    \end{align}
откуда, зная, что $\ln\mu = \xb$ и $\pa\E[\nu|T\ge t]/\pa x_j = -\mu t^{\al}\V[\nu|T\ge t]\beta_j$, для модели смеси Вейбулла получим
    \begin{align}
        \label{eq:18.16}
        \frac{\pa\ln\la(t|\mu,\nu)}{\pa x_j} &= \beta_j \left[ 1 - \frac{\mu t^\al \V[\nu|T\ge t]}{\E[\nu|T\ge t]} \right] \\
                                             &< \beta_j. \notag
    \end{align}
То есть, при данной гетерогенности пропорциональный эффект изменения $x_j$ меньше и зависит от $t$, и кроме того, больше не относится к типу пропорциональных рисков. Следовательно, полученные оценки могут быть неверными и вести к неправильным выводам, даже если составляющая  гетерогенности некоррелирована с наблюдаемыми регрессорами.

Аналогичные результаты для более общих, чем Вейбулла, моделей представлены в Ланкастер и Никелл (1980).




\section{Идентификация в моделях смеси}\label{sec:18.3}

\noindent
Моделям смеси свойственна общая \textbf{проблема идентификации}, которая относится к возможности (или невозможности) последовательного разложения индивидуальных вкладов на среднюю вероятность выживания базового риска, ненаблюдаемую гетерогенность и объясняющие переменные для наблюдаемого набора данных $(t,\x)$ по отношению к единственному состоянию. Точнее говоря, если модель неидентифицируема, такое разложение невозможно. Как и в аналогичных обсуждениях идентификации прочих моделей, формулировка
основывается на определенных ограничениях. Модели (смешанных) пропорциональных рисков в эконометрической литературе представлены детально. В частности, Хекман и Сингер (1984b) и Элберс и Риддер (1982) доказали идентифицируемость модели смешанных пропорциональных рисков (MPH) при определенных условиях. Эти и более поздние доказательства можно также найти в работе Ван ден Берга (2001).

Рассуждения об идентифицируемости модели смешанных пропорциональных рисков (MPH) следует начать с определения \textbf{средней}, или \textbf{агрегированной, функции выживания}
    \begin{align}
        \label{eq:18.17}
        S(t|\x) &= \E_\nu[S(t|\x,\nu)] \\
                &= \int\exp(-\nu\La_0(t)\phi(\x)) g(\nu)d\nu, \notag
    \end{align}
где предполагается, что риски пропорциональны как в \ref{eq:18.1}, а формулировки для пропорциональных рисков (PH) взяты из раздела \ref{sec:17.8} 
но без предположений о распределении параметров $\La_0, \phi$ или $g$. Здесь $\La_0(t) = \int^{T}_{0}\la_0(s)ds$. Модель называется непараметрически идентифицируемой, если для набора данных функции $\la_0, \phi$ и $g$ единственные; ``непараметрически'' --- ввиду отсутствия каких-либо предположений о функциональной форме.

Разброс наблюдаемых моментов выживания можно объяснить изменениями ковариат $\x$, $\nu$ и базового риска (функции зависимости от длительности). Идентифицируемость означает единственность такой декомпозиции, следовательно, доказательство должно основываться на том, что отдельные компоненты, в принципе, идентифицируемы. Большинство существующих доказательств предполагают использование сложных математических методов для того, чтобы показать, что функцию правдоподобия можно разложить единственным образом. Мелино и Сейоши (1990) предлагают более простое доказательство.

Для того чтобы модель была непараметрически идентифицируема, требуется, чтобы выполнялись следующие условия:
(i) Составляющая гетерогенности $\nu$ не зависит от времени и распределен независимо от набора переменных $\x$.
(ii) $g(\nu)$ невырождена с конечным математическим ожиданием $\E[\nu] < \infty$.
(iii) $\phi(\x) > 0$ для всех $\x$.
(iv) $\la_0(t)$ непрерывна и положительна в интервале $[0, \infty)$
(v) Наблюдаемые объясняющие переменные $\x$ линейно независимы и обладают достаточной изменчивостью.
Различные доказательства используют различные модификации перечисленных условий, но мы не будем вдаваться в подробности.

Тогда как непараметрическая идентифицируемость предполагает использование сложных математических инструментов, проблема актуальна и для параметрических моделей. Предположив определенную параметрическую форму для $\la_0(t|\al)$, $\phi(\x|\be)$ и $g(\nu|\ga)$, можем ли мы утверждать что она единственна? К сожалению, во многих случаях ответ на этот вопрос будет отрицательным. То есть, мы можем оценить определенную модель смеси без каких-либо вычислительных трудностей и получить вроде бы неплохие результаты со значимыми коэффициентами. Но в то же время при других предпосылках мы получим не менее хорошие результаты, но выводы на этот раз будут другими. Таким образом, наблюдаемая функция выживания согласуется с различными предположениями о базовом риске и распределении гетерогенности (Ланкастер, 1990, глава 4). В терминах из раздела 2.2 % \ref{sec:2.2} # UNCOMMENT AFTER 2 CH
это означает, что различные структурные модели при существенно разных выводах могут иметь одинаковую приведенную форму. Это, соответственно, поднимает вопрос о способах применения параметрических методов. Такими способами могут быть выбор гибкой параметрической формы, или же применение полупараметрического подхода к анализу методом частичного правдоподобия. Мы продолжим обсуждение в следующем разделе.




\section{Спецификация распределения неоднородности}\label{sec:18.4}

\noindent
Чувствительность оценок коэффициентов к различным предположениям о распределении гетерогенности изучена в литературе подробно. На основе предыдущих исследований можно выделить две, на первый взгляд, противоположные точки зрения:

\begin{enumerate}
    \item
Выбор параметрической формы ненаблюдаемой гетерогенности осуществляется зачастую произвольным образом, поэтому выводы о поведении функции риска могут быть существенно искажены. Следовательно, предпочтительно применение гибкой параметрической формы или же непараметрической спецификации. См. Хекман и Сингер (1984a).

    \item
Последствия выбора некорректной параметрической формы ненаблюдаемой гетерогенности относительно безобидны, если спецификация функции базового риска верна. Если же форма функции риска неоднозначна, то оценки на основе различных предпосылок о распределении гетерогенности могут приводить к различным оценкам маргинального распределения данных. См. Мантон, Сталлард и Вопель (1986).
\end{enumerate}

Кажущееся противоречие можно разрешить следующим образом. Спецификация функции риска воздействует на первый момент распределения $f(t)$, в то время как спецификация гетерогенности воздействует на второй момент, при условии, что гетерогенность независима от наблюдаемых ковариат. Если функция риска специфицирована верно, то выбор распределения гетерогенности будет оказывать эффект, в основном, на относительную эффективность оценок.




\subsection{Гамма гетерогенность для PH в дискретном времени}\label{sec:18.4.1} % гамма гетерогенность для пропорциональных рисков в дискретном времени

\noindent
На основе предыдущих рассуждений можно ожидать, что определенное распределение гетерогенности должно сочетаться с функцией риска произвольной формы в модели пропорциональных рисков. Хан и Хаусман (1990) и Мейер (1990) предложили объединить гамма-распределенную гетерогенность с моделью пропорциональных рисков в дискретном времени, представленной в разделе \ref{sec:17.10} % UNCOMMENT AFTER 17 CH.
Согласно полученным результатам, оценки слабо чувствительны к альтернативным спецификациям функциональной формы $g(\nu)$, если базовый риск не зависит от параметров.

Для определенности, уравнение \ref{eq:17.43} % UNCOMMENT AFTER 17 CH
с учетом гетерогенности можно перезаписать как
    $$\e_i = \ln \left( \int \la_0 (\tau) d\tau - \xib -\nu_i \right),$$
и подставить в выражение логарифма правдоподобия (17.44). % \ref{eq:17.44} # UNCOMMENT AFTER 17 CH
Составляющая гетерогенности следует исключить с помощью интегрирования. Решение в аналитическом виде для гамма гетерогенности представили Хан и Хаусман, где указали на относительно незначительную чувствительность по отношению к параметрическим распределениям при данной гибкой спецификации риска.


\subsection{Другие модели с гетерогенностью}\label{sec:18.4.2}

%%%%%% ПЕРЕСМОТРЕТЬ %%%%%%%
\noindent
Как уже упоминалось, модели смеси Вейбулла--Гамма удобны для математических вычислений, поскольку имеют аналитическое решение.

Однако если частное распределение имеет более тяжелые хвосты, нежели гамма или лог-нормальное, имеет смысл выбрать распределение из \textbf{семейства устойчивых распределений} Мандельброта. В частности, довольно широкий класс, включающий, в том числе, гамма и обратное гауссовское распределения, представлен в работе Хугаард (1986) (см. также Джаггиа, 1991b). Распределение называется строго устойчивым, если сумма его $p$ независимых реализаций подчиняется исходному распределению, умноженному на $p$.
% ALT: Распределение называется строго устойчивым, если сумма его $p$ независимых реализаций имеет то же распределение, что и исходное, умноженное на $p$.
Краткое изложение свойств можно найти в Хугаард (2000, аппендикс 3.3).

Хотя распределение гетерогенности с большим числом параметров в силу своей универсальности выглядит привлекательнее, его применение может быть связано с трудностями следующего характера. Во-первых, если объем данных относительно небольшой, степеней свободы может быть недостаточно для идентификации или точной оценки параметров. Обычно, трудно понять, что это так, не попытавшись сперва оценить модель.

Во-вторых, трудности могут возникнуть при вычислениях. Если плотность смеси не может быть выражена аналитически, она записывается в форме интеграла. В результате логарифм правдоподобия состоит из элементов, являющихся интегралами, и для максимизации требуется произвести значительный объем вычислений, используя численные методы интегрирования, или интегрирование по методу Монте-Карло (см. главу 12). % \ref{ch:12} # UNCOMMENT IN THE END OF THE BOOK
Примером модели смеси, требующей применения таких методов, является смесь Вейбулла и лог-нормального распределений, где ненаблюдаемая гетерогенность распределена лог-нормально. Оценивание моделей гетерогенности на основе симуляционных методов представлено в работе Gouri\'eroux и Monfort (1991, 1996) и рассмотрено в качестве примера в разделе 12.2. % \ref{sec:12.2} # UNCOMMENT AFTER 12 CH




\section{Дискретная гетерогенность и анализ латентных классов}\label{sec:18.5}

\noindent
До сих пор анализ подразумевал оценивание параметров непрерывного распределения ненаблюдаемой гетерогенности.

Альтернативный подход предполагает, что выборка объектов построена на основе генеральной совокупности, которая состоит из конечного числа \textbf{латентных классов} $q$, таким образом, что каждый каждый элемент в выборке принадлежит к одному из этих классов, или страт. Такая модель известна как модель конечной смеси, \textbf{полупараметрическая модель гетерогенности} (Хекман и Сингер, 1984a) или \textbf{модель латентных классов} (Айткен и Рубин, 1985). Результатом модели является гибкое параметрическое распределение, что является значительным преимуществом при ее оценивании. В моделировании длительностей сторонниками данного подхода можно назвать Хекмана и Сингер (1984a), которые занимались анализом и применением модели.

Несмотря на то, что эти модели представлены в контексте анализа времени жизни, их применение гораздо шире, в связи с чем мы будем использовать более общие обозначения; см. раздел 20.4. %\ref{sec:20.4}


\subsection{Модель конечной смеси}\label{sec:18.5.1}

\noindent
Рассмотрим следующую \textit{модель конечной смеси}. Если выборка является вероятностной смесью двух групп (классов) с плотностями распределения $f_1(t|\mu_1(\x))$ и $f_2(t|\mu_2(\x))$, то $\pi f_1(\cdot) + (1 - \pi)f_2(\cdot)$, где $0\le\pi\le1$, задает двухкомпонентную конечную смесь. То есть, с вероятностью $\pi$ мы наблюдаем объект из группы с плотностью $f_1$ и с вероятностью $1-\pi$ --- из группы с плотностью $f_2$. Следовательно, требуется оценить параметры $(\pi,\mu_1,\mu_2)$, где параметр $\pi$ может выступать как в роли экзогенной константы, так и в роли эндогенной переменной, для оценки которой можно использовать, например, логит-функцию. Тогда $\pi = \exp(\la)/[1+\exp(\la)]$, и далее с помощью наблюдаемых ковариат оценивается $\la$. Таким образом, мы подразумеваем два типа индивидов, наблюдаемых с вероятностями $f_1(\cdot)$ и $f_2(\cdot)$. Возможно, есть некоторая априорная теория, которая предсказывает принадлежность объектов к одной из этих групп, например, если существует некая латентная характеристика, которая разделяет выборку на две части. Иначе говоря, % ALT: интерпретация, альтернативная формулировка идеи
идея заключается в том, что линейная комбинация функций плотности лучше описывает наблюдаемое распределение длительностей $t$.

Обобщение модели до аддитивных смесей с тремя и более компонентами, в принципе, не связано ни с какими проблемами, за исключением возможной неидентифицируемости компонент. Этот случай будет рассмотрен в главе далее. % ALT: момент, обсуждение будет представлено и т.д.
Таким образом, в эмпирических приложениях желательно, чтобы компоненты были обоснованы с интуитивной точки зрения. Наиболее простым образом компоненты можно интерпретировать как ``типы'', но в некоторых ситуациях возможна более информативная интерпретация (см. Линдсей, 1995).

Иначе, модель конечной смеси можно сформулировать, представив гетерогенность совокупности объектов в дискретном виде. Пусть, совокупность состоит из $m$ гомогенных групп, обычно называемых \textbf{компонентами}. Параметрическая модель, например, экспоненциальная или Вейбулла, применяется к каждой из компонент. Предположим, доля $j$-ой компоненты в генеральной совокупности равна $\pi_j$, при этом $\sum\pi_j = 1$.

Формально, задача может быть представлена следующим образом: До сих пор носитель распределения ненаблюдаемой гетерогенности имел бесконечное число точек. Если непрерывное смешиваемое распределение $g(\nu_i)$ может быть аппроксимировано дискретным распределением, $\pi_j(j = 1, \ldots, m)$, с конечным числом точек $m$, то маргинальное распределение (смеси) равно
    \begin{align}
        \label{eq:18.18}
        h(t_i|\x_i,\pi_j,\be) = \sum^{m}_{j=1} f(t_i|\x_i, \nu_j, \be) \pi_j(\nu_j),
    \end{align}
где $\pi_j$ --- вероятность, соответствующая оценке точки определения $\nu_j$. В моделировании длительностей такая полупараметрическая форма ненаблюдаемой гетерогенности была рассмотрена Хекман и Сингер, 1984a. Подобной работой является работа Веделя и др. (1993), где гетерогенность сформулирована в виде латентных классов. Если смешиваемое распределение $\pi_j$ не зависит от параметрических предпосылок, то модель смеси называется \textbf{полупараметрической моделью смеси} для $t$.

Оценивание модели конечной смеси можно проводить, предполагая, что число компонент или известно, или же нет. Если известны доли $\pi_j$, то оценки распределений компонент можно получить методом максимального правдоподобия. Зачастую же доли $\pi_j$, $j=1,\ldots,m$, неизвестны, и требуется оценить не только $\pi_j$, но и параметры компонент. Функция оценки ММП в таком случае называется функцией оценки непараметрическим методом максимального правдоподобия (NPMLE), где непараметрической составляющей является число классов. Заметим, однако, что это строго полупараметрический метод, поскольку он совмещен с параметрической моделью для компонент. Обычно число компонент неизвестно, и поэтому в отношении выводов требуется аккуратность; см. раздел 18.5.4. % \ref{sec:18.5.4}

Применение моделей конечной смеси обосновано тем, что зачастую разумней и проще предполагать, что неоднородность совокупности состоит из небольшого числа латентных классов, а не из континуума ``типов'', как было представлено ранее в разделе \ref{sec:18.2}.




\subsection{Интерпретация в виде латентных классов}\label{sec:18.5.2} % ALT: Формулировка в виде латентных классов?

\noindent
Модель конечной смеси относится к \textbf{латентному анализу классов} (Айткен и Рубин, 1985; Ведел и др., 1993). Обозначим переменную $d_{i} = (d_{i1}, \ldots, d_{im})$, $d_{ij} = \textbf{1}(\sum_jd_{ij} = 1)$, как индикатор (дамми), что наблюдаемая длительность $t_i$ относится к $j$-ой латентной группе, или классу, для $i = 1, \ldots, N$. То есть, каждое наблюдение может принадлежать одной из $m$ латентных групп, классов, или же ``типов''. Далее будем предполагать, что модель идентифицирована.

Согласно модели, $(t_i|d_i,\bmu, \bpi)$ распределены независимо с плотностями распределения
    \begin{align}
        \label{eq:18.19}
        \sum^{m}_{j=1} d_{ij}f(t_i|\mu_j) = \prod^{m}_{j=1}f(t_i|\mu_j)^{d_{ij}},
    \end{align}
где $\mu_j = \mu(\x_j, \be_j)$, $\bmu = (\mu_1, \ldots, \mu_m)$ и $d_i|\bmu,\bpi$ независимо и одинаково распределены  c полиномиальным распределением
    \begin{align}
        \label{eq:18.20}
        \prod^{m}_{j=1}\pi_j^{d_{ij}}, 0<\pi_j<1, \sum^{m}_{j=1}\pi_j=1.
    \end{align}
Из последних двух выражений следует, что
    $$(t_i|\bmu,\bpi)\thicksim\!\!\!\!\!\!\!^{{}^{\textit{iid}}}\hspace{0.1cm}\sum^{m}_{j=1}\pi_j^{d_{j}}f_j(t|\mu_j)^{d_{ij}}.$$
Следовательно, функция правдоподобия равна
    \begin{align}
        \label{eq:18.21}
        \mL(\be,\bpi|\bt) = \prod^{N}_{i=1}\sum^{m}_{j=1}\pi_j^{d_{ij}}f_j(t;\mu_j)^{d_{ij}}.
    \end{align}


\subsection{EM алгоритм}\label{sec:18.5.3}

\noindent
Данную функцию правдоподобия можно максимизировать непосредственным вычислением, % ALT: напрямую
или же с помощью EM алгоритма, где переменные $\bd = (d_1, \ldots, d_n)$ воспринимаются как пропущенные % ALT: недостающие
данные, см раздел 10.3. % \ref{sec:10.3}
Если бы $\bd$ были наблюдаемы, то логарифм правдоподобия был бы равен
    \begin{align}
        \label{eq:18.22}
        \ln L(\bmu,\bpi|\bt,\bd) = \sum^{N}_{i=1}\sum^{m}_{j=1} d_{ij}\ln f_j (\bt_i; \mu_j) + \sum^{N}_{i=1}\sum^{m}_{j=1} d_{ij} \ln\pi_j.
    \end{align}
Если $\pi_j$, $j = 1,\ldots,m$ известны, то апостериорная вероятность, что $t_i$ принадлежит к группе $j$, $j = 1, 2, \ldots, m$, обозначенная как $\zeta_{ij}$, равна
    \begin{align}
        \label{eq:18.23}
        \zeta \equiv \Pr[y_i\in\textrm{population }j] = \frac{\pi_jf_j(y_i|\x_i,\be_j)}{\sum^{m}_{j=1}\pi_jf_j(y_i|\x_i,\be_i)}.
    \end{align}
Среднее по $i$ значение $\zeta_{ij}$ равно вероятности, что случайно выбранная длительность принадлежит к группе $j$, то есть, $\pi_j$:
    $$\E[\zeta_{ij}] = \pi_j.$$

Предположим, что оценка $\hat{\zeta}_{ij}$ для $\E[d_{ij}]$ известна. % логично: математическое ожидание индикатора равно вероятности, т.е. дзете.
Тогда, при условии, что мы ее знаем, получим
    \begin{align}
        \label{eq:18.24}
        \mathrm{EL}(\be_1, \ldots, \be_m,\bpi|\bt,\hat{\mathbf{z}},\x_1, \ldots, \x_m)) = \sum^{N}_{i=1}\sum^{m}_{j=1} \hat{\zeta}_{ij}\ln f_j(t_i, \mu(\x_j,\be_j)) + \sum^{N}_{i=1}\sum^{m}_{j=1} \hat{\zeta}_{ij}\ln\pi_j,
    \end{align}
что соответствует E-шагу EM алгоритма. На M-шаге мы максимизируем EL, решая условия первого порядка
    \begin{align}
        \label{eq:18.25}
        \hat{\pi}_j - N^{-1}\sum^{m}_{j=1} \hat{\zeta}_{ij} = 0, j = 1, \ldots, m,
    \end{align}
    \begin{align}
        \label{eq:18.26}
        \sum^{N}_{i=1}\sum^{m}_{j=1} \hat{\zeta}_{ij} \frac{\pa\ln f_j(t_i|\be_j)}{\pa\be_j} = \0.
    \end{align}
Далее, мы можем получить новые значения $\hat{\zeta}_{ij}$, используя уравнение \ref{eq:18.23}, и повторять шаги E и M до тех пор, пока процесс не сойдется.
Дисперсию можно рассчитать после, с помощью информационной матрицы или робастной формулы. % не одно и то же: что такое робастная формула, можно посмотреть, например, здесь, в разделе 4.2:
% http://www.actuaries.org/ASTIN/Colloquia/Berlin/Dupin_Montfort_Verle.pdf


\subsection{Выбор количества латентных классов}\label{sec:18.5.4}

\noindent
В связи с количеством компонент возникает два важных вопроса. Первый относится непосредственно к выбору числа компонент, $m$. Зачастую, мы не имеем теории, предсказывающей определенное количество групп, и выбор приходится осуществлять исходя из практических соображений. В частности, количество оцениваемых параметров может быть довольно большим, поскольку их размерность равна $m\dim[\be + m -1]$. Чтобы уменьшить число параметров, можно положить, что некоторые коэффициенты из набора $\be$ равны между собой. В качестве другого способа часто предполагают, что константа принимает различные значения, в то время как коэффициенты наклона между группами одинаковы (как в уравнении \ref{eq:18.18}). Однако, если допустить, что изменяются все параметры, логично выбрать малое число компонент $m$. Часто используют $m = 2$, даже если только константа может варьироваться между группами. Разумной стратегией будет начать с $m = 2$ и затем проверить качество модели с помощью диагностических тестов. Если качество плохое, добавляется еще одна компонента. Добавление компонент, которые невозможно однозначно различить, бессмысленно --- если межгрупповые различия незначительны, нет необходимости представлять выборку в виде конечной смеси. Желательно, чтобы компоненты имели интерпретацию. Выбор между моделями с различным числом компонент можно осуществить на основе штрафных критериев правдоподобия (AIC или BIC), см. раздел 8.5.1. % \ref{sec:8.5.1} % Термин должен совпадать!!!
Отношение правдоподобия \textit{(likelihood ratio)} неприменимо, поскольку параметр находится на границе проверяемой гипотезы.
Бейкер и Мелино (2000) провели эксперимент по методу Монте Карло, который указывает на возможные проблемы перепараметризации, приводящие к неверным выводам, в случае когда и базовый риск, и гетерогенность имеют гибкую форму в целях избежания проблемы неправильной спецификации.
Для выбора модели среди моделей с различным количеством латентных классов авторы рекомендуют использовать штрафные критерии правдоподобия с более высоким штрафом за большее число параметров.

Если модель перепараметризована, параметры невозможно идентифицировать. Проблема возникает из-за наличия нескольких максимумов, или же плоской поверхности функции правдоподобия. Как следствие, алгоритм будет показывать % ALT: выдавать, рассчитывать
различные оценки параметров в зависимости от начальных условий.

Заметим, что минимальное значение штрафного критерия не гарантирует качество подгонки выбранной модели, которое может быть проверено только с помощью соответствующих тестов. По сути, систематическая составляющая объясняет выборочную дисперсию достаточно хорошо, если отклонение между истинными и предсказанными длительностями несущественно.

        \begin{center}{Замечания по поводу вычисления}\end{center}
        \noindent
Второй из вопросов относится к выбору вычислительного алгоритма. Несмотря на то, что EM алгоритм помогает разобраться в структуре решения задачи, на практике он работает довольно медленно. Существует множество примеров, когда алгоритм Ньютона-Рафсона решает ту же задачу быстрее. Обзор работы других алгоритмов можно найти в Хоутон (1997). Заметим, что разницы в применении нет, если межгрупповые различия малы, поскольку поверхность правдоподобия будет содержать несколько локальных максимумов. В любом случае, нет гарантии, что максимум единственный.

Все модели конечной смеси неидентифицированы в том смысле, что распределение данных не изменится, если поменять компоненты местами (например, назвать ``компоненту 1'' ``компонентой 2'', и наоборот). Проблему можно решить, положив, что $\pi_j$ и $\mu_j$ не убывают. Желательно, чтобы компоненты имели содержательную интерпретацию.

Одно из замечаний к модели конечной смеси заключается в том, что, вводя дополнительные компоненты для учета гетерогенности, мы можем попросту найти выбросы. Однако, это необязательно плохо, поскольку такая информация также полезна. Чтобы понять, что компонента включает в себя выбросы, можно воспользоваться формулой \ref{eq:18.23} для расчета апостериорной вероятности. В случае выброса эта вероятность будет велика по отношению к одной из компонент и мала по отношению к остальным.




\section{Выборка типа поток и запас}\label{sec:18.6}

\noindent
Во многих практических ситуациях часто возникает вопрос: В чем заключается взаимосвязь между двумя и более доступными мерами длительностей? Например, различия известны между такими концепциями как средний возраст и ожидаемая продолжительность жизни в демографии, период продажи существующей собственности и ожидаемый период продажи новой в недвижимости. Часто говорят о первом типе, в то время как к делу больше подходит второй. В экономике возникает аналогичный вопрос о различных мерах длительности безработицы, которая публикуется государственными статистическими агентствами. С этими обсуждениями тесно связана ненаблюдаемая гетерогенность, так как она относится и к запасу (совокупности) безработных, и к потоку. Одной из ранних и значимых работ по этой теме является Салант (1977).

Рассмотрим идею на примере с длительностью безработицы.
Одним из показателей, измеряющих продолжительность поисков работы уже безработных индивидов, является \textbf{средняя прекращенная длительность} (\textit{average interrupted duration}, AID). Показатель равен средней длительности поисков индивидов, находящихся в текущем запасе безработных, и его можно воспринимать как переменную запаса.
Это оценка \textbf{ожидаемого пройденного времени} (\textit{expected elapsed duration}), то есть, времени, которое индивид, оставшийся без работы, может ожидать, что потратит на ее поиски.
Часто такую оценку также называют средней длительностью завершенного состояния безработицы (average completed duration, ACD), о чем и идет речь в этой и предыдущих главах. В свою очередь, это является оценкой ожидаемой \textbf{завершенной длительности} (\textit{completed duration}), которую можно воспринимать как переменную потока.
По сути, AID аналогична среднему возрасту, а ACD --- ожидаемой продолжительности жизни. Вопрос же заключается в том, каким образом они взаимосвязаны.

Для решения подобных вопросов используют такой статистический инструмент, как \textbf{теорию восстановления} (\textit{renewal theory}).
Примером \textbf{процесса восстановления} (\textit{renewal process}) является стационарный пуассоновский процесс с постоянной интенсивностью. Количество обновлений в интервале $dt$ является числом событий. Длительность равна промежутку между успешными наступлениями событий (то есть, обновлениями). Для объекта в определенном состоянии пройденное с момента обновления время называется \textbf{обратным временем повторения} (\textit{backward recurrence time}). \textbf{Прямое время повторения} (\textit{forward recurrence time}) обозначает продолжительность периода с текущего момента до момента перехода. Ожидаемое число событий $\E[N(t)]$ в промежутке $(0, t]$ называется \textbf{функцией восстановления} (\textit{renewal function}), а $\lim_{dt\rightarrow0}\mathrm{d}\E[N(t)]/dt$ --- \textbf{интенсивностью восстановления} (\textit{renewal intensity}), которая и определяет зависимость между ACD и средним обратным временем повторения. Далее будут представлены некоторые важные результаты.

Салант (1977) показал, что гетерогенность, присутствующая в коэффициентах риска, может играть ключевую роль в объяснении взаимосвязи между AID и ACD. Представленный им график помогает интуитивно понять два фактора, влияющих на расчет средних. На вертикальной оси рисунка \ref{fig:18.1} отображено календарное время, где горизонтальная ось соответствует дате начала исследования. \textbf{Выборка типа запас} строится на основе запаса объектов, находящихся в данном состоянии. \textbf{Выборка типа поток}, напротив, означает, что мы отбираем только тех, кто попадает в данное состояние в течение периода исследования. Продолжительности состояния объектов изображены вертикальными линиями. На момент начала наблюдения четыре из девяти объектов (S6, S7, S8 и S9) уже находились в исследуемом состоянии. Еще пять (S1, S2, S3, S4 и S5) начались и завершились в течение 12-месячного периода исследования. Пусть $u_j$ обозначает длительность $j$-го продолжающегося наблюдения, а $t_i$ --- длительность $i$-го завершенного наблюдения. Тогда в нашем примере $\mathrm{AID} = 1/4(\sum_ju_j)$, а $\mathrm{ACD} = 1/5(\sum_i t_i)$.

Заметим, что длительных наблюдений в выборке, вероятно, больше, чем коротких. Как следствие, оценка средней длительности будет завышена, и такая выборка будет \textbf{смещенной по длительностям} (\textit{length-biased sampling}). В результате, соотношение между показателями будет выглядеть как AID > ACD. Однако, поскольку выборка также включает неполные длительности, то среднее, вероятно, будет ниже, чем среднее по полностью наблюдаемым объектам. Такой эффект называется \textbf{смещением прекращенных длительностей} (\textit{interruption bias}). Итоговый эффект зависит от распределения длительностей, который, в свою очередь, зависит от распределения коэффициентов риска, где гетерогенность играет ключевую роль.

    \begin{figure}[ht!]\caption{Смещенная по длительностям выборка типа запас: примеры}\label{fig:18.1}
    \centering
%    \includegraphics[scale=0.7]{fig.png}
    \end{figure}

Основной предпосылкой является стационарность, то есть, когда количество переходов в состояние и количество выходов из него равны между собой. Пусть плотность распределения неполных длительностей равна $f(u)$, а плотность полных --- $g(t)$. Тогда распределение $u$ будет равно
    \begin{align}
        \label{eq:18.27}
        f(u) = \frac{\bar{G}(u)}{\int \bar{G}(u)du} = \frac{\bar{G}(u)}{\E[t]},
    \end{align}
где
    $$\bar{G}(u) = \int g(x)dx$$
соответствует функции выживания с плотностью $g(u)$, а $\E(t)$ --- математическое ожидание распределения полных длительностей. Полный вывод формулы со всеми предпосылками можно найти в работах Салант (1977) и Ланкастер (1990, раздел 5.3).

Таким образом, если $g(t)$ экспоненциальная функция плотности, то стохастический процесс наступления событий является процессом Пуассона, $f(u)$ также экспоненциальная, а средние для обеих мер равны.

На основе (\ref{eq:18.27}) можно получить общее соотношение между моментами распределений $u$ и $t$. В частности, математическое ожидание $u$ можно найти, зная математическое ожидание и дисперсию $t$:
    \begin{align}
        \label{eq:18.28}
        \E[u] = \frac{1}{2}\left( \E[t] + \frac{\V[t]}{E[t]} \right).
    \end{align}

Также мы можем определить взаимосвязь между $\E(t)$ и средней завершенной длительностью для \textit{постоянной генеральной совокупности} с продолжающимися (еще не завершенными) наблюдениями (то есть, средним значением по запасу незавершенных наблюдений). Такое соотношение будет равно % опечатка в Кэмероне

    \begin{align}
        \label{eq:18.29}
        \E[t^{S(t)}] = \E[t] + \frac{\V[t]}{E[t]} > \E[t],
    \end{align}
что означает, что средняя длительность по постоянной совокупности, обозначенная $\E[t^{S(t)}]$, превышает среднюю ожидаемую длительность нового объекта. Если $f(t)$ является экспоненциальным, то $\E[t^{S(t)}] = 2 \E[t]$ и $\E[u] = 1/2\E[t^{S(t)}]$. То есть, в среднем, прекращенная длительность будет равна половине завершенной.

Что, если коэффициент риска не постоянен? Если он возрастает с длительностью объекта (то есть, положительная зависимость от состояния), то $\E[u] < \E[t]$, если же он убывает (негативная зависимость от состояния), то $\E[u] > \E[t]$.

Несмотря на то, что мы получили эти результаты при предпосылке о постоянности совокупности, они помогают понять взаимосвязь между распространенными мерами средних длительностей. Причина наступления события при этом не играет роли. Следовательно, форму функции риска нужно моделировать аккуратно, принимая во внимание описанные рассуждения.




\section{Тестирование спецификации}\label{sec:18.7}

\noindent
В моделях времени жизни тестирование спецификации включает в себя:
\begin{itemize}
\item тесты на включение переменных (значимость коэффициентов),\\
\item тесты на функциональную форму функции выживания,\\
\item тесты на ненаблюдаемую гетерогенность,\\
\item совместные тесты на зависимость от состояния % state dependence, предшествующий опыт; то же самое, что и duration dependence
и ненаблюдаемую гетерогенность.
\end{itemize}


Первый тип не подразумевает ничего нового, поскольку такие тесты проводятся с помощью статистики Вальда.

Тесты с ограничением на функциональную форму не отличаются от тестов на ненаблюдаемую гетерогенность, где в качестве ограничения мы предполагаем отсутствие ненаблюдаемой гетерогенности. Поскольку неучет гетерогенности может служить причиной смещения оценок, как было показано в разделе \ref{sec:18.2}, желательно проведение диагностических тестов.

Стандартная задача заключается в том, чтобы проверить, отличается ли составляющая гетерогенности от нуля, или нет. Если для проверки гипотезы мы используем модель с ограничением на отсутствие гетерогенности, то применяется скор-тест. Отношение правдоподобия и тест Вальда, напротив, неприменимы, поскольку основываются на модели без ограничений, где проверяемый параметр находится на границе гипотезы.
Например, смесь Вейбулла--Гамма \ref{eq:18.9} при $1/\de = 0$ равна модели Вейбулла, но такая гипотеза будет граничной. % ALT?
Стандартный хи-квадрат тест с одной степенью свободы при нулевой гипотезе имеет взвешенное хи-квадрат распределение.


\subsection{Тестирование гипотез}\label{sec:18.7.1}

\noindent
Одним из типов тестов на спецификацию является скор-тест на ненаблюдаемую гетерогенность в экспоненциальной модели. Поскольку гетерогенность легко перепутать с зависимостью от длительности, желательно проводить совместный, а не отдельный тест. Сделать это можно в рамках модели Вейбулла с локальной гетерогенностью (Ланкастер, 1985).

\textbf{Локально гетерогенная плотность} получается с помощью разложения в ряд Тейлора около $\nu = 1$ с плотностью Вейбулла с мультипликативной гетерогенностью $\nu$, то есть
    \begin{align}
        S(t|\nu) &= e^{-\mu t^{\al}\nu} = e^{-\e\nu} \notag \\
                 &= e^{-\e}[1 + (-\e)(\nu - 1) + (\e^2/2)(\nu - 1)^2 + O(\e^3)], \notag
    \end{align}
где $\e = \mu t^{\al}$. Из второй строки получим
    $$\E[e^{-\e\nu}] = e^{-\e}[1 + (\e^2\sis/2)] \equiv S_m(t),$$
где $\sis$ равен параметру дисперсии в распределении гетерогенности.

Тогда
    \begin{align}
        f_m(t) &= - \frac{\pa S_m(t)}{\pa t} \notag \\
               &= \al\mu t^{\al-1}e^{-\e}[1 + (\e^2\sis/2)] - e^{-\e}[2\e(\al\mu t^{\al-1})\sis/2] \notag \\
               &= \al\mu t^{\al-1}e^{-\e}[1 + \sis(\e^2-2\e)/2]. \notag
    \end{align}
Используя последний результат и допуская наличие цензурированных наблюдений, логарифм правдоподобия можно записать как
    \begin{align}
        \ln\mL (\al, \be, \sis) &= \sum^{N}_{i=1}\ln\{[f_m(t)]^{\de_i}[S_m(t)]^{1-\de_i}\} \notag \\
                                &= \sum^{N}_{i=1}\de_i[\ln\al + (\al-1)\ln t_i + \ln\mu_i +\ln(1 + \sis(\e^{2}_{i}-2\e_i)/2) - \e_i \notag\\
                                &  + (1-\de_i)\ln(1 + \sis\e^{2}_{i}/2)], \notag
    \end{align}
где $\de_i$ --- индикатор (отсутствия) цензурирования, равный 1 для нецензурированных наблюдений и 0 в обратном случае, $\ln\mu_i = \beta_0 + \xib_1$, и $\e_i = \mu_i t_i^\al$ --- \textbf{обобщенная ошибка} (см. раздел \ref{sec:18.7.2}).

Нулевая гипотеза записывается как $H_0: \sis = 0$ и $\al = 1$, и проверяет одновременное отсутствие ненаблюдаемой гетерогенности и спецификацию экспоненциального распределения. Пусть $\bttt = (\bttt^{'}_1, \bttt^{'}_2)$, $\bttt^{'}_2 = (\beta_0, \be_1)$ и $\bttt^{'}_0 = (0,1,\beta_0,\be_1)$, а $\bttt'_0 = (0, 1, \beta_0, \be_1)$ обозначает вектор ограничения. Для простоты будем рассматривать только нецензурированные наблюдения. Тогда совместная скор-статистика равна
    \begin{align}
        \label{eq:18.30}
        \mathrm{LM}_{\mathrm{HD}} = \frac{1}{d}\bs^{'}\left[ \begin{array}{cc}
        \Psi'(1) & 1 \\
        1        & 1
        \end{array} \right] \bs,
    \end{align}
где $\bs^{'} = [\frac{1}{2}\sum_i(\e^{2}_{i} - 2\e_i)]$, $\sum_i(1+1(1-\e_i)\ln t_i)$ и $\Psi^{'}(r)$ соответствует первой производной дигамма функции $d\ln\Ga(r)/dr$ и $d = 1/(N(\Psi^{'}(1) - 1))$. Для проведения теста статистика $\mathrm{LM}_{\mathrm{HD}}$ оценивается при нулевой гипотезе (то есть, для экспоненциального распределения) и имеет асимптотическое $\chi^2(2)$ распределение (Джаггиа и Триведи, 1994).

Заметим, что матрица квадратичной формы в статистике $\mathrm{LM}_{\mathrm{HD}}$ не является диагональной, поскольку компоненты совместного теста коррелированы друг с другом. Отдельный тест на гетерогенность (зависимость от длительности) может также указывать на зависимость от длительности (гетерогенность). % ENG: has power against duration dependence ? correct?
Чтобы пояснить, рассмотрим два отдельных теста
    \begin{align}
        \label{eq:18.31}
        \mathrm{LM}_{\mathrm{H}} = \frac{1}{4N}(\sum_i(\e^{2}_{i} - 2\e_i))^2,
    \end{align}
    \begin{align}
        \label{eq:18.32}
        \mathrm{LM}_{\mathrm{D}} = \frac{1}{d}\sum_i(1 + (1-\e_i)\ln t_i))^2,
    \end{align}
каждый из которых имеет распределение $\chi^2(1)$ при нулевой гипотезе. Поскольку тесты коррелированы, см. (\ref{eq:18.30}.), каждый из них будет иметь мощность против другой нулевой гипотезы. Как следствие, выводы о характере неправильной пефицикации на основе отдельных тестов могут быть ошибочны.

Поскольку спецификация ненаблюдаемой гетерогенности тесно связана с зависимостью от состояния, тестирование гипотез по отдельности может приводить к неправильным результатам (Джаггиа и Триведи, 1994). Формально, тесты на зависимость от состояния без учета ненаблюдаемой гетерогенности, и наоборот, смещены. Джаггиа (1991c) проводит повторный анализ данных по длительностям забастовок, которые ранее были ошибочно представлены в экономической литературе. Джаггиа и Триведи (1994) предлагают несколько совместных тестов для класса параметрических моделей. Бера и Юн (1993) рассматривают более общие вопросы тестирования гипотез при неправильной спецификации моделей.

Эти тесты полезны в простых параметрических моделях, так что анализ можно начать с моделей Вейбулла, Вейбулла--Гамма или пропорциональных рисков. Тестирование ненаблюдаемой гетерогенности или любой другой ошибки спецификации возможно с помощью функции интегрального риска, поскольку интегральный риск без гетерогенности является нормированно экспоненциально распределенной случайной величиной. Далее мы представим обсуждение методов для оценки качества модели, основанных на интегральном риске.




\subsection{Графические способы выявления неправильной спецификации}\label{sec:18.7.2} % ALT: обнаружения

\noindent
В разделе 8.7.2 % \ref{sec:8.7.2}
мы обозначили понятие обобщенных остатков. В нелинейный моделях выбор такой меры обычно неочевиден, однако в данном контексте можно найти подходящий вариант.

        \begin{center}{Обобщенные остатки}\end{center}
        \noindent
Графические непараметрические тесты представляют собой практичный инструмент для оценки качества подгонки модели. Тесты используют обобщенный остаток как некоторую функцию от данных и оценок параметров. Для верно специфицированной модели остатки должны быть приблизительно независимо и одинаково распределены. Такому критерию удовлетворяет интегральный риск, и, следовательно, может быть использован для проведения теста на основе остатков. В контексте моделей времени жизни из раздела 17.3.1
    \begin{align}
        S(t|\mu) &= \exp[\La(t|\mu)], \notag \\
        f(t|\mu) &= \la(t|\mu)\exp[\La(t|\mu)], \notag
    \end{align}
\textbf{обобщенный остаток} равен
    \begin{align}
        \label{eq:18.33}
        \eps &= \La(t|\mu) \\
             &= -\ln (S(t|\mu)). \notag
    \end{align}
Якобиан преобразования может быть записан как
    \begin{align}
        |J| &= dt/d\eps \notag \\
            &= \frac{1}{d\La(t|\mu) / dt} \notag \\
            &= 1/\la(t|\mu). \notag
    \end{align}
Зная $f(t|\mu)$, преобразование (\ref{eq:18.33}) и Якобиан преобразования, получим плотность распределения $\eps$
    \begin{align}
        \label{eq:18.34}
        \la(t|\mu) \exp(-\eps)\frac{1}{\la(t|\mu)} = \exp(-\eps),
    \end{align}
которая не зависит от $\mu$ и имеет нормированное экспоненциальное распределение. Данный результат был использован в разделах
17.3.1 и % \ref{sec:17.3.1} # UNCOMMENT 17
17.6.7. % \ref{sec:17.6.7} # UNCOMMENT 17


        \begin{center}{Диагностические тесты на основе интегрального риска}\end{center}
        \noindent
Используя свойство нормированного экспоненциального распределения обобщенного остатка, можно построить диагностический тест при нулевой гипотезе, что спецификация верна. Функция выживания для обобщенного остатка равна $S(\eps) = \exp(-\eps)$, откуда следует, что $-\ln S(\eps) = \La(\eps) = \eps$. Для верно специфицированной модели зависимость между оценкой интегрального риска и обобщенным остатком $\hat{\eps}$ изображается прямой линией под углом $45^{\circ}$. Значительное отклонение от этой линии означает, что модель специфицирована неверно.

Например, \textbf{оценка интегрального риска} в модели Вейбулла равна $\hat{\eps} = \hat{\mu}t^{\hat{\al}}$, а функция выживания $\hat{S}(\hat{\eps}) = N^{-1}$ (количество наблюдений в выборке $\ge \hat{\eps}$). Формально, можно построить регрессию $-\ln S(\hat{\eps})$ на $\hat{\eps}$ и проверить совместную гипотезу, что константа равна 0, а коэффициент наклона --- 1.

Такой метод применим к любой параметрической модели, для которой можно выразить интегральный риск. Например, для смеси Вейбулла--Гамма (и смеси экспоненциального и гамма распределений при $\al = 1$) \textit{обобщенная ошибка} равна $\eps = k\ln[(k + \mu t^{\al})/k]$. Чтобы применить тест, нужно сперва найти $\hat{\eps}$ при данных $(\mu,\al,k)$, а затем построить графическую зависимость $\hat{\eps}$ от $-\ln \hat{S}(\hat{\eps})$.


        \begin{center}{Цензурированные данные}\end{center}
        \noindent
Если наблюдения цензурированы, длительность равна $t = \min[T,L]$, где $L$ обозначает предел цензурирования справа. Если длительность превышает $L$, то она оказывается цензурирована в момент $L$. В таком случае обобщенная ошибка $\eps(t)$ больше не подчиняется нормированному экспоненциальному закону распределения. Соответствующая поправка для цензурированных данных выглядит следующим образом
    \begin{align}
        \label{eq:18.35}
        \E[\eps(T)|T\ge L] &= \int^{\infty}_{\eps(L)}\frac{\eps f(\eps)}{S(\eps (L))} d\eps \notag \\
                           &= \frac{1}{e^{\eps(L)}}\left[ \int^{\infty}_{\eps(L)} \eps e^{-\eps} d\eps \right] \notag \\
                           &= \frac{1}{e^{\eps(L)}}[1 + \eps(L)e^{-\eps(L)} + e^{-\eps(L)} - 1] \notag \\
                           &= 1 + \eps(L),
    \end{align}
где требуется интегрирование по частям и упрощение.

Таким образом, обобщенная ошибка может быть оценена как $\tilde{\eps}(t) = \hat{\eps}(t)$ для нецензурированных наблюдений и как $\tilde{\eps}(t) = 1 + \hat{\eps}(L)$ для цензурированных. Исследования показывают, что такой метод работает достаточно хорошо в экспоненциальной модели с цензурированными данными, где доля таких наблюдений относительно невелика (Джаггиа и Триведи, 1994; Джаггиа, 1997).

\subsection{Тесты на условный момент}\label{sec:18.7.3} % !!!Термин должен совпадать с термином из главы 8!!! ок
\noindent

Для тестирования спецификации к обобщенным остаткам можно применить идею \textbf{условного момента} (см. раздел 8.2), % \ref{sec:8.2} # UNCOMMENT IN THE END
представив ее в контексте тестирования ненаблюдаемой гетерогенности.

Ранее было показано, что интегральная функция риска имеет нормированное экспоненциальное распределение с математическое ожиданием и дисперсией, равными 1. Тогда соответствующее ограничение на условный момент второго порядка равно $\E[(\eps - 1)]^2 = \V[\eps] = 1$, что эквивалентно
    $$\E[\eps^2 - 2] = 0.$$
Ограничения более высокого порядка можно также построить и проверить, как совместно, так и по отдельности. Детали можно найти в работе Джаггиа (1991a).




\section{Пример ненаблюдаемой гетерогенности: длительность безработицы}\label{sec:18.8}

\noindent
В этом разделе мы вернемся к примеру из раздела 17.11, % \ref{sec:17.11} # UNCOMMENT AFTER 17 CHAPTER
предположив, что в модели присутствует ненаблюдаемая гетерогенность, выраженная в аналитической форме.

\begin{figure}[ht!]\caption{Длительность безработицы: обобщенные остатки в экспоненциальной модели. Данные США по 3343 наблюдениям в 1986--92 гг., некоторые наблюдения неполные.}\label{fig:18.2}
\centering
%\includegraphics[scale=0.7]{fig.png}
\end{figure}

Как уже обсуждалось в разделе \ref{sec:18.7.2}, возможное наличие ненаблюдаемой гетерогенности можно проверить графически, построив предсказанные значения модели. Для верно специфицированной модели остатки должны следовать нормированному экспоненциальному распределению. Качество модели можно также оценить неформально, рассчитав эмпирическую функцию кумулятивного риска и сравнив ее с обобщенными остатками. Для верно специфицированной модели должна получиться почти прямая линия с наклоном, равным единице.

На рисунках \ref{fig:18.2} и \ref{fig:18.3} изображены обобщенные остатки в экспоненциальной модели с ненаблюдаемой гетерогенностью и без, соответственно. По графикам видно, что качество модели с учетом ненаблюдаемой гетерогенности незначительно выше.
    \begin{figure}[ht!]\caption{Длительность безработицы: обобщенные остатки в модели смеси экспоненциального и гамма распределений. Данные те же, что и на рисунке \ref{fig:18.2}.}\label{fig:18.3}
    \centering
    %\includegraphics[scale=0.7]{fig.png}
    \end{figure}

    \begin{table}[!htbp]\caption{\textit{Длительность безработицы: экспоненциальная модель с гамма и обратной гауссовской гетерогенностью}}\label{tab:18.1}
    \begin{center}
\begin{tabular}{lcccc}
\hline \hline
&\multicolumn{2}{c}{\textbf{Экспоненциальное--Гамма}}&\multicolumn{2}{c}{\textbf{Экспоненциальное--IG}}\\
\cmidrule(r){2-3}\cmidrule(r){4-5}
\textbf{Переменная} &\textbf{Коэффициент}   &$t$      &\textbf{Коэффициент}   &$t$ \\
\hline
RR                  &0.501  &0.817  &0.504  &0.821 \\
DR                  &-0.882 &-1.118 &-0.807 &1.032 \\
UI                  &-1.585 &-6.043 &-1.545 &-5.994 \\
RRUI                &1.091  &1.725  &1.057  &1.686 \\
DRUI                &0.057  &0.055  &-0.013 &-0.012 \\
LNWAGE              &0.379  &3.184  &0.373  &3.156 \\
CONS                &-4.095 &-4.507 &-4.097 &-4.545 \\
$\sis$              &0.232  &3.178  &0.207  &2.925 \\
$-\ln\textrm{L}$    &\multicolumn{2}{c}{2695.35}&\multicolumn{2}{c}{2696.48} \\
\hline \hline
\end{tabular}
    \end{center}
    \end{table}

Графические результаты могут быть подтверждены расчетами, представленными в таблице \ref{tab:18.1}, где также указаны оценки экспоненциальной модели с обратной гауссовской гетерогенностью (IG, \textit{inverse-Gaussian}). Несмотря на то, что мы знаем, что в данных присутствует ненаблюдаемая гетерогенность, оценки коэффициентов лишь незначительно отличаются от тех, что были получены в предыдущей главе. Значит, она должна оказывать существенный эффект на параметр зависимости от длительности, так как он не учитывается в экспоненциальной модели.

Следовательно, более интересный случай для нас представляет модель, учитывающая как ненаблюдаемую гетерогенность, так и зависимость от длительности, например, модель Вейбулла со смесью IG. Для удобства сопоставления, мы представим оценки в таблице \ref{tab:18.2} рядом с оценками без учета гетерогенности.

Согласно представленным расчетам, ненаблюдаемая гетерогенность оказывает существенный эффект на параметр зависимости от длительности, который возрастает с $1.129$ в таблице 17.8 % \ref{tab:17.8} # UNCOMMENT AFTER 17 CHAPTER
до $1.753$ в таблице \ref{tab:18.2}, что соответствует более крутому наклону коэффициента риска в модели с гетерогенностью. Вспомним из раздела \ref{sec:18.2.4}, что одно из последствий неучета ненаблюдаемой гетерогенности в модели пропорциональных рисков заключается в недооценке коэффициента риска; следовательно, данный результат согласуется с теорией. О наличии гетерогенности свидетельствует также то, что $t$-статистика для оценки параметра дисперсии $\sis$ превышает $11$. Наконец, качество модели, измеренное с помощью логарифма правдоподобия, также повысилось с $-2687.6$ до $-2616.6$. Хотя изменения в оценках коэффициентов и незначительны, эффекты значимых коэффициентов (UI, LNWAGE и CONS) стали более выражены.

    \begin{table}[!htbp]\caption{\textit{Длительность безработицы: модель Вейбулла с обратной гауссовской гетерогенностью и без.}}\label{tab:18.2}
    \begin{center}
\begin{tabular}{lcccc}
\hline \hline
&\multicolumn{2}{c}{\textbf{Weibull-IG}}&\multicolumn{2}{c}{\textbf{Weibull}}\\
\cmidrule(r){2-3}\cmidrule(r){4-5}
\textbf{Переменная} &\textbf{Коэффициент}   &$t$      &\textbf{Коэффициент}   &$t$ \\
\hline
RR                  &0.736  &0.812  &0.448  &0.70 \\
DR                  &-1.073 &-0.933 &-0.427 &-0.53 \\
UI                  &-2.575 &-6.698 &-1.496 &-5.67 \\
RRUI                &1.734  &1.857  &1.105  &1.57 \\
DRUI                &-0.061 &-0.039 &-0.299 &-0.28 \\
LNWAGE              &0.576  &3.259  &0.37   &2.99 \\
CONS                &-5.303 &-3.953 &-4.358 &-4.74 \\
$\al$               &1.753  &44.19  &1.129  &51.44 \\
$\sis$              &6.377  &11.149 &-      &- \\
$-\ln\textrm{L}$    &\multicolumn{2}{c}{2616.6}&\multicolumn{2}{c}{2687.6} \\
\hline \hline
\end{tabular}
    \end{center}
    \end{table}

Тем не менее, несмотря на повышение качества модели, полученная модель смеси по прежнему может быть мисспецифицирована. Графический инструментарий снова может быть использован в качестве неформального теста на спецификацию. Изображенные на рисунках \ref{fig:18.4} и \ref{fig:18.5} обобщенные остатки из моделей Вейбулла с ненаблюдаемой гетерогенностью и без указывают на неверную спецификацию модели смеси.

Заметим, что для ``улучшенной'' модели, учитывающей как ненаблюдаемую гетерогенность, так и зависимость от длительности, проще определить определить ошибочность спецификации по сравнению с более простой моделью, не учитывающей ни один из факторов; похожий результат представлен в работе Джаггиа (1991c). Объяснить такой результат можно с помощью взаимосвязи между гетерогенностью и зависимостью от длительности. В то время как модель Вейбулла предполагает монотонные риски, МакКолл (1996) показывает, что для этих же данных лучше подходит U-образная форма функции риска. В частности, он использует более гибкую полиномиальную спецификацию функции риска. Следовательно, основной вывод здесь заключается в том, что для модели, которая допускает наличие обоих факторов, обнаружить мисспецификацию проще.

\begin{figure}[ht!]\caption{Длительность безработицы: обобщенные остатки в модели Вейбулла. Данные те же, что и на рисунке \ref{fig:18.2}.}\label{fig:18.4}
\centering
%\includegraphics[scale=0.7]{fig.png}
\end{figure}

\begin{figure}[ht!]\caption{Длительность безработицы: обобщенные остатки в модели смеси Вейбулла и обратного гауссовского распределения. Данные те же, что и на рисунке \ref{fig:18.2}.}\label{fig:18.5}
\centering
%\includegraphics[scale=0.7]{fig.png}
\end{figure}

Наконец, мы проведем параметрический тест на наличие ненаблюдаемой гетерогенности, целью которого является показать, как можно применить теорию, описанную в разделе \ref{sec:18.7}. Однако скор-тест на ненаблюдаемую гетерогенность из раздела \ref{sec:18.7.1} предназначен для работы с нецензурированными наблюдениями. Поскольку используемые данные содержат наблюдения, цензурированные справа, мы воспользуемся скор-тестом, представленном в работе Джаггиа (1997) для работы с цензурированной выборкой.

Мы хотим проверить отсутствие ненаблюдаемой гетерогенности в экспоненциальной модели времени жизни, $H_0:\sis=0$. Обозначим параметры как $\bttt = (\sis, \be)$, а информационную и скор-матрицы как $\cI(\bttt_0)$ и $\bs(\bttt_0)$, рассчитанные при нулевой гипотезе. Используя логарифм правдоподобия, полученный в разделе \ref{sec:18.7.1}, мы можем записать $\bs(\bttt_0) = (\bs_1(\bttt_0),\bs_2(\bttt_0))$, где $\bs_1(\bttt_0) = \frac{\pa\cL}{\pa\sis}\big|_{H_0} = \frac{1}{2}\sum(\eps^{2}_{i} - 2C_i\eps_i)$ и $\cI(\bttt_0) = -\E\left[\frac{\pa^2\cL}{\pa\ttt\pa\ttt^{'}}\right]\big|_{H_0}$. Тогда тестовая статистика будет выглядеть следующим образом
    \begin{align}
        \label{eq:18.36}
            \mathrm{LM} = \bs^{'}_{1}(\tilde{\bttt}_0)\cI^{11}(\tilde{\bttt}_0)\bs_1(\tilde{\bttt}_0)\sim\chi^2(1),
    \end{align}
где $\cI^{11} = [\cI_{11} - \cI_{12}(\cI_{22})^{-1}\cI_{21}]^{-1}$ является первым диагональным элементом блочной матрицы, обратной от $\cI(\bttt)$, предложенной Джаггиа (1997), а волны над оценками соответствуют методу максимального правдоподобия с ограничениями.

Для нашей выборки статистика равна $\mathrm{LM} = 44.25$, что значительно превышает критическое значение $\chi^{2}(1)$. Следовательно, мы отвергаем нулевую гипотезу, что $\sis = 0$. Результат согласуется с моделями смеси Вейбулла--Гамма и Вейбулла--IG, где существенное повышение качества модели является результатом учета ненаблюдаемой гетерогенности. Как было сказано ранее, данный тест может также указывать на неправильную спецификацию зависимости от длительности.




\section{Практические соображения}\label{sec:18.9}

\noindent
Вопрос о взаимосвязи функции риска и ненаблюдаемой гетерогенности послужил поводом для многочисленных исследований. Одна достаточно изученная точка зрения заключается в том, что спецификация распределения гетерогенности не играет существенной роли, если функция риска специфицирована правильно (Мантон и др., 1986). Другими словами, вместо параметрического моделирования ненаблюдаемой гетерогенности можно попросту использовать робастные оценки дисперсии, при условии, что спецификация функции риска верна. Другие авторы утверждают, что выбор формы гетерогенности, напротив, имеет значение (Хекман и Сингер, 1984a), и поэтому следует применять непараметрическую спецификацию. Некоторые известные работы также используют дискретную модель с гибкой спецификацией риска в сочетании с параметрическими предпосылками о распределении гетерогенности (Мейер, 1990; Хан и Хаусман, 1990). Наконец, в качестве компромисса некоторые исследователи предлагают объединить предыдущие подходы, совместив дискретную модель Хан-Хаусман, или же функцию риска в виде полинома высшего порядка, и непараметрическую гетерогенность Хекман-Сингер. Однако, Бейкер и Мелино (2000) показали, что такая общая модель может привести к другой проблеме --- перепараметризации. В таком случае нужно действовать аккуратно и отдавать предпочтение моделям с относительно небольшим числом параметров.

Модель PH Кокса является центральным объектом анализа в биометрической литературе. Если для исследования не требуется определенная форма базового риска, то такая модель является довольно привлекательной в отношении функциональной формы риска. Обычно, удобнее начинать именно с нее. При этом, поскольку ненаблюдаемая гетерогенность имеет место во многих эконометрических приложениях, нельзя игнорировать ее роль.

Многие статистические пакеты умеют работать с набором стандартных параметрических моделей, которые могут быть комбинированы с любой из стандартных спецификаций гетерогенности (гамма, обратной гауссовской или лог-нормальной). Но несмотря на удобство таких моделей в применении, дискретные модели риска обычно являются более гибкими и лучше соответствуют данным.

EM алгоритм, применяемый для оценки моделей латентных классов, работает довольно медленно, и максимизация правдоподобия напрямую зачастую оказывается более эффективна.




\section{Библиографические заметки}\label{sec:18.10}

\noindent

\begin{itemize}
    \item[\textbf{18.2}]
Существует довольно много работ, в которых обсуждается спецификация распределения гетерогенности и последствия неправильной спецификации. Вопель и др. (1979) предлагает качественное описание свойств модели гамма. Хугаард (1984) рассматривает различные альтернативы этой модели. Работа Хугаард (1995) представляет собой обзор моделей гетерогенности. Хекман и Сингер (1984a) указывают на преимущества непараметрической спецификации и чувствительность результатов к неправильной спецификации. В работе Мантон и др. (1986) представлена попытка сравнить последствия неправильной спецификации риска и гетерогенности.

    \item[\textbf{18.3}]
Ван ден Берг (2001) предлагает детальное и понятное описание идентификации модели смешанных пропорциональных рисков MPH и ссылки на работы по этой теме.

    \item[\textbf{18.4}]
Хан и Хаусман (1990) и Мейер (1990) представляют собой качественные эмпирические исследования, в которых гибкая спецификация риска сочетается с параметрическими предпосылками о распределении гетерогенности.

    \item[\textbf{18.5}]
Хекман и Сингер (1984a) является одной из первых работ, где предлагается обсуждение дискретной модели гетерогенности. Модель конечной смеси также называется ``непараметрической моделью гетерогенности''. Бейкер и Мелино (2000) проводят эксперимент по методу Монте Карло для оценивания модели с более гибкой спецификацией базового риска с непараметрической гетерогенностью. Результат заключается в том, что наличие множества компонент конечной смеси приводит к существенным смещениям и недостоверным выводам. В этом случае может быть полезным применение критериев BIC и Ханнана-Куинна, штрафующих за перепараметризацию.

    \item[\textbf{18.6}]
Ланкастер (1990) и Салант (1977) являются
Ланкастер
Тэйлор и Карлин (1994).
    \item[\textbf{18.7}]
Кифер (1988)
Джаггиа (1991a)
Грин (2003)
Эндрюс (1997)
Кэмерон и Триведи (1998, 6)
Хосмер и Лемешов (1999, 196-240).
    \item[\textbf{18.8}]
Ланкастер (1979)
Джаггиа (1991c)
19.
\end{itemize}




\section{Упражнения}\label{sec:18.ex}

\noindent

\begin{itemize}
    \item[\textbf{18--1}]
(Сапра, 2002)
В разделе \ref{sec:18.2} представлены последствия неучета гетерогенности для безусловной, или средней, функции риска. Согласно приведенному анализу, неучет гетерогенности приводит к недооценке коэффициента наклона средней функции риска. Пусть условная функция риска равна $\la_C(t|\nu) = \la_0(t)$, где $\la_0$ обозначает базовый, или безусловный, риск. Покажите, что (i) безусловный риск $\la_U(t) < \la_0(t)$ и (ii) $\pa\la_U(t)/\pa t < 0$ для каждого из следующих пунктов.
        \item[\textbf{(a)}]
$\nu \sim \mathrm{Uniform}[0,1]$ и $\la_0(t) = 1$ $\forall$ $t$.
        \item[\textbf{(b)}]
$\nu$ подчиняется нормированному экспоненциальному закону распределения с функцией плотности $g(\nu) = e^{-\nu}$ и $\la_0(t) = \rho \exp(\ga t)$, $\rho > 0$, $\ga < 0$.

    \item[\textbf{18--2}]
Рассмотрим модель Вейбулла--Гамма из раздела \ref{sec:18.2.3}, где вместо гамма-распределенной гетерогенности предполагается лог-нормальное распределение с математическое ожиданием, равным единице.
        \item[\textbf{(a)}]
Проверьте, что выражение для безусловной функции риска невозможно представить в аналитическом виде.
        \item[\textbf{(b)}]
Подставьте безусловный риск в виде интеграла в функцию логарифма правдоподобия, представленную в разделе 17.6.3. % \ref{sec:17.6.3} # UNCOMMENT AFTER 17 CHAPTER
Используя основанный на симуляции метод максимального правдоподобия (см. раздел 12.4), % \ref{sec:12.4} # UNCOMMENT AFTER 12 CHAPTER
опишите последовательность действий, с помощью которых находится максимум правдоподобия.

    \item[\textbf{18--3}]
Рассмотрим смесь экспоненциального и гамма распределений, которая является частным случаем модели смешанных пропорциональных рисков, MPH. Для экспоненциальной модели условная функция выживания относительно мультипликативного $\nu$ равна $S(t|\nu) = \exp(-\mu t \nu)$, $\la > 0$. Безусловная функция выживания дана функция выживания, усредненная по плотности распределения неоднородной совокупности $g(\nu)$, то есть, $S(t) = \int^{\infty}_{0} S(t|\nu)g(\nu)d\nu$. Пусть $\nu$ имеет (двухпараметрическое) гамма распределение с функцией плотности $g(\nu) = \de^k\nu^{k-1}\exp(-\de\nu)/\Ga(k)$.
        \item[\textbf{(a)}]
Покажите, что для гамма-распределенной гетерогенности $S(t) = (1 + \mu t /\de)^{-k}$.
        \item[\textbf{(b)}]
Выведите выражения для безусловной плотности распределении длительности $f(t)$ и безусловной функции риска $\la(t)$. Эти выражения можно упростить, положив, что $\E[\nu] = 1$, или же $k = \de$. В результате должна получиться смесь экспоненциального и гамма распределений. Сравните математическое ожидание и дисперсию для экспоненциального распределения и распределения смеси.

        \item[\textbf{(c)}]
Пусть случайная величина $\nu$ может принимать только два значения: $\nu_1$ с вероятностью $\pi$ и $\nu_2$ с вероятностью $(1 - \pi)$. Как это повлияет на безусловную функцию риска? Объясните ответ.

    \item[\textbf{18--4}]
Используя выборку по данным МакКолла из упражнения 17--3, оцените заново модель Вейбулла для тех индивидов, кто переходит на полную ставку (CENSOR1 = 1), при условии, что ненаблюдаемая гетерогенность (в некоторых пакетах используется термин уязвимость, \textit{frailty}) имеет гамма распределение.
        \item[\textbf{(a)}]
Используя обобщенные остатки, представленные в разделе \ref{sec:18.7.2}, проверьте гипотезу о неверной спецификации модели.
        \item[\textbf{(b)}]
Описывает ли новая модель зависимость от длительности соответствующим образом? Подходит ли она лучше для описания данных? % качество подгонки?
Объясните ответ через взаимосвязь ненаблюдаемой гетерогенности и зависимости от длительности.
        \item[\textbf{(c)}]
Повторите пункт (a) при условии, что гетерогенность распределена лог-нормально. Значительны ли различия в результатах?
\end{itemize}



\chapter{Модели множественных рисков}

% Тонкости перевода:
%
% 1. competing risks vs hazard function
% В данной главе вводится понятие "конкурирующих рисков", что в оригинале звучит как "competing risks". По сути, здесь имеется в виду конкуренция "причин".
% Очевидно, "(функция) риска" переводит другое слово "hazard". Хотя и там, и там "риски" обозначают "риски завершения длительности" (то бишь, это синонимы), иногда полезно понимать, что когда мы произносим термин "конкурирующие риски", "риски наступления события/причины q", "независимые риски" и т.д., мы переводим слово "risks", а когда произносим термин "функция риска", "базовый" и "интегральный" "риски", "пропорциональные риски" и т.д., мы переводим слово "hazard".
% Моя гипотеза --- что эти термины пришли из независимых концепций о похожих вещах, отсюда и различные названия.
%
% 2. marginal distributions
% переводим как "маргинальные распределения"
% в журнале "прикладная эконометрика" (Фантаццини про копула-функции) используется другой термин "частные распределения"
%
% 3. spell
% хотя это и означает буквально "период", я перевожу его чаще всего как "событие" (иногда "состояние" --- в главе 17), поскольку Кэмерон иногда говорит то event, то spell. Здесь имеется в виду "период, месяц, время года, время и состояние жизни в который наблюдалась длительность", то есть, непосредственно факт события. Например, многократные события (multiple spells) означают, что одно и то же событие повторилось много раз. Грубо говоря, один и тот же период (например, период состояния безработицы или беременности) повторился много раз. А это есть не что иное как повторяющееся событие.
%

% 341, 351, 454 лишние пустые строки 

\section{Введение}\label{sec:19.1}

\noindent
В данной главе мы рассмотрим несколько различных моделей времени жизни, которые составляют класс многомерных моделей, включающих параллельные и повторяющиеся переходы. Любая модель перехода с двумя и более состояниями может быть классифицирована как многомерная модель, поскольку такой анализ подразумевает оценку совместного распределения нескольких длительностей. Несмотря на то, что изучаемые модели могут относиться к различным ситуациям и типам данных, мы представим их в рамках одной главы для организационного удобства.

В качестве пояснения рассмотрим несколько примеров. Уже знакомая из экономики труда модель предполагает, что безработный индивид может быть как нанят на работу, так и исключен из состава рабочей силы. Первый переход можно конкретизировать далее, указав, что индивид вернулся на старую работу или нашел новую. Такие переходы являются взаимоисключающими. Поиски работы могут быть завершены переходом в любое из этих состояний. Как вариант, индивид также может устроиться на полставки, полную ставку или же остаться безработным. В таком случае, существует три возможных состояния. Ранее в главах 17 и 18 мы рассматривали модели перехода для двух состояний, которые применимы и в данном случае, если предположить, что 1 означает полную занятость, а 0 --- любое другое состояние. Для этого, как мы знаем, требуется смоделировать единственный коэффициент риска. Однако данную ситуацию возможно представить в полном виде, то есть, для трех состояний, с двумя переходами и двумя функциями риска, по одной для каждого перехода. В общем же случае, допустимо множество типов отказов, где переход из текущего состояния может произойти по причине любого отказа из этого множества. В данной главе мы хотим представить методы для моделирования таких переходов, и усовершенствовать представленный ранее инструментарий таким образом, чтобы получить возможность анализа множественных рисков, или многомерной модели времени жизни.

В данном контексте представляется важным понять ответы на следующие вопросы:
\begin{enumerate}
\item Как следует моделировать отношение между регрессорами и отказами различных типов?
\item Как следует моделировать взаимосвязь между типами отказов в рамках конкретного исследования?
\item Как оценивать коэффициенты отказа для конкретных типов отказов при условии исключения одного или всех остальных типов?
\end{enumerate}

\textbf{Многомерная модель времени жизни} подразумевает одновременное моделирование всех переходов, то есть, совместную спецификацию и оценивание двух и более функций риска. Существует несколько различных концепций анализа многомерных данных по длительностям, наиболее популярной из которых является идея \textbf{конкурирующих рисков} (\textit{competing risks}). На основе данной модели МакКолл (1996) представляет эмпирическое приложение по анализу данных по безработице с акцентом на роль страхования. Используя аналогичный подход, Денг, Куигли и Ван Ордер (2000) исследуют предварительную выплату или полное погашение ипотечных кредитов.

Однако в чем заключается смысл совместного моделирования отказов? Если отказы, по сути, независимы, то мы получим, в принципе, тот же результат. Ответ на вопрос состоит в том, что различные отказы могут быть связаны друг с другом; например, если в функциях риска присутствует общий параметр ненаблюдаемой гетерогенности, то отказы будут коррелированы.


Вторая группа моделей представляет анализ совместного распределения длительностей в ситуации, когда возможны параллельные переходы. Например, набор $(T_1, T_2)$ может обозначать длительности поисков работы и периода отсутствия страхового полиса. В таком случае идея совместного оценивания рисков аналогична предыдущему примеру.

Третья группа предполагает совместное распределение \textbf{повторяющихся периодов}, или \textbf{длительностей},
того же состояния (например, безработицы или отсутствия страхового полиса). То есть, для данного индивида, мы хотим одновременно смоделировать риски завершения длительности. Если длительности независимы, то анализ может быть представлен в контексте единственного перехода, как было показано ранее. Если же нас интересует структура зависимости переходов, то подходит совместное моделирование длительностей в данном состоянии. Когда длительности зависимы, требуется применение новых методов и моделей. Этот пример сложнее, чем первые два, из-за возможной зависимости между событиями, разбросанными во времени. Например, на вероятность последующих событий и их длительности может влиять предыдущая длительность и тип события, или же вся предыстория. Коррелированная ненаблюдаемая гетерогенность образует связь между повторяющимися событиями так, что вероятность наступления даже того же самого события будет зависеть от предыдущих исходов. Хекман и Борхас (1980) описывают несколько структурных типов зависимости от состояния для индивида, используя такие понятия, как \textbf{зависимость от наступления события} и (марковскую) \textbf{лаговую зависимость от длительности.}

Множеству различных ситуаций и типов данных соответствует разнообразие типов моделей, которые на первый взгляд могут показаться не имеющими друг к другу никакого отношения. Однако, на самом деле, в каждой из них присутствуют общие черты. Мы начнем анализ с определения базовых концепций в разделе \ref{sec:19.2}, а затем представим известную модель конкурирующих рисков. В разделе \ref{sec:19.3} мы рассмотрим многомерную модель, основанную на маргинальных распределениях моментов выживания, и представим подход, использующий копулы для оценки таких распределений. Модели с множественными событиями будут представлены в разделе \ref{sec:19.4}.




\section{Конкурирующие риски}\label{sec:19.2}
% вообще говоря, термин ``конкурирующий'' можно назвать устоявшимся, хотя в некоторых ситуациях ``более по-русски'' звучит ``альтернативный'', например, ``альтернативные причины смерти''.

\noindent
В начале мы определим основные понятия, которые используются в \textbf{модели конкурирующих рисков} (CRM, \textit{competing risks model}) и других многомерных моделях. Часто эти понятия являются общим видом ранее представленных концепций.  Так, базовая формулировка модели CRM применима к моделированию длительности в одном состоянии, где переход происходит по ряду конкурирующих причин, таких как различные способы смерти. Модель CRM является привлекательной благодаря относительной простоте в применении, если базовая модель --- типа PH.


\subsection{Основные понятия}\label{sec:19.2.1}

\noindent
Мы будем рассматривать CRM с $m$ латентными длительностями, или моментами отказа, при соответствующих конкурирующих причинах.

        \begin{center}Латентные длительности\end{center}
        \noindent
Модель устроена следующим образом. Каждый объект характеризуется моментом отказа, который может быть цензурирован. Момент отказа может быть любого типа из множества $J = \{1, \ldots , m\}$, что можно представить как ситуацию, когда существует $m$ различных причин перехода из данного состояния (<<смертей>>). При этом, наступление одного отказа избавляет нас от рисков наступления отказа любого другого типа. В таком случае, при цензурировании оставшихся $(m-1)$ длительностей мы наблюдаем не более одного отказа.

В CRM с $m$ типами отказов существует $m + 1$ возможных состояний $\{0, 1, \ldots , m\}$, где 0 обозначает текущее состояние, а $\{1, \ldots , m\}$ --- возможные состояния в результате перехода. Для $i$-го объекта вектор данных выглядит как $(\x_i, t_i, d_{1i}, \ldots , d_{mi}, d_{ci})$, где $\x_i$ --- это вектор слабо экзогенных регрессоров, соответствующих характеристикам объекта $i$, $t_i = \min(t_{1i}, \ldots , t_{mi}, t_{ci})$, где $t_{ki}$ --- момент перехода в $k$-ое состояние, а $t_{ci}$ --- момент цензурирования, и $d_{ji} \equiv \textbf{1}(t_{ji} = t_i)$, $j = 1, \ldots , m, c$ обозначают дамми переменные, равные единице при $t_{ji} = t_i$. Поскольку мы наблюдаем только $t_{ji}$, остальные длительности подразумеваются латентными.

Цензурирование можно воспринимать как конкурирующий риск, так как оно применяется к объектам в соответствии с распределением вероятностей. В данной главе предполагается, что переменная цензурирования независима от $(t_1, \ldots , t_m)$.

Ненаблюдаемые характеристики объекта $i$ учитываются в виде ненаблюдаемой гетерогенности с параметром $\nu$. Если $\nu$ варьируется с причинами перехода, то параметр записывается как $\nu_j$, $j = 1, \ldots , m$.

        \begin{center}Конкурирующие причины\end{center}
        \noindent
Стандартным примером конкурирующих рисков является смерть, наступившая по одной из конкурирующих причин. Представим, что некоторый индивид перенес операцию по пересадке почки и ожидает либо выздоровления, либо отказа новой почки, либо осложнения, например, со стороны печени. Переход в одно из этих состояний означает невозможность перехода в другие. То есть, в рамках модели с $m$ возможными событиями, каждому событию соответствует одна завершенная длительность и $m - 1$ цензурированных. Таким образом, в модели присутствуют <<конкурирующие риски>>, где конкуренция заключается в том, какой из рисков окажется причиной перехода.

Несмотря на то, что для большинства эмпирических приложений задача формулируется в дискретном времени, мы будем моделировать совместный риск в непрерывном, в соответствии с работой Меалли и Падни (1996). Также мы будем работать с одноразовыми (немногократными) событиями.

Модель предполагает совместное распределение \textbf{длительности события}, обозначенной как $\tau$, и \textbf{способа перехода} (\textit{exit route}) $r$, переменной, принимающей целые значения из множества $(1, 2, \ldots , m)$.

Для простоты мы не рассматриваем цензурированные наблюдения и предполагаем, что для каждого способа перехода, который может завершить событие, существует латентная переменная из набора $(t_1, \ldots , t_m)$, которая соответствует длительности события, при отсутствии прочих факторов риска, которые могут завершить событие раньше. Регрессоры для каждого способа перехода записываются как $\x_j$ $(j = 1, \ldots , m)$. Мы наблюдаем только наиболее короткую длительность $\tau$, равную
    \begin{align}
    \label{eq:19.1}
    \tau    &= \min(t_1, \ldots , t_m)\\
            &= \min_{j}(t_j), t_j > 0, \notag
    \end{align}
в то время как все остальные длительности цензурированы. Помимо типов перехода мы не рассматриваем никакие другие причины цензурирования. Тогда
    \begin{align}
    \label{eq:19.2}
    \Pr[\tau > t]   &= \Pr[t_1 > t, \ldots ,  t_m > t]\\
                    &= S_{\tau}(t), \notag
    \end{align}
что представляет собой функцию выживания. Если риски независимы, то
    \begin{align}
    \label{eq:19.3}
    \Pr [\tau > t]  &= \Pr [t_1 > t] \times \Pr [t_2 > t] \times \ldots  \times \Pr [t_m > t]
    \end{align}
Соответствующий способ перехода $r$ равен
    \begin{align}
    \label{eq:19.4}
    r = \arg \min_{j \in J}(t_j)
    \end{align}

Если $g_j(t)dt$ обозначает вероятность перехода по причине $j$ в промежутке $(t, t + dt)$, то общий коэффициент риска, соответствующий всем причинам равен
    $$\lambda_\tau(t) \equiv - d/dt \ln S_\tau(t) = \sum_{j=1}^m g_j(t).$$
В биостатистике данный показатель называется \textbf{общей интенсивностью смертности} (\textit{total force of mortality}) (Дэвид и Мошбергер, 1978). Если риски независимы, то коэффициент риска для определенной причины $j$ равен $\la_j(t) = g_j(t)$, что означает, что вероятность отказа по причине $j$ в промежутке $(t, t + dt)$ при условии дожития до момента $t$ одинакова, независимо от того, является ли $j$ одним из рисков или единственным риском.

Вероятность пережить риск % выживания при риске
$j$ в интервале $(T_1, T_2)$ при условии дожития до момента $T_1$ равна
    \begin{align}
    \label{eq:19.5}
    \int_{T_1}^{T_2} \lambda_j(t)dt     &= \int_{0}^{T_2} \lambda_j(t)dt - \int_{0}^{T_1} \lambda_j(t)dt \\
                                        &= \ln S(T_2) - \ln S(T_1) \notag \\
                                        &= - \ln \frac{\Pr [t_j > T_2]}{\Pr [t_j > T_1]}, \notag
    \end{align}
что эквивалентно
    \begin{align}
    \label{eq:19.6}
    \exp{\left( - \int_{T_1}^{T_2} \lambda_j(t)dt \right)} = \frac{\Pr [t_j > T_2]}{\Pr [t_j > T_1]}.
    \end{align}
Единица за вычетом левой части уравнения называется \textit{чистой вероятностью смерти по причине} $j$ (\textit{net probability of death from cause} $j$) в интервале $(T_1, T_2)$. Также с помощью выражения в (\ref{eq:19.6}) можно получить функцию правдоподобия.

        \begin{center}Независимые риски\end{center}
        \noindent
На данном этапе мы можем представить регрессоры, влияющие на коэффициент риска. Мы будем предполагать \textbf{независимость рисков} (в противопоставление их коррелированности) и рассмотрим распределение длительности $t_j$. Коэффициент риска отказа типа $j$ записывается как
    \begin{align}
    \lambda_j(t_j|\x_j) = \lim_{\Delta t_j \rightarrow 0} \frac{\Pr [t_j \le T \le t_j + \Delta t, | T \ge t_j, \x_j]}{\Delta t_j}, \notag
    \end{align}
а интегральный риск $\La_j(t_j|\x_j)$ отказа $j$-го типа определен как
    \begin{align}
    \Lambda_j(t_j|\x_j) = \int_{0}^{t_j} \lambda_j(s|\x_j)ds. \notag
    \end{align}
Тогда, используя отношение между функциями выживания и интегрального риска, получим плотность распределения длительности
    \begin{align}
    f_j(t_j|\x_j, \be_j)    &= \lambda_j(t_j|\x_j, \be_j) S_j(t_j|\x_j, \be_j), \notag \\
                            &= \lambda_j(t_j|\x_j, \be_j) \exp{[-\Lambda_j(t_j|\x_j, \be_j)]}. \notag
    \end{align}
Пусть $\x = [\x_1, \ldots , \x_m]'$ и $\be = [\be_1, \ldots , \be_m]'$, тогда совместная плотность $\tau$ и $r$ равна
    \begin{align}
    \label{eq:19.7}
    f_j(\tau,r|\x,\be)  &= f_r(\tau|\x_r,\be_r) \prod_{j \ne r} \exp{[- \Lambda_j(\tau |\x_j,\be_j)]} \\
                        &= \lambda_r(\tau|\x_r,\be_r) \exp{[-\Lambda_r(\tau|\x_r,\be_r)]} \notag \\
                        &  \times \prod_{j \ne r} \exp{[-\Lambda_j(\tau|\x_j,\be_j)]} \notag \\
                        &= \lambda_r(\tau|\x_r,\be_r) \prod_{j = 1}^{m} \exp{[-\Lambda_j(\tau|\x_j,\be_j)]}, \notag
    \end{align}
где результат в первой строке следует из произведения условной и маргинальной вероятностей, а второй элемент в правой части выражения равен произведению вероятностей выживания для всех способов перехода кроме $r$, при условии что риски независимы.

Из уравнения (\ref{eq:19.7}) следует, что
    \begin{align}
    \label{eq:19.8}
    & \lambda_j(\tau|\x_j,\be_j) \exp{\left[\sum_{j=1}^{m} - \Lambda_j(\tau|\x_j,\be_j) \right]} \\
    &= \lambda_j(\tau|\x_j,\be_j) \exp{[- \Lambda^a (\tau|\x,\be)]}, \notag
    \end{align}
где $\Lambda^a (\tau|\x,\be) = \sum_{j=1}^{m} \Lambda_j(\tau|\x_j,\be_j)$ представляет агрегированный, или общий, интегральный риск. Другими словами, общий риск перехода из начального состояния равен сумме рисков отказа всех типов. Общая функция выживания равна
    \begin{align}
    \label{eq:19.9}
    S(t) = \exp{(-\Lambda^a (t))}
    \end{align}

Функция правдоподобия при условии независимости рисков представляет собой произведение элементов типа (\ref{eq:19.7}) по всем наблюдениям. Зная спецификацию соответствующих функциональных форм, можно получить конкретное выражение для правдоподобия. Такие представленные ранее вопросы, как гибкость функциональной формы и ненаблюдаемая гетерогенность, остаются актуальными в контексте CRM. Мы не будем продолжать обсуждение в общем виде, а рассмотрим вместо этого определенные функциональные формы. Одной из популярных спецификаций в литературе является форма пропорциональных рисков, которая и будет представлена далее.


\subsection{CRM c пропорциональными рисками}\label{sec:19.2.2}

\noindent
Идея заключается в том, чтобы получить совместную плотность длительности события и причины перехода. Сделать это можно с помощью агрегирования интегрального риска по типам перехода.

Рассмотрим модели PH в форме
    $$\lambda_j(t;\x) = \lambda_{0j}(t) \exp{[\x'(t) \be_j]}, j = 1,\ldots ,m,$$
где базовый риск $\la_{0j}$ и $\be_j$ соответствуют риску типа $j$, а $t_{j1} < \ldots  < t_{jk_j}$ --- $k_j$ упорядоченным отказам типа $j$. Например, для $m = 2$ $k_1$ соответствует количеству объектов, зарегистрировавших отказ типа 1, а $k_2$ --- числу объектов с отказом типа 2.

Функция правдоподобия \textbf{CRM Кокса} в таком случае равна

    \begin{align}
    \label{eq:19.10}
    \mathrm{L}(\be_1,\ldots ,\be_m) &= \prod_{j=1}^{m} \prod_{i=1}^{k_j} \frac{\exp{[\x'_{ji}(t_{ji}) \be_j]}}{\sum_{l \in R(t_{ji})} \exp{[\x'_l(t_{ji})\be_j]}}, \\
    &= \prod_{j=1}^{m} \mathrm{L}_j(\be_j), \notag
    \end{align}
где
    \begin{align}
    \label{eq:19.11}
    \mathrm{L}_j(\be_j) = \prod_{i=1}^{k_j} \frac{\exp{[\x'_{ji}(t_{ji}) \be_j]}}{\sum_{l \in R(t_{ji})} \exp{[\x'_l(t_{ji})\be_j]}}.
    \end{align}
    
Обратим внимание на следующие четыре особенности данной функции правдоподобия:
(1) $\mathrm{L}_j(\be_j)$ представляет собой частное правдоподобие, полученное в разделе 17.8.2. % \ref{sec:17.8.2} # UNCOMMENT AFTER 17 CHAPTER
Функция базового риска отсутствует, и в отношении асимптотического распределения применяются результаты, также полученные ранее.
(2) Максимизация $\mathrm{L}(\be_1, \ldots , \be_m)$ возможна с помощью максимизации каждого элемента $\mathrm{L}_j(\be_j)$ в отдельности, при условии, что риски независимы. То есть, совместная максимизация и максимизация по отдельности дают эквивалентные результаты.
Оценивание и сравнение $\be_j$ осуществляется с помощью стандартных асимптотических методов, применимых к каждому элементу из $m$ элементов правдоподобия.
(3) Идеи, представленные в разделах
17.7 % \ref{sec:17.7} # UNCOMMENT AFTER 17 CHAPTER
и 17.8, % \ref{sec:17.8} # UNCOMMENT AFTER 17 CHAPTER
могут быть обобщены. Если формулировка в дискретном виде (в виде дамми переменных) применяется для каждого типа риска, то идентифицируемые компоненты функции риска можно оценить для каждого типа совместно с $\be_j$.
(4) Ненаблюдаемая гетерогенность может быть представлена согласно модели пропорциональных рисков с одноразовым событием и двумя состояниями в главе 18. % \ref{ch:18} # UNCOMMENT IN THE END OF THE BOOK

\subsection{Идентификация CRM}\label{sec:19.2.3}

\noindent
Кокс (1962a) и Тсиатсис (1975) показали, что при отсутствии регрессоров модель CRM неидентифицируема. Другими словами, это означает, что любая модель CRM с зависимыми рисками по внешнему виду % с точки зрения наблюдателя
эквивалентна CRM с независимыми рисками. Хекман и Оноре (1989) показали, что CRM со смешанной PH формой с регрессорами идентифицируема  при определенных предпосылках. Описание основополагающих предпосылок предлагает Ван ден Берг (2001, стр. 3438 --- 3441). В частности, помимо допущений, представленных в главе 17, % \ref{ch:17} # UNCOMMENT IN THE END OF THE BOOK
необходимы дополнительные, такие как <<достаточная изменчивость>> регрессоров или отсутствие между ними идеальной коллинеарности. Также мы вправе требовать, чтобы базовые риски для различных типов не были коррелированы идеально.


\subsection{Интерпретация коэффициентов регрессии}\label{sec:19.2.4}

\noindent
В модели CRM с пропорциональными рисками эффект регрессоров на коэффициент риска аналогичен модели PH в главе 17, % \ref{ch:17} # UNCOMMENT IN THE END OF THE BOOK
однако при непосредственной интерпретации коэффициентов регрессии мы сталкиваемся с той же проблемой, что и в разделе 15.4.3 % \ref{sec:15.4.3} # UNCOMMENT AFTER 15 CH
при интерпретации логит модели множественного выбора. % multinomial logit model (Магнус, стр. 332)

Кроме стандартных выводов нас также может интересовать, как изменение регрессоров повлияет на вероятность перехода способом $r$, что подразумевает более сложные расчеты. Чтобы показать это, запишем непосредственно вероятность перехода способом $r$
    \begin{align}\label{eq:19.12}
    \Pr [r|\tau,\x,\be] = \frac{\lambda_r(\tau|\x_r,\be_r)}{\sum^{m}_{j=1} \lambda_j(\tau|\x_j,\be_j)}.
    \end{align}
Поскольку регрессоры присутствуют и в числителе, и в знаменателе, где знаменатель равен сумме всех рисков, итоговый знак частной производной $\pa\Pr[r|\tau, \x, \be] / \pa x_{rk}$ будет зависеть от параметров модели. Следовательно, знак $\beta_{rk}$ будет отличаться от направления частной производной. (Ситуация полностью аналогична описанной в 15 % \ref{ch:15} # UNCOMMENT IN THE END OF THE BOOK
главе по моделям множественного выбора.) Тем не менее, если конкурирующие риски пропорционального типа, то справедлив следующий результат (Томас, 1996, стр. 31). Если $\beta_{rk} > \beta_{jk}$, $\forall j \ne r$, то $\pa \Pr[r | \tau, \x, \be] / \pa x_{rk} > 0$. Иначе говоря, увеличение $x_k$ приведет к увеличению условной вероятности перехода способом $r$, если оценка данного коэффициента в функции риска $\la_r(\cdot)$ выше, чем соответствующий коэффициент в любой другой функции риска.


\subsection{CRM при наличии ненаблюдаемой гетерогенности}\label{sec:19.2.5}

\noindent
Если конкурирующие риски пропорционального типа, то метод, представленный в предыдущей главе можно обобщить на случай ненаблюдаемой гетерогенности. Общая спецификация ненаблюдаемой гетерогенности допускает наличие случайной компоненты, соответствующей конкретному состоянию. Пусть $\bm{\nu} = (\nu_1, \ldots , \nu_m)$ обозначает вектор параметров ненаблюдаемой гетерогенности с функцией совместного распределения $\bm{G}(\bm{\nu})$, тогда
    \begin{align}
    f_j(\tau,r| \x, \be, \bnu) &= \lambda_j(\tau|\x_j,\be_j,\nu_j) \exp{\left[ \sum^{m}_{j=1} -\Lambda_j(\tau|\x_j,\be_j,\nu_j) \right]} \notag \\
    &= \lambda_j(\x_j,\be_j)\nu_j \exp{\left[ \sum^{m}_{j=1} -\Lambda_j(\tau|\x_j,\be_j)\nu_j \right]}, \notag
    \end{align}
где во второй строке используется предпосылка о мультипликативной гетерогенности.

Данный пример представляет собой модель конкурирующих рисков со случайными эффектами для состояний. Маргинальное по отношению к $\bm{\nu}$ распределение может быть получено за счет исключения $\bm{\nu}$ с помощью интегрирования
    $$f_j(\tau,r|\x,\be) = \int \ldots  \int \lambda_j(\tau|\x_j,\be_j)\nu_j \exp{\left[ \sum^{m}_{j=1} -\Lambda_j(\tau|\x_j,\be_j)\nu_j \right]} d\bm{G}(\bnu),$$
что подразумевает запись $m$-кратного интеграла.

Очевидно, не каждая такая модель может быть рассчитана. Один из случаев, когда вычисления реализуемы, основывается на том, что $m$ параметров $\bm{\nu}$ являются независимыми гамма распределенными случайными величинами. Тогда $m$-кратный интеграл можно представить в виде произведения $m$ интегралов. Примером является модель, предполагающая смесь Вейбулла-Гамма для каждой функции риска при соответствующей причине, поскольку в таком случае конкурирующие риски независимы.

Если допустить коррелированность параметров $\bm{\nu}$, мы получим более интересный случай с зависимыми конкурирующими рисками. На самом деле, такой <<прием>> специально применяется для моделирования зависимости между конкурирующими рисками. Например, если $\bm{\nu}$ имеет многомерное лог-нормальное распределение $\begin{bmatrix} \ln\nu_1 & \ldots  & \ln\nu_m \end{bmatrix}' \sim \mathcal{N}[\bm{0}, \bm{\Sigma}]$, то зависимость реализуется через параметр гетерогенности. При этом, оценка максимума правдоподобия связана с гораздо более сложными вычислениями, поскольку представленный $m$-кратный интеграл не имеет решения в аналитическом виде. В таком случае, применяется интегрирование по методу Монте Карло. Если $m$ равно двум или трем, как во многих прикладных задачах, то вычисления, в принципе, реализуемы, хотя и нетривиальны. Уменьшить размерность интеграла можно с помощью ограничения на структуру матрицы ковариаций. Например, мы можем определить факторную структуру для каждого параметра $\nu_j$, специфицировав ее как линейную функцию от двух независимо идентично распределенных ($iid$) случайных величин с неизвестными весами (факторной нагрузкой, \textit{factor loadings}).


\subsection{CRM с зависимыми конкурирующими рисками}\label{sec:19.2.6}

\noindent
Модель CRM с независимыми рисками имеет значительные вычислительные преимущества перед моделью, где зависимость выражена с помощью параметров гетерогенности, коррелированных между конкурирующими рисками. С другой стороны, такая модель содержит ценную дополнительную информацию о структуре гетерогенности, например, о параметрах связи (\textit{association parameter}). Тем не менее, остается вопрос практического характера --- насколько ограничительной должна быть спецификация коррелированной гетерогенности? В качестве ответа мы рассмотрим случай с двумя параметрами, используя постановку, аналогичную (17.20): % \ref{eq:17.20} # UNCOMMENT AFTER 17 CH
    $$\ln\left[ \int \lambda_1(u)du \right] = -\x'\beta_1 - \nu_1 + \e,$$
    $$\ln\left[ \int \lambda_2(u)du \right] = -\x'\beta_2 - \nu_2 + \e.$$
Предпосылка $\nu_1 = \nu_2 = \nu$ означает, что в обеих моделях риска присутствует одинаковый параметр гетерогенности. Идея заключается в том, что на оба события могут влиять одинаковые параметры, но их масштаб может различаться. Такая ситуация относится к идеально коррелированной гетерогенности между обоими рисками. При меньших ограничениях мы можем предположить коррелированность между $\nu_1$ и $\nu_2$ и оценить параметр связи. Оба предположения можно рассматривать как модели с одним и двумя факторами гетерогенности, соответственно. Обоснованность того или иного подхода зависит от конкретной задачи. Например, если оба риска относятся к одно и тому же объекту, $\nu_1$ и $\nu_2$ можно воспринимать как два индивидуальных фактора, и тогда двухфакторная модель привлекательнее. При этом, есть некоторые теоретические и экспериментальные доказательства, что применение однофакторной модели, когда верна двухфакторная, приводит к значительным искажениям результатов (Линдебум и Ван ден Берг, 1994).




\section{Совместные распределения длительностей}\label{sec:19.3}

\noindent
В данном разделе мы рассмотрим случай с параллельными зависимыми событиями. Предполагается, что моменты выживания непрерывны. Описание представлено в общем виде, где для простоты мы будем рассматривать только нецензурированные длительности, обладающие параметрическими распределениями.

Прикладной анализ логично начать анализ с определенной функциональной формы, описывающей совместные функции выживания или функции плотности. Доступны ли <<стандартные>> функциональные формы, и есть ли общий метод записи моделей, представленных в предыдущих главах, в многомерном виде? Ответы на эти вопросы и будут представлены далее.


\subsection{Обобщение концепции выживаемости на многомерный случай}\label{sec:19.3.1}

\noindent
Имеет смысл начать анализ с обобщения представленных ранее базовых понятий на многомерный случай.

Многомерная функция выживания $S(\bt)$ задается как
    \begin{align}\label{eq:19.13}
    S(\bt)    &= S(t_1,\ldots ,t_q) \\
            &= \Pr [T_1>t_1,\ldots ,T_q>t_q], \notag
    \end{align}
где $T_1, \ldots , T_q$ соответствуют $q$ моментам выживания для одномерных функций выживания $S_j(t_j)$. По определению,
    \begin{align}\label{eq:19.14}
    S_j(t_j)    &= \Pr [T_j>t_j] \\
                &= S(T_1 \ge 0,\ldots , T_j \ge t_j, \ldots , T_q \ge 0) \notag \\
                &= S(0,\ldots ,t_j, \ldots ,0). \notag
    \end{align}
В отличие от одномерного случая
    $$ S(t_1, \ldots , t_q) \ne 1-F(t_1, \ldots , t_q).$$
Например, $S(t_1,t_2) = 1-F(t_1)-F(t_2)+F(t_1,t_2)$.

Совместная плотность $(t_1, \ldots , t_q)$ обозначается как $f(t_1, \ldots , t_q)$. Если $F(t_1, \ldots , t_q)$ непрерывно, то
    \begin{align}\label{eq:19.15}
    f(t_1, \ldots , t_q) = (-1)^q \frac{\partial^q F(t_1,\ldots  , t_q)}{\partial t_1\ldots \partial t_q}.
    \end{align}
\textbf{Совместная функция риска} $\la(t_1, \ldots , t_q)$ записывается аналогично одномерному случаю:
    \begin{align}\label{eq:19.16}
    \lambda(t_1, \ldots , t_q) = \frac{f(t_1, \ldots , t_q)}{S(t_1, \ldots , t_q)}.
    \end{align}
Совместный интегральный риск $\La(t_1, \ldots , t_q)$ представляет собой $q$-кратный интеграл от $\la(t_1, \ldots , t_q)$. Однако такого же простого как и ранее соотношения между $S(t_1, \ldots , t_q)$ и $\La(t_1, \ldots , t_q)$ теперь нет.

Ключевой вопрос заключается в том, возможно ли вывести совместные функции выживания на основе данных определений. Клейтон и Куцик (1985) предложили двумерную модель, которая иллюстрирует представленные понятия. Согласно авторам, сначала следует определить функцию \textbf{отношения перекрестных рисков} (\textit{cross-hazard ratio}), заданную как функцию от двух условных функций риска для $t_1$ при $T_2 = t_2$ и $T_2 \ge t_2$. Так мы получим нелинейное уравнение в частных производных второго порядка, решение которого и даст нам совместную функцию выживания. Детали можно найти в самой статье; при этом следует заметить, что для размерности больше двух применение такого подхода может быть затруднено.


\subsection{Двумерные распределения, основанные на маргинальных}\label{sec:19.3.2}
% примечание: надеюсь, здесь не возникнет ассоциации с бомжами? Такая "калька", обозначающая маргинальные распределения, присутствует в том числе на википедии:
% \url{http://ru.wikipedia.org/wiki/%D0%9A%D0%BE%D0%BF%D1%83%D0%BB%D0%B0}
% нужно ли заменить?

\noindent
Данный раздел представляет краткое описание некоторых методов построения двумерных моделей времени жизни, которые основываются на предположениях о маргинальных функциях выживания. Такие методы будут полезны в том случае, если мы имеем представление о маргинальных распределениях и хотим использовать их как составные элементы для построения модели. Разумеется, выбор составных элементов накладывает определенные ограничения на форму итогового совместного распределения.

Один из подходов, согласно работе Маршалла и Олкина (1990), предполагает, что модель с мультипликативной ненаблюдаемой гетерогенностью, присутствующей в обоих маргинальных распределения, устроена следующим образом. Пусть $f_i(t_i|\x_i, \nu)$, $i = 1, 2$ обозначает маргинальные распределения $t_1$, $t_2$ при данных наборах регрессоров $\x_1$, $\x_2$, где $\nu$ является общим параметром ненаблюдаемой гетерогенности и образует связь между обоими рисками; это является единственной причиной корреляции $t_1$ и $t_2$. В анализе выживаемости такую модель иногда называют моделью <<распределенной уязвимости>> (<<shared frailty>>). Пусть $\nu$, $\nu > 0$ имеет распределение с функцией плотности $g(\nu)$. Тогда двумерное распределение $t_1$, $t_2$ формально записывается как
    \begin{align}\label{eq:19.17}
    f(t_1,t_2|\x_1,\x_2) = \int^{\infty}_{0} f_1(t_1|\x_1,\nu) f_2(t_2|\x_2,\nu) g(\nu) d\nu,
    \end{align}
где параметры распределения пропущены для удобства записи.

Двумерное распределение, полученное в виде \textbf{смеси}, может как иметь, так и не иметь решения в аналитическом виде. Поэтому без определенной параметрической спецификации трудно сказать, насколько оно применимо для расчетов. Более того, итоговое совместное распределение подразумевает положительную корреляцию между $t_1$ и $t_2$, что не всегда так.

Такой подход применим к любому типу данных и может быть адаптирован для анализа выживаемости посредством замены маргинальных распределений на \textbf{маргинальные функции выживания} и выводом \textbf{совместной функции выживания} с помощью интегрирования по $\nu$; то есть,
    \begin{align}\label{eq:19.18}
    S(t_1, t_2|x_1,x_2) = \int^{\infty}_{0} S_1(t_1|x_1,\nu) S_2(t_2|x_2,\nu) g(\nu) d\nu.
    \end{align}
Одним из примеров применения этой идеи является работа Клейтона и Куцика (1985), где авторы использовали данную формулировку, чтобы получить \textbf{двумерную функцию выживания} для маргинальных пропорциональных рисков с гамма гетерогенностью.

Как показано, применение данного подхода связано с определенными ограничениями, одним из которых является предпосылка об однофакторной ненаблюдаемой гетерогенности. Как правило, это ограничение легко обойти, заменив $\nu$ на вектор двух коррелированных элементов $\begin{pmatrix} \nu_1 & \nu_2 \end{pmatrix}$, $\nu_1 > 0$, $\nu_2 > 0$, каждый из которых соответствует функции выживания с совместной вероятностью распределения $g(\nu_1, \nu_2)$. Тогда
    \begin{align}\label{eq:19.19}
    S(t_1, t_2|\x_1, \x_2) = \int^{\infty}_{0} \int^{\infty}_{0} S_1(t_1|\x_1, \nu_1) S_2(t_2|\x_2, \nu_2) g(\nu_1, \nu_2) d\nu_1 d\nu_2.
    \end{align}
Для определенности предположим, что
    \begin{align}
    \nu_1           &= \omega_{11}\e_1 + \omega_{12}\e_2 \notag \\
    \nu_2           &= \omega_{21}\e_1 + \omega_{22}\e_2 \notag \\
    \varepsilon_j   &\sim \mathcal G[1,\sigma^{2}_{j}] , j=1,2, \notag
    \end{align}
где $\{\om_{ij}, i, j = 1, 2\}$ --- неизвестные параметры (веса), часто называемые \textbf{<<факторной нагрузкой>>} (\textit{<<factor loadings>>}). Такая запись означает, что компоненты гетерогенности $(\nu_1, \nu_2)$ являются коррелированными линейными комбинациями независимо идентично распределенных ($iid$) случайных компонент $\e_1$ и $\e_2$, если факторная нагрузка отличается от нуля. Прочие распространенные предпосылки, связанные с ограничениями, заключаются в том, что (i) $(\ln\e_1, \ln\e_2)$ подчиняются стандартному двумерному нормальному распределению или что (ii) $\nu_1$, $\nu_2$ имеют дискретное распределение (конечной смеси). Таким образом, модель (\ref{eq:19.19}) имеет двумерную форму смеси. Также необходимы дополнительные ограничения идентифицируемости (например, нормализация $\om_{11} = 1$). Коэффициент корреляции Пирсона между $\nu_1$ и $\nu_2$, равный $\Cov[\nu_1, \nu_2]/[\V[\nu_1]\V[\nu_2]]^{1/2}$, зависит от $\{\om_{ij}, \sis_j, i, j = 1, 2\}$. Легко убедиться, что его значение не находится в привычном интервале от $-1$ до $+1$. (Заметим также, что соответствующий параметр связи для моментов отказа отличается и равен $\Cov[t_1, t_2]/[\V[t_1]\V[t_2]]^{1/2}$.) Ван ден Берг (1997) вывел точные границы для $\Cor[t_1, t_2|\x]$. В частности, для смешанной модели пропорциональных рисков с постоянным базовым риском справедливо неравенство $-1/3 < \Cor[t_1, t_2|\x] < 1/2$. Также он показал, что эти границы не зависят ни от регрессоров $x$, ни от распределения гетерогенности. Однако если базовый риск меняется, границы корреляции зависят от его изменений.

Спецификация компонент в виде факторной нагрузки имеет вычислительные преимущества по сравнению с той, когда компоненты ненаблюдаемой гетерогенности представлены в произвольном виде. Несмотря на то, что однофакторная модель может быть чересчур ограничительной, модели без ограничений допускают наличие интеграла любой кратности. Как следствие, совместная функция выживания, являющаяся результатом интегрирования, может не иметь выражения в аналитическом виде, и в  таком случае применяется подход на основе симуляций. На момент написания этой книги для оценки подобной модели потребовалось бы выйти за рамки стандартных статистических пакетов.

Спецификация в виде факторной нагрузки накладывает определенные ограничения на модель (Ван ден Берг, 2001; Линдебум и Ван ден Берг, 1994). Например, если одно из маргинальных распределений не указывает на наличие ненаблюдаемой гетерогенности, то $\Cov[\nu_1, \nu_2]$ должна равняться нулю; если $\V[\nu_1] > 0$ и $\V[\nu_2] > 0$, то $\Cov[\nu_1, \nu_2] \ne 0$. Тогда, если $\Cov[\nu_1, \nu_2] = 0$, то одно из маргинальных распределений не содержит ненаблюдаемой гетерогенности.

С практической точки зрения многомерная функция выживания должна быть гибкой. Помимо представленного подхода, связанного с некоторыми ограничениями, существуют альтернативные методы. Один такой перспективный подход основан на анализе копула-функций. Хугаард (2000, стр. 435 --- 437) предлагает описание основ в контексте анализа выживаемости.


\subsection{Подход на основе копула-функций}\label{sec:19.3.3}

\noindent
Копулы, изначально представленные Скларом в 1959 г. на французском (см. также Склар, 1973), были предложены как эффективный метод получения совместных распределений на основе маргинальных, что в особенности удобно, если мы работаем с распределениями, отличными от нормального. Данная идея нашла широкое применение в анализе выживаемости, но может быть использована для работы с совместными распределениями любого набора дискретных, непрерывных или смешанных случайных величин.

Изученные подходы (например, \textbf{метод Маршалла--Олкина}) определяют зависимость между переменными с помощью ненаблюдаемой гетерогенности, что вполне разумно, поскольку трудно представить, что наблюдаемые регрессоры могут отражать все возможные аспекты экономических событий.


        \begin{center}Свойства копулы\end{center}
        \noindent
Для определения копулы рассмотрим набор зависимых равномерно распределенных случайных величин $U_1, \ldots , U_q$ в интервале $[0, 1]$. Взаимосвязь описывается с помощью функции совместного распределения
    \begin{align}\label{eq:19.20}
    C(u_1, \ldots  , u_q) = \Pr[U_1 \le u_1, \ldots  , U_q \le u_q],
    \end{align}
где функция $C(\cdot)$ обозначает \textbf{копулу}, а $u_j$ --- конкретную реализацию $U_j$, $j = 1, \ldots , q$.

Правая часть уравнения обозначает совместную функцию распределения, $F(\cdot)$, где $q$ аргументов копулы можно заменить на $q$ маргинальных функций распределения $F_1(\cdot), \ldots , F_q(\cdot)$. Тогда совместную функцию распределения можно определить как
    $$C(F_1(u_1), \ldots  , F_q(u_q)) = F(u_1, \ldots , u_q).$$
Чтобы получить совместное распределение на основе копула-функции, необходимо объединить набор маргинальных распределений. Поскольку копула представляет собой функциональную форму для объединения выбранных функций, то для различных вариантов $C(\cdot)$ получаются различные совместные распределения. Согласно \textbf{теореме Склара} (\textit{Sklar's theorem}), любое многомерное распределение может быть представлено в виде копулы (\ref{eq:19.20}); более того, если маргинальные распределения непрерывны, то копула определена единственным образом.

В отношении многомерной функции выживания теорема Склара утверждает, что многомерная функция выживания $S(t_1, \ldots , t_q)$ размерности $q$ может быть представлена в виде соответствующей копулы $C(S_1(t_1), \ldots , S_q(t_q))$.

В случае, если $q = 2$, уравнения выглядят следующим образом
    \begin{align}
    F(t_1, t_2)     &= \Pr[T_1 \le t_1, T_2 \le t_2] \notag \\
                    &= 1 - \Pr[T_1 > t_1] - \Pr[T_2 > t_2] + \Pr[T_1>t_1, T_2>t_2] \notag
    \end{align}
и
    \begin{align}
    S(t_1, t_2)     &= \Pr[T_1>t_1, T_2>t_2] \notag \\
                    &= 1-F(t_1) - F(t_2) + F(t_1,t_2) \notag \\
                    &= S_1(t_1) + S_2(t_2) - 1 + C(1-S_1(t_1), 1 - S_2(t_2)), \notag
    \end{align}
где $C(\cdot)$ называется \textbf{копулой выживания} (\textit{survival copula}). Заметим, что здесь $S(t_1, t_2)$ является функцией только от маргинальных распределений.

Копулы обладают свойством симметричности, которое позволяет работать как с копулами, так и с копулами выживания (Нелсен, 1999). Джо (1997) определяет двумерную копулу $C(u, v)$, соответствующую распределению $F(\cdot)$, как двумерную функцию распределения вероятности на единичном пространстве $[0, 1]^2$ с равномерными маргинальными распределениями, определенными на $[0, 1]$. Для всех $(u, v) \in [0, 1]$, $C(u, 0) = C(0, v)$, $C(u, 1) = u$ и $C(1,v) = v$. Заменив $u$ на маргинальную функцию выживания $S(t_1)$, а $v$ --- на $S(t_2)$, получим свойства копулы выживания. В данном случае теорема Склара показывает, что существует функция-копула $C$, такая что
    \begin{align}\label{eq:19.21}
    F(u,v) = C(F_u(u), F_v(v)),
    \end{align}
где $F(u, v) = \Pr[U < u, V < v]$ обозначает двумерную функцию распределения случайных величин $U$ и $V$, а $F_u(u)$ и $F_v(v)$ являются соответствующими маргинальными распределениями.

Если $F$ непрерывна и для одномерных маргинальных распределений существуют квантильные функции $F_{u}^{-1}$ и $F_{v}^{-1}$, то копула (\ref{eq:19.21}) может быть представлена как
    $$C(u_1,u_2) = F(F^{-1}_u(u), F^{-1}_v(v)).$$

Подход на основе копула-функций требует спецификации не только маргинальных распределений каждой случайной величины, но и соответствующей функции (копулы), которая их объединяет. Копула-функция может быть задана так, чтобы измерять зависимость между маргинальными распределениями. Если зависимость не найдена, предполагается, что оба распределения независимы, и оценивание можно производить по отдельности. Если же зависимость присутствует, то оценки можно улучшить с помощью построения совместного распределения на основе копула-функции. Так как копула позволяет выявлять структуру зависимости независимо от формы маргинальных распределений, данный подход может быть весьма полезен для моделирования взаимосвязанных переменных. Степень зависимости, представленную копула-функцией, можно оценить с помощью \textbf{границ Фреше}. Несмотря на очевидные отличия, данный подход имеет фундаментальное сходство с подходом из раздела \ref{sec:19.3.2}, основанного на распределениях смеси, поскольку оба используют маргинальные распределения для получения двумерной функции выживания (19.19).

Рассмотрим пример с $q$ условно независимыми длительностями $(T_1, \ldots , T_q)$ при общей неучтенной гетерогенности $\nu$; для простоты регрессоры исключены. Тогда условная совместная функция выживания равна
    \begin{align}
    \Pr[T_1>t_1, \ldots  , T_q>t_q|v]   &= \Pr[T_1>t_1|v] \times \ldots  \times \Pr[T_q>t_q|v] \notag \\
                                    &= S_1[(t_1)|v]\ldots S_q[(t_q)|v], \notag
    \end{align}
и многомерная функция выживания записывается как
    \begin{align}\label{eq:19.22}
    \Pr[T_1>t_1, \ldots  , T_q>t_q] = \E_v[S_1(t_1)|v, \ldots  , S_q(t_q)|v].
    \end{align}


        \begin{center}Оценка зависимости\end{center} % Измерение?
        \noindent
Функциональная форма копула-функций как таковая не зависит от формы одномерных маргинальных распределений, однако зачастую включает параметр взаимосвязи, являющийся скалярной величиной. Для простоты мы будем рассматривать только двумерные копулы.

Многомерному распределению дискретных случайных величин может соответствовать несколько копула-функций (Джо, 1997, стр. 14). Однако это не является основной проблемой, которая заключается в аппроксимации неизвестного совместного распределения. Поэтому ключевой аспект моделирования состоит в выборе достаточно гибкой параметрической формы для копулы.

\textbf{Параметр взаимосвязи} не обязательно находится в пределах от 0 до 1 и поэтому не поддается интерпретации, в связи с чем его обычно преобразуют в более привычные \textbf{$\bm{\tau}$-Кендалла} и \textbf{$\bm{\rho}$-Спирмена}; см. Джо (1997). Швайцер и Вольф (1981) показали, что корреляция Спирмена может быть выражена исключительно в терминах копула-функции
    $$\rho(t_1,t_2) = 12 \int\int{\{C(u,v) - uv\}}dudv.$$

Рассмотрим любую двумерную совместную функцию распределения $F(t_1, t_2)$ с одномерными маргинальными распределениями $F_1(t_1)$ и $F_2(t_2)$. По определению, $0 \le F_1(t_1)$, $F_2(t_2) \le 1$. Тогда совместная функция распределения ограничена нижней и верхней границами Фреше, $F^-$ и $F^+$, заданными как
    \begin{align}
    F(t_1,t_2) \ge F^{-}(t_1,t_2) &\equiv \max[F_1(t_1) + F_2(t_2) - 1, 0], \notag \\
    F(t_1,t_2) \le F^{+}(t_1,t_2) &\equiv \min[F_1(t_1), F_2(t_2)]. \notag
    \end{align}
Поскольку копулы представляют собой совместные функции распределения, для них также можно найти \textbf{границы Фреше}, которые, в свою очередь, влияют на выбор конкретной копулы. Для каждой копулы существует область допустимых значений \textbf{параметра взаимосвязи} $\ttt$. При этом, по мере того, как $\ttt$ приближается к нижней (верхней) границе допустимых значений, копула приближается к нижней (верхней) границе Фреше. Однако параметрическая форма копулы может быть задана таким образом, что область допустимых значений будет исключать границы Фреше. По этой причине для определенного набора данных одна копула может подходить лучше, чем другая.



        \begin{center}Примеры\end{center}
        \noindent

В таблице \ref{tab:19.1} приведены примеры некоторых двумерных копул, используемых в литературе. Джо (1997) предлагает описание их свойств.

    \begin{table}[!htbp]\caption{\textit{Некоторые стандартные функции Копула}}\label{tab:19.1}
    \begin{center}
\begin{minipage}{12cm}
\begin{tabular}{lll}
\hline \hline
\textbf{Тип}                   &   \textbf{Функция}                         &\textbf{Область}\\
\textbf{копулы}                & $C(u, v)$                                  &\textbf{определения} $\bttt$\\
\hline
Произведение    
&$uv$                                                       
&na\footnote{na, не определена.} \\
FGMS\footnote{Копула Фарли-Гамбла-Моргенштерна, Farlie --- Gumble --- Morgenstern.}        
&$uv(1 + \ttt(1 - u)(1 - v))$                                       
&$-1 < \ttt < +1$ \\
Нормальная\footnote{$\Phi$ обозначает функцию двумерного нормального распределения.}  
&$\Phi[\Phi^{-1}(u) \Phi^{-1}(v);\ttt]$                             
&$-1 < \ttt < +1$\\
Клейтона        
&$(u^{-\ttt} + v^{-\ttt} - 1)^{-1/\ttt}$                           
&$\ttt \in (0, \infty)$\\
Франка         
&$-\ttt^{-1}\ln(\eta - (1 - e^{-\ttt u})(1 - e^{-\ttt v}))/\eta,$   
&$\ttt \in (-\infty, \infty)$\\
&$\eta = 1 - e^{-\ttt}$\\
\hline \hline
\end{tabular}
\end{minipage}
    \end{center}
    \end{table}

Нормальная копула и копула Франка включают границы Фреше в диапазон допустимых значений. Копула Клейтона принадлежит \textbf{семейству Архимедовых копул} и имеет вид $C(u, v) = \phi (\phi^{-1}(1 - u) + \phi^{-1}(1 - v))$; см. Смит (2003).

Предположим, что мы хотим смоделировать набор длительностей $\begin{pmatrix} t_1 & t_2 \end{pmatrix}$, используя копулу Клейтона. Тогда двумерное распределение, записанное в терминах маргинальных функций выживания $S(t_1)$ и $S(t_2)$, будет выглядеть как
    $$(S(t_1)^{-\theta} + S(t_2)^{-\theta} - 1)^{-1/\theta},$$
где мы предполагаем, что маргинальные функции выживания определены с точностью до неизвестных параметров. Как и ранее, эти функции могут учитывать зависимость от регрессоров и ненаблюдаемой гетерогенности, например, в виде модели пропорциональных рисков. Для оценивания применяется метод максимального правдоподобия, основанный на полученных двумерных копулах.

Такой подход также не лишен недостатков, в частности, следует отметить два из них. Во-первых, модель непросто обобщить на случай, когда размерность больше двух. Во-вторых, при выборе функциональной формы следует помнить о возможных ограничениях, например, что конкретная функциональная форма копулы может разрешать только положительную корреляцию между переменными.


        \begin{center}Функции правдоподобия на основе копула-функций\end{center}
        \noindent
Модель на основе копула-функций (в терминах функций распределения) оценивается в два этапа. На первом требуется выбрать копулу, а затем на ее основе выразить функцию правдоподобия (в терминах функций плотности распределения). Рассмотрим частный случай, двумерную модель с нецензурированными моментами отказа $(t_1, t_2)$. Определив $f_j(t_j) = \pa F_j(t_j) / \pa t_j$ и $C_j(F_1, F_2) / \pa t_j$ для $j = 1, 2$, а также $C_{12}(F_1, F_2) = \pa C(F_1, F_2) / \pa t_1 \pa t_2$, получим плотность вероятности
    \begin{align}\label{eq:19.23}
    f(t_1,t_2) = f_1(t_1) f_2(t_2) C_{12} (F_1(t_1), F_2(t_2)),
    \end{align}
где формула $f(t_1, t_2) = \pa^2 F(t_1, t_2) / \pa t_1 \pa t_2$ применяется для вывода функции правдоподобия. Если наблюдения цензурированы, требуется соответствующая поправка (Фрис и Вальдес, 1998, стр. 15 --- 16; Джорджес и др., 2001). 

Различные копулы соответствуют невложенным моделям, % DEF: эконометрические модели, каждая из которых не может быть получена путем наложения ограничений на параметры другой модели
выбор среди которых может быть основан на \textbf{функции правдоподобия со штрафом} (\textit{penalized log-likelihood}). % ALT: выбор среди которых может быть осуществлен

\section{Многократные события}\label{sec:19.4}

\noindent
Различия между параллельными и повторяющимися состояниями были упомянуты в начале главы. Параллельные состояния подразумевают параллельные события, такие как пребывание в безработном состоянии или наличие страхового полиса; повторяющиеся состояния подразумевают последовательные события, такие как первые, вторые и последующие роды. Термин <<многократные длительности>> относится к \textbf{промежуткам времени между повторами} одного и того же события. Совместное моделирование таких данных имеет сходство с совместным моделированием параллельных состояний, так как в обоих случаях используется концепция многомерного распределения. Однако в отличие от параллельных событий, последовательные могут порождать временную (динамическую) зависимость в рисках.

Рассмотрим несколько примеров повторяющихся событий. На рынке труда индивиды могут испытывать последовательность переходов между периодами работы и ее отсутствия. Например, молодые специалисты могут оставаться без работы не один раз. Ньюман и МакКуллах (1984) рассматривают распределение родов во времени в рамках модели рисков. В этом случае, при моделировании коэффициента риска нужно учитывать корреляцию между продолжительностями перерывов между родами. Триведи и Александер (1989) анализируют многократные длительности безработицы среди молодежи в Австралии. В литературе по фертильности ключевой интерес представляет количество времени между последовательными родами (Хекман, Хотц и Уолкер, 1985). Меалли и Падни (1996) исследуют положительную связь между стажем работы и пенсионным статусом на основе данных из обследования населения по выходу на пенсию в Великобритании. Энгл и Расселл (1998) анализируют временные ряды интервалов между последовательными операциями по определенным ценным бумагам на фондовом рынке. На основе анализа многократных длительностей Стивенс (1999) исследует устойчивость бедности в течение жизни индивидов.

Вышеперечисленные примеры имеют некоторые особенности. В частности, коэффициент риска наступления события может зависеть от предыдущего события, при условии, что предыдущее событие наступило. Во-вторых, важна сама форма зависимости. Длительность предыдущего события, или периода, может быть включена как переменная, объясняющая риск более позднего события; наступление предыдущего события может оказывать воздействие на базовый риск более позднего периода; и, наконец, ненаблюдаемая гетерогенность может демонстрировать автокорреляцию.
Каждая из этих особенностей затрагивает важные аспекты моделирования.

\textbf{Многократные события}, или \textbf{периоды}, порождают лонгитюдные, или панельные, данные, которые помогают разрешить проблему идентификации в отношении влияния динамической зависимости (\textit{the hand of past}) по сравнению с моделями с гетерогенностью в функции риска. При некоторых предпосылках такие данные упрощают учет гетерогенности и выводы в отношении временной зависимости.

Может показаться, что оценивать модели выживания с гетерогенностью и зависимостью довольно трудно. Однако данные по многократным событиям позволяют исследовать те аспекты, анализ которых возможен только при наличии панельных данных. Примерами являются зависимость от наступления события, лаговая зависимость от длительности и автокоррелированная гетерогенность. И лаговая зависимость от длительности, и зависимость от наступления события относятся к зависимости между вероятностью завершения текущего периода и количеством или длительностями предыдущих периодов. С учетом этого было бы неверно исследовать периоды по отдельности, игнорируя их взаимосвязь.

Один из возможных подходов к анализу многократных событий заключается в моделировании зависимости с помощью совместных функций выживания, представленных в предыдущих разделах. Другой подход основан на анализе панельных данных, где нижний индекс вместо времени обозначает номер повтора события. При этом, календарное время также может оказывать влияние. Зависимость событий затрагивает аспекты, которые будут рассмотрены в разделах 22.5 % \ref{sec:22.5} # UNCOMMENT AFTER 22 CH
и 23.6, % \ref{sec:23.6} # UNCOMMENT AFTER 23 CH
посвященных динамическим моделям панельных данных. Различие в обоих подходах возникает из-за возможного цензурирования, возникающего по причине истощения панели или того, что недавние периоды еще не завершены.




\subsection{Модель с двукратными событиями}\label{sec:19.4.1}

\noindent
Некоторые особенности моделей с многократными событиями можно проиллюстрировать на примере модели пропорциональных рисков с двумя периодами, или повторами события. Применительно к эконометрике такие модели представлены в работах Оноре (1993) и Хоровиц и Ли (2003).

Оноре (1993) рассматривает модель пропорциональных рисков вида
        \begin{align}\label{eq:19.24}
        \lambda_s(t|\x,v) = \lambda_{0,s}(t) \phi(\x,\beta)v, \hspace{0.5cm} s=1,2.
        \end{align}
где гетерогенность представлена мультипликативно, а базовый риск, в отличие от параметра гетерогенности, определен для каждого периода. Следовательно, $\nu$ представляет собой фиксированные, или постоянные, характеристики объектов, и такая модель называется моделью с \textbf{фиксированными эффектами}. При условиях, аналогичных требуемым для идентификации смешанной модели PH в главе 18, автор показывает, что модель идентифицируема. При этом, ни предпосылки о распределении $\nu$, ни наличие регрессоров не играют значительной роли.

Оноре также рассматривает другую модель, где предполагается, что гетерогенность определена для каждого периода в отдельности, $\nu_1$ и $\nu_2$, с совместной функцией плотности $g(\nu_1, \nu_2)$. Корреляция между $\nu_1$ и $\nu_2$ показывает автокорреляцию в гетерогенности. Такая модель является моделью со \textbf{случайными эффектами}. Совместная функция выживания $S(t_1, t_2|\x)$ выводится через формулу  (\ref{eq:19.19}) на основе смешиваемого распределения $g(\nu_1, \nu_2)$. Совместная функция выживания идентифицируема в том случае, если идентифицируема маргинальная функция выживания. По сути, условия идентифицирумеости точно такие же, как для модели PH.

Оноре анализирует случай \textbf{лаговой зависимости от длительности} во втором периоде, предполагая, что длительность первого периода $t_1$, воздействует на риск завершения второго периода мультипликативно. Он формулирует достаточные условия для идентифицируемости параметров в условной модели для второго периода при данном наборе регрессоров и $t_1$. При таких условиях, модель пропорциональных рисков с несколькими повторами имеет вид
        \begin{align}\label{eq:19.25}
        \lambda_1(t_1|\x_1,v_1) &= \lambda_{0,1}(t) \phi(\x_1,\beta_1)v_1, \\
        \lambda_1(t_2|\x_2,v_2) &= \lambda_{0,2}(t) \phi(\x_2,\beta_2)v_2, \notag
        \end{align}
где $\x^{a}_{2} = (\x_2, t_1)$ обозначает расширенный вектор регрессоров. Мы не будем обсуждать эти условия здесь. Заметим, что в модели присутствует проблема эндогенности, поскольку, если $\nu_1$ и $\nu_2$ коррелированы, то $t_1$ и $\nu_2$ не являются независимыми.

Наступление события в предыдущем периоде может не только сдвинуть функцию риска в следующем, но и изменить саму спецификацию за счет включения новых регрессоров. Например, безработный индивид может поступить на программу профессиональной подготовки, которая вполне вероятно повлияет на риск завершения следующего периода безработицы. Предположение о слабой экзогенности переменной, отвечающей за получение или повышение квалификации, ставит под вопрос идентификацию такой модели. Аргумент имеет место и для анализа модели с одним периодом (единственным переходом): Предпосылка о некоррелированности регрессоров и ненаблюдаемой гетерогенности вовсе не является безобидной.

В некоторых случаях требуется смоделировать не только многократные длительности для одного состояния, но и длительности для связанных (смежных) состояний. Например, при анализе двух состояний, безработицы и занятости, интерес заключается не только в том, насколько длительность предыдущего периода безработицы влияет на длительность текущего, но и в том, какой эффект оказывает период занятости на последующую вероятность найти работу (риск завершения безработицы). Кроме того, данные о смежных состояниях могут быть недоступны. Например, официальная статистика может предоставлять информацию об индивидах, получающих пособие, в то время как о тех, кто пособие не получает, неизвестно ничего.




\subsection{Общая модель с многократными событиями}\label{sec:19.4.2}

\noindent
Для иллюстрации возможных вычислительных трудностей, возникающих при оценивании моделей с многократными событиями, мы представим краткое описание модели Меалли и Падни (1996).

Пусть $\bm{\tau} = (\tau_1, \ldots , \tau_k)$ обозначает вектор завершенных длительностей размерности $k$, $r_{k-1}$ --- индекс начального состояния, а $r_k$ --- индекс конечного состояния. Предположим, что длительности между периодами независимы после учета возможной лаговой зависимости от длительности. Пусть $\la_j(\x_j, \be_j)$ обозначает функцию риска для каждого состояния, $\x = [\x_1, \ldots , \x_k]$ и $\be = [\be_1, \ldots , \be_k]$.

Тогда совместная плотность периодов и способов перехода задается как
        \begin{align}\label{eq:19.26}
        &f(\tau_1, r_1, \tau_2, r_2,\ldots , \tau_k|\x_1,\ldots , \x_k,r_0,\beta) \\
        &= f(\tau_1, r_1|\x_1,r_0;\beta)\ldots f(\tau_{k-1},r_{k-1}|\x_{k-1}, r_0, r_1,\ldots , r_{k-2},\beta) \notag \\
        &\times S(\tau_k|\x_k,r_0,r_1,\ldots , r_{k-1}, \beta) \notag \\
        &= \prod^{k-1}_{j=1} \lambda_{r_j}(\tau_j|\x_j,\beta_{r_j}) \exp{\left(-\sum^{k}_{l=1} \Lambda_0(\tau_l|\x_l,\beta) \right)}, \notag
        \end{align}
где применяются определения (17.4) % \ref{eq:17.4} # UNCOMMENT AFTER 17 CH
и (17.6); % \ref{eq:17.6} # UNCOMMENT AFTER 17 CH
также предполагается, что текущая $k$-ая длительность цензурирована (поскольку еще не была завершена).
Допускается, что набор регрессоров может включать лаговые длительности и некоторые регрессоры, которые варьируются между периодами. Данная формулировка сопоставима с формулировкой модели CRM с единственным периодом (\ref{eq:19.7}).

Меалли и Падни (1996) предлагают подробную модель, используя такую формулировку в качестве основы. Поскольку они допускают наличие ненаблюдаемой гетерогенности с более сложной структурой по сравнению с той, которая была рассмотрена в главе, их вычислительная процедура также требует более сложных методов оценивания. В частности, авторы используют метод имитационного максимального правдоподобия (см. раздел 12.4). % \ref{sec:12.4} # UNCOMMENT AFTER 12 Ch
% ТЕРМИН "имитационного максимального правдоподобия" должен совпадать!!!
% http://www.quantile.ru/08/08-AT.pdf




\section{Пример конкурирующих рисков: длительность безработицы}\label{sec:19.5}

\noindent
Ранее в главах 17 % \ref{ch:17} # UNCOMMENT AFTER 17 CH
и 18 % \ref{ch:18} # UNCOMMENT AFTER 18 CH
мы анализировали продолжительность периода безработицы, игнорируя все прочие состояния после перехода. В данной главе мы представим анализ конкурирующих рисков, используя данные из работы МакКолла (1996). Данные содержат три возможных состояния: полная занятость, частичная занятость и полная либо частичная занятость. Следовательно, мы можем убрать предпосылку о том, что функция риска не зависит от последующего состояния, и построить модель конкурирующих рисков, в которой независимые конкурирующие риски определяют длительность безработицы.

В базе данных МакКолла содержатся 1073, 339 и 574 перехода в каждое из вышеперечисленных состояний, соответственно. Поскольку нет четкой и понятной интерпретации третьего состояния, мы не будем обсуждать его детально. В таблицах \ref{tab:19.2} и \ref{tab:19.3} для каждого перехода представлены результаты оценивания четырех параметрических моделей времени жизни, экспоненциальной и Вейбулла, с обратной гауссовской гетерогенностью и без. Как и ранее, использовалось лишь ограниченное число переменных. Гамма гетерогенность не рассматривалась по причине нестабильности вычислений. Наконец, поскольку риски независимы, оценивание уравнений можно производить по отдельности.

    \begin{table}[!htbp]\caption{\textit{Длительность безработицы: }}\label{tab:19.2}
    \begin{center}
\begin{tabular}{lcccccc}
\hline \hline
&\multicolumn{3}{c}{\textbf{No heterogeneity}}&\multicolumn{3}{c}{\textbf{IF heterogeneity}}\\
\cmidrule(r){2-4}\cmidrule(r){5-7}
\textbf{Risk}                    &\textbf{Риск 1} &\textbf{Риск 2} &\textbf{Риск 3} &\textbf{Риск 1} &\textbf{Риск 2} &\textbf{Риск 3}\\
\textbf{Coefficient Transitions} &\textbf{1,073}  &\textbf{339}    &\textbf{574}    &\textbf{1,073}  &\textbf{339}    &\textbf{574}\\
\hline
RR                  &.472   &--.092 &--.600 &.504   &--.185 &--.562 \\
                    &(.601) &(.976) &(.725) &(.614) &(1.025)&(.744) \\
DR                  &--.575 &--.959 &1.122  &--.806 &--1.051&1.078 \\
                    &(.762) &(1.247)&(.901) &(.781) &(1.295)&(.921) \\
UI                  &--1.424&--1.047&--.966 &--1.544&--1.092&--.963 \\
                    &(.249) &(.524) &(.449) &(.258) &(.544) &(.456) \\
RRUI                &.966   &--.669 &--.432 &1.057  &--.742 &--.482 \\
                    &(.612) &(1.192)&(1.014)&(.627) &(1.23) &(1.033) \\
DRUI                &--.198 &1.987  &2.102  &--0.12 &2.18   &2.158 \\
                    &(1.019)&(1.727)&(1.303)&(1.041)&(1.788)&(1.323) \\
LNWAGE              &.351   &--.257 &.003   &.373   &--.321 &--.007 \\
                    &(.116) &(.179) &(.145) &(.118) &(.191) &(.147) \\
TENURE              &0      &.005   &--.047 &.0006  &.007   &-.047 \\
                    &(.006) &(.013) &(.012) &(.007) &(.014) &(.012) \\
$-\ln\textrm{L}$    &\multicolumn{3}{c}{5,693.63}&\multicolumn{3}{c}{5,687.64} \\
\hline \hline
\end{tabular}
    \end{center}
    \end{table}


\subsection{Оценки модели конкурирующих рисков}\label{sec:19.5.1}

\noindent
Попарное сравнение экспоненциальных моделей с гетерогенностью и без указывает на более высокое качество модели, учитывающей ненаблюдаемую гетерогенность, что аналогично результатам, представленным в разделе 18.8. % \ref{sec:18.8} # UNCOMMENT AFTER 18 CH
При этом, модель Вейбулла без учета гетерогенности лучше экспоненциальной, что соответствует росту логарифму правдоподобия с $-5,666$ до $-5,693$. Наилучшей является модель Вейбулла с обратной гауссовской гетерогенностью с логарифмом правдоподобия, равным $-5,543$. Однако это не означает, что эта модель позволяет осуществлять корректные выводы --- этот вопрос остается открытым. Перейдем к обсуждению результатов из таблицы \ref{tab:19.3}.

С учетом ненаблюдаемой гетерогенности в модели Вейбулла коэффициенты наклона во всех трех функциях риска существенно возрастают: с $1.29$ до $1.75$ для 1 риска и с $1.08$ до $1.65$ для 2 риска, что указывает на убывающую зависимость от длительности, или более крутой наклон коэффициента риска. Такие изменения вполне согласуются с теорией из раздела 18.5. % \ref{sec:18.5} # UNCOMMENT AFTER 18 CH
Ненаблюдаемая гетерогенность оказывает существенный эффект на величину страховки от безработицы (\textit{unemployment insurance}, UI) в модели Вейбулла, который еще более существенен в абсолютных величинах. Коэффициенты при RR, DR, RRUI и DRUI определены неоднозначно. Коэффициент при LNWAGE значителен и положителен в первой функции риска и отрицателен во второй. Это означает, что увеличение LNWAGE ускоряет переход из состояния безработицы для тех, кто ищет работу на полную ставку, но имеет отрицательный эффект для тех, кто находит работу на полставки. Таким образом, модель конкурирующих рисков позволяет оценить роль переменных в различных функциях риска.

\begin{sidewaystable}[!htbp]\caption{\textit{Длительность безработицы: }}\label{tab:19.3}
    \begin{center}
\begin{tabular}{lccccccccc}
\hline \hline
&\multicolumn{3}{c}{\textbf{No heterogeneity}}&\multicolumn{3}{c}{\textbf{IF heterogeneity}}&\multicolumn{3}{c}{\textbf{Модель Кокса}}\\
\cmidrule(r){2-4}\cmidrule(r){5-7}\cmidrule(r){8-10}
\textbf{Risk}                    &\textbf{Риск 1} &\textbf{Риск 2} &\textbf{Риск 3} &\textbf{Риск 1} &\textbf{Риск 2} &\textbf{Риск 3} &\textbf{Риск 1} &\textbf{Риск 2} &\textbf{Риск 3}\\
\textbf{Coefficient Transitions} &\textbf{1,073}  &\textbf{339}    &\textbf{574}    &\textbf{1,073}  &\textbf{339}    &\textbf{574} &\textbf{1,073}  &\textbf{339}    &\textbf{574}\\
\hline
RR                  &.448   &--.085 &--.694 &.736   &--.379 &--.432 &.522   &--.071 &--.469 \\
                    &(.638) &(.992) &(.763) &(.906) &(1.452)&(1.111)&(--.752)&(.951)&(.715) \\
DR                  &--.427 &--.938 &1.361  &--1.072&--1.689&1.167  &--.571 &--1.023&.875 \\
                    &(.809) &(1.279)&(.969) &(1.149)&(1.78) &(1.378)&(.721) &(1.193)&(.878) \\
UI                  &--1.496&--1.109&--1.097&--2.574&--2.063&--1.761&--1.317&--.906 &--.905 \\
                    &(.264) &(.527) &(.46)  &(.384) &(.747) &(.623) &(.237) &(.510) &(.444) \\
RRUI                &1.015  &--.616 &--.305 &1.734  &--.301 &--.515 &.882   &--.781 &--.539 \\
                    &(.646) &(1.204)&(1.047)&(.933) &(1.702)&(1.418)&(.582) &(1.166)&(1.002) \\
DRUI                &--.299 &1.973  &1.991  &--.06  &3.263  &3.669  &--.095 &2.031  &2.293 \\
                    &(1.065)&(1.757)&(1.37) &(1.538)&(2.47) &(1.935)&(.977) &(1.671)&(1.274) \\
LNWAGE              &.366   &--.243 &.043   &.576   &--.494 &--.006 &.335   &--.280 &--.0140 \\
                    &(.122) &(.183) &(.153) &(.177) &(.261) &(.216) &(.110) &(.173) &(.141) \\
TENURE              &--.001 &.005   &--.049 &--.0009&.017   &--.067 &.000   &.005   &--.046 \\
                    &(.007) &(.013) &(.013) &(.01)  &(.019) &(.017) &(.006) &(.012) &(.011) \\
$\al$               &1.29   &1.08   &1.17   &1.75   &1.65   &1.79   &--     &--     &-- \\
                    &(.022) &(.033) &(.028) &(.04)  &(.06)  &(.048) &--     &--     &-- \\
$-\ln\textrm{L}$    &\multicolumn{3}{c}{5,666.13}&\multicolumn{3}{c}{5,543.33} \\
\hline \hline
\end{tabular}
    \end{center}
\end{sidewaystable}

\begin{figure}[ht!]\caption{Длительность безработицы: оценки базовых функций выживания в модели Кокса с конкурирующими рисками. Данные США по 3343 наблюдениям в 1986--92 гг., некоторые наблюдения неполные.}\label{fig:19.1}
\centering
%\includegraphics[scale=0.7]{fig.png}
\end{figure}

Представим модель Кокса в виде модели с конкурирующими рисками из раздела \ref{sec:19.2}. В этой спецификации отсутствует ненаблюдаемая гетерогенность и параметрическая спецификация базового риска. Последний, однако, можно оценить, как было показано в разделе 17.8.3. % \ref{sec:17.8.3} # UNCOMMENT AFTER 17 CH
Точечные оценки, сопоставимые с экспоненциальной моделью в таблице \ref{tab:19.2}, указаны в последних трех столбцах таблицы \ref{tab:19.3}. Стандартные ошибки больше, поскольку спецификация Кокса менее ограничительна, чем экспоненциальная. Оценки коэффициентов при страховке от безработицы ближе к оценкам экспоненциальной модели, чем к оценкам модели Вейбулла--IG, которые превышают настоящие почти в два раза. Коэффициент при LNWAGE также выше в модели Вейбулла--IG. Однако, при том, что ненаблюдаемая гетерогенность неучтена, базовый риск неидентифицируем. На рисунках \ref{fig:19.1} и \ref{fig:19.2}, соответственно, изображены рассчитанные базовые функции выживания и функции кумулятивного риска для трех состояний, которые можно интерпретировать как отражение некоторой ненаблюдаемой гетерогенности и зависимости от длительности. По графикам видно, что базовая функция выживания находится ниже остальных для тех, кто переходит в состояние частичной занятости, в то время как для тех, кто устраивается на полную ставку, она более пологая и лежит выше других. Соответственно, наклон функции кумулятивного риска наиболее крутой для тех, кто устраивается на полставки.

\begin{figure}[ht!]\caption{Длительность безработицы: оценки базовых кумулятивных рисков в модели Кокса с конкурирующими рисками. Данные те же, что и на рисунке \ref{fig:19.1}.}\label{fig:19.2}
\centering
%\includegraphics[scale=0.7]{fig.png}
\end{figure}

Описанный в данном разделе анализ не может быть заключительным и представлен здесь лишь в качестве иллюстрации. На самом деле, существуют серьезные подозрения, что вейбулловская функция риска мисспецифицирована. В частности, анализ МакКолл (1996), проведенный на этих же данных при более гибкой полиномиальной функции риска, свидетельствует об U-образной форме функции риска, что означает, что риск убывает для коротких длительностей и растет для более длительных наблюдений. Монотонная функция риска Вейбулла такую возможность не учитывает. Другие авторы, занимавшиеся моделированием длительностей безработицы на основе данных США, указывают, что при более гибкой функции риска учет ненаблюдаемой гетерогенности не оказывает значительного эффекта на результаты (Мейер, 1990; Хан Хаусман, 1990). Так как мы не наблюдаем этого в нашем примере, возможно имеет смысл попробовать более гибкую спецификацию, например, из раздела 17.10. % \ref{sec:17.10} # UNCOMMENT AFTER 17 CH



\section{Практические соображения}\label{sec:19.6}

\noindent
При моделировании многомерных моделей выживания имеет смысл начать с маргинальных моделей, прежде чем перейти к одновременному оцениванию. Такой подход может быть полезен для того, чтобы определить статистическую адекватность исходной спецификации.

На момент написания этой книги применение многомерных моделей выживания и риска в большинстве случаев потребовало бы написания собственного программного модуля на основе того или иного языка программирования, хотя такое задание и можно было бы упростить за счет использования встроенных библиотек, позволяющих максимизировать или минимизировать функции.

Оценивание модели CRM с независимыми рисками сводится к оцениванию ряда моделей выживания, практическая информация по применению которых была приведена в разделе 17.12. % \ref{sec:17.12} # UNCOMMENT AFTER 17 CH
Среди коммерческого программного обеспечения непросто найти программы для оценивания общих многомерных моделей CRM. Часто доступны лишь некоторые многомерные модели с определенной структурой зависимости. Например, STATA предоставляет расчеты \textbf{распределенной модели уязвимости} (\textit{shared frailty model}). Распределенная модель уязвимости является моделью со случайными эффектами, где компоненты ненаблюдаемой гетерогенности являются общими для каждой группы объектов или периодов и между группами распределены случайно.

Если основной интерес находится в области моделировании структуры зависимости между длительностями, то наиболее привлекательным по отношению к имитационному методу максимального правдоподобия для двумерного случая является подход на основе копула-функций, поскольку он не предполагает численного интегрирования. Данный подход применим и в случае, когда размерность больше двух, например, для моделей с многократными событиями, однако существует относительно мало таких примеров в литературе. Оценивание и тестирование маргинальных моделей возможно на основе стандартных одномерных моделей выживания, где параметр взаимосвязи оценивается с помощью последовательной двухшаговой процедуры. Даже если параметры оцениваются одновременно, оценки маргинальных моделей предоставляют полезную информацию в отношении стартовых значений для итерационных вычислений. Однако нам неизвестно программное обеспечение, которое бы позволяло оценивать такие модели.




\section{Библиографические заметки}\label{sec:19.7}

\begin{itemize}

    \item[\textbf{19.2}]
Хан и Хаусман (1990) предлагают эмпирический пример модели CRM, обобщенной на случай ненаблюдаемой гетерогенности.
В рамках модели CRM со случайными эффектами для состояний МакКолл (1996) анализирует эффект проводимой политики на поведение застрахованных безработных индивидов в поисках работы на полставки, используя для этого модель CRM с коррелированными рисками. В работе Батлера, Андерсона и Буркхаузера (1989) риски смерти и принятия предложения о работе также моделируются на основе CRM с коррелированными рисками.

    \item[\textbf{19.3}]
Первая работа по копула-функциям была написана Скларом в 1959 году на французском языке, однако статья Склара (1973) является хорошим аналогом на английском. Радулович и Вегкамп (дата не указана) приводят доказательство Теоремы Склара. Полезный обзор литературы по копулам с аннотированной библиографией можно найти в работе Фриса и Вальдеса (1998).

    \item[\textbf{19.4}]
Анализ многократных событий представлен в работах Меалли и Падни (1996) и Флинна и Хекмана (1982). Меалли и Падни (1996) исследуют переходы среди работ, учитываемых и не учитываемых при расчете пенсии, а также других <<состояний>> на рынке труда, используя симуляционные методы оценивания.

\end{itemize}


\section*{Упражнения}
\begin{enumerate}

    \item[\textbf{19--1}]
(Sapra, 2000; 2001). Данное упражнение иллюстрирует неидентифицируемость модели с конкурирующими рисками, которую привели Кокс и Тсиатсис и упомянутую в разделе \ref{sec:19.2}. Рассмотрим следующую модель с \textit{зависимыми} конкурирующими рисками, для которой известны $T = \min(T_1, T_2)$ и $\de$, где $\de = 1$, если $T = T_1$, и $\de = 2$, если $T = T_2$. Здесь $T_1$ и $T_2$ являются латентными длительностями рисков 1 и 2, соответственно. Предположим, что дана двумерная функция выживания $S(t_1, t_2) = \exp[-(\la_1 t_1 + \la_2 t_2)^\al]$, $0 < \al \le 1$, $\la_1, \la_2 > 0$. Постройте модель с независимыми конкурирующими рисками, эквивалентную обозначенной модели.

    \item[\textbf{19--2}]
Для модели, обозначенной в предыдущем упражнении, выпишите функцию логарифма правдоподобия в терминах коэффициентов риска и коэффициентов интегрального риска, если известны $T$ и $\de$. Исследуйте информационную матрицу параметров и покажите, что все параметры локально идентифицируемы, поскольку матрица невырождена.

    \item[\textbf{19--3}]
Рассмотрим две параллельные длительности, например, длительности периода безработицы, $T_1$, и периода отсутствия частного страхового полиса, $T_2$. Предположим, что длительности независимы условно на ненаблюдаемую гетерогенность и экспоненциально распределены со средними значениями $\beta_0 + \beta_1\x$ и $\ga_0 + \ga_1\x$, соответственно. Обозначим параметры ненаблюдаемой гетерогенности для обеих длительностей как $\nu_1$ и $\nu_2$ с математическим ожиданием $\E[\nu_1] = \E[\nu_2] = 1$.
        \begin{enumerate}
        \item
Запишите такой алгоритм получения коррелированных реализаций $(\nu_1, \nu_2)$ для произвольных значений параметров, чтобы с учетом $\x$, но без учета $(\nu_1, \nu_2)$, обе длительности были коррелированы. Также можно делать любые предположения о совместном распределении $(\nu_1, \nu_2)$, основываясь, например, на соображениях удобства математических вычислений. Объясните, как можно учесть степень корреляции между длительностями.
        \item
Используя для получения двумерного совместного распределения метод, представленный в разделе \ref{sec:19.3.2}, запишите совместное распределение длительностей.
        \item
Опишите, как можно расширить анализ из пункта (b) для наблюдений, цензурированных справа.
        \end{enumerate}


    \item[\textbf{19--4}]
Используя ту же выборку данных МакКолла, что и в главе 18, % \ref{ch:17} # UNCOMMENT AFTER 18 CH
оцените модель с двумя состояниями, безработицы и занятости (то есть, игнорируя различия между полным и неполным рабочим днем как два альтернативных состояния)
        \begin{enumerate}
        \item
Оцените модель одного уравнения Вейбулла и сравните результаты с моделью с независимыми конкурирующими рисками со спецификацией Вейбулла.
        \item
Определите, насколько повысилось качество подгонки модели со спецификацией CRM.
        \item
Оцените и сравните предсказанные значения <<риска найти работу>> при средних по выборке значениях объясняющих переменных в модели одного уравнения и модели CRM.
        \end{enumerate}

\end{enumerate}


    % Section 20.3. Эксперимент по страхованию здоровья:
    % http://www.d22d.ru/load/28-1-0-261

\chapter{ Модели счетных данных}


\section{Введение}\label{sec:20.1}

\noindent
Во многих экономических ситуациях зависимая переменная принимает целые неотрицательные значения, представляющие собой количество, число или подсчет произошедших событий, которое мы и хотим объяснить с помощью набора регрессоров. В отличие от классической модели регрессии зависимая переменная является дискретной, с ненулевой вероятностью принимающей только целые неотрицательные значения. Можно показать, что некоторые модели, представленные ранее в книге, такие как модель бинарного выбора и модель времени жизни, тесно связаны с моделью регрессии для счетных данных. Как и другие модели с дискретными переменными, например, логит и пробит, эта модель является нелинейной, свойства которой объясняются дискретностью и нелинейностью.

Рассмотрим несколько примеров из микроэконометрики на выборке независимых пространственных данных. В исследованиях по фертильности часто моделируют число рождений у женщины в определенном возрастном интервале как зависимость от таких факторов, как количество лет обучения, возраст и доход домохозяйства (Winkelmann, 1995). В некоторых моделях принятия решений в домохозяйстве в качестве объясняющей переменной используют количество детей, принимая во внимание ее эндогенность. В рамках анализа несчастных случаев часто исследуют взаимосвязь между безопасностью, измеренной как количество несчастных случаев на рейсах авиакомпании, и прибыльностью, а также другими финансовыми показателями авиакомпании (Rose, 1990). Для оценки спроса на рекреационные услуги определяют ценность природных ресурсов, таких как национальные парки, измеренную с помощью количества путевок в зоны отдыха (Gurmu и Trivedi, 1996). Для оценки спроса на услуги здравоохранения обычно моделируют данные по количеству потребления таких услуг, например, посещений доктора или дней, проведенных в госпитале в прошлом году (Cameron et al., 1988). В случае если мы хотим определить взаимосвязь между такими переменными, как состояние здоровья и страхование на случай болезни, регрессии для счетных данных также подходят.

    \begin{table}[!ht]\caption{\textit{Доля нулевых значений в эмпирических исследованиях}}\label{tab:20.1}
    \begin{center}
\begin{tabular}{llcc}
\hline \hline
                          &               &Размер         &Доля нулевых\\
Исследование              &Переменная     &выборки        &значений\\
\hline
Cameron et al. (1988)     &Визиты к врачу&$5,190$&$0.798$\\
Pohlmeier и Ulrich (1995) &Визиты к специалисту&$5,096$&$0.678$\\
Grootendorst (1995)       &Лекарства по рецепту&$5,743$&$0.224$\\
Deb и Trivedi (1997)      &Число дней в госпитале&$4,406$&$0.806$\\
Gurmu и Trivedi (1996)    &Путевки в зоны отдыха&$659$&$0.632$\\
Geil et al. (1997)        &Число госпитализаций&$30,590$&$0.899$\\
Greene (1997)             &Негативные отметки&$1,319$&$0.803$\\
                          &в кредитной истории&&\\
\hline \hline
    \end{tabular}
    \end{center}
    \end{table}

Основные подходы к моделированию представлены в разделах \ref{sec:20.2}--\ref{sec:20.5}. Детальное описание модели регрессии Пуассона можно найти в разделе \ref{sec:20.2}. Применение модели на примере данных RHIE рассмотрено в разделе \ref{sec:20.3}. Поскольку модель регрессии Пуассона довольно ограничительна, в разделе \ref{sec:20.4} мы рассмотрим другие, также широко используемые параметрические модели счетных данных. Здесь же можно найти менее известные альтернативные параметрические методы для счетных данных, такие как модель дискретного выбора. Частично параметрический подход к моделированию условных матожидания и дисперсии подробно представлен в разделе \ref{sec:20.5}. Раздел \ref{sec:20.6} представляет собой введение в многомерные модели счетных данных и модели с эндогенными регрессорами. Раздел \ref{sec:20.7} иллюстрирует применение различных моделей на данных RHIE. В обучающих целях мы подробно рассмотрим модель регрессии Пуассона на пространственных данных.  Другие модели, хотя и более подходящие, чем модель Пуассона, из соображений краткости представлены менее детально. Полное описание можно найти в Cameron и Trivedi (1998), а также в работах, указанных в разделе \ref{sec:20.9}.




\section{Основные модели регрессии счетных данных}\label{sec:20.2}

\noindent
В некоторых случаях нас интересуют переменные в том виде, в котором они есть, например, число рождений. В других случаях интересующие нас переменные, такие как спрос на медицинские услуги и расходы на НИОКР, выраженные в долларах, являются непрерывными, однако доступные по ним данные --- счетные. Зачастую выборка состоит из \textbf{нескольких дискретных значений}, например, 0, 1 и 2. Это подтверждает таблица \ref{tab:20.1}, где представлена доля нулевых значений в некоторых опубликованных эконометрических исследованиях; заметим, что иногда эта доля может достигать 90\%. Также распределение данных может быть \textbf{скошено вправо}. Наконец, в данных по определению присутствует \textbf{гетероскедастичность}, где дисперсия возрастает вместе со средним.


\subsection{Регрессия Пуассона}\label{sec:20.2.1}

\noindent
Несмотря на то, что модель Пуассона зачастую плохо соответствует данным, изучение анализа счетных данных обычно начинают именно с нее. В разделах \ref{sec:20.2.1}--\ref{sec:20.2.3} мы представим модель регрессии Пуассона, рассмотренную ранее в разделе 5.2, % \ref{sec:5.2} # UNCOMMENT AFTER 5 CH
оценивание методом максимального правдоподобия, интерпретацию оценок коэффициентов и соответствующие модификации для урезанных и цензурированных данных. В разделе \ref{sec:20.2.3} мы также представим квази-ММП, основанный на распределении Пуассона с верно специфицированным условным матожиданием, но с неверно специфицированной функцией дисперсии. Недостатки модели Пуассона, в частности, свойство равенства матожидания и дисперсии, будут рассмотрены в разделе \ref{sec:20.2.4}

    \begin{table}[!ht]\caption{\textit{Описательные статистики данных в некоторых исследованиях по НИОКР}}\label{tab:20.2}
    \begin{center}
\begin{tabular}{lccccc}
\hline \hline
                        &Размер         &           &Стандартная&Максимальное            &Доля нулевых\\
Исследование            &выборки        &Среднее    &ошибка     &кол-во патентов         &значений\\
\hline
Cincera (1997)          &$181$          &$60.8$     &$721.6$    &$925$                  &$<0.19$\\
Crepon и Duguet (1997b)   &$698$          &$11.6$     &na$^a$     &na                   &$0.441$\\
Crepon и Duguet (1997a)   &$451$          &$2.73$     &$11.45$    &na                   &$0.729$\\
Hausman et al. (1984)   &$346$          &$32.1$     &$66.36$    &$515$                  &$0.220$\\
Wang et al. (1998)      &$70$           &$23.46$    &$39.10$    &$173$                  &$0.186$\\
\hline \hline
\multicolumn{6}{l}{$^a$ \scriptsize{na, not available}}
    \end{tabular}
    \end{center}
    \end{table}

Следует оговориться, что в некоторых случаях высокая доля нулей в выборке может сочетаться с высокими ненулевыми значениями, что создает определенные сложности при моделировании. Это подтверждает таблица \ref{tab:20.2}, где представлены описательные статистики данных из пяти исследований о зависимости между количеством патентов и расходами на НИОКР (R\&D). Следует обратить внимание, что максимальное наблюдаемое значение существенно превышает среднее. Сложности же заключаются в выборе такой функциональной формы, которая будет учитывать высокую долю нулевых значений вместе с высоким средним. В других примерах, практически все данные являются однозначными числами (например, число рождений), поэтому среднее количество событий невелико.

Такие особенности требуют применения специальных методов и моделей счетных регрессий, среди которых можно выделить два подхода.

Первый подход является \textbf{полностью параметрическим}, описывающим распределение данных, которое учитывает целочисленность и неотрицательность $y$. Данный подход применялся в первых работах, в основном, в биостатистике, где счетные регрессии рассматривались как способ обобщения литературы на распределение независимых и идентичных наблюдений. Также он был представлен в известной работе Hausman et al. (1984).

Второй подход основан на спецификации \textbf{матожидания и дисперсии} (\textit{mean--variance approach}) и описывает неотрицательность условного матожидания и условную дисперсию как функцию от матожидания, что неплохо соответствует неотрицательности и гетероскедастичности данных, но не учитывает их дискретность. Данный подход был представлен Nelder и Wedderburn (1972) в рамках анализа не только счетных данных и впоследствии послужил основой для обобщенной линейной модели, широко используемой в статистике (McCullagh и Nelder, 1989). В эконометрике он был представлен Gouri\'eroux, Monfort и Trognon (1984a,b), и чаще всего рассматривался как частный случай обобщенного метода моментов.


\subsection{ММП и квази-ММП Пуассона}\label{sec:20.2.2}

\noindent
Метод квази-максимального правдоподобия Пуассона (квази-ММП, \textit{QMLE}) был представлен и изучен в главе 5 % \ref{ch:5} # UNCOMMENT IN THE END OF THE BOOK
как пример оценивания $m$ плотностей. В данном разделе мы представим более полное описание.

Обычная стохастическая модель для счетных данных является точечным процессом Пуассона. Это подразумевает, что число наступивших событий описывается \textbf{распределением Пуассона} с функцией вероятности
    \begin{align}\label{eq:20.1}
    \Pr [Y=y]=\frac{e^{-\mu}\mu^y}{y!}, \hspace{0.5cm} y=0,1,2,...,
    \end{align}
где $\mu$ --- параметр интенсивности. Обозначим распределение как $\mathcal{P}[\mu]$, первые два момента которого равны
    \begin{align}\label{eq:20.2}
    \E[Y] = \mu, \\
    \V[Y] = \mu, \notag
    \end{align}
что соответствует свойству \textbf{равенства матожидания и дисперсии} (\textit{equidispersion}) в распределении Пуассона.

Добавив нижний индекс $i$ для $y$ и $\mu$, мы можем расширить данный подход на случай регрессии. \textbf{Модель регрессии Пуассона} получается из распределения Пуассона с помощью выражения среднего $\mu$ через набор регрессоров $\x$. Стандартной предпосылкой является моделирование среднего в виде экспоненты
    \begin{align}\label{eq:20.3}
    \mu = \exp{(\xib)}, \hspace{0.5cm} i=1,...,N,
    \end{align}
с $K$ линейно независимых ковариат, включая константу. Зная (\ref{eq:20.2}) и (\ref{eq:20.3}), можно найти дисперсию $\V[y_i|\x_i] = \exp(\xib)$, поэтому регрессия Пуассона гетероскедастична по определению.

Из (\ref{eq:20.1}) и (\ref{eq:20.3}), а также предположения о независимости наблюдений $(y_i|\x_i)$ следует, что подходящим методом оценивания является метод максимального правдоподобия. Логарифм функции правдоподобия записывается как
    \begin{align}\label{eq:20.4}
    \ln \L(\be) = \sum^{N}_{i=1} {y_i\xib - \exp{(\xib) - \ln y_i !}}.
    \end{align}
\textbf{Оценки, полученные ММП Пуассона}, $\hat{\be}_P$, являются решением $K$ нелинейных уравнений, соответствующих условию первого порядка для нахождения максимума правдоподобия
    \begin{align}\label{eq:20.5}
    \sum^{N}_{i=1} (y_i - \exp{(\xib)})\x_i = \textbf{0}.
    \end{align}
Если $\x_i$ включает константу, то сумма остатков $y_i - \exp(\xib)$ равна нулю согласно (\ref{eq:20.5}). Функция правдоподобия является глобально вогнутой, следовательно, оценки параметров единственны. Сами оценки можно найти с помощью методов Гаусса--Ньютона или Ньютона--Рафсона.

В эконометрической литературе \textbf{псевдо-ММП} (\textit{PML}) или \textbf{квази-ММП} (\textit{QML}) относится к оценке максимального правдоподобия при мисспецификации функции плотности (Gourieroux et al., 1984a). Термины псевдо-ММП и квази-ММП взаимозаменяемы. Оценки не требуют столь же строгих предпосылок о процессе, генерирующем данные, что и обозначенная функция правдоподобия; см. раздел 5.7. % \ref{sec:5.7} # UNCOMMENT AFTER 5 CH
В статистической литературе квази-ММП часто относят к нелинейному обобщенному методу наименьших квадратов. В этом смысле квази-ММП эквивалентен стандартной максимизации правдоподобия для регрессии Пуассона.

Из (\ref{sec:20.5}) следует, что условия первого порядка для оценки псевдо-ММП, $\hat{\be_\mP}$, равны $\sum^{N}_{i = 1} (y_i - \exp(\xib))\x_i = \0$. Как уже упоминалось, левая часть уравнения имеет матожидание, равное нулю, если $\E[y_i|\x_i] = \exp(\xib)$. Следовательно, оценки квази-ММП Пуассона состоятельны при более слабых предпосылках о корректной спецификации условного матожидания; то есть, необязательно, чтобы данные имели распределение Пуассона. Используя результаты, полученные в разделе 5.2.3, % \ref{sec:5.7} # UNCOMMENT AFTER 5 CH
запишем матрицу ковариаций в виде
    \begin{align}\label{eq:20.6}
    \V_{\mathrm{PML}}[\hat{\be_\mP}] = \left( \sum^{N}_{i = 1} \mu_i \x_i\x'_i \right)^{-1} \left( \sum^{N}_{i = 1} \omega_i \x_i\x'_i \right) \left( \sum^{N}_{i = 1} \mu_i \x_i\x'_i \right)^{-1},
    \end{align}
где $\omega_i = \V[y_i|\x_i]$ является условной дисперсией $y_i$.

Если ввести более строгие предпосылки о корректной параметрической спецификации регрессии Пуассона, что есть $\omega_i = \mu_i$, то $\hat{\be_\mP}$ будет состоятельной и асимптотически нормальной оценкой $\be$ с матрицей выборочных ковариаций
    \begin{align}\label{eq:20.7}
    \V[\hat{\be_\mP}] = \left( \sum^{N}_{i = 1} \mu_i \x_i \x'_i \right)^{-1},
    \end{align}
в случае, если $\mu_i$ имеет экспоненциальную форму вида (\ref{eq:20.3}).

Оценки, полученные ММП и псевдо-ММП Пуассона, идентичны, но обладают разной дисперсией. Практическое применение более робастных оценок вида (\ref{eq:20.6}) представлено в разделе \ref{sec:20.5.1}.


\subsection{Интерпретация коэффициентов регрессии}\label{sec:20.2.3}

\noindent
В линейных моделях с $\E[y|\x] = \xib$ коэффициенты $\be$ показывают, насколько изменяется условное матожидание при изменении соответствующих регрессоров на одну единицу. В нелинейных моделях интерпретация будет отличаться; см. обсуждение в разделе 5.2.4. Взяв производную в любой модели с экспоненциальным условным матожиданием, получим
    \begin{align}\label{eq:20.8}
    \frac{\pa\E[y|\x]}{\pa x_j} = \beta_j \exp(\xib),
    \end{align}
где скаляр $x_j$ обозначает $j$-ый регрессор. Например, если $\hat{\beta}_j = 0.25$ и $\exp(\x'_i\hat{\be}) = 3$, то изменение $j$-го регрессора на единицу приведет к изменению $y$ на $0.75$. Значит, частичный отклик зависит от выражения $\exp(\x'_i\hat{\be})$, которое принимает различные значения в зависимости от индивида. Легко показать, что $\beta_j$ равняется относительному изменению $\E[y|x]$ при изменении $x_j$ на единицу. Если $x_j$ выражен в логарифмах, то $\beta_j$ представляет собой эластичность.

Так как эффекты от изменения регрессоров индивидуальны, можно рассчитать средний отклик, $N^{-1} \sum_i \pa \E[y_i|\x_i]/\pa x_{ij} = \hat{\beta}_j \times N^{-1} \sum_i \exp(\x'_i\hat{\be})$. Для моделей регрессии Пуассона со свободным членом это выражение упрощается до $\hat{\beta}_j \bar{y}$.

Из (\ref{eq:20.8}) также следует, что если оценка $\beta_j$ в два раза превышает оценку $\beta_k$, то эффект от изменения $j$-го регрессора на единицу будет также вдвое больше эффекта от изменения $k$-го регрессора.


\subsection{Избыточная дисперсия}\label{sec:20.2.4}

\noindent
Зачастую модель регрессии Пуассона плохо соответствует счетным данным, в связи с чем в разделах \ref{sec:20.3} и \ref{sec:20.4} предлагаются альтернативные модели. Фундаментальная проблема заключается в том, что распределение Пуассона задается единственным параметром $(\mu)$, и, как следствие, все остальные моменты $y$ являются функциями от $\mu$. Для сравнения, нормальное распределение имеет различные параметры для среднего и разброса, $(\mu)$ и $(\sis)$, соответственно. По той же причине однопараметрическое экспоненциальное распределение хуже соответствует данным, чем двухпараметрическое распределение Вейбулла. Заметим, однако, что проблема отсутствует в бинарных данных. В этом случае в качестве распределения подходит однопараметрическое распределение Бернулли, которое определяет вероятность успеха $p$ и вероятность проигрыша $1 - p$, где $p$ необходимо объяснить с помощью набора регрессоров.

Одно из последствий такого однопараметрического моделирования заключается в том, что вероятность нулевых значений, предсказанная по модели Пуассона, значительно ниже, чем их доля в выборке, что называется проблемой \textbf{избыточных нулевых значений}.

Другим недостатком является то, что модель Пуассона подразумевает равенство дисперсии и матожидания (см. (\ref{eq:20.2})), в то время как в счетных данных дисперсия обычно превышает среднее, что соответствует проблеме \textbf{избыточной дисперсии}.

Избыточная дисперсия в модели Пуассона имеет те же последствия, что и нарушение предпосылки о гомоскедастичности в модели линейной регрессии. При условии, что спецификация условного матожидания (\ref{eq:20.3}) верна, оценки ММП Пуассона состоятельны, что следует из условий первого порядка (\ref{eq:20.5}), так как матожидание левой части уравнения (\ref{eq:20.5}) равно нулю при $\E[y_i|\x_i] = \exp(\xib)$. В общем случае свойство состоятельности применяется к квази-ММП, когда заданная плотность принадлежит экспоненциальному семейству распределений. Хотя и распределение Пуассона, и нормальное распределение являются членами экспоненциального семейства распределений, рассмотренного ранее в разделе 5.7.3, учитывать избыточную дисперсию важно по ряду причин. Во-первых, при более сложной структуре данных, например, при наличии урезанных и цензурированных наблюдений, оценки больше не являются состоятельными. Во-вторых, даже для простой структуры данных избыточная дисперсия приводит к существенно завышенным стандартным ошибкам и $t$-статистикам по сравнению с обычным ММП, что указывает на важность робастного оценивания дисперсии. В-третьих, для оценивания вероятностей числа событий требуется больше дополнительных параметров, чем для оценивания условного матожидания.

Избыточная дисперсия может указывать на мисспецификацию, особенно при наличии урезанных и цензурированных наблюдений, если они не были учтены при оценивании. В таком случае условное матожидание специфицировано неверно и наличие избыточной дисперсии приводит не только к неэффективности, но и к несостоятельности оценок ММП.

Поэтому сразу после оценивания регрессии Пуассона желательно провести тест на избыточную дисперсию. Во многих моделях счетных данных избыточная дисперсия имеет вид
    \begin{align}\label{eq:20.9}
    \V[[y_i|\x_i] = \mu_i + \al g(\mu_i)
    \end{align}
где $\al$ является неизвестным параметром, а $g(\cdot)$ --- известной функцией, обычно задаваемой как $g(\mu) = \mu^2$ или $g(\mu) =\mu$. Предполагается, что при обеих гипотезах спецификация матожидания верна, а при нулевой гипотезе также $\al = 0$, что соответствует равенству дисперсии и матожидания $\V[y_i|\x_i] = \mu_i$.
Простая \textbf{тестовая статистика} для проверки гипотезы об \textbf{избыточной дисперсии} $H_0: \al = 0$ против альтернативной $H_1: \al \ne 0$ или $H_1: \al > 0$ может быть рассчитана с помощью оценки вспомогательной МНК регрессии (без свободного члена) на основе предсказанных значений $\hat{\mu}_i = \exp(\x'_i\hat{\be})$ по модели Пуассона
    \begin{align}\label{eq:20.10}
    \frac{(y_i - \hat{\mu}_i)^2 - y_i}{\hat{\mu}_i} = \al\frac{g(\hat{\mu}_i)}{\hat{\mu}_i} + u_i,
    \end{align}
где $u_i$ обозначает ошибку. Рассчитанные $t$-статистики для коэффициента $\al$ асимптотически нормальны при нулевой гипотезе (Cameron и Trivedi, 1990). Тест также может быть использован для проверки гипотезы о \textbf{недостаточной дисперсии}, $\al < 0$, то есть, когда условная дисперсия меньше условного матожидания. См. также Gurmu и Trivedi (1992).




\section{Пример на счетных данных: Визиты к врачу}\label{sec:20.3}

\noindent
Для иллюстрации методов и моделей, описанных выше, мы будем использовать данные из эксперимента по страхованию здоровья, проводимого корпорацией RAND (\textit{RAND Health Insurance Experiment}) с 1974 по 1982 гг., ранее использованные в работе Deb и Trivedi (2002). Авторы провели более глубокий анализ данных, чем это требуется и хотелось бы в рамках моделей, представленных в этой главе. Данный эксперимент является, пожалуй, наиболее длительным и контролируемым экспериментом в исследованиях медицинского обслуживания. Цель эксперимента заключалась в том, чтобы оценить влияние способов страхования здоровья, случайно распределенных (\textit{randomly assigned}) между пациентами, на потребление медицинских услуг.
% (\textit{fee-for-service})
% (\textit{health maintenance organizations, HMOs})
В ходе эксперимента были собраны данные о $8,000$ участниках из $2,823$ семей из шести городов США. Каждая семья принимала участие в одной из 14 различных программ страхования здоровья в течение трех или пяти лет. Программы варьировались от бесплатной медицинской помощи до 95\%-го сострахования (соплатежа) в пределах максимальных долларовых затрат (\textit{maximum dollar expenditure, MDE}), а также включали приписку к тому или иному медицинскому центру.

Ключевая идея заключается в том, что, поскольку программы страхования не выбирались участниками, а были распределены случайно, то в анализе отсутствует эндогенный эффект воздействия, что позволяет определить причинную связь.

Данные были собраны на основе потребления участниками медицинских услуг и их состояния здоровья в течение случайно распределенного периода участия, трех или пяти лет. Детали можно найти в работах Manning et al. (1987), Newhouse et al. (1993) и Deb и Trivedi (2002). Выборка, используемая в данном примере, исключает программы страхования с бесплатной медициной.

Данные включают информацию о потреблении, затратах, демографических характеристиках, состоянии здоровья и типе страхования. Данные по затратам были проанализированы в разделе 16.6. % \ref{sec:16.6} # UNCOMMENT AFTER CH 16
Ставка сострахования в данной выборке может принимать четыре различных значения. Тем не менее, аналогично исследованиям RAND, мы будем рассматривать ее как непрерывную переменную. Итоговая выборка состоит из $20,186$ наблюдений; каждое наблюдение отображает информацию по соответствующему участнику эксперимента в определенном году. Для простоты мы не будем рассматривать кластеры данных; см. раздел
24.5. % \ref{sec:24.5} # UNCOMMENT AFTER CH 24

    \begin{table}[!htbp]\caption{\textit{Визиты к врачу: распределение частот}}\label{tab:20.3}
    \begin{center}
\begin{tabular}{lccccccccccc}
\hline \hline
Визиты                  &0&1&2&3&4&5&6&7&8&9&10\\
Относительная частота   &31.2&18.9&13.8&9.3&6.7&4.8&3.4&2.6&2.0&1.4&1.0\\
\hline
Визиты                  &11&12&13&14&15&16&...&$>21$& Max&&\\
Относительная частота   &0.9&0.6&0.5&0.4&0.3&0.3& &1.0&77&&\\
\hline \hline
\end{tabular}
    \end{center}
    \end{table}

Потребление в данном примере измеряется в количестве визитов к врачу (MDU). Распределение относительных частот MDU в процентах представлено в таблице \ref{tab:20.3}. MDE обозначает максимальные долларовые затраты, то есть, максимальную сумму платежей, после достижения которой все медицинские счета оплачиваются страховой компанией. Заметим, что порядка 31\% наблюдений содержат нули. Тяжелый правый хвост распределения и существенное превышение дисперсии над средним указывают на то, что в данных присутствует (безусловная) избыточная дисперсия.

В данном разделе мы будем рассматривать оценки регрессии ММП и квази-ММП Пуассона, другие спецификации будут рассмотрены позже. Список объясняющих переменных представлен в таблице \ref{tab:20.4}.

    \begin{table}[!htbp]\caption{\textit{Визиты к врачу: описание переменных}}\label{tab:20.4} % ALT: Посещения врача
    \begin{center}
\begin{tabular}{llcc}
\hline \hline
\textbf{Переменная}&\textbf{Определение}&\textbf{Среднее}&\textbf{Ст. Откл.}\\
\hline
MDU     & Число визитов к врачу &2.861&4.505\\
LC      & $\ln(\textrm{сострахование} + 1)$, $0 \le \textrm{сострахование} \le 100$ &1.710&1.962\\
IDP     & 1, если план страхования с индивидуальной франшизой&0.220&0.414\\
        & 0, в противном случае&&\\
LPI     & $\ln(\max(1, \textrm{ежегодная стимулирующая выплата}$  &4.709&2.697\\
        & \textrm{за участие}))&&\\
FMDE    & 0, если IDP = 1  &3.153&3.641\\
        & $\ln(\max(1, \textrm{MDE}$/(0.01 \textrm{coinsurance}))), в противном случае&&\\
LINC    & $\ln(\textrm{доход домохозяйства})$ &8.708&1.228\\
LFAM    & $\ln(\textrm{размер домохозяйства})$  &1.248&0.539\\
AGE     & Возраст в годах  &25.718&16.678\\
FEMALE  & 1, если женщина  &0.517&0.500\\
CHILD   & 1, если возраст меньше 18  &0.402&0.490\\
FEMCHILD& FEMALE*CHILD  &0.194&0.395\\
BLACK   & 1, если глава домохозяйства афроамериканец &0.182&0.383\\
EDUCDEC & Количество лет обучения главы домохозяйства  &11.967&2.806\\
PHYSLIM & 1, если имеет ограничения физического характера  &0.124&0.322\\
NDISEASE& Количество хронических заболеваний  &11.244&6.742\\
HLTHG   & 1, если самооценка состояния здоровья хорошая  &0.362&0.481\\
HLTHF   & 1, если самооценка состояния здоровья средняя  &0.077&0.267\\
HLTHP   & 1, если самооценка состояния здоровья плохая  &0.015&0.121\\
        & \multicolumn{3}{l}{Пропущенная переменная --- отличная самооценка состояния здоровья}\\
\hline\hline
\end{tabular}
    \end{center}
    \end{table}

    \begin{table}[!htbp]\caption{\textit{Визиты к врачу: оценки модели счетных данных}}\label{tab:20.5}
    \begin{center}
\begin{tabular}{lccccc}
\hline \hline
&\multicolumn{2}{c}{\textbf{Poisson}}&\textbf{PPML}&\multicolumn{2}{c}{\textbf{NB2-PML}}\\
\cmidrule(r){2-3}\cmidrule(r){4-4}\cmidrule(r){5-6}
\textbf{Модель}&\textbf{Коэффициент}&\textbf{$t$-статистика}&\textbf{$t$-статистика}&\textbf{Коэффициент}&\textbf{$t$-статистика}\\
\hline
LC      &$-.0427$&$-7.030$  &$-2.835$   &$-0.0504$  &$-3.228$\\
IDP     &$-.1613$&$-13.881$ &$-5.773$   &$-0.1475$  &$-4.889$\\
LPI     &$0.0128$&$6.999$   &$2.912$    &$0.0158$   &$3.574$\\
FMDE    &$-.0206$&$-5.803$  &$-2.319$   &$-0.0213$  &$-2.351$\\
PHYSLIM &$0.2684$&$21.711$  &$8.240$    &$0.2751$   &$8.068$\\
NDISEASE&$0.0231$&$38.124$  &$13.487$   &$0.0259$   &$15.324$\\
HLTHG   &$0.0394$&$4.109$   &$1.699$    &$0.0065$   &$0.275$\\
HLTHF   &$0.2531$&$15.613$  &$5.894$    &$0.2368$   &$5.425$\\
HLTHP   &$0.5216$&$19.150$  &$6.966$    &$0.4256$   &$6.205$\\
$\al$   &$-$     &$-$       &$-$        &$1.1822$   &$8.926$\\
$-\ln\mL$&$60087$ &          &           &$42777$    &\\
\hline \hline
\end{tabular}
    \end{center}
    \end{table}

Оценки коэффициентов с соответствующими $t$-статистиками, логарифм правдоподобия и информационные критерии представлены в таблице \ref{tab:20.5}. Интерес представляют коэффициенты при переменных, отвечающих за страхование (LC, IDP, LPI и FMDE), поскольку они отражают чувствительность потребления к цене. Нас интересуют также коэффициенты при переменных, отвечающих за состояние здоровья (PHYSLIM, NDISEASE, HLTHG, HLTHF и HLTFP).

Рассмотрим коэффициент при ставке сострахования, измеренной в логарифмах, LC. Эта переменная представляет ключевой интерес, поскольку отражает эффект цены. % ALT: стоимостной эффект
Чем выше ставка сострахования, тем выше будет сумма (со)платежа пациентом за посещение, следовательно, тем ниже будет среднее число визитов к врачу. Оценка коэффициента в регрессии Пуассона (1-ый столбец в таблице \ref{tab:20.5}) отрицательна $(-0.42)$ с $t$-статистикой, равной $2.835$, что указывает на значимую отрицательную зависимость от цены, как и предсказывает классическая теория. Эластичность числа визитов к врачу по отношению к LC равна $-.042$. Однако при интерпретации следует быть аккуратным, поскольку ставка сострахования может принимать лишь четыре значения. С этой оговоркой, оценку можно интерпретировать как эластичность. Аналогично, оценка коэффициента при логарифме дохода (LINC) составляет $0.174$, что означает, что увеличение дохода приводит к увеличению среднего числа посещений.

Как узнать, насколько хорошо модель Пуассона соответствует данным? Один простой способ заключается в том, чтобы сравнить действительные и предсказанные значения вероятностей при различных количествах посещений врача. Такое сравнение представлено в таблице \ref{tab:20.6} для первых девяти визитов; мы не рассматриваем остальные значения, поскольку они составляют менее 10\% выборки. Для расчета предсказанной вероятности $\Pr[y_i|\x'_i\hat{\be}]$ для $y_i = 0, 1, ..., 9$, необходимо подставить оценку $\hat{\mu}_i$ в уравнение \ref{eq:20.1} и усреднить по наблюдениям. Заметим, что регрессия Пуассона значительно недооценивает вероятность нулевых значений и переоценивает долю ненулевых значений для количества посещений меньше семи. Следовательно, регрессия Пуассона плохо подходит для работы со счетными данными, что можно объяснить неучетом избыточной дисперсии в данных (Cameron и Trivedi, 1998, глава 4).

    \begin{table}[!htbp]\caption{\textit{}}\label{tab:20.6}
    \begin{center}
\begin{tabular}{lcccccccccc}
\hline \hline
\textbf{Относительная частота} &$\mathbf{0}$&$\mathbf{1}$&$\mathbf{2}$&$\mathbf{3}$&$\mathbf{4}$&$\mathbf{5}$&$\mathbf{6}$&$\mathbf{7}$&$\mathbf{8}$&$\mathbf{9}$\\
\hline
Относительная частота           &31.2&18.9&13.8&9.3&6.7&4.8&3.4&2.6&2.0&1.4\\
Предсказанные значения          &10.6&19.2&20.9&17.6&12.6&7.99&4.69&2.64&1.46&0.8\\
по модели Пуассона              &&&&&&&&&&\\
Предсказанные значения          &30.9&19.6&13.6&9.67&6.97&5.07&3.70&2.72&2.0&1.47\\
по модели NB2                   &&&&&&&&&&\\
\hline \hline
\end{tabular}
    \end{center}
    \end{table}

Можно ожидать, что при неучете избыточной дисперсии $t$-статистики для ММП Пуассона будут завышены. Для сравнения в 3-ем столбце (PPML) таблицы \ref{tab:20.5} представлены робастные $t$-статистики. Например, при робастном оценивании $t$-статистика для LC возрастает с $-7.03$ до $-2.83$. В таблицах \ref{tab:20.5} и \ref{tab:20.6} также указаны оценки модели NB2, которая будет рассмотрена позже в разделе \ref{sec:20.7}. Из таблицы видно, что модель NB2 лучше соответствует данным, чем модель Пуассона.




\section{Параметрические модели регрессии для счетных данных}\label{sec:20.4}

\noindent
Модель регрессии Пуассона зачастую довольно ограничительна, поэтому в данном разделе мы представим несколько более гибких параметрических моделей.

Во-первых, избыточная дисперсия в счетных данных может являться следствием ненаблюдаемой гетерогенности. В таком случае предполагается, что наблюдения сгенерированы процессом Пуассона (с последовательно независимыми событиями), но параметр интенсивности этого процесса определить невозможно. Более того, параметр интенсивности сам является случайной величиной. Подход на основе смеси, представленный в разделах \ref{sec:20.4.1} и \ref{sec:20.4.2}, приводит к широко применяемой отрицательной биномиальной модели.

Во-вторых, избыточная и в некоторых случаях недостаточная дисперсия может возникать потому, что процесс, определяющий первое событие, может отличаться от процесса, определяющего остальные. Например, первичное обращение к врачу может быть обосновано исключительно решением пациента, однако последующие визиты могут быть назначены врачом. Следовательно, требуется модификация моделей счетных данных, что представлено в разделе \ref{sec:20.4.5}.

В-третьих, избыточная дисперсия в счетных данных может являться следствием нарушения предпосылки о независимости событий, что неявно предполагается в процессе Пуассона. Тогда зависимость можно определить таким образом, чтобы последующие визиты к врачу были более вероятны, чем первый. (Этот подход не нашел широкого применения в анализе счетных данных. В анализе выживаемости такая особенность данных называется истинной зависимостью от состояния.) Предпосылки о ненаблюдаемой гетерогенности или зависимости от состояния также приводят к отрицательной биномиальной модели; см. Winkelmann (1995). Модель дискретного выбора, моделирующая вероятности $\Pr[y = j|y\ge j - 1]$, представлена в разделе \ref{sec:20.4.6}.

В-четвертых, можно обратиться к литературе, посвященной одномерным независимо идентично распределенным счетным данным, например, логарифмическому ряду или гипергеометрическому распределению (Johnson, Kotz и Kemp, 1992). Новые модели регрессии можно получить с помощью моделирования одного или двух параметров распределения как функцию от объясняющих переменных. Этот подход не имеет тех же оснований, что и первые три, поэтому маловероятно, что такие модели могут оказаться лучше.

Помимо избыточной дисперсии может возникать и недостаточная, например, если выборочные данные содержат нулевые и единичные значения, а также небольшое количество двоек. Такое распределение будет близко к биномиальному с матожиданием, превышающим дисперсию. Также могут быть использованы распределения Каца и другие распределения, основанные на методах разложения в ряд, которые представлены, например, в работе Cameron и Johansson (1997); см. также Cameron и Trivedi (1998, глава 12).


\subsection{Отрицательная биномиальная модель}\label{sec:20.4.1}

\noindent
Несмотря на то, что отрицательная биномиальная модель является примером модели непрерывной смеси, она может быть получена различными способами. Представленный далее подход на основе смеси, однако, является одним из первых и наиболее популярных.

Предположим, что случайная величина $y$ следует распределению Пуассона при условии, что параметр $\la$ известен, $f(y|\la) = \exp(-\la)\la^y/y!$. Более того, пусть параметр $\la$ также содержит случайную компоненту так, что его нельзя точно определить с помощью регрессоров $\x$. То есть, $\la = \mu\nu$, где $\mu$ представляет собой детерминированную функцию от регрессоров $\x$, например, $\exp(\xib)$, а компоненты $\nu > 0$ независимо идентично распределены с плотностью $g(\nu|\al)$. Поскольку наблюдения могут иметь различные $\la$ (гетерогенность) со случайной (ненаблюдаемой) составляющей $\nu$, такая модель является примером модели с ненаблюдаемой гетерогенностью. Заметим, что $\E[\la|\mu] = \mu$, если $\E[\nu] = 1$, следовательно, интерпретация коэффициентов наклона аналогична интерпретации в модели Пуассона.

Маргинальную плотность $y$, при условии информации о параметрах $\mu$ и $\al$, но безусловно на $\nu$, можно получить за счет исключения $\nu$ при интегрировании. Таким образом,
    \begin{align}\label{eq:20.11}
    h(y|\mu,\al) = \int f(y|\mu, \nu) g(\nu|\al) d\nu,
    \end{align}
где $g(\nu|\al)$ называется смешиваемым распределением с неизвестным параметром $\al$. То есть, интегрирование позволяет найти ``усредненное'' распределение. В зависимости от выбора $f(\cdot)$ и $g(\cdot)$ интеграл может как иметь, так и не иметь аналитическую форму.

Пусть $f(y|\la)$ является плотностью распределений Пуассона, а $g(\nu) = \nu^{\de - 1}e^{-\nu\de}\de^\de / \Ga(\de)$, $\nu, \de > 0$ обозначает плотность распределения гамма с матожиданием $\E[\nu] = 1$ и дисперсией $\V[\nu] = 1/\de$. Тогда можно записать \textbf{отрицательную биномиальную} модель в виде плотности смеси % ALT: смеси плотностей
    \begin{align}\label{eq:20.12}
    h[y|\mu, \de]   &= \int^{\infty}_{0} \frac{e^{-\mu\nu}(\mu\nu)^y}{y!} \frac{\nu^{\de - 1}e^{-\nu\de}\de^\de}{\Ga(\de)} d\nu \\
                    &= \int^{\infty}_{0} \frac{e^{-(\mu + \de)\nu} \mu^y}{y!} \frac{\nu^{y + \de - 1} \de^\de}{\Ga(\de)} d\nu \notag \\
                    &= \frac{\mu^y \de^\de}{\Ga(\de) y!} \int^{\infty}_{0} e^{-(\mu + \de)\nu} \nu^{y + \de - 1} d\nu \notag \\
                    &= \frac{\mu^y \de^\de \Ga(y + \de)}{\Ga(\de) y! (\mu + \de)^{y + \de}} \notag \\
                    &= \frac{\Ga(\al^{-1} + y)}{\Ga(\al^{-1}) \Ga(y + 1)} \left( \frac{\al^{-1}}{\al^{-1} +\mu } \right)^{\al^{-1}} \left( \frac{\mu}{\mu + \al^{-1}} \right)^{y}, \notag
    \end{align}
где $\al = 1/\de$. $\Ga(\cdot)$ обозначает гамма интеграл, который упрощается до факториала от целочисленного аргумента. Для преобразований в четвертой строке используется определение гамма функции. Модель Пуассона $(\al = 0)$, модель `с заменой параметров $\de$ на $\al$' (`\textit{the advantage of reparametrization from $\de$ to $\al$}') и геометрическая модель $(\al = 1)$ являются частными случаями отрицательной биномиальной модели.

Как и в случае с несколькими распределениями смеси, отрицательное биномиальное распределение имеет собственное обоснование; см. Cameron и Trivedi (1998, глава 4). То есть, это распределение не всегда подразумевает распределение смеси, поскольку может быть получено несколькими различными способами.

Алгебраическое выражение отрицательной биномиальной модели как \textbf{смеси Пуассона--гамма} имеет Байесовскую интерпретацию. Априорное распределение $\mu$ является гамма распределением при данных $\al$ и сопряженных априорных распределениях (\textit{conjugate priors}) экспоненциального семейства из раздела 13.2.4. Предполагается, что апостериорное распределение может быть выражено в аналитической форме. Следовательно, оценка ММП и Байесовское апостериорное среднее будут совпадать при предпосылке о неопределенном априорном распределении (\textit{vague prior}) $\al$.

Первые два момента отрицательного биномиального распределения равны
    \begin{align}\label{eq:20.13}
    \E[y|\mu, \al] &= \mu \\
    \V[y|\mu, \al] &= \mu(1 + \al\mu). \notag
    \end{align}
Следовательно, дисперсия превышает матожидание, поскольку $\al > 0$ и $\mu > 0$. Более того, можно легко показать, что избыточная дисперсия возникает всегда, когда $y|\la$ следует распределению Пуассона, а ненаблюдаемая гетерогенность имеет мультипликативную форму $\la = \mu\nu$ с матожиданием $\E[\nu] = 1$. Избыточная дисперсия вида (\ref{eq:20.9}) также представлена в разделе \ref{sec:20.2.4}.

В регрессионном анализе применяются два стандартных типа отрицательной биномиальной модели с $\mu_i = \exp(\xib)$.
Наиболее распространенный из них предполагает оценивание параметра $\al$, и в таком случае условная функция дисперсии $\mu + \al \mu ^2$ из уравнения (\ref{eq:20.13}) имеет квадратичную зависимость от среднего.

Другой тип предполагает линейную функцию дисперсии $\V[y|\mu, \al] = (1 + \ga)\mu$, полученную с помощью замены $\al$ на $\ga/\mu$ в уравнении (\ref{eq:20.12}). Модель можно оценить непосредственно методом максимального правдоподобия, где логарифм правдоподобия легко записать с помощью уравнения (\ref{eq:20.12}). Иногда такой тип называют отрицательной биномиальной моделью 1 (NB1), а модель с квадратичной функцией дисперсии --- отрицательной биномиальной моделью 2 (NB2) (Cameron и Trivedi, 1998). Детали относительно оценивания ММП можно найти, например, в Cameron и Trivedi (1998). Интерпретация коэффициентов в обоих случаях одинакова, поскольку $\E[y|\x] = \exp(\xib)$. Модель NB2 рассматривается далее в разделе \ref{sec:20.7}.

Модель NB2 получила широкое распространение в прикладных исследованиях. В силу своей гибкости она обеспечивает высокое качество подгонки для различных типов счетных данных. Отчасти это объясняется тем, что квадратичная дисперсия является хорошей аппроксимацией во многих эмпирических примерах. К сожалению, сложилось негласное правило, что, если не получается подогнать модель Пуассона, то следует использовать отрицательную биномиальную модель.Такой механистический подход является неверным, поскольку низкое качество модели Пуассона может объясняться, например, неправильной спецификацией функции условного матожидания, а она в обеих моделях совпадает.

Отрицательная биномиальная модель менее робастна к мисспецификации распределения по сравнению с моделью Пуассона. Даже при верной спецификации условного матожидания оценки ММП в отрицательных биномиальных моделях несостоятельны, за исключением модели NB2, где оценки ММП для $\be$ (но не для $\al$) сохраняют состоятельность.

В качестве исходной плотности распределения $f(y|\mu, \nu)$ в моделях смеси для счетных данных логично выбрать плотность распределения Пуассона, поскольку процесс Пуассона является логичным и естественным способом моделирования счетных данных. В отличие от исходного распределения, выбор смешиваемого распределения $g(\nu)$ в (\ref{eq:20.12}) не имеет подобных оснований и остается за исследователем. Обсуждение данного вопроса представлено ранее в разделах 18.2--18.4. В роли смешиваемых распределений могут выступать также лог-нормальное и обратное гауссовское распределения. См. Willmot (1987) и Guo и Trivedi (2002). В таком случае маргинальное распределение не имеет аналитической формы, и для оценивания требуется использовать такие методы, как имитационное максимальное правдоподобие. Современные вычислительные мощности легко позволяют оценивать такие модели. Оценка различных типов смешанных моделей Пуассона становится возможной при использовании методов на основе симуляций, представленных в главе 12.


\subsection{Имитационное максимальное правдоподобие}\label{sec:20.4.2}

\noindent
Исключительно в иллюстративных целях здесь мы покажем, как оценивать модель NB2 \textbf{методом имитационного максимального правдоподобия}. Следует понимать, что на практике в этом нет необходимости, поскольку данная модель уже представлена в аналитическом виде. Предположим, что мы не этого не знаем и попробуем оценить ее с помощью симуляции.

Заметим, что $h(y|\al, \mu)$ в уравнении (\ref{eq:20.12}) может быть аппроксимировано выражением
    $$\frac{1}{S}\sum^{S}_{s = 1} \frac{e^{-\mu\nu_S}(\mu\nu_S)^y}{y!},$$
где $\nu_S(s = 1, ..., S)$ являются псевдо-случайными реализациями распределения $g(\nu|\al)$, а $S$ обозначает число симуляций. Случайные реализации гамма распределения с матожиданием 1 и дисперсией $\al$ можно получить, выбрав значение из равномерного распределения и применив к нему соответствующее преобразование. Пусть $u_S$ обозначает равномерно распределенные случайные величины, а $\nu_S = -\ln u_s/\al$, тогда формула для симуляции будет выглядеть следующим образом
    $$\tilde{f}(y|\nu_S, \al, \mu)\frac{e^{-\mu(-\ln u_S / \al)}(\mu(-\ln u_S/\al))^y}{y!}.$$
Следовательно, оценка $\hat{\ttt}_{\mathrm{MSL}}$ максимизирует сумму
    \begin{align}\label{eq:20.14}
    Q_N(\bttt) = \sum^{N}_{i = 1}\ln \left( \frac{1}{S}\sum^{S}_{s = 1}\tilde{f}(y_i|\x_i, u^S_i, \bttt)\right),
    \end{align}
где $\mu_i = \exp(\xib)$ и $\bttt = (\al, \be)$.

Такой метод прост в применении, но требует компьютерных вычислений. Полное описание свойств метода имитационного максимального правдоподобия представлено ранее в разделе 12.4. % Опечатка в Кэмероне (не глава, а раздел)
Следует также обратить внимание, что при $S, N \rightarrow \infty$, $S/\sqrt{N} \rightarrow 0$ оценки $\hat{\bttt}_{\mathrm{MSL}}$ и $\hat{\bttt}_{\mathrm{ML}}$ асимптотически эквивалентны.\footnote{MSL обозначает метод имитационного максимального правдоподобия, а ML --- метод максимального правдоподобия.}


\subsection{Модели конечной смеси}\label{sec:20.4.3}

\noindent
Модель смеси, рассмотренная в предыдущем разделе, представляла собой модель непрерывной смеси, так как смешиваемая случайная величина $\nu$ имела непрерывное распределение. Альтернативный подход предполагает дискретный вид ненаблюдаемой гетерогенности, что порождает класс моделей, называемых \textbf{моделями конечной смеси} (\textit{finite mixture models}); см. раздел 18.5. Этот класс является определенным подклассом \textbf{моделей латентных классов}. % звучит отвратительно, но это придумал Кэмерон, а не я!
Некоторые типы и частные случаи таких моделей также называются \textbf{дискретными факторными моделями}.

В прикладных исследованиях наиболее распространенной альтернативой непрерывным моделям являются модифицированные модели счетных данных, рассмотренные в следующем разделе. Однако, прежде чем перейти к анализу таких моделей, имеет смысл представить модели конечной смеси как логичное продолжение анализа счетных данных. Позже мы покажем, что подкласс модифицированных моделей является частным случаем конечных смесей.

Предположим, что плотность распределения $y$ является линейной комбинацией $m$ различных плотностей, где $j$-ая плотность равняется $f_j(y|\bttt_j)$, $j = 1, 2, ..., m$. Тогда конечная смесь с $m$ компонентами записывается как
    \begin{align}\label{eq:20.15}
    f(y|\bttt, \bpi) = \sum^{m}_{j = 1} \pi_j f_j (y|\bttt_j), \hspace{0.5cm} 0 \le \pi_j \le 1, \sum^{m}_{j = 1} \pi_j = 1.
    \end{align}

Данная формулировка представляет собой общий случай, когда все компоненты смеси различаются по параметрам. Менее общие модели допускают вариацию среди компонент только некоторых параметров (например, свободного члена), в то время как все остальные параметры одинаковы для каждой компоненты.

Для более простого понимания рассмотрим случай с количеством компонент $m = 2$. Предположим, что совокупность делится на два ``типа'' наблюдений, где исходы $y$ следуют распределениям $f_1 (y|\bttt_1)$ и $f_2 (y|\bttt_2)$ с различными моментами распределения. Предположим, что среднее для 1-го типа совокупности равно $\mu(\bttt_1)$, а для 2-го --- $\mu(\bttt_2)$, где $\mu(\bttt_2) < \mu(\bttt_1)$. В контексте исследования потребления медицинских услуг первая группа может относиться, например, к индивидам, привыкшим посещать врача регулярно, а вторая --- к индивидам, посещающим врача не так часто. Предположим, что доли обеих групп в совокупности равны, соответственно, $\pi_1$ и $\pi_2 = 1 - \pi_1$. Тогда случайная выборка будет содержать $\pi_1$ и $\pi_2$ процентов наблюдений первого и второго типов, хотя мы и не сможем наблюдать, к какой группе относится каждое наблюдение. Таким образом, ``типы'' являются \textbf{латентными классами}.

В данном контексте цель исследователя заключается в оценивании неизвестных параметров $\bttt_j$, $j = 1, ..., m$. Модели регрессии легко построить на основе уравнения (\ref{eq:20.15}). Например, для модели NB2 $f_j(y|\bttt_j)$ является плотностью распределения (\ref{eq:20.12}) с параметрами $\mu_j = \exp(\xb_j)$ и $\al_j$, где $\bttt_j = (\be_j, \al_j)$. Если число компонент $m$ известно, то можно оценить параметры $(\pi, \bttt_j)$, $j = 1, ..., m$ методом максимального правдоподобия при некоторых условиях регулярности.

Полное обсуждение преимуществ и недостатков моделей конечной смеси в контексте моделей времени жизни было представлено ранее в разделе 18.5. Здесь же мы упомянем их вкратце. Во-первых, модель конечной смеси содержит ограниченное число параметров, но при этом является довольно гибким методом моделирования данных, поскольку каждая компонента смеси представляет собой локальное приближение некоторой части истинного распределения. Во-вторых, подход на основе конечной смеси в некотором смысле \textbf{полупараметрический}, так как не требует каких-либо предпосылок о распределении смешиваемой переменной. Наконец, в большинстве случаев результаты легко интерпретируются. Модель особенно привлекательна, если исследователь заинтересован в поведении групп(ы) для анализа проводимой политики. Если забыть про латентные классы, что соответствует случаю $m = 1$, то оценки параметров будут равны взвешенным суммам по параметрам латентного класса.

Трудности могут заключаться в отсутствии теоретического обоснования для выбора числа компонент; более того, некоторые компоненты могут быть неразличимы друг с другом из-за отсутствия существенных различий в данных. В таком случае поступают следующим образом: обычно начинают с небольшого количества компонент, а затем добавляют дополнительные компоненты, если такое действие сопровождается значительным повышением качества модели. Иногда допускается вариация только свободного члена, в то время как коэффициенты наклона должны быть одинаковы между компонентами. Однако следует быть аккуратным, поскольку выборочные свойства оценок максимального правдоподобия полностью не изучены для случая с неизвестным числом параметров $m$.

Существует ряд работ, свидетельствующих о хорошем качестве моделей конечной смеси для счетных данных по медицинскому обслуживанию (Deb и Trivedi, 1997; 2002). Это можно объяснить, например, тем, что совокупность индивидов делится на основе латентной переменной, отвечающей за состояние здоровья. В среднем, здоровые индивиды могут предъявлять низкий спрос на медицинские услуги, в то время как менее здоровые будут поддерживать высокий спрос. Другими словами, модели конечной смеси позволяют разделить совокупность на группы, когда состояние здоровья полностью ненаблюдаемо.


\subsection{Урезанные и цензурированные данные}\label{sec:20.4.4}

\noindent
В некоторых исследованиях в выборку включаются только те наблюдения, которые удовлетворяют интересующим характеристикам. Поскольку мы не наблюдаем объекты, которые им не удовлетворяют, и, как следствие, объекты с нулевыми значениями, счетные данные оказываются \textbf{урезаны} (усечены). Примером таких данных может быть число поездок, совершенных автобусом в течение недели, число походов в торговый центр индивидом на основе опроса, проводимого в торговом центре, или число периодов безработицы среди запаса (совокупности) безработных. Во всех этих случаях мы не наблюдаем нулевые значения, поэтому мы говорим, что такие данные содержат \textbf{урезанные нулевые значения}. В общем случае, такие данные называются урезанными слева. Урезание (усечение) справа возникает из-за потери наблюдений, значение которых превышает определенную величину.

Полное описание урезанных и цензурированных моделей, оцениваемых ММП, представлено в разделе 16.2. Здесь мы рассмотрим их применение по отношению к счетным данным.

Урезание приводит к несостоятельным оценкам параметров в случае, если отсутствует соответствующая поправка для функции правдоподобия. Рассмотрим пример с урезанными нулевыми значениями. Пусть $f(y|\bttt)$ обозначает функцию плотности, а $F(y|\bttt) = \Pr[Y \le y]$ --- кумулятивную функцию распределения дискретной случайной величины, где $\bttt$ --- вектор параметров. Если реализация $y$, не превышающая единицу, пропущена, то плотность с учетом урезанных нулей будет равна
    \begin{align}\label{eq:20.16}
    f(y|\bttt, y \ge 1) = \frac{f(y|\bttt)}{1 - F(0|\bttt)}, \hspace{0.5cm} y = 1, 2, ... .
    \end{align}
На основе уравнения \ref{eq:20.16} можно построить модель Пуассона с \textbf{урезанными нулевыми значениями}, например, с функцией $f(y|\mu, y \ge 1) = e^{-\mu}\mu^y / [y!(1 - \exp(-\mu))]$. Тогда легко найти логарифм правдоподобия и получить оценки ММП.

\textbf{Цензурированные счетные данные} чаще всего возникают из-за агрегирования счетных данных, значение которых превышает определенную величину. Агрегирование характерно для обследований, где общая вероятность по агрегированным данным относительно невелика.
Важное различие между урезанными и цензурированными наблюдениями заключается в том, что при цензурировании ковариаты наблюдаемы, в то время как при урезании ни исходы, ни ковариаты ненаблюдаемы. Ценузирование также приводит к несостоятельным оценкам параметров, если применяется ошибочная функция правдоподобия для нецензурированных данных; см. раздел 16.2.

Например, число событий, превышающее некоторое пороговое значение $c$, может быть объединено в одну группу. Тогда некоторые значения $y$ будут наблюдаемы неполностью: для определенного наблюдения будет известно лишь то, что его значение равно или превышает некую константу $c$. Тогда плотность наблюдаемых данных будет равна
    \begin{align}\label{eq:20.17}
    g(y|\bttt) =\begin{cases}
                f(y|\bttt),              & \text{ если }y < c, \\
                1 - F(c - 1|\bttt),      & \text{ если }y \ge c
                \end{cases}
    \end{align}
при известном значении $c$.

Подобные трудности возникают при \textbf{отборе выборки} (Terza, 1998). Мы наблюдаем счетную переменную $y$ только тогда, когда другая случайная величина, возможно коррелированная с $y$, превышает определенное пороговое значение. Например, попасть к врачу-специалисту можно лишь предварительно записавшись к терапевту, который и выписывает соответствующие направления.


\subsection{Модифицированные модели счетных данных}\label{sec:20.4.5}

\noindent
Задача модифицированных моделей счетных данных, рассматриваемых в данном разделе, заключается в том, чтобы решить так называемую проблему \textbf{избыточных нулевых значений}, возникающую, когда в данных содержится больше нулей, чем предсказывает модель Пуассона или NB2.

        \begin{center}{Модель преодоления порогов}\end{center} % или модель, состоящая из двух частей
\noindent
\textbf{Модель преодоления порогов} (\textit{hurdle model}), или \textbf{двухчастная модель} (\textit{two-part model}) (см. раздел 16.4), снимает % убирает/отказывается/избавляется/ослабляет
предпосылку о том, что нулевые и ненулевые значения порождаются одним и тем же процессом.
В то время как нули распределены с плотностью $f(\cdot)$ так, что $\Pr[y = 0] = f_1(0)$, положительные значения имеют усеченную плотность распределения $f_2(y|y > 0) = f_2(y) / (1 - f_2(0))$, умноженную на $\Pr[y > 0] = 1 - f_1(0)$, что гарантирует, что сумма вероятностей равняется единице. Таким образом,
    \begin{align}\label{eq:20.18}
    g(y) =\begin{cases}
                f_1(0),                                     & \text{ если }y = 0, \\
                \frac{1 - f_1(0)}{1 - f_2(0)} f_2(y),       & \text{ если }y \ge 1.
                \end{cases}
    \end{align}
Следовательно, в модифицированной модели процессы, порождающие нулевые и положительные значения, различаются. Стандартная модель получается при $f_1(\cdot) = f_2(\cdot)$. Хотя модель и построена таким образом, чтобы решить проблему избыточных нулей, она также подходит для ситуаций, когда количество нулей слишком мало.

Оценивание модели преодоления порогов методом максимального правдоподобия подразумевает максимизацию двух частей правдоподобия по отдельности, одна из которых отвечает за нулевые значения, а другая за положительные.

Интерпретация модели преодоления порогов заключается в том, что она отражает двухэтапный процесс принятия решения. Например, решение о первичном посещении врача может быть принято самим пациентом, но второй и последующие визиты могут определяться другими механизмами (Pohlmeier и Ulrich, 1995).

В регрессионном анализе применяются различные варианты модели преодоления порогов, построенные на основе спецификации распределений $f_1(\cdot)$ и $f_2(\cdot)$ в виде плотности распределений Пуассона или отрицательного биномиального, представленных ранее. Набор регрессоров в первой части модели не должен совпадать с набором во второй (усеченной), хотя на практике они часто одинаковы. В силу своей гибкости отрицательная биномиальная модель преодоления порогов получила широкое применение в анализе счетных данных. Вместе с этим она требует оценивания множества параметров, количество которых обычно удваивается по сравнению со стандартной моделью. При этом, их интерпретация также не всегда очевидна.

Выбор распределения в модели преодоления порогов играет важную роль. Использование более гибких распределений создает преимущества для отрицательной биномиальной модели перед моделью Пуассона. Условное матожидание в модели преодоления препятствия является произведением вероятности положительных значений и условного матожидания для плотности с урезанными нулевыми значениями. Следовательно, применение регрессии Пуассона, в то время, когда модель преодоления порогов является истинной моделью, приводит к мисспецификации, и, как следствие, к несостоятельным оценкам. Более того, из-за формы условного среднего, расчет предельных эффектов оказывается сложнее, аналогично модели, состоящей из двух частей, из раздела 16.4.

        \begin{center}{Модель с нулевыми значениями}\end{center}
\noindent
Следующей модифицированной моделью является \textbf{модель с нулевыми значениями} (\textit{with-zeros model} или \textit{zero-inflated model}). Идея заключается в том, что бинарный процесс с плотностью $f_1(\cdot)$ дополняет плотность распределения счетных данных $f_2(\cdot)$. То есть, если бинарный процесс принимает значение $0$ с вероятностью $f_1(0)$, то $y = 0$. Если же значение равно $1$ с вероятностью $f_1(1)$, то $y$ может принимать любые целые неотрицательные значения $0, 1, 2, ...$, в соответствии с плотностью распределения $f_2(\cdot)$. Таким образом, нули могут возникать двумя способами: как реализация бинарного процесса и как реализация счетного процесса при условии, что бинарный процесс принимает значение $1$. Следовательно, плотность равна
    \begin{align}\label{eq:20.19}
    g(y) =\begin{cases}
                f_1(0) + (1 - f_1(0)) f_2(0),              & \text{ если }y = 0, \\
                (1 - f_1(0))f_2(y),                        & \text{ если }y \ge 1.
                \end{cases}
    \end{align}
В регрессионных моделях плотность $f_1(\cdot)$ задается логит моделью, а $f_2(\cdot)$ --- плотностью распределения Пуассона или отрицательного биномиального. Данная модель менее популярна, чем модель преодоления порогов. При этом, она также подходит для моделирования ситуаций с недостаточным количеством нулей.

Заметим также, что в эконометрике счетные модели с нулевыми значениями используются гораздо реже, чем в других статистических дисциплинах.


\subsection{Модели дискретного выбора}\label{sec:20.4.6}

\noindent
Для моделирования счетных данных также подходят модели дискретного выбора. Предварительная группировка данных может потребоваться, чтобы ограничить число категорий. Например, если незначительное число наблюдений превышает четыре, то данные можно сгруппировать по пяти категориям как 0, 1, 2, 3 и 4 и более. Неупорядоченные модели, такие как логит модель множественного выбора из раздела 15.4, используют слишком много параметров и, что более важно, не подходят для работы со счетными данными. Следовательно, нужно использовать последовательные модели, то есть, те, которые учитывают порядок.

Одной из таких моделей является \textbf{модель упорядоченного множественного выбора} (\textit{ordered model}). % Термин должен совпадать!!!
Зависимая переменная принимает значения $y = 0, 1, 2, ...$, соответствующие пороговым значениям латентной переменной $y^* = \xb + u$, которые также подлежат оценке. Логит (или пробит) модель упорядоченного множественного выбора можно получить, предположив, что $u$ следует логистическому (или стандартному нормальному) распределению. Упорядоченные модели (см. раздел 15.9) особенно полезны в случае, если счетные данные могут принимать отрицательные значения, как, например, при моделировании изменений, таких как изменение числа фирм в отрасли.

Другую последовательную модель, хотя и с б\'{о}льшим числом оцениваемых параметров, можно построить на основе последовательности моделей бинарного выбора с вероятностями $\Pr[y = 1|y \ge 0]$, $\Pr[y = 2|y \ge 1]$ и так далее.

Наконец, в некоторых случаях помимо счетных данных могут быть доступны данные по длительностям. Например, если известны даты посещения врача, то можно смоделировать как число визитов в месяц, так и длительности между визитами. В общем случае, последний подход более эффективен, поскольку он использует более детальные данные. Однако регрессия для счетных данных также предоставляет полезную информацию о роли ковариат (Dean и Balshaw, 1997).




\section{Частично параметрические модели}\label{sec:20.5}

\noindent
Частично параметрические модели подразумевают моделирование данных с помощью условных матожидания и дисперсии, однако даже их спецификация может быть неполностью определена. В разделе \ref{sec:20.5.1} мы рассмотрим модели, основанные на спецификации условных матожидания и дисперсии. В разделе \ref{sec:20.5.2} мы представим применение методов наименьших квадратов и их критику. В разделе \ref{sec:20.5.3} мы рассмотрим еще менее параметрические модели с неполной спецификацией условного матожидания.

Такой подход аналогичен нелинейному МНК, за исключением того, что мы допускаем наличие гетерогенности, заданной в виде функции от условного матожидания.


\subsection{Оценивание квази-ММП}\label{sec:20.5.1}

\noindent
Как было сказано в разделе \ref{sec:20.2.1}, при использовании псевдо- или квази-ММП итоговое распределение оценок не требует столь же строгих предпосылок о процессе, генерирующем данные, что и определенная функция правдоподобия.

Пересмотрим формулу (\ref{eq:20.6}). При определенной предпосылке о функциональной форме $\omega_i$ и состоятельной оценке $\hat{\omega_i}$ для $\omega_i$ можно получить состоятельную оценку матрицы ковариаций. В качестве предпосылки можно использовать распределение Пуассона с $\omega_i = \mu_i$, однако, как уже упоминалось, для счетных данных характерна избыточная дисперсия, что означает, что $\omega_i > \mu_i$. Другими распространенными функциями являются $\omega_i = (1 + \al \mu_i) \mu_i$ в модели NB2, представленной в разделе \ref{sec:20.4.2}, и $\omega_i = (1 + \al) \mu_i$ в модели NB1. Заметим, что в последнем случае выражение (\ref{eq:20.6}) упрощается до $\V_{\mathrm{PML}}[\hat{\be}_{\mathrm{P}}] = (1 + \al) (\sum_i \mu_i \x_i \x'_i)^{-1}$, следовательно, при наличии избыточной дисперсии $(\al > 0)$ матрица ковариаций (\ref{eq:20.7}) недооценивает истинную дисперсию.

Если спецификация функциональной формы $\omega_i = \E[(y_i - \xib)^2|\x_i]$ отсутствует, то состоятельную оценку $\V_{\mathrm{PML}}[\hat{\be}_{\mathrm{P}}]$ можно получить, применив формулу Эйкера–-Уайта (\textit{Eicker--White}). Сумма в середине уравнения (\ref{eq:20.6}) подлежит оцениванию. Если $\hat{\mu}_i \xrightarrow{\text{p}} \mu_i$, то $N^{-1} \sum_i (y_i - \hat{\mu}_i)^2 \x_i \x'_i \xrightarrow{\text{p}} \lim N^{-1} \sum_i \omega_i \x_i \x'_i$. Следовательно, произведя замену $\omega_i$ и $\mu_i$ на $(y_i - \hat{\mu}_i)^2$ и $\hat{\mu}_i$ в уравнении (\ref{eq:20.6}), найдем состоятельную оценку для $\V_{\mathrm{PML}}[\hat{\be}_{\mathrm{P}}]$.

В случае если существуют сомнения относительно функциональной формы дисперсии, рекомендуется использование псевдо-ММП. В вычислительном плане данный метод не отличается от ММП Пуассона, с той лишь оговоркой, что матрица дисперсии должна быть пересчитана. Стандартные статистические пакеты, как правило, предоставляют возможность расчета робастных оценок дисперсии.

Результаты оценивания псевдо-ММП Пуассона качественно похожи на оценивание линейной модели псевдо-ММП при условии нормальности. В общем случае, их можно обобщить до оценивания псевдо-ММП на основе плотностей экспоненциального семейства распределений. Так или иначе, для состоятельности требуется только верная спецификация условного матожидания (Nelder и Wedderburn, 1972; Gouri\'eroux et al., 1984a), что позволяет осуществлять корректные выводы и строить различные модели --- непрерывную (нормальное распределение), счетных данных (распределение Пуассона), дискретную (биномиальное распределение), положительную (гамма распределение), что представлено в разделе 5.7.4. Это является основой для значительного объема литературы, посвященной обобщенным линейным моделям (McCullagh и Nelder, 1989). Многие методы, такие как временные ряды и панельные данные, представлены в контексте обобщенной линейной модели (\textit{generalized linear model, GLM}).

Некоторые эконометристы считают, что вместо обобщенной линейной модели логично использовать обобщенный метод моментов (\textit{generalized method of moments, GMM}), где исходной точкой для анализа является условный момент $\E[y_i - \exp(\xib)|\x_i] = \0$. Если данные независимы по наблюдениям $i$ и условная дисперсия является функцией от матожидания, то можно показать, что оптимальный набор инструментов равен $\x_i$, что в результате приводит к оцениваемым уравнениям (\ref{eq:20.5}); детали можно найти в Cameron и Trivedi (1998, стр. 37--44). Обобщенный метод моментов хорошо зарекомендовал себя при работе с панельными счетными данными (см. раздел \ref{sec:20.5.3}) и \textbf{эндогенными регрессорами}. Анализ полностью параметрических моделей с одновременными уравнениями для счетных данных еще только начинает развиваться, поэтому методы инструментальных переменных выглядят довольно многообещающе. При данных инструментах $\bz_i$, $\dim (\bz) \ge \dim(\x)$, удовлетворяющих условию $\E[y_i - \exp(\xib)|\bz_i] = \0$, состоятельная оценка для $\be$ минимизирует выражение
    \begin{align}\label{eq:20.20}
    Q(\be) = \left[ \sum^{N}_{i = 1} (y_i - \exp(\xib))\bz_i\right]' \mathbf{W} \left[ \sum^{N}_{i = 1} (y_i - \exp(\xib))\bz_i\right],
    \end{align}
где $\mathbf{W}$ является симметричной взвешивающей матрицей.

Среди преимуществ такого подхода можно выделить небольшое число предпосылок о распределениях, что позволяет избежать возможной мисспецификации. Однако, данный подход не учитывает дискретность данных и свойственную им гетероскедастичность, что приводит к потере эффективности. Проблему можно отчасти решить с помощью соответствующей матрицы $\mathbf{W}$. Более того, акцентируя внимание на первом моменте распределения, можно упустить важную дополнительную информацию, которая содержится в моментах более высокого порядка, в результате чего инструментальная оценка будет чувствительна к наличию больших чисел. В таблице \ref{tab:20.2} представлены некоторые особенности типов данных, для работы с которыми обобщенный метод моментов не подходит.


\subsection{Оценивание МНК}\label{sec:20.5.2}

\noindent
При моделировании только условного матожидания методы наименьших квадратов уступают подходам, рассмотренным в предыдущем разделе.

Оценки параметров в \textbf{линейной регрессии МНК} $y$ на $\x$ состоятельны, если условное матожидание линейно по $\x$. Однако, спецификация $\E[y|\x] = \xib$ неадекватна счетным данным, так как допускает отрицательные значения $\E[y|\x]$. По той же причине линейная модель вероятности не подходит для бинарных данных.

Для учета неотрицательности можно рассматривать функциональные преобразования $y$, в частности, логарифмическое, что соответствует регрессии $\ln y$ на $\x$. Однако, такое преобразование может быть проблематично, так как счетные данные обычно содержат нулевые значения. В качестве стандартного решения можно прибавить константу, например $0.5$, и работать уже с $\ln (y + .5)$. В свою очередь, такой специальный метод связан с проблемами обратного преобразования, если мы заинтересованы в анализе $\E[y|\x]$, а не $\E[\ln y|\x]$; см. Mullahy (1998).
Тем не менее, переход к линейной модели удобен, если в уравнении содержится эндогенная правая часть, которая требует применения инструментальных методов.

Вместо логарифмического преобразования можно использовать нелинейный МНК (НМНК) с экспоненциальным матожиданием, что соответствует оцениванию нелинейной регрессии $y = \exp(\xib) + u$. Важно, чтобы статистические выводы были основаны на робастных стандартных ошибках Эйкера--Уайта, поскольку ошибка в данной регрессии гетероскедастична.

Как правило, нелинейный МНК на счетных данных менее эффективен, чем псевдо-ММП Пуассона. Условие первого порядка для НМНК выглядит как $\sum_i (y_i - \exp(\xib))\exp(\xib)\x = \0$. То есть, остатки при оценивании НМНК взвешены иначе, чем при оценивании псевдо-ММП Пуассона (см. (\ref{eq:20.5})). Веса НМНК оптимальны, если дисперсия $\V[y_i|\x_i]$ постоянна (гомоскедастична), веса псевдо-ММП оптимальны, если $\V[y_i|\x_i]$ является функцией от $\E[y_i|\x_i]$. Следовательно, последняя модель лучше учитывает свойственную счетным данным гетероскедастичность.


\subsection{Полупараметрические модели}\label{sec:20.5.3}

\noindent
Под \textbf{полупараметрическими моделями} подразумеваются частично параметрические модели, содержащие компоненту с бесконечной размерностью и представленные в разделе 9.7. Проклятие размерности служит основанием для того, чтобы определять функциональную форму матожидания.

Один из классов полупараметрических моделей, включающий одноиндексные и частично линейные модели, предполагает неполную спецификацию условного матожидания. В частности, одноиндексные модели определяют $\mu_i = g(\xib)$, где $g(\cdot)$ неизвестна. Частично линейные модели определяют $\mu_i = \exp(\xib + g(\bz_i))$, где функциональная форма $g(\cdot)$ остается неопределена. В обоих случаях $\sqrt{N}$-состоятельные асимптотически нормальные оценки $\be$ могут быть получены при неизвестной $g(\cdot)$

Другой класс моделей предполагает, что $\mu_i = \exp(\xib)$, в то время как $\V[y_i|\x] = \omega_i$ остается неизвестной. Бесконечная размерность имеет место, так как при $N \rightarrow \infty$ существует бесконечное множество параметров дисперсии $\omega_i$. Оптимальная оценка для $\be$, называемая адаптивной оценкой, обладает той же эффективностью, что и оценка при известном параметре $\omega_i$. Delgado и Kniesner (1997) рассматривают случай линейной модели регрессии для счетных данных с экспоненциальной функцией условного матожидания, используя методы ядерной регрессии для оценки весов, применяемых на втором шаге оценивания регрессии НМНК. Такая оценка обладает незначительным преимуществом по сравнению со спецификацией избыточной дисперсией в форме NB2, $\omega_i = \mu_i (1 + \al\mu_i)$.


\section{Многомерные счетные данные и эндогенные регрессоры}\label{sec:20.6}

\noindent
В данном разделе мы представим краткое описание моделей для пространственных данных, обобщенных на другие типы счетных данных (для дальнейшего анализа см. Cameron и Trivedi, 1998). Несмотря на то, что существует значительное количество моделей для анализа многомерных счетных данных, какие-либо общепринятые методы отсутствуют. Относительно панельных данных в эконометрической литературе больше договоренности о том, какие методы использовать, хотя в статистической литературе рассматривается еще более широкий круг моделей; см. раздел 23.7.


\subsection{Многомерные данные}\label{sec:20.6.1}

\noindent
Иногда данные содержат информацию по нескольким переменным. Например, могут быть доступны данные по потреблению различных медицинских услуг, таких как число визитов к врачу и число дней, проведенных в госпитале. В случае если такие переменные коррелированы, совместное моделирование повышает эффективность оценок и позволяет строить более качественные модели. Данный раздел представляет собой краткий обзор \textbf{двумерных моделей счетных данных}, построенных на основе базовых моделей, представленных в этой главе. Читатель, знакомый с такими моделями, как \textbf{модель с внешне несвязанными уравнениями} (\textit{SUR}) из раздела 6.9.3, может сопоставить их с моделями счетных данных, содержащими несколько уравнений с коррелированными ошибками. Предположим, что для определенного индивида мы наблюдаем несколько переменных (например, число посещений врача и число прописанных лекарств), причина коррелированности которых может заключаться в ненаблюдаемой гетерогенности. Совместное оценивание, учитывающее коррелированность ошибок, позволяет получить более эффективные оценки, однако для этого потребуются дополнительные компьютерные вычисления.

        \begin{center}{Полупараметрические методы}\end{center}
        \noindent
Частично параметрический подход можно рассматривать как задачу оценки внешне несвязанных уравнений с применением методов для оценки линейных моделей регрессии к счетным данным, с учетом гетероскедастичности и нелинейности условных матожиданий; см. раздел 6.10.3.

Gouri\'eroux, Monfort и Trognon (1984b) предлагают основанный на моментах метод, который позволяет получить двумерную модель типа Пуассона. В частности, авторы определяют первые два момента распределения $y_1$ и $y_2$ и оценивают их квази-обобщенным псевдо-ММП. Такая модель является более общей, чем модель Пуассона, и допускает наличие избыточной дисперсии; однако она не учитывает целочисленность счетных данных.

Delgado (1992) рассматривает многомерную модель счетных данных как многомерную нелинейную модель и предлагает полупараметрический обобщенный МНК. Матрица ковариаций остатков оценивается с помощью метода $k$-NN. Этот подход отличается от подхода Gouri\'eroux, Monfort и Trognon (1984b) типом оценки матрицы ковариаций.

Многие параметрические исследования используют \textbf{двумерное распределение Пуассона}. Распределение можно найти, предположив, например, что две счетных переменных $y_1$ и $y_2$ получены как $y_1 = \zeta_1 + \omega$ и $y_2 = \zeta_2 + \omega$, где все $\zeta_1$, $\zeta_2$ и $\omega$ независимы и распределены по Пуассону с положительными параметрами $\la_1$, $\la_2$ и $\la_{12}$, соответственно. Эти параметры, в свою очередь, задаются в виде функции от экзогенных ковариат. Такой метод называется \textbf{трехмерной редукцией} (\textit{trivariate reduction}).

Маргинальное распределение $y_j$ является распределением Пуассона с параметром $[\la_j + \la_{12}]$.
% $\sim \mathrm{Poisson}[\la_j + \la_{12}]$ ???
Следовательно, эта модель предполагает равенство условных матожидания и дисперсии для каждой счетной переменной, то есть
    \begin{align}\label{eq:20.21}
    \E[y_j|\x_j] = \V[y_j|\x_j]
    \end{align}
для $j = 1, 2$, где $\x_j$ обозначает вектор объясняющих переменных. Так как $\la_{12} > 0$, коэффициент корреляции положителен и равен
    \begin{align}\label{eq:20.22}
    \mathrm{Cor}[y_1, y_2] = \frac{\la_{12}}{\sqrt{(\la_1 + \la_{12})(\la_2 + \la_{12})}}.
    \end{align}


        \begin{center}{Полностью параметрические методы}\end{center}
        \noindent
Параметрические модели можно улучшить за счет включения коррелированной ненаблюдаемой гетерогенности для каждой счетной переменной. С этой идеей тесно связаны вопросы, рассмотренные в разделах 6.10.1 и 19.3.

Marshall и Olkin (1990) рассматривают следующую модель с \textbf{мультипликативной ненаблюдаемой гетерогенностью} в маргинальных распределениях обеих счетных переменных. Пусть $y_j \sim \mathcal{P}[\la_j\nu]$, $j = 1, 2$, где $\mathcal{P}$ обозначает распределение Пуассона с матожиданием $\la_j\nu$. Гетерогенность $\nu$ имеет гамма распределение с плотностью
        $$g(\nu) = \frac{\nu^{\al - 1}\exp(-\nu)}{\Ga(\al)}.$$
Случайную переменную $\nu$ можно интерпретировать как общую (распределенную) ненаблюдаемую гетерогенность. Итоговая модель является \textbf{однофакторной моделью}. \textbf{Двумерное отрицательное распределение} (\textit{BVNB}) двух счетных переменных задается как
    \begin{align}\label{eq:20.23}
    f(y_1, y_2|\x_1, \x_2)  &= \int^{\infty}_{0} f_1(y_1|\x_1, \nu) f_2(y_2|\x_2, \nu) g(\nu) d\nu \\
                            &= \int \left[ \prod^2_{j = 1}\frac{\exp(-\la_j\nu)(\la_j\nu)^{y_j}}{y_j!}  \right] \frac{\nu^{\al - 1} \exp(-\nu)}{\Ga(\al)} d\nu \notag \\
                            &= \frac{\Ga(y_1 + y_2 + \al)}{y_1! y_2! \Ga(\al)} \left[ \frac{\la_1}{\la_1 + \la_2 + 1} \right]^{y_1} \left[ \frac{\la_2}{\la_1 + \la_2 + 1} \right]^{y_2} \notag \\
                            &\times \left[ \frac{1}{\la_1 + \la_2 + 1} \right]^{\al}. \notag
    \end{align}

Данная смесь имеет аналитическое решение, однако ненаблюдаемая гетерогенность должна быть идентична для обеих счетных переменных. Совместное правдоподобие состоит из элементов, аналогичных (\ref{eq:20.23}). Маргинальные распределения представляют собой одномерные отрицательные биномиальные распределения. Корреляция между переменными положительна и равна
    \begin{align}\label{eq:20.24}
    \mathrm{Cor}[y_1, y_2] = \frac{\la_1 \la_2}{\sqrt{(\la_1^2 + \al \la_1)(\la_2^2 + \al \la_2)}}.
    \end{align}

Другие модели с более \textbf{гибкой структурой корреляций}, но при этом требующие применения продвинутых вычислительных методов, были предложены в работах Cameron и Johansson (1998), Munkin и Trivedi (1999) и Chib и Winkelmann (2001).

Munkin и Trivedi (1999) рассматривают обобщенную BVNB модель в виде
    \begin{align}\label{eq:20.25}
    f(y_1, y_2 | \x_1, \x_2) = \int^{\infty}_{0} \int^{\infty}_{0} f_1(y_1|\x_1, \nu_1) f_2(y_2|\x_2, \nu_2) g(\nu_1, \nu_2) d\nu_1 d\nu_2,
    \end{align}
где совместное распределение состоит из двух маргинальных моделей, каждая из которых определена при условии соответствующей переменной ненаблюдаемой гетерогенности, $\nu_1$ или $\nu_2$. Гетерогенность задается таким образом, чтобы получалось двумерное нормальное распределение. При условии $(\x_1, \x_2, \nu_1, \nu_2)$ каждое маргинальное распределение является распределением Пуассона с мультипликативной ненаблюдаемой нормальной гетерогенностью. Следовательно, такая модель называется \textbf{двумерной смесью Пуассона и лог-нормального распределений}. Функция правдоподобия представляет собой произведение элементов, аналогичных (\ref{eq:20.25}). В отличие от предыдущей, данная модель является ``\textbf{двухфакторной моделью}''. Такая спецификация является более гибкой, поскольку не накладывает ограничений на размер или знак коэффициента корреляции между двумя ненаблюдаемыми компонентами. Поскольку интеграл в уравнении (\ref{eq:20.25}) не имеет аналитического решения, такая дополнительная гибкость создает трудности, связанные с вычислением, которое подразумевает применение методов, основанных на симуляции (представленных в главе 12 и в Munkin и Trivedi (1999)). % опечатка в Кэмероне
При увеличении числа переменных $y$, увеличивается и порядок численного интегрирования, что наряду с большим объемом выборки может создавать серьезные вычислительные проблемы. Chib и Winkelmann (2001) предлагают альтернативный Байесовский MCMC подход, сочетающий как гибкость вышеописанной спецификации, так и высокую размерность. Авторы демонстрируют доступность такого подхода с помощью шестимерной смеси Пуассона и лог-нормального распределений.

Еще одним альтернативным подходом к моделированию коррелированных счетных данных является \textbf{подход на основе копула-функций}, описанный ранее в разделе 19.3. Вначале определяется спецификация маргинальных распределений, а затем за счет их объединения с помощью копула-функции получается совместное распределение. Примеры с зависимыми длительностями представлены в разделе 19.3. См. также Cameron, Li, Trivedi и Zimmer (2004).


\subsection{Модели счетных данных с эндогенными регрессорами}\label{sec:20.6.2}

\noindent
Модели одновременных уравнений для счетных данных подходят для описания различных ситуаций. Например, в работе Cameron et al. (1988) в качестве зависимой счетной переменной выступает потребление медицинских услуг, и, как следствие, регрессор, отражающий состояние здоровья объекта, является эндогенным. Mullahy (1997) и Cre\'epon и Duguet (1997b) применяют обобщенный метод моментов к моделям счетных данных с эндогенными регрессорами на пространственных и панельных данных, соответственно. В известном примере по моделированию таких медицинских услуг, как визиты к врачу, один из регрессоров соответствует состоянию здоровья индивида. Поэтому предпосылка о независимости выбора программы страхования и ошибки нереалистична, и, как следствие, переменная страхования является эндогенной. Примеры и подробности для панельных моделей счетных данных с эндогенными регрессорами можно найти в главе 22.

На данный момент в эконометрике существует два подхода к оцениванию моделей с эндогенными регрессорами, один из которых основан на процедуре GMM/IV, а второй использует более строгие предпосылки о максимальном правдоподобии. Мы рассмотрим каждый из них по очереди.

Первый подход (Mullahy, 1997) вначале определяет условия на момент. Рассмотрим модель с экспоненциальным матожиданием, где ошибка аддитивна и обладает нулевым средним
    \begin{align}\label{eq:20.26}
    y_i = \E[y_i|\x_i] + \nu_i = \exp(\xib) + \nu_i,
    \end{align}
    \begin{align}\label{eq:20.27}
    \E[\nu_i|\x_i] \ne 0.
    \end{align}
Предположим, что имеется набор инструментальных переменных $\bz_i$, которые удовлетворяют моментным тождествам
    \begin{align}\label{eq:20.28}
    \E[y_i|\bz_i]    &= 0, \\
    \E[y_i - \exp(\xib)\bz_i] &= 0. \notag
    \end{align}
Тогда процедура GMM/IV реализуема, при условии, что мы имеем достаточно моментных тождеств. Подробности по вопросам ее применения и другим возникающим вопросам можно найти в разделе 6.5.3. Заметим, что данный подход не учитывает счетный характер переменной, поэтому итоговая модель аналогична любой другой нелинейной модели с экспоненциальным матожиданием. Также, в данных, скорее всего, присутствует гетероскедастичность, следовательно, модель GMM/IV должна учитывать и эту характерную черту данных.

Mullahy отметил, что мультипликативная спецификация ошибки имеет определенные преимущества, что, однако, соответствует другим условиям на момент. Пусть
    \begin{align}\label{eq:20.29}
    \E[y_i|\x_i, \nu_i] = \exp(\xib)\nu_i.
    \end{align}
Тогда условие на момент является частным случаем нелинейного моментного тождества $\E[r(y_i, \x_i, \be)|\bz_i] = 0$, рассмотренного в разделе 6.5, и равняется
    \begin{align}\label{eq:20.30}
    \E\left[ \frac{y_i}{\exp(\xib)} - 1| \bz_i \right] = 0,
    \end{align}
Метод GMM применим, если доступны подходящие и достаточные моментные тождества. Повторим, что для счетной переменной свойственна гетероскедастичность и потери эффективности возникают из-за того, что счетный характер переменной проигнорирован.

Альтернативные подходы, учитывающие одновременно и счетность зависимой переменной, и проблему эндогенных регрессоров, являются более \textbf{параметрическими} (Terza, 1998). Deb и Trivedi (2004) предложили совместную модель подсчетов $(Y)$ с планом страхования $(D)$ в качестве объясняющей переменной и модель бинарного выбора плана страхования. Эндогенность в данном случае возникает из-за наличия коррелированной ненаблюдаемой гетерогенности в уравнении счетных данных и уравнении бинарного выбора.
Модель имеет следующую структуру:
    \begin{align}\label{eq:20.31}
    \Pr[Y_i = y_i|\x_i, D_i, l_i] = f(\xib + \ga_1 D_i + \la l_i),
    \end{align}

    \begin{align}\label{eq:20.32}
    \Pr[D_i = 1|\bz_i, l_i] = g(\bz'_i \bm{\al} + \de l_i),
    \end{align}
где $l_i$ являются \textbf{латентными факторами}, отражающими ненаблюдаемую гетерогенность, а $\de$ и $\la$ обозначают соответствующую факторную нагрузку. Предполагается, что $(Y, D)$ условно независимы, и совместное распределение зависимой (счетной) переменной и переменной отбора, условных на общие латентные факторы, может быть записано как
    \begin{align}\label{eq:20.33}
    \Pr[Y_i = y_i, D_i = 1|\x_i, \bz_i, l_i] = f(\xib + \ga_1 d_i + \la l_i) g(\bz'_i \bm{\al} + \de l_i).
    \end{align}

Так как $l_i$ неизвестны, при оценивании могут возникнуть проблемы. Однако, хотя мы и не знаем $l_i$, мы можем предположить, что известно их распределение $h$. Следовательно, его можно исключить из функции совместной плотности с помощью интегрирования
    \begin{align}\label{eq:20.34}
    \Pr[Y_i = y_i, D_i = 1|\x_i, \bz_i] = \int [f(\xib + \ga_1 D_i + \la l_i)g(\bz'_i\bm{\al} + \de l_i)]g(l_i)dl_i.
    \end{align}
В такой форме оценки неизвестных параметров модели можно получить методом максимального правдоподобия.

Для простоты предположим, что все параметры распределения $h(l_i)$ известны. Тогда оценки ММП соответствуют максимуму совместной функции правдоподобия $\mL(\bttt_1, \bttt_2|y_i, D_i, \x_i, \bz_i)$, где $\bttt_1 = (\be, \ga_1, \la)$ и $\bttt_2 = (\bm{\al}, \de)$ обозначают набор параметров в уравнении счетных данных и уравнении бинарного выбора, соответственно, а $\mL$ обозначает совместное правдоподобие, где $i$-ая компонента соответствует уравнению (\ref{eq:20.34}). Для идентификации, однако, могут потребоваться дополнительные условия нормализации.

С практической точки зрения основной проблемой является то, что в общем случае интеграл не может быть представлен в аналитическом виде при наличии подходящих распределений $f$, $g$ и $h$. Тогда применяется метод имитационного максимального правдоподобия, который предполагает замену матожидания на симулированные выборочные моменты (средние), то есть,
    \begin{align}\label{eq:20.35}
    \tilde{\Pr}[Y_i = y_i, D_i = 1|\x_i, \bz_i] = \frac{1}{S} \sum^{S}_{s = 1}[f(\xib + \ga_1 D_i +\la \tilde{l}_{is})g(\bz'_i\bm{\al} + \de\tilde{l}_{is})],
    \end{align}
где $\tilde{l}_{is}$ обозначает $s$-ую (из возможных $S$) реализацию псевдо-случайной величины с плотностью $h$, а $\tilde{\Pr}$ --- симулированную вероятность. Оценки имитационного ММП соответствуют максимуму функции имитационного правдоподобия, построенной на основе \ref{eq:20.35}.

Хотя такой подход и был изначально предложен для анализа моделей счетных данных с эндогенной объясняющей дамми-переменной, он может быть также применен к многомерным дамми и счетным, дискретным или непрерывным, переменным. Ограничения на применение могут возникать в связи с объемом вычислений, довольно значительным по сравнению с оцениванием типа IV. Кроме того, как и в любой модели одновременных уравнений, под вопросом находится ее идентифицируемость. Также в прикладных работах вектор $\bz$ обычно включает некоторые нетривиальные регрессоры, исключенные из вектора переменных $\x$.




\section{Пример на счетных данных: дальнейший анализ}\label{sec:20.7}

\noindent
В данном разделе мы заново проведем анализ из раздела \ref{sec:20.3}, но с использованием более гибких параметрических моделей, начав с NB2 вместо Пуассона.

Расчеты по модели NB2, включая робастные стандартные ошибки и $t$-статистики, находятся в последних столбцах таблицы \ref{tab:20.5}, представленной в разделе \ref{sec:20.3}. Заметим, что коэффициент, соответствующий наличию избыточной дисперсии, $\al$, статистически значим.
Тестовая статистика Вальда составляет $8.926$, что отвергает нулевую гипотезу о равенстве матожидания и дисперсии $(\al = 0)$. С этим также согласуется значительный рост логарифма правдоподобия, с $-60,087$ до $-42,777$, что соответствует существенному повышению качества модели. Поскольку модели являются вложенными, нет необходимости указывать AIC и BIC.

Сравнив первую и третью строки таблицы \ref{tab:20.6}, можно увидеть, что рассчитанная по модели NB2 частота довольно точно предсказывает наблюдаемую, что подтверждает повышение качества модели как результат учета избыточной дисперсии.

Коэффициенты достаточно стабильны по сравнению с альтернативными методами оценивания, и все эффекты измерены с точностью, соответствующей большому размеру выборки. Такие результаты свидетельствуют о том, что применение модели NB2 обосновано. Как предсказывает классическая теория, потребление и коэффициент сострахования (LC) отрицательно коррелированы. При этом, сама оценка коэффициента слабо чувствительна к наличию избыточной дисперсии.

Возможно дальнейшее усовершенствование анализа. Например, Deb и Trivedi (2002) сравнивают модель преодоления порогов с моделью двухкомпонентной конечной смеси и показывают, что последняя лучше соответствует данным. Однако, даже модель преодоления порогов показывает лучшие результаты, чем NB2. Но то, что эти модели предоставляют дополнительную информацию, не говорит о том, что результаты, представленные в данной главе, являются неверными в отношении главного вопроса о чувствительности потребления к цене.

Модель NB2 хорошо работает на данных о количестве визитов к врачу. Для других же счетных данных могут потребоваться более гибкие модели, чем NB2.




\section{Практические соображения}\label{sec:20.8}

\noindent
Большинство статистических и эконометрических пакетов позволяют оценивать модель Пуассона. При наличии опыта работы с нелинейным методом наименьших квадратов использование такого статистического программного обеспечения не должно представлять трудностей. Однако необходимо убедиться, что полученные стандартные ошибки являются робастными. Многие эконометрические пакеты также позволяют оценить отрицательную биномиальную регрессию и базовые модели панельных данных. Модели регрессии для счетных данных обычно включены в модули по оцениванию обобщенных линейных моделей. Стандартные пакеты также рассчитывают меры качества подгонки модели для регрессии Пуассона, такие как псевдо-$R^2$, см. раздел 8.7.1.

Применение более новых моделей, в частности, моделей конечной смеси, большинства моделей временных рядов и динамических моделей панельных данных, скорее всего, потребовало бы написания собственного программного модуля на основе матричного языка программирования и программного обеспечения, предназначенного для оценивания задаваемых пользователем целевых функций. Большинство программ умеют оценивать простые модели методом максимального правдоподобия и производить расчет робастной дисперсии для задаваемых пользователем функций.

Помимо оценок коэффициентов, полезно иметь представление о степени влияния объясняющих переменных на объясняемую, что обсуждалось ранее в разделе \ref{sec:20.2.3}. Также, как уже было сказано в разделе \ref{sec:20.2.4}, важно проследить, чтобы стандартные ошибки и $t$-статистики в модели регрессии Пуассона были основаны на оценках дисперсии, робастных к наличию избыточной дисперсии.

Кроме того, для оценки адекватности модели желательно проводить тесты на спецификацию. Так, тесты на избыточную дисперсию легко применить для регрессии Пуассона на пространственных данных. Для любой параметрической модели можно сопоставить действительные и предсказанные значения подсчетов, хотя в этом случае не всегда легко понять, в чем заключается недочет модели, если распределение наблюдаемых счетных данных характеризуется высокой дисперсией. Далее, можно провести формальные тесты на статистическую спецификацию и качество подгонки модели, также основанные на сравнении действительных и предсказанных значений.

Во многих практических ситуациях возникает проблема выбора модели. Для невложенных моделей на основе правдоподобия можно использовать такие критерии, как информационный критерий Акаике, построенный с помощью предсказанных значений логарифма правдоподобия, но со штрафом за большое число параметров.




\section{Литература}\label{sec:20.9}

\noindent
\begin{itemize}
    \item[\textbf{20.2}]
Все вопросы, рассмотренные в данной главе, более подробно представлены в работе Cameron и Trivedi (1998); там же можно найти подробный список литературы. Winkelmann (1997) представляет обзор эконометрической литературы по счетным данным. В статистической литературе анализ счетных данных представлен в контексте обобщенной линейной модели. Классической работой является McCullagh и Nelder (1989). Эконометрическая литература уделяет относительно немного внимания обобщенной линейной модели. С точки зрения эконометрики эту модель рассматривают Fahrmeier и Tutz (1994). Материал из раздела \ref{sec:20.2} является базовым и применим во многих ситуациях.

    \item[\textbf{20.3}]
Deb и Trivedi (2002) проводят детальный анализ данных RHIE.

    \item[\textbf{20.4}]
Cameron и Trivedi (1986) является одной из первых работ, где предлагается обсуждение отрицательной биномиальной модели. Hausman et al. (1984) рассматривает применение этой модели к панельным данным. Для справки по моделям конечной смеси из раздела \ref{sec:20.4.3} см. Deb и Trivedi (1997). Модель преодоления порогов из раздела \ref{sec:20.4.5} была впервые предложена Mullahy (1986), и также была рассмотрена в работах Pohlmeier и Ulrich (1995) и Gurmu и Trivedi (1996).

    \item[\textbf{20.5}]
Квази-ММП, рассмотренный в разделе \ref{sec:20.5.1}, детально представлен в работах Gouri\'eroux et al. (1984 a,b) и Cameron и Trivedi (1986).

    \item[\textbf{20.6}]
Модели регрессии для типов данных, рассмотренных в главе \ref{sec:20.6}, еще не изучены полностью и только начинают развиваться. Исключение составляет (статическая) модель панельных счетных данных, которая достаточно хорошо представлена в классической работе Hausman et al. (1984). См. также Br\"ann\"as и Johansson (1996). Анализ и применение адекватных моделей регрессии для многомерных счетных данных и моделей с эндогенными регрессорами является довольно перспективной и активно развивающейся областью исследований; см. Terza (1998) и Deb и Trivedi (2004).
\end{itemize}




\section{Упражнения}\label{sec:20.ex}

\noindent
\begin{itemize}
    \item[\textbf{20--1}]
Предположим, что $Y$ следует распределению Пуассона с матожиданием $\mu$.
        \item[\textbf{(a)}]
Проверьте, что первые четыре момента распределения равны, соответственно, $\mu$, $\mu$, $\mu$ и $3\mu^2 + \mu$.
        \item[\textbf{(b)}]
Покажите, что существует линейное соотношение между $\Pr[Y = j]$ и $\Pr[Y = j - 1]$, $j = 1, 2, ...$.
        \item[\textbf{(c)}]
Рассмотрим оценку ММП Пуассона в регрессии с $\mu_i = \exp(\xib)$. Оценка дисперсии может быть равна $\hat{\V}[\hat{\be}] = [\sum_i \hat{\mu}_i \x_i \x'_i]^{-1}$ или $\tilde{\V}[\hat{\be}] = [\sum_i (y_i - \hat{\mu}_i)^2 \x_i \x'_i]^{-1}$.
Покажите, что эти оценки асимптотически эквивалентны (после умножения на $N$), если спецификация плотности распределения данных верна.

    \item[\textbf{20--2}]
Рассмотрим избыточную дисперсию в модели Пуассона.
        \item[\textbf{(a)}]
Пусть $Y|\mu \sim \mathcal{P}[\mu]$, где $\mu = \exp(\beta_0 + \beta_1\x)$, $\beta_0 = \ga_0 + \e$ и ошибка $\e$ является ненаблюдаемой случайной величиной с матожиданием и дисперсией $\E[\e] = 0$ и $\V[\e] = \sis > 0$, соответственно. Покажите, что $\V[Y] > \E[Y]$.
        \item[\textbf{(b)}]
Рассмотрим модель NB2 с функцией дисперсии $\mu + \al \mu^2$ и функцией распределения вероятностей (\ref{eq:20.12}). Для четырех различных значений $\al \in [0, 3]$ графически опишите поведение вероятности для различных реализаций $Y$; особое внимание следует обратить на область начала координат и правый хвост распределения.
        \item[\textbf{(c)}]
Для плотности NB2 (\ref{eq:20.12}) из раздела \ref{sec:20.4.1} покажите, что при $\al \rightarrow 0$ плотность стремится к плотности распределения Пуассона [может быть непросто].

    \item[\textbf{20--3}]
Рассмотрим модель регрессии Пуассона с условным матожиданием $\mu = \exp(\xb)$. Сформулируем задачу в виде нелинейного метода невзвешенных наименьших квадратов, где $y = \E[y|\x] + \e$, $\E[y|\x] = \exp(\xb)$ и $\e \sim \mathrm{iid}[0, \sis]$.
        \item[\textbf{(a)}]
Выведите уравнения (первого порядка) НМНК для $(\be, \sis)$. Сравните уравнения НМНК и ММП для $\be$ и объясните разницу между ними.
        \item[\textbf{(b)}]
Выведите уравнения \textit{взвешенным} НМНК для $\be$ и обоснуйте выбор весов. [Веса используются для учета гетероскедастичности].
        \item[\textbf{(c)}]
Сравните уравнения, полученные взвешенным НМНК и ММП, и объясните сходства, если такие присутствуют.

    \item[\textbf{20--4}]
Рассмотрим плотность конечной смеси $f(y|\ttt) = \sum^{C}_{j = 1} \pi_j f_j (y|\ttt_j)$, представляющую собой аддитивную смесь $C$ различных латентных классов, или групп, с неизвестными пропорциями $\pi_1, ..., \pi_C$, где $\sum^{C}_{j = 1} \pi_j = 1$, $\pi_j > 0$. Счетной переменной является $y$, и $j$-ая компонента плотности $f_j(y_i|\ttt_j)$ для $i$-го наблюдения записывается как
        $$f_j(y_j) = \frac{\Ga(y_i + \psi_{ji})}{\Ga(\psi_{ji})\Ga(y_i + 1)} \left( \frac{\psi_{ji}}{\la_{ji} + \psi_{ji}} \right)^{\psi_{ji}} \left( \frac{\la_{ji}}{\la_{ji} + \psi_{ji}} \right)^{y_{i}}$$
где $\la_{ji} = \exp(\xib_j)$, $\psi_{ji} = \la^k_{ji} / \al_j$, $\al_j > 0$ и $\bttt_j = (\be_j, \al_j)$. Параметр $k$ может принимать значения $0$ или $1$. Данная модель является отрицательной биномиальной конечной смесью с $C$ компонент, которая соответствует конечной смеси Пуассона при $\al_j = 0$.
        \item[\textbf{(a)}]
Покажите, что $\E[y_i|\x_i] = \tilde{\la}_i = \sum^{C}_{j = 1} \pi_j \la_{ji}$ и $\V(y_i|\x_i) = \sum^{C}_{j = 1} \pi_j \la^2_{ji} [1 + \al_j \la^{-k}_{ji}] + \tilde{\la}_i - \tilde{\la}_i^2$.
        \item[\textbf{(b)}]
Покажите, что любая модель смеси, основанная только на первом моменте, неидентифицируема.
        \item[\textbf{(c)}]
Покажите, что смесь Пуассона с $C$ компонентами, основанная на первых двух моментах, идентифицируема.

    \item[\textbf{20--5}]
(Baltagi и Li, 1999) Простой тест на избыточную дисперсию в модели Пуассона, представленный в разделе \ref{sec:20.2.4}, проверяет нулевую гипотезу о равенстве коэффициента нулю в регрессии $[(y_i - \hat{\mu}_i)^2 - y_i] / \hat{\mu}_i$ на $\hat{\mu}_i$. Альтернативный тест, предложенный Baltagi и Li (1999), проверяет ту же гипотезу в регрессии $(y_i - \hat{\mu}_i)^2$ на $\hat{\mu}_i$. Идея последнего теста аналогична тестам, основанным на регрессиях Гаусса--Ньютона (см. раздел 10.3.9). Проанализируйте различия между тестами и последствия таких различий в контексте применения второго теста.

    \item[\textbf{20--6}]
Для данного упражнения используйте 50\% выборки данных, представленных в данной главе.
        \item[\textbf{(a)}]
Оцените регрессию Пуассона и отрицательную биномиальную регрессию, взяв в качестве зависимой переменной MDU, а в качестве объясняющих факторов следующий набор переменных: LC, IDP, LINC, FEMALE, EDUDEC, XAGE, BLACK, HLTHG, HLTHF и HLTHP. Проведите тест отношения правдоподобия, чтобы проверить гипотезу о том, что переменные LC и IDP не оказывают влияния на MDU.
        \item[\textbf{(b)}]
Проведите тест на избыточную дисперсию в регрессии Пуассона, используя формулы дисперсии (\ref{eq:20.9}) с $g(\mu) = \mu$ и (\ref{eq:20.10}) с $g(\mu) = \mu^2$. % что?
Какой вариант формулы лучше соответствует данным? Какой вывод можно сделать на основе данного упражнения?
        \item[\textbf{(c)}]
Оцените отрицательную биномиальную модель (NB2). Сравните оценки параметра избыточной дисперсии в пункте (b). Объясните сходства и различия.
        \item[\textbf{(d)}]
Используя результаты по оцениванию отрицательной биномиальной модели, сравните оценку предельного эффекта от изменения LC для среднего индивида с отличным состоянием здоровья и среднего индивида с плохим состоянием здоровья (HLTHP = 1).
        \item[\textbf{(e)}]
Для спецификации Пуассона оцените модель ``преодоления порогов'', состоящую из части с нулевыми значениями (логит или пробит) и части с положительными (распределение Пуассона с урезанными нулями). Сравните эти результаты с обычной моделью Пуассона. Проанализируйте сходства и различия между выводами, следующими из обеих моделей. Какая модель лучше объясняет данные?
\end{itemize}





\part{Модели анализа панельных данных}


Модели анализа данных пространственного типа имеют ряд неотъемлемых ограничений.  Главным образом, речь идет о равновесных моделях, которые не проливают свет на межвременную зависимость событий. Кроме того, они не могут разрешить фундаменталные вопросы о неизменных поведенческих процессах. Такие неизменные процессы могут быть поведенческими, т.е. являющимися результатом истинной зависимости, или могут быть кажущимися, т.е. являющимися результатом невозможности учитывать гетерогенное поведение. Вследствие того, что панельные данные, также называемые лонгитюдными, содержат периодически повторяющиеся наблюдения одних и тех же субъектов, они обладают большим потенциалом в разрешении таких вопросов, которые модели анализа пространственных данных не в состоянии решить.
Главы с 21 по 23 описывают методы анализа панельных данных. Мы систематически двигаемся от линейных моделей для непрерывных переменных в главе 21 к нелинейным моделям панельных данных для ограниченных зависимых переменных в главе 23, рассматривая фиксированные и случайные эффекты. На протяжении этих трех глав 
мы постоянно возвращаемся к вопросу о важности использования робастных для панельных данных методов получения статистических выводов.

Глава 21, в которой содержится обзор основных результатов для моделей линейной регрессии для панельных данных, легко осваивается на базе знаний о линейных регрессионных моделях; она не требует знания материала частей 2-4. Даже тем, кто заинтересован исключительно в продвинутом материале, мы советуем в первую очередь внимательно ознакомится  с содержанием этой главы, чтобы получить представление об основных концепциях и определениях, касающихся анализа панельных данных.

В главе 22 рассматриваются расширенные модели, в особенности динамические модели анализа панельных данных, которые позволяют анализировать марковскую зависимость текущих переменных. Анализ проводится в рамках ММП, который широко используется практиками этой области. Анализ довольно-таки трудный и влечет за собой множество деталей, которые необходимо учитывать. Хорошее знание ММП облегчает понимание основных результатов данной главы.

Нелинейные модели панельных данных главы 23 в общем не являются продолжением результатов глав 21 и 22. Там уже представлено меньше результатов для панельных данных с ограниченными зависимыми переменными. Несмотря на это, глава 23 начинается с анализа общих вопросов и подходов. Дальнейшие разделы главы содержат результаты для аналогичных моделей пространственных данных, изученных в главе 4. В этих частях описывается анализ четырех категорий моделей: бинарных, количественных, цензурированных и моделей длительности. Они должны быть доступны для среднего читателя, имеющего знания о соответствующих моделях  данных пространственного типа.

\chapter{Линейные модели панельных данных: основы}
\section{Вступление}
\textbf{Панельные данные} представляют собой одни и те же повторяющиеся наблюдения, относящиеся к разным периодам времени. В микроэкономических приложениях это обычно фирмы или индивидуумы. По-другому, эти данные называют \textbf{лонгитюдными данными} или повторяющимися измерениями (\textbf{repeated measures}). Мы будем рассматривать главным образом \textbf{короткие панели}, соответствующие большим выборкам индивидуумов. Они наблюдаются в течение короткого периода времени, в то время как длинные панели, к примеру, небольшие группы стран, наблюдаются в течение многих периодов.

Основное преимущество панельных данных  --- это увеличение точности оценивания. Это результат увеличения количества наблюдений вследствие комбинирования (\textbf{pooling}) нескольких периодов для каждого индивидуального наблюдения. Однако, для получения верных статистических выводов для данного наблюдения необходимо учитывать возможную корреляцию во времени ошибок регрессионной модели. В частности, обычная формула для нахождения стандартных ошибок в модели сквозной регрессии переоценивает выигрыш от увеличения точности, приводя к недооцененным стандартным ошибкам и  резко увеличенным t-статистикам.

Во-вторых, панельные данные привлекают возможностью получить состоятельную оценку параметров в модели с \textbf{фиксированными эффектами}, которые учитывают коррелируемую с регрессорами ненаблюдаемую гетерогенность между индивидуами. Такая ненаблюдаемая гетерогенность приводит к \textbf{смещению в связи с опущенной переменной} (omitted variable bias), которое можно скорректировать с помощью метода инструментальных переменных с использованием только одной группы наблюдений. Это однако сложно осуществить на практике, так как найти годные инструменты достаточно трудно. Данные короткой, к примеру, даже двухпериодной панели позволяют преодолеть эту проблему альтернативным способом в предположении, что ненаблюдаемые индивидуальные эффекты аддитивны и не изменяются со временем.

Большинство дисциплин в прикладной статистике, кроме микроэконометрики, предполагают, что ненаблюдаемые индивидуальные эффекты распределены независимо от регрессоров. В таком случае они называются \textbf{случайными эффектами}, хотя лучшим определением было бы {\it чисто} случайные эффекты. Это более строгое предположение дает преимущество  по сравнению с фиксированными эффектами. Оно позволяет получить состоятельные оценки всех параметров, включая коэффициенты регрессоров, не изменяющихся во времени. Однако, случайные эффекты и оценки коэффициентов не состоятельны, если истинной моделью является модель с фиксированными эффектами. Экономисты считают, что часто данное предположение модели со случайными эффектами не выполняется на реальных данных.

Третье преимущество панельных данных состоит в том, что они позволяют более детально изучить \textbf{динамику} индивидуального поведения, чем при использовании пространственных данных. Так анализ пространственных данных может выявить уровень бедности в 20\%, однако для выявления того, входят ли в эти 20 \% те же самые индивидуумы или нет, необходим анализ панельных данных. Также панельные данные позволяют определить, например, является ли  причиной высокой автокорреляции индивидуальных доходов или продолжительности безработицы
специфическая индивидуальная тенденция к получению высоких доходов или к длительному безработному периоду, или же это является следствием того, что индивидуумы в прошлом имели высокие доходы или были безработными. Это рассматривается в главе 22.

Линейные модели панельных данных и связанные с ними оценки просты для понимания в отличие от фундаментальных вопросов о необходимости использования фиксированных эффектов. Рассматриваемые алгебраические методы, используемые для выведения свойств оценок в моделях панельных данных, не должны уводить от понимания основ: статистические свойства оценок в моделях панельных данных меняются в зависимости от предполагаемой модели и от предположений о ненаблюдаемых эффектах. Кроме того, значительная доля алгебраических выводов не распространяются на нелинейные модели панельных данных.

Текущая глава описывает основные виды оценок линейных моделей панельных данных. В подробном введении в разделе 21.2 и 21.3 представлены часто используемые модели и оценки, а также объяснен практический пример зависимости годовых часов работы от зарплат. Важное различие между моделями с фиксированными и случайными эффектами объясняется в разделе 21.4. Разделы 21.5-21.7 содержат дополнительные детали, связанные с оцениванием соответственно модели сквозной регрессии, модели с индивидуальными фиксированными эффектами и модели с индивидуальными случайными эффектами. Раздел 21.8 рассматривает другие основные аспекты, такие как получение статистических выводов и построение прогнозов в линейных моделях панельных данных.

\section{Обзор моделей и оценок}
Панельные данные содержат информацию о поведении индивидуумов как среди индивидуумов, так и во времени. 

Даже в случае линейной регрессии стандартный анализ панельных данных предполагает использование гораздо большего ряда моделей и оценок, чем в случае с пространственными данными. Несколько стандартных моделей представлены в разделе 21.2.1, после чего в разделе 21.2.2 описаны некоторые оценки. В таблице \ref{Tab:21.1}, обобщены основные модели и оценки. В таблице указано, что некоторые оценки несостоятельны, если процесс, порождающий данные, является моделью с фиксированными индивидуальными эффектами.

В случае с панельными данными получить верные стандартные ошибки для оценок сложнее, чем в случае с пространственными данными. Необходимо учитывать корреляцию во времени для данного индивидуума, а также возможную гетероскедастичность. Эта тема затрагивается в разделе 21.2.3.

\begin{table}[ht]
\centering
\caption[]{\itЛинейные модели панельных данных: основные оценки и модели ${}^a$}
\begin{tabular}{p{4cm} p{3.2cm} p{3.2cm} p{3.2cm}}
\hline \hline
				& \multicolumn{3}{c}{\bf{Предполагаемая модель}} \\
\bf{Оценка} $\beta$ 	& \bf{Модель сквозной регрессии (21.1)}	& \bf{Модель со случайными эффектами (21.3)  и (21.5)} &  \bf{Модель с фиксированными эффектами (21.3)} \\
МНК оценка модели сквозной регрессии (21.1)	&	Состоятельная	 &	Состоятельная	&	Несостоятельная	 \\

Between (21.7)	&	Состоятельная	 &	Состоятельная	&	Несостоятельная	 \\
Within (или оценка модели с фиксированными эффектами) (21.8)	&	Состоятельная	 &	Состоятельная	 &	Состоятельная	 \\
Модель в первых разностях & Состоятельная	 &	Состоятельная	 &	Состоятельная	 \\
Оценка модели со случайными эффектами	& Состоятельная	 &	Состоятельная	 &	Несостоятельная	 \\
\hline \hline
\multicolumn{4}{p{15cm}}{${}^a$ В этой таблице рассматриваются только состоятельность оценок $\beta$.  Для корректного вычисления стандартных ошибок см. раздел 21.2.3.}
\end{tabular}
\label{Tab:21.1}
\end{table}

\subsection{Модели анализа панельных данных}

Самая общая линейная модель для панельных данных предполагает, что свобоный член и коэффциенты наклона могут варьироваться по индивидуальным наблюдениям и во времени:
\begin{align}
& y_{it} = \alpha _{it} + x'_{it} \beta_{it}+u_{it},
& i= 1, \dots, N, &
& t=1, \dots, T,
\nonumber
\end{align}
где $y_{it}$ --- это скалярная зависимая переменная, $x_{it}$ --- вектор независимых переменных размерности $K \times 1$, $u_{it}$ --- ошибки модели, $i$ --- индивидуальный индекс (индивида, фирмы или страны), $t$ --- индекс временного периода.

Эта модель слишком общая, и оценить ее не представляется возможным, так как количество параметров для оценки больше, чем количество наблюдений. Необходимо наложить ограничения на степень, в которой $\alpha$ и $\beta$ изменяются в зависимости от $i$ и $t$, а также на поведение ошибок.

\vspace{0.8cm}

{\centering
Объединенная регрессия\\}


Модель, накладывающая наибольшее ограничение, --- это \textbf{модель сквозной регрессии}, которая базируется на предположении о \textbf{постоянных коэффициентах}, как это обычно предполагается при анализе пространственных данных:
\begin{align}
y_{it}= \alpha + \x'_{it} \bm\beta + u_{it}.
\label{Eq:21.1}
\end{align}
Если эта модель правильно специфицирована, и регрессоры некоррелированы с ошибками, тогда оценивание с помощью МНК сквозной регрессии дает состоятельные оценки. Ошибки могут быть коррелированы во времени для данного индивидуального наблюдения; в этом случае не следует пользоваться стандартными ошибками, так как они могут быть смещены вниз. Более того, МНК оценка сквозной регрессии будет несостоятельна, если моделью, описывающей данные, является модель с фиксированными эффектами, описанная ниже.

{\centering
Индивидуальные и временные эффекты\\}

Простой вариант модели \ref{Eq:21.1} предполагает, что свободный член может изменяться по индивидуальным наблюдениям и во времени, в то время как коэффициенты наклона остаются неизменными. Тогда $y_{it}= \alpha_{i} + \gamma_t + \mathbf x'_{it} \bm\beta+ u_{it}$, или
\begin{align}
y_{it}=\sum \limits_{j=1}^{N} \alpha_j d_{j,it} + \sum \limits_{s=2}^{T} \gamma_s d_{s,it} +  \mathbf x'_{it} \bm\beta,
\label{Eq:21.2}
\end{align}
где N индивидуальных фиктивных переменных $d_{j,it}$, равные единице, если $i=j$ и равны 0 в противном случае, $(T-1)$ временных фиктивных переменных $d_{s,it}$, равные единице, если $t=s$ и нулю в противном случае; предполагается, что $\mathbf x'_{it}$ не включает свободный член (Если свободный член включается, тогда необходимо исключить одну из индивидуальных фиктивных переменных).

Эта модель включает $N+(T-1)+\dim[\x]$ параметров, для которых могут быть получены состоятельные оценки, если $N\rightarrow \infty$ и $T \rightarrow \infty$. Мы главным образом рассматриваем \textbf{короткие панели}, где $N\rightarrow \infty$, но не $T$. Тогда для $\gamma_s$ может быть получена состоятельная оценка, так что $(T-1)$ временных фиктивных переменных просто включены в регрессоры $\mathbf x'_{it}$. В таком случае трудность состоит в оценивании параметров $\bm\beta$ при учете $N$ свободных членов для каждого индивидуума. Одно из решений состоит в том, чтобы вместо дамми для каждого индивидуального наблюдения, использовать дамми для групп, для чего можно использовать кластерный анализ, описанный в главе 24. В текущей главе мы используем спецификацию с полным набором $N$ свободных членов, что приводит к проблеме увеличения параметров $N \rightarrow \infty$.


{\centering
Модели с фиксированными и случайными эффектами\\}
Модели с \textbf{индивидуальными эффектами} предполагают, что каждая пространственная единица имеет свой свободный член, в то время как коэффициенты наклона одинаковы, так что 
\begin{align}
y_{it}=\alpha_i + \mathbf x'_{it} \bm \beta + \e_{it},
\label{Eq:21.3}
\end{align}
где $\e_{it}$ независимы и одинаково распределены по $i$ и $t$.  Это более компактный вид модели \ref{Eq:21.2} с фиктивными переменными, включенными в регрессоры $\mathbf x'_{it}$. $\alpha_i$ --- случайные переменные, которые охватывают \textbf{ненаблюдаемую гетерогенность}, которая обсуждалась в разделах 18.2-18.5 и 20.4.

В данной главе мы делаем предположение о \textbf{сильной экзогенности} (или \textbf{строгой экзогенности}) :
\begin{align}
E[\e_{it} | \alpha_i, \mathbf x'_{i1}, \dots, \mathbf x'_{iT}]=0, &
& t=1, \dots, T,
\label{Eq:21.4}
\end{align}
так что предполагается, что ошибки имеют нулевое условное математическое ожидание при прошлых, настоящих и будущих значениях регрессоров. Чемберлин (1980) детально обсуждает предположения об экзогенности и тесты для проверки экзогенности для панельных данных. Строгая экзогенность исключает использование моделей с лаговыми переменными или с эндогенными переменными в регрессорах, такие модели мы отложим до главы 22.

Один вариант модели \ref{Eq:21.3} предполагает, что $\alpha_i$ --- это ненаблюдаемые случайные величины, которые могут быть коррелированы с наблюдаемыми регрессорами. Такая модель называется \textbf{моделью с фиксированными эффектами}, так как в первую очередь эффекты моделировались как параметры $\alpha_i \dots \alpha_N$ для оценки. Если фиксированные эффекты присутствуют и коррелированы с $\mathbf x'_{it}$, то многие оценки, такие как МНК оценки модели сквозной регрессии будут несостоятельны. Вместо этого, для получения состоятельных оценок $\beta$ в коротких панелях должны быть использованы альтернативные методы оценки, которые элиминируют $\alpha_i$.

Другой вариант модели \ref{Eq:21.3} предполагает, что ненаблюдаемые индивидуальные эффекты $\alpha_i$ --- это случайные переменные, которые распределены независимо от регрессоров. Такая модель называется \textbf{моделью со случайными эффектами}, которая обычно накладывает дополнительные предположения:
\begin{align} \label{Eq:21.5}
 & \alpha_i  \thicksim [\alpha, \sigma^2_{\alpha}],  \\ 
& \epsilon_{it}  \thicksim [0, \sigma^2_{\epsilon}], %\nonumber
\end{align}
т.е. и случайные эффекты, и ошибки в \ref{Eq:21.3} независимы и одинаково распределены по предположению. Заметим, что в  \ref{Eq:21.5} не специфицировано никакое конкретное распределение. Для того, чтобы отличать эту модель от более общих моделей со случайными эффектами, таких как \textbf{смешанных линейных моделей} (mixed linear models), представленных в разделе 22.8, такой модели дано более точное название  --- \textbf{модель с индивидуальными случайными эффектами} (one-way individual-specific random effects model), или, что проще, \textbf{модель со случайными свободными членами} (random intercept model). Кроме того, другое название модели --- \textbf{модель со случайными компонентами} (random components model). 

Термин фиксированные эффекты может вводить в заблуждение, а вместо термина случайные эффекты более корректно было бы употреблять термин чисто случайные эффекты. Чтобы избежать путаницы, М.-Дж. Ли (2002)  называет фиксированные эффекты <<связанными эффектами>>, а случайные эффекты --- <<несвязанными эффектами>>. Мы используем традиционные названия и терминологию, но необходимо помнить, что $\alpha_i$  --- это случайная переменная как в моделях с фиксированными, так и со случайными эффектами.


{\centering
Равнокоррелированная модель\\}
Модель со случайными эффектами может рассматриваться как частный случай модели сквозной регрессии, так как $\alpha_i$ можно отнести к ошибке. Тогда \ref{Eq:21.3} может рассматриваться как регрессия $y_{it}$ на $\mathbf x'_{it}$ с составной ошибкой $u_{it}=\alpha_i+\e_{it}$. В \ref{Eq:21.5} указано предположение, что 
\begin{align}
Cov[(\alpha_i+\e_{it}), (\alpha_i+\e_{is})]= 
	\begin{cases}
		\sigma^2_{\alpha}, & t\neq s, \\
		\sigma^2_{\alpha} + \sigma^2_{\e}, & t=s.
	\end{cases}
\label{Eq:21.6}
\end{align}

Модель со случайными эффектами накладывает ограничение, что составная ошибка $u_{it}$ \textbf{равнокоррелирована} (equicorrelated), так как $Cor[u_{it},u_{is}]=\sigma^2_{\alpha}/[\sigma^2_{\alpha} + \sigma^2_{\epsilon}]$ для $t\neq s$ не меняется в зависимости от $t-s$. Очевидно, что МНК оценка модели сквозной регрессии будет состоятельна, но неэффективна в модели со случайными эффектами. Модель со случайными эффектами также называется \textbf{равнокоррелированной моделью} (equicorrelated model) или \textbf{моделью взаимозаменяемых ошибок} (exchangeable errors model).

{\centering
Модель с фиксированными эффектами против модели со случайными эффектами\\}

Модели с и без фиксированных эффектов фундаментально различаются между собой. Современная эконометрическая литература фокусируется на фиксированных эффектах, но мы также остановимся и на модели со случайными эффектами.

Некоторые авторы, включая Чемберлина (1980, 1984) и  Вулдриджа (2002), используют запись
\begin{align}
y_{it} = c_i + \mathbf x'_{it} \bm{\beta} + \e_{it}
\nonumber
\end{align}
в \ref{Eq:21.3}, чтобы было ясно, что индивидуальный эффект --- это случайная величина как в модели с фиксированными, так и со случайными эффектами. Обе модели предполагают, что
\begin{align}
E[y_{it} | c_i,  \mathbf x'_{it}] = c_i +  \mathbf x'_{it} \bm{\beta}.
\nonumber
\end{align}

Индивидуальный эффект $c_i$ неизвестен, и  в коротких панелях для него не может быть получена состоятельная оценка, поэтому мы не можем оценить $E[y_{it} | \mathbf x'_{it}]$. Вместо этого мы можем элиминировать $c_i$ взятием математического ожидания по $c_i$, 
\begin{align}
E[y_{it} | \mathbf x'_{it}] = E [c_i | \mathbf x'_{it}] + \mathbf x'_{it} \bm{\beta}\nonumber.
\end{align}

В модели со случайными эффектами предполагается, что $E [c_i | \mathbf x'_{it}] = \alpha$, и как следствие $E[y_{it} | \mathbf x'_{it}] = \alpha + \mathbf x'_{it} \bm{\beta}$ и поэтому    
$E[y_{it} | \mathbf x'_{it}]$ возможно идентифицировать. В модели с фиксированными эффектами однако $E [c_i | \mathbf x'_{it}]$ изменяется с $\mathbf x'_{it}$. Каким образом изменяется $\mathbf x'_{it}$, неизвестно. Поэтому идентифицировать $E[y_{it} | \mathbf x'_{it}]$ мы не можем. И тем не менее, возможность получить состоятельную оценку $\bm{\beta}$ в модели с фиксированным эффектом в короткой панели сохраняется (это будет обсуждаться далее). Таким образом, в модели с фиксированными эффектами можно идентифицировать предельный эффект
\begin{align}
\bm{\beta} = \partial E[y_{it} | c_i, \mathbf x'_{it}]/ \partial \mathbf x'_{it},
\nonumber
\end{align}
но даже в этом случае условное математическое ожидание не может быть идентифицировано. Можно, например, идентифицировать эффект на отдачу от дополнительного года образования, учитывая индивидуальный эффект, но даже в таком случае индивидуальные эффекты и условное математическое ожидание не идентифицируемы.

В коротких панелях модели с фиксированными эффектами позволяют только идентифицировать  предельные эффекты $\partial E[y_{it} | c_i, \mathbf x'_{it}]/ \partial \mathbf x'_{it}$, но даже в этом случае только для регрессоров, меняющихся во времени,. Поэтому предельные эффекты расы или пола, к примеру, не идентифицируемы. Модель со случайными эффектами позволяет идентифицировать все компоненты $\bm{\beta}$ и $E[y_{it}| \mathbf x'_{it}]$, но ключевое предположение модели со случайными эффектами заключается в том, что $E[c_{it}| \mathbf x'_{it}]$ является константой, что не выполняется во многих микроэконометрических приложениях.

\subsection{Оценки параметров в моделях панельных данных}

Мы сейчас представим несколько часто используемых оценок параметра $\bm{\beta}$ в моделях панельных данных, детальное описание которых представлено в разделах 21.5 --- 21.7. Оценки различаются в зависимости от степени использования изменчивости пространственных и временных данных, а их свойства меняются в соответствии с тем, является ли модель с фиксированными эффектами подходящей.

Регрессор $x_{it}$ может быть либо \textbf{неизменным}, с $x_{it}=x_i$ при $t=1, \dots , T$, или \textbf{изменяющимся во времени}. Для некоторых оценок, в особенности для межгрупповой оценки и оценки в первых разностях, описанных далее, идентифицированы только коэффициенты перед регрессорами, меняющимися во времени.

{\centering
 Модель сквозной регрессии \\}

Оценка \textbf{модели сквозной регрессии} может быть получена с помощью оценивания регрессии МНК, полученной с помощью составления всех имеющихся данных в одну группу длинного формата с $NT$ наблюдениями
\begin{align}
& y_{it} = \alpha +  \mathbf x'_{it} \bm{\beta} + u_it, 
& i= 1, \dots, N, &
& t=1, \dots, T.
\nonumber
\end{align}
Если $Cov[u_{it}, \mathbf x_{it} ] = \mathbf 0$, тогда для состоятельности оценок достаточно либо $ N \rightarrow  \infty$ , либо $T\rightarrow \infty$

Очевидно, что оценка модели сквозной регрессии состоятельна, если модель сквозной регрессии \ref{Eq:21.1} наилучшим образом описывает данные и регрессоры некоррелированы с ошибкой. Обычная ковариационая матрица МНК основывается на независимых и одинаково распределенных ошибках, однако не является подходящей, так как ошибки для данного $i$ почти наверняка коррелируют  во времени. $NT$ коррелированных наблюдений вмешают в себя меньше информации, чем $NT$ независимых наблюдений.

Для понимания этой корреляции заметим, что для данного индивидуального наблюдения мы ожидаем существенную корреляцию $y$  во времени, так что $Cor[y_{it}, y_{is}]$ велика. Даже после включения регрессоров $Cor[u_{it}, u_{is}]$ может по прежнему не равняться нулю, и быть все еще достаточно высокой. Например, если модель завышает оценку будущих индивидуальных доходов в одном году, она может также завысить оценку будущих доходов того же индивидуума в другой год. Модель со случайными эффектами учитывает такую корреляцию $Cor[u_{it}, u_{is}] = \sigma^2_\alpha / [ \sigma^2_\alpha + \sigma^2_\epsilon]$ при $t \neq s$ (см. \ref{Eq:21.6}).

Обычный МНК использует $T$ лет как независимые частицы информации, однако при наличии положительной корреляции ошибок информации в этих частицах содержится меньше, чем предполагается МНК. Это приводит к переоцениванию точности оценок, которое может быть довольно большим, как показано в разделе 21.3.2 и формально продемонстрировано в разделе 21.5.4. Поэтому необходимо использовать скорректированные (panel-corrected) стандартные ошибки (см. раздел 21.2.3)  всегда, когда для оценивания панельных данных применяется МНК. Можно применить различные корректировки для стандартных ошибок в зависимости от предполагаемой структуры корреляции и гетероскедастичности. Корректировки могут различаться в зависимости от того, какая панель, короткая или длинная, используется (см. раздел 21.5).

Оценка МНК модели сквозной регрессии несостоятельна, если истинная модель --- это модель с фиксированными эффектами. Чтобы это обнаружить, необходимо переписать модель (\ref{Eq:21.3}) как 
\begin{align}
y_{it} = \alpha +  \mathbf x'_{it} \bm{\beta} + (\alpha_i --- \alpha + \e_{it}).
\nonumber
\end{align}
Сквозная регрессия  $y_{it}$ на $\mathbf x_{it}$ с включением константы, оцениваемая с помощью МНК, дает несостоятельные оценки $\bm{\beta}$, если индивидуальный эффект $\alpha_i$ коррелирован с регрессорами $\mathbf x'_{it}$, вследствие того, что такая корреляция подразумевает, что составная ошибка $(\alpha_i --- \alpha + \epsilon_{it})$ коррелирована с регрессорами.

Таким образом, МНК оценка модели сквозной регрессии  
является подходящей, если моделями, хорошо описывающими данные, являются модели с постоянными коэффициентами или модели со случайными эффектами. Однако, для статистических выводов необходимо использовать скорректированные стандартные ошибки и t-статистики. Модель объединенной регрессии МНК не будет состоятельна, если наилучшим образом описывает данные модель с фиксированными эффектами.

{\centering
Оценка between \\}
При оценке коэффициента $\be$ с помощью объединенной модели регрессии используется изменчивость и во времени, и в пространстве.

Оценка between использует только пространственную изменчивость в коротких панелях. Начнем с модели с фиксированными индивидуальными эффектами \ref{Eq:21.7}. Усредняя по годам, получаем $\bar{y}_{it} = \alpha_i +  \bar{\mathbf x}'_{it} \bm{\beta} +\bar{\varepsilon}_{it}$ и переписываем в виде  \textbf{модели between}
\begin{align}
& \bar{y}_{it} = \alpha +  \bar{\mathbf x}'_{it} \bm{\beta} + (\alpha_i --- \alpha + \bar{\varepsilon}_{it}),
& i=1, \dots, N,
\label{Eq:21.7}
\end{align}
где $\bar{y}_{it} =T^{-1}\sum_t {y}_{it}$, $\bar{\varepsilon}_{it}=\sum_t {\varepsilon}_{it}$ и $\bar{\mathbf x}'_{i}=T^{-1}\sum_t {\mathbf x}_{it}$.

\textbf{Оценка between} представляет собой МНК оценку регрессии $\bar{y}_{it}$ на константу и $\bar{\mathbf x}'_{i}$. Эта оценка использует изменчивость между индивидуальными наблюдениями и является аналогом обычной регрессионной модели пространственных данных. Регрессионная модель пространственных данных является частным случаем модели between, когда $T=1$.

Оценка between состоятельна, если регрессоры $\bar\x$ не зависят от составной ошибки  $(\alpha_i --- \alpha + \bar{\varepsilon}_{it})$ в \ref{Eq:21.7}. Это выполняется для модели с постоянными коэффициентами и модели со случайными эффектами. Для модели с фиксированными эффектами оценка between несостоятельна, так как предполагается, что $\alpha_i$ коррелирует с $\x_{it}$, а следовательно коррелирует и с $\bar\x_{it}$.

{\centering 
Оценка within или оценка с фиксированным эффектом\\}

Оценка within в отличие от оценки модели сквозной регрессии или оценки between использует особые качества панельных данных. В короткой панели она измеряет связь между индивидуальными отклонениями регрессоров от своих средних по времени значений и индивидуальными отклонениями зависимой переменной от своего среднего по времени значения. Это возможно благодаря использованию вариации данных во времени.

Начнем с модели с индивидуальными эффектами \ref{Eq:21.3}, которая является частными случаем модели \ref{Eq:21.1} при $\alpha_i=\alpha$. Усредняя по времени, получаем $\bar{y}_i=\alpha_i+\bar{\x}'_{i} \bm{\beta}+\bar{\e}_i$. Вычитаем полученное из $y_{it}$ в  \ref{Eq:21.3} и получаем \textbf{модель within}
\begin{align}
& y_{it} --- \bar{y}_{it} = (\x_{it} --- \bar{\mathbf x}'_{it}) \be + (\e_{it}-\bar\e_i),
& i=1, \dots, N, &
& t=1, \dots, T,
\label{Eq:21.8}
\end{align}
так как $\alpha_i$ взаимно уничтожаются.

\textbf{Оценка within} представляет собой МНК оценку модели \ref{Eq:21.8}. Особенностью этой модели явялется то, что она дает состоятельные оценки $\be$ в модели с фиксированными эффектами в отличие от МНК оценки в модели сквозной регрессии и оценки between.

Оценка within имеет несколько интерпретаций (см. раздел 21.6). Она называется \textbf{оценкой с фиксированным эффектом}, так как это эффективная оценка $\be$ в модели \ref{Eq:21.3}, где $\alpha_i$ --- это фиксированные эффекты, а ошибки $\e_{it}$ независимы и одинаково распределены. В этой главе внимание фокусируется на фиксированных эффектах как \textbf{вспомогательных параметрах (nuisance parameters)}, которые могут быть проигнорированы, ведь интерес заключается только в оценке параметра $\be$. В случае необходимости фиксированные эффекты также могут быть оценены. В коротких панелях  оценки индивидуальных эффектов $\alpha$ несостоятельны, хотя их распределение или их ковариация с зависимой переменной могут быть информативными. Если $N$ не слишком велико, то альтернативным и более простым путем получить оценку within будет \textbf{МНК оценка с фиктивными переменными} (Least Squares Dummy Variable estimator, LSDV). Она получается путем оценивания \ref{Eq:21.2} МНК регрессии $y_{it}$ на $x_{it}$ и  N индивидуальных фиктивных переменных. В результате получается  оценка within коэффициента $\be$ наряду с оценками N фиксированных эффектов (см. раздел 21.6.4). Другая интерпретация оценки within --- оценка ковариации. Наконец, взятие отклонений средних индивидуальных значений эквивалентно взятию остатков из вспомогательной регрессии $y_{it}$ и $\x_{it}$ на индивидуальные фиктивные переменные и последующему использованию этих остатков.

Основное ограничение оценки within состоит в том, что коэффициенты регрессоров, которые не меняются во времени, не идентифицируемы в данной модели, так как если $x_{it}=x_i$, то $\bar{x}_i=x_i$ и $(x_{it}-\bar{x}_i)=0$. Было осуществлено множество попыток оценить влияние регрессоров, не меняющихся во времени. Например, в модели заработной платы мы можем быть заинтересованы в эффекте пола или расы. По этой причине многие не используют оценку within. МНК оценка модели сквозной регрессии или оценка со случайным эффектом позволяют оценивать коэффициенты перед переменными, которые не меняются во времени, но эти оценки несостоятельны в случае, если модель с фиксированным эффектом оказывается истинной моделью.


{\centering
Оценка модели в первых разностях\\}

Оценка модели в первых разностях также использует особенности панельных данных. Она измеряет связь между индивидуальными однопериодными изменениями регрессоров и индивидуальными однопериодными изменениями зависимой переменной в короткой панели.

Начнем с модели с индивидуальными эффектами \ref{Eq:21.3}. Вычитая однопериодное запаздывание $y_{i,t-1}=\alpha_i+\x'_{i,t-1} \be + \e_{i,t-1}$ из $y_{it}$ в  \ref{Eq:21.3}, получаем \textbf{модель в первых разностях}
\begin{align}
& y_{it} --- {y}_{i,t-1} = (\x_{it} --- \mathbf x'_{i,t-1})' \be + (\e_{it}-\bar\e_{i,t-1}),
& i=1, \dots, N, &
& t=2, \dots, T,
\label{Eq:21.9}
\end{align}
так как $\alpha_i$ взаимно уничтожаются.

Оценка модели в первых разностях --- это МНК оценка в \ref{Eq:21.9}. Как и в случае оценки within, это оценивание приводит к состоятельным оценкам коэффициентов $\be$  в модели с фиксированным эффектом, хотя коэффициенты не меняющихся во времени регрессоров не идентифицированы. Оценка модели в первых разностях менее эффективна, чем внутригрупповая оценка для $T>2$, если $\e_{it}$ независимы и одинаково распределены.

{\centering
Оценка со случайным эффектом\\}
Оценка со случайным эффектом также использует особенности панельных данных.

Начнем с модели с индивидуальными эффектами \ref{Eq:21.3}, но будем предполагать модель со случайным эффектом, в которой $\alpha_{i}$ и $\e_{it}$ независимы и одинаково распределены, как в \ref{Eq:21.5}. МНК оценка модели сквозной регрессии состоятельна, но ОМНК оценка будет более эффективна. \textbf{Доступная ОМНК-оценка} (см. раздел 4.5.1) модели со случайным эффектом, называемая \textbf{оценкой со случайным эффектом}, может быть посчитана посредством МНК оценивания преобразованной модели 
\begin{align}
y_{it} --- \hat{\lambda}\bar{y}_{i} = (1-\hat{\lambda})\mu+(\x_{it} --- \hat{\lambda}\bar{\x}_{i})' \be +v_{it},
\label{Eq:21.10}
\end{align}
где $v_{it}=(1-\hat{\lambda})\alpha_{i}+(\e_{it}-\hat{\lambda}{\bar{\e}}_{i})$ асимптотически независимы и одинаково распределены, и $\hat{\lambda}$ состоятельна для
\begin{align}
\lambda=1-\frac{\sigma_{\e}}
				{\sqrt{\sigma^2_{\e}+T\sigma^2_\alpha}}.
\label{Eq:21.11}
\end{align}
Вывод \ref{Eq:21.10} и способы оценки $\sigma^2_{\alpha}$  и $\sigma^2_{\e}$, а следовательно и $\lambda$ представлены в разделе 21.7. Заметим, что $\hat{\lambda}=0$ соответствует МНК оценке в модели сквозной регрессии, $\hat{\lambda}=1$ соответствует оценке within, и $\hat{\lambda} \rightarrow 1$ при $T \rightarrow \infty$. Это двухшаговая оценка $\be$.

Оценка со случайным эффектом является эффективной, если модель со случайными эффектами является истинной, хотя выигрыш в эффективности в сравнении с МНК не должен быть большим. Однако, если истинная модель --- это модель с фиксированными эффектами, то оценка несостоятельна.

\subsection{Статистические выводы робастные для панельных данных}
Различные модели панельных данных включают ошибки, обозначаемые $u_{it}$, $\e_{it}$ и $\alpha$. Во многих микроэконометрических приложениях стоит использовать предположение о независимости по $i$. Однако, ошибки могут быть (1) \textbf{коррелированы} (например, во времени для данного $i$) и/или (2) \textbf{гетероскедастичны}. Для получения надежных статистических выводов необходимо учитывать оба этих фактора.

Состоятельная оценка при наличии гетероскедастичности в форме Уайта, рассмотренная в разделе 4.4.5, может применяться и к коротким панелям, так как для i-го наблюдения ковариационная матрица ошибок имеет конечную размерность $T\times T$ при $N \rightarrow \infty$. Таким образом, робастные к гетероскедастичности стандартные ошибки могут быть получены без предположения о специальной функциональной форме для внутригрупповой корреляции ошибок или гетероскедастичности. Более эффективные оценки полученные с помощью ОММ будут рассматриваться позже в разделе 22.2.3. 

Важно заметить, что часто команды для работы с панельными данными во многих статистических пакетах по умолчанию считают стандартные ошибки в предположении, что ошибки модели независимы и одинаково распределены, что приводит к ошибочным выводам. В частности, для модели сквозной регрессии $y_{it}$ на $\x_{it}$ без какого-либо учета индивидуальных эффектов очень вероятно, что $Cov[u_{it}, u_{is}] > 0$ для $t \neq s$. Игнорирование этой корреляции может привести к сильно завышенным стандартным ошибкам и переоцененным $t$-статистикам, что продемонстрировано на примере в разделе 21.3 и алгебраически показано в разделе 21.5.4. Как только в модель включаются фиксированные или индивидуальные случайные эффекты, корреляция в ошибках значительно снижается, но все же не может быть устранена полностью.
К тому же, вероятно, необходимо учитывать и возможную гетероскедастичность как это обычно делается для пространственных данных.
 
{\centering
Робастные для панельных данных стандартные ошибки в сэндвич форме\\}

Оценки применимые к панельным данным, описанные в разделе 21.2.2 могут быть получены с помощью МНК оценивания $\bm\theta$ в сквозной регрессионной модели
\begin{align}
\tilde{y}_{it}=\tilde{\mathbf w}'_{it} \bm\theta+\tilde{u}_{it},
\label{Eq:21.12}
\end{align}
где разные оценки панельных данных соответствуют разным преобразованиям $\tilde{y}_{it}$,
 $\tilde{w}'_{it}$ и $\tilde{u}_{it}$ для $y_{it}$, $\tilde{w}'_{it}=[1\; \x'_{it}]$, и $u_{it}$. Дело в том, что $\tilde{y}_{it}$ --- известная функция только для $y_{i1}, \dots, y_{iT}$, что аналогично для  $\tilde{w}'_{it}$ и $\tilde{u}_{it}$.

В самом простом случае модели сквозной регрессии нет необходимости в преобразовании и $\bm\theta=[\alpha \; \be']'$. Для оценки within $\tilde{y}_{it}=y_{it}-\bar{y}_{it}$, $\tilde{\mathbf w}'_{it}=(\x_{it}-\bar\x_i)$, где присутствуют только регрессоры, меняющиеся во времени, и $\bm\theta$ равно коэффициентам этих регрессоров. Для оценивания уравнения в первых разностях $\tilde{y}_{it}=y_{it}-{y}_{i,t-1}$, $\tilde{\mathbf w}'_{it}=(\x_{it}-\x_{i,t-1})$ и снова идентифицированы коэффициенты только регрессоров, меняющихся во времени. В случае со случайными эффектами $\tilde{y}_{it}=y_{it}-\hat{\lambda}\bar{y}_{it}$, $\tilde{\mathbf w}'_{it}=(\mathbf w_{it}-\hat{\lambda} \bar{\mathbf w}_i)$ и $\bm\theta=[\bm\alpha \; \be']'$. Такие преобразования могут стать причиной корреляции даже если исходные ошибки не были коррелированы.

Для записи удобно соединять наблюдения по временным периодам для данного индивида:
\begin{align}
\widetilde{\mathbf y}_{i}=\widetilde{\mathbf W}'_{i} \bm\theta+\tilde{\mathbf u}_{i},
\nonumber
\end{align}
где $\tilde{\mathbf{y}}_{i}$ --- это вектор размерности $T \times 1$, как в предыдущих примерах (кроме модели в первых разностях, где размерность $(T-1) \times 1$), и $\tilde{\mathbf{W}}_i$  --- это матрица размерности $T \times q$ (или $(T-1) \times q$ --- для модели в первых разностях). Если далее соединить наблюдения по индивидуумам, то запись модели будет выглядеть следующим образом:
 \begin{align}
\tilde{\mathbf y}=\widetilde{\mathbf W} \bm\theta+\tilde{\mathbf u}.
\nonumber
\end{align}
Поэтому МНК оценку можно записать тремя разными способами:
 \begin{align}
\hat{\theta}_{OLS}
&=[\tilde{\mathbf W}'\tilde{\mathbf W}]^{-1}\tilde{\mathbf W}'\tilde{\mathbf y} \nonumber \\
& = \left[\sum^{N}_{i=1} \tilde{\mathbf W}_i'\tilde{\mathbf W}_i\right]^{-1}\sum_i \tilde{\mathbf W}_i'\tilde{\mathbf y}_i \nonumber \\
& =\left[\sum^{N}_{i=1} \sum^{T}_{t=1}\tilde{\mathbf w}_{it}'\tilde{\mathbf w}_{it}\right]^{-1}\sum_{i=1}^{N} \sum_{t=1}^{T} \tilde{\mathbf w}_{it}'\tilde{\mathbf y}_{it}
\nonumber
\end{align}
где в случае модели в первых разностях пределы суммирования по времени для третьего выражения меняются на от $t=2$ до $T$. В зависимости от контекста используется наиболее удобная форма записи.

В целях проверки состоятельности, заметим, что если модель правильно специфицирована, то с помощью стандартных алгебраических преобразований получаем $\hat{\bm \theta}_{OLS}=\bm\theta+[\tilde{\mathbf W}'\tilde{\mathbf W}]^{-1}\tilde{\mathbf W}'\tilde{\mathbf u}$ или 
 \begin{align}
\hat{\theta}_{OLS}
& = \bm\theta + \left[\sum^{N}_{i=1} \tilde{\mathbf W}_i'\tilde{\mathbf W}_i\right]^{-1}\sum_i^N \tilde{\mathbf W}_i'\tilde{\mathbf u}_i.
\nonumber
\end{align}
При условии независимости индивидуальных наблюдений необходимое условие состоятельности --- $E[\tilde{\mathbf W}_i'\tilde{\mathbf u}_i]=0$, что влечет за собой более строгое предположение $E[u_{it}|\mathbf w_{it}]=0$. Существенным условием также является условие строгой экзогенности, данное в \ref{Eq:21.4}. Оценка, использующая более слабые предположения, чем строгая экзогенность, описана в главе 22. Она позволит, к примеру, включить лаговые переменные.

Асимптотическая дисперсия $\hat{\theta}_{OLS}$ будет равна
 \begin{align}
V[\hat{\theta}_{OLS}]= \left[\sum^{N}_{i=1} \tilde{\mathbf W}_i'\tilde{\mathbf W}_i\right]^{-1} \sum^{N}_{i=1} \tilde{\mathbf W}_i' \mathrm E[\tilde{\mathbf u}_i \tilde{\mathbf u}'_i|\tilde{\mathbf W}_i]\tilde{\mathbf W_i} \left[\sum^{N}_{i=1} \tilde{\mathbf W}_i'\tilde{\mathbf W}_i\right]^{-1},
\nonumber
\end{align}
при условии независимости ошибок по индивидуальным наблюдениям. Получение состоятельной оценки $V[\hat\theta_{OLS}]$ в данном случае аналогично проблеме получения состоятельных оценок $V[\hat{\theta}_{OLS}]$ робастных к гетероскедастичности в случае пространственных данных. Единственное усложнение --- это вектор $\mathbf u_i$ вместо скалярной $u_i$, что не представляет особой проблемы в случае короткой панели, так как в таком случае размерность $\mathbf u_i$ конечна.

Таким образом, мы получаем \textbf{робастную для панельных данных оценку} асимптотической ковариационной матрицы МНК оценки в  модели сквозной регрессии, которая учитывает и корреляцию, и гетероскедастичность:
 \begin{align}
V[\hat{\theta}_{OLS}]= \left[\sum^{N}_{i=1} \tilde{\mathbf W}_i'\tilde{\mathbf W}_i\right]^{-1} \sum^{N}_{i=1} \tilde{\mathbf W}_i' \hat{\mathbf u}_i \hat{\mathbf u}'_i\tilde{\mathbf W}_i]\tilde{\mathbf W_i} \left[\sum^{N}_{i=1} \tilde{\mathbf W}_i'\tilde{\mathbf W}_i\right]^{-1},
\label{Eq:21.13}
\end{align}
где $\hat{\mathbf u}_i=\hat{\tilde{\mathbf u}}_i=\tilde{\mathbf y}_i-\tilde{\mathbf W}_i \tilde{\theta}$. Оценка \ref{Eq:21.13} предполагает независимость индивидуальных наблюдений и $N \rightarrow \infty$ в случае короткой панели, в противном случае $V[u_it]$  и  $Cov[u_it, u_is]$ могут меняться в зависимости от $i, t$ и $s$. По-другому выражение \ref{Eq:21.13} можно записать следующим образом:
\begin{align}
V[\hat{\theta}_{OLS}]=\left[\sum^{N}_{i=1} \sum^{T}_{t=1} \tilde{\mathbf w}_{it}\tilde{\mathbf w}'_{it}\right]^{-1} \sum_{i=1}^{N} \sum_{t=1}^{T} \sum_{s=1}^T \tilde{\mathbf w}_{it}\tilde{\mathbf w}'_{is} \hat{u}_{it} \hat{ u}_{is} \left[\sum^{N}_{i=1} \sum^{T}_{t=1}\tilde{\mathbf w}_{it}\tilde{\mathbf w}'_{it}\right]^{-1}, 
\nonumber
\end{align}
где $\hat{u}_{it}=\hat{\tilde{y}}_{it}-\tilde{\mathbf w}_{it} \hat{\bm\theta}$. Эта оценка была предложена Ареллано (1987) для оценки с фиксированным эффектом.

Робастные стандартные ошибки, основанные на \ref{Eq:21.13} могут быть вычислены с помощью стандартной команды для вычисления стандартных ошибок в МНК-регрессии, если в команде есть опция вычисления \textbf{кластерно-робастных} стандартных ошибок (см. раздел 24.5.2). Так как кластеризация здесь проводится на уровне индивидуумов, индивидулаьный индекс $i$ используется как \textbf{кластерная переменная}. Этот метод был использован для получения робастных стандартных ошибок, представленных в таблице 24.1. %\ref{Tab:24.1}.

Термин <<робастные>> стандартные ошибки может ввести читателя в заблуждение. Распространенная ошибка в оценивании модели сквозной регрессии --- использование обычных робастных стандартных ошибок, используемых при оценивании регрессии МНК \ref{Eq:21.12}  (см. раздел 4.4.5). Однако, обычные робастные стандартные ошибки решают только проблемы гетероскедастиности, а на практике при работе с панельными данными намного более важно устранить проблему корреляции ошибок  индивидуальных наблюдений. Другая распространенная ошибка, хоть и не влекущая за собой серьезных последствий, --- использование кластерных робастных стандартных ошибок, которые предполагают гомоскедастичность, т.е. что $\mathrm E[\mathbf u_i \mathbf u'_i]$ постоянно для каждого $i$.


{\centering
Робастные для панельных данных стандартные ошибки, полученные методом бутстрэп \\}

Альтернативный способ получения робастных для панельных данных стандартных ошибок --- \textbf{бутстрэп метод}. Ключевым предположением метода является независимость индивидуальных наблюдений, так что ресэмплинг проводится \textbf{с заменой по $i$}, и используются все наблюдаемые периоды для данного индивидуального наблюдения.
Таким образом, мы получаем $B$ псевдо-выборок для данных $\{(\mathbf y_i, \mathbf X_i), i = 1, \dots, N\}$, и для каждой \textbf{псевдо-выборки} оценивается МНК регрессия $\tilde{y}_{it}$ на $\tilde{\mathbf w}_{it}$. В результате оценивания получается $B$ оценок $\hat{\bm \theta}_b, b=1, \dots, B$.

\textbf{Оценка ковариационной матрицы для панельных данных, полученная методом бутстрэп:}
\begin{align}
\hat{\mathrm V}_{\mathrm {Boot}}[\hat{\bm\theta}]=\frac{1}{B-1} \sum_{b=1}^B \left(\hat{\bm\theta}_b --- \bar{\hat{\bm\theta}} \right) \left( \hat{\bm\theta}_b --- \bar{\hat{\bm\theta}} \right)',
\label{Eq:21.14}
\end{align}
где $\bar{\bm{\hat{\theta}}}=B^{-1} \sum\nolimits_b \bm{\hat{\theta}}$. Бутстрэпирование не дает \textbf{асимптотических улучшений} (см. гл. 11.2.2). При условии независимости индивидуальных наблюдений оценка состоятельна при $N \rightarrow \infty$. Оценка \ref{Eq:21.14} асимптотически эквивалетна оценке \ref{Eq:21.3}, как в случае с данными пространственного типа бутстрэп пары асимптотически эквивалентны состоятельной в случае гетероскедастичности оценке Уайта. Этот бутстрэп не дает асимптотического улучшения, хотя оно возможно (см. раздел 11.6.2).

Такой бутстрэп метод  можно применить к любой оценке, которая основывается на независимости индивидуальных наблюдений и на том, что $N \rightarrow \infty$, в том числе и к доступной ОМНК-оценке модели сквозной регрессии для коротких панелей (см. раздел 21.5.2). Идея заключается в том, что нужно проводить ресэмплинг только по индивидуальным наблюдениям, а не по $i$ и $t$.

 
{\centering
Обсуждение\\}

Невозможно преувеличить важность корректирования стандартных ошибок на корреляцию в остатках на уровне индивидуальных наблюдений. Статистические пакеты еще не решают эту проблему автоматически. Бертран, Дуфло, и Маллайнатан (2004) иллюстрируют смещение вниз вычисляемых стандартных ошибок в контексте оценивания методом <<разность разностей>> (см. раздел 22.6). Они выяснили, что робастные методы для панельных данных и бутстрэп методы успешно справляются с проблемой, хоть и количество индивидуальных наблюдений $N$ (количество штатов) в их выборке достаточно мало, в то время как асимптотическая теория предполагает $N \rightarrow \infty$.

Следующий пример (см. таблицу 21.2) также показывает значимость корректирования стандартных ошибок при наличии любой корреляции или автокорреляции ошибок.

\section{Пример линейной модели панельных данных: количество часов работы и заработная плата}

Один из важных вопросов экономики рынка труда  --- чувствительность предложения труда к заработной плате. Стандартная модель предложения труда предполагает, что для уже работающих людей эффект увеличения заработной платы на предложение труда неоднозначен. Эффект дохода, стимулирующий меньше работать, компенсирует эффект замещения, стимулирующий работать больше.

Анализ данных пространственного типа о взрослых мужчинах выявил слабую положительную зависимость заработной платы и количества часов работы. Однако, вполне верятно, что это взаимосвязь ложная, так как она просто отображает ненаблюдаемое желание больше работать, связанное с более высокой оплатой труда. Анализ панельных данных позволяет учитывать такой феномен в предположении, что это ненаблюдаемое желание больше работать не меняется во времени. Например, оценка within учитывает этот феномен, измеряя насколько больше (меньше), чем среднее количество часов, работает сотрудник тогда, когда заработная плата выше (ниже) среднего.

Допустим, имеются данные о 532 мужчинах для каждого из 10 лет с 1979 по 1988 гг. (Зилиак, 1997). Исследуемая переменная --- lnhrs, натуральный логарифм ежегодного количества часов работы. Единственная объясняющая переменная --- lnwg, натуральный логарифм почасовой заработной платы. Рассмотрим регрессионную модель
\begin{align}
\mathrm {lnhrs}_{it}=\alpha_i +\beta \mathrm {lnwg}_{it}+\e_{it},
\nonumber
\end{align}
где индивидуальный эффект $\alpha_i$ в некоторых моделях упрощается до $\alpha$, и $\beta$ измеряет эластичность предложения труда. Предполагается, что ошибки $\e_{it}$ независимы по $i$, но могут быть коррелированы по $t$ для данного $i$. Как уже было упомянуто, ожидается слабая положительная эластичность предложения труда $\beta$.

Зилиак (1997) дополнительно включал в модели квадрат возраста, количество детей, индикатор плохого здоровья. Эти регрессоры и фиктивные переменные для каждого года в незначительной степени меняют оценку $\beta$ и стандартных ошибок. Для краткости они здесь пропущены. В главе 22 мы рассмотрим более общие модели, которые не исключают эндогенность lnwg и позволяют использовать лаги lnhrs в качестве регрессоров.

\subsection{Результаты оценивания}
Для 5320 наблюдений выборочные средние значения lnhrs и lnwg --- 7.66 и 2.61 соответственно, геометрическое среднее значение 2,120 часов в год и 13.60\$  в час. Выборочные стандартные отклонения  --- 0.29 и 0.43 соответственно, что говорит о более высокой дисперсии заработной платы, чем часов работы.

При анализе панельных данных полезно знать, где главным образом присутствует дисперсия --- среди индивидуальных наблюдений или же во времени. Общая дисперсия ряда $x_{it}$ может быть представлена в виде декомпозиции
\begin{align}
\sum_{i=1}^N \sum_{t=1}^T (x_{it}-\bar x)^2
& =\sum_{i=1}^N \sum_{t=1}^T [(x_{it}-\bar x_i)+(\bar x_i-\bar x)]^2 \nonumber \\
& =\sum_{i=1}^N \sum_{t=1}^T(x_{it}-\bar x_i)^2+\sum_{i=1}^N \sum_{t=1}^T(\bar x_i --- \bar x)^2,
\nonumber
\end{align}
так как $2 \sum_{i=1}^N \sum_{t=1}^T (x_{it}-\bar x_i)(\bar x_i-\bar x)=0$. Другими словами, общая сумма квадратов равна сумме \textbf{внутригрупповой суммы квадратов} и \textbf{межгрупповой суммы квадратов}. Такая декомпозиция позволяет определить \textbf{внутригрупповое стандартное отклонение} $s_W$ и \textbf{межгрупповое стандартное отклонение} $s_B$, где
\begin{align}
s^2_W =\frac{1}{NT-N} \sum_{i=1}^N \sum_{t=1}^T (x_{it}-\bar x_i)^2
\nonumber
\end{align}
и
\begin{align}
s^2_B =\frac{1}{N-1} (\bar x_i-\bar x)^2.
\nonumber
\end{align}

\begin{table}[ht]
\caption{{\it Заработная плата и количество часов работы: Стандартные оценки линейных моделей панельных данных} ${}^a$} 
\centering
\begin{tabular}{p{2cm} p{1.5cm} p{2cm} p{1.6cm} p{2cm} p{2cm} p{2cm}}
\hline \hline
				& \textbf{POLS} &  \textbf{Between} & \textbf{Within} & \textbf{First Diff} & \textbf{RE-GLS} & \textbf{RE-MLE}\\
\hline
$\alpha$ 	& 7.442	& 7.483  & 7.220& .001&  7.346&7.346\\
 $\beta$& .083&.067 &.168 &.109 & .119& .120\\
 Робастные ст.ош. ${}^b$ & (.030)&(.024) &(.085) &(.084) & (.051)&(.052) \\
 Ст.ош. (бутстрэп) & [.030] & [.019]& [.084]& [.083]& [.056]&[.058] \\
 Дефолтные ст.ош.&{.009} &{.020} & {.019}& {.021}& {.014}& {.014}\\
\hline
$R^2$& .015 &.021 & .016& .008&.014& .014\\
RMSE & .283& .177&.233&.296 & .233&.233 \\
RSS & 427.225& 0.363& 259.398& 417.944& 288.860 & 288.612\\
TSS & 433.831& 17.015& 263.677& 420.223&293.023 & 292.773\\
$\sigma_{\alpha} $&.000 & & .181& &.161 & .162\\
$\sigma_{\e}$ & .283& & .232& & .233& .233\\
$\lambda$ & 0.000& -&1.000 &- &.585 &.586 \\
$N$ &5320 & 532& 5320& 4788& 5320& 5320\\
\hline \hline
\multicolumn{7}{p{15cm}}{${}^a$ Представленные оценки: объединенная МНК-оценка (POLS), внутригрупповая (between), межгрупповая (within), в первых разностях (first diff), ОМНК-оценка со случайным эффектом (RE-GLS) и ММП-оценка линейной регрессии lnhrs на lnwg. В круглых скобках представлены стандартные ошибки для наклонов коэффициентов робастные к гетероскедастичности, в квадратных скобках --- стандартные ошибки, полученные бутстрэп методом, в фигурных скобках --- стандартные ошибки, вычисленные по умолчанию в предположении, что ошибки независимы и одинаково распределены. $R^2$, корень из среднеквадратичной ошибки (RMSE), сумма квадратов остатков (RSS), общая сума квадратов (TSS) и количество наблюдений в выборке вычислены для соответствующих моделей, описанных в разделе 21.2. Параметр $\lambda$ определен после \ref{Eq:21.11}.} \\
\multicolumn{7}{l}{${}^b$ ст.ош. --- стандартные ошибки}
\end{tabular}
\label{Tab:21.2}
\end{table}

Внутригрупповые и межгрупповые выборочные стандартные отклонения --- 0.22 и 0.18 соответственно для lnhrs и 0.19 и 0.39 для lnwg. Таким образом, большая изменчивость заработной платы по сравнению с часами работы объясняется более высокой межгрупповой изменчивостью для заработной платы. Заметим, что внутригрупповая изменчивость несколько меньше для заработной платы, чем для количества рабочих часов.

\subsection{Сравнение оценок, используемых при анализе для панельных данных}
Таблица \ref{Tab:21.2} обобщает результаты применения стандартных оценок панельных данных, представленных в разделе 21.2.2, наряду с тремя разными оценками стандартных ошибок. Как показано ниже, для надежных статистических выводов необходимо использовать робастные для панельных данных стандартные ошибки  или робастные стандартные ошибки, полученные бутстрэп методом.

{\centering
Оценки коэффициентов наклона\\}

Оценка коэффициента наклона $\beta$ различается в зависимости от метода оценивания.  Оценка between, которая использует только пространственную изменчивость, меньше, чем МНК оценка модели сквозной регрессии. Оценка within или оценка с фиксированным эффектом, равная 0.168, значительно выше, чем МНК оценка модели сквозной регрессии, равная 0.083. Оценки статистически значимы, как показывает двусторонний тест на 5\% уровне значимости, оценки стандартных ошибок равны 0.084 и 0.085. %?? 
Оценка в первых разностях, равная 0.109, также превышает МНК оценку модели сквозной регрессии, но оказывается значительно ниже оценки within, которая тоже использует только изменчивость во времени. Оценка со случайным эффектом, равная 0.119 или 0.120, лежит между оценками between и within. Это вполне ожидаемо, так как оценки со случайным эффектом могут быть представлены как \textbf{взвешенное среднее within и between оценок}. Две оценки со случайным эффектом очень близки друг к другу, так как здесь оценки дисперсий $\sigma^2_{\alpha}$ и $\sigma^2_{\e}$ одинаковы для обеих оценок, что является причиной близких значений $\hat{\alpha}=0.585$ и $\hat{\alpha}=0.586$ в регрессии \ref{Eq:21.10}. Удивительно, что оценки со случайным эффектом оказались менее эффективными, чем МНК оценки сквозной регрессии. Это говорит о том, что модель со случайным эффектом плохо описывает данные с корреляцией в ошибках. 

Какая оценка более предпочтительна? Оценка within и оценка модели в первых разностях состоятельны при любой истинной модели (модели сквозной регрессии, модели со случайным эффектом, модели с фиксированным эффектом), в то время как другие оценки несостоятельны в случае истинности модели с фиксированным эффектом. Вследствие этого  самыми робастными оценками являются оценка within и оценка модели в первых разностях, равные 0.168 и 0.109.

Однако, используя более робастные оценки, мы теряем в эффективности. Стандартные ошибки, равные 0.83 и 0.85, намного выше тех, которые получены при получении МНК оценивании сквозной регрессии и оценки со случайным эффектом. Можно использовать формальный тест Хаусмана (для более подробного описания и обсуждения см. раздел 21.4.3), чтобы определить, являются ли индивидуальные эффекты фиксированными. При относительной неточности оценивания в нашем примере, тест Хаусмана не отвергает нулевую гипотезу о случайных эффектах, несмотря на большую разницу между оценками со случайными и фиксированными эффектами. Так, более эффективная оценка со случайным эффектом могла бы быть использована здесь. Другое преимущество оценки со случайным эффектом состоит в том, что она позволяет оценивать коэффициенты регрессоров, меняющихся во времени.

{\centering
Оценивание стандартных ошибок\\}

Перейдем к сравнению оценок стандартных ошибок. Как отмечено в разделе 21.2.3, статистические выводы должны быть основаны на робастных стандарных ошибках, которые учитывают корреляцию ошибок во времени для данного индивидуального наблюдения и разные дисперсии и ковариации индивидуумов. Также, как подробнее описано в последующих разделах, стандартные ошибки, основанные на отклонениях от средних значений, таких как \ref{Eq:21.8} и \ref{Eq:21.10},  должны учитывать потерю не $K$, а $N+K$ степеней свободы.

Первая оценка стандартной ошибки вычислена с помощью робастного метода, описанного в  \ref{Eq:21.13}, а вторая вычислена с помощью бутстрэп метода (\ref{Eq:21.14}) с 500 симуляциями. Для краткости эти оценки называются робастными для панельных данных, хотя они еще робастны к гетероскедастичности. Эти две оценки очень близки друг к другу и далеки от оценок модели со случайным эффектом, где робастные для панельных данных стандартные ошибки недооценены вследствие того, что вычислены для регрессии \ref{Eq:21.10}, которые игнорируют ошибки в оценивании $\hat{\lambda}$.

Третья оценка стандартной ошибки  --- значение, вычисляемое компьютером по умолчанию в предположении, что ошибки независимы и одинаково распределены. В нашем примере правильно специфицированные стандартные ошибки выше, чем стандартные ошибки по умолчанию в три-четыре раза. Исключение составляет оценка between, оценка со стандартными ошибками, которые необходимо скорректировать только на гетероскедастичность, так как она использует только изменчивость между кросс-секциями. 

Например, для МНК оценки сквозной регрессии коэффициента $\beta$ стандартная ошибка по умолчанию равна 0.09, что приводит к неверной $t$-статистике 9.07. Робастная для панельных данных стандартная ошибка намного выше --- 0.30, поэтому верные $t$-статистики намного меньше --- 2.83. Стандартные ошибки, вычисляемые по умолчанию, предполагают независимость ошибок в модели по времени для данного $i$, в то время как на практике они часто положительно коррелированы. Из-за этого ошибочного предположения переоценивается выгода от дополнительных временных периодов, что приводит к смещению стандартных ошибок вниз (см. раздел 21.5.4). К тому же, игнорирование гетероскедастичности в ошибках также приводит к смещению, хотя это смещение могло иметь любое направление. В случае с нашими данными неудавшаяся попытка учесть гетероскедастичность также приводит к большому смещению вниз: стандартная ошибка $\hat{\beta}_{POLS}$, учитывающая гетероскедастичность, но не учитывающая корреляцию во времени для данного $i$, равна 0.020. Для любых других данных коррекция на гетероскедастичность обычно менее важна, чем коррекция на корреляцию во времени. 

Включение $\alpha_i$ для оценок between и within позволяет учитывать некоторую корреляцию ошибок во времени для данного индивидуального наблюдения. В случае с нашими данными разница между робастными для панельных данных и неробастными стандартными ошибками остается значительной, частично из-за неудавшейся попытки дополнительно учесть гетероскедастичность.

Очевидно, необходимо использовать робастные для панельных данных стандартные ошибки. 

\subsection{Графический анализ}

Проведение графического сравнения сквозной, between регрессий и регрессий с фиксированным эффектом (within или в первых разностях) очень полезно. Такие графики редко представлены в регрессиях для панельных данных, но их легко использовать в данном случае, так как у нас имеется только один регрессор.

Все графики включают линии непараметрической регрессии с использованием локально взвешенного сглаживания (Lowess, см. раздел 9.6.2) и линию линейной регрессии, которая соотносится с оценками, представленными в таблице \ref{Tab:21.2}.

Рисунок \ref{Fig:21.2} изображет зависимость lnhrs от lnwg для всех фирм за все годы (5320 наблюдений). На графике показана положительная взаимосвязь, почти линейная за исключением крайних значений. Наклон равен 0.083, а $R^2=0.015$ достаточно низок (см. табл. \ref{Tab:21.2}).

Оценка between (\ref{Eq:21.7}) вычисляется из регрессии $\bar{y}_i$  на $\bar{x}_i$. Соответствующий график представлен на рисунке \ref{Fig:21.2}, где снова замечается положительная взаимосвязь. 

Оценка within и оценка с фиксированным эффектом (21.8) вычисляется из регрессии $(y_{it}-\bar{y}_i)$  на $(x_{it}-\bar{x}_i)$. На рисунке \ref{Fig:21.3} представлен связанный график $(y_{it}-\bar{y}_i+\bar{y})$  на $(x_{it}-\bar{x}_i+\bar{x})$, где $\bar{y}=N^{-1}\sum_{i} \bar{y}_i$ и $\bar{x}=N^{-1} \sum_{i} \bar{x}_i$ и общих средних по $y$ и $x$. Сравнение с рисунком \ref{Fig:21.1} показывает, что вычитание индивидуального среднего приводит к значительному снижению изменчивости в lnwg и менее значительному снижению изменчивости в lnhrs. Наклон оказывается более крутой, чем для сквозной регрессии, и по сравнению с таблицей \ref{Tab:21.2} наклон увелиился с 0.083 до 0.168.

Оценка модели в первых разностях \ref{Eq:21.9} получается посредством построения регрессии $(y_{it}-y_{i,t-1})$ на $(x_{it} --- x_{i,t-1})$. Соответствующий график для lnhrs --- lnwg данных представлен на рис \ref{Fig:21.4}. Рисунок качественно не отличается от рисунка \ref{Fig:21.3}.
 
Вывод предшествующего анализа состоит в том, что количество часов работы чувствительно к изменению заработной платы в большей степени ввиду изменчивости во времени, а не изменчивости внутри кросс-секций.

  \begin{figure}[ht]
                \begin{center}
                    \includegraphics[scale=1.2]{chapters/fig211}
                    \caption{Количество часов работы и заработная плата: объединенная регрессия. По оси ординат --- натуральный логарифм ежегодных часов работы, по оси абсцисс --- натуральный логарифм заработной платы за час. Данные для 532 мужчин, проживающих в США, для каждого из 1979-88 гг. }
                    \label{Fig:21.1}
                \end{center}
     \end{figure}

 \begin{figure}[ht]
                \begin{center}
                    \includegraphics[scale=1.2]{chapters/fig212}
                    \caption{Количество часов работы и заработная плата: межгрупповая регрессия. По оси ординат --- среднее за 10 лет логарифма рабочих часов, по оси абсцисс  --- среднее за 10 лет логарифма заработной платы для 532 мужчин. Та же выборка, что и на рисунке \ref{Fig:21.1}}
                    \label{Fig:21.2}
                \end{center}
     \end{figure}

\subsection{Анализ остатков}

Было бы полезно рассмотреть также автокорреляцию данных и остатков. Например, для остатков $\hat{u}_{it}=y_{it} --- \hat{y}_{it}$ автокорреляция между периодом $s$ и периодом $t$ рассчитывается как $\hat{\rho}_{st} = c_{st}/\sqrt{c_{ss} c_{tt}}, s, t = 1, \dots, T$, где оценка ковариации  $c_{st}=(N-1)^{-1} \sum_i (\hat{u}_{it} = \bar{\hat{u}}_t)(\hat{u}_{is} = \bar{\hat{u}}_s)$ и $\bar{\hat{u}}_t= N^{-1} \sum_i \hat{u}_{it}$.

 \begin{figure}[ht]
                \begin{center}
                    \includegraphics[scale=1.2]{chapters/fig213}
                    \caption{Количество часов работы и заработная плата: внутригрупповая (с фиксированным эффектом) регрессия. По оси ординат --- отклонение логарифма рабочих часов от среднего за 10 лет , по оси абсцисс --- отклонение логарифма заработной платы от среднего за 10 лет  по данным о 532 мужчинах. Та же выборка, что и на рисунке \ref{Fig:21.1}}
                    \label{Fig:21.3}
                \end{center}
     \end{figure}

 \begin{figure}[ht]
                \begin{center}
                    \includegraphics[scale=1.2]{chapters/fig214}
                    \caption{Количество часов работы и заработная плата: оценка модели в первых разностях. По оси ординат --- первая разность логарифма рабочих часов, по оси абсцисс --- первая разность логарифма заработной платы по данным о 532 мужчинах. Та же выборка, что и на рисунке \ref{Fig:21.1}}
                    \label{Fig:21.4}
                \end{center}
     \end{figure}

В таблице \ref{Tab:21.3} представлена автокорреляция остатков после оценивания модели сквозной регрессии lnhrs на lnwg. Автокорреляция обычно лежит в пределах от 0.2 до 0.4 для наблюдений, находящихся в двух-девяти годах друг от друга. Степень угасания очень мала, и автокорреляция, скорее, тяготит к модели со случайным эффектом, которая предполагает, что $Cor[u_{it}, u_{is}]$  постоянная для $t \neq s$, чем к экспоненциально угасающей модели AR(1).

Автокорреляции lnhrs для регрессии очень близки к тем, что даны в таблице \ref{Tab:21.3}, так как $\hat{u}_{it} \simeq y_{it}$, что ожидаемо исходя из слабой объяснительной силы сквозной регрессии с $R^2=0.015$. Автокорреляция регрессора lnwg, не представленная здесь, значительно больше, она принимает значения от 0.9 для одного лага до 0.7 для девяти лагов.

Корреляции остатков из внутригрупповой регрессии даны в таблице \ref{Tab:21.4}. Если исходные ошибки $\e_{it}$ в \ref{Eq:21.3} независимы и одинаково распределены, тогда можно показать, что автокорреляция преобразованных ошибок $\e_{it}-\bar{\e}_i$ для любого лага равна $-1/(T-1)=-0.11$. Исключение может составлять только первый лаг, для которого корреляция всегда положительна.


\begin{table}[ht]
\caption{{\it Заработная плата и количество часов работы: Автокорреляции объединенной МНК-регрессии} ${}^a$} 
\centering
\begin{tabular}{ccccccccccc}
\hline \hline
	&	\textbf{u79} & \textbf{u80} & \textbf{u81} & \textbf{u82} & \textbf{u83} & \textbf{u84} & \textbf{u85} & \textbf{u86} & \textbf{u87} & \textbf{u88}  \\
\hline
upols79 & 1.00 & & & & & & & & & \\
upols80 & .33 	& 1.00 & & & & & & & & \\
upols81 & .44	& .40 & 1.00 & & & & & & & \\
upols82 & .30	& .31 & .57 & 1.00 & & & & & & \\
upols83 & .21	& .23 & .37 & .47 & 1.00 & & & & & \\
upols84 & .20	& .23 & .32 & .34 & .64 & 1.00 & & & & \\
upols85 & .24	& .32 & .41 & .35 & .39 & .58 & 1.00 & & & \\
upols86 & .20	& .19 & .28 & .25 & .31 & .35 & .40 & 1.00 & & \\
upols87 & .20	& .32 & .33 & .29 & .31 & .34 & .39 & .35 & 1.00 & \\
upols88 & .16	& .25 & .30 & .26 & .21 & .25 & .34 & .55 & 0.53 &  1.00\\
\hline \hline
\multicolumn{11}{p{15cm}}{${}^a$ Прим.: Автокорреляции остатков взяты из объединенной МНК-регрессии lnhrs на lnwg для 532 мужчин за 10 лет. Автокорреляции медленно угасают.}
\end{tabular}
\label{Tab:21.3}
\end{table}

\begin{table}[ht]
\caption{{\it Заработная плата и количество часов работы: Автокорреляции остатков внутригрупповой регрессии} ${}^a$} 
\centering
\begin{tabular}{ccccccccccc}
\hline \hline
	&	\textbf{u79} & \textbf{u80} & \textbf{u81} & \textbf{u82} & \textbf{u83} & \textbf{u84} & \textbf{u85} & \textbf{u86} & \textbf{u87} & \textbf{u88}  \\
\hline
upols79 & 1.00 & & & & & & & & & \\
upols80 & .10 	& 1.00 & & & & & & & & \\
upols81 & .21	& .08 & 1.00 & & & & & & & \\
upols82 & .00	& -.04 & .26 & 1.00 & & & & & & \\
upols83 & -.26& -.27 & -.21 & .01 & 1.00 & & & & & \\
upols84 & -.26& -.27 & -.30 & -.20 & .32 & 1.00 & & & & \\
upols85 & -.18& -.10 & .41 & .35 & .39 & .58 & 1.00 & & & \\
upols86 & -.19& -.25 & -.26 & -.27 & -.17 & -.14 & -.08 & 1.00 & & \\
upols87 & -.15& -.05 & -.16 & -.20 & -.24 & -.21 & -.09 & -.09 & 1.00 & \\
upols88 & -.17& -.11 & -.14 & -.18 & -.38 & -.31 & .13 & .24 & .24 &  1.00\\
\hline \hline
\multicolumn{11}{p{15cm}}{${}^a$ Прим.: Автокорреляции остатков взяты из внутригрупповой (с фиксированным эффектом) регрессии lnhrs на lnwg для 532 мужчин за 10 лет.}
\end{tabular}
\label{Tab:21.4}
\end{table}

Корреляции остатков регрессии со случайным эффектом довольно похожи на корреляции остатков регрессии с фиксированным эффектом, представленные в таблице \ref{Tab:21.4}. Корреляции остатков регрессии в первых разностях качественно близки к теоретическому результату, что если исходные ошибки $\e_{it}$ в \ref{Eq:21.3} независимы и одинаково распределены, то преобразованные ошибки $\e_{it} --- \e_{it-1}$ имеют автокорреляцию от 0.5 для первого лага до 0 для остальных лагов.

\section{Модели с фиксированным эффектом против моделей со случайным эффектом}

Модель с фиксированным эффектом привлекательна тем, что она позволяет использовать панельные данные для выявления взаимосвязей, основываясь на более слабых предположениях (представленных в разделе 21.4.1), чем те, которые необходимы при использовании данных пространственного типа или панельных данных без фиксированных эффектов, таких как объединенные модели или модели со случайным эффектом.

Для некоторых данных зависимость очевидна, поэтому модель со случайным эффектом может адекватно описывать данные, как, например, в контролируемом эксперименте об уровне урожайности в зависимости от различного количества удобрений. В других случаях использование модели со случайным эффектом может оказаться целесообразным для измерения степени корреляции.  При этом для определение причинности нужны другие подходы. Хороший пример --- связь курения и рака легких. Однако, экономисты обычно предпочитают модели со случайным эффектом модель с фиксированным эффектом, так как желают измерить \textbf{причинность}, несмотря на наблюдаемые данные.

У модели с фиксированным эффектом есть некоторые практические слабые стороны. Оценивание коэффициента любого регрессора, не меняющегося во времени, как, например, индикаторной переменной {\it пол} невозможна, так как ее влияние поглощается в индивидуальном эффекте. Коэффициенты регрессоров, меняющихся во времени, могут быть оценены, но эти коэффициенты могут быть очень неточными, если большая часть изменчивости регрессоров отражается не во времени, а  внутри кросс-секций. Прогнозирование условного среднего невозможно. Вместо этого могут быть прогнозированы только изменения условного среднего, обусловленного изменениями регрессоров, меняющихся во времени. Даже коэффициенты регрессоров, меняющихся во времени, может оказаться сложно или теоретически невозможно идентифицировать в нелинейных моделях с фиксированных эффектах (см. главу 23).  Ввиду этих причин экономисты также используют модели со случайным эффектом, даже если причинная интерпретация может быть необоснованна.

\subsection{Пример применения модели с фиксированными эффектами}

Рассмотрим влияние использования компьютера на заработную плату.  В некоторых исследованиях, использющих данные пространственного типа, (например, Крюгер (1993), ДиНардо и Пишке (1997)) выявляется, что использование компьютера в работе связано с существенно более высокими заработными платами, даже при учете многих детерминант заработной платы, таких как образование, возраст, пол, отрасль и занятость. Как было отмечено в работе ДиНардо и Пишке (1997), это необязательно говорит о причинности. Если регрессоры коррелированы с ошибкой, это может идентифицировать проблему эндогенности или пропущенных переменных.

 А именно, рассмотрим пространственную модель:
\begin{align}
y_i=\x'_i \bm\beta+\alpha_i +\e_i,
\nonumber
\end{align}
где $y$ --- это натуральный логарифм заработной платы, $\x$ --- вектор индивидуальных характеристик, включая фиктивную переменную для использования компьютера в работе, и $\e$ --- ошибка, которая по предположению независима от $\x$. Трудность состоит в добавлении ненаблюдаемой переменной $\alpha$, которая по предположению коррелирована с переменной <<использование компьютера на работе>>, а поэтому и с наблюдаемым регрессором $\x$, хотя компоненты $\x$ такие, как занятость и образование, могут частично учитывать влияние переменной {\it использование компьютера на работе}. Оценивание регрессии $y$ на $\x$ приводит к проблеме \textbf{смещения в связи с пропущенной переменной}, что в свою очередь является причиной появления несостоятельных оценок $\beta$, так как комбинированная ошибка $(\alpha + \e)$ коррелирована с $\x$.

Благодаря панельным данным можно обойти эту проблему, если предположить, что ненаблюдаемая переменная $\alpha_i$ не меняется во времени. Тогда 
\begin{align}
y_{it}=\x'_{it} \bm\beta+\alpha_i +\e_{it},
\nonumber
\end{align}
где снова $\e$  некоррелирована с $\x$ и $\alpha$ коррелирована с $\x$. Путем взятия разностей мы избавляемся от $\alpha_i$ (см. раздел 21.2.2), что позволяет получить состоятельную оценку $\beta$. В примере с компьютером, влияние использования компьютера на заработную плату измеряется взаимосвязью между индивидуальными изменениями заработной плате и индивидуальными изменениями в использовании компьютера в работе. Хайскен-Денью и Шмидт (1999) не находят подтверждение этому эффекту на панельных немецких данных.

Подход к оценке панельных данных с использованием фиксированных эффектов позволяет определить причинность при более слабых предпосылках, чем при анализе данных пространственного типа. Однако, сделать предположения все же необходимо. Ключевое предположение состоит в том, что ненаблюдаемая $\alpha_i$ неизменна во времени (иначе она содержала бы индекс $t$: $\alpha_{it}$). В нашем примере предполагается, что индивидуальная предрасположенность к использованию компьютера на работе может быть эндогенна, но $\alpha_i$, ненаблюдаемая часть эффекта этой склонности к использованию компьютера  на заработную плату, постоянна во времени, как только мы учтем наблюдаемые  $\x_{it}$. По существу, когда мы учитываем влияние ненаблюдаемой $\alpha_i$ и наблюдаемого $\x_{it}$, отдельные временные периоды, в которые работа индивидуума предполагает или не предполагает использование компьютера, предполагаются чисто случайными.

Метод оценивания объединенной регрессии и метод оценивания с использованием случайных эффектов не имеют таких свойств. Метод оценивания с использованием случайных эффектов не предполагает коррелированность $\alpha$ и $\x$, напротив, он предполагает, что 
 $\alpha$ независимы и одинаково распределены с параметрами $[0,\sigma^2]$ и поэтому некоррелированы с $\x$. Если на самом деле $\alpha$ коррелирована с $\x$, метод оценивания с использованием случайных эффектов дает несостоятельные оценки, в то время как оценка с фиксированным эффектом состоятельна, даже если $\alpha$ коррелирует с $\x$ при условии неизменности  $\alpha$ во времени. 

\subsection{Условный анализ против предельного анализа}

Оценивание модели с фиксированным эффектом  --- это \textbf{условный анализ}, исследующий влияние $\x_{it}$ на $y_{it}$, учитывая индивидуальный эффект $\alpha_i$. Построение проноза возможно только для индивидуумов той же самой выборки, и даже тогда нужна длинная панель для получения состоятельной оценки $\alpha_i$. Оценивание модели со случайным эффектом, напротив, является примером \textbf{предельного анализа} или \textbf{усредненного (population-averaged) анализа}, так как индивидуальные эффекты трактуются как независимые и одинаково распределенные случайные переменные. Выводы оценивания моделей со случайным эффектом могут распространяться не только на выборку, но и на генеральную совокупность.

Если истинная модель  --- это модель со случайным эффектом, тогда тип анализа (условный или предельный) выбирается в зависимости от конкретных данных. Если анализ проводится для случайной выборки стран, тогда необходимо использовать модель со случайным эффектом. Однако, если исследователю в действительности интересны конкретные страны, входящие в выборку, тогда применяется оценивание с фиксированным эффектом, хотя это может повлечь за собой потерю эффективности. 

Однако, если истинная модель вместо этого содержит индивидуальные эффекты, коррелированные с регрессорами, тогда использование оценки со случайным эффектом уже не имеет смысла, так как оценка со случайным эффектом несостоятельна. Вместо этого в таком случае необходимо использовать альтернативные оценки, такие как оценка с фиксированным эффектом и оценки модели в первых разностях. В целях определить причинные взаимосвязи в микроэкономических приложениях чаще используются последние методы оценки.

\subsection{Тест Хаусмана}

Если индивидуальные эффекты фиксированны, то оценка within $\hat{\beta}_W$ состоятельна, в то время как оценка со случайным эффектом $\tilde{\beta}_{RE}$ несостоятельна. Здесь $\beta$ относится к вектору коэффициентов только тех регрессоров, которые не меняются во времени. С помощью теста Хаусмана можно проверить, имеют ли место фиксированные эффекты. Тест проверяет, есть ли статистически значимая разница между этими оценками. В качестве альтернативы можно протестировать любую другую пару оценок с похожими свойствами, например, оценку модели в первых разностях и МНК оценку модели сквозной регрессии.

При большом значении статистики теста Хаусмана нулевая гипотеза о том, что индивидуальные эффекты некоррелированы с регрессорами, отвергается, и делается вывод о том, что модель с фиксированным эффектом лучше описывает данные. При этом все еще можно избежать использования модели с фиксированным эффектом. Если регрессоры коррелированы с индивидуальными эффектами, вызванными пропущенными переменными, тогда можно добавлять дополнительные регрессоры, меняющиеся или неменяющиеся во времени, и снова провести тест Хаусмана в этой расширенной модели для повторной проверки значимости фиксированных эффектов. Даже если такая корреляция сохранится, то можно оценить модель со случайным эффектом, используя метод инструментальных переменных (см. раздел 22.4.3-22.4.4). 

{\centering
Ситуация эффективности модели со  случайным эффектом\\}

Мы начнем с предположения о том, что истинная модель  --- это модель со случайным эффектом \ref{Eq:21.3}, где $\alpha_i$ независимы и одинаково распределены с параметрами $[0,\sigma^2_{\alpha}]$ и некоррелированы с регрессорами и ошибками $\e_{it}$, которые в свою очередь независимы и одинаково распределены с параметрами $[0,\sigma^2_{\e}]$.

Тогда оценка $\tilde{\beta}_{RE}$ эффективна, и из раздела 8.3 статистика \textbf{теста Хаусмана} упрощается до
\begin{align}
H=\left(\tilde{\beta}_{1,RE} --- \hat{\beta}_{1,W}\right)' \left[\hat{V}[\hat{\beta}_{1,W}]-\hat{V}[\tilde{\beta}_{1,RE}]\right]^{-1} \left(\tilde{\beta}_{1,RE} --- \hat{\beta}_{1,W}\right),
\nonumber
\end{align}
где $\beta_1$ обозначает часть вектора $\beta$, соответствующую регрессорам, меняющимся во времени, так как только эта часть может быть оценена с помощью оценки within. При нулевой гипотезе эта статистика асимптотически распределена как $\chi^2(\dim[\bm\beta_1])$.

Хаусман (1978) показал, что асимптотически эквивалентной версией этого теста является тест Вальда о $\gamma=0$ во вспомогательной регрессии,
\begin{align}
y_{it}-\hat{\lambda}\bar{y}_i=(1-\hat{\lambda})\mu + (\x_{1it}-\hat{\lambda}\bar{\x}_{1i})'\beta_1+(\x_{1it}-\bar{\x}_{1i})'\gamma+v_{it},
\label{Eq:21.15}
\end{align}
где $\x_{1it}$ обозначает регрессоры, меняющиеся во времени, и $\hat{\lambda}$ определена в \ref{Eq:21.11}. Используются только регрессоры, меняющиеся во времени. Этот результат может быть интерпретирован следующим образом. Из модели с индивидуальными эффектами \ref{Eq:21.10} следует, что $v_{it}=(1-\hat{\lambda})\alpha_i+(\e_{it}-\hat{\lambda}\bar{\e}_i)$. Оценка со случайным эффектом в действительности получена МНК-оцениванием модели \ref{Eq:21.15} c $\gamma=0$ (cм. \ref{Eq:21.10}). Если же более подходящей является спецификация с фиксированным эффектом, тогда ошибка $v_{it}$ будет коррелирована с регрессорами из-за корреляции $\alpha_i$ с регрессорами. Эта корреляция приводит к тому, что такие регрессоры, как $(\x_{it}-\bar{\x_i})$, становятся статистическими значимыми переменными в \ref{Eq:21.15}.

{\centering
Ситуация неэффективности модели со случайным эффектом\\}

Простая форма теста Хаусмана недействительна, когда $\alpha_i$ либо $\e_{it}$ не являются независимыми и одинаково распределенными, что очень часто встречается при гетероскедастичности в микроэкономических данных. Тогда оценка со случайным эффектом не эффективна при нулевой гипотезе, поэтому выражение $\hat{V}[\hat{\beta}_W]-\hat{V}[\tilde{\beta}_{RE}]$ необходимо заменить на более общее $\hat{V}[\tilde{\beta}_W-\tilde{\beta}_{RE}]$ (см. раздел 8.3).

В случае коротких панелей эта ковариационная матрица может быть состоятельно оценена с помощью бутстрэп по $i$ (см. раздел 21.2.3). Тогда робастной для панельных данных статистика Хаусмана будет
\begin{align}
\hat{V}_{Boot}[\tilde{\beta}_{1,W}]=\frac{1}{B-1} \sum^B_{b=1} 
\left(\hat{\delta}_b-\bar{\hat{\delta}}\right)
\left(\hat{\delta}_b-\bar{\hat{\delta}}\right)',
\end{align}
где $b$ обозначает $b$-ю репликацию (из $B$) (см. раздел 21.2.3), и $\hat{\delta}=\tilde{\beta}_{1,RE}-\hat{\beta}_{1,W}$. Эта статистика может быть применена к компоненте $\beta_1$, а вместо $\hat{\beta}_{1,RE}$ может использоваться  оценка $\tilde{\beta}_{1,POLS}$, и вместо  $\hat{\beta}_{1,W}$ --- оценка $\hat{\beta}_{1,FD}$.

Как альтернативу Вулдридж (2002) предлагает оценивание вспомогательной МНК регрессии \ref{Eq:21.15} и тестирование $\gamma=0$ с помощью робастных для панельных данных стандартных ошибок. Если эффекты случайны, хотя и необязательно, чтобы $\alpha_i$ и $\e_{it}$ были независимыми и одинаково распределенными, тогда $v_{it}=(1-\hat{\lambda})\alpha_i+(\e_{it}-\hat{\lambda}\bar{\e}_i)$ все еще будут коррелированы с регрессорами, даже не будучи асимптотически независимыми и одинаково распределенными. Таким образом, нужно использовать \textbf{кластерные робастные стандартные ошибки}. Если эффекты фиксированы, тогда ошибка  $v_{it}$ коррелирована с регрессорами, и такие регрессоры, как $(\x_{it}-\bar{\x_i})$, становятся статистическими значимыми переменными. Эта робастная версия вспомогательной регрессии для теста Хаусмана предпочтительнее той, которая предполагает, что  $v_{it}$ асимптотически независимы и одинаково распределены, основываясь на минимальных предположениях о распределении. Однако, не совсем ясно, совпадает ли этот тест с тестом Хаусмана, когда оценка со случайным эффектом неэффективна.


{\centering
Пример теста Хаусмана\\}

Для примера lnhrs-lnwg, представленном в таблице \ref{Tab:21.2}, сравнение оценок с фиксированным и случайным эффектом с использованием стандартных ошибок, вычисляемых по умолчанию, дает значение статистики: $H \simeq (0.168 --- 0.119)^2/(0.019^2-0.014^2)$. Так $H=14 > \chi^2_{.05}(1)=3.84$, т.е. модель со случайным эффектом отвергается.

Однако, в нашем случае тест неприменим. Статистика $H$ завышена, так как обычные стандартные ошибки в этом примере сильно смещены вниз (см. раздел 21.3.2). Более того, это смещение  --- признак того, что оценка со случайным эффектом не эффективна при нулевой гипотезе, поэтому необходимо использовать более общую форму теста Хаусмана.

Использование вспомогательной регрессии \ref{Eq:21.15}  дает робастную для панельных данных $t$-статистику  $\gamma$, равную 1.28, и как следствие $H*=1.28^2=1.65$. Так, модель со случайными эффектами не отвергается на 5\% уровне значимости. Хотя оценки эластичности заработной платы отличаются на 0.049, оценки недостаточно точные, поэтому разница статистически незначима. Заметим, что если вместо этого использовать неробастную $t$-статистику для $\hat{\gamma}$, то $t^2=13.69$, что близко к предыдущей неверной статистике теста Хаусмана.

\subsection{Более сложные модели для случайных эффектов}

Модель со случайными эффектами предполагает, что случайный эффект $\alpha_i$ распределен независимо от регрессоров. Более сложные модели, которые ближе к моделям с фиксированными эффектам, ослабляют это предположение. 

Мундлак (1978) предполагает, что в модели панельных данных \ref{Eq:21.3} индивидуальные эффекты определены \textbf{средними по времени} значениями регрессоров: $\alpha_i=\bar{\x}'_i\pi+w_i$, где $w_i$ независимы и одинаково распределены. Тогда оценивание $\beta$ и $\pi$ методом ОМНК в этой расширенной модели дает оценку $\beta$, равную оценке с фиксированным эффектом в модели \ref{Eq:21.3}. Обычная оценка $\beta$ со случайным эффектом в модели \ref{Eq:21.3}, которая ошибочно предполагает независимые и одинаково распределенные случайные эффекты, будет несостоятельна.

Чемберлин (1982, 1984) рассматривал еще более сложную модель для случайных эффектов, в которой $\alpha_i$ --- это \textbf{взвешенная сумма} регрессоров $\alpha_i=\x'_{1i}\pi_1+ \dots + \x'_{Ti} \pi_T+w_i$. Он предложил производить оценку с помощью метода минимальных расстояний (см. раздел 22.2.7), который дает оценку  $\beta$, равную оценке с фиксированным эффектом.

Смешанные линейные модели  и иерархические линейные модели раздела 24.6 позволяют применять вполне общие модели для случайных свободных членов, а также для случайных параметров наклона. Байесовский анализ панельных данных также использует такую систему (см. раздел 22.8).

В линейных моделях подход оценивания с фиксированным эффектом используется, если ненаблюдаемый индивидуальный эффект коррелирован с регрессорами. В более сложных моделях, таких как нелинейные модели, модели с фиксированными эффектами не всегда могут быть оценены. В таком случае в качестве альтернативного подхода можно использовать модели с более сложными случайными эффектами.

\section{Модели сквозной регрессии}

\textbf{Модель сквозной регрессии} или \textbf{модель с постоянными коэффициентами}:
\begin{align}
y_{it}=\alpha+\x'_{it}\bm\beta + u_{it}.
\label{Eq:21.17}
\end{align}
В статистической литература модель называется \textbf{усредненной моделью}, так как в ней явном виде не присутствует зависимость $y_{it}$ от индивидуальных эффектов. Вместо этого, индивидуальные эффекты были неявным образом усреднены. Модель со случайным эффектом  --- это частный случай такой модели, в котором $u_{it}$ равнокоррелирована по $t$ для данного $i$ (см. раздел 21.2.1).

Основная сложность предположения об отсутствии фиксированных эффектов для статистических выводов состоит в том, что распределение МНК-оценок меняется в зависимости от распределения $u_{it}$. В коротких панелях, робастные для панельных данных стандартные ошибки могут быть получены с помощью \ref{Eq:21.13}.

Однако, здесь мы фокусируемся на ОМНК оценивании, используя множество различных спецификаций, делая предположение о равнокоррелированности для  ковариационной матрицы $u_{it}$  по времени и среди индивидуальных наблюдений, как это было предложено в литературе.

Хотя мы рассматриваем преимущественно оценивание сквозной ОМНК регрессии \ref{Eq:21.17}, модели без индивидуальных фиксированных эффектов, методы, описанные в данном разделе, могут быть применены в более общих случаях для оценивания сквозной модифицированной ОМНК модели \ref{Eq:21.12} раздела 21.2.3.

\subsection{МНК, доступная ОМНК и взвешенная МНК оценки модели сквозной регрессии}

Здесь будет удобно использовать матричные обозначения. Комбинируя наблюдения по времени для данного индивидуума, определим
\begin{align}
\mathbf y_{i}=\mathbf W_{i}\bm\delta+\mathbf u_{i},
\label{Eq:21.18}
\end{align}
где $\delta=[\alpha \; \beta']'$ --- вектор размерности  $(K+1)\times 1$, $\mathbf y_i$ и $\mathbf u_i$ --- вектора размерности $T \times 1$ c $t$-ми элементами $y_{it}$ и $u_{it}$ соответственно и $\mathbf W_i$ --- матрица размерности $T \times (K+1)$, $t$-ая строка которой равна $\mathbf w'_{it}=[1 \; \x_{it}]'$. Соединяя все индивидуальные наблюдения, получаем
\begin{align}
\mathbf y=\mathbf W\bm\delta+ \mathbf u,
\label{Eq:21.19}
\end{align}
где $\mathbf y$  и $\mathbf u$ --- вектора размерности $NT \times 1$, например, $\mathbf y=[\mathbf y'_1 \dots \mathbf y'_N]'$ и $\mathbf W$ --- матрица регрессоров размерности $NT \times (K + 1)$, первым столбцом которой является единичный вектор. Мы предполагаем, что $E[\mathbf u| \mathbf W]=0$, таким образом ошибки строго экзогенны. $\Omega=E[\mathbf u \mathbf u'|\mathbf W]$.

Есть несколько возможных МНК оценок для данной модели. Они обобщены в таблице \ref{Tab:21.5}.

Во-первых, \textbf{МНК оценка модели сквозной регрессии} состоятельна и асимптотически нормальна. Однако, мала вероятность, что $\Omega=\sigma^2 \mathbf I_{NT}$, поэтому МНК оценка состоятельна за исключением некоторых особых случаев, например, когда все регрессоры не меняются во времени.  Следует использовать не обычную МНК оценку дисперсии  $\sigma^2 (\mathbf W' \mathbf W)^{-1}$, а робастную для панельных данных оценку, как в \ref{Eq:21.13}. 

Во-вторых, \textbf{доступная ОМНК оценка сквозной модели регрессии} (PFGLS) состоятельна и эффективна, если $\Omega$ правильно специфицирована и $\hat{\Omega}$ является состоятельной оценкой для $\Omega$. Некоторые из огромного количества структур $u_{it}$, а следовательно и $\bm\Omega$, которые были предложены в литературе, посвященной анализу панельных данных, и реализованы в статистических пакетах, представлены в разделах 21.5.2 и 21.5.3 для коротких и длинных панелей соответственно.

В-третьих, \textbf{взвешенная МНК оценка модели сквозной регрессии} (PWLS) является хорошим инструментом защиты от неправильной спецификации $\Omega$. Она кладет в основу ковариационной матрицы ошибок $\bm\Omega$ \textbf{вспомогательную матрицу $\bm\Sigma$}, но затем дает статистические выводы, которые достоверны даже при $\bm\Sigma \neq \bm\Omega$. Обычный МНК --- это частный случай с $\bm\Sigma = \sigma^2 \mathbf I_{NT}$. Другие выборы $\bm\Sigma$ могут увеличить эффективность.

Оценивание ковариационной матрицы МНК оценки модели сквозной регрессии требует такую $\hat{\bm\Omega}$, что $(NT)^{-1} \mathbf W' \hat{\bm\Omega} \mathbf W$ будет состоятельной оценкой для $(NT)^{-1} \mathbf W' \bm\Omega \mathbf W$.

Для коротких панелей это возможно посредством прямого применения результатов разделов 21.2.3. Для оценивания ковариационной матрицы взвешенной МНК оценки модели сквозной регрессии необходима такая $\hat{\bm\Omega}$, что $(NT)^{-1} \mathbf W' \bm\Sigma^{-1} \hat{\bm\Omega}\bm\Sigma^{-1} \mathbf W$ является состоятельной оценкой для $(NT)^{-1} \mathbf W' \bm\Sigma^{-1} \bm\Omega \bm\Sigma^{-1} \mathbf W$. Робастная для панельных данных оценка для МНК, данная в \ref{Eq:21.13}, может быть приспособлена к взвешенному МНК для оценивания сквозной регрессии путем замены $\mathbf W' \bm\Sigma^{-1} \bm\Omega \bm\Sigma^{-1} \mathbf W$ (или эквивалентного $\sum_i \mathbf W_i' \hat{\bm\Sigma}^{-1}_i \mathrm E[\mathbf u_i \mathbf u_i' | \mathbf W_i] \hat{\bm\Sigma}^{-1}_i \mathbf W_i$ при условии независимости по $i$) на  $\sum_i \mathbf W_i' \bm\Sigma^{-1}_i \hat{\mathbf u}_i \hat{\mathbf u}_i'  \hat{\bm\Sigma}^{-1}_i \mathbf W_i$, где $\hat{\mathbf u}_i=\mathbf y_i --- \mathbf W_i \hat{\delta}$. В качестве альтернативы можно использовать панельный бутстрэп.

\begin{table}[ht]
\caption{{\it Заработная плата и количество часов: Автокорреляции объединенной МНК-регрессии}} 
\centering
\begin{tabular}{p{4.5cm} c p{4cm}}
\hline \hline
	Оценка & Формула ${}^a$ & Ковариационная матрица	${}^b$ \\
\hline
МНК оценка: $\hat{\delta}_{POLS}$ & $(\mathbf W' \mathbf W)^{-1} \mathbf W' \mathbf y$ & $(\mathbf W' \mathbf W)^{-1} \mathbf W' \hat{\bm\Omega} \mathbf W (\mathbf W' \mathbf W)^{-1}$ \\
доступная ОМНК оценка: $\hat{\delta}_{PFGLS}$& $(\mathbf W' \hat{\bm\Omega}^{-1} \mathbf W)^{-1} \mathbf W' \hat{\bm\Omega}^{-1} \mathbf y$ &  $(\mathbf W' \hat{\bm\Omega}^{-1} \mathbf W)^{-1}$\\
взвешеная МНК оценка:  $\hat{\delta}_{PWLS}$ & $(\mathbf W' \hat{\bm\Sigma}^{-1} \mathbf W)^{-1} \mathbf W' \hat{\bm\Sigma}^{-1} \mathbf y$ & $(\mathbf W' \hat{\bm\Sigma}^{-1} \mathbf W)^{-1} \mathbf W' \hat{\bm\Sigma}^{-1} \hat{\bm\Omega} \mathbf W \times (\mathbf W' \hat{\bm\Sigma}^{-1} \mathbf W)^{-1}$  \\
\hline \hline
\multicolumn{3}{p{15cm}}{${}^a$ Формулы для модели $\mathbf y =\mathbf W \sigma + \mathbf u$ определены в \ref{Eq:21.19} и матрица ошибок $\Omega$.} \\
\multicolumn{3}{p{15cm}}{${}^b$ Для вычисления $\hat{\Omega}$ для коварационных матриц POLS и PWLS см. текст; в этих случаях $\hat{\Omega}$ не обязательно должна быть состоятельна для $\Omega$. Для 
доступной ОММ оценки модели сквозной регрессии предполагается, что $\hat{\Omega}$ состоятельна для $\Omega$.}
\end{tabular}
\label{Tab:21.5}
\end{table}

\subsection{Ковариационная матрица ошибок для коротких панелей}

В коротких панелях мы обладаем всего лишь несколькими временными периодами и большим количеством индивидуальных наблюдений (обычно это индивидуумы или фирмы). Предполагается, что ошибки независимы по индивидуальным наблюдениям, т.е. $Cov[u_{it}, u_{js}]=0, i \neq j$. В таких случаях удобно перейти к обозначениям суммирования. Например, PFGLS оценку, представленную в таблице \ref{Tab:21.5}, можно записать в виде
\begin{align}
\hat{\bm\beta}_{PFGLS}=\left[\sum^N_{i=1} \mathbf W'_i \hat{\bm\Omega}^{-1}_i \mathbf W_i \right]^{-1} \sum^N_{i=1} \mathbf W'_i \hat{\bm\Omega}^{-1}_i \mathbf y_i,
\label{Eq:21.20}
\end{align}
где $\hat{\bm\Omega}_i$ является состоятельной оценкой для 
\begin{align}
\bm\Omega_i =\mathrm E[\mathbf u_i \mathbf u'_i | \mathbf W_i],
\label{Eq:21.21}
\end{align}
и $\bm\Omega_i$ является недиагональной матрицей, так как ошибки для данного индивидуального наблюдения, скорее всего, будут коррелированы по времени. Заметим, что $\hat{\bm\Omega}_i$ должна быть получена из оценивания модели, специфицированной специально для $\bm\Omega_i$. Мы не можем использовать $\hat{\bm\Omega}_i=\hat{\mathbf u}_i\hat{\mathbf u}'_i$ (см. относящееся к этому обсуждение после уравнения 5.88. %\ref{Eq:5.88}).

{\centering
Равнокоррелированные ошибки\\}

Наиболее часто используемая структура ошибок описана в рамках модели со случайными эффектами в разделе 21.2.1. Из (21.6) видно, что $\bm\Omega_i$ имеет одинаковые диагональные элементы $\sigma^2_{\alpha}+\sigma^2_{\e}$ и одинаковые недиагональные элементы $\sigma^2_{\alpha}$. Другими словами, ошибки \textbf{равнокоррелированы}, когда $\bm\Omega_i$ имеет одинаковые диагональные элементы $\sigma^2$ и одинаковые недиагональные элементы $\rho\sigma^2$. При использовании доступной ОМНК оценки требуется оценить только $\sigma^2_{\alpha}$ и $\sigma^2_{\e}$ или $\sigma^2$ и $\rho$ (см. разделы 21.2.2 и 21.7).

{\centering
Ошибки, имеющие ARMA структуру\\}

Альтернативно можно предположить, что ошибки имеют структуру ARMA модели. Например, модель ошибок AR(1) предполагает, что $u_{it}=\rho u_{i,t-1}+\e_{it}$, где $e_{it}$ независимы и одинаково распределены. Тогда  $Cov[u_{it}, u_{is}]=\rho^{|t-s|}\sigma^2$. В этом случае ковариация ошибок падает по мере увеличения временных промежутков между ними. Модель со случайными эффектами и модель AR(1) сравниваются в разделе 21.5.4.

Бальтаджи и Ли (1991) комбинируют эти две модели и рассматривают модель со случайным эффектом с ошибками вида AR(1). Эту модель можно без труда обобщить до случая AR(p), а методы, использующие ошибки вида MA и ARMA (см. раздел 5.8.7) в моделях со случайными эффектами, были разработаны относительно недавно. Краткий обзор этих моделей представлен у Бальтаджи (2001, глава 5).  

{\centering
Гомоскедастичные ошибки и неструктурированная автокорреляция \\}

При использовании доступной ОМНК оценки в коротких панелях в действительности нет необходимости налагать на ошибки такие структуры, как в модели со случайным эффектом или AR(1) модели, если сделано предположение о неизменности матрицы $\bm\Omega_i$ размерности $T \times T$ для индивидуальных наблюдений. В таком случае остается оценить только $T(T+1)/2$ ковариационных параметров. Тогда состоятельной оценкой $\bm\Omega_i$ будет $\hat{\bm\Omega}_i$ с $(t,s)$-м элементом $\hat{\sigma}_{ts}=N^{-1} \sum^N_{i=1} \hat{u}_{it} \hat{u}_{is}$. Предшествующие модели также предполагают гомоскедастичность, но накладывают дополнительный структурированный вид на $\bm\Omega_i$.

{\centering
Робастные статистические выводы\\}

Как и в предыдущих спецификациях предположим, что ковариации ошибок одинаковы для индивидуальных наблюдений, что решает проблему гетероскедастичности. В случае с короткой панелью можно тем не менее использовать предшествующие модели матрицы ошибок как основу для взвешенной МНК оценки сквозной регрессии, но тогда следует использовать робастные стандартные ошибки, как это обсуждалось после таблицы \ref{Tab:21.5}. Вместо этого также можно использовать  более сложные  смешанные модели, представленные в главе 22.

В главах 21-23 выполняется предположение о независимости по $i$. На самом деле оно может быть ослаблено даже для малых $T$, если на корреляцию наложить определенную структуру. Хорошим примером может послужить модель для пространственной корреляции для панельных данных по регионам, таким как штаты или страны. Корреляции снижаются по мере увеличения физического расстояния между индивидуальными наблюдениями. 

\subsection{Ковариационная матрица ошибок для длинных панелей}

В {\bf длинных панелях} мы располагаем множеством временных периодов, но относительно небольшим количеством индивидуальных наблюдений. Такие данные имеют место быть в микроэконометрическом анализе, если индивидуальными наблюдениями являются небольшое количество регионов, таких как штаты, страны, или фирмы. Однако эти объекты наблюдаются в течение достаточного количества периодов, чтобы статистические выводы основывались на предположении $T \rightarrow \infty$. 

Корреляция во времени для данного индивидуального наблюдения может быть описана с помощью ARMA  модели ошибок, в которых параметры ARMA модели меняются только по $i$, так как $N$ сейчас фиксировано, а  $T \rightarrow \infty$. Например, рассмотрим ошибку вида AR(1): $u_{it}=\rho u_{i,t-1}+\e_{it}$, где ошибка $\e_{it} \thicksim [0,\sigma^2_i]$ гетероскедастична и $\rho_i$ также различается по индивидуальным наблюдениям. Отдельные регрессии $y_{it}$ на $\mathbf w_{it}$
для каждого индивидуального наблюдения с ошибками вида AR(1) с использованием $T$ временных периодов дают состоятельные оценки $\hat \rho_i$ и $\hat \sigma^2_i$, так как  $T \rightarrow \infty$. Эти оценки могут быть использованы для получения доступной ОМНК оценки $\delta$ при использовании уже всех $NT$ наблюдений. Более подробно информацию можно найти у Кмента (1986). Эта модель позволяет учитывать как гетероскедастичность по индивидуальным наблюдениям, так и корреляцию во времени для данного индивидуального наблюдения. Песаран (2004) предлагает более расширенную модель, оцениваемую при помощи ОМНК.

При оценивании длинных моделей можно представить корреляцию индивидуальных наблюдений, т.е. $Cov[u_{it}, u_{jt}] \neq 0$ для $i \neq j$, так как $N$ фиксировано и асимптотически результаты основываются на $T \rightarrow \infty$. В частности, можно применить ОМНК оценку сквозной регрессии, как это предлагалось ранее, предполагая независимость по $i$, а затем посчитать стандартные ошибки с использованием упомянутого в разделе 6.4.4 метода Ньюи и Веста (1987b), который позволяет учитывать случайную зависимость между индивидуальными наблюдениями  и во времени, если зависимость во времени нивелируется достаточно быстро. Более подробно см. у Ареллано (2003, p.19). 

Аспекты анализа временных рядов для панельных данных обсуждаются более подробно в разделе 22.5 для моделей с использованием лагов зависимых переменных в качестве регрессоров.

\subsection{Влияние автокоррелированных ошибок}

Ошибки регрессионных моделей панельных данных обычно автокоррелированы во времени для данного индивидуального наблюдения. Если фиксированные эффекты отсутствуют, тогда модель сквозной регрессии в этом случае дает состоятельные оценки параметров. Однако \textbf{корреляция ошибок} может привести к \textbf{большому смещению} в стандартных ошибках сквозной регрессии, если автокорреляция не учитывается, и относительно меньшему увеличению эффективности в связи с увеличением длины панели.

Анализ достаточно прост для оценивания среднего $y$ на основе $T$ наблюдений для одного индивидуального наблюдения ($N=1$) c равнокоррелированными ошибками. Тогда $y_t=\beta + u_t$, и МНК оценка --- это выборочное среднее: $\hat{\beta}=\bar{y}=T^{-1} \sum_t y_t$. Истинная дисперсия МНК оценки равна $V[\hat{\beta}]=V[\bar{y}]=T^{-2}\sum_t \sum_s Cov[u_{t},u_{s}]$. В предположении о равнокоррелированности двойная сумма состоит из $T$ дисперсий, равных $\sigma^2$, и $T(T-1)$ ковариаций, равных $\rho \sigma^2$. Тогда $V[\bar{y}]=T^{-1}\sigma^2(1+\rho (T-1))$. Получается, что $V[\bar{y}]=T^{-1}\sigma^2$ должно быть домножено на $(1+\rho (T-1))$. В частности, $V[\bar{y}]$ стремится к $\sigma^2$ по мере того, как $\rho \rightarrow \infty$.

\begin{table}[ht]
\caption{{\it Дисперсии оценок сквозной регрессии с равнокоррелированными ошибками$^a$}} 
\centering
\begin{tabular}{ccccccc}
\hline \hline
	\textbf{T} & $\rho=\mathbf{0.0}$& $\rho=\mathbf{0.2}$	& $\rho=\mathbf{0.4}$	& $\rho=\mathbf{0.6}$	& $\rho=\mathbf{0.8}$	& $\rho=\mathbf{1.0}$ \\
\hline
1	&1.00	&1.00	&1.00	&1.00	&1.00	&1.00\\
2	&0.50	&0.60	&0.70	&0.80	&0.90	&1.00\\
5	&0.20	&0.36	&0.52	&0.68	&0.84	&1.00\\
10	&0.10	&0.28	&0.46	&0.64	&0.82	&1.00\\
\hline \hline
\multicolumn{7}{p{14cm}}{$^a$Представлены диспресии МНК оценки сквозной регрессии при растущей корреляции $\rho$ равнокоррелированных ошибок для модели, включающей только свободный член, с дисперсией ошибок, пронормированной к единице. Предполагается, что ошибки коррелированы, хотя гомоскедастичны.} \\
\end{tabular}
\label{Tab:21.6}
\end{table}

Таблица \ref{Tab:21.6} показывает влияние корреляции на дисперсию $\bar{y}$ для разных значений $T$ и $\rho$, где для простоты мы положим, что $\sigma^2=1$. Точность оценивания значительно падает с увеличением $\rho$, и оценка $V[\bar{y}]$ в предположении независимости (предполагая для простоты, что $\sigma^2$ известна) может сильно недооценить истинную дисперсию. Более того, для $\rho > 0$ увеличение точности в связи с увеличением количества временных периодов намного меньше, чем в случае с независимыми данными, когда увеличение временного интервала вдвое уменьшит дисперсию оценки в два раза. Например, если $\rho=0.4$, тогда в случае использования пяти временных периодов дисперсия оценки только в два раза меньше ($1/0.52$), чем в случае оценивания одномоментной выборки. В случае же с независимыми данными уменьшение дисперсии оценки меньше в целых пять раз ($1/0.2$). Кроме того, удвоение количества периодов с 5 до 10 приводит лишь к небольшому снижению дисперсии оценки с 0.52 до 0.46.

Этот результат применим в более общем виде для сбалансированных панелей с равнокоррелированными ошибками и регрессорами, не изменяющимися во времени, где истинная дисперсия МНК оценки равна $(1+\rho(T-1))$, умноженное на дисперсию, в предположении о независимости ошибок (см. Kloek, 1981). На практике регрессоры, меняющиеся во времени, тоже включены в регрессию, что делает более трудным получение каких-либо точных аналитических результатов. Скотт и Холт (1982) показали, что для регрессий со свободным членом и одним регрессором, меняющимся во времени, дисперсия коэффициента наклона увеличивается на множитель $(1+\hat{\rho}_x\rho(T-1))$, где $\hat{\rho}_x$ может рассматриваться как оценка индивидуальной автокорреляции в $x$. В случае с панельными данными  $\hat{\rho}_x$ обычно высок, так что увеличение достаточно значительное. Эти результаты также применимы к другим формам кластеризованных  данных и более подробно представлены в разделе 24.5.2.

Вышеописанный анализ предполагает равнокоррелированные ошибки. Это свойство модели со случайным эффектом. Если вместо этого ошибки описываются процессом AR(1), то выгода от увеличения длины панели будет значительно больше. В этом случае $\mathrm{Cov}[u_{t}, u_{s}]=\rho^{|t-s|}\sigma^2$, а $V[\bar{y}]=T^{-2}\sigma^2[T+2\sum^{t-1}_s=1(T-s)\rho^s]$. Например, если $\rho=0.8$, тогда $V[\bar{y}]=0.72\sigma^2$ для $T=5$ и $0.54\sigma^2$ для $T=10$, что ниже, чем соответствующие значения из таблицы \ref{Tab:21.6} $0.84\sigma^2$ и $0.82\sigma^2$ для равнокоррелированных ошибок с $\rho=0.8$, но тем не менее выше значений $0.2\sigma^2$ и $0.1\sigma^2$ для $\rho=0.0$.

Микроэконометристы для коротких панелей чаще всего используют модель со случайным эффектом или модели с равнокоррелированными ошибками Например, рассмотрим данные для большого количества семей о разных близнецах в семье. Разумно предполагать, что корреляции ненаблюдаемых характеристик близнецов в одной и той же семье одни и те же для разных семей. Например, корреляция между первыми и вторыми близнецами равна корреляции между первыми и третьими близнецами. Те, кто используют длинные панели, зачастую обладают исторической информацией и предполагают, что корреляция снижается со временем. Поэтому для ошибок они используют модель AR(1). 

Какая модель корреляции ошибок во времени будет более оправданной, зависит от данных. Во множестве коротких панелей, используемых в микроэкономических приложениях, при МНК оценивании сквозной регрессии присутствуют автокорреляции ошибок. Эти автокорреляции качественно близки к тем, что даны в таблице \ref{Tab:21.3}. Они более точно описываются моделью со случайным эффектом, чем AR(1), хотя может также хорошо подходить и ARMA(1,1). Однако модель со случайным эффектом с ошибками вида AR(1) может быть лучше. Во всех случаях корреляция ошибок приводит к потере информации и к тому, что обычные стандартные ошибки МНК недооценивают истинные стандартные ошибки. Для коротких панелей статистические выводы можно основывать на робастных для панельных данных стандартных ошибках (см. раздел 21.2.3), которые не требуют спецификации модели для корреляции ошибок.

\subsection{Количество часов работы и заработная плата. Пример сквозной ОМНК регрессии}

ОМНК оценки сквозной регрессии и соответствующие им робастные стандартные ошибки, а также стандартные ошибки, вычисляемые по умолчанию, модели $y_{it}=\alpha_i+\beta x_{it} + u_{it}$ для регрессии lnhrs на lnwg представлены в таблице 21.7. Предполагается, что ошибки $u_{it}$ независимы и одинаково распределены по $i$. Предположения относительно корреляции $u_{it}$ по $t$ различны.

Первая колонка таблицы \ref{Tab:21.7}, предназначенная для МНК оценки сквозной регрессии, повторяет первую колонку таблицы \ref{Tab:21.2}.

ОМНК оценки сквозной регрессии, предполагающие равнокоррелированные ошибки, записаны во второй колонке таблицы \ref{Tab:21.7}. Они совпадают с колонкой RE-GLS в таблице \ref{Tab:21.2}, так как модель со случайным эффектом предполагает равнокоррелированные ошибки (см. (21.6)).

ОМНК оценки сквозной регрессии, предполагающие ошибки вида AR(1): $u_{it}=\rho u_{it-1}+\e_{it}$, где $\e_{it}$ независимы и одинаково распределены, представлены в третей колонке таблицы \ref{Tab:21.7}. Оценка коэффициента наклона близка к МНК оценке сквозной регрессии.

ОМНК оценки сквозной регрессии, не накладывающие никакой структуры на корреляцию ошибок, кроме гомоскедастичности, т.е. $Cov[u_{it}, u_{is}]=\sigma_{ts}$, даны в четвертой колонке таблицы \ref{Tab:21.7}. Тогда оценка $\sigma_{ts}$ будет состоятельна при малом $T$ для всех $t$ и $s$: $\hat{\sigma}_{ts}=N^{-1}\sum^N_{i=1}\hat{u}_{it}\hat{u}_{is}$. Эти оценки также близки к МНК оценкам сквозной регрессии.

Из таблицы \ref{Tab:21.7} видно, что робастные для панельных данных стандатные ошибки следует предпочесть стандартным ошибкам, вычисленным по умолчанию, и предполагающим гомоскедастичность и правильно специфицированную модель для корреляции во времени.


 \begin{table}[ht]
\caption{{\it Количество часов работы и заработная плата: МНК и ОМНК оценки сквозной регрессии$^a$}} 
\centering
\begin{tabular}{cc p{2cm} cc}
\hline \hline
	Оценка & POLS & \multicolumn{3}{c}{PFGLS}\\
 Корреляция ошибок & --- & Equi & AR(1) & General\\
\hline
$\alpha$ & 7.442 & 7.346 & 7.440 & 7.426\\
$\beta$	& .083	& .120 	&.084 & .091 \\
Робастные станд. ош.		& (.029) &(.052) & (.037) & (.050) \\
Бутстрэп станд. ош.			& [.032] & [.060] & [.050] &[-] \\
Станд. ош. по умолчанию	& \{.009\} & \{.014\}& \{.012\}  & \{.014\} \\
\hline \hline
\multicolumn{5}{p{14cm}}{$^a$МНК и ОМНК модели линейной сквозной регреccии lnhrs на lnwg для короткой панели. Предполагается независимость и одинаковая распределенность по $i$ и отсутствие фиксированных эффектов.  ОМНК оценка сквозной регрессии предполагает равнокоррелированность  ошибки или ошибки со случайными эффектами, ошибки вида AR(1), или отсутствие какой-либо структуры корреляций. Стандартные ошибки для коэффициента наклона робастные для панельных данных выписаны в скобках, полученные с помощью бутстрэп --- в квадратных скобках, и оценки, вычисляемые по умолчанию, предполагающие, что ошибки независимы и одинаково распределены --- в фигурных скобках.} \\
\end{tabular}
\label{Tab:21.7}
\end{table}

\section{Модели с фиксированными эффектами}
 

\textbf{Модель с фиксированными эффектами} предполагает 
\begin{align}
y_{it}=\alpha_i +\x'_{it}\bm\beta +\e_{it},
\label{Eq:21.22}
\end{align}
где индивидуальные эффекты $\alpha_1, \dots , \alpha_N$ измеряют ненаблюдаемую гетерогенность, которая, вероятно, коррелирует с регрессорами, $\x_{it}$ и $\bm\beta$ --- вектора размерности $K \times 1$. Для начала ошибки $\e_{it}$ предполагаются независимыми и одинаково распределенными с параметрами $[0, \sigma^2]$.

Трудность оценивания составляет наличие $N$ индивидуальных эффектов, которые увеличиваются в количестве по мере того, как $N \rightarrow \infty$. С практической точки зрения мы больше заинтересованы в $K$ коэффициентах наклона $\bm\beta$, которые отображают  предельный эффект изменения регрессоров $\delta \mathrm E[y_{it}]/ \partial \x_{it}=\bm\beta$. Параметры $\alpha_1, \dots, \alpha_N$ являются \textbf{вспомогательными параметрами} или \textbf{второстепенными параметрами}, они не представляют существенного интереса. Тем не менее, их присутствие мешают оцениванию параметров $\bm\beta$, которые являются предметом изучения.

Есть несколько способов получить состоятельные оценки $\bm\beta$ для линейной модели, несмотря на присутствие вспомогательных параметров. Это (1) МНК в модели within \ref{Eq:21.8}; (2) прямое МНК оценивание модели \ref{Eq:21.2} с фиктивными переменными для каждого из $N$ фиксированных эффектов; (3) ОМНК модели within \ref{Eq:21.8}; (4) оценка ММП при условии индивидуальных средних $\bar{y}_i$, $i=1, \dots, N$; и (5) МНК оценка модели в первых разностях \ref{Eq:21.9}.

Первые два метода всегда приводят к одинаковой оценке $\bm\beta$. К этой же оценке приводят третий метод, если дополнительно предположить, что $\e_{it}$ в модели \ref{Eq:21.22} независимы и одинаково распределены, и четвертый метод, если $\e_{it} \thicksim \mathcal N[0, \sigma^2]$. Последний метод отличается от остальных для случая  $T>2$. Для нелинейных моделей, рассматриваемых в главе 23, такие эквивалентные соотношения не характерны. 

Важный результат для оценки within представлен в следующем разделе. Оценка в первых разностях, описанная в разделе 21.6.2, используется и в главе 22, когда регрессоры уже не являются строго экзогенными. Другие оценки представлены в приложении раздела 21.6, которое некоторые читатели могут предпочесть пропустить.

\subsection{Оценка within или оценка с фиксированным эффектом}

Модель within получается посредством вычитания усредненной по времени модели $\bar{y}_i=\alpha_i+\bar{\x}'_i\bm\beta +\bar{\e}_i$ из первоначальной модели:
\begin{align}
y_{it}-\bar{y}_i=(\x'_{it}-\bar{\x}_i)'\bm\beta +(\e_{it}-\bar{\e}_i).
\label{Eq:21.23}
\end{align}
Таким образом, фиксированный эффект $\alpha_i$ уничтожается наряду с регрессорами, не меняющимися во времени, так как $\x_{it}-\bar{\x}'_i=\mathbf 0$, если $\x_{it}=\x_i$ для всех $t$.

Использование МНК дает \textbf{оценку within} или \textbf{оценку с фиксированным эффектом} $\hat{\bm\beta}_W$, где 
\begin{align}
\hat{\bm\beta}_W=\left[ \sum^N_{i=1} \sum^T_{t=1} (\x_{it}-\bar{\x}_i)(\x_{it}-\bar{\x}_i)'\right]^{-1} \sum^N_{i=1} \sum^T_{t=1} (\x_{it}-\bar{\x}_i)(y_{it}-\bar{y}_i).
\label{Eq:21.24}
\end{align}

 Индивидуальные эффекты $\alpha_i$ могут быть оценены следующим образом:
\begin{align}
&\hat{\alpha}_i=\bar{y}_i-\x'_{it}\hat{\bm\beta}_W, & i=1, \dots, N.
\label{Eq:21.25}
\end{align}

Оценка $\hat{\alpha}_i$ является несмещенной оценкой $\alpha_i$. Она также является состоятельной при $T \rightarrow \infty$, так как $\hat{\alpha}_i$ усредняет $T$  наблюдений. В коротких панелях оценки $\hat{\alpha}_i$ несостоятельны, но $\hat{\bm\beta}_W$ остается состоятельной оценкой $\bm\beta$. Параметры  $\alpha_i$ являются \textbf{вспомогательными}. Их, к счастью, не требуется оценивать состоятельно в целях получения состоятельных оценок более существенных параметров $\beta$. Этот удивительный результат не обязательно верен  в более сложных моделях с фиксированными эффектами, например, в нелинейных моделях.

{\centering
Состоятельность оценки within\\}

Оценка within параметра $\bm\beta$ состоятельна, если  $\mathrm{plim}(NT)^{-1}\sum_i\sum_t(\x_{it}-\bar{\x_i})(\e_{it}-\bar{\e}_i)=\mathbf 0$. Это выполняется, если $N \rightarrow \infty$ или $T \rightarrow \infty$ и 
\begin{align}
\mathrm E[\e_{it}-\bar{\e}_i|\x_{it}-\bar{\x}_i]=0.
\label{Eq:21.26}
\end{align}
 
В связи с присутствием средних $\bar{\x}_i=T^{-1}\sum_i\x_{it}$ и $\bar{\e}_i$ это условие сильнее, чем $\mathrm E[\e_{it}|\x_{it}]=0$. Существенное условие для \ref{Eq:21.26} --- строгая экзогенность: $\mathrm E[\e_{it}|\x_{i1}, \dots, \x_{iT}]=0$. Это условие мешает использовать оценки within при включении лаговых эндогенных переменных как регрессоров (см. раздел 22.5).

{\centering
Асимптотическое распределение и оценка within\\}

Распределение $\hat{\bm\beta}_W$ может быть достаточно сложным, так как ошибка $(\e_{it}-\bar{\e}_i)$ в модели within \ref{Eq:21.8} коррелирована по $t$ для данного $i$. Ниже показывается, что обычный МНК тем не менее применим. При строгом предположении, что $\e_{it}$ независимы и одинаково распределены,
\begin{align}
\mathrm V[\hat{\bm\beta}_W]=\sigma^2_{\e} \left[ \sum^N_{i=1} \sum^T_{t=1} \ddot{\x}_{it}\ddot{\x}'_{it} \right]^{-1},
\label{Eq:21.27}
\end{align}
где $\ddot{\x}_{it}=\x_{it}-\bar{\x}_i$. Состоятельной и несмещенной оценкой $\sigma^2_{\e}$ будет $\hat{\sigma}^2_{\e}=[N(T-1)-K]^{-1}\sum_i\sum_t \hat{\e}^2_{it}$, где количество степеней свободы равно размеру выборки $NT$ за вычетом числа параметров модели $K$ и количества $N$ индивидуальных эффектов. Заметим, что если регрессия \ref{Eq:21.23} оценивается с помощью стандартного подхода МНК, то необходимо увеличить вычисленную дисперсию на $[N(T-1)-K]^{-1}[NT-K]$.

Для коротких панелей формула \ref{Eq:21.13} дает робастную оценку асимптотической дисперсии
\begin{align}
\mathrm V[\hat{\bm\beta}_W]= \left[ \sum^N_{i=1} \sum^T_{t=1} \ddot{\x}_{it}\ddot{\x}'_{it} \right]^{-1}
\sum^N_{i=1} \sum^T_{t=1} \sum^T_{s=1} \ddot{\x}_{it}\ddot{\x}'_{is} \hat{\ddot{\e}}_{it} \hat{\ddot{\e}}_{is}
\left[ \sum^N_{i=1} \sum^T_{t=1} \ddot{\x}_{it}\ddot{\x}'_{it} \right]^{-1},
\label{Eq:21.28}
\end{align}
где $\ddot{\e}_{it}=\e_{it}-\bar{\e}_i$. Эта оценка допускает любую автокорреляцию $\e_{it}$ и гетероскедастичность, поэтому более предпочтительна. 


{\centering
Вывод дисперсии оценки within\\}

Здесь мы  выведем оценку дисперсии \ref{Eq:21.27} и \ref{Eq:21.28} оценки within при помощи матричной алгебры. Начнем с модели для $i$-го наблюдения
\begin{align}
y_{it}=\alpha_i+\x'_{it}\bm\beta+\e_{it},
\nonumber
\end{align}
где $\x_{it}$ и $\bm\beta$ --- вектора размерности $K \times 1$. Для $i$-го индивидуального наблюдения, запишем все $T$ наблюдений:
\begin{align}
&\begin{bmatrix}
 y_{i1} \\
 \vdots \\
 y_{iT}
\end{bmatrix}
=
\begin{bmatrix}
1 \\ \vdots  \\ 1
\end{bmatrix}
\alpha_i + 
\begin{bmatrix}
\x'_{i1} \\ \vdots \\  \x'_{iT} 
\end{bmatrix}
 \bm\beta + 
\begin{bmatrix}
 \e_{i1} \\ \vdots \\ \e_{iT} 
\end{bmatrix}
, & i=1, \dots N,
\nonumber
\end{align}
или
\begin{align}
& \mathbf y_{it}=\mathbf e \alpha_i+\mathbf X'_{it}\bm\beta+\bm\e_{it},
& i=1, \dots, N,
\label{Eq:21.29}
\end{align}
где $\mathbf e=(1,1, \dots, 1)'$  --- единичный  вектор размерности $T \times 1$, $\mathbf X_i$  --- матрица размерности $T \times K$, и $\mathbf y_i$ и $\bm\e_i$ --- вектора размерности $T \times 1$.

Для преобразования модели \ref{Eq:21.29} в модель within, которая получается посредством вычитания среднего по индивидуальным наблюдениям, введем матрицу размерности $T \times T$
\begin{align}
\mathbf Q=\mathbf I_T --- T^{-1} \mathbf e \mathbf e'.
\label{Eq:21.30}
\end{align}
Умножение на матрицу $\mathbf Q$ дает отклонения от средних, так как
\begin{align}
\mathbf Q \mathbf W_i= \mathbf  W_i --- \mathbf e \bar{\mathbf w}'_i,
\label{Eq:21.31}
\end{align}
где $\mathbf W_i$ --- матрица размерности $T \times m$ с $t$-ой строкой $\mathbf w'_{it}$ и $\bar{\mathbf w}'_i=T^{-1} \sum^T_{t=1} \mathbf w_{it}$ --- вектор средних размерности $m \times 1$. Для получения результата \ref{Eq:21.31} используется равенство $\mathbf e' \mathbf W_i=T\bar{\mathbf w}'_i$. Заметим также, что $\mathbf Q\mathbf Q'=\mathbf Q$, кроме того $\mathbf e \mathbf e'=T$ и $\mathbf Q \mathbf e =\mathbf 0$,  поэтому $\mathbf Q$ идемпотентна.

Умножая на $\mathbf Q$ модель с фиксированными эффектами \ref{Eq:21.29} для $i$-го индивидуального наблюдения, используя $\mathbf Q \mathbf e= \mathbf 0$, получаем
\begin{align}
&\mathbf Q \mathbf y_i= \mathbf Q\mathbf  X_i \bm\beta + \mathbf Q \bm\e_i,
&i=1, \dots, N,
\label{Eq:21.32}
\end{align}
Это модель within \ref{Eq:21.23}, так как это эквивалентно выражению $\mathbf y_i --- \mathbf e \bar{y}'_i = (\mathbf X_i --- \mathbf e \bar{\mathbf x}'_i ) \bm\beta + (\e_i- \mathbf e \bar{\e}_i)$, полученному с помощью \ref{Eq:21.31}. Умножение на $\mathbf Q$ дает модель within. МНК оценивание \ref{Eq:21.32} дает $\hat{\bm\beta}_W$ с ковариационной матрицей, предполагающей независимость по $i$, равной
\begin{align}
V[\hat{\bm\beta}]=\left[ \sum_{i=1}^N \mathbf X'_i \mathbf Q' \mathbf Q \mathbf X_i \right]^{-1} 
\sum_{i=1}^N \mathbf X'_i \mathbf Q' \mathrm V[\mathbf Q \bm\e_i | \mathbf X_i] \mathbf Q \mathbf X_i 
\left[ \sum_{i=1}^N \mathbf X'_i \mathbf Q' \mathbf Q \mathbf X_i \right]^{-1}.
\label{Eq:21.33}
\end{align}
Начнем с сильного предположения, что $\e_{it}$ независимы и одинаково распределены с параметрами $[0, \sigma^2_{\e}]$, и $\bm\e_i$ независимы и одинаково распределены с параметрами $[\mathbf 0, \sigma^2_{\e} \mathbf I]$. Тогда вектор ошибок $\mathbf Q \mathbf \e_i$ размерности $T \times 1$ независим по $i$ с нулевым ожиданием и дисперсией $\mathrm V[\mathbf Q \bm\e_i]=\mathbf Q \mathrm V[\bm\e_i]\mathbf Q'=\sigma^2_\e \mathbf Q \mathbf Q' = \sigma^2_\e \mathbf Q$. Тогда
\begin{align}
\sum_{i=1}^N \mathbf X'_i \mathbf Q' \mathrm V[\mathbf Q \e_i | \mathbf X_i] \mathbf Q \mathbf X_i=
&\sum_{i=1}^N \mathbf X'_i \mathbf Q' \mathrm \sigma^2_\e \mathbf Q \mathbf Q \mathbf X_i  \nonumber \\
&=\sigma^2_\e \sum_{i=1}^N \mathbf X'_i \mathbf Q'  \mathbf Q \mathbf X_i.
\nonumber
\end{align}
Таким образом \ref{Eq:21.33} можем упростить до оценки, данной в \ref{Eq:21.27}, используя
\begin{align}
(\mathbf Q \mathbf X_i)'(\mathbf Q \mathbf X_i)=\sum^T_{t=1}(\x_{it}-\bar{\x}_i)(\x_{it}-\bar{\x}_i)'.
\nonumber
\end{align}
В настоящее время большинство пакетов вычисляют дисперсию по формуле \ref{Eq:21.27}, однако следует помнить, что альтернативные оценки могут быть лучше. В частности, достаточно просто можно ослабить предположение о некоррелированности ошибок $\e_{it}$ во времени. Если $\bm\e_i$ независимы и одинаково распределены с параметрами $[\mathbf 0, \sum_i]$, то используется более общая формула ковариационной матрицы \ref{Eq:21.33} с $\mathrm{Cov}[\mathbf Q \bm\e_i, \mathbf Q\bm \e_i]= \mathbf 0$ для $i \neq j$, и заменой $\mathrm V[\mathbf Q \mathbf \e_i]$ на $(\mathbf Q  \hat{\bm\e}_i) (\mathbf Q  \hat{\bm\e}_i)'$, где $\hat{\bm\e}_i=\mathbf y_i --- \mathbf X_i \hat{\bm\beta}_W$, что дает нам дисперсию оценки вида \ref{Eq:21.28}.

Оценка $\hat{\bm\beta}_W$ также является состоятельной в модели со случайными эффектами, хотя, как показано в разделе 21.7, она менее эффективна, чем оценка со случайным эффектом в случае, если модель со случайными эффектами более адекватно описывает данные.

{\centering
ОМНК оценивание модели within\\}

Модель within \ref{Eq:21.32} может быть оценена с помощью доступного ОМНК.

Но если в действительности $\e_{it}$  независимы и одинаково распределены с параметрами $[0, \sigma^2_\e]$, то применение ОМНК не дает преимуществ. Продемонстрируем это. Заметим, что $\mathbf Q \bm\e_i$ не зависит от $\mathbf Q \bm\e_j$ при $i \neq j$, $\mathrm V[\mathbf Q \bm\e_i]=\sigma^2_\e \mathbf Q$.     \textbf{ОМНК оценка} будет меть вид
\begin{align}
\hat{\bm\beta}_{W,GLS}=\left[ \sum_{i=1}^N \mathbf X'_i \mathbf Q' \mathbf Q^{-} \mathbf Q \mathbf X_i \right]^{-1} 
\sum_{i=1}^N \mathbf X'_i \mathbf Q' \mathbf Q^{-} \mathbf Q \mathbf y_i,
\nonumber
\end{align}
где используется квазиобратная матрица $\mathbf Q^-$ из-за того, что матрица  $\mathbf Q$ имеет неполный ранг. Однако $\mathbf Q' \mathbf Q^{-} \mathbf Q =\mathbf Q' \mathbf Q$, так как $\mathbf Q' \mathbf Q^{-} \mathbf Q = \mathbf Q$ для квазиобратной матрицы, и $\mathbf Q= \mathbf Q \mathbf Q'$, так как $\mathbf Q$ идемпотентна. Заменяя  $\mathbf Q' \mathbf Q^{-} \mathbf Q $ на $\mathbf Q' \mathbf Q$ в формуле для $\hat{\bm\beta}_{W,GLS}$, получаем МНК оценку \ref{Eq:21.32}.

Преимущества от использования ОМНК могут иметь место, если для $\e_{it}$ предполагаются другие модели. По существу это тот же подход, что и в разделе 21.5.2 для ОМНК сквозной регрессии без фиксированный эффектов, за исключением того, что предварительно фиксированные эффекты должны быть элиминированы. Это приводит к ошибке $\mathbf Q\bm\e_i$, имеющей неполный ранг. Поэтому вначале необходимо выбросить один временной период и применять ОМНК лишь к $(T-1)$ временным периодам. Проще, и зачастую без большой потери эффективности, использовать обычную оценку within c фиксированным эффектом, а затем вычислить робастные для панельных данных стандратные ошибки, используя \ref{Eq:21.28}.

МакКарди (1928b) предлагает анализ типа Бокса-Дженкинса для идентификации и оценки процессов ARMA для $\e_{it}$ в модели с фиксированными эффектами в случае короткой панели. При использовании коротких панелей нет необходимости предполагать, что $\e_{it}$ описывается ARMA процессом, или даже, что $\e_{it}$ стационарна, так как для $N \rightarrow \infty$ всегда можно состоятельно оценить $\mathrm Cov[u_{it}, u_{is}]$ с помощью $N^{-1}\sum_i \hat{u}_{it}\hat{u}_{it}$. Тем не менее, определение ARMA процесса для ошибок может представлять интерес.


\subsection{Оценка в первых разностях}

Модель within получается путем вычитания усредненной по времени модели $\bar{y}_i=\alpha_i+\bar{\x}'_i\bm\beta +\bar{\e}_i$ из первоначальной модели. Альтернативно можно вычесть первый временной лаг модели. Тогда
\begin{align}
&(y_{it}-y_{i,t-1})=(\x_{it}-\x_{i,t-1})'\bm\beta+(\e_{it}-\e_{i,t-1}), &t=2, \dots, T.
\label{Eq:21.34}
\end{align}
Фиксированные эффекты $\alpha_i$ элиминированы. МНК оценивание дает \textbf{оценку в первых разностях}
\begin{align}
\hat{\bm\beta}_{FD}=\left[\sum^N_{i=1}\sum^T_{t=2}(\x_{it}-\x_{i,t-1})(\x_{it}-\x_{i,t-1})'\right]^{-1}
\sum^N_{i=1}\sum^T_{t=2}(\x_{it}-\x_{i,t-1})(y_{it}-y_{i,t-1}).
\label{Eq:21.35}
\end{align}
Заметим, что в регрессии присутствует только $N(T-1)$ наблюдений. Распространенная ошибка в применении данного метода заключается в том, что сначала составляются все $NT$ наблюдений в один вектор, а потом вычитается первый лаг. В таком случае удаляется лишь одно наблюдение $(1, 1)$, в то время, как необходимо выбросить все $T$ наблюдений  первого временного периода $(i, 1), i=1, \dots, N$.

 {\centering
Состоятельность оценки в первых разностях\\}

Для состоятельности оценки в первых разностях необходимо выполнение условия $\mathrm E[\e_{it}-\e_{i,t-1}| \x_{it}-\x_{i, t-1}]$. Это более сильное условие, чем  $\mathrm E[\e_{it}|\x_{it}]=0$, но слабее, чем условие строгой экзогенности, требуемое для состоятельности оценки within.


 {\centering
Асимптотическое распределение оценки в первых разностях\\}

Для  статистических выводов необходима корректировка стандартных ошибок обычного МНК, учитывающая корреляцию во времени ошибок $\e_{it}-\e_{i,t-1}$. С целью получить асимптотическую дисперсию $\hat{\bm\beta}_{FD}$, запишем модель для $i$-го индивидуального наблюдения как
\begin{align}
\Delta \mathbf y_i=\Delta \mathbf X'_i \bm\beta + \Delta \bm\e_i,
\nonumber
\end{align}
где $\Delta \mathbf y_i$ --- вектор размерности $(T-1)\times 1$, элементами которого являются $(y_{i2}-y{i1}), \dots, (y_{iT}-y_{i,T-1})$, и $\Delta \mathbf X_i$ --- вектор размерности $(T-1)\times K$, рядами которого являются $(\x_{i2}-\x_{i1})', \dots, (\x_{iT}-\x_{i,T-1})'$. Тогда оценка
\begin{align}
\hat{\bm\beta}_{FD}=\left[\sum^N_{i=1} (\Delta \mathbf X_i)'(\Delta \mathbf X_i)\right]^{-1}
\sum^N_{i=1} (\Delta \mathbf X_i)'(\Delta \mathbf y_i)
\label{Eq:21.36}
\end{align}
в предположении о независимости по $i$ имеет ковариационную матрицу
\begin{align}
\mathrm V[\hat{\bm\beta}_{FD}]=\left[\sum^N_{i=1} (\Delta \mathbf X_i)'(\Delta \mathbf X_i)\right]^{-1}
\left[\sum^N_{i=1} (\Delta \mathbf X_i)' 
\mathrm V[\Delta \bm\e| \Delta \mathbf X_i]
(\Delta \mathbf X_i)\right]
\left[\sum^N_{i=1} (\Delta \mathbf X_i)'(\Delta \mathbf X_i)\right]^{-1}.
\label{Eq:21.37}
\end{align}

Самое простое предположение состоит в том, что $\e_{it}$ независимы и одинаково распределены с параметрами $[0,\sigma^2_\e]$. Тогда ошибка $\e_{it}-\e_{i,t-1}$ описывается процессом MA(1) с дисперсией $2\sigma^2_\e$ и ковариацией с предыдущим периодом $\sigma^2_\e$ для индивидуального наблюдения $i$. Из этого следует, что $\mathrm V[\Delta \bm\e_i]$  равна $\sigma_\e^2$, умноженной на матрицу размерности $(T-1)\times (T-1)$, состоящую из двоек на диагонали, единиц --- на диагоналях, соседних с главной, и нулей на остальных местах.

Более реалистично предположить, что $\e_{it}$ коррелированы во времени для данного $i$, т.е. $\mathrm{Cov}[\e_{it}, \e_{is}] \neq 0$ для $t \neq s$, но независимы по $i$. Из \ref{Eq:21.13} следует, что для коротких панелей оценкой, которая робастна к общим видам автокорреляции и гетероскедастичности, является \ref{Eq:21.37} с $\mathrm V[\Delta \bm\e_i]$, замененной на $(\widehat{\Delta \bm\e_i})'(\widehat{\Delta \bm\e_i})$. Никогда не следует использовать обычные стандартные ошибки из МНК регрессии модели в первых разностях \ref{Eq:21.37}, так как они являются верными только в том редком случае, когда $(\e_{it}$) описывается процессом случайного блуждания и, как следствие, $(\e_{it}-\e_{i,t-1})$ независимы и одинаково распределены.

Для $T=2$ оценка within и оценка в первых разностях равны, так как $\bar{y}=(y_1+y_2)/2$, что $(y_1-\bar{y})=(y_1-y_2)/2$ и $(y_2-\bar{y})=-(y_1-y_2)/2$ (аналогично для $\x$). В случае $T>2$ эти две оценки различаются. В самом простом предположении, что $\e_{it}$ независимы и одинаково распределены, может быть показано, что ОМНК оценка модели в первых разностях \ref{Eq:21.34} равна оценке within. Вместо этого оценка $\hat{\bm\beta}_{FD}$ получается посредством оценивания  \ref{Eq:21.34} МНК  и является менее эффективной, чем $\hat{\bm\beta}$. По этой причине оценка в первых разностях больше не описывается в большинстве вводных курсов. Однако она широко применяется, когда в модели присутствуют лаги зависимых переменных (см. главу 22). В таком случае оценка within состоятельна. Оценка в первых разностях также состоятельна, но она основывается на более слабых предположениях об экзогенности, что позволяет использовать состоятельную IV оценку.

\subsection{Оценка условного ММП}

Условный ММП заключается в максимизации совместного правдоподобия $y_{11}, \dots, y_{NT}$ при условии индивидуальных средних $\bar{y}_1, \dots, \bar{y}_T$. Этот метод привлекателен тем, что для линейной модели панельных данных в предположении нормальности фиксированные эффекты $\alpha_i$ устраняются, так что максимизация функции правдоподобия осуществляется только по $\bm\beta$.

Предположим, что $y_{it}$ при условии регрессоров $\x_{it}$ и параметров $\bm\alpha_i, \bm\beta$, и  $\sigma^2$ независимы и одинаково нормально распределены $\mathcal N [\bm\alpha_i+\x'_{it}\bm\beta, \sigma^2]$. Тогда \textbf{условная функция правдоподобия} равна
\begin{align}
\mathbf L(\bm\beta, \sigma^2, \bm\alpha)=
& \prod^N_{i=1} f(y_{i1}, \dots, y_{iT}|\bar{y}_i) =  \label{Eq:21.38} \\
& \prod^N_{i=1} \frac{f(y_{i1}, \dots, y_{iT}, \bar{y}_i)} {f(\bar{y}_i)} = \nonumber \\
& \prod^N_{i=1} \frac{(2\pi\sigma^2)^{-T/2}}{(2\pi\sigma^2/T)^{-1/2}}\mathrm{exp}
\left\{ \sum^{T}_{t=1} -[(y_{it}-\x'_{it}\bm\beta)^2+(\bar{y}_i-\bar{\x}'_i\bm\beta)^2]/2\sigma^2 \right\}. \nonumber
\end{align}
Первое равенство определяет условную функцию правдоподобия в предположении о независимости по $i$. Второе равенство выполняется всегда, так как $f(y_1, \dots, y_T|\bar{y})=f(y_i, \dots, y_T, \bar{y})/f(\bar{y})$ и $f(y_1, \dots, y_T, \bar{y})=f(y_1, \dots, y_T)$, а знание о том, что $\bar{y}=T^{-1}\sum_i y_i$, не приносит дополнительной информации при том, что уже известны $y_1, \dots, y_T$ (индекс $i$ здесь опущен для простоты). Третье равенство получается при условии нормальности посредством алгебраических преобразований, которые оставлены для читателя в качестве упражнения.

Ключевой результат заключается в том, что фиксированные эффекты $\bm\alpha$ отсутствуют в последнем выражении в \ref{Eq:21.38}, так $\mathbf L_{COND}(\bm\beta, \sigma^2, \bm\alpha)$ в действительности равно $\mathbf L_{COND}(\bm\beta, \sigma^2)$. Таким образом, необходимо максимизировать условную функцию максимального правдоподобия \ref{Eq:21.38} только по $\bm\beta$ и $\sigma^2$. \textbf{Оценка условного ММП} $\hat{\bm\beta}_{CML}$ получается в результате решения условий первого порядка
\begin{align}
\frac{1}{\sigma^2} \sum^T_{t=1} \sum^N_{i=1} [(y_{it}-\x'_{it}\bm\beta)\x_{it}-(\bar{y}_i-\bar{\x}'_i\bm\beta)\bar{\x}_i]=\mathbf 0,
\nonumber
\end{align}
или, что эквивалентно,
 \begin{align}
 \sum^T_{t=1} \sum^N_{i=1} [(y_{it}-\bar{y}_i)-(\x_{it}-\bar{\x)}'_i\bm\beta)]
(\x_{it}-\bar{\x}_i)=\mathbf 0.
\nonumber
\end{align}
Однако это всего лишь условия первого порядка из МНК регрессии $(y_{it}-\bar{y}_i)$ на $(\x_{it}-\bar{\x}_i)$.

Поэтому оценка условного ММП $\hat{\bm\beta}_{CML}$ равна оценке within $\hat{\bm\beta}_W$.

То, что метод дает состоятельную оценку, интуитивно объясняется тем, что условие $\bar{y}_i$ в \ref{Eq:21.38} уничтожило фиксированные эффекты. Более формально, $\bar{y}_i$ --- достаточная статистика для $\alpha_i$, а условие на достаточную статистику дает состоятельную оценку $\bm\beta$ (см. раздел 23.2.2).


\subsection{МНК оценка с фиктивными переменными}

Рассмотрим стандартную модель с фиксированными эффектами \ref{Eq:21.22} до взятия разностей. МНК анализ можно применить к модели напрямую, одновременно оценивая параметры $\bm\alpha$ и $\bm\beta$.

В принципе не требуется никакого специального программного обеспечения. Просто оценивается МНК регрессия $y_{it}$ на $\x_{it}$ и набор $N$ фиктивных переменных $d_{1,it}, \dots, d_{N,it}$, где $d_{j,it}$ равно единице, если $j=i$, и нулю в противном случае. Однако по мере того, как растет $N$, регрессия обрастает слишком большим числом регрессоров, чтобы обращать матрицу регрессоров размерности $(N+K) \times (N + K)$. Некоторые приемы матричной алгебры снижают проблему обращения матрицы размерности $(K \times K)$.

Конечная оценка $\bm\beta$ будет равна оценке within. Это частный случай теоремы Фриша-Вау. Строится регрессия всех переменных на фиктивные переменные, и если остатки из этой регрессии используются на втором шаге оценивания регрессии, то мы получим те же оценки, что и в полной регрессии. Но эти остатки являются просто отклонениями от своих средних, т.е. это регрессия within. Для полноты описания представим соответствующую матричную алгебру.

Запишем векторы размерности $T \times 1$ в \ref{Eq:21.29} всех $N$ индивидуальных наблюдений для получения \textbf{модели с фиксированными эффектами в виде фиктивных переменных}
\begin{align}
\begin{bmatrix}
 \mathbf y_{1} \\
 \vdots \\
 \mathbf y_{N}
\end{bmatrix}
=
\begin{bmatrix}
\mathbf e & \mathbf 0 & \mathbf 0 \\
\mathbf 0 & \ddots & \mathbf 0\\
\mathbf 0 & \mathbf 0 & \mathbf e
\end{bmatrix}
\begin{bmatrix}
\alpha_i \\ \vdots \\  \alpha_N 
\end{bmatrix}
+
\begin{bmatrix}
 \mathbf X_{1} \\ \vdots \\ \mathbf X_N
\end{bmatrix}
\bm\beta
+
&\begin{bmatrix}
 \bm\e_{1} \\
 \vdots \\
 \bm\e_{N}
\end{bmatrix}
\nonumber
\end{align}
или
\begin{align}
\mathbf y=[(\mathbf I_N \otimes \mathbf e) \mathbf X] 
\begin{bmatrix}
 \bm\alpha \\ \bm\beta
\end{bmatrix}
+\bm\e,
\label{Eq:21.39}
\end{align}
где $\mathbf y$ --- вектор размерности $NT \times 1$, кронекерово произведение $(\mathbf I_N \otimes \mathbf e)$ --- блочная диагональная матрица размерности $NT \times K$, а $ \mathbf X$ --- матрица регрессоров, не являющихся константами, размерности $NT \times K$.

Применение МНК к данной модели дает \textbf{МНК оценку с фиктивными переменными (LSDV)}
 \begin{align}
\begin{bmatrix}
 \hat{\bm\alpha}_{LSDV} \\ \hat{\bm\beta}_{LSDV}
\end{bmatrix}
&=
\begin{bmatrix}
(\mathbf I_N \otimes \mathbf e)'(\mathbf I_N \otimes \mathbf e) & (\mathbf I_N \otimes \mathbf e)'\mathbf X\\
\mathbf X'(\mathbf I_N \otimes \mathbf e) & \mathbf X' \mathbf X
\end{bmatrix}
^{-1} \times
\begin{bmatrix}
(\mathbf I_N \otimes \mathbf e)'\mathbf y\\
\mathbf X'\mathbf y
\end{bmatrix} 
\nonumber \\
&=
\begin{bmatrix}
T \mathbf I_N & T\bar{\mathbf X} \\
T \bar{\mathbf X}' & \mathbf X' \mathbf X
\end{bmatrix}
^{-1} \times
\begin{bmatrix}
 \bar{\mathbf y} \\ \mathbf X' \mathbf y,
\end{bmatrix}
\nonumber
\end{align}
где матрица выборочных средних $\bar{\mathbf X}=[\bar{\x}'_1 \dots \bar{\x}'_N]', \bar{\x}_i=T^{-1} \sum^T_{t=1} \x_{it}, \bar{\mathbf y}=[\bar{y} \dots \bar{y}_N]'$, и $\bar{y}_i=T^{-1} \sum^T_{t=1} y_{it}$. Используя формулу обращения разделенной на блоки матрицы, выполняем последующие алгебраические преобразования, получаем
 \begin{align}
\begin{bmatrix}
 \hat{\bm\alpha}_{LSDV} \\ \hat{\bm\beta}_{LSDV}
\end{bmatrix}
=
\begin{bmatrix}
\bar{\mathbf y}-\bar{\mathbf X} \bm\beta_W\\
[\mathbf X'\mathbf X-\bar{\mathbf X}'\bar{\mathbf X}]^{-1}(\mathbf X' \mathbf y --- \bar{\mathbf X}'\bar{\mathbf y}) 
\end{bmatrix}.
\label{Eq:21.40}
\end{align}
Перезаписав это в обозначениях сумм, имеем $\hat{\bm\beta}_{LSDV}=\hat{\bm\beta}_{W}$, определенное в \ref{Eq:21.24}, и $\hat{\bm\alpha}_{LSDV}=\hat{\bm\alpha}_{FE}$, определенное в \ref{Eq:21.25}, так что LSDV оценка равна оценке within или оценке с фиксированным эффектом. 

Для коротких панелей потенциальная очевидная проблема состоит в том, что получение состоятельной оценки для $\bm\beta$ и $\bm\alpha$ не гарантировано, так как необходимо оценивать $N+K$ параметров, а $N \rightarrow \infty$. Однако состоятельное оценивание $\bm\beta$  возможно при $T \rightarrow \infty$, даже если для $\bm\alpha$ получены несостоятельные оценки.

Эта оценка 'эффективна, если $\e_{it}$ независимы и одинаково распределены с параметрами $[0, \sigma^2]$. Из этого следует, что оценка within для $\bm\beta$ более эффективна, чем альтернативные оценки в разностях, например, такие, как вычитание первого наблюдения или наблюдения предыдущего периода, которые также элиминируют $\bm\alpha_i$. Если дополнительно ошибки нормально распределены, LSDV оценка равна оценке ММП ввиду обычной эквивалентности МНК и ММП в линейной модели со сферичными нормальными ошибками.

\subsection{Оценка ковариации}

Предположим, что данные принадлежат одному из $N$ классов, с $y_{it}$, обозначающим $t$-е наблюдение в $i$-м классе. \textbf{Анализ дисперсии} представляет общую изменчивость $y_{it}$ вокруг общего среднего $\bar{y}$, $\sum_i \sum_t (y_{it}-\bar{y})^2$, в виде декомпозиции на \textbf{внутригрупповую} дисперсию $\sum_i \sum_t (y_{it}-\bar{y}_i+\bar{y})^2$ и \textbf{межгрупповую} дисперсию $\sum_i(\bar{y}_i-\bar{y})^2$, где $\bar{y}_i$ --- среднее $i$-й группы. Принадлежность к группе становится более важной по мере увеличения межгрупповой дисперсии. \textbf{Анализ ковариации} расширяет этот подход и использует регрессоры, в этом случае сумма квадратов остатков подвергается декомпозиции похожим образом. Этот прием часто используется в в прикладной статистике.

В коротких панелях каждое индивидуальное наблюдение рассматривается как класс, наблюдаемый в течение нескольких временных периодов. Модель \ref{Eq:21.3} называется \textbf{моделью анализа ковариации}, так как она позволяет средним остаткам в $i$-м классе меняться в зависимости от класса. Оценка этой модели, оценка within, соответственно называется \textbf{оценкой ковариации}.

\section{Модель со случайными эффектами}

\textbf{Модель со случайными эффектами} \ref{Eq:21.3} может быть переписана в виде
 \begin{align}
&y_{it}=\mu + \x'_{it} \bm\beta +\alpha_i + \e_{it},
&i=1, \dots, N,
& t=1, \dots, T,
\label{Eq:21.41}
\end{align}
или
 \begin{align}
y_{it}=\mathbf w'_{it} \bm\delta+\alpha_i + \e_{it},
\label{Eq:21.42}
\end{align}
где $\mathbf w_{it} = [1  \; \x_{it}]$ и $\bm\delta=[\mu \; \bm\beta']'$. Предполагается, что индивидуальные эффекты $\alpha_i$ --- независимы и одинаково распределены  с параметрами $[0, \sigma^2_{\alpha}]$, а ошибки $\e_{it}$ независимы и одинаково распределены с параметрами $[0, \sigma^2_\e]$. Неслучайный скалярный свободный член $\mu$ добавляется таким образом, что в отличие от \ref{Eq:21.5} случайные эффекты могут быть нормированы и будут иметь нулевое среднее.

С другой стороны модель может рассматриваться как частный случай \textbf{модели со случайным коэффициентом} или \textbf{модели с изменяющимся коэффициентом}, где случаен только свободный член. Модель может быть переписана как $y_{it}=\mu+\x'_{it}\beta+u_{it}$, где ошибка состоит их двух компонент $u_{it}=\alpha_i+\e_{it}$. По этой причине модель со случайным эффектом также называется \textbf{моделью с компонентами ошибок}. Еще более точно эту модель можно было бы назвать \textbf{моделью со случайным свободным членом}. Более общие модели также могут содержать случайные коэффициенты наклона, см. главу 22.

Существует множество состоятельных оценок модели со случайными эффектами, включая (1) ОМНК оценивание в модели \ref{Eq:21.42}; (2) ММП в модели \ref{Eq:21.42} в предположении, что $\alpha_i$ и $\e_{it}$ нормально распределены; (3) МНК оценивание в модели \ref{Eq:21.42}; и (4) оценки с фиксированными эффектами, такие как оценка within или оценка в первых разностях, хотя эти оценки применимы лишь для оценивания коэффициентов, не меняющихся во времени. Первые две оценки асимптотически эквивалентны, но могут различаться в ограниченных выборках в зависимости от выбора оценок для $\sigma^2_{\alpha}$ и $\sigma^2_\e$. Оставшиеся оценки состоятельны, хотя неэффективны в том случае, если в действительности $\alpha_i$ и $\e_{it}$ независимы и одинаково распределены.

\subsection{ОМНК оценка}

\textbf{Оценка со случайным эффектом} $\mu$ и $\bm\beta$ --- это доступная ОМНК оценка модели \ref{Eq:21.42}. Ниже в этом разделе будет показано, что эта оценка может быть получена с помощью МНК оценивания измененного уравнения
 \begin{align}
y_{it}-\hat{\lambda}\bar{y}_i=(1-\hat{\lambda})\mu+(\x_{it}-\hat{\lambda}\bar{\x}_i)'\bm\beta+v_{it},
\label{Eq:21.43}
\end{align}
где $v_{it}=(1-\hat{\lambda})\alpha_i+(\e_{it}-\hat{\lambda}\bar{\e}_i)$ и $\hat{\lambda}$ состоятельна для 
 \begin{align}
\lambda=1-\sigma_\e/(T\sigma^2_{\alpha}+\sigma^2_\e)^{1/2}.
\label{Eq:21.44}
\end{align}
Эквивалентно,
 \begin{align}
\hat{\delta}_{RE}=
\begin{bmatrix}
 \hat{\mu}_{RE} \\ \hat{\bm\beta}_{RE}
\end{bmatrix}
\left[ \sum^N_{i=1} \sum^T_{t=1} (\mathbf w_{it}  -\hat{\lambda} \bar{\mathbf w}_i)
 (\mathbf w_{it}  -\hat{\lambda} \bar{\mathbf w}_i)' \right]^{-1}
\sum^N_{i=1} \sum^T_{t=1} (\mathbf w_{it}  -\hat{\lambda} \bar{\mathbf w}_i)(y_{it}-\hat{\lambda}\bar{y}_i),
\label{Eq:21.45}
\end{align}
где $\mathbf w_{it}=[1 \; \x_{it}]$ и $\bar{\mathbf w}_i=[1\; \bar{\x}_i]$. Для состоятельности необходимо условие $NT \rightarrow \infty$, для этого будет достаточно условия $N \rightarrow \infty$, или условия $T \rightarrow \infty$.

Предполагая, что $\e_{it}$ и $\alpha_i$ независимы и одинаково распределены, результаты оценивания регрессии \ref{Eq:21.43} обычным МНК могут быть использованы для получения оценки ковариационной матрицы,
 \begin{align}
\mathrm V
\begin{bmatrix}
 \hat{\mu}_{RE} \\ \hat{\bm\beta}_{RE}
\end{bmatrix}
=
\sigma^2_\e
\left[ \sum^N_{i=1} \sum^T_{t=1} (\mathbf w_{it} --- \hat{\lambda} \bar{\mathbf w}_i)
 (\mathbf w_{it} -\hat{\lambda} \bar{\mathbf w}_i)' \right]^{-1}.
\label{Eq:21.46}
\end{align}
В случае коротких панелей робастная оценка дисперсии, которая допускает более общую структуру для $\alpha_i+\e_{it}$, может быть получена, используя \ref{Eq:21.13}, из чего следует
 \begin{align}
\mathrm V
\begin{bmatrix}
 \hat{\mu}_{RE} \\ \hat{\bm\beta}_{RE}
\end{bmatrix}
=
\left[ \sum^N_{i=1} \sum^T_{t=1} \tilde{\mathbf w}_{it} \tilde{\mathbf w}_{it}' \right]^{-1}
\sum^N_{i=1} \sum^T_{t=1} \sum^T_{s=1} \tilde{\mathbf w}_{it} \tilde{\mathbf w}_{it}' \hat{\tilde{\e}}_{it} \hat{\tilde{\e}}_{is}
\left[ \sum^N_{i=1} \sum^T_{t=1} \tilde{\mathbf w}_{it} \tilde{\mathbf w}_{it}' \right]^{-1},
\label{Eq:21.47}
\end{align}
где $\tilde{\mathbf w}_{it}=\mathbf w_{it} -\hat{\lambda} \bar{\mathbf w}_{it}$ и $\tilde{\e}_{it}=\hat{\e}_{it}-\hat{\lambda}\bar{\hat{\e}}_i$, где $\hat{\e}_{it}$ --- остаток в модели со случайным эффектом. Эта оценка допускает произвольную автокорреляцию $\e_{it}$ и произвольную гетероскедастичность.

Для уравнения \ref{Eq:21.46} необходимы состоятельные оценки \textbf{компонент дисперсии} $\sigma^2_\e$ и $\sigma^2_\alpha$. Из регрессии within (или регрессии с фиксированными эффектами) $(y_{it}-\bar{y}_i)$ на $(\x_{it}-\bar{\x}_i)$ получаем
 \begin{align}
\hat{\sigma}^2_\e=\frac{1}{N(T-1)-K}
\sum_i \sum_t ((y_{it}-\bar{y}_i)-((\x_{it}-\bar{\x}_i)'\hat{\bm\beta}_W)^2.
\label{Eq:21.48}
\end{align}
Из регрессии between $\bar{y}_i$ на свободный член и $\bar{\x}_i$, уравнение, содержащее ошибку с дисперсией $\sigma^2_\alpha+\sigma^2_\e/T$, получаем
 \begin{align}
\hat{\sigma}^2_\alpha=\frac{1}{N-(K+1)}
\sum_i  (\bar{y}_{i}-\hat{\mu}_B-\bar{\x}'_i\hat{\bm\beta}_B)^2-\frac{1}{T}\hat{\sigma}^2_\e.
\label{Eq:21.49}
\end{align}
Возможны более эффективные оценки компонент дисперсии $\sigma^2_\e$ и $\sigma^2_\alpha$ (см., например, работу Амэмия, 1985), но такие оценки не обязательно будут приводит к увеличению эффективности $\hat{\bm\beta}_{RE}$. Выбор оценок достаточно широк. Оценка дисперсии \ref{Eq:21.43} может быть отрицательной. В таких случаях статистические пакеты зачастую присваивают нулевое значение дисперсии $\alpha$: $\hat{\sigma}^2_{\alpha}=0$. Тогда $\hat{\lambda}=0$, и оценивание производится посредством МНК для сквозной регрессии.

В целях проверки того, что доступный ОМНК упрощается до МНК оценивания \ref{Eq:21.43}, запишем \ref{Eq:21.42} по наблюдениям из всех $T$ временных периодов для данного $i$ таким же образом, как для модели с фиксированным эффектом. Тогда
 \begin{align}
\mathbf y_i=\mathbf W_i \bm\delta + (\mathbf e \alpha_i + \bm\e_i),
\label{Eq:21.50}
\end{align}
где $\mathbf y_i, \mathbf e, \bm\e_i$ и $\mathbf X_i$ определены так же, как и после \ref{Eq:21.29}, $\mathbf W'_i = [\mathbf e \; \mathbf X'_i]$. Для оценивания с помощью ОМНК необходимо получить ковариационную матрицу $\bm\Omega$ вектора ошибок $(\mathbf e \alpha_i +\e_i)$ размерности $T \times 1$. Учитывая, что $\alpha_i$ и $\e_{it}$ независимы, $\mathrm E[( \mathbf e \alpha_i+\bm\e_{i})( \mathbf e \alpha_i+\bm\e_{i})']=\mathrm E[\bm\e_i \bm\e'_i]+\mathrm E[\alpha^2_i]\mathbf e \mathbf e'$.

Так как $\e_{it}$ независимы и одинаково распределены с параметрами $[0, \sigma^2_\e]$ и $\alpha_i$ независимы и одинаково распределены с параметрами $[0,\sigma^2_\alpha]$ получаем
 \begin{align}
\bm\Omega=\sigma^2_\e \mathbf I_T + \sigma^2_{\alpha} \mathbf e \mathbf e'=\sigma^2_\e
\left[\mathbf Q+ \frac{1}{\psi^2}(\mathbf I_T-\mathbf Q) \right],
\nonumber
\end{align}
где $\mathbf Q= \mathbf I_T-T^{-1} \mathbf e \mathbf e'$ было преставлено в \ref{Eq:21.30} и $\psi^2=\sigma^2_\e/[\sigma^2_\e+T \sigma^2_\alpha]$. Используя 
$\mathbf Q \mathbf Q'= \mathbf Q$, можем легко проверить, что $\bm\Omega^{-1}=\sigma^{-2}_\e[\mathbf Q + \psi^2(\mathbf I_T-\mathbf Q)]$ и 
 \begin{align}
\bm\Omega^{-1/2}=\frac{1}{\sigma_\e}[\mathbf Q +\psi (\mathbf Q +\psi (\mathbf I_T- \mathbf Q)].
\label{Eq:21.51}
\end{align}
ОМНК оценка получается путем домножения \ref{Eq:21.50} на скалярный множитель $\bm\Omega^{-1/2}$. Сейчас
 \begin{align}
[\bm\Omega+\psi (\mathbf I_T --- \mathbf Q)] \mathbf y_i
&=\mathbf y_i --- \mathbf e \bar{y}'_i + \psi(\mathbf y_i --- (\mathbf y_i --- \mathbf e \bar{y}'_i)) \nonumber \\
&=\mathbf y_i --- \lambda \mathbf e \bar{y}'_i,
\nonumber
\end{align}
где $\lambda=(1-\psi)$. Применяя похожие алгебраические преобразования для $\mathbf W_i, \mathbf e \alpha_i$, и $\bm\e_i$ в \ref{Eq:21.50}, получаем следующую модель:
 \begin{align}
\mathbf y_i --- \lambda \mathbf e \bar{y}'_i=(\mathbf W_i --- \lambda \bar{\mathbf W})'\bm\delta+(1-\lambda)\alpha_i+(\bm\e_i-\lambda \mathbf e \bar{\e}'_i),
\label{Eq:21.52}
\end{align}
где преобразованная ошибка в \ref{Eq:21.52} имеет ковариационную матрицу $\sigma^2_\e \mathbf I_T$. ОМНК оценка --- это МНК оценка в \ref{Eq:21.52}, но \ref{Eq:21.52} --- это та же версия \ref{Eq:21.43}, записанная в векторной форме, со скалярной величиной $\lambda$, замененной на состоятельную оценку.

Оценка со случайным эффектом $\hat{\bm\beta}_{RE}$ параметра наклона сходится к оценке within при $T \rightarrow \infty$, так как в таком случае $\lambda \rightarrow 1$. В противном случае может быть показано, что $\hat{\bm\beta}_{RE}$ равна \textbf{матрично-взвешенной комбинации} оценки within и оценки between. Если модель со случайными эффектами адекватно описывает данные, это взвешенное среднее предпочитается простой оценке within. Однако если данные описываются  моделью с фиксированными эффектами, то взвешенное среднее не является состоятельной оценкой, так как оценка between в этом случае будет несостоятельна. Можно показать, что оценка свободного члена упрощается до $\hat{\mu}_{RE}=\bar{y}-\bar{\mathbf X}\hat{\bm\beta}_{RE}$. Более подробно см., например, работу Хсяо (2003, с.36) или Грина (2003).


\subsection{Оценка ММП}

В предыдущем разделе в ходе вывода не предполагается нормальность ошибок. Если они действительно являются \textbf{нормальными}, мы можем максимизировать логарифмическую функцию правдоподобия по $\bm\beta, \mu, \sigma^2_\e$, и $\sigma^2_{\alpha}$. Для данных  $\sigma^2_\e$ и $\sigma^2_{\alpha}$ оценка ММП для $\bm\beta$  и $\mu$ совпадает с оценкой ОМНК, но оценки ММП $\tilde{\sigma}^2_\e$ и $\tilde{\sigma}^2_{\alpha}$ отличаются от тех, что даны в \ref{Eq:21.48} и \ref{Eq:21.49}. 

Таким образом, оценки ММП для $\bm\beta$ и $\mu$ даны в \ref{Eq:21.45} с $\hat{\lambda}$, замененной на другую состоятельную оценку $\tilde{\lambda}=1-\tilde{\sigma}_\e/(T\tilde{\sigma}^2_\alpha+\tilde{\sigma}^2_\e)^{1/2}$. Асимптотически, оценки ММП и ОМНК модели со случайными эффектами эквивалентны, однако в ограниченных выборках они будут различаться.

Для оценки ММП возможны два локальных максимума функции правдоподобия при $0 < \psi^2 \leq 1$, а не один. Поэтому необходимо позаботиться о том, чтобы получить глобальный максимум.


\subsection{Другие оценки}

Множество других оценок $\bm\beta$ состоятельны, если модель со случайными эффектами является истинной моделью. В частности, оценки МНК сквозной регрессии, within, в первых разностях, и between будут состоятельны. Однако они не будут эффективными, если $\alpha_i$ и $\e_{it}$ независимы и одинаково распределены, а оценки within и в первых разностях могут оценивать коэффициенты только  регрессоров, меняющихся во времени.

\section{Особенности моделирования}

В этом разделе мы рассмотрим некоторые практические особенности, которые имеют место при оцениваниии линейных панельных данных, даже в отсутствие таких трудностей как эндогенность или лаговые зависимые переменные и других сложностей, которые будут рассмотрены в главе 22.


\subsection{Тесты на объединение}

В модели со случайными эффектами все параметры регрессии одинаковы для разных кросс-секций и временных периодов, в то время как в моделях с фиксированными эффектами параметры постоянны за исключением свободного члена, который может изменяться в зависимости от индивидуальных наблюдений. \textbf{Тесты на  объединение} проверяют адекватность данных ограничений.

Эти тесты обычно реализуются при помощи теста Чоу (см. работу Грина, 2003, c. 130), основанного на тестах на равенство регрессоров в двух линейных регрессиях в предположении об одинаковых дисперсиях. В зависимости от предположений относительно ошибок, тест Чоу может быть применен к моделям, оцененным с помощью МНК или ОМНК. Бальтаджи (2001, глава 4) и Хсяо (2003, глава 2) подробно это описывают.

В случае с короткими панелями невозможно предполагать, что параметры наклона будут различаться по индивидуальным наблюдениям, так как в таком случае количество параметров возрастает слишком сильно. Однако параметры наклона могут меняться во времени. Тестируется модель $y_{it}=\gamma+\x'_{it} \bm\beta+u_{it}$ против модели $y_{it}=\gamma_t+\x'_{it} \bm\beta+u_{it}$. Самым очевидным будет предположить случайные эффекты вида $u_{it}=\e_{it}+\alpha_i$, оценить модель с ограничениями ($\gamma_t=\gamma$ и $\bm\beta_t=\bm\beta$), используя ОМНК оценку со случайным эффектом, и сравнить сумму квадратов остатков моделей с ограничениями и без в преобразованной модели. Если необходимо получить более робастные статистичесике выводы, тогда необходимо вычислить робастные для панельных данных стандартные ошибки и провести тест Вальда. Чаще всего для коротких панелей специфицируется модель с постоянными параметрами наклона $\bm\beta$, хотя свободный член $\gamma_t$ может меняться во времени посредством включения временных фиктивных переменных в качестве дополнительных регрессоров.

\subsection{Тесты на индивидуальные эффекты}

Бройш и Паган (1980) вывели тест множителей Лагранжа, проверяющий гипотезу наличия индивидуальных случайных эффектов против нулевой гипотезы о независимых и одинаково распределенных ошибках. Преимущество состоит в том, что тест легко осуществляется благодаря вспомогательной регрессии, для которой необходимы лишь остатки из МНК оценивания сквозной регрессии. Альтернативно можно предположить нормальность и воспользоваться тестом отношения правдоподобия для  оценки ММП со случайным эффектом против оценки ММП модели с постоянными коэффициентами, или тестом Вальда для гипотезы $\sigma_{\alpha}=0$ в модели со случайными эффектами.

На практике нулевая гипотеза о том, что ошибки в модели с постоянными коэффициентами независимы и одинаково распределены, отвергается. Самый простой способ --- оценить модель с помощью МНК сквозной регрессии с вычислением робастных для панельных данных стандартных ошибок или с помощью ОМНК со случайным эффектом.

Для коротких панелей тест на присутствие индивидуальных фиксированных эффектов невозможен в том числе из-за проблемы с параметрами. Невозможно протестировать, будут ли $N$ параметров равны нулю, когда в наличии только $NT$ наблюдений и $T$ достаточно мало. Вместо этого, тест Хаусмана, описанный в разделе 21.4.3, используется для тестирования нулевой гипотезы со случайными эффектами против альтернативной гипотезы о фиксированных эффектах.

\subsection{Прогнозирование}

Прогнозирование в моделях без индивидуальных эффектов довольно простое: $\hat{y}_{js}=\x'_{js}\hat{\bm\beta}$. Это прогнозирование среднего генеральной совокупности $\mathrm E[y_{js}|\x_{js}]$.

Прогнозирование для данного индивидуального наблюдения при условии индивидуального эффекта менее тривиально. Это прогнозирование $\mathrm E[y_{js}|\x_{js}, \alpha_i]$. Мы рассмотрим прогноз наблюдения, не входящего в выборку, для $i$-го индивидуального наблюдения, используя модель со случайными эффектами \ref{Eq:21.42}. Тогда $y_{i,t+s}=\mathbf w'_{it} \bm\delta + u_{i,t+s}$, где $u_{i,t+s}=\alpha_i+\e_{i,t+s}$. При очевидном прогнозировании $\bm\delta$ заменяется на $\hat{\delta}_{RE}$ и $u_{i,t+s}$ --- на 0 или $\bar{\hat{u}}_i$, где $\bar{\hat{u}}_i=\bar{y}_i-\mathbf w'_i\hat{\bm\delta}_{RE}$ --- выборочное среднее остатков модели within для $i$-го наблюдения. Однако такое прогнозирование не будет эффективным, так как оно игнорирует корреляцию между $u_{i,t+s}$ и ошибками, вызванную индивидуальным случайным эффектом $\alpha_i$. Эта проблема является, скорее, примером более общей проблемы прогнозирования в рамках ОМНК, а не МНК. Лучшее линейное несмещенное прогнозирование для этого специального случая (см. раздел 22.8.3) --- это $\hat{y}_{i,t+s}=\x'_{it}\hat{\delta}_{RE}+(T\sigma^2_\alpha/(T\sigma^2_\alpha+\sigma^2_\e))\bar{\hat{u}}_i$.
Аналогичное очевидное прогнозирование для модели с фиксированными эффектами: $\hat{y}_{i,t+s}=\x'_{it}\hat{\bm\beta}_W+\hat{\alpha}_{i,FE}$. Опять же это прогнозирование не будет состоятельным в случае короткой панели.

\subsection{Модели с двусторонними  эффектами}

До этого момента анализ был сфокусирован на модели только с индивидуальными эффектами \ref{Eq:21.1}, где $u_{it}=\alpha_i+\e_{it}$. Более общая модель --- \textbf{модель с двусторонними (two-way) эффектами}, где $u_{it}=\alpha_i+\gamma_t+\e_{it}$, т.е. которая предполагает также наличие временных эффектов. Тогда
 \begin{align}
& y_{it}=\alpha_i+\gamma_t+\x'_{it}\bm\beta+\e_{it}
& i=1, \dots, N, &
& t=1, \dots, T.
\label{Eq:21.53}
\end{align}
Эта модель была представлена в \ref{Eq:21.2}.

Как уже было отмечено, обычным подходом для коротких панелей было предположение о том, что временные эффекты фиксированы, и оценивание их как коэффициентов фиктивных переменных, включенных в состав регрессоров, наряду с анализом о природе индивидуальных эффектов.

Если $\alpha_i$ и $\gamma_t$ фиксированы, то оценка МНК коэффициентов $\bm\beta$ в \ref{Eq:21.53} эквивалентна регрессии $y_{it}-\bar{y}_i-\bar{y}_t+\overline{\bar{y}}$ на $\x_{it}-\bar{\x}_i-\bar{\x}_t+\overline{\bar{\x}}$, где $\bar{y}_i=T^{-1}\sum^T_{t=1} y_{it}, \bar{y}_t=N^{-1}\sum^N_{i=1}y_{it}$, и $\overline{\bar{y}}=(NT)^{-1}\sum^N_{i=1} \sum^T_{i=1}y_{it}$. $\bar{\x}_i, \bar{\x}_t,$ и $\overline{\bar{\x}}$ определены аналогично. Этот метод дает состоятельные оценки, если $T$ достаточно большое.

Если вместо этого $\alpha_i$ и $\gamma_t$  случайны, то ошибки будут содержать компоненту $\gamma_t$, которая вызывает корреляцию ошибок по индивидуальным наблюдениям, в то время как мы предполагали независимость по $i$. Можно показать, что ОМНК оценка может быть посчитана посредством МНК оценивания $y^{\star}_{it}$ на константу и $\x_{it}^{\star}$,
 \begin{align}
 y_{it}^{\star}=y_{it}-\lambda_1\bar{y}_i-\lambda_2\bar{y}_t+\lambda_3\overline{\bar{y}},
\nonumber
\end{align}
где $\bar{y}_i$, $\bar{y}_t$ и $\overline{\bar{y}}$ уже были определены, а $\x^{\star}_{it}$ определяется аналогично $y^{\star}_{it}$. Этот и другие результаты для модели с двусторонними эффектами представлены у Хсяо (2003) или Бальтаджи (2001). 

\subsection{Несбалансированные панельные данные}

До этого момента предполагалось, что панель сбалансирована, что означает доступность данных для каждого индивидуального наблюдения в каждый из временных периодов. Для региональных панельных данных это предположение часто выполняется. Для панельных исследований индивидуумов, напротив, зачастую характерно сокращение или \textbf{истощение} со временем пропорции индивидуумов, которые продолжают отвечать на вопросы исследования. Более того,  в некоторых случаях некоторые индивидуумы могут пропустить один или более периодов, а затем снова вернуться в выборку, согласно  дизайну \textbf{чередующихся панелей}, таких как CPS, где домохозяйства опрашиваются в течение четырех месяцев, затем восемь месяцев не опрашиваются, а потом снова наблюдаются четыре месяца. Такие панели, где разные индивидуумы появляются в разные годы, называются \textbf{несбалансированными или неполными панелями}.

Пусть $d_{it}$ будет индикаторной переменной, равной 1, если $i$-е наблюдение наблюдается, и равной нулю в противном случае. Тогда для модели с индивидуальными эффектами \ref{Eq:21.3} оценка с фиксированным эффектом будет состоятельной, если  предположение \ref{Eq:21.4} о строгой эндогенности будет выглядеть следующим образом:
 \begin{align}
\mathrm E[u_{it}|\alpha_i, \x_{i1}, \dots, \x_{iT}, d_{i1}, \dots, d_{iT}]=0,
\label{Eq:21.54}
\end{align}
и оценка со случайным эффектом состоятельна, если дополнительно $\alpha_i$ не зависит от остальных переменных условия. Оценки с фиксированным и случайным эффектом могут применяться к несбалансированным данным с относительно небольшой корректировкой. Это должно быть понятно из изначального представления оценок как МНК оценок в различных моделях, представленных в разделе 22.2.2. Например, замена $\hat{\lambda}$ в \ref{Eq:21.10} для модели со случайными эффектами на $\hat{\lambda}_i=1-\sigma_\e/(T_i\sigma^2_\alpha+\sigma^2_\e)^{1/2}$, где $T_i$ --- количество наблюдений для индивидуального наблюдения $i$ (см. работы Бальтаджи, 1985, Вансбика и Каптейна, 1989). Дэвис (200) рассматривает модель со случайными многосторонними (multi-way) эффектами. Индивидуальное наблюдение для модели с фиксированными эффектами должно наблюдаться по крайней мере дважды в выборке, а число степеней свободы должно быть скорректировано. Бальтаджи (2001) довольно широко описывает, как нужно обращаться с несбалансированными панелями. Эконометрические пакеты, которые оценивают основные  модели панельных данных, описанные в главах 21-23, обычно автоматически справляются с пропущенными наблюдениями.

Иногда удобно конвертировать несбалансированную панель в сбалансированную, включая в выборку только тех индивидуумов, данные по которым известны для всех периодов. Очевидно, это сильно снизит эффективность в связи с потерей большого количества наблюдений. Более того, если данные пропущены неслучайно, это может усилить потенциальную проблему нерепрезентативности выборки.

Одной из причин отсутствия данных может быть ненаблюдаемость по меньшей мере одной переменной, в то время как все остальные переменные наблюдаемы. Например, \textbf{уровень отсутствия ответов} на вопросы о доходах может быть вполне высоким. Чем выбрасывать целое наблюдение в связи с отсутствием данных об одном регрессоре, доходе, может оказаться выгодным использование методов восстановления данных, описанных в главе 27.

Необходимо использовать особые методы для анализа несбалансированных панелей, если причина выпадания индивидуумов из выборки коррелирована с ошибкой, так что \ref{Eq:21.52} не выполняется. Например, индивидуумы с необычно низкими заработными платами (после учета наблюдаемых характеристик) могут выпадать из выборки с большей вероятностью. В результате нерепрезентативная панель приведет к \textbf{смещению в связи с истощением} (attrition bias), если заработная плата выступает в качестве зависимой переменной. Для получения состоятельной оценки нужно использовать методы формирования выборки, расширенные для панельных данных (см. раздел 23.5.2).

\subsection{Ошибки измерения}

Ошибки измерения в регрессорах приводят к несостоятельным оценкам параметров в моделях с данными пространственного типа. Если используются методы панельных данных, включающие в себя взятие разностей, проблема несостоятельности оценок может в результате только усугубиться.  Несостоятельность будет  вызвана ошибкой измерения и будет зависеть от предположений, сделанных относительно процесса, порождающего данные.

\subsection{Практические соображения}

Различные оценки, представленные в данной главе просты в применении. Самый проверенный метод --- использование команд для панельных данных, доступных в таких эконометрических пакетах, как LIMDEP, STATA и TSP, каждый из которых можно использовать для  несбалансированных панелей. Большинство оценок могут быть получены с использованием адекватной сквозной МНК регрессии на преобразованных данных, для которой необходим лишь стандартный алгоритм для анализа данных пространственного типа. Однако стандартные ошибки при этом могут отличаться от тех, что получаются при использовании алгоритма для панельных данных, так как пространственный алгоритм может игнорировать автокорреляцию, вызванную преобразованием данных, и алгоритмы могут использовать разное количество степеней свободы.

Недостаток команд статистических пакетов состоит в том, что в настоящий момент они вычисляют стандартные ошибки, которые основываются на  предположениях о распределениях: в моделях с фиксированными эффектами ошибки независимы и одинаково распределены,  индивидуальные эффекты и ошибки независимы и одинаково распределены в модели со случайными эффектами. Для расчета более робастных оценок стандартных ошибок, описанных в данной главе, требуется оценивание панельных данных с панельным бутстрэпом или МНК оценивание сквозной регрессии с опцией расчета кластерных робастных стандартных ошибок.

В микроэконометрическом анализе модели с и без фиксированных эффектов существенно различаются. Если выбрана модель без фиксированных эффектов, то это должно быть подтверждено тестом Хаусмана. Если этот тест отвергает модель со случайными эффектами, тогда возможно состоятельно оценить коэффициенты  регрессоров, не меняющихся во времени,  используя инструментальные переменные, представленные в следующей главе.

\subsection{Библиографические заметки}

Большинство учебников, таких как учебник Грина (2003), содержат по меньшей мере главу, посвященную анализу панельных данных. В книге Вулдриджа (2002) содержится несколько глав, которые посвящены линейным и нелинейным моделям панельных данных. Эконометричеcкие монографии о панельных данных  --- Хсяо (1986, 2003), Бальтаджи (1995, 2001), Матиас и Севестр (1995), М.-Дж. Ли (2002) и Ареллано (2003). Последние три книги уделяют большое внимание методам, представленным в главах 22 и 23. Диггл, Лианг и Цегер (1994, 2002) представляет собой стандартный статистический справочник.

\textbf{21.4} Мундлак (1978) написал классическую статью о выборе между моделями с фиксированными и случайными эффектами. Хаусман (1978) для иллюстрации своего подхода использовал тесты для выбора между этими двумя моделями.

\textbf{21.6} Кух (1959) и Хох (1962) были авторами одного из самых первых исследований на панельных данных, в котором оценивали инвестиционные функции и производственные функции Кобба-Дугласа. Эти работы сравнивают использование оценок within с использованием изменчивости во времени и оценками between с использованием изменчивости внутри кросс-секций.


 {\centering
{\bf Упражнения}\\}

\textbf{21-1} (Бальтаджи, 1999) Рассмотрим модель анализа панельных данных $y_{it}=\alpha + \beta \x_{it}+u_{it}$, где $\alpha$ и $\beta$ --- скалярные величины.
\begin{itemize}
\item[{\bf (a)}] Покажите вычитанием, что из этой модели следует
 \begin{align}
y_{it}-\bar{y}=\beta(\x_{it}-\bar{\x}_i) + \beta(\bar{\x}_{i}-\bar{\x}) + (u_{it}-\bar{u}),
\nonumber
\end{align}
где $\bar{y}=(NT)^{-1}\sum_{i,t} y_{it}$, $\bar{x}=(NT)^{-1}\sum_{i,t} x_{it}$ и $\bar{\x}_i=T^{-1}\sum_t\x_{it}$.

\item[{\bf (b)}] Для соответствующей неограниченной  МНК регрессии
\begin{align}
y_{it}-\bar{y}=\beta(\x_{it}-\bar{\x}_i) + \beta(\bar{\x}_{i}-\bar{\x}) + (u_{it}-\bar{u})
\nonumber
\end{align}
покажите, что МНК оценка $\beta_1$ --- это оценка within, а $\beta_2$ --- оценка between.

\item[{\bf (с)}] Покажите, что если $u_{it}=\mu_i+v_{it}$, где $\mu_i$ независимы и одинаково распределены с параметрами $[0,\sigma^2_\mu]$ и $v_{it}$ независимы и одинаково распределены с параметрами $[0,\sigma^2_v]$, и они независимы по  $i$ и $t$, то МНК и ОМНК оценки эквивалентны.
\end{itemize}

\textbf{21-2} Рассмотрите оценивание фиксированных эффектов линейной регрессионной модели $y_{it}=\alpha_i+\x'_{it}\bm\beta+\e_{it}$, где $\alpha_i$ --- фиксированные эффекты, возможно коррелированные с $\x_{it}$. Объединение всех  $T$ наблюдений в один вектор для индивидуума $i$ дает $\mathbf y_i=\alpha_i \mathbf e + \mathbf X_i \beta + \e_i$ (см. \ref{Eq:21.29}). Рассмотрим оценку $\hat{\beta}=\sum^N_{i=1}[ \mathbf X'_i \mathbf J' \mathbf J \mathbf X_i]^{-1} \times \sum^N_{i=1}\mathbf X'_i \mathbf J' \mathbf J \mathbf y_i$, где $\mathbf J$ --- это матрица размерности $T \times T$ известных констант. Эти константы таковы, что $\mathbf J \mathbf e =\mathbf 0$. [Заметим, что примером $\mathbf J$ может служить $\mathbf Q = \mathbf I_T --- T^{-1} \mathbf e \mathbf e'$.]

\begin{itemize}
\item[{\bf (a)}] Объясните, почему необходима такая оценка $\hat{\beta}$.

\item[{\bf (b)}] Найдите $\mathrm E[\hat{\beta}]$. Для простоты используйте предположение, что $\mathbf X_i$ --- фиксированные регрессоры, и что $\e_{it} \thicksim [0, \sigma^2]$. Является ли $\hat{\beta}$ несмещенной оценкой для $\beta$?

\item[{\bf (с)}] Найдите $V[\hat{\beta}]$. Для простоты используйте предположение, что $\mathbf X_i$ --- фиксированные регрессоры, и что $\e_{it} \thicksim [0, \sigma^2]$.

\item[{\bf (d)}] Теперь предположите, что $\e_{it}$ независимы по $i$, но коррелированы по $t$ с $V[\e_i]=\Omega_i$. Найдите $V[\hat{\beta}]$.

\item[{\bf (e)}] Предположим, что $\alpha_i$ --- случайные эффекты с $(0, \sigma^2_\alpha)$, а не фиксированные. Будет ли в этом случае оценка состоятельна?

\end{itemize}

\textbf{21-3} (Бальтаджи, 1998) Рассмотрим фиксированные эффекты, модель панельных данных с временными и индивидуальными эффектами
 \begin{align}
y_{it}=\alpha + \x'_{it} \bm\beta +\bm\mu_i+\lambda _t+\e_{it},
\nonumber
\end{align}
где $\alpha$  это скалярная величина, $\x_{it}$ --- вектор экзогенных переменных размерности $k \times 1$, $\bm\beta$ --- вектор размерности $K \times 1$, $\mu$ и $\lambda$ обозначают фиксированные индивидуальные и временные эффекты соответственно, и $\e_{it}$ независимы и одинаково распределены с параметрами $[0, \sigma^2]$.

\begin{itemize}
\item[{\bf (a)}] Покажите, что оценка within параметра $\bm\beta$, которая является лучшей линейной несмещенной оценкой, может быть получена посредством двух преобразований within (one-way). Первое преобразование --- преобразование within, игнорирующее фиксированные эффекты. За ним следует преобразование within, игнорирующее индивидуальные эффекты.

\item[{\bf (b)}]  Покажите, что порядок этих двух преобразований неважен. Дайте интуитивное объяснение этому результату.

\end{itemize}

\textbf{21-4} Используйте 50\%-ю случайную подвыборку данных о заработной плате и количестве часов работы в разделе 21.3.
\begin{itemize}
\item[{\bf (a)}] Может ли $\beta$ интерпретироваться как эластичность предложения труда? Объясните.

\item[{\bf (b)}] Для следующих оценок: (1) МНК сквозной регрессии, (2)  between, (3) within, (4)  в первых разностях, (5) ОМНК со случайным эффектом, (6) оценка ММП со случайным эффектом запишите (i) $\hat{\beta}$ (оцененный коэффициент lnwg), (ii) стандартные ошибки, вычисляемые по умолчанию и (iii) робастные для панельных данных стандартные ошибки, полученные методом бутстрэп, используя 200 репликаций.

\item[{\bf (c)}] Схожи ли оценки $\beta$?

\item[{\bf (d)}] Есть ли систематическая разница между стандартными ошибками, вычисляемыми по умолчанию, и робастными для панельных данных стандартными ошибками.

\item[{\bf (e)}] Будет ли МНК оценка сквозной регрессии в части (b)  состоятельной оценкой $\beta$ в модели с фиксированными эффектами? Будет ли МНК оценка сквозной регрессии состоятельна для $\beta$ в модели со случайными эффектами?

\item[{\bf (f)}] Выполните тест Хаусмана и сделайте выбор между ОМНК оценкой со случайным эффектом и оценкой с фиксированным эффектом параметра $\beta$ в этой модели. Проведите тест вручную, используя результаты предыдущей регрессии со стандартными ошибками, вычисленными по умолчанию. Какой вывод Вы сделаете и какую модель предпочтете?

\item[{\bf (g)}] Верите ли Вы, что кривая предложения труда имеет положительный наклон? Объясните.

\end{itemize}


\chapter{Линейные модели анализа панельных данных: дополнения}

\subsection{Введение}
В предыдущей главе рассмотрены варианты линейных моделей панельных данных с фиксированными или случайными эффектами и строго экзогенными регрессорами. Сейчас мы перейдем к рассмотрению различных видов линейных моделей, делая акцент на ослабление предпосылок экзогенности, с целью состоятельного оценивания моделей с эндогенными переменными и/или лаговыми зависимыми переменными.

Стандартный метод, применяемый при эндогенных регрессорах, --- использование инструментальных переменных. Получить инструментальные переменные для панельных данных гораздо проще, чем в случае с данными пространственного типа, так как в качестве инструментов для текущего периода могут быть использованы экзогенные регрессоры других периодов. Единственная трудность заключается в том, чтобы сначала учесть фиксированные или случайные эффекты.

При использовании панельных данных в состав регрессоров могут быт дополнительно включены лаговые зависимые переменные, что невозможно в случае наличия одного периода. Благодаря этому мы можем оценивать динамические модели, и можем различать причины постоянства доходов, например, отличить изменчивость вокруг ненаблюдаемых индивидуальных эффектов (как в главе 21), от влияния предыдущих значений на значения текущего периода. Оценки главы 21, которые учитывают индивидуальные  эффекты становятся несостоятельными, когда одним из регрессоров становится лаговая переменная. Метод инструментальных переменных с использование более длинных лагов в качестве инструментальных переменных дает состоятельные оценки.

Благодаря обилию инструментов панельные данные обеспечивают излишек доступных для оценивания моментных условия. Ошибки моделей панельных данных обычно не являются независимыми и одинаково распределенными. Естественный подход к оцениванию --- ОММ, описанный подробно в разделе 22.2. Подход проиллюстрирован на примере оценивания эластичности предложения труда в разделе 22.3. Дальнейшие подробности оценивания с индивидуальными эффектами и эндогенными регрессорами или лаговыми зависимыми переменными представлены в разделе 22.4 и 22.5. Обсуждение достаточно объемно, так как в нем затронуто множество различных усложнений моделей. Это включение индивидуальных фиксированных или случайных эффектов, различные предположения об экзогенности, идентифицируемые и сверхидентифицируемые модели.

Остаток главы посвящен другим отдельным темам, которые в принципе не требуют знания разделов 22.2---22.5. Модели,  близко связанные с моделями анализа панельных данных, рассмотрены в разделах 22.6---22.8, а именно модели повторяющихся пространственных данных, <<разность разностей>> и иерархические модели.

\section{ОММ оценивание линейных моделей панельных данных}

Регрессионные модели панельных данных в главе 21 накладывали ограничение на скалярную зависимую переменную $y_{it}$. Предполагалось, что она зависит только от одновременных значений регрессоров $\x_{it}$, хотя вполне возможно, что все значения $\x_{i1}, \dots, \x_{iT}$ влияют на $y_{it}$ в предположении главы 21 о строгой экзогенности. Таким образом, нам предоставлена возможность более эффективно оценить параметры модели благодаря использованию невключенных регрессоров других периодов в качестве инструментов для значений регрессоров текущего периода.

Более того, регрессоры других периодов могут быть валидными инструментами для эндогенных регрессоров текущего периода, либо для лагов зависимых переменных. Так инструменты обеспечивают состоятельные оценки в ситуациях, когда из-за нарушения предположения о строгой экзогенности оценки главы 21 оказываются несостоятельными.

В этом разделе рассматривается оценивание ОММ для панельных данных. Это очень полезная база для \textbf{оценивания панельных данных с помощью инструментальных переменных}, к которому мы будем прибегать на протяжении разделов 22.2---22.5. Затем описывается использование экзогенных переменных (регрессоров или инструментов)  других периодов в качестве инструментов. Когда эти основные аспекты будут рассмотрены, останется только добавить в модели фиксированные или случайные эффекты. Внедрение в модели фиксированных и случайных эффектов отложено на последующие разделы. 

\subsection{ОММ для панельных данных}

Рассмотрим \textbf{линейную модель панельных данных}
\begin{align}
y_{it}=\x'_{it}\bm\beta+u_{it},
\label{Eq:22.1}
\end{align}
где $\x_{it}$ может содержать как регрессоры, меняющиеся во времени, так и регрессоры, не меняющиеся во времени, а также могут включать свободный член. Здесь пока нет \textbf{индивидуальных эффектов} $\alpha_i$ (предположение перестает действовать в разделе 22.3), $\x_{it}$  включает переменные только текущего периода (предположение перестает действовать в разделе 22.5). Предполагается, что наблюдения независимы по $i$, используется \textbf{короткая панель}, т.е. $T$ фиксировано, и $N \rightarrow \infty$.

Начнем с записи $T$ наблюдений для $i$-го индивидуума в столбик
\begin{align}
\mathbf y_{i}=\mathbf X'_{i}\bm\beta+\mathbf u_{i},
\label{Eq:22.2}
\end{align}
где $\mathbf y_i$ и $\mathbf u_i$ --- векторы размерности $T \times 1$ и $\mathbf X_i$ --- матрица размерности $T \times K$ c $t$-ой строкой $\x'_{it}$, т.е.
\begin{align}
\mathbf y_{i}=
\begin{bmatrix}
y_{i1} \\
\vdots \\
y_{iT}
\end{bmatrix};
\mathbf X'_{i}=
\begin{bmatrix}
\x'_{i1} \\
\vdots \\
\x'_{iT}
\end{bmatrix};
\mathbf u_i=
\begin{bmatrix}
u_{i1} \\
\vdots \\
u_{iT}
\end{bmatrix}.
\nonumber
\end{align}
Модель \ref{Eq:22.2} определяет линейную систему уравнений. Можно напрямую применять результаты раздела 6.9.5 для оценивания методом инструментальных переменных на данных, независимых по $i$.

Предположим, что существует \textbf{матрица инструментов $Z_i$} размерности $T \times r$, где $r \geq K$ --- количество \textbf{инструментов}, которые удовлетворяют $r$ моментным условиям
\begin{align}
\E [\mathbf Z'_i \mathbf u'_i]=\mathbf 0.
\label{Eq:22.3}
\end{align}
ОMM оценка, основанная на этих моментных условиях, минимизирует квадратичную форму
\begin{align}
Q_N(\be)=\left[ \sum^N_{i=1} \mathbf Z'_i \mathbf u_i \right]'
\mathbf W_N
\left[ \sum^N_{i=1} \mathbf Z'_i \mathbf u_i \right],
\nonumber
\end{align}
где $\mathbf W_N$ обозначает взвешивающую матрицу $r \times r$. Используя $\mathbf u_i=\mathbf y_i - \mathbf X_i\be$, с помощью  алгебраических преобразований можно получить \textbf{ОMM оценку для панельных данных} (Panel Generalized Method of Moments, PGMM) 
\begin{align}
\hat{\be}_{PGMM}=
\left[ \left( \sum^N_{i=1} \mathbf X'_i \mathbf Z_i \right) \mathbf W_N
\left( \sum^N_{i=1} \mathbf Z'_i \mathbf X_i \right) \right]^{-1}
\left[ \left( \sum^N_{i=1} \mathbf X'_i \mathbf Z_i \right) \mathbf W_N
\left( \sum^N_{i=1} \mathbf Z'_i \mathbf y_i \right) \right].
\nonumber
\end{align}
Существенное условие состоятельности этой оценки --- предположение \ref{Eq:22.3}.

Во многих приложениях $\mathbf Z_i$ состоит из текущих и лаговых значений экзогенных регрессоров. Например, предположим, что все регрессоры одновременно экзогенны. Тогда $\E[\x_{it} u_{it}]= \mathbf 0$ означает \ref{Eq:22.3} с $\mathbf Z_i=[\x'_{it} \dots \x'_{iT}]$. В этом случае модель \textbf{идентифицируема} и, так как $\mathbf Z_i=\mathbf X_i$, $\hat{\be}_{PGMM}$ упрощается до МНК оценки сквозной регрессии главы 21. Если дополнительно предполагается, что $\E[\x_{it-1}u_{it}]=\mathbf 0$, тогда $\x_{it-1}$ доступна как дополнительный инструмент для $i$-го наблюдения, модель \textbf{сверх-идентифицируема}, и более эффективное оценивание возможно при использовании ОММ оценки для панельных данных (PGMM).

Использование \textbf{различных предположений об экзогенности} для формирования матрицы инструментов $\mathbf Z_i$  подробно рассмотрено в разделе 22.2.4. Анализ требует внедрения в модели панельных данных индивидуальных эффектов $\alpha_i$. Это проиллюстрировано в эмпирическом примере в разделе 22.3 и детально обсуждается в разделах 22.4 и 22.5.

\subsection{Робастные статистические выводы для панельных данных }

Чтобы выразить формулу распределения ОMM оценки для панельных данных, удобно использовать более компактную запись. Перепишем
\begin{align}
\hat{\be}_{PGMM}=[\mathbf X' \mathbf Z \mathbf W_N \mathbf Z' \mathbf X]^{-1} \mathbf X' \mathbf Z \mathbf W_N \mathbf Z' \mathbf y,
\label{Eq:22.4}
\end{align}
где  $\mathbf X'=[\mathbf X'_1 \dots \mathbf X'_N], \mathbf Z'=[\mathbf Z'_1 \dots \mathbf Z'_N]$, $\mathbf y'=[\mathbf y'_1 \dots \mathbf y'_N]$. Тогда $\hat{\be}_{PGMM}$ \textbf{асимптотически нормальна}  с оцененной асимптотической ковариационной матрицей
\begin{align}
\hat{\V} [\hat{\be}_{PGMM}]=[\mathbf X' \mathbf Z \mathbf W_N \mathbf Z' \mathbf X]^{-1} \mathbf X' \mathbf Z \mathbf W_N \mathbf Z' 
(N \hat{\mathbf S}) \mathbf W'_N \mathbf Z' \mathbf X
[\mathbf X' \mathbf Z \mathbf W_N \mathbf Z' \mathbf X]^{-1},
\label{Eq:22.5}
\end{align}
см. уравнение \ref{Eq:6.97}, где $\hat{\mathbf S}$ --- состоятельная оценка матрицы размера $r \times r$ 
\begin{align}
\mathbf S =\mathrm{plim} \frac{1}{N} \sum^N_{i=1} \mathbf Z'_i \mathbf u_i \mathbf u'_i \mathbf Z_i,
\label{Eq:22.6}
\end{align}
 также предполагается независимость по $i$. Существенным предположением для этого результата является $N^{-1/2} \mathbf Z' \mathbf u = N^{-1/2} \sum_i \mathbf Z'_i \mathbf u_i \overset{d}\rightarrow \mathcal N[\mathbf 0, \mathbf S]$. Робастная ошибка $\mathbf S$ в форме Уайта  
\begin{align}
\hat{\mathbf S} = \frac{1}{N} \sum^N_{i=1} \mathbf Z'_i \hat{\mathbf u}_i \hat{\mathbf u}'_i \mathbf Z_i,
\label{Eq:22.7}
\end{align}
где  $\hat{\mathbf u}_i=\mathbf y_i- \mathbf X_i \hat{\be}$ --- вектор оцененных остатков размерности $T \times 1$. 

Оценка \ref{Eq:22.5} дает \textbf{робастные} к гетероскедастичности и корреляции во времени \textbf{стандартные ошибки для панельных данных}. В качестве альтернативы можно использовать \textbf{панельный бутстреп}. Подробнее см. обсуждение в разделе 21.2.3, где  применяется данный метод.

\subsection{Одношаговый и двухшаговый ОMM для панельных данных}

Различные взвешивающие матрицы полного ранга $\mathbf W_N$ в \ref{Eq:22.4} приводят к различным системам ОММ оценок, за исключением случая идентифицируемости, когда $r=K$, а оценка ОММ для панельных данных упрощается до оценки инструментальных переменных $[\mathbf Z' \mathbf X]^{-1}\mathbf Z'\mathbf y$ для любого $\mathbf W_N$. Обсуждение этого результат приводится в разделе 6.4.2. Два самых используемых вида матрицы $\mathbf W_N$ даны в данном разделе.

{\centering Одношаговый ОMM}

В \textbf{одношаговом ОMM} или \textbf{двухшаговой МНК оценке} используется взвешивающая матрица $\mathbf W_N=[\sum_i \mathbf Z'_i \mathbf Z_i]^{-1}=[\mathbf Z' \mathbf Z]^{-1}$.  Тогда оценка имеет вид
\begin{align}
\hat{\be}_{2SLS} = [\mathbf X' \mathbf Z ( \mathbf Z' \mathbf Z)^{-1} \mathbf Z' \mathbf X]^{-1} \mathbf X' \mathbf Z (\mathbf Z' \mathbf Z)^{-1} \mathbf Z' \mathbf y.
\label{Eq:22.8}
\end{align}
На основании \ref{Eq:22.3} может быть показано, что это оптимальная оценка ОММ для панельных данных, если $\mathbf u_i | \mathbf Z_i$ независимы и одинаково распределены  с параметрами $[\mathbf 0, \sigma^2 \mathbf I_T]$.

Эта оценка называется одношаговой ОMM, потому что она может быть напрямую вычислена по формуле \ref{Eq:22.8}. Она также называется двухшаговой МНК оценкой (2SLS), так как может быть получена в два шага: (1) МНК $\mathbf X_i$ на $\mathbf Z_i$, откуда получаем предсказанные значения $\hat{\mathbf X}_i$, и (2) МНК $\mathbf y_i$ на $\hat{\mathbf X}_i$. В качестве оценки ковариационной матрицы $\hat{\be}_{\mathrm{2SLS}}$, которая является робастной в случае гетероскедастичности и панельных данных, используется \ref{Eq:22.5} с $\mathbf W_N=[\mathbf Z' \mathbf Z]^{-1}$.

{\centering Двухшаговый ОМM}

Самая эффективная ОMM оценка, основанная на безусловном моментном условии \ref{Eq:22.3}, использует взвешивающую матрицу $\mathbf W_N=\hat{\mathbf S}^{-1}$, где $\hat{\mathbf S}$  является состоятельной оценкой для $\mathbf S$, определенной в \ref{Eq:22.6}; для более общего результата см. раздел 6.4.2. Используя $\hat{\mathbf S}$ в \ref{Eq:22.7}, получаем \textbf{двухшаговую ОMM оценку}
\begin{align}
\hat{\be}_{2SGMM} = [\mathbf X' \mathbf Z \hat{\mathbf S}^{-1} \mathbf Z' \mathbf X]^{-1} \mathbf X' \mathbf Z \hat{\mathbf S}^{-1} \mathbf Z' \mathbf y.
\label{Eq:22.8}
\end{align}
Тогда выражение \ref{Eq:22.5} значительно упрощается и $\hat{\mathrm V}[\hat{\be}_{2SGMm}]=[\mathbf X' \mathbf Z (N \hat{\mathbf S})^{-1} \mathbf Z' \mathbf X]^{-1}$.

Оценка называется двухшаговой ОMM оценкой, так как состоятельная оценка $\be$ на первом шаге, например $\hat{\be}_{\mathrm 2SLS}$, нужна для вычисления остатков $\hat{\mathbf u}_i$, которые в свою очередь используются для вычисления $\hat{\mathbf S}$.

{\centering Увеличение эффективности}

В этой главе мы сосредоточим внимание на ситуациях, в которых $\mathbf Z$ не содержит $\mathbf X$ вследствие того, что некоторые компоненты $ \mathbf X$ эндогенны. ОMM для панельных данных дает состоятельные оценки в отличие от МНК. Двухшаговый ОMM дает наиболее эффективную оценку на основании моментного условия $\E[ \mathbf Z'_i \mathbf u_i]=\mathbf 0$.

Даже если регрессоры строго экзогенны, двухшаговый ОMM будет \textbf{более эффективным}, чем МНК сквозной регрессии. Чтобы это продемонстрировать, предположим, что $\mathbf X$ строго экзогенны. Когда $\mathbf Z = \mathbf X$, двухшаговая ОMM оценка упрощается до  $[\mathbf X' \mathbf X]^{-1}\mathbf X' \mathbf y$, и использование ОMM для панельных данных не дает никакого преимущества. Однако если вместо этого в $\mathbf Z$  входит $\mathbf X$ и некоторые дополнительные переменные, такие как степени регрессоров или значения регрессоров в других периодах, тогда двухшаговый ОMM по меньшей мере такой же эффективный, как МНК при выполнении равенства, если ошибки $u_{it}$ независимы и одинаково распределены.

Даже более эффективные, чем $\hat{\be}_{\mathrm{2SGMM}}$, оценки возможны благодаря расширению определения $\mathbf Z_i$ и использованию оптимального моментного условия $\E[ \mathbf u_i| \mathbf Z_i]= \mathbf 0$, а не $\E[ \mathbf Z'_i \mathbf u_i] = \mathbf 0$ (см. раздел 22.4.3), а также использованию дополнительных моментных ограничений. Мы не решаемся назвать двухшаговую ОММ \textbf{оптимальной ОММ} оценкой,  как в разделе 6.3, так как она оптимальна только при выполнении условия \ref{Eq:22.3}.

{\centering Тест на сверх-идентифицирующие ограничения}

Если есть $r$  инструментов и только $K$  параметров для оценивания, то в ОММ для панельных данных остается $(r-K)$ сверх-идентифицирующих ограничений. Из раздела 6.3.8 это позволяет сделать \textbf{тест на сверх-идентифицирующие ограничения}
\begin{align}
OIR= \left[ \sum^N_{i=1} \hat{\mathbf u}'_i \mathbf Z_i \right] 
(N \hat{\mathbf S})^{-1}
 \left[ \sum^N_{i=1} \mathbf Z'_i \hat{\mathbf u}_i \right] 
\label{Eq:22.10}
\end{align}
где $\hat{ \mathbf u}_i=\mathbf y_i- \mathbf Z'_i \hat{\be}_{\mathrm{2SGMM}}, \hat{\mathbf S}$ дано в \ref{Eq:22.7}. Предполагается независимость по $i$, но допускаются гетероскедастичность и корреляция по $t$ для данного $i$. Заметим, что должна использоваться $\hat{\be}_{\mathrm{2SGMM}}$, а не $\hat{\be}_{\mathrm{2SLS}}$.

Эта тестовая статистика  распределена как $\chi^2(r-K)$ при выполнении нулевой гипотезы, что сверх-идентифицирующие ограничения верны. Если OIR  велика, то сверх-идентифицирующие моментные условия отвергаются, и мы делаем вывод, что некоторые из инструментов в $\mathbf Z_i$ коррелированы  с ошибкой, а следовательно эндогенны.

\subsection{Выбор инструментальных переменных}

В обсуждении предполагалось существование матрицы инструментов $\mathbf Z_i$ размерности $T \times r$, удовлетворяющей условию \ref{Eq:22.3}. Перейдем к подробному обсуждению, как получить инструменты в случае использования панельных данных.

В моделях пространственных данных, для эндогенных переменных применяются переменные, не являющиеся регрессорами в оцениваемом уравнении. Такие переменные также могут быть использованы и в случае с панельными данными. В случае с моделями панельных данных, однако, дополнительные временные периоды обеспечивают дополнительные моментные условия и дополнительные инструменты, которые легко могут привести к идентифицируемости или сверх-индентифицируемости $\be$.

Число доступных моментных условий и инструментов увеличивается по мере того, как делаются более строгие предположения о корреляции между $u_{it}$ и $\mathbf z_{is}$, $s,t=1, \dots, T$. Мы рассмотрим эффект более сильного \textbf{предположения об экзогенности} согласно М.Дж. Ли (2002), см. раздел 2.3. Акцент сделан на использовании экзогенных компонент регрессоров в качестве инструментов более чем один раз. Однако в методе также применяются более традиционные инструменты, т.е. переменные, не включенные в регрессию \ref{Eq:22.1}.

{\centering  Предположение о сумме}

Определим $\mathbf Z_i$ аналогично определению $\mathbf X_i$. Тогда
\begin{align}
\mathbf Z_i=
\begin{bmatrix}
\mathbf z'_{i1} \\
\mathbf z'_{i2} \\
\vdots \\
\mathbf z'_{iT}  
\end{bmatrix},
\mathbf u_i=
\begin{bmatrix}
u_{i1} \\
u_{i2} \\
\vdots \\
u_{iT}
\end{bmatrix},
\label{Eq:22.11}
\end{align}
где $\mathbf z_{it}$ имеет размерность $r \times 1$ и $\E[ \mathbf Z'_i \mathbf u_i]= \mathbf 0$, если \textbf{предположение о сумме}
\begin{align}
\E \left[ \sum^T_{i=1} \mathbf z_{it} u_{it} \right] =\mathbf 0
\label{Eq:22.12}
\end{align}
выполнено.

Это предположение соответствует предположению, использованному в сквозной МНК регрессии $y_{it}$ на $\mathbf x_{it}$, так как если $\mathbf z_{it}=\mathbf x_{it}$ в \ref{Eq:22.12}, то панельная ОММ оценка, определенная в \ref{Eq:22.4}, упрощается до $(\sum_i \mathbf Z'_i \mathbf X_i)^{-1} \sum_i \mathbf Z'_i \mathbf y_i$.

Чтобы эта оценка была доступна, как минимум необходимо, чтобы выполнялось порядковое условие, т.е. $r \geq K$. При выполнении предположения о сумме, найти инструменты в случае с панельными данными так же сложно, как и в случае с пространственными данными.

{\centering Предположение об одновременной экзогенности}

Более сильное и естественное предположение --- \textbf{это предположение об одновременной экзогенности}:
\begin{align}
&\E [\mathbf z_{it} u_{it}] =\mathbf 0
& t=1, \dots, T,
\label{Eq:22.13}
\end{align}
т.е. предполагается, что инструменты одновременно не коррелированы с ошибкой.

Этому предположению соответствует намного больше моментных условий. Получается $Tr$ моментных условий, где $r=\mathrm{dim}[\mathbf z_{it}]$. Чтобы их использовать, определим
\begin{align}
\mathbf Z_i=
\begin{bmatrix}
\mathbf z'_{i1}  &\mathbf 0 & \dots & \mathbf 0\\
\mathbf 0 & \mathbf z'_{i2} & & \vdots \\
\vdots & & \ddots & \mathbf 0 \\
\mathbf 0 & \hdots & \mathbf 0 & \mathbf z'_{iT}  
\end{bmatrix},
\mathbf u_i=
\begin{bmatrix}
u_{i1} \\
u_{i2} \\
\vdots \\
u_{iT}
\end{bmatrix},
\label{Eq:22.14}
\end{align}
где $\mathbf Z_i$ имеет размерность $Tr \times T$. Моментное условие \ref{Eq:22.3} выполнено, так как $\E[ \mathbf Z'_i \mathbf u_i]= \mathbf 0$ из \ref{Eq:22.13}, но \ref{Eq:22.3} сейчас определяет $Tr$ моментных условий, которые могут быть использованы для оценки $K$ компонент $\be$.

Этот примечательный результат очевидного избытка моментных ограничений исходит из неявного предположения, что $\be$ не меняется во времени, так что каждый дополнительный временной период добавляет дополнительные моментные ограничения.

Количество дополнительных моментных ограничений уменьшается в той мере, в которой $\be$ меняется во времени. В частности, свободный член зачастую меняется во времени при включении в $\x_{it}$ $(T-1)$ временных фиктивных переменных $d_{s,it}=1$, если $t=s$ и 0 иначе, для $s=2, \dots, T$. Тогда условие $\E[d_{s,it} u_{it}]=0$ не может быть использовано, так как оно повторяет условие $\E[1 \times u_{it}]=0$ включением свободного члена в $\x_{it}$. В предыдущем примере, если $\x_{1it}$ включает временные фиктивные переменные, то доступно только $TK-(T-1)$ моментных условий. Любые меняющиеся во времени регрессоры могут быть использованы как инструменты.

{\centering  Слабое предположение об экзогенности}


В моментном условии \ref{Eq:22.13} рассматривается только одновременная корреляция между инструментами и регрессорами. Более строгое предположение  --- \textbf{слабое предположение об экзогенности} или \textbf{предположение о предопределенных  инструментах}. Оно состоит в том, что лаговые значения инструментов не коррелированы с ошибкой в текущем периоде, т.е.
\begin{align}
&\E [\mathbf z_{it} u_{it}] =\mathbf 0
& s \leq t &
& t=1, \dots, T.
\label{Eq:22.15}
\end{align}
В условии \ref{Eq:22.15} $\mathbf z_{i1}, \dots, \mathbf z_{it}$ могут быть инструментами для $u_{it}$, хотя будущие значения $\mathbf z_{is}$ не могут быть использованы. Инструмент $ \mathbf Z_i$ имеет структуру, аналогичную \ref{Eq:22.14}, за тем исключением, что $\mathbf z'_{it}$  заменяется на вектор инструментов $[\mathbf z'_{i1}, \dots, \mathbf z'_{it}]$, увеличивающийся по мере увеличения $t$.

Условия такого типа используются в моделях рациональных ожиданий и в моделях межвременного принятия решений в условиях неопределенности, что приводит к \textbf{уравнениям Эйлера} вида $\E[u_{it} | \mathcal I_{it}]=0$, где $\mathcal I_{it}$ --- это информация, доступная в момент $t$, а пример $u_{it}$ дан в разделе 6.2.7. Если информация включает текущие и прошлые значения $\mathbf z_{it}$, то $\E[u_{it} | \mathbf z_{is}]=0, s \leq t$, что приводит к \ref{Eq:22.15}.

Эти условия становятся существенными в динамических моделях с лаговыми зависимыми переменными (см. раздел 22.5). В некоторых примерах одновременная корреляция не исключена, так что неравенство $s \leq t$ в \ref{Eq:22.15} заменено на $s < t$.

Заметим, что инструменты, не меняющиеся во времени, могут быть использованы лишь однажды. Т.е. если $\mathbf z_{it} = [ \mathbf z_{1i} \; \mathbf z_{2it}]$, то $\mathbf z_{1i}$ и $\mathbf z_{2i1}, \dots, \mathbf z_{2it}$ доступны в качестве инструментов.

{\centering  Сильное предположение об экзогенности}

Более сильное предположение  о слабой экзогенности  --- \textbf{предположение о сильной экзогенности}. Оно состоит в том, что будущие значения инструментов не коррелированы с ошибкой текущего периода, так что
\begin{align}
&\E [\mathbf z_{is} u_{it}] =\mathbf 0
& s, t=1, \dots, T.
\label{Eq:22.16}
\end{align}
Тогда текущие, прошлые и будущие значения $\mathbf z_{is}$ --- годные инструменты для $u_{it}$.

Это предположение было выполнено для регрессоров $\mathbf x_{it}$  в главе 21, так как из $\E[u_{it}| \mathbf x_{i1}, \dots, \mathbf x_{iT}]=0$ следует $\E[u_{it} | \x_{is}]=0, 1 \leq s \leq T$, а следовательно $\E[\x_{is} u_{it}]= \mathbf 0$. Прошлые предпосылки подходят для статических моделей, но в случае динамических моделей самое большое, что может предполагаться, --- это слабая экзогенность инструментов.

Условие \ref{Eq:22.16} показывает, что $\mathbf z_{i1}, \dots, \mathbf z_{iT}$ могут быть инструментами для $u_{it}$. Структура инструментов $\mathbf Z_i$ имеет вид \ref{Eq:22.14}, за исключением того, что $\mathbf z'_{it}$ в \ref{Eq:22.14} заменяется на расширенный вектор инструментов $[\mathbf z'_{i1}, \dots, \mathbf z'_{iT}]$.

В случае слабой экзогенности не меняющиеся во времени инструменты могут быть использованы только один раз. Если $\mathbf z_{it}=[\mathbf z_{1i} \; \mathbf z_{2it}]$, тогда доступно $T(r_{TI}+Tr_{TV})$ моментных условий, где $r_{TI}$ и $r_{TV}$  обозначают номера не меняющихся (Time Invariant, TI) и меняющихся (Time Varying, TV) во времени инструментов.

 Огромное количество моментных условий, $rT^2$, объясняется исключающими ограничениями, неявно сделанными в модели панельных данных \ref{Eq:22.1}. Для простоты предположим, что все компоненты $\x_{it}$ строго экзогенны и мы используем их в качестве инструментов всегда, когда это возможно. В общем $y_{it}$ может зависеть от регрессоров всех временных периодов, $\x_{i1}, \dots, \x_{iT}$. В модели панельных данных $y_{it}=\x'_{it}\be+u_{it}$ c $\E[\x_{it} u_{it}]=\mathbf 0$, напротив, $y_{it}$ зависит только от $\x_{it}$. Предположение о сильной экзогенности, что $\E[\x_{is} u_{it}]=\mathbf 0$, делает возможным использование регрессоров других периодов $\x_{is}, s \neq t$ (не только $\x_{it}$)  в качестве инструментов. 

{\centering  Излишние инструменты}
 
Если $\mathbf z_{it}$ меняется по $i$ и $t$, то лаги могут быть использованы как инструменты, в зависимости от сделанных предположений об экзогенности. Для $i$-го наблюдения в случае одновременной экзогенности доступны инструменты $\mathbf z_{it}$, в случае слабой экзогенности  --- $\mathbf z_{i1}, \dots, \mathbf z_{it}$, а в случае сильной экзогенности --- $\mathbf z_{i1}, \dots, \mathbf z_{iT}$. Благодаря этому идентифицирование возможно при использовании только экзогенных регрессоров в качестве инструментов. Трудности нахождения годных инструментов, сравнимые с трудностями в случае использования пространственных данных, возникают только при выполнении предположения о сумме.

На практике, однако, нет такого количества доступных инструментов, как обещано выше. \textbf{Не меняющиеся во времени инструменты} $\mathbf z_{it}= \mathbf z_{i}$ могут быть использованы только один раз, так как в этом случае $\mathbf z_{it}= \mathbf z_{is}$ для любых $s$ и $t$. Например, это относится к свободному члену или идентификаторам расы и пола. Если в качестве инструмента в модели используется сам регрессор или его лаговые значения, то количество доступных инструментов снижается. Инструменты, изменяющиеся во времени систематическим образом, могут быть доступны не во всех периодах. Поэтому инструменты,  построенные как произведение временных дамми и регрессоров, не меняющихся во времени, должны быть включены в регрессию только один раз в случае использования полного набора временных дамми. В примерах присутствуют временные дамми и произведения временных дамми с индикаторами расы или пола. Инструменты, имеющие линейную функцию по времени, следует использовать только один раз. Например,  если год является инструментальной переменной, то не следует использовать лаговые значения временных дамми. Этот комментарий не относится к возрасту, который увеличивается линейно для каждого индивидуума, но изменяется по $i$.

Довольно легко непреднамеренно использовать \textbf{излишние инструменты}. ОММ оценки для панельных данных по прежнему доступны и обычные результаты верны, если  есть достаточное количество годных инструментов. Например, если используется $r$ инструментов, и два из них излишние, то модель все еще может быть оценена при $r \geq K +2$, так как $\mathbf Z' \mathbf X$ имеет полный ранг $K$. Если используется слишком много излишних инструментов, что приводит к недоидентифицируемости модели, могут возникнуть проблемы сингулярности в оценивании ОММ. Даже если модель сверх-идентифицируема, количество степеней свободы в тесте на сверх-идентифицирующие ограничения будет уменьшено, если некоторые инструменты излишни.

{\centering  Слабые инструменты}
 

Слабые инструменты не стоит путать со слабой экзогенностью, которая описана в разделе 4.9. Не существует устоявшегося формального теста на \textbf{слабые инструменты}. Стандартные $R^2$ и $F$-статистика представлены в разделе 4.9. Важно именно увеличение объясняющею силы при использовании инструментов. Следует использовать частичный $R^2$, который учитывает экзогенные регрессоры, входящие в набор инструментов. Более того, в то время как строится регрессия эндогенных регрессоров на  все инструменты, $F$-статистика должна показывать значимость части инструментальных переменных, которые не являются экзогеными регрессорами.

Так как ошибки здесь  не являются независимыми и одинаково распределенными, $F$-статистика должна быть вычислена на основании робастных стандартных ошибок для  панельных данных. Она может быть вычислена как $W/r^*$, где $W$  --- это тестовая статистика Вальда, распределенная как $\chi^2$, для ограничений раздела 7.2.7, и $r^*$ --- количество инструментов, не являющихся регрессорами первоначальной модели.

\subsection{Вычисление оценок ОММ для панельных данных}

Моментные условия, которые обсуждались в предыдущем разделе, приводят к матрице инструментов $\mathbf Z_i$. Тогда, зная $\mathbf Z_i$, можно оценить $\be$ с помощью $\hat{\be}_{\mathrm{2SLS}}$, определенной в \ref{Eq:22.8}, или $\hat{\be}_{\mathrm{2SGMM}}$, определенной в \ref{Eq:22.9}.

Применение двухшаговой МНК оценки проще, чем двухшаговой ОММ оценки. Рассмотрим оценивание в предположении о сумме; $\mathbf Z_i$ определено в \ref{Eq:22.11}. Тогда $\hat{\be}_{\mathrm{2SLS}}$ дана в \ref{Eq:22.8}, где $\mathbf Z' \mathbf X =\sum_i \mathbf Z_i' \mathbf X_i=\sum_i \sum_t \mathbf z_{it} \x'_{it}$. Похожие алгебраические преобразования применяются для других произведений. В результате этих преобразований получается стандартная формула из учебника для двухшагового МНК, за исключением того, что суммирование производится по $i$ и по $t$. Таким образом, $\hat{\be}_{\mathrm{2SLS}}$ может быть получена с помощью двухшаговой МНК регрессии $y_{it}$ на $\x_{it}$ в пакете анализа данных пространственного. \textbf{Робастные для панельных данных} стандартные ошибки могут быть получены благодаря использованию кластерно-робастной опции с кластеризацией по $i$, или \textbf{панельного бутстрепа} с ресемплингом только по $i$, а не по  $i$ и $t$. Подходы похожи на те, что применялись для МНК сквозной регрессии, подробно описанной в разделе 21.2.3. 

Для предположений отличных от предположения о сумме можно использовать статистический пакет для реализации двухшагового МНК для данных пространственного типа, подходящим образом определяя матрицу инструментов $\mathbf Z_i$, которая имеет более сложную форму. Для  предположения об  одновременной экзогенности, $\mathbf Z_i$ определена в \ref{Eq:22.14}. Это будет та же самая форма, как и в \ref{Eq:22.11}, если $t$-ую строку в \ref{Eq:22.11}, $\mathbf z'_{it}$, заменить на 
\begin{align}
[\mathbf 0'_{r_1} \dots \mathbf 0'_{r_{t-1}} \mathbf z'_{it} \mathbf 0'_{r_{t+1}} \dots \mathbf 0'_{r_{T}}],
\label{Eq:22.17}
\end{align}
где $r_s=\mathrm{dim}[\mathbf z_{is}]$ и $\mathbf 0_{r_s}$ обозначает нулевой вектор размерности $r_s \times 1$. Подобным образом, для предположения о слабой экзогенности, $\mathbf Z_i$  --- это \ref{Eq:22.11} с $t$-ой строкой $\mathbf z'_{it}$, замененной на
\begin{align}
[\mathbf 0'_{r_1} \dots \mathbf 0'_{r_{t-1}} (\mathbf z^t_{it})' \mathbf 0'_{r_{t+1}} \dots \mathbf 0'_{r_{T}}],
\label{Eq:22.18}
\end{align}
где $(\mathbf z^t_{it})'=[\mathbf z'_{i1} \dots \mathbf z'_{it}]$ и $r_s=\mathrm{dim}[\mathbf z^s_{is}]$, и для предположения о строгой экзогенности, $\mathbf Z_i$ --- это \ref{Eq:22.11} c $t$-ой строкой $\mathbf z'_{it}$, замененной на
\begin{align}
[\mathbf 0'_{r_1} \dots \mathbf 0'_{r_{t-1}} (\mathbf z^T_{it})' \mathbf 0'_{r_{t+1}} \dots \mathbf 0'_{r_{T}}],
\label{Eq:22.19}
\end{align}
где $(\mathbf z^T_{it})'=[\mathbf z'_{it} \dots \mathbf z'_{iT}]$ и $r_s=\mathrm{dim}[\mathbf z^T_{is}]$. Практический пример генерирования инструментов представлен в разделе 22.3.

На практике может быть очень много моментных условий. Например, для данных с 10 временными периодами и 5 изменяющимися во времени регрессорами предположение о сильной экзогенности дает $5 \times 10^2=500$ моментных условий (и рассмотренный вектор-строка имеет 500 элементов) с всего лишь 5 параметрами для оценки. Предельная значимость инструментов может быть очень маленькой в связи с увеличивающейся мультиколлинеарностью инструментов, что приводит к ситуации слабых инструментов. Один из полезных приёмов приемов --- те инструменты, которые незначительно меняются во времени, использовать как не меняющиеся во времени инструменты. Например, в качестве инструментов использовать только значения первого периода. Даже у регрессоров, которые значительно меняются во времени, могут быть использованы значения не во все возможные временные периоды, а только за несколько периодов.

Вычисление более эффективных оценок $\hat{\be}_{\mathrm{2SGMM}}$ невозможно при использовании только статистического пакета для двухшагового МНК. Вместо этого необходимо либо использовать более специализированные пакеты, либо запрограммировать оценку, используя матричные операции.

В таблице 22.1 кратко представлены четыре предположения об экзогенности и соответствующие годные инструменты.

\subsection{Другие типы оценивания ОММ}

Хотя $\hat{\theta}_{\mathrm{2SLS}}$ более эффективна, чем $\hat{\theta}_{\mathrm{2SLS}}$, некоторые исследования показывают, что она имеет большее смещение в связи с ограниченной выборкой,  чем $\hat{\theta}_{\mathrm{2SLS}}$, особенно когда $r$ намного больше, чем $K$. Для объяснения см. обсуждение смещения оптимального ОММ в связи с ограниченной выборкой в разделе 6.3.5.

Один из подходов состоит в том, чтобы быть умеренным в использовании инструментов, хотя в таком случае возможные выгоды в эффективности от использования дополнительных инструментов будут потеряны.

Несколько авторов предложили альтернативные ОММ оценки, которые с меньшей вероятностью страдают от смещения в ограниченных выборках. Многие из них представлены в разделе 6.4.4 и используются в исследовании панельных данных Зилиак (1997).

\begin{table}[ht]
\caption{{\it Предположения об экзогенности для панельных данных и инструменты}} 
\centering
\begin{tabular}{ccc}
\hline \hline
	Предположение об экзогенности & Моментное условие & Вектор инструментов  $^a$ \\
\hline
О сумме & $\E[\sum_t \mathbf z_{it} u_{it}=\mathbf 0$ & $[\mathbf z_{it}]$\\
Одновременная  & $\E[\mathbf z_{it} u_{it}=\mathbf 0$ для всех $t$ & $[\mathbf 0'_{r_1} \dots \mathbf 0'_{r_{t-1}} \mathbf z^t_{it} \mathbf 0'_{r_{t+1}} \dots \mathbf 0'_{r_{T}}]$ \\
Слабая	 & $\E[\mathbf z_{it} u_{it}=\mathbf 0$ $s\leq t$ для всех $t$ & $[\mathbf 0'_{r_1} \dots \mathbf 0'_{r_{t-1}} (\mathbf z^t_{it})' \mathbf 0'_{r_{t+1}} \dots \mathbf 0'_{r_{T}}]$\\
Сильная & $\E[\mathbf z_{it} u_{it}=\mathbf 0$ для всех $t$ и $s$ & $[\mathbf 0'_{r_1} \dots \mathbf 0'_{r_{t-1}} (\mathbf z^T_{it})' \mathbf 0'_{r_{t+1}} \dots \mathbf 0'_{r_{T}}]$ \\
\hline \hline
\multicolumn{3}{p{14cm}}{$^a$ Вектор инструментов --- это $t$-ая строка $\mathbf Z_i$ в \ref{Eq:22.11}; $(\mathbf z^t_{it})'=[\mathbf z'_{i1} \dots \mathbf z'_{it}]$, $(\mathbf z^T_{it})'=[\mathbf z'_{i1} \dots \mathbf z'_{iT}]$;  и $r_s=\mathrm{dim}[\mathbf z'_{is}]$ или $\mathrm{dim}[\mathbf z^s_{is}]$ или $\mathrm{dim}[\mathbf z^T_{is}]$.}\\
\end{tabular}
\label{Tab:22.1}
\end{table}

\subsection{Оценка Чемберлина}

Рассмотрим модель с индивидуальными эффектами
\begin{align}
y_{it}=\alpha_i +\x'_{it}\be +u_{it},
\label{Eq:22.20}
\end{align}
где регрессоры строго экзогенны, как в главе 21. В разделах 21.2.3 и 21.6.1 представлены методы получения робастных стандартных ошибок для оценки within для панельных данных.

Здесь требуется использование робастных статистических выводов для панельных данных в виду того, что $\e_{it}$ не являются независимыми и одинаково распределенными. Оценки, описанные в главе 21, не будут эффективными. Более эффективные оценки возможны благодаря использованию оптимального ОММ, примененного к сверх-идентифицируемой модели. Здесь $\x_{is}$, $s \neq t$, доступны как дополнительные инструменты и ОММ можно применить к преобразованной модели, если $\alpha_i$ необходимо элиминировать (см. раздел 22.4.2). Улучшение эффективности такое же, что и при использовании данных пространственного типа с гетероскедастичностью (см. раздел 6.3.5).

Чемберлин (1982, 1984) предложил следующую более эффективную оценку. Модель  \ref{Eq:22.20} можно представить в виде 
\begin{align}
\mathbf y_{i}=\mathbf e \alpha_i + (\mathbf I_T \otimes \be')\x_i+\mathbf u_i,
\label{Eq:22.21}
\end{align}
где $\mathbf e=(1, 1, \dots, 1)'$ --- единичный вектор размера $T \times 1$, $\x_i=[\x'_{i1} \dots \x'_{iT}]$ вектор размера $TK \times 1$, и $\mathbf y_i$ и $\mathbf u_i$ вектора размера $T \times 1$. Из уравнения \ref{Eq:22.21} понятны ограничения, которые сделаны неявно в статических моделях, в которых $y_{it}$ зависит только от $\x_{it}$ одновременных периодов. Чемберлин использовал аргументы линейной проекции, которые опираются на более слабые предположения, чем при условном ожидании. Пусть
\begin{align}
\E^*[\alpha_i| \x_i]=\mu+\sum_t \bm\lambda'_t \x_{it}=\bm\mu + \bm\lambda' \x_i,
\nonumber
\end{align}
где $\E^*$ обозначает линейную проекцию. Зная, что $\E[ \mathbf u_i | \alpha_i, \x_i]=\mathbf 0$, из \ref{Eq:22.21} получаем
\begin{align}
\E^*[\mathbf y_i| \x_i]=\mathbf e \mu+( \mathbf I_T \otimes  \be' +\mathbf e \bm\lambda') \x_i.
\nonumber
\end{align}
Это накладывает ограничения на неограниченную линейную проекцию
$\E^*[\mathbf y_i| \mathbf \x_i]=\pi_o + \pi'\x_i$, а именно $\pi-\mathbf I_T \otimes \be' + \mathbf e \bm\lambda'=\mathbf 0$.

Вместо использования ОММ, Чемберлин предложил следующую двухшаговую процедуру. Во-первых, с помощью многомерной МНК регрессии $\mathbf y_i$ на $\x_i$ и свободный член получить $\hat{\pi}$. Во-вторых, получить \textbf{оптимальную  оценку минимального расстояния} (см. раздел 6.7), при которой минимально
\begin{align}
Q_N(\be, \lambda)=(\mathrm{Vec}[\hat{\pi}-\mathbf I_T \otimes \be' - \mathbf e \lambda'])'\mathbf W_N(\mathrm{Vec}[\hat{pi}-\mathbf I_T \times \be' - \mathbf e \la']),
\nonumber
\end{align}
где оптимальная взвешивающая матрица $\mathbf W_N=(\hat{\mathrm{V}}[\mathrm{Vec}[\pi]])^{-1}$. Полученная оценка $\hat{\be}$ более эффективная, чем МНК оценка \ref{Eq:22.20}, если $u_{it}$ гетероскедастична.

Оценка минимального расстояния была вытеснена ОММ; см. Ареллано (2003, c. 22---23) и Крепон и Мересс (1995) для сравнения оценки минимального расстояния с ОММ оценкой. Однако подход Чемберлина получения моментных ограничений через предположения об экзогенности и предположения об индивидуальных эффектах оказали большое влияние на литературу, посвященную анализу панельных данных. Его оценка минимального расстояния также использовалась для оценивания ковариационных структур (см. раздел 22.5.4). 


\section{Пример оценивания ОММ для панельных данных: Часы и заработная плата}

Вернемся к примеру раздела 21.3 о заработной плате и количестве часов работы. В отличие от главы 21 регрессоры теперь могут быть эндогенными. Включим также фиксированные индивидуальные эффекты. Оценивание производится с помощью методов раздела 22.2 после взятия первых разностей для уничтожения фиксированных эффектов.

Регрессионная модель
\begin{align}
lnhrs_{it}=\alpha_i+\beta_1 lnwg_{it} + \beta_2 kids_{it} +\beta_3 age_{it} + \beta_4 agesq_{it} + \beta_5 disab_{it} + u_{it},
\nonumber
\end{align}


где нас интересует межвременная эластичность замещения предложения труда по заработной плате, $\be_1$, коэффициент lnwg. Дополнительные регрессоры - количество детей, возраст, возраст в квадрате, и индикатор нетрудоспособности.

МаКарди (1981) вывел эту взаимосвязь, используя модель жизненного цикла предложения труда в условиях неопределенности. Эта модель является <<$\la$-постоянной>>, где $\alpha_i$  равно $\la_i$ и пропорционально предельной полезности от первоначальное богатство, неизменно во времени, но изменяется по $i$. Так как $\la_i$ зависит от переменных и ограничений, его необходимо воспринимать как фиксированный, а не случайный эффект. В литературе, посвященной анализу предложения труда, представлено несколько методов для учета фиксированных эффектов.

Один метод, который будет обсуждаться далее в разделе 22.4.2, заключается во взятии разностей в регрессионной модели:
\begin{align}
\Delta lnhrs_{it}=\beta_1 \Delta lnwg_{it} + \beta_2\Delta kids_{it} +\beta_3 \Delta age_{it} + \beta_4 \Delta agesq_{it} + \beta_5 \Delta disab_{it} + \Delta u_{it}.
\label{Eq:22.22}
\end{align}
Если все регрессоры экзогенны, то МНК оценивание даст состоятельные оценки для $\be$. Заметим, что взятие разностей приводит к тому, что ошибки становятся коррелированными во времени, даже если $u_{it}$ независимы и одинаково распределены. Вследствие этого нужно использовать робастные стандартные ошибки для панельных данных.

Зилиак (1997) же предполагает, что $lnwg_{it}$ одновременно коррелирует с $u_{it}$ из-за ошибки измерения в заработной плате или из-за изломов кривой бюджетного ограничения. Тогда МНК оценка \ref{Eq:22.22} несостоятельна.

Зилиак предложил оценивание методом инструментальных переменных c использованием в качестве инструментов соответствующих лагов регрессоров. Предположим, что заработные платы прошлых периодов не коррелированы с ошибкой, вследствие чего lnwg будет слабо экзогенным помимо того, что он одновременно коррелирован с ошибкой. Тогда из $\E[lnwg_{is}u_{it}]=0$ при $s \leq t-1$ следует, что для ошибки модели в разностях $\E[lnwg_{is}\Delta u_{it}]=0$ при $s \leq t-2$. Значит, в качестве инструментов в модели в первых разностях могут быть использованы лаги второго и более высокого порядка. Заметим, что по крайней мере три периода данных нужны для идентифицируемости $\be$.

Центральным предметом исследования Зилиака являются характеристики ОММ оценок с эндогенными регрессорами. Он считал, что все регрессоры эндогенны и использовал в качестве инструментов лаги первого и более высокого порядков в уровнях других  четырех регрессоров. Для простоты не были включены свободный член и временные дамми, а также инструменты, неизменные для индивидуумов и используемые только один раз. В связи с включением свободного члена в состав зависимых переменных в модель в разностях результаты немного отличаются. Так как $lnwg_{i,t-2}$ всегда использовалось в качестве инструмента, первые два года были исключены и для оценки \ref{Eq:22.22} использовалось данные только за восемь лет 1981---1988.


\begin{table}[ht]
\caption{{\it  Часы и заработная плата: GMM-IV оценки линейных моделей панельных данных$^a$}} 
\centering
\begin{tabular}{cccccc}
\hline \hline
 & &	\multicolumn{2}{c}{Базовый случай} & \multicolumn{2}{c}{Составной}  \\
 & \textbf{МНК} &	\textbf{2SLS} & \textbf{2SGMM} & \textbf{2SLS} & \textbf{2SGMM}   \\
\hline
 $\beta_1$ & 0.112 &	0.209 & 0.547 & 0.543 & 0.330   \\
Панельные ст.ош. & (.096) &	 (.374)  & (.327)  & (.209) & (.110)   \\
Гетеро ст.ош. & [.079] & [.423] & [-] & [.226] & [-]   \\
Cт.ош. по умолчанию & \{.023\} &	 \{.389\}  & \{-\}  & \{.169\} & \{-\}   \\
RMSE & .283 &	 .296  & .307  & .307 & .298   \\
Инструменты & 5 &	 9  & 9  & 72 & 72   \\
OIR Tест & - &	 -  & 5.45  & - & 69.51   \\
ст. своб. & - &	 -  & 4  & - & 67   \\
P-значение & - &	 -  & .244  & - & .393   \\
N & 4256 & 4256  & 4256  & 4256 & 4256   \\
\hline \hline
\multicolumn{6}{p{14cm}}{$^a$ Для регрессии в разностях используются ежегодные данные за период с 1981 по 1988 гг. для 532 мужчин. В таблице представлены значения $\be_1$, коэффициент $\Delta lnwg$, и три типа оцененных стандартных ошибок: робастные для панельных данных в круглых скобочках, робастные к гетероскедастичности в квадратных скобках, и обычные оценки по умолчанию в фигурных скобках, которые предполагают независимые и одинаково распределенные ошибки. Все регрессии дополнительно включают $\Delta kids$, $\Delta age$, $\Delta agesq$ и $\Delta disab$ в качестве регрессоров, но оценки их коэффициентов не представлены. В качестве инструментов используются лаг lnwg второго порядка и лаги  kids, age, agesq и disab первого и второго порядка. Для базового случая используется 9 инструментов и $9 \times 8=72$ составных инструмента. RMSE -  средняя квадратичная ошибка остатков. OIR --- статистика теста на сверх-идентифицируемость ограничений, ст. своб. --- степени свободы и Р-значение --- это точное Р-значение для этого теста.}\\
\end{tabular}
\label{Tab:22.2}
\end{table}

Таблица \ref{Eq:22.2} представляет небольшую подвыбрку результатов, данных в таблицах 1 и 2 Зилиак (1997). Для полноты даны оценки различных стандартных ошибок, но использоваться должны робастные стандартные ошибки для панельных данных.

\textbf{МНК}: В колонке МНК представлены результаты МНК оценивания модели \ref{Eq:22.22}. Эластичность предложения труда 0.112 немного отличается от оценки 0.109 в колонке первых разностей таблицы \ref{Tab:21.2}, так как здесь четыре демографические переменные включены как регрессоры и данные дополнительного года были исключены из анализа. Так как моделируются первые разности, качество подгонки модели оставляет желать лучшего, и $R^2$ с включением свободного члена равен  0.006.

\textbf{2SLS  с инструментами базового случая}: Инструменты базового случая используют матрицу $\mathbf Z_i$, определенную в \ref{Eq:22.11}, где $\mathbf z_{it}$ состоит из девяти компонент: $\mathrm{lnwg}_{i,t-2}$, $\mathrm{kids}_{i,t-1}$, $\mathrm{age}_{i,t-1}$, $\mathrm{agesq}_{i,t-1}$, $\mathrm{disab}_{i,t-1}$, $\mathrm{kids}_{i,t-2}$, $\mathrm{age}_{i,t-2}$, $\mathrm{agesq}_{i,t-2}$ и $\mathrm{disab}_{i,t-2}$. Модель сверх-идентифицируема и включает девять инструментов  и пять параметров для оценки. Двухшаговая МНК оценка (2SLS) $\be_1$ менее точна, чем МНК оценка. Стандартные ошибки этой оценки превышают МНК оценки в четыре раза: 0.096 против 0.374. Для других регрессоров, которые не представлены в таблице, потеря эффективности намного меньше.

\textbf{2SLS  с составными инструментами}: Базовый случай --- это ОММ, основанный на девяти моментных условиях $\E[\sum^{10}_{t=3} \mathbf z_{it} u_{it}]=\mathbf 0$. В случае же составных инструментов используется 72 ($8 \times 9$) моментных условия $\E[\mathbf z_{it} u_{it}]=\mathbf 0, t=3,\dots,10$, где $\mathbf z_{it}$ то же, что и в базовом случае. Здесь используется матрица $\mathbf Z_i$, определенная в \ref{Eq:22.14}, где $\mathbf Z_i$ включает в себя 8 лет по 72 инструмента. $t$-ая строка $\mathbf Z_i$ представлена в \ref{Eq:22.17}, где $\mathbf z_{it}$ --- это вектор инструментов размерности $9 \times 1$ для базового случая. Для конструирования инструментов в первую очередь генерируется 72 переменных $ztj$, равные нулю для всех $i$ и $t$, где $t$ обозначает год и $j$ обозначает $j$й инструмент. Затем $ z sj_{it}$ заменяется на $ z_{it,j}$, если $t=s$, и остается равным нулю, если $t \neq s$. Например, если $t=3$ (третий год) $ z 35$ заменяется на $\mathrm{disabl}_{i,2}$, если пятый инструмент $\mathrm{disabl}_{i,t-1}$ и $ z t5$ равно нулю при $t \neq 3$. 2SLS оценки могут получены с помощью стандартной двухшаговой МНК регрессии (2SLS) $\Delta \mathrm{lnhrs}_{it}$ на пять регрессоров в \ref{Eq:22.22} c этими 72 переменными, сконструированными в качестве инструментов. При использовании расширенных инструментов стандартные ошибки упали с 0.374 до 0.209, и теперь они всего лишь в два раза больше стандартных ошибок МНК оценки.

\textbf{Двухшаговый ОММ}:  Двухшаговые ОММ оценки в таблице \ref{Eq:22.2} отличаются от оценок таблицы 1 Зилиака (1997), так как робастная оценка $\hat{\mathbf S}$, определенная в   \ref{Eq:22.7}, здесь используется для формирования взвешивающей матрицы, в то время как Зилиак использовал робастную к гетероскедастичности оценку $\hat{\mathbf S}=N^{-1} \sum_i\hat{u}^2_{it} \mathbf z_{it} \mathbf z'_{it}$. Как и ожидалось, двухшаговая ОММ оценка более эффективна, чем оценка двухшагового МНК (2SLS). Стандартные ошибки уменьшились с 0.374 до 0.327 в базовом случае, и с 0.209 до 0.110 --- в случае с составными инструментами. Эта последняя стандартная ошибка ненамного больше, чем стандартная ошибка для МНК оценки.

\textbf{Тест на сверх-идентифицирующие ограничения}: Тестовая статистика на сверх-идентифицирующие ограничения, когда $q >k$, дана в \ref{Eq:22.10}. Из таблицы \ref{Tab:22.2} для обоих базовых случаев и случаем в составными инструментами P-значения тестовых статистик намного выше 0.05, поэтому ограничения не отвергаются и мы делаем заключение, что перед нами годные инструменты. 

\textbf{Тест на слабые инструменты}: Способы диагностирования слабых инструментов были представлены  в разделе 22.2.4 и разделе 5.9. Так как ни один из регрессоров не используется как инструмент, используется общая F-статистика из регрессии первого шага, а не F-статистика подмножества регрессоров. Для инструментов базового случая, в случае регрессии $\Delta \mathrm{lnwg}$ на девять инструментов и константу получается робастная для панельных данных $F=2.80$, и в случае схожей регрессии для 72 составных инструментов  $F=1.90$. Это говорит о том, что, скорее всего, смещение в связи с конечностью выборки имеет место. В случае схожих регрессий для $\Delta \mathrm{kids}$, $\Delta \mathrm{age}$, $\Delta \mathrm{agesq}$, $\Delta \mathrm{disab}$, регрессоров модели \ref{Eq:22.22}, которые тоже считались эндогенными, $F > 8.5$  во всех случаях. Частный коэффициент детерминации $R^2$ (см. раздел 4.9.1) Шеа равен 0.0036 для $\Delta \mathrm{lnwg}$ и превышает 0.075 для  четырех других эндогенных регрессоров. Проблема слабых инструментов обуславливается трудностями в нахождении хороших инструментов для $\Delta \mathrm{lnwg}$.

\textbf{Увеличение эффективности}: В этом примере оценки ОММ были использованы для того, чтобы решить проблему эндогенности. Однако даже если все регрессоры строго экзогенны, ОММ оценки остаются привлекательными, так как они более эффективны, чем МНК, при независимых и одинаково распределенных ошибках $u_{it}$; см. обсуждение после \ref{Eq:22.20}. Например, двухшаговое ОММ оценивание, где в качестве инструментов используются базовые инструменты и пять первоначальных регрессоров в \ref{Eq:22.22}, дает $\hat{\be}_1=0.016$ со стандартными ошибками 0.076 (ниже, чем стандартные ошибки при МНК, равные 0.096).

\section{ОММ для панельных данных со случайными и фиксированными эффектами}

Теперь расширим модель панельных данных \ref{Eq:22.1}, включив в нее аддитивные \textbf{индивидуальные эффекты} $\alpha_i$, не меняющиеся во времени:
\begin{align}
y_{it}=\alpha_i+\x'_{it}\be + \e_{it}.
\label{Eq:22.23}
\end{align}
Тогда ошибка в \ref{Eq:22.1} имеет вид $u_{it}=\alpha_i +\e_{it}$. Для простоты одни и те же  обозначения используются и в моделях с фиксированными, и со случайными эффектами. В случае модели со случайными эффектами общий свободный член $\mu$ в разделе 21.7 включен в $\x'_{it} \be$.

Предполагается, что некоторые компоненты $\x_{it}$ \textbf{эндогенны}, $\E[\x_{it}(\alpha_i+\e_{it}] \neq \mathbf 0$, вследствие чего МНК оценка $\be$ несостоятельна. В этом разделе мы предлагаем оценивание методом инструментальных переменных, которое дает состоятельные оценки $\be$ в различных постановках моделей, включая модели с фиксированными и случайными эффектами, гибрид этих двух моделей и системы уравнений.

\subsection{Фиксированные или случайные эффекты?}

Вспомним из 21 главы, что индивидуальные эффекты $\alpha_i$ могут рассматриваться как случайные эффекты в моделях как с фиксированными, так и со случайными эффектами. Эта случайная переменная $\alpha_i$ не зависела от $\x_{it}$ в модели со случайными эффектами, но была коррелирована  с $\x_{it}$ в модели с фиксированными эффектами. В модели со случайными эффектами могут быть оценены коэффициенты любых регрессоров, в то время как в модели с фиксированными эффектами невозможно оценить коэффициенты регрессоров, не  меняющихся во времени. Это объясняется тем, что для состоятельного оценивания при взятии разности  уничтожаются $\alpha_i$ и регрессоры, не меняющиеся во времени.

В текущей главе, имея дело в том числе с эндогенными регрессорами, мы рассматриваем модель со \textbf{случайными эффектами}, если существуют инструменты $\mathbf Z_i$, удовлетворяющие $\E[\mathbf Z'_i(\alpha_i+\e_{it})]=\mathbf 0$. Тогда используя методы раздела 22.2, можно получить состоятельные оценки параметров всех регрессоров. Если возможно найти только те инструменты, для которых $\E[\mathbf Z'_i \e_{it}]=\mathbf 0$ и $\E[\mathbf Z'_i \alpha_i] \neq 0$, то модель рассматривается как модель с \textbf{фиксированными эффектами}. Тогда $\alpha_i$ 
должны быть уничтожены при взятии разности. И в этом случае идентифицируемы будут только коэффициенты изменяющихся во времени регрессоров.

\subsection{Инструментальные переменные для моделей с фиксированными эффектами}

Применяя к \ref{Eq:22.23} различные комбинации взятия разностей, данные в разделе 21.2, получаем \textbf{преобразованную модель} вида
\begin{align}
\tilde{y}_{it}= \tilde{\x}'_{it}\be+\tilde{\e}_{it},
\nonumber
\end{align}
где тильда $\tilde{}$  обозначает преобразование, при которой уничтожается $\alpha_i$. Основные примеры преобразований представлены ниже. Будем использовать запись
\begin{align}
\tilde{\mathbf y}_{i}= \tilde{\mathbf X}'_{i}\be+\tilde{\bm\e}_{i}.
\label{Eq:22.24}
\end{align}
Если $\E[\x_{it} \e_{it}] \neq 0$, тогда $\E[\tilde{\x_{it}} \tilde{\e}_{it}] \neq 0$ и МНК оценивание \ref{Eq:22.24} дает несостоятельные оценки.

Сейчас рассмотрим оценивание методом инструментальных переменных, предполагая существование инструментов $\mathbf Z_i$, которые удовлетворяют  условию $\E [\mathbf Z'_i \tilde{\bm\e}_i]=\mathbf 0$. Тогда ОММ оценивание (инструментальные переменные, IV, двухшаговый МНК, 2SLS, или двухшаговый ОММ, 2SGMM), модели \ref{Eq:22.24} с инструментами $\mathbf Z_i$ дает состоятельные оценки коэффициентов изменяющихся во времени регрессоров. Робастные стандартные ошибки для панельных данных могут быть вычислены так, как это обсуждалось в разделе 22.2.2.

Один из способов: инструменты могут быть получены аналогично случаю пространственных данных. Годный инструмент --- это переменная, которая коррелирует с регрессором, но не корелирована с ошибкой,  а также та, которая может быть исключена из правой части уравнения \ref{Eq:22.23}. Другой способ получить инструменты — через экзогенные регрессоры не текущих периодов с использованием предположения об экзогенности, которое подробно обсуждается в разделе 22.2.4.

Обычные предположения для пригодности использования инструментов основаны на корреляции между $\mathbf z_{is}$ и $\e_{it}$. Однако в текущем случае для нас важна корреляция между $\mathbf z_{is}$ и разности ошибки $\tilde \e_{it}$. В общем случае взятие разностей, необходимое для уничтожения фиксированных эффектов, уменьшает количество доступных инструментов. Некоторые операции взятия разностей  приводят к большим потерям, чем другие, и могут даже приводить к несостоятельному оцениванию методом инструментальных переменных. Мы рассмотрим три операции взятия разностей, уделяя особое внимание \textbf{слабо экзогенным инструментам}. На практике это может быть более реалистичным предположением, особенно применительно к динамическим моделям.

{\centering  Инструментальные переменные для модели в первых разностях}

\textbf{Оценка инструментальных переменных, IV, в первых разностях} --- это оценка инструментальных переменных, IV, или оценка двухшагового МНК, 2SLS, или ОММ оценка \textbf{модели в первых разностях}
\begin{align}
& y_{it}-y_{i,t-1}=(\x_{it}-\x_{i,t-1})'\be + (\e_{it}- \e_{i,t-1}),
& t=2, \dots, T.
\label{Eq:22.25}
\end{align}
Из предположения о слабой экзогенности $\E[ \mathbf z_{is} \e_{it}]=\mathbf 0$ для $s \leq t$ следует
$\E[\mathbf z_{is} (\e_{it}-\e_{i,t-1})]=\mathbf 0$ для $s \leq  t-1$. Взятие первых разностей уменьшает временной ряд доступных инструментов на один период, поэтому только $\mathbf z_{i,t-1}, \mathbf z_{i,t-2}, \dots$ доступны в качестве инструментов. Предполагая слабую экзогенность, получаем состоятельную IV оценку для $\be$.

Использование лагов регрессоров в качестве инструментов было впервые предложено Андерсоном и Хсяо (1981) в контексте динамических моделей панельных данных и было расширено Хольтц-Экином, Ньюи и Розеном (1988) и Ареллано Бондом (1991) (см. раздел 22.5.3). В разделе 22.3  дан подробный эмпирический пример этого подхода.

Заметим, что вместо этого можно использовать преобразованные инструменты $\tilde{\mathbf z}_{is}=\Delta \mathbf z_{is}=\mathbf z_{is}-\mathbf z_{i,s-1}$, $s \leq t-1$. Однако это не дает преимуществ, так как использование $\Delta \mathbf z_{i,t-1}, \dots, \Delta \mathbf z_{i2}, \mathbf z_{i1}$ эквивалентно использованию $\mathbf z_{i,t-1}, \dots, \mathbf z_{i2}, \mathbf z_{i1}$ в качестве инструментов, и, если данные начинаются в момент $t=1$, то в нашем распоряжении только $\mathbf z_{i1}$, а не $\Delta \mathbf z_{i1}$.

{\centering  IV для модели within или модель отклонения от среднего }

\textbf{IV оценка within} --- это IV или 2SLS или ОММ оценка \textbf{модели within} или \textbf{модели отклонения от среднего}
\begin{align}
& y_{it}-\bar{y}_{i}=(\x_{it}-\bar{\x}_{i})'\be + (\e_{it}- \bar{\e}_{i}).
\label{Eq:22.26}
\end{align}
Тогда $\E[\mathbf z_{is} \e_{it}]=\mathbf 0$ для $s \leq t$ больше не подразумевает, что $\E[\mathbf z_{is} (\e_{it}-\bar{\e}_i)]=\mathbf 0$ даже для $s$ значительного меньше, чем $t$. Продемонстрируем это. Предположим, что $\E[\mathbf z_{is} \e_{it}] \neq \mathbf 0$ для $s > t$. Тогда $\E[\mathbf z_{is} \bar{\e}_i] \neq \mathbf 0$ для всех $s$, так как в $\bar{\e}_i=T^{-1}\sum \e_{it}$ содержатся значения $\e_{it}$ прошлых периодов, которые коррелируют с $\mathbf z_{is}$.

Поэтому IV оценивание модели within дает несостоятельную оценку $\be$, если инструменты слабо экзогенны или удовлетворяют более слабому предположению об одновременной экзогенности или условию в виде суммы. Если инструменты строго экзогенны, можно использовать только преобразование within.

{\centering  IV для модели форвардных ортогональных отклонений }

Ареллано и Бовер (1995) предложили альтернативный взятию первых разностей метод, для которого требуется только слабая, а не строгая экзогенность инструментов. Мы также продемонстрируем этот метод, хотя метод взятия первых разностей используется гораздо чаще.

Для модели \ref{Eq:22.2}  для $i$-го наблюдения, после преобразования в первые разности получается модель $\mathbf D \mathbf y_i=\mathbf D \mathbf X_i \be + \mathbf D \bm\e_i$, где $\mathbf D$ --- это матрица размерности $(T-1) \times T$ с элементами $\mathbf D_{st}$, $t=1, \dots, T-1, s=1, \dots, T$, равными минус единице, если  $s=t$, единице, если $s=t+1$, и нулю в противном случае. Если $\e_{it}$ независимы и одинаково распределены, преобразованная ошибка описывается процессом MA(1)  и $\mathrm{V}[\mathbf D \mathbf u_i]=\sigma^2 \mathbf D \mathbf D'$. ОМНК оценка домножает $\mathbf D \mathbf \e_i$ на $(\mathbf D \mathbf D')^{-1/2}$, или домножает $\bm\e_i$ на $(\mathbf D \mathbf D')^{-1/2} \mathbf D$. Получается преобразованная модель  вида \ref{Eq:22.24}, где тильда обозначает предварительное умножение на $(\mathbf D \mathbf D')^{-1/2} \mathbf D$.

Если для получения $(\mathbf D \mathbf D')^{-1/2}$ была использована факторизация верхне-треугольной матрицы Холецкого  \textbf{модель форвардных ортогональных отклонений}
\begin{align}
c_t(y_{it}-\bar{y}^F_{it})=c_t(\x_{it}-\bar{\x}_{it}^F)'\be + c_t (\e_{it} - \bar{\e}^F_{it})
\label{Eq:22.27}
\end{align}
(см. Ареллано, 2003, p.17), где $c^2_t=(T-t)/(T-t+1)$, и верхний индекс $F$ обозначает, что для формирования среднего используются только будущие значения. Например, $\bar{y}^F_{it}=(T-t)^{-1} \sum^T_{s=t+1} y_{is}$.

Такое преобразование называется \textbf{ортогональными отклонениями (orthogonal deviations)}, так как преобразованные ошибки $c_t(\e_{it}-\bar{\e}_i^F)$ не коррелированы между собой и имеют единичную дисперсию. Прилагательное \textbf{форвардные} добавлено, так как преобразованные ошибки зависят только от текущих и будущих значений первоначальных ошибок. МНК оценивание \ref{Eq:22.27} дает оценку within, представленную в главе 21. Поэтому преобразование ортогональных отклонений оптимально, если $\e_{it}$ независимы и одинаково распределены.

\textbf{IV оценка форвардных ортогональных отклонений} --- это IV или 2SLS или ОММ оценка модели \ref{Eq:22.27}. Для слабо экзогенных инструментов из $\E[\mathbf z_{is} \e_{it}]=\mathbf 0$ для $s \leq t$ следует, что $\E[\mathbf z_{is} (\e_{it}-\bar{\e}^F_i)]=\mathbf 0$ для $s \leq t$. Следовательно, форвардные ортогональные отклонения не приводят к потере  количества доступных инструментов. Обычно к инструментам преобразование не применяется, так как в $(\mathbf z_{it} - \bar{\mathbf z}^F_i)$ включены будущие значения $\mathbf z_{it}$, которые зачастую коррелированы с $\e_{it}$.

\subsection{IV для моделей со случайными эффектами}

Модель для $i$-го наблюдения 
\begin{align}
\mathbf y_{i}=\mathbf X_i \be + \mathbf e \alpha_i + \bm\e_i,
\nonumber
\end{align}
где $ \mathbf e$ --- это единичный вектор размерности $T \times 1$. Состоятельные, но неэффективные оценки могут быть получены с помощью прямого применения ОММ для панельных данных раздела 22.2. При этом используются инструменты $ \mathbf Z_i$, полученные с помощью использования ограничений исключения или подходящих условий об экзогенности, т.е. $\E[ \mathbf Z'_i( \mathbf e \alpha_i + \e_i)]= \mathbf 0$. Здесь мы заглянем несколько глубже и рассмотрим более эффективное оценивание, в котором, как и в главе 21, учитывается корреляция ошибок во времени, когда ошибки модели имеют вид $u_{it}=\alpha_i +\e_{it}$. 

{\centering  IV оценивание преобразованной модели }

Предположим, что инструменты $\mathbf Z_i$  удовлетворяют условию $\E[\mathbf u_i| \mathbf Z_i]=\mathbf 0$ и $\mathrm V[\mathbf u_i | \mathbf Z_i]=\bm\Omega_i$, где $\bm\Omega_i$ имеет ту же форму, что и в случае стандартной модели со случайными эффектами, где на диагонали стоят $\sigma^2_{\alpha}+\sigma^2_{\e}$ и вне диагонали - $\sigma^2_{\alpha}$. Заметим, что это более строгое предположение, чем $\E[\mathbf Z'_i \mathbf u_i]=\mathbf 0$. Поэтому наложим ограничения на доступные инструменты.

При условном моментном ограничении $\E[\mathbf u_i| \mathbf Z_i]=\mathbf 0$, из раздела 6.3.7 оптимальное безусловное моментное ограничение имеет вид
\begin{align}
\E[\mathbf Z'_i \bm{\Omega}^{-1}_i \mathbf u_i]= \E[(\bm{\Omega}^{-1/2}_i \mathbf Z_i)'(\bm{\Omega}^{-1/2}_i \mathbf u_i)]=\mathbf 0.
\nonumber
\end{align}
Это приводит к ОММ оцениванию преобразованной системы $\mathbf y^*_i=\mathbf X^*_i\be+\mathbf u^*_i$ с преобразованными инструментами $\mathbf Z^*_i$. Звездочка обозначает умножение на матрицу $\bm\Omega^{-1/2}_i$ размерности $T \times T$ или ее состоятельную оценку $\hat{\bm\Omega}^{-1/2}_i$.

Из раздела 21.7.1 умножение на  $\hat{\bm\Omega}^{-1/2}_i$ приведет к модели
\begin{align}
y_{it}-\hat{\la}\bar{y}_i=(\x_{it}-\hat{\la}\bar{\x}_i)'\be+\{(1-\hat{\la})\alpha_i+(\e_{it}-\hat{\la}\bar{\e}_i)\},
\label{Eq:22.28}
\end{align}
где $\hat{\la}$ --- это состоятельная оценка $\la=1-\sigma_{\e}/\sqrt{\sigma^2_{\e}+T\sigma^2_{\alpha}}$. \textbf{IV оценка со случайным эффектом} --- это IV или 2SLS оценка этой модели с преобразованными инструментами $\tilde{\mathbf z}_{it}=(\mathbf z_{it} - \hat{\la} \bar{\mathbf z}_i$), или, что эквивалентно, с инструментами $\mathbf z_{it} - \bar{\mathbf z}_i$ и $\bar{\mathbf z}_i$.

Этот метод требует состоятельной оценки $\la$. Для $\sigma^2_{\e}$ мы используем $\hat{\sigma}^2_{\e}=\sum_i \tilde{\e}^2_{it}/N(T-1)$, где $\tilde{\e}_{it}$ --- остатки IV регрессии within $y_{it}-\bar{y}_i$ на $(\x_{it}-\bar{\x}_i)$ c инструментами $(\mathbf z_{it} - \bar{\mathbf z}_i)$ (см. \ref{Eq:22.26}). Также $(\sigma^2_{\e}+T\sigma^2_{\alpha})$ может быть оценено с помощью $\sum_i \bar{u}^2_i/N$, где $\bar{u}_i$ --- это остатки IV регрессии between $\bar{y}_i$ на $\bar{\x}_i$ с инструментами $\bar{\mathbf z}_i$. Результирующая IV оценка $\be$ называется \textbf{2SLS оценка с составной ошибкой} (Error Components 2SLS) (см. Бальтаджи, 1981).

Эти результаты зависят от спецификации определенной функциональной формы $\bm\Omega_i$. Результат в разделе 22.2.2 позволяет делать статистические выводы, которые робастны к неверной спецификации $\bm\Omega_i$, с использованием \ref{Eq:22.5}, где $\mathbf y, \mathbf X, \mathbf Z$ и $\mathbf W_N=[\mathbf Z' \mathbf Z]^{-1}$ заменяются на преобразованные переменные в \ref{Eq:22.28}.

Более важное ограничение состоит в том, что этот метод может быть использован, только если первоначальные инструменты строго экзогенны. Здесь для состоятельности необходимо выполнение предположения $\E[\mathbf Z'_i \bm\Omega^{-1}_i \mathbf u_i]=\mathbf 0$, более сильного предположения, чем $\E[\mathbf Z'_i \mathbf u_i]=\mathbf 0$, которое неизбежно требует $\E[\mathbf u_i| \mathbf Z_i]= \mathbf 0$. Например, предположим $\E[\mathbf z_{it} \alpha_i]=\mathbf 0$ для всех $t$, в то время как $\E[\mathbf z_{is} \e_{it}]=0$ для $ s \leq t$, а $\E[ \mathbf z_{it} \e_{it}] \neq \mathbf 0$ для $s > t$. Тогда $\E[\mathbf z_{it} \e_{it}] \neq \mathbf 0$, что приведет к корреляции инструментов с ошибкой в \ref{Eq:22.28}.

\subsection{IV для гибридной модели Хаусмана-Тейлора}

Один из распространенных примеров эндогенности --- включение регрессоров, коррелированых с индифидуальными эффектами $\alpha$. Это приводит к несостоятельности оценки со случайным эффектом главы 21. Очевидное решение  --- вместо этого использовать состоятельную оценку within (или оценку с фиксированным эффектом). Однако в таком случае коэффициенты индивидуальных регрессоров, не меняющихся во времени, не могут быть идентифицированы. Это лишает многие исследования на панельных данных их основной цели --- оценивание эффектов не меняющихся во времени регрессоров, таких как эффект уровня образования  в регрессиях, описывающих уровень заработка.

Хаусман и Тейлор (1981) рассмотрели следующий вариант \ref{Eq:22.23}:
\begin{align}
y_{it}=\x'_{1it}\be_1+\x'_{2it}\be_2 + \mathbf w'_{1i}\gamma_1+ \mathbf w'_{2i} \gamma_2 + \alpha_i + \e_{it},
\label{Eq:22.29}
\end{align}
где некоторые регрессоры по предположению коррелируют с $\alpha_i$, а другие нет. $\mathbf w$ обозначает регрессоры, не меняющиеся во времени. А именно $\x_{1it}$ и $\mathbf w_{1i}$ не коррелированы c $\alpha_i$, и $\x_{2it}$ и $\mathbf w_{2i}$ коррелированы c $\alpha_i$. Предполагается, что все регрессоры некоррелированы с $\e_{it}$. В этой модели $\alpha_i$ можно рассматривать как своеобразный \textbf{гибрид} случайных и фиксированных эффектов.

Хаусман и Тейлор (1981) предложили два способа использования меняющихся во времени экзогенных регрессоров $\x_{1it}$: для оценивания $\be_1$  и в качестве инструментов для  $\mathbf w_{2i}$, позволяющих оценивание $\gamma$. Тогда $\gamma$ идентифицируемо, если количество меняющихся во времени экзогенных регрессоров равно или превышает количество неизменных во времени эндогенных регрессоров. Амэмия и МаКарди (1986) предложили более эффективную оценку, которая использует $\x_{1it}$ $(T+1)$ способами: для оценивания $\be_1$ и в качестве $T$ инструментов для $\mathbf w_{2i}$, позволяющих идентификацию, если $\mathrm{dim}[\mathbf w_{2i}] \geq T \mathrm{dim}[\x_{1it}]$. Такой подход получения инструментов из значений экзогенных регрессоров не текущего периода подробно обсуждался в разделе 22.2.4.

Различные \textbf{проекции}, некоторые эквиваленты могут быть использованы для формирования подходящих инструментов. Бройш, Мизон и  Шмидт (1989) предложили более простой вид и проекцию, которая позволяет оценивание при использовании только пакета для двухшагового МНК (2SLS).

Для начала рассмотрим состоятельную, но неэффективную оценку, которая игнорирует корреляционную структуру $(\alpha_i+\e_{it})$. После преобразования within уничтожается корреляция с $\alpha_i$, благодаря чему $\ddot{\x}_{2it}=\x_{2it}-\bar{\x}_{2i}$ может быть использован как инструмент для эндогенного $\x_{2it}$. Инструмент для $\x_{1it}$ --- $\ddot{\x}_{1it}$, а не $\x_{1it}$. Тогда $\bar{\x}_{1i}$ используется как инструмент для эндогенного $\mathbf w_{2i}$, а экзогенный $\mathbf w_{1i}$ используется как инструмент для самого себя.

Сейчас рассмотрим эффективное оценивание в предположении случайных эффектов о том, что компоненты $\alpha_i$ и $\e_{it}$ гомоскедастичны. Тогда из \ref{Eq:22.27} после применения \textbf{преобразования взятия разностей со случайным эффектом} (см. \ref{Eq:22.28}) получается
\begin{align}
\tilde{y}_{it}=\tilde{\x}'_{1it}\be_1+\tilde{\x}'_{2it}\be_2 + \tilde{\mathbf w}'_{1i}\gamma_1+ \tilde{\mathbf w}'_{2i} \gamma_2  + v_{it},
\label{Eq:22.30}
\end{align}
где, например, $\tilde{\x}_{1it}=\tilde{\x}_{1it}-\hat{\la}\bar{\x}_{1i}$, где оценка для скалярной величины $\hat{\la}$  была представлена в конце предыдущего раздела. Оценка Хаусмана-Тейлора эквивалентна IV оцениванию \ref{Eq:22.30}  с использованием инструментов $\ddot{\x}_{1it}, \ddot{\x}_{2it}, \mathbf w_{1i}$ и $\bar{\x}_{1i}$. Экзогенные меняющиеся во времени регрессоры $\x_{1it}=\ddot{\x}_{1it}+\bar{\x}_{1i}$ дважды используются как инструменты, в которых преобразование within $\ddot{\x}_{1it}$ используется как инструмент для $\x_{1it}$ и среднее по времени $\bar{\x}_{1i}$ используется как инструмент для $\mathbf w_{2i}$. Оценка Амэмия и МаКарди (1986) в качестве инструментов использует $\ddot \x_{1it}, \ddot \x_{2it}, \mathbf w_{1i}$ и $\x_{1it}, \dots, \x_{1iT}$, так что в качестве инструментов используются значения во все значения временные периоды, а не только среднее по $t$. Для этого необходимо, чтобы  $\E[\x_{1it} \alpha_i] = \mathbf 0$ для $t=1, \dots, T$. Это более сильное предположение, чем  $\E[\bar{\x}_{1i} \alpha_i]=\mathbf 0$ (см. раздел 22.2.4). Бройш и др. (1989) предложили даже более эффективную оценку с использованием $\ddot{\x}_{2is}, s \neq t$ в качестве дополнительных инструментов.

Основное ограничение этого подхода состоит в том, что он требует разграничения регрессоров на коррелированные и некоррелированные с $\alpha_i$. В регрессии отдачи от образования Хаусман и Тейлор начинают с предположения, что все три меняющихся во времени регрессора (опыт работы, плохое здоровье и прошлогодний уровень безработицы) экзогенны, два не меняющихся регрессора (раса и семейное положение) экзогенны, и не меняющийся во времени регрессор (уровень образования) эндогенный. В такой спецификации есть два сверх-идентифицирующих ограничения. Тестирование спецификации модели возможно с помощью теста Хаусмана, основанном на разнице между $\hat{\be}_{HT}$ и $\hat{\be}_W$, так как оценка within для $\be$, несмотря на то, какие компонент $\x_{it}$ и $\mathbf w_i$ коррелированы с $\alpha_i$. В своем эмпирическом исследовании Корнуолл и Руперт (1988) сравнивают различные оценки.

\subsection{Внешне не связанные уравнения и оценка одновременных уравнений}

В предшествующем анализе панельных данных внимание было сосредоточено исключительно на оценивании одного уравнения в отдельности. В некоторых случаях желательно оценивать систему уравнений, такую как систему уравнений спроса, где зависимые переменные и регрессоры наблюдаются для многих индивидуумов в разные периоды времени. Если нет ограничений на параметры, то оценивание одного уравнения дает состоятельные оценки.  Более эффективная оценка возможна благодаря совместному оцениванию уравнений, которая учитывает корреляцию между ошибками разных уравнений.

В рамках подхода со строго экзогенными регрессорами главы 21, более эффективная оценка --- это обобщение оценки внешне не связанных уравнений (seemingly unrelated regressions, SUR) от случая пространственных данных до панельных данных. В \textbf{модели внешне не связанных уравнений с составной ошибкой, error components SUR}  $g$-е из $G$ уравнений задается в виде
\begin{align}
& y_{git}=\x'_{git}\be+\alpha_{gi}+\e_{git},
&g=1, \dots, G,
\label{Eq:22.31}
\end{align}
где, как и в случае панельных данных, для $\alpha_{gi}$ выполняется независимость по $i$, $\e_{dit}$ --- независимость по $i$ и $t$, $\alpha_{gi}$ и $\e_{git}$ не зависят друг от друга. Однако компоненты ошибок могут коррелировать между собой, т.е. $\mathrm{Cov}[\alpha_{gi},\alpha_{hi}] \neq 0$ и $\mathrm{Cov}[\alpha_{git},\alpha_{hit}] \neq 0$ для $g \neq h$. Тогда методы главы 21 дают состоятельные оценки. Очевидная оценка --- это оценка со случайным эффектом, т.е. доступный ОМНК, учитывающий корреляцию внутри каждого уравнения. Более эффективные ОМНК оценки, которые дополнительно учитывают корреляцию  ошибок между уравнениями, подробно описаны у Авери (1997) и Бальтаджи (1980).

Подобное увеличение эффективности возможно в случае, когда одно из \textbf{одновременных уравнений}, где регрессор $\x_{git}$ в \ref{Eq:22.31} может включать один или более эндогенных регрессоров $y_{hit}$  из других уравнений. Тогда IV или ОММ каждого уравнения в отдельности дает состоятельные оценки. Очевидной оценкой при составной структуре ошибок будет IV или EC2SLS оценка раздела 22.4.3. Более эффективные оценки получаются при использовании \textbf{оценки трехшагового МНК с составной ошибкой} (Error components 3SLS, EC3SLS), предложенной Бальтаджи (1981).

Оценки систем уравнений получить сложнее, поэтому более адекватным может быть отдельное оценивание каждого уравнения. Но даже при таком подходе к оцениванию, выгодно специфицировать систему одновременных уравнений, так как это позволяет идентификацию коэффициентов эндогенных регрессоров с использованием экзогенных регрессоров, исключенных из основного уравнения, в качестве инструментов.  Это более традиционный подход получения инструментов, чем использование экзогенных регрессоров остальных периодов в качестве инструментов.


\section{Динамические модели}

В этом разделе мы будем рассматривать обычные модели панельных данных с индивидуальными эффектами с тем усложнением, что регрессоры будут включать первые лаги зависимых переменных. Тогда мы получаем \textbf{динамическую модель}
\begin{align}
& y_{it}=\gamma y_{i,t-1} +\x'_{it} \be +\alpha_i+\e_{it},
& i=1, \dots, N &
&t=1, \dots, T.
\label{Eq:22.32}
\end{align}
Как обычно панель является короткой, а данные  независимы по $i$. Предполагается, что $|\gamma| < 1$. Это предположение ослабляется в разделе 22.5.4.

Важным результатом является то, что даже если $\alpha_i$  --- это случайный эффект, МНК оценивание \ref{Eq:22.32} приводит к несостоятельной оценке $\gamma$  и $\be$. Это объясняется тем, что регрессор $y_{i,t-1}$ коррелирован с $\alpha_i$, а поэтому и с составной ошибкой $(\alpha_i+\e_{it})$. Даже в случае со случайными эффектами необходимы альтернативные оценки.

Мы будем рассматривать оценивание, когда $\alpha_i$ --- это фиксированный эффект, $|\gamma| <1$, ошибка $\e_{it}$ не коррелирована во времени, а панель является короткой (см. раздел 22.5.3). Хотя это базовый случай для микроэконометрических приложений, существует обширный объем литературы, в которой меняется одно или более из этих предположений. Индивидуальные эффекты могут быть чисто случайными, ошибки могут быть коррелированы во времени, данные могут быть нестационарными. Кроме того, панель может быть длинной, но мы всего лишь частично затронем эту литературу. 

\subsection{Зависимость от состояния и ненаблюдаемая гетерогенность}

Прежде чем рассматривать оценивание, заметим, что корреляция $y_{it}$ во времени теперь напрямую вызывается $y_{i,t-1}$ в дополнении к косвенным эффектам через $\alpha_i$, которые уже рассматривалось в главе 21. Эти две причины приводят к довольно  разным интерпретациям \textbf{корреляции во времени}, например, в индивидуальной заработной плате или восприятии богатства.

Для простоты пусть $\be=\mathbf 0$, так что $y_{it}=\gamma y_{i,t-1}+\alpha_i+\e_{it}$. Тогда $\E[y_{it}|y_{i,t-1}, \alpha_i]=\gamma y_{i,t-1} + \alpha_i$  и $Cor[y_{it}, y_{i,t-1}|\alpha_i]=\gamma$. При фиксированных $\alpha_i$ применимы стандартные результаты для модели AR(1) с зависимостью по времени в $y_{it}$, определяемой только авторегрессионным параметром $\gamma$. Однако $\alpha_i$  неизвестно и мы на самом деле наблюдаем $\E[y_{it}|y_{i,t-1}]=\gamma y_{i,t-1} + \E[\alpha_i|y_{i,t-1}]$  и $Cor[y_{it},y_{i,t-1}]= \neq \gamma$. Из \ref{Eq:22.32} с $\be=\mathbf 0$
\begin{align}
\mathrm{Cor}[y_{it}, y_{i,t-1}]
&=\mathrm{Cor}[\gamma y_{i,t-1}+\alpha_i+\e_{it},y_{i,t-1}] \nonumber \\
&=\gamma+ \mathrm{Cor}[\alpha_i,y_{i,t-1}] \nonumber \\
&=\gamma+ \frac{(1-\gamma)}{1+(1-\gamma)\sigma^2_{\e}/(1+\gamma)\sigma^2_{\alpha}}, \nonumber \\
\label{Eq:22.33}
\end{align},
где второе равенство предполагает $\mathrm{Cor}[\e_{it},y_{i,t-1}]=\mathrm 0$, и третье равенство получается после некоторых алгебраических преобразований для частного  случая случайных эффектов при независимых и одинаково распределенных $\e_{it}$ с параметрами $[0, \sigma^2_\e]$ и независимых и одинаково распределенных $\alpha_i$ с параметрами $[0, \sigma^2_{\alpha}]$.

Из результата \ref{Eq:22.33} становится понятно, что есть две возможные причины корреляции $y_{it}$ и $y_{it-1}$.

\textbf{Зависимость от состояния, True state dependence} имеет место тогда, когда корреляция во времени объясняется обычным механизмом: $y_{i,t-1}$ последнего периода определяет $y_{it}$ текущего периода. Эта зависимость относительно велика, если индивидуальный эффект $\alpha_i \simeq 0$, так как в таком случае $Cor[y_{it}, y_{i,t-1}] \simeq \gamma$. Как правило, эту случается когда $\sigma^2_{\alpha}$ очень мала по сравнению с $\sigma^2_{\e}$. 

Корреляция, обусловленная \textbf{ненаблюдаемой гетерогенностью}, имеет место быть даже в отсутствие вышеописанного механизма, т.е. если  $\gamma=0$. Но тем не менее корреляция все же присутствует, так как $\mathrm Cor[y_{it},y_{i,t-1}]$ упрощается до $\sigma^2_{\alpha}/(\sigma^2_{\alpha}+\sigma^2_{\e})$,  если $\gamma=0$, как в главе 21.

Обе крайности допускают то, что корреляция будет близка к единице, так как либо $\gamma \rightarrow 1$, либо $\sigma^2_{\e}/\sigma^2_{\alpha}$. Однако этим двум случаям соответствуют два совершенно разных объяснения с различными выводами для принятия решений. Объяснение зависимости от сложившегося состояния, что доходы $y_{it}$ постоянно высоки даже при учете регрессоров $x_{it}$, состоит в том, что будущие доходы определяются прошлыми и $\gamma$ большое. Ненаблюдаемая гетерогенность объясняется тем, что в действительности $\gamma$ маленькое, но некоторые существенно важные переменные были пропущены, что привело к высокому значению $\alpha_i$ в каждом периоде. Для данных длительности состояний различие между зависимости от состояния и ненаблюдаемой гетерогенностью было рассмотрено в главе 18. В статических линейных моделях панельных данных главы 21 принималась во внимание только ненаблюдаемая  гетерогенность.

\subsection{Несостоятельность стандартных оценок панельных данных}

Оценки из предыдущей главы \textbf{состоятельны}, если в регрессоры включены лаги зависимых переменных, даже в случае модели со случайными эффектами. Рассмотрим оценивание модели \ref{Eq:22.32}, где обычно предполагается, что $\e_{it}$ некоррелированы во времени.

Во-первых, рассмотрим \textbf{МНК оценивание} $y_{it}$  на $y_{i,t-1}$ и $\x_{it}$. Ошибка имеет вид $(\alpha_i+\e_{it})$. Она коррелирована с регрессором $y_{i,t-1}$, так как  $y_{i,t-1}=\gamma y_{i,t-2}+\x'_{i,t-1}\be+\alpha_i+\e_{i,t-1}$, вследствие чего $y_{i,t-1}$ коррелирована с $\alpha_i$. Заметим, что это отличается от предыдущего результата для МНК оценивания модели со случайными эффектами, не включающих лаги зависимых переменных. В том случае МНК регрессия $y_{it}$ на $\x_{it}$ дает состоятельные, хотя и неэффективные, оценки. Это также отличается от обычного результата, что МНК регрессия $y_{it}$  на $y_{i,t-1}$ дает состоятельные оценки (хотя и слегка смещенные на малых выборках), если ошибка не коррелирована во времени.

Во-вторых, рассмотрим \textbf{оценку within}, которая получается из оценивания регрессии $(y_{it}-\bar{y}_i)$ на $(y_{i,t-1}-\bar{y}_{i,t-1})$ и $(\x_{it}-\bar{\x}_i)$. В этой регрессии ошибка имеет вид $(\e_{it}-\bar{\e}_i)$. Сейчас по \ref{Eq:22.32} $y_{it}$ коррелирован с $\e_{it}$,   $y_{i,t-1}$ коррелирован с $\e_{i,t-1}$, а следовательно и с $\bar{\e}_i$. Однако из этого следует, что регрессор $(y_{i,t-1}-\bar{y}_i)$  коррелирован с ошибкой $(\e_{it}-\bar{\e}_i)$. Тогда МНК оценивание модели within приводит к несостоятельным оценкам параметров, так как регрессор коррелирован с ошибкой. Для состоятельности нужно, чтобы $\bar{\e}_i$ была очень мала по сравнению с $\e_{it}$. Для этого необходимо, чтобы $T \rightarrow \infty$, т.е. чтобы панель была длинная, а не короткая. Это рассматривает Никелл (1981).

 \textbf{Оценка со случайным эффектом}, данная в главе 21, тоже может быть несостоятельна, так как это линейная комбинация оценок within и between. Для моделей со случайными эффектами Андерсен и Хсяо (1981) рассматривали ММП оценивание, когда $\e_{it} \thicksim \mathcal N[0. \sigma^2]$; см. также Бхаргава и Сарган (1981). В коротких панелях распределение оценки ММП зависит от предположений относительно $y_{i0}$, первоначального значения зависимой переменной. Андерсен и Хсяо (1981) различали несколько предположений относительно \textbf{начальных условий}: (1) фиксированные начальные наблюдения, (2) случайные начальные наблюдения с общим средним, (3) случайные начальные наблюдения с разными средними, и (4) случайные начальные наблюдения со стационарными распределениями.

\textbf{МНК оценка в первых разностях} также несостоятельна. Состоятельные оценки могут быть получены с помощью IV подхода. Сейчас приступим к описанию этой оценки.

\subsection{Оценка Ареллано-Бонда}
Модель \ref{Eq:22.32} приводит к модели в первых разностях
\begin{align}
& y_{it}-y_{i,t-1}=\gamma (y_{i,t-1} - y_{i,t-2})+(\x'_{it} -\x_{i,t-1})'\be + (\e_{it}-\e_{i,t-1}),
&t=2, \dots, T.
\label{Eq:22.34}
\end{align}
МНК оценка несостоятельна, так как $y_{i,t-1}$ коррелирован с $\e_{i,t-1}$ из \ref{Eq:22.32}, поэтому регрессор $(y_{i,t-1}-y_{i,t-2})$ коррелирован с ошибкой $(\e_{it}-\e_{i,t-1})$  в \ref{Eq:22.34}.

Андерсен и Хсяо (1981) предложили оценивать \ref{Eq:22.34}, используя \textbf{IV оценку}  с инструментом $y_{i,t-2}$ для $(y_{i,t-1}-y_{i,t-2})$. Этот инструмент является годным, так как $y_{i,t-2}$ не коррелирован с $(\e_{it}-\e_{i,t-1})$ в предположении некоррелированности ошибок $\e_{it}$ во времени. Более того, $y_{i,t-2}$ --- хороший инструмент, так как он коррелирован с $(y_{i,t-1}-y_{i,t-2})$. Метод требует наличия хотя бы трех временных периодов данных для каждого индивидуума. Можно использовать $\Delta y_{i,t-2}$ как инструмент для $\Delta y_{i,t-1}$, для чего требуется как минимум четыре периода для одного наблюдения. Андерсон и Хсяо (1981) сделали вывод, что в обычном случае ($\gamma > 0$) IV оценка более эффективна, если в качестве инструмента используется $\Delta y_{i,t-2}$, а не $y_{i,t-2}$. В другом случае $(\x_{it}-\x_{i,t-1})$ используется как инструмент для самого себя.

\textbf{Более эффективное оценивание} возможно благодаря использованию дополнительных лагов зависимых переменных в качестве инструментов. Например, и $y_{i,t-2}$, и $y_{i,t-3}$ можно использовать как инструменты. Тогда модель сверх-идентифицируема, и оценивание должно проводиться с помощью двухшагового МНК или ОММ для панельных данных. Более того, количество доступных инструментов для зависимой переменной будет наибольшим в момент $t$, который наиболее близок к $T$. В момент времени 3 в качестве инструмента доступно только $y_{i1}$, в момент времени 4 доступны  $y_{i1}$ и  $y_{i2}$, в 5 ---  $y_{i1}$,  $y_{i2}$
 и  $y_{i3}$ и т.д. Хольтц-Экин и др. (1988), а также  Ареллано и Бонд (1991) предложили ОММ оценки с использованием более широких несбалансированных наборов инструментов.

В литературе по микроэконометрике результирующая ОММ оценка для панельных данных называется \textbf{оценкой Ареллано-Бонда}. Общая процедура уже была представлена в разделе 22.4.2, где динамический аспект не был описан явно. Оценка
\begin{align}
\hat{\be}_{AB}=\left[ \left( \sum^N_{i=1} \tilde{\mathbf X}'_i \mathbf Z_i \right) \mathbf W_N
\left( \sum^N_{i=1} \mathbf Z_i  \tilde{\mathbf X}'_i \right) \right]^{-1}
\left( \sum^N_{i=1} \tilde{\mathbf X}'_i \mathbf Z_i \right) \mathbf W_N
\left( \sum^N_{i=1} \mathbf Z_i  \tilde{\mathbf y}'_i \right)
\label{Eq:22.35}
\end{align}
где $\tilde{\mathbf X}_i$ --- это матрица размерности $(T-2) \times (K +1)$ с $t$-ой строкой $(\Delta y_{i,t-1}, \Delta \x'_{it}$, $t=3, \dots, T, \tilde{\mathbf y}_i$ --- это вектор размерности $(T-2) \times 1$ с $t$-ой строкой $\Delta y_{it}$, и $\mathbf Z_i$  --- матрица инструментов размерности $(T-2) \times r$
\begin{align}
\mathbf Z_i=
\begin{bmatrix}
\mathbf z'_{i3} & \mathbf 0 & \dots & \mathbf 0 \\
\mathbf 0 & \mathbf z'_{i4} & & \vdots \\
\vdots & & \ddots & \mathbf 0 \\
\mathbf 0 & \dots & \mathbf 0 & \mathbf z'_{iT} 
\end{bmatrix},
\label{Eq:22.36}
\end{align}
где $\mathbf z'_{it} = [y_{i,t-2}, y_{i,t-3}, \dots, y_{1i}, \Delta \x'_{it}]$. Лаги $\x_{it}$ или $\Delta \x_{it})$ могут быть дополнительно использованы как инструменты, и для средних или больших $T$ максимальный лаг $y_{it}$, используемый в качестве инструмента, может не превышать $y_{i,t-4}$. Двухшаговый МНК и двухшаговый ОММ соответствуют разным взвешивающим матрицам $\mathbf W_N$ (см. раздел 22.2.3).

Метод достаточно просто может быть применен к модели AR(p), с $\gamma y_{i,t-1}$ в \ref{Eq:22.32}, замененным на $\gamma_1 y_{i,t-1}+\gamma_2 y_{i,t-2}+ \dots + \gamma_p y_{i,t-p}$, хотя для состоятельной оценки нужно более, чем три периода данных.

Пример в разделе 22.3 --- это главным образом  пример  оценивания Ареллано-Бонда, так как  к модели в первых разностях применяется IV оценивание с лаговыми значениями регрессоров в качестве инструментов.

Ан и Шмидт (1995) заметили, что возможно еще более эффективное оценивание с использованием дополнительных моментных условий. Рассмотрим версию \ref{Eq:22.32}, где $\be= \mathbf 0$ и сделаем стандартное предположение, что $\e_{it}$ некоррелировано с $\alpha_i$, $\e_{is}$ для $s \neq t$ и наблюдения $y_{i1}$. Оценка Ареллано-Бонда  использует моментное условие $\E[y_{is} \Delta u_{it}]=0$ для $s \leq t-2$, где  $u_{it}=\e_{it}+\alpha_i$. Ан и Шмидт (1995) получили более эффективную оценку, используя дополнительно моментные условия $\E[u_{iT}\Delta u_{it}]=0$. Они показали, что эта оценка эффективно использует предположения о моментах второго порядка и асимптотически эквивалентна оптимальной оценке  минимального расстояния Чемберлина (1982, 1984).

Дополнительные предположения позволяют использовать дополнительные моментные условия, и поэтому получать более эффективные оценки. Если $\mathrm V[\e_{it}]=\mathrm V[\e_{is}]$, то в предположении о гомоскедастичности $\e_{it}$ $\E[\bar{u}_i \Delta u_{it}]=0$ (см. Ан и Шмидт, 1995). Ареллано и Бовер (1995) предложили использовать условие $\E[u_{it} \Delta y_{is}]=0$  для $s \leq t-1$. Бланделл и Бонд (1998) рассматривают эти и другие предположения и показывают, что выигрыш может быть значительным, особенно когда $\gamma$ принимает высокое значение, а $T$ --- маленькое. Ареллано и Оноре (2001) представляют множество возможных предположений и соответствующие им моментные условия, которые могут быть использованы в оценивании.

Хсяо, Песаран, и Тамишоглу (2002) предлагают \textbf{преобразованную ММП оценку}. Предположим, что $\e_{it}$ независимы и имеют нормальное распределение $\mathcal N [0, \sigma^2]$. Это предположение может быть ослаблено. Вместо того, чтобы формировать функцию максимального правдоподобия на основе $\e_{i1}, \dots, \e_{iT}$, они основывают функцию правдоподобия на разностях ошибок $\Delta \e_{i1}, \dots, \Delta \e_{iT}$. Для модели временного ряда AR(1) $\Delta \e_{it}=\Delta y_{it}  - \gamma \Delta y_{i,t-1}$ для $t >1$. Плотность $\Delta \e_{i1}$ зависит от предположений относительно первоначальных условий: или $\Delta \e_{i1} = \Delta y_{i1}$, или $\Delta \e_{i1} = \Delta y_{i1}-b$, где $b=\E[\Delta y_{i1}]$ --- дополнительный параметр для оценки. Результирующая оценка --- квази-ММП оценка, которая обеспечивает состоятельность, даже когда $\e_{it}$ не имеет нормального распределения. Если  $\e_{it}$ независимы и одинаково распределены с параметрами $[0,\sigma^2]$, то преобразованная оценка ММП более эффективна, чем предшествующие ОММ оценки.


\subsection{Оценивание ковариационных структур}

\textbf{Ковариационные структуры} --- это модели, которые специфицируют структуру ковариационной матрицы ошибки регрессии. Приложения включают структуры для динамики ошибок и для ошибки измерения. Цель --- оценить параметры структуры.

Например, предположим, что процесс, генерирующий $y_{it}$, является моделью со случайными эффектами с ошибками вида MA(1), т.е.
\begin{align}
y_{it}=\alpha_i+\e_{it}+\phi \e_{i,t-1},
\nonumber
\end{align}
где $\alpha_i \thicksim [0, \sigma^2_{\alpha}]$ и $\e_{it} \thicksim [0, \sigma^2_\e]$ и $|\phi| <1$. Тогда автокорреляции $\gamma_j=\mathrm{Cov}[y_{it}, y_{i,t-j}]$ удовлетворяют $\gamma_0=\sigma^2_{\alpha}+(1+\phi^2)\sigma^2_\e, \gamma_1=\sigma^2_{\alpha}+\phi \sigma^2_{\e}$ и $\gamma_j=\sigma^2_{\alpha}$ для $j \geq 2$. Если $T=3$, то из этих уравнений получаются оценки $\hat{\sigma}^2_{\alpha}$,  $\hat{\sigma}^2_{\e}$, и $\hat{\phi}$ при автоковариациях $\hat{\gamma}_0$, $\hat{\gamma}_1$ и $\hat{\gamma}_2$. Если $T>3$, то модель сверх-идентифицируема и остается только три параметра для оценки, но более трех оценок ковариаций. Самым очевидным будет использование оценки минимального расстояния.

Пусть $\bm\theta$ обозначает $q$  структурных параметров, и предположим, что $\mathbf{g}(\bm\theta)=\bm\gamma$, где $\gamma=[\gamma_0, \dots, \gamma_{T-1}]'$ --- это вектор $T \geq q$ автоковариаций. Тогда \textbf{оценка минимального расстояния} $\hat{\bm\theta}_{MD}$ минимизирует
\begin{align}
Q_N(\bm\theta)=(\hat{\bm\gamma}-\mathbf{g}(\bm\theta))'\mathbf{W}_N (\hat{\bm\gamma}-\mathbf{g}(\bm\theta)),
\label{Eq:22.37}
\end{align}
где $\hat{\gamma}=[\hat{\gamma}_1, \dots, \hat{\gamma}_{T-1}]'$,
\begin{align}
\hat{\gamma}_j=[N(T-j)]^{-1} \sum^T_{t=j+1} \sum^N_{i=1} (y_{it}-\bar{y}_t)(y_{i,t-j}-\bar{y}_{t-j}),
\label{Eq:22.38}
\end{align}
и $\bar{y}_{t-j}=N^{-1} \sum_i y_{i,t-j}$. Взвешивающая матрица $\mathbf{W}_N$ и другие подробности  оценивания методом минимального расстояния содержатся в разделе 6.7. Ограничения модели могут быть проверены с помощью тестовой статистики $\chi^2$, данной в разделе 6.7. Таким образом мы наложили  ограничение слабой стационарности. В более общем виде
$\gamma_{tj} \neq \gamma_{sj}$ для $t \neq s$, где  $\gamma_{tj}=\mathrm{Cov}[y_{it},y_{i,t-1}]$. Тогда $\bm\gamma$ содержит $T(T+1)/2$ компонент $\gamma_{tj}, t= j+1, \dots, T$ и $j=0, \dots, T-1$. Предположение о стационарности  можно протестировать. Более того, регрессоры могут быть включены заменой $y_{it}$ на остатки $y_{it}-\x'_{it} \be$.

Абоуд и Кард (1989) одними из первых применили этот подход к совместному моделированию заработной платы и количества часов работы. Альтоньи и Сегал (1996) продемонстрировали, что оптимальная оценка минимального расстояния может быть достаточно смещенной в ограниченных выборках (см. раздел 6.3.5). С помощью этой оценки в основном моделируются заработные платы; см. недавний пример Бейкер и Солон (2003).

Подход минимального расстояния, MD,  хорошо подходит для оценивания ковариационных структур. Набор панельных данных может быть большим, но после оценивания ковариаций всё оценивание сводится к  минимизации \ref{Eq:22.37}. Другие подходы к оцениванию возможны. В частности см. МаКарди (1982b), где используются модели типа Бокса-Дженкинса для панельных данных. 

\subsection{Нестационарные панели}

В литературе, посвященной анализу единичных корней и нестационарности панельных данных, делается акцент на панели, в которых $N$ и $T$ велики. Одну из первых ключевых работ, посвященных \textbf{тестам на единичные корни}, написали Левин и Лин (1992) и позже опубликовали Левин, Лин и Чу (2002); Песаран и Смит (1999) одними из первых написали работу о \textbf{коинтеграции}. Филлипс и Мун (1999)  и Педрони (2004) приводят общую теорию статистических выводов в случае нестационарных панельных данных. Этот анализ довольно прост, в нем используется \textbf{последовательное взятие пределов}, где сначала фиксируется  $N$, а $T \rightarrow \infty$. После этого $N \rightarrow \infty$. Более робастный подход использует \textbf{совместные пределы}, когда  $T \rightarrow \infty$ и $ N \rightarrow \infty$ одновременно. Среди недавних обзоров литературы следует отметить работы Филлипса и Муна (2000) и Бальтаджи (2001, глава 12). 

В меньшей степени были рассмотрены нестационарные данные в \textbf{коротких панелях}. Харрис и Тцавалис (1999) применяют тест на единичные корни Левина и Лина (1992) в коротких панелях. Пусть $\hat{\gamma}$ обозначает оценку within $\gamma$ в модели с фиксированными эффектами вида AR(1) $y_{it}=\alpha_i+\gamma y_{i,t-1} + \e_{it}$, где $\e_{it} \thicksim \mathcal N [0, \sigma^2]$. Мы проверяем нулевую гипотезу о единичном корне, т.е. $\gamma=1$ и отсутствие свободного члена $\alpha_i=0$, которая соответствует случаю 2 во временных рядах у Гамильтона (1994, c. 490). При нулевой гипотезе статистика тестирования единичного корня

\begin{align}
\frac{\sqrt{N}(\hat{\gamma}-1+3/(T+1))}
{[3(17T^2-20T+17)]/[5(T-1)(T+1)^3]} \rightarrow \mathcal N [0,1],
\nonumber
\end{align}
если $N \rightarrow \infty$ для фиксированного $T$. При больших отрицательных значениях статистики гипотеза о наличии единичного корня отвергается. Левин и Лин (1992) предложили дополнительные тесты, как, например, для моделей с индивидуальными временными трендами.

Байндер, Хсяо и Песаран (2003) изучают оценивание динамических моделей панельных данных с фиксированными эффектами с наличием единичных корней и коинтеграцией. При наличии единичных корней оценка Ареллано-Бонда несостоятельна, хотя видоизмененные оценки Ана Шмидта (1995) и др., которые обсуждаются в разделе 22.5.3, дают состоятельные оценки. Байндер и др. (2003) предлагают оценки квази-ММП, которые более предпочтительны для коротких панелей при наличии единичных корней.

\section{Оценка разность разностей}

Литература, представленная в главе 25, фокусируется на измерении \textbf{эффекта воздействия}. В самом простом случае влияние или предельный эффект одного бинарного регрессора равен единице, если происходит какое-то изменение внешних условий, и равен нулю иначе.
Например, мы можем быть заинтересованы в измерении эффекта изменения политики (воздействие бинарного типа) на заработную плату. Изменение политики состоит в изменении налоговой ставки или благосостояния или доступа отдельных индивидуумов к обучению.

В этом разделе мы связываем один из методов главы 25 с методами анализа панельных данных. В частности, эффект воздействия может измеряться с помощью стандартных методов анализа панельных данных, если данные доступны до и после данного изменения, и если не все индивидуумы попадают под данное влияние. Тогда оценка в первых разностях для модели с фиксированными эффектами сводится к простой оценке, называемой оценкой <<разность разностей>>, представленной в разделе 3.4.2, и проанализированной в разделе 25.5. 
Преимущество последней оценки состоит в том, что она может применяться даже тогда, когда доступны повторные данные пространственного типа, а не панельные данные. Однако эта оценка зависит от предположений модели, которые зачастую явно не определены. Изложение этого раздела следует подходу Бланделла и МаКарди (2000).

\subsection{Фиксированные эффекты и бинарное воздействие}

Пусть бинарный регрессор, интересующий нас, будет иметь вид 
\begin{align}
D_{it}=
\begin{cases}
 1, \text{если индивид $i$ подвержен внешнему влиянию в момент $t$}, \\
0, \text{в противном случае}.
\end{cases}
\label{Eq:22.39}
\end{align}
Предположим, что модель содержит фиксированные эффекты для $y_{it}$ с 
\begin{align}
y_{it}=\phi D_{it} + \delta_t + \alpha_i + \e_{it},
\label{Eq:22.40}
\end{align}
где $\delta_t$ --- это временной фиксированный эффект, а $\alpha_i$ --- индивидуальный фиксированный эффект. Как замечено в разделе 21.2.1, это эквивалентно регрессии $y_{it}$ на $D_{it}$ и набор временных фиктивных переменных, только с индивидуальными фиксированными эффектами. Для простоты пусть в регрессии не содержится других регрессоров.

Индивидуальные эффекты $\alpha_i$  можно уничтожить взятием первых разностей. Тогда
\begin{align}
\Delta y_{it}=\phi \Delta D_{it} + (\delta_t - \delta_{t-1}) + \Delta \e_{it}.
\label{Eq:22.41}
\end{align}
$\phi$, эффект воздействия, может быть состоятельно оценен с помощью МНК сквозной регрессии $\Delta y_{it}$ на $\Delta D_{it}$ и временные дамми.

\subsection{Разность разностей}

Сейчас рассмотрим случай только двух периодов. Более того, предположим, что воздействие происходит в период 2, т.е. в период 1 $D_{i1}=0$ для всех индивидуумов и в периоде 2 $D_{i2}=1$ для всех, на кого было оказано воздействие (опытная группа), и $D_{i2}=0$ для тех, кто не был подвержен влиянию (контрольная группа). Тогда нижний индекс $t$ в \ref{Eq:21.41} можно отбросить  и 
 \begin{align}
\Delta y_{i}=\phi  D_{i} + \delta+ v_{i}.
\label{Eq:22.42}
\end{align}
где $D_i$ --- это бинарная переменная, обозначающая, был ли индивидуум подвержен воздействию.

Эффект воздействия может быть оценен МНК регрессией $\Delta y$ на свободный член и бинарный регрессор $D$. Пусть $\Delta\bar{y}^{tr}$ обозначает выборочное среднее $\Delta y_i$ для тех, кто был подвержен влиянию $(D_i=1)$, и  $\Delta \bar{y}^{nt}$ обозначает выборочное среднее $\Delta y_i$ для тех, кто не был подвержен влиянию $(D_i=0)$. Тогда МНК оценка будет иметь вид
 \begin{align}
\hat{\phi}=\Delta \bar{y}^{tr} - \Delta \bar{y}^{nt}.
\label{Eq:22.43}
\end{align}
Эта оценка называется \textbf{оценкой разность разностей, differences-in-differences, DID}, так как оцениваются разности для подверженных и не подверженных воздействию, а затем берется разность этих разностей.

Эта оценка привлекательна своей простотой. К тому же, она может быть применена не только к панельным данным, но и для данных пространственного типа, значения которых доступны для двух периодов во времени. Во втором периоде считаются средние $\bar{y}^{tr}_2$ и $\bar{y}^{nt}_2$ для двух групп (подверженных и не подверженных воздействию). Аналогично вычисляются средние для первого периода $\bar{y}^{tr}_1$ и $\bar{y}^{nt}_1$. Предполагается, что в первом периоде можно идентифицировать, подвержено ли индивидуальное наблюдение воздействию.  Это легко сделать, когда, к примеру, влияние оказывается только на женщин, и доступны данные о поле. Тогда вычисляется
 \begin{align}
\hat{\phi}=(\bar{y}^{tr}_2- \bar{y}^{tr}_1) - ( \bar{y}^{nt}_2 - \bar{y}^{nt}_1).
\label{Eq:22.44}
\end{align}

Например, если средняя ежегодная заработная плата для группы, которая будет подвержена воздействию, равна 10,000 до изменений и 13,000 после, тогда $\bar{y}^{tr}_2- \bar{y}^{tr}_1=3,000$. Аналогично, если средняя ежегодная заработная плата для группы, которая не будет подвержена воздействию, равна 15,000 до изменений и 17,000 после, то  $\bar{y}^{nt}_2 - \bar{y}^{nt}_1=2,000$. DID оценка эффекта $\hat{\phi}$ будет равна $3,000-2,000=1,000$.

\subsection{Предположения для DID оценки}

Предыдущая формулировка DID оценки выявляет необходимые предположения для состоятельного оценивания $\phi$.

Во-первых, предполагается, что временные эффекты $\delta_t$ характерны как для индивидуумов, подверженных воздействию, так и для тех, на кого изменения не будут оказывать влияния. Например, временной тренд может различаться в зависимости от пола. В этом случае идентификация $\phi$  будет проблематичной, если влияние зависит от пола. Предположение об общем тренде необходимо как в случае панельных, так и в случае пространственных данных.

Во-вторых, если используются данные пространственного типа, то структуры обеих групп постоянны до и после воздействия. В случае с панельными данными после взятия разностей уничтожаются фиксированные эффекты $\alpha_i$. С повторяющимися данными пространственного типа при первоначальной модели \ref{Eq:22.40} $\bar{y}^{tr}_t=\phi+\delta_t + \bar{\alpha}^{tr}_t+\bar{\e}^{tr}_t$ и $\bar{y}^{nt}_t=\delta_t+\bar{\alpha}^{nt}_t+\bar{\e}^{nt}_t$. Если изменение происходит только во втором периоде, то 
\begin{align}
\phi=(\bar{y}^{tr}_2- \bar{y}^{tr}_1) - ( \bar{y}^{nt}_2 - \bar{y}^{nt}_1)+ (\bar{\alpha}^{tr}_2- \bar{\alpha}^{tr}_1) - ( \bar{\alpha}^{nt}_2 - \bar{\alpha}^{nt}_1) + v,
\nonumber
\end{align}
где $v=(\e^{nt}_2 - \bar{\e}^{nt}_1)-(\bar{\e}^{nt}_2-\bar{\e}^{nt}_1)$. Оценка $\hat{\phi}$ будет состоятельной в \ref{Eq:22.44}, если $\mathrm{plim} (\bar{\alpha}^{tr}_2-\bar{\alpha}^{tr}_1)=0$ и $\mathrm{plim} (\bar{\alpha}^{nt}_2-\bar{\alpha}^{nt}_1)=0$. Это будет выполняться, если индивиды, подверженные воздействию, выбраны случайно. Однако это зачастую не выполняется на практике.


\subsection{Другие более сложные модели}

На практике используются более сложные модели. Очевидным является включение в модели других регрессоров помимо временных дамми и индикатора влияния. Индивидуальные эффекты могут по меньшей мере различаться в среднем по группам. Общий алгоритм оценивания состоит в оценке уравнения
\begin{align}
y_{igt}=\phi D_{igt} + \delta_t + \alpha_i + \e_{it},
\nonumber
\end{align}
где $g$ обозначает  $g$-ю группу.

В классическом примере DID оценивания, Кард (1990) изучал эффект внезапного приток иммигрантов из Кубы на безработицу среди рабочих с низкой заработной платой в Маями. Этот пример также рассматривают Ангрист и Крюгер (1999). Этей и Имбенс (2002) приводят обобщения для нелинейных моделей.

\section{Повторяющиеся пространственные данные и псевдо-панели}

Ключевое преимущество панельных данных возникает благодаря возможности наблюдения субъектов в разные периоды времени. Благодаря этому можно  учитывать ненаблюдаемую индивидуальную гетерогенность, разницу в первоначальных условиях, динамическую зависимость исходов. Во многих случаях, однако, подлинные панельные данные недоступны.

\subsection{Повторяющиеся пространственные данные}

Мы рассмотрим анализ, когда имеются данные для нескольких \textbf{повторяющихся пространственных выборок}. Данные взяты из ответов на серию независимых исследований, где независимость означает, что каждый субъект появляется только в одном исследовании. Пример такого исследования --- Исследование расходов английских домохозяйств, U.K. Family Expenditure Survey, в котором ежегодно собирается информация о расходах домохозяйств, но каждый год в исследовании принимают участие разные семьи. Также, если доступна только очень короткая панель (например, $T=2$), то  предпочтительными являются данные повторяющихся пространственных выборок, если они генерируют более большую и богатую выборку.

Для модели \textbf{со случайными эффектами} данные повторяющихся пространственных выборок не составляют особых трудностей. Необходимо просто оценить сквозную регрессию $y_{it}$  на $\x_{it}$ (см. раздел 21.5). Получение статистических выводов довольно простое, так как нужна только коррекция на гетероскедастичность вследствие того, что ошибки независимы по $i$ и $t$.

С фиксированными эффектами, однако, сквозная регрессия приводит к несостоятельным оценкам параметров. Более того, альтернативные методы, такие как оценивание within или в первых разностях недоступны, если индивидуальные наблюдения наблюдаются лишь однажды. В данном разделе будут использоваться данные пространственного типа для конструирования \textbf{псевдо-панелей} или \textbf{синтетических панельных данных}, которые имеют некоторые преимущества настоящих панельных данных (самое существенное из которых --- возможность учета фиксированных эффектов). Специальный случай --- это DID оценивание, представленное в разделе 22.6.

\subsection{Псевдо-панели}

Браунинг, Дитон, и Айриш (1985), а также Дитон (1985) в своих эмпирических исследованиях, основанных на U.K. Family Expenditure Survey, рассматривали методы для анализа повторяющихся пространственных данных. Их предложением было конвертировать индивидуальные данные в \textbf{когорты}. Хотя индивидуальные расходы домохозяйств не могут регулярно записываться на протяжении некоторого времени, то для когорт это возможно.

\textbf{Когорта} --- это <<группа постоянных членов, индивидуумы которой могут идентифицироваться по мере того, как они <<показываются>> в исследованиях>> (Дитон, 1985, c.109). Примером может служит возрастная когорта, такая как мужчины с годом рождения от 1965 до 1970. Для больших выборок, последующие исследования будут генерировать случайные выборки членов каждой когорты.

Временные ряды выборочных средних когорт могут формировать основу регрессионных моделей. Ключевой вопрос: могут ли служить синтетические панели, основанные на когортных данных, заменой настоящим панелям. Для таких моделей процедуры получения статистических выводов рассматриваются в рамках темы о повторяющихся пространственных данных. Здесь мы фокусируемся на статических моделях псевдо панелей. Колладо (1997) и Гирма (2000) тоже рассматривают динамический случай;

Отправной точкой является статическая линейная регрессия с индивидуальными эффектами $\alpha_i$, основанная на $T$ последующих пространственных выборках,
 \begin{align}
& y_{it}=\alpha_i+\x'_{it} \be + u_{it},
& t=1, \dots, T.
\label{Eq:22.45}
\end{align}
Предполагается, что объясняющие переменные строго экзогенны по отношению к исследуемым параметрам, $\be$, $\E[\x'_{it} u_{is}]=\mathbf 0$ для любых $t$ и $s$. Для простоты мы предполагаем, что в каждой кросс секции доступно $N$ наблюдений. Каждое индивидуальное наблюдение присутствует только в одном периоде, поэтому индивидуальные эффекты $\alpha_i$ не могут быть уничтожены с помощью взятия разностей по индивидуальным данным.

Пусть $g$  --- это случайная переменная, которая определяет членство в когорте для каждого $i$.  $i$ принадлежит кластеру $c$ тогда и только тогда, когда $g_i$ относится к набору $I_c$. Предположим, что имеется  $C$ когорт, и $c$ --- это индекс когорты, $c=1, \dots, C$. Взяв условное математическое ожидание, получаем
 \begin{align}
\E[y_{it}|g_i I_c]=\E[\alpha_i|g_i I_c]+\E[\x'_{it}|g_i I_c]\be+\E[u_{it}|g_i I_c].
\label{Eq:22.46}
\end{align}
Мы получаем когортную версия модели \ref{Eq:22.45}
 \begin{align}
y*_{ct}=\alpha*_c+\x*_{ct}'\be +u*_{ct},
\label{Eq:22.47}
\end{align}
где звездочка обозначает ненаблюдаемые средние по индивидам когортам. Например, $y_{ct}^*=\E[y_{it}|g_i \in I_c]$.

Параметр $\alpha^*_c=\E[\alpha_i|g_i \in I_c]$ является \textbf{фиксированным эффектом когорты}. Важное предположение, которое делается в случае фиксированных эффектов, состоит в том, что генеральная совокупность стационарна. Поэтому можно предполагать, что $\alpha^*_c$ постоянна во времени. Это предположение аналогично тому, что необходимо для состоятельности DID оценки, описанному в конце раздела 22.6.3. В условиях обычного предположения о слабой экзогенности $\E[u^*_{ct}|\x^*_{ct}]=0$. Однако ненаблюдаемый фиксированный эффект $\alpha^*_c$ будет коррелирован с $\x^*_{ct}$, если $\alpha_i$ коррелировано с $\x_{it}$ в первоначальной версии модели \ref{Eq:22.45}. Для оценивания необходимо учитывать фиксированные эффекты.

На практике средние по индивидам когорты ненаблюдаемы, вместо этого мы работаем со \textbf{средними когорт по времени} $\bar{y}_{ct}$ и $\bar{\x}_{c}$. Тогда регрессия имеет вид
 \begin{align}
& \bar{y}_{ct}=\bar{\alpha}_c+\bar{\x}'_{ct}\be+\bar{u}_{ct}
& c=1, \dots, C, &
& t=1, \dots, T.
\label{Eq:22.48}
\end{align}

Это шаг представляет нам дополнительный источник ошибки, так как оценки $\bar{y}_{ct}$ и $\bar{\x}_{ct}$ для средних по индивидам когорты загрязнены  ошибками, т.е.
 \begin{align}
& \bar{y}_{ct}=y^*_{ct}+\theta_{ct},
\label{Eq:22.49} \\
& \bar{\x}_{ct}=\x^*_{ct}+v_{ct}. \nonumber
\end{align}

Если \textbf{ошибка измерения} очень мала благодаря тому, что количество наблюдений в когорте для одного временного периода ($N_{ct}$) достаточно велико, то $\bar{y}_{ct} \simeq y^*_{ct}$ и $\bar{\x}_{ct}=\x^*_{ct}$  и ошибка измерения может быть проигнорирована. Состоятельная оценка $\be$ может быть получена с помощью оценивания within уравнения \ref{Eq:22.48}, т.е. МНК регрессии $(\bar{y}_{ct}-\bar{y}_c)$ на $(\bar{\x}_{ct}-\bar{\x}_c)$, где $\bar{y}_c=T^{-1} \sum_t \bar{y}_{ct}$ и $\bar{\x}_c=T^{-1}\sum_t \bar{\x}_{ct}$.

К сожалению, ошибка измерения часто слишком велика, чтобы ее игнорировать. Тогда within оценивание уравнения \ref{Eq:22.48}, или даже МНК оценивание \ref{Eq:22.48}, когда $\bar{\alpha}_c$  --- это случайный эффект, приводит к несостоятельной оценке $\be$. Вместо этого необходимо использовать оценки для случая ошибок измерения в переменных. Такие оценки могут здесь применяться, так как индивидуальные данные дают необходимые оценки моментов ошибок измерения, см. раздел 26.3.3.


\subsection{Оценки ошибок измерения для псевдо-панелей}

Классический способ решения проблемы ошибок измерения --- это использование повторных наблюдений для оценки ковариационной матрицы ошибок измерения, а затем использование этих оценок для <<корректирования>>  выборочных моментов переменных, содержащих ошибки, до применения процедуры МНК (см. раздел 26.3.4). Дитон (1985) предложил использование этого метода в текущей задаче.

Предположим, что индивидуальные наблюдения удовлетворяют уравнениям
 \begin{align}
y_{it}=y^*_{ct}+\e_{it} \nonumber \\
\x_{it}=\x^*_{ct}+\bm\eta_{it} \nonumber.
\end{align}
Cистема похожа на ту, что в разделе 26.2.1, за исключением того, что здесь также присутствует ошибка измерения в зависимой переменной. Предположим, что для любого индивидуума в данной когорте $c$ 
 \begin{align}
\begin{bmatrix}
\e_{it} \\
\bm\eta_{it}
\end{bmatrix}
\thicksim
\mathrm{iid}
\left[
\begin{bmatrix}
0, \\
\mathbf 0
\end{bmatrix}, 
\begin{bmatrix}
\sigma^2_0 & \sigma'_{01} \\
\bm\sigma_{01} & \bm\sum
\end{bmatrix}
\right].
\nonumber 
\end{align}
Выборочные оценки $(\sum, \bm\sigma_{01})$, обозначенные как $(\hat{\sum}, \bm\sigma_{01})$, могут быть получены при данных $(\bar{y}_{ct}, \bar{\x}_{ct})$ из индивидуальных данных. Определим $\mathbf d_c$ как столбец вектора фиктивных переменных размерности $C \times 1$, соответствующий фиксированным эффектам $\alpha^*_c$  (см. раздел 21.2.1). Этот вектор-регрессор, безусловно, не содержит ошибку измерения. Тогда, если $T$ достаточно большое, и существуют необходимые обратные матрицы, то регрессия
 \begin{align}
\begin{bmatrix}
\hat{\bar{\alpha}}_{ct} \\
\hat{\bar{\be}}_{ct}
\end{bmatrix}
=\left(
\sum^C_{c=1}
\sum^T_{t=1}
\begin{bmatrix}
\mathbf d'_c \mathbf d_c & \mathbf d'_c \bar{\x}_{ct} \\
\bar{\x}'_{ct} \mathbf d_c & \bar{\x}'_{ct} \bar{\x}_{ct}-\hat{\sum}
\end{bmatrix}
\right)^{-1}
\left[
\sum^C_{c=1}
\sum^T_{t=1}
\begin{pmatrix}
\mathbf d'_c \bar{y}_{ct} \\
\bar{\x}'_{ct} \mathbf d_c - \hat{\bm\sigma}_{01}
\end{pmatrix}
\right]
\label{Eq:22.50}
\end{align}
будет давать состоятельные оценки при $CT \rightarrow \infty$. Эта оценка та же, что дана в разделе 26.3.4, только с поправкой, так как $\bar{y}_{ct}$ тоже измеряется с ошибкой, и с упрощением, так как только подмножество регрессоров, $\hat{\bar{\x}}_{ct}$, измеряется с ошибкой. Вербик и Нейман (1992) приводят более подробное обсуждение выборочных свойств, и Дитон (1985) обсуждает оценивание дисперсии. См. также Вербик (1995).

Предыдущая оценка учитывает фиксированные эффекты путём оценивания модели с фиктивными переменными, делая поправку на ошибку измерения посредством использования повторных данных и оценки раздела 26.3.4.

Колладо (1997) рассматривал альтернативный подход для уничтожения эффектов когорт взятием первых разностей, а затем учетом ошибки измерения с помощью IV оценивания. Эта альтернативная стратегия идентификации ошибки измерения обсуждается в разделе 26.3.2.

Заменяя \ref{Eq:22.49} на \ref{Eq:22.47}, получаем 
 \begin{align}
\bar{y}_{ct}-\theta_{ct}=\alpha^*_c+(\bar{\x}'_{ct}-\mathbf v'-{ct}) \be+u*_{ct}, \nonumber \\
\bar{y}_{ct}=\alpha^*_c+\bar{\x}'_{ct}\be+w_{ct}
\nonumber,
\end{align}
где ошибка $w_{ct}=u^*_{ct}-\mathbf v'_{ct}\be+\theta_{ct}$. При взятии первых разностей уничтожаются $\alpha^*_c$, что приводит к 
 \begin{align}
& \Delta \bar{y}_{ct}=\Delta \bar{\x}'_{ct}\be+\Delta w_{ct},
&t=2, \dots, T.
\label{Eq:22.51}
\end{align}
Сейчас из-за ошибки измерения объясняющие переменные $\Delta \bar{\x}'_{ct}$ будут коррелированы с $\Delta w_{ct}$, и поэтому МНК будет давать несостоятельные оценки. Состоятельные оценки могут быть получены с помощью IV оценивания, основанного на лагах экзогенных переменных, т.е. $\bar{\x}'_{c,t-1}$. Этот подход легко обобщить на модели с  лаговыми зависимыми переменными. Подробности см. в Колладо (1997).

\section{Смешанные линейные модели}

В модели, называемой эконометристами моделью со случайными эффектами, только свободный член является случайным. В более общих моделях со случайными эффектами, которые широко используются в прикладной статистике, случайными могут быть и коэффициенты наклона. В этом разделе мы представляем смешанные линейные модели. Они также называются моделями смешанных эффектов, иерархическими или многоуровневыми линейными моделями (см. главу 24), моделями со случайными коэффициентами, и моделями с составной дисперсией.

Эти модели применяются в таких случаях, когда МНК оценка сквозной регрессии все еще состоятельна. В частности, нет фиксированных эффектов.  Предпосылки смешанных линейных моделей позволяют использовать доступный ОМНК для получения более эффективных оценок.

 \subsection{Смешанные линейные модели}

\textbf{Смешанная линейная модель} специфицирована следующим образом
 \begin{align}
y_{it}=\mathbf z'_{it} \be + \mathbf w'_{it} \bm\alpha_i + \e_{it},
\label{Eq:22.52}
\end{align}
где регрессоры $\mathbf z_{it}$ включают свободный член, $\mathbf w_{it}$  --- это вектор наблюдаемых характеристик, $\bm\alpha_i$  --- это случайный вектор с нулевым математическим ожиданием, и $\e_{it}$ обозначает ошибку. Эта модель называется \textbf{смешанной моделью}, так как в ней есть \textbf{фиксированные параметры} $\be$ и  \textbf{случайные параметры} с нулевым средним  или \textbf{случайные эффекты} $\alpha_i$.

Модель со случайным свободным членом $y_{it}=\mathbf z'_{it}\be+\alpha_i+\e_{it}$ --- это частный случай модели \ref{Eq:22.52}  с $\mathbf w'_{it} \bm\alpha_i=\alpha_i$.

Другой частный случай \ref{Eq:22.52} --- это \textbf{модель со случайными коэффициентами} или \textbf{модель со случайными параметрами}. В терминах регрессионной модели мы предполагаем, что
 \begin{align}
y_{it}=\mathbf z'_{it} \be + \e_{it}.
\nonumber
\end{align}
Это обычная линейная регрессия за исключением того, что компоненты вектора параметров регрессии сейчас различаются 
 \begin{align}
\be_i=\be+\bm\alpha_i
\nonumber
\end{align}
где $\bm\alpha_i$ --- случайный вектор с нулевым ожиданием. Подставляя это выражение в предыдущее, получаем $y_{it}=\mathbf z'_{it} \be + \mathbf z'_{it} \bm\alpha_i+\e_{it}$, что дает нам \ref{Eq:22.52} с $\mathbf w_{it}=\mathbf z_{it}$.

Множество моделей находятся где-то между моделью со случайным свободным членом и моделью со случайным коэффициентом, где $\mathbf w_{it}$ является подмножеством $\mathbf z_{it}$. В частности,  стандартные смешанные и случайные \textbf{ANOVA} модели также являются частным случаем, где $k$-й элемент вектора $\mathbf w_{it}$ равен нулю или единице, в соответствии с различными возможными моделями кластеризации данных. Например, один из элементов в $\mathbf z_{it}$ может быть переменная-индикатор расы или пола. Тогда условное среднее $y_{it}$ изменяется в зависимости от расы и пола. Может случиться, что условная дисперсия $y_{it}$ тоже меняется в зависимости от расы и пола и может быть учтена  в $\mathbf w_{it}$. Смешанная модель выросла из моделей ANOVA. \textbf{Иерархичечские линейные модели} или \textbf{многоуровневые линейные модели} (см. раздел 24.6.2) тоже могут быть представлены как частный случай модели \ref{Eq:22.52}. 

 \subsection{Оценивание}

Цель --- оценить фиксированные параметры регрессии $\be$, дисперсию и ковариации распределений для $\bm\alpha_i$ и $\e_{it}$. Одна из первых попыток оценивания этой модели была осуществлена в байесовском контексте  (Линдлей и Смит, 1972). Простым примером их подхода является модель со случайными коэффициентами с $y_{it} \thicksim \mathcal N [\mathbf z'_{it}\be_i, \sigma^2]$, где $\be_i \thicksim \mathcal N [\bm\gamma, \bm\Gamma]$. Подробнее \textbf{байесовский анализ} линейной модели панельных данных см., например,  у Купа (2003).

Здесь мы будем следовать \textbf{классическому подходу}, основанному на работе Харвилла (1977), который ссылается на более раннюю литературу. Смешанная модель \ref{Eq:22.52}  может быть разделена на детерминистическую компоненту $\x'_{it} \be$ и случайную компоненту $\mathbf w'_{it} \bm\alpha_i+\e_{it}$. Стохастические предположения включают в себя также предположение, что регрессоры $\x'_{it}$ независимы от случайных компонент $\bm\alpha_i$ и $\e_{it}$  с нулевым ожиданием. Поэтому сквозная МНК регрессия $y_{it}$ на $\x_{it}$  дает состоятельные оценки $\be$. Мы будем придерживаться рамок раздела 21.5, где для оценки можно  использовать доступный ОМНК, так как на ковариационную матрицу ошибки $\mathbf w'_{it} \bm\alpha_i+\e_{it}$ была наложена структура. В этом разделе мы представляем доступную ОМНК оценку наряду с двумя различными методами для оценки дисперсий и ковариаций $\bm\alpha$ и $\e_{it}$. Мы также рассмотрим предсказание случайной компоненты $\bm\alpha_i$.

Совместим наблюдения по времени для данного индивидуума обычным способом, так что \ref{Eq:22.52} превратится в
\begin{align}
\mathbf y_i=\mathbf Z_i \be+(\mathbf W_i \bm\alpha_i+\bm\e_i).
\label{Eq:22.53}
\end{align}
Обычно предполагается, что $\bm\alpha_i$ и $\bm\e_i$  независимы по $i$  и между друг другом с $\bm\alpha_i \thicksim [\mathbf 0, \sum_{\alpha}]$ и $\bm\e_i \thicksim [ \mathbf 0, \sum_{\bm\e}]$, так что ошибка 
 \begin{align}
\mathbf W_i \bm\alpha_i+\bm\e_i \thicksim [\mathbf 0, \bm\Omega_i = \mathbf W_i \bm\Sigma_{\alpha} \mathbf W'_i + \bm\Sigma_{\bm\e}].
\nonumber
\end{align}
Тогда \textbf{доступная ОМНК оценка} будет иметь вид
\begin{align}
 \hat{\be}_{FGLS}=\left[ \sum^N_{i=1} \mathbf Z'_i \hat{\bm\Omega}^{-1}_i \mathbf Z_i 
\right] ^{-1}
\sum^N_{i=1} \mathbf Z'_i \hat{\bm\Omega}^{-1}_i \mathbf y_i,
\label{Eq:22.54}
\end{align}
где $\hat{\bm\Omega}_i$ является состоятельной оценкой для $\bm\Omega_i$.

Осуществление этого подхода требует состоятельного оценивания $\bm\Omega_i$. Это уже обсуждалось в разделе 21.7 для более простого случая случайного свободного члена. В распоряжении имелось несколько различных способов состоятельно оценить компоненты дисперсии $\sigma^2_{\alpha}$ и $\sigma^2_{\e}$ с такими возможными осложнениями, как смещение или возможность получения отрицательных оценок. Схожие проблемы появляются здесь при оценивании $\bm\Sigma_{\bm\alpha}$ и $\bm\Sigma_{\bm\epsilon}$.

Мы представляем две оценки, основанные на дополнительном предположении о нормальном распределении случайных компонент. Более общая модель имеет вид 
\begin{align}
\mathbf y= \mathbf Z \be + (\mathbf W \bm\alpha + \e),
\label{Eq:22.55}
\end{align}
который может быть получен, например, соответствующим совмещением наблюдений модели \ref{Eq:22.53}. Предполагается, что $\bm\alpha \thicksim \mathcal N [\mathbf 0, \mathbf G]$ и  $\e \thicksim \mathcal N[\mathbf 0, \mathbf R]$, где в текущем случае $\mathbf G$ и $\mathbf R$  --- функции $\bm\Sigma_{\bm\alpha}$ и $\bm\Sigma_{\bm\e}$. \textbf{Доступная ОМНК оценка} для смешанной модели
\begin{align}
\hat{\be}_{FGLS}=[\mathbf Z' \hat{\mathbf V}^{-1} \mathbf Z]^{-1}
\mathbf Z' \hat{\mathbf V}^{-1} \mathbf y,
\nonumber
\end{align}
где $\hat{\mathbf V}$ --- это состоятельная оценка для $\mathbf V=\mathrm V[\mathbf W \bm\alpha +\bm\e]=\mathbf W \mathbf G \mathbf W' + \mathbf R$. См. Свами (1970).

Очевидный метод для получения $\hat{\mathbf V}$ --- метод максимального правдоподобия. Функция логарифма максимального правдоподобия основана на многомерном нормальном, после замены вектора $\be$ на его ОМНК оценку $[\mathbf Z' \mathbf V^{-1} \mathbf Z]^{-1} \mathbf Z' \mathbf V^{-1} \mathbf y$,
\begin{align}
\mathrm{ln} L(\mathbf G, \mathbf R)=-\frac{1}{2} \mathrm{ln}|\mathbf V| - \frac{NT}{2} \mathrm{ln} \mathbf r' \mathbf V^{-1} \mathbf r - \frac{NT}{2} \left[ 1+ \mathrm{ln}(\frac{2\pi}{NT})\right],
\nonumber
\end{align}
где $\mathbf r= \mathbf y - \mathbf Z[\mathbf Z'\mathbf V^{-1} \mathbf Z]^{-1} \mathbf Z' \mathbf V^{-1} \mathbf y$ и $|\mathbf V|$ обозначает определитель $\mathbf V$. Максимизация по параметрам в $\mathbf G$ и $\mathbf R$ дает $\hat{\mathbf V}=\mathbf W \hat{\mathbf G} \mathbf W' + \hat{\mathbf R}$.

Недостаток оценок ММП компонент дисперсий заключается в том, что они смещены в малых выборках. Например, для линейных пространственных регрессий с гомоскедастичными ошибками оценка ММП $\hat{\sigma}^2=N^{-1} \sum_i \hat{u}^2_i$ смещена, и поэтому делить лучше на $(N-K)$. Для модели \ref{Eq:22.53} коррекции на степени свободы обеспечиваются оценкой \textbf{ограниченной функцией правдоподобия}, которая максимизирует
\begin{align}
\mathrm{ln} L_R(\mathbf G, \mathbf R)=-\frac{1}{2} \mathrm{ln}|\mathbf V| - \frac{NT-p}{2} \mathrm{ln} \mathbf r' \mathbf V^{-1} \mathbf r - \frac{NT-p}{2} \left[ 1+ \mathrm{ln}(\frac{2\pi}{NT-p})\right]
-\frac{1}{2}\mathrm{ln}| \mathbf Z' \mathbf V^{-1} \mathbf Z|,
\nonumber
\end{align}
где  $p$  --- ранг $\mathbf Z$. Преимущества использования  $\mathrm{ln} L_R(\mathbf G, \mathbf R)$ см. у Харвилла (1977).

В качестве примера смешанной линейной модели рассмотрим регрессию ln(hours)-ln(wage) раздела 21.3, в которой и свободный член и параметры наклона могут быть случайными. Тогда модель со случайными коэффициентами будет выглядеть как lnhrs=7.734-0.02lnwg, стандартная ошибка коэффициента наклона равна 0.046 (по умолчанию) или 0.020 (панельный бутстреп). Коэффициент наклона отличается от оценок, данных в таблице 21.2.

 \subsection{Прогнозирование}

Мы можем желать \textbf{спрогнозировать} случайные параметры $\bm\alpha$  в дополнение к фиксированным параметрам $\be$ и ковариациям.

Совместные  уравнения для $\hat{\be}$ и $\hat{\bm\alpha}$ при условии, что   оценки $\hat{\be}$ и $\hat{\bm\alpha}$ состоятельны, могут быть записаны как
\begin{align}
\begin{bmatrix}
\mathbf Z' \hat{\mathbf R}^{-1} \mathbf Z & \mathbf Z' \hat{\mathbf R}^{-1} \mathbf W \\
\mathbf W' \hat{\mathbf R}^{-1} & \mathbf W' \hat{\mathbf R}^{-1} \mathbf W + \hat{\mathbf G}^{-1}
\end{bmatrix}
\begin{bmatrix}
\hat{\be} \\
\hat{\bm\alpha}
\end{bmatrix}
=
\begin{bmatrix}
\mathbf Z' \hat{\mathbf R}^{-1} \mathbf y \\
\mathbf W' \hat{\mathbf R}^{-1} \mathbf y
\end{bmatrix}.
\nonumber
\end{align}
Решая уравнение относительно $\hat{\be}$, получаем прежнюю оценку $\hat{\be}_{FGLS}$, в то время как
\begin{align}
\hat{\bm\alpha}=\hat{\mathbf G} \mathbf W' \hat{\mathbf V}^{-1} (\mathbf y - \mathbf Z' \hat{be}).
\nonumber
\end{align}
В случае независимости по $i$ $\hat{\bm\alpha}_i=\hat{\sum_{\bm\alpha}}\mathbf W'_i \hat{\mathbf V}_i^{-1} (\mathbf y_i - \mathbf Z'_i \hat{\be})$. Это \textbf{лучший линейный несмещенный прогноз} при известных ковариационных матрицах.


\section{Практические соображения}

Оценки двухшагового МНК для панельных данных могут быть получены с помощью обычного двухшагового алгоритма для пространственных данных (см. раздел 22.2.5), хотя вычисленные стандартные ошибки должны быть робастными. Оптимальные ОММ оценки могут быть получены с помощью команд для работы с матрицами в статистических пакетах или в языках программирования, таких как GAUSS или R. Статистические пакеты все чаще предлагают команды для работы с панельными данными, которые автоматически вычисляют оценки этой главы, в том числе оценку Ареллано-Бонда.

\section{Литература}

В этой главе активно описывается развивающаяся область исследования, которая освещается в нескольких недавних работах, посвященных панельным данным, а именно Бальтаджи (1995, 2001), Хсяо (1986, 2003), М.–Дж. Ли (2002), и Ареллано (2003). Более сложные методы представлены в работах Матиас и Севестр (1995) и Ареллано and Оноре (2001).

\textbf{22.2} Чемберлин (1982, 1984) делал акцент на использовании предположений об экзогенности. Он использовал оценку минимального расстояния. Впоследствии чаще использовался ОММ. М.–Дж. Ли (2002) и Ареллано (2003) придавали особое значение ОММ оцениванию. См. также работу Ана и Шмидта (1999).

\textbf{22.4} Привлекательна модель Хаусман and Тейлор (1981). Предполагая некоррелированность некоторых регрессоров с индивидуальными эффектами, можно  идентифицировать коэффициенты регрессоров, не меняющихся во времени.

\textbf{22.5} Линейные динамические модели представлены в литературе достаточно кратко по сравнению с объемом литературы, которая началась с Балестра and Нерлова (1966). Более сложные обсуждения даны в Бальтаджи (2001, глава 8), Хсяо (2003, глава 4), и Ареллано (2003, глава 5–8). Особенно популярна оценка Ареллано–Бонда (1991), так как она подходит для динамических моделей с фиксированными эффектами.

\textbf{22.6} Из-за своей простоты очень популярен метод <<разность разностей>>. Хотя он может быть использован для повторяющихся пространственных данных, интерпретация для панельных данных помогает сделать явными первоначальные предположения. Бертран и др. (2004) демонстрируют важность корректирования корреляции во времени на индивидуальном уровне с использованием методов раздела 22.2.3.

\textbf{22.8} Смешанные линейные модели особенно популярны в статистической литературе. Они реже используются в эконометрической литературе из-за того, что они не накладывают структуру на индивидуальные фиксированные эффекты, не меняющиеся во времени.

 {\centering
{\bf Упражнения}\\}

\textbf{22-1} Рассмотрим ОММ оценку для панельных данных, представленную в разделе 22.2.1.
\begin{itemize}
\item[{\bf (a)}] Покажите, что минимизируя по $\be$  квадратичную функцию $Q_N(\be)$, выписанную после уравнения \ref{Eq:22.3}, мы получим ОММ оценку для панельных данных, выражение для которой дано после $Q_N(\be)$ с использованием обозначения суммы.
\item[{\bf (b)}] Покажите, что эта оценка эквивалентна оценке, определенной в \ref{Eq:22.4}.
\item[{\bf (с)}] Предположим для простоты, что матрицы $\mathbf Z$ и $\mathbf X$ в \ref{Eq:22.4} не являются стохастическими, и что $\mathbf y = \mathbf X \bm\be + \mathbf u$, где среднее и дисперсия $\mathbf u$ равны $0$ и $\Omega$ соответственно. Получите ковариационную матрицу оценки в \ref{Eq:22.4} для ограниченной выборки и сравните ее с результатами для асимптотики в \ref{Eq:22.5}.
\item[{\bf (d)}] Упростите ОММ оценку для панельных данных для случая, когда $r=K$.
\end{itemize}

\textbf{22-2} Рассмотрим модель анализа панельных данных $y_{it}=\alpha + \beta \x_{it}+\gamma w_{it}+u_{it}, i=1, \dots, N, t=1, \dots, T$, где для простоты отсутствуют индивидуальные эффекты. Предположим, что регрессор-скаляр $x_{it}$  коррелирован с $u_{it}$ для всех $t$ и $s$. Для каждого варианта определите, возможно ли состоятельное IV оценивание параметров $\be$ и $\gamma$, и если это так запишите все подходящие инструменты, основываясь на обсуждении в главе 22.2. Предполагайте, что доступно три периода данных, $T=3$. Заметьте, что переменная может не быть доступна как инструмент во всех периодах, и что в разные годы могут быть доступны разные инструменты.
\begin{itemize}
\item[{\bf (a)}] Регрессор $w_{it}$ удовлетворяет предположению о сумме $\E[\sum_t w_{it} u_{it}]=0$.
\item[{\bf (b)}] Регрессор $w_{it}$ удовлетворяет предположению об одновременной экзогенности $\E[w_{it} u_{it}]=0, t=1, \dots, 3$.
\item[{\bf (с)}] Регрессор $w_{it}$ удовлетворяет предположению о слабой экзогенности $\E[w_{is} u_{it}]=0, s \leq t, t=1, \dots, 3$.
\item[{\bf (d)}] Регрессор $w_{it}$ удовлетворяет предположению о строгой экзогенности $\E[w_{it} u_{it}]=0, s, t=1, \dots, 3$.
\end{itemize}

\textbf{22-3} Повторите задание 2, тоже с тремя периодами данных, но сейчас возьмите модель $y_{it}=\alpha + \beta \x_{it}+\gamma w_{it}+u_{it}$, где $\alpha_i$  --- это фиксированный эффект, и используйте IV оценивание, основанное на модели в первых разностях, $y_{it}-y_{i,t-1}=\beta (x_{it} - x_{i,t-1})+\gamma (w_{it}-w_{i,t-1})+(u_{it}-u_{i,t-1})$.

\textbf{22-4} Рассмотрим оценку <<разность разностей>> (DID), представленную в разделе  22.6. Предположим, временной тренд $(\delta_t-\delta_{t-1})$ различается для опытной и контрольной групп.
\begin{itemize}
\item[{\bf (a)}] Будет ли DID оценка $\phi$, основанная на повторяющихся пространственных данных состоятельна? Объясните свой ответ.
\item[{\bf (b)}] Возможно ли получить состоятельную оценку $\phi$, если используются панельные данные. Объясните Ваш ответ.
\end{itemize}
 
\textbf{22-5} Используя данные о часах и заработной плате из работы Зилиака (1997), воспроизведите таблицу 22.2, насколько это возможно, с хорошим объяснением, когда набор инструментов расширяется и включает третий лаг lnwg, kids, age, agesq, и disab, и для оценки \ref{Eq:22.22} используется семь лет 1982–88.

 


\chapter{Нелинейные модели панельных данных} 

\subsection{Введение}
Эта глава является продолжением глав 21 и 22, посвященных линейным моделям панельных данных. Здесь рассматриваются нелинейные модели регрессии, представленные в главах 14 --- 20. Во внимание берутся короткие панели и модели с фиксированными или случайными индивидуальными эффектами, не меняющимися во времени. Рассматриваются и статическая, и динамические модели.

Не существует какого-то единого предписания для нелинейных моделей с индивидуальными эффектами. Если используются фиксированные индивидуальные эффекты, а панель короткая, то получение состоятельных оценок для параметров наклона возможно только для некоторых нелинейных моделей. Если же используются случайные индивидуальные эффекты, то состоятельные оценки можно получить уже для более широкого ряда моделей.

В разделе 23.2 представлены общие подходы, которые могут или не могут быть применены для конкретных моделей. В разделе 23.3 описан пример применения нелинейных моделей с мультипликативными индивидуальными эффектами. Особенности основных классов нелинейных моделей --- моделей дискретных данных, моделей выбора, моделей с данными о переходах и счетными данными представлены в разделах 23.4-23.7. Полупараметрические модели оценивания исследуются в разделе 23.8.

\section{Общие результаты}

В этом разделе представлены общие подходы расширения методов линейных моделей. Во-первых, мы представляем различные модели --- с фиксированными и случайными эффектами, модели сквозной регрессии, проводя различие между параметрическими  моделями и моделями условного среднего. Затем представлены методы оценки этих моделей и способы получения робастных стандартных ошибок для панельных данных. Дальнейшие подробности спецификаций нелинейных моделей панельных данных даны в следующих разделах.

\subsection{Модели с индивидуальными эффектами}

В линейной модели с индивидуальными эффектами (см. раздел 21.2.1) зависимая переменная $y_{it}$ зависит от индивидуальных эффектов $\alpha_i$, обычных регрессоров $\x_{it}$ и параметров регрессии $\be$. Модель записывается как $y_{it}=\alpha_i+\x'_{it}\be + u_{it}$, где $u_{it}$ --- ошибка.

Для нелинейных моделей, таких как логит-модели или модели Пуассона, аддитивная ошибка $u_{it}$ не вводится. Вместо этого, более естественно напрямую моделировать условную плотность или условное среднее. В линейном случае это выражается как $\E[y_{it}|\alpha_i, \x_{it}]=\alpha_i+\x'_{it}\be$.

{\centering Параметрические модели \\
}

Общим для многих нелинейных моделей является полностью непараметрический подход. Особенно это касается моделей бинарного и множественного выбора, а также цензурированных выборок, описанных в главах 14-16.

Стандартные модели для пространственных данных  --- одноиндексные модели, или одноиндексные модели с дополнительными коэффициентами масштаба. В \textbf{параметрических моделях с индивидуальными эффектами}, представленных в последующих разделах, специфицируется условная плотность
\begin{align}
f(y_{it}|\alpha_i, \x_{it})=f(y_{it}, \alpha_i+\x'_{it}\be, \bm\gamma)
\label{Eq:23.1}
\end{align}
где $\bm\gamma$  обозначает дополнительные параметры, например, дисперсию. Эта модель называется одноиндексной моделью с регрессорами $\x_{it}$  и индивидуальными эффектами $\alpha_i$.

Обычно предполагается, что $y_{it}|\x_{it}, \alpha_i$ независимы по $i$ и по $t$.  Предположение может быть ослаблено. Возможна зависимость по $t$ для данного $i$ (см. раздел 23.2.6).

 {\centering Модели условного среднего \\}

Общая нелинейная модель для условного среднего с ненаблюдаемыми не меняющимися во времени индивидуальными эффектами:
\begin{align}
&\E[y_{it}|\alpha_i, \x_{it}]=\mathrm g(\alpha_i, \x_{it}, \be),
& i=1, \dots, N &
&t=1,  \dots, T,
\label{Eq:23.2}
\end{align}
для заданной функции $\mathrm g(\cdot)$. Три распространенных спецификации  --- это \textbf{модель с аддитивными индивидуальными эффектами}
\begin{align}
\mathrm g(\alpha_i, \x_{it}, \be)=\alpha_i+ \mathrm g(\x_{it}, \be),
\label{Eq:23.3}
\end{align}
\textbf{модель с мультипликативными индивидуальными эффектами}
\begin{align}
\mathrm g(\alpha_i, \x_{it}, \be)=\alpha_i \mathrm g(\x_{it}, \be),
\label{Eq:23.4}
\end{align}
и \textbf{одноиндексная модель с индивидуальными эффектами}
\begin{align}
\mathrm g(\alpha_i, \x_{it}, \be)= \mathrm g(\alpha_i+\x_{it}, \be).
\label{Eq:23.5}
\end{align}
Функция $\mathrm g(\cdot)$ специфицирована в каждом случае. Регрессоры $\x_{it}$ могут быть изменяющимися или неизменными во времени, а могут включать временную дамми переменную.

Модель с аддитивными эффектами подходит для случаев, когда значения $y_{it}$  не ограничены, как это неявно предполагалось в случае линейной регрессии. Модель с мультипликативными эффектами подходит для случаев, когда $y_{it}$ не принимает отрицательные значения. Например, это счетные данные, для которых $\alpha_i>0$ и  $\mathrm g(\cdot)>0$. Одноиндексная модель --- это естественное начало для пробит-модели, например, с $\mathrm g(\alpha_i+\x'_{it})=\Phi(\ali+\x'_{it}\be)$, где $\Phi(\cdot)$ --- это функция стандартного нормального распределения. Одноиндексная модель сводится к аддитивной модели, если $\mathrm g(\cdot)$ --- это тождественное отображение. 
Она сводится к мультипликативной модели, если $\mathrm g(\cdot)$  --- экспоненциальная функция. В таком случае $\exp (\ali+\x'_{it}\be)=\exp (\ali)\exp (\x'_{it}\be)$.

Моментное тождество \ref{Eq:23.2} обуславливается только текущим периодом $\x_{it}$ и предполагает, что регрессоры \textbf{одновременно экзогенны} (см. раздел 22.2.4). Для устранения индивидуальных эффектов $\alpha_i$ могут потребоваться более строгие предположения об экзогенности. Регрессоры \textbf{слабо экзогенны}, если 
\begin{align}
\E[y_{it}|\ali, \x_{i1}, \dots, \x_{it}]=\mathrm g(\ali, \x_{it},\be)
\label{Eq:23.6}
\end{align}
и \textbf{сильно экзогенны} или \textbf{строго экзогенны}, если
\begin{align}
\E[y_{it}|\ali, \x_{i1}, \dots, \x_{iT}]=\mathrm g(\ali, \x_{it},\be).
\label{Eq:23.7}
\end{align}

Нелинейная модель с аддитивными эффектами усложняется относительно незначительно. В частности, если модель панельных данных имеет вид $y_{it}=\ali+\mathrm g(\x_{it},\be)+u_{it}$, то подходы глав 21 и 22 должны претерпеть некоторые изменения, включая использование нелинейных МНК оценок и нелинейных оценок инструментальных переменных вместо линейных.

В этой главе внимание акцентируется на моделях с неаддитивными индивидуальными эффектами, такими как в моделях \ref{Eq:23.4} и \ref{Eq:23.5}. Эти эффекты могут рассматриваться как фиксированные или как случайные эффекты.

\subsection{Модели с фиксированными эффектами}

В \textbf{модели с фиксированными эффектами} индивидуальные эффекты $\ali$ фигурируют в качестве ненаблюдаемых случайных переменных, которые могут быть коррелированы с регрессорами $\x_{it}$. В коротких панелях совместное оценивание фиксированных эффектов $\alpha_i, \dots, \alpha_N$ и других параметров модели $\be$ и, возможно, $\bm\gamma$, вообще приводит к несостоятельным оценкам всех параметров. Вместо этого, было предложено множество методов для устранения фиксированных эффектов в некоторых специальных постановках модели, которые позволяют состоятельно оценивать другие параметры модели.

 {\centering Проблема второстепенных параметров\\}
 
Нейман и Скотт (1948) рассматривали статистические выводы, когда некоторые параметры являются общими для всех наблюдений, но в то же время имеется бесконечное число параметров, каждый из которых зависит от конечного количества наблюдений. Нас интересуют \textbf{общие параметры}. Последние же параметры называются \textbf{второстепенными параметрами}.

Здесь $\be$ и $\bm\gamma$ --- общие параметры, а $\alpha_1, \dots, \alpha_N$ --- второстепенные параметры. Это соответствует короткой панели, когда каждый $\ali$  зависит от фиксированного количества $T$ наблюдений и есть бесконечно много $\ali$, так как $N \rightarrow \infty$. Оценки второстепенных параметров несостоятельны при $N \rightarrow \infty$, так как используется только $T$ наблюдений для оценки каждого параметра. \textbf{Проблема второстепенных параметров} состоит в том, что это загрязняет оценивание общих параметров. В общем случае и оценки общих параметров также несостоятельны, даже если их количество ограничено, и их оценивание производится с помощью $NT \rightarrow \infty$ наблюдений.

Просто проиллюстрировать загрязнение второстепенными параметрами можно, предполагая, что $y_{it} \thicksim \mathcal{N}[\ali, \sigma^2]$. Оценка ММП дает $\hat{\ali}=\bar{y_i}, i=1, \dots, N$ и $\hat{\sigma}^2=(NT)^{-1}\sum_i \sum_t (y_{it}-\bar{y}_i)^2$. Тогда $\E[\hat{\sigma}^2]=\sigma^2(T-1)/T$, т.е. $\hat{\sigma}^2$ --- несостоятельная оценка для $\sigma^2$ при  $N \rightarrow \infty$ в коротких панелях с фиксированным $T$. Несостоятельность может быть значительной. Например, $\hat{\sigma}^2 \xrightarrow{\text{p}} 0.5 \sigma^2$, когда $T=2$.

В общем случае при возникновении проблемы второстепенных параметров, необходимы альтернативные методы оценивания, которые на первом шаге позволяют избавиться от второстепенных параметров. Для некоторых известных моделей, особенно пробит-модели панельных данных, существует решение проблемы второстепенных параметров. Даже когда  существуют методы для получения состоятельных оценок $\be$, эти методы, скорее, будут более специфическими, как отметил Ланкастер (2000). Не существует единого решения проблемы второстепенных параметров.

 {\centering Условная функция правдоподобия \\}

Статистика $t$  называется \textbf{достаточной} для параметра $\theta$, если распределение выборки при данном $t$ не зависит от $\theta$. В моделях панельных данных с индивидуальными эффектами, если существует достаточная статистика для вспомогательного параметра $\ali$, то вспомогательный параметр уничтожается, когда мы включаем в условие эту достаточную статистику. Результирующая условная плотность зависит от общих параметров, что позволяет их состоятельно оценивать.

Пусть $\mathbf{y}_i=[y_{i1}, \dots, y_{iT}]'$ --- это вектор зависимой переменной для индивидуального наблюдения $i$ размерности $T \times 1$ по всем $T$ периодам, и пусть $\mathbf{X}_i=[\x_{it}, \dots, \x_{iT}]'$ обозначает соответствующую матрицу регрессоров размерности $T \times K$. Для статической модели плотность $\mathbf{y}_i$ равна
\begin{align}
f(\mathbf{y}_i|\mathbf{X}_i, \ali, \be, \gamma)= \prod^T_{t=1} f(y_{it}|\x_{it}, \ali, \be, \gamma).
\label{Eq:23.8}
\end{align}

Оценивание методом максимального правдоподобия, основанное на этой плотности в общем случае приводит к несостоятельному оцениванию $\be$ в коротких панелях из-за проблемы второстепенных параметров.

Предположим, что существует \textbf{достаточная статистика} $\mathbf{s}_i$ для $\alpha_i$. Тогда использование ее в качестве условия в дополнение к обычным регрессорам дает \textbf{условную плотность}
\begin{align}
f(\mathbf{y}_i|\mathbf{X}_i, \ali, \be, \bm\gamma,\mathbf{s}_i)= f(\mathbf{y}_{it}|\mathbf{X}_{i}, \be, \bm\gamma, \mathbf{s}_i),
\label{Eq:23.9}
\end{align}
где исчезает $\ali$. Например, для линейной регрессионной модели в предположении нормальности $\mathbf{s}_i=\bar{y}_i$ (см. раздел 21.6.3). Тогда \textbf{оценка условного МП} максимизирует условную функцию правдоподобия
\begin{align}
\mathrm{ln L_{COND}}(\be,\bm\gamma)= \sum^N_{i=1} ln f(\mathbf{y}_{it}|\mathbf{X}_{i}, \be, \bm\gamma, \mathbf{s}_i).
\label{Eq:23.10}
\end{align}
Прилагательное <<условная>> добавлено для обозначения того, что берется функция  не только при условии$\mathbf{X}_i$, но и $\mathbf{s}_i$.

Андерсен (1970) проделал детальный анализ оценки условного ММП. Он показал, что оценка условного ММП будет состоятельна, если плотность $f(\mathbf{y}_i|\mathbf{X}_i, \ali, \be)$  правильно специфицирована, что информационно матричное равенство выполняется для условной функции правдоподобия. Однако вообще происходит потеря эффективности, так как оценка условного ММП необязательно достигает  нижнюю границу неравенства Рао-Крамера. Для нормального распределения и распределения Пуассона, однако, потеря эффективности не происходит.

Для данного подхода требуется, чтобы существовала подходящая достаточная статистика. Это выполняется только для нескольких моделей, в основном для моделей из семейства линейных экспоненциальных распределений. Андерсен останавливал свое внимание на моделях без регрессоров, используя в качестве примеров нормальное распределение, распределение Пуассона, биномиальное и гамма распределения. С регрессорами найти подходящую достаточную статистику становится еще сложнее. Общее обсуждение данного подхода можно найти в работе МакКуллах и Нелдер (1989). Диггл и др. (2002) в своей работе уделяли внимание специализированным обобщенным линейным моделям с каноническими связующими функциями. 

Основные примеры моделей, когда достаточные статистики доступны, --- линейные модели в условиях нормальности (см. раздел 21.6.2), логит-модели (но не пробит-модели) для бинарных данных (см. раздел 23.4.3), модели распределения с одним параметром гамма (включая экспоненциальное) и особые параметризации Пуассона и отрицательные биномиальные модели для счетных данных (см. раздел 23.7.3).

{\centering Преобразование <<отклонение от среднего>> \\}

Для некоторых моделей условных средних с аддитивными или мультипликативными эффектами индивидуальные эффекты $\ali$ могут быть устранены с помощью подходящего взятия разностей. Это позволяет получить моментные тождества, которые можно использовать для метода моментов или ОММ оценивания, что подробно представлено в разделе 23.2.6.

\textbf{Преобразование <<отклонение от среднего>>} --- это обобщение преобразования within для линейной модели, описанного в разделе 21.2.2, которое позволяет избавиться от $\ali$ посредством вычитания индивидуальных \textbf{средних}. Для этого преобразования необходимы строго экзогенные регрессоры, см. \ref{Eq:23.7}.

Для модели с аддитивными эффектами \ref{Eq:23.3} со строго экзогенными регрессорами
\begin{align}
\E[(y_{it}-\bar{y}_i)-(\mathrm g(\x'_{it}\be)-\bar{g}_i(\be))|\x_{i1},\dots,\x_{iT}]=0,
\label{Eq:23.11}
\end{align}
где $\bar{g}_i(\be)=T^{-1}\sum^T_{t=1}\mathrm g(\x'_{it}\be)$, и для получения результата используется $\E[\bar{y}_i|\x_{i1}, \dots, \x_{iT}]=\ali+\bar{g}_i(\be)$. Для линейных моделей \ref{Eq:23.11} значительно упрощается, так как в случае линейности $\mathrm g(\x'_{it}\be)-\bar{g}_i(\be)=(\x_{it}-\bar{x}_i)'\be$.

Для модели с мультипликативными эффектами \ref{Eq:23.4} алгебраические преобразования приводят к 
\begin{align}
\E\left[y_{it}-\frac{\mathrm g(\x'_{it}\be)}{\bar{g}_i(\be)}\times \bar{y}_i|\x_{i1},\dots, \x_{iT} \right]=0.
\label{Eq:23.12}
\end{align}
В ходе преобразований используется тот факт, что $\E[\bar{y}_i|\x_{i1}, \dots, \x_{iT}]=\ali\bar{g}_i(\be)$. Для простоты мы называем это преобразованием \textbf{<<отклонение от среднего>>}, хотя строго говоря это \textbf{квази-разность}. Это преобразование также называется (условным) \textbf{преобразованием, масштабирующим по среднему}, так как оно эквивалентно
\begin{align}
\E \left[y_{it}-\frac{\bar{y}_i}{\bar{g}_i(\be)}\mathrm g(\x'_{it}\be)|\x_{i1}, \dots, \x_{iT} \right] = 0.
\nonumber
\end{align}

{\centering Преобразование <<первые разности>> \\}

\textbf{Преобразование <<первые разности>>} --- обобщение преобразования <<взятие первых разностей>> для линейной модели, представленной в разделе 21.2.2, который устраняет $\ali$  посредством вычитания первого лага. Мы предполагаем, что регрессоры слабо экзогенны (см. \ref{Eq:23.6}).

Для модели с аддитивными эффектами
\begin{align}
\E[(y_{it}-y_{i,t-1})-(\mathrm g(\x'_{it}\be)-\mathrm g(\x'_{i,t-1}\be))|\x_{i1},\dots,\x_{i,t-1}]=0,
\label{Eq:23.13}
\end{align}
где мы использовали $\E[y_{i,t-1}|\x_{i1}, \dots, \x_{i,t-1}]=\ali+\mathrm g(\x'_{i,t-1}\be)$.

Для моделей с мультипликативными эффектами \ref{Eq:23.4}
\begin{align}
\E\left[ y_{it}-\frac{\mathrm g(\x'_{it}\be)}{\mathrm g(\x'_{i,t-1}\be)}\times y_{i,t-1}|\x_{i1}, \dots, \x_{i,t-1} \right]=0,
\label{Eq:23.14}
\end{align}
где мы использовали $\E[y_{i,t-1}|\x_{i1}, \dots, \x_{i,t-1}]=\ali \mathrm g(\x'_{i,t-1}\be)$. Для простоты мы называем это \textbf{преобразованием <<первые разности>>}, хотя строго говоря, это \textbf{квази-разность}.

Преобразование <<первые разности>> основывается на более слабых предположениях, в качестве условий используются только периоды меньше $t$. Это позволяет оценивать динамические модели в том числе и нелинейные модели в продолжение моделей раздела 22.5. Для динамических мультипликативных эффектов Вулдридж (1997) и Чемберлин (1992) предложили использовать версию \ref{Eq:23.14}
\begin{align}
\E\left[ \frac{\mathrm g(\x'_{i,t-1}\be)}{\mathrm g(\x'_{it}\be)}y_{it}-y_{i,t-1}|\x_{i1}, \dots, \x_{i,t-1} \right]=0.
\label{Eq:23.15}
\end{align}

{\centering Оценивание модели с фиктивными переменными \\}

Если игнорировать проблему второстепенных параметров, можно попытаться оценить все параметры, включая индивидуальные эффекты. Возьмем набор $N$ дамми переменных $d_{j,it}$, равных 1, если $i=j$ и 0 иначе. Затем совместно оцениваем индивидуальные параметры $\alpha_1, \dots, \alpha_N$ наряду с другими параметрами модели.

Эта оценка доступна с вычислительной точки зрения, несмотря на большое число параметров из-за большого $N$. Но результирующие оценки $\be$ и, возможно, $\bm\gamma$ несостоятельны. Здесь мы рассматриваем только параметрические модели, хотя есть сходства с моделями условного среднего.

Итак, рассмотрим параметрическую модель с индивидуальными эффектами \ref{Eq:23.1}. Тогда способ получения оценок ММП $\be$, $\gamma$ и  $\bm\alpha=[\alpha_1 \dots \alpha_N]'$ --- максимизировать полную функцию правдоподобия в логарифмах
\begin{align}
\mathrm{ln L}_{FE}(\be,\bm\gamma,\alpha)=\sum^N_{i=1} \sum^T_{t=1} \mathrm{ln} f(y_{it}, \mathbf{d}'_{it} \bm\alpha + \x'_{it} \be, \bm\gamma),
\label{Eq:23.16}
\end{align}
где $\mathbf{d}_{it}=[d_{1,it} \dots, d_{N,it}]'$. Условия первого порядка для $\bm\delta = [\be' \; \bm\gamma]'$ и $\alpha$:
\begin{align}
\sum^N_{i=1} \sum^T_{t=1} \partial \mathrm{ln} f (y_{it}, \mathbf{d}'_{it}\bm\alpha + \x'_{it} \be, \bm\gamma) / \partial \delta = \mathbf{0}, 
\nonumber \\
\sum^T_{t=1} \partial \mathrm{ln} f (y_{it}, \ali + \x'_{it} \be, \bm\gamma) / \partial \ali = 0,&
& i=1, \dots, N.
\nonumber
\end{align}

Эту оценку просто посчитать несмотря на  большое количество параметров $N$ плюс размерность $\bm\delta$. Как подробно описано у Грина (2004b), обратить матрицу Гессе легко с помощью разбиения по $\bm\delta$  и $\bm\alpha$  и применения стандартной формулы обращения блочной матрицы. Используя тот факт, что $\partial \mathrm{ln L}(\bm\delta, \bm\alpha)/\partial\ali \partial\alpha_j=0$ для $j \neq i$,  получить обратную матрицу к блоку размерности $N \times N$, соответствующей $(\bm\alpha, \bm\alpha)$, довольно просто.


В двух особых случаях проблемы второстепенных параметров не возникает. Во-первых, если $y_{it} \thicksim \mathcal{N} [\ali + \x'_{it}\be, \sigma^2]$ тогда из раздела 21.6.4, оценка ММП для $\be$  --- оценка within, которая состоятельна для $\be$ даже для конечных $T$. Здесь проблема второстепенных параметров возникает при оценивании $\sigma^2$, но не $\be$. Во-вторых, для $y_{it} \thicksim  \mathcal{P} [\exp (\ali+\x'_{it}\be)]$ также нет проблемы второстепенных параметров при оценивании $\be$ (см. раздел 23.7.3).

Однако вообще проблема второстепенных параметров имеет место. Для взятия производной по $\ali$ необходимо только $T$, а не $NT$ наблюдений. Это обычно приводит к несостоятельности $\hat{\be}_{ML}$ и $\hat{\bm\gamma}_{ML}$ в коротких панелях. Возможно, несостоятельность умеренна в панелях средней длины с $T=10$ или $T=20$. С помощью симуляций Грин (2004a) показал, что природа и степень смещения сильно зависят от конкретной нелинейной модели. Развитие методов, робастных к присутствию фиксированных эффектов, хотя и несостоятельных для коротких панелей, является актуальной темой исследований.

\subsection{Модели со случайными эффектами}

В \textbf{модели со случайными эффектами} индивидуальные эффекты $\ali$ рассматриваются как случайные переменные с заданным распределением, и $\ali$ уничтожаются посредством взятия интеграла по этому распределению. Случайные эффекты обычно используют в случае параметрических моделей.

{\centering Параметрические модели \\}

Предположим $i$-е наблюдение $\mathbf{y}_i$ имеет безусловную совместную плотность $f(\mathbf{y}_i|\mathbf{X}_i, \ali, \be, \bm\gamma)$ \ref{Eq:23.8} и случайный эффект имеет плотность
\begin{align}
\ali \thicksim \mathrm g(\ali|\bm\eta),
\label{Eq:23.17}
\end{align}
где $\mathrm g(\ali|\bm\eta)$ не зависит от наблюдаемых переменных. Тогда безусловная совместная плотность для $i$-го наблюдения будет иметь вид
\begin{align}
f(\mathbf{y}_i|\mathbf{X}_i, \be, \bm\gamma, \bm\eta) = \int \left[ \prod^T_{t=1} f(y_{it} | \x_{it}, \ali, \be, \bm\gamma) \right] \mathrm g(\ali|\bm\eta)d \ali,
\label{Eq:23.18}
\end{align}
где под безусловным подразумевается то, что в условии больше не содержится $\ali$. \textbf{Оценки ММП случайных эффектов} $\be$, $\bm\gamma$ и $\bm\eta$ максимизируют логарифмическую функцию правдоподобия.
\begin{align}
\mathrm{ln L}_{\mathrm{RE}}(\be, \bm\gamma, \bm\eta)=\sum^N_{i=1} \mathrm{ln}\left( \int \left[ \prod^T_{t=1} f(y_{it}|\x_{it}, \ali, \be) \right] \mathrm g(\ali| \bm\gamma) d \ali \right).
\label{Eq:23.19}
\end{align}

В некоторых случаях этот интеграл можно выразить аналитически, в основном когда $\prod_t f(y_{it}|\ali)$ и $\mathrm g(\ali)$ --- сопряженные распределения (см. таблицу 23.2). Например, если оба распределения --- нормальные, то на выходе получается нормальное распределение. Или одно --- распределение Пуассона, а другое --- гамма-распределение для счетных данных, то на выходе получается отрицательное биномиальное распределение.

Во многих случаях аналитические результаты не доступны, но работают численные методы или методы, основанные на симуляциях, так как это однократный интеграл. Обычный подход состоит в том, чтобы для $f(y_{it})$ использовать плотность, наиболее адекватно описывающую данные  в отсутствие индивидуальных эффектов, а для $\mathrm g(\ali)$ --- нормальную плотность. Тогда мы получим однократный интеграл по случайной переменной, имеющей нормальное распределение. При малом $T$ для приближенного вычисления интеграла можно использовать метод Гаусса (см. раздел 12.3.1), которая приближает интеграл по нормальной плотности с помощью взвешенной суммы. Батлер и Моффитт (1982) подробно описывают пробит-модель со случайными эффектами. Скрондал и Рэйб-Хаскет (2004) используют метод Гаусса. В качестве альтернативы основой для оценивания методом симуляционного максимального правдоподобия могут быть  повторяющиеся выборки из $\mathrm g(\ali)$ (см. раздел 12.4.2).

В предыдущем обсуждении предполагалась независимость по $t$ для данного $i$. Если вместо этого $y_{it}$ и $y_{is}$ коррелированы по $i$, тогда более эффективно будет заменить $\prod_t f(y_{it}|\x_{it}, \ali, \be, \bm\gamma)$ на $ f(\mathbf{y}_i|\mathbf{X}_i, \ali, \be, \bm\gamma)$ в \ref{Eq:23.18} и \ref{Eq:23.19}.

{\centering Модель со случайными коэффициентами \\}

Модель со случайными эффектами может быть расширена до \textbf{модели со случайными коэффициентами}, т.е. со случайными коэффициентами наклона и свободным членом подобно линейному случаю раздела 22.8.

Естественная модель --- это одноиндексная модель с условной плотностью $f(y_{it}, \x'_{it}(\be+\ali),\bm\gamma)$ или условным средним $\mathrm g(y_{it}, \x'_{it}(\be+\ali))$ и однократный интеграл по скаляру $\ali$  станет многократным интегралом по вектору $\bm\alpha_i$. Обычно предполагается нормальное распределение.

{\centering Модель с коррелированными случайными эффектами \\}
  
Основной недостаток модели со случайными эффектами состоит в сильном предположении о независимых от регрессоров случайных коэффициентах.  Чемберлин (1980, 1982) решил ослабить это предположение и предложил \textbf{модель с коррелированными случайными эффектами}. Обсуждение см. в разделе 21.4.4, где $\ali$ специфицируется следующим образом
\begin{align}
\ali=\x'_{it} \bm\pi_1+ \dots + \x'_{Ti}\bm\pi_T+\xi_i.
\label{Eq:23.20}
\end{align}
Затем функция правдоподобия максимизируется по $\be, \bm\gamma, \bm\pi$, и параметрам плотности $\xi$. В отличие от линейных моделей эта модель  приводит к оценкам, отличным от тех, которые получаются с помощью спецификации Мундлака (1978), 
\begin{align}
\ali=\bar{\x}'_i\bm\pi+\xi_i.
\label{Eq:23.21}
\end{align}
Уравнение \ref{Eq:23.20} можно рассматривать как пример иерархической модели. В более общих иерархических моделях коэффициенты наклона могут быть случайными. Эти модели оцениваются классическими или байесовскими методами. В разделе 22.8 представлены подробности для линейной  модели.

{\centering Модель смеси распределений \\} 

Модель смеси распределений (см. раздел 18.5.1) --- это альтернативная модель для ненаблюдаемых индивидуальных эффектов. Если есть $m$ \textbf{латентных классов} или типов индивидуумов и для $j$-го типа $\ali=\alpha_j$ тогда \ref{Eq:23.18}
\begin{align}
f(\mathbf{y}_i|\mathbf{X}_i, \be, \bm\gamma, \pi) = \sum^m_{j=1} \left[\sum^T_{t=1}f(y_{it}|\x_{it}, \ali, \be, \bm\gamma) \right] \bm\pi_j.
\nonumber
\end{align}
Эта модель наиболее часто используется для моделей длительности состояний (см. раздел 18.5.2).

\subsection{Модели сквозной регрессии}

В модели сквозной регрессии индивидуальные эффекты не фигурируют явно. Сейчас расширим линейную сквозную регрессию (см. раздел 21.5) на нелинейный случай.

{\centering Модели условного среднего \\}

Для моделей условного среднего \textbf{модель сквозной регрессии} имеет вид
\begin{align}
\E[y_{it}|\x_{it}]=\mathrm g(\x_{it}, \be),
\label{Eq:23.22}
\end{align}
для специфицированной функции $\mathrm g(\cdot)$.

Модель \ref{Eq:23.22} можно оценить напрямую с помощью нелинейного МНК и использовать статистические выводы, основанные на робастных стандартных ошибках для панельных данных, которые учитывают возможную условную гетероскедастичность и условную корреляцию между $y_{it}$ и  $y_{is}$. Более эффективные оценки можно получить, только моделируя гетероскедастичность и корреляцию. Подробности см. в разделе 23.2.6.

{\centering Модели сквозной регрессии против модели со случайными эффектами \\}

Каковы издержки игнорирования индивидуальных случайных эффектов?

Модель с аддитивными эффектами $\E[y_{it}|\ali, \x_{it}]=\ali \mathrm g(\x_{it}, \be)$ сводится к \ref{Eq:23.22}, если $\E[\ali|\x_{it}]=0$. Модель с мультипликативными эффектами $\E[y_{it}|\ali, \x_{it}]=\ali \mathrm g(\x_{it}, \be)$ подразумевает, что   \ref{Eq:23.22}, если $\E[\ali|\x_{it}]=1$. Модель сквозной регрессии будет давать состоятельные оценки $\be$ в модели со случайными эффектами, если эффекты являются аддитивными или мультипликативными, и в модели использовалось стандартное нормирование  среднего $\ali$.

Иначе в модели сквозной регрессии вряд ли получатся те же оценки параметров, что и в модели со случайными эффектами. Например, рассмотрим пробит-модель со случайными эффектами, для которой $\E[y_{it}|\ali, \x_{it}]=\Phi(\ali+\x_{it}'\be)$, где $\ali \thicksim N[0,\sigma^2_{\alpha}]$. Тогда можно показать, что $\E[y_{it}|\x_{it}]=\Phi(\x_{it}'\be/\sqrt{1+\sigma^2_{\alpha}})$, что отличается от естественной пробит-модели сквозной регрессии, где $\E[y_{it}|\x_{it}]=\Phi(\x_{it}'\be)$. В отличие от линейной модели главы 21, если истинная модель содержит индивидуальные эффекты, то при игнорировании этих эффектов будут получены несостоятельные оценки параметра $\be$.

В статистической литературе часто используется сквозная регрессия для таких версий \textbf{обобщенных линейных моделей} панельных данных, как, например, модели бинарных данных и счетных данных. Результирующие оценки параметров называются \textbf{усредненными по генеральной совокупности}, так как случайные эффекты устранены. Такой подход называет \textbf{маргинальным анализом}, так как $\E[y_{it}|\x_{it}]$ --- это модель, которая является предельно по отношению к случайным эффектам.

{\centering Параметрические модели \\}

Отправной точкой в случае с \textbf{моделями сквозной регрессии} обычно выступает
\begin{align}
f(y_{it}|\x_{it})=f(y_{it}, \x'_{it}\be, \be, \bm\gamma)
\label{Eq:23.23}
\end{align}
для плотности $f(\cdot)$. Эту модель можно оценить ММП, используя для статистических выводов робастные стандартные ошибки, которые учитывают условную гетероскедастичность и корреляцию (см. раздел 23.2.6).

В общем оценки $\be$ и $\bm\gamma$ параметрической модели сквозной регрессии, скорее всего, не будут соответствовать оценкам, полученным из параметрической модели со случайными эффектами. Аргументы те же, что и для модели условного среднего.

\subsection{Фиксированные эффекты против случайных эффектов}

Важным результатом является то, что оценки модели со случайными эффектами и модели сквозной регрессии несостоятельны в нелинейных моделях, если в модели присутствуют индивидуальные эффекты, и они коррелированы с регрессорами. В связи с этим модели с фиксированными эффектами используются с большей предпочтительностью, хотя с другой стороны возникает проблема потери эффективности в оценивании. Проверить, необходимо ли использовать модель с фиксированными эффектами, можно с помощью теста Хаусмана (см. раздел 21.4.4), если возможно состоятельно оценить модель с фиксированными эффектами.

Другие методы сравнения линейных моделей фиксированных и случайных эффектов  (см. раздел 21.4) нужно видоизменить, чтобы адаптировать к нелинейным моделям.

Из-за проблемы второстепенных параметров не все нелинейные модели с фиксированными эффектами допускают состоятельные оценки параметров. Поэтому, моделирование фиксированных эффектов не всегда доступно.

Если нелинейную модель с фиксированными эффектами можно оценить состоятельно, то, в отличие от линейного случая, коэффициенты регрессоров, не меняющихся во времени, идентифицируемы. Чтобы это продемонстрировать, рассмотрим преобразование <<отклонение от среднего>> для модели с аддитивными эффектами. Для линейной модели $\E[(y_{it}-\bar{y}_i)-(\x_{it}-\bar{\x_i})'\be|\x_{i1},\dots,\x_{iT}]=\mathbf{0}$, и очевидно присутствуют проблемы с не меняющимися во времени регрессорами, так как для $j$-го регрессора  $x_{itj}-\bar{x}_{ij}=x_{ij}-x_{ij}=0$. В общем виде, из \ref{Eq:23.11}
\begin{align}
\E[(y_{it}-\bar{y}_i)-(\mathrm g(\x'_{it}\be-\bar{\mathrm{g}}_i(\be))|\x_{i1}]=\mathbf{0},
\nonumber
\end{align}
Такое упрощение возможно для нелинейного $\mathrm g(\cdot)$ только, если все $K$ компоненты $\x_{it}$ не меняются во времени.

В моделях с фиксированными эффектами с неаддитивными эффектами  не возможно предсказать изменения зависимой переменной при изменении регрессоров. Для общей модели \ref{Eq:23.2} \textbf{предельный эффект} $\partial \E[y_{it}|\x_{it},\ali,\be]/\partial \x_{it} = \partial \mathrm g(\x_{it}, \ali, \be)/\partial\x_{it}$ зависит от $\ali$.

Предельный эффект может быть измерен в двух частных случаях. Для аддитивных эффектов (см. \ref{Eq:23.3}) предельный эффект составляет $\partial \mathrm g(\x_{it}, \be)/\partial \x_{it}$. Для мультипликативных эффектов (см. \ref{Eq:23.4}) предельный эффект составляет $\ali \partial \mathrm g(\x_{it}, \be)/\partial \x_{it}$. Тогда возможно измерить размер предельных эффектов для изменений разных регрессоров. В частности, если $\E [y_{it}|\x_{it}, \ali, \be]=\ali \exp (\x_{it}' \be)$, то $(\partial \E[y_{it}]/\partial x_{it})/(\partial \E[y_{it}]/\partial x_{itk}) = \beta_j/\beta_k$.

\subsection{Оценивание и робастные статистические выводы}

В предыдущем анализе акцент был на устранении второстепенных параметров $\ali$. Сейчас, когда $\ali$ устранены, мы подробно будем рассматривать оценку параметров моделей с индивидуальными эффектами.

Мы предполагаем, что используется короткая панель, и наблюдения независимы по $i$.
Зависимая переменная $y_{it}$ может быть условно гетероскедастичной и условно коррелированной по $t$ для данного $i$. Ситуация похожа на ситуацию в разделе 21.2.3, за исключением того, что нелинейные оценки используются вместо обычного линейного МНК оценивания. По стандартным результатам выдаваемым статистическими пакетами, в которых игнорируется гетероскедастичность или коррелированность, можно сделать неверные статистические выводы. Далее мы представим выражения для робастных оценок ковариационной матрицы оценок параметров для панельных данных. В качестве альтернативы можно использовать панельный бутстреп (см. раздел 11.6.2).

{\centering ОMM оценивание \\}

ОММ оценивание подходит для моделей, основанных на условном среднем. Ключевой является спецификация моментных тождеств, что составляет основу ОММ оценивания. Согласно разделу 22.2.1, естественной отправной точкой будет
\begin{align}
&\E[\mathbf{Z}'_i \mathbf{u}_i (\mathbf{\theta})]=\mathbf{0},
&i=1, \dots, N
\label{Eq:23.24}
\end{align}
где $\mathbf{Z}_i$ --- это матрица размерности $T \times r$, которая зависит от регрессоров, $\mathbf{u}_i(\mathbf{\theta})$ --- вектор остатков размерности $T \times 1$, и $\mathbf{\theta}$ --- вектор параметров размерности  $q \times 1$. В различных моделях панельных данных $\mathbf{u}_i$ и $\mathbf{Z}_i$. Ниже дан пример. В качестве основной отправной точки главы 22 мы берем тот факт, что остатки $\mathbf{u}_i(\mathbf{\bm\theta})$ будут нелинейны по $\bm\theta$.

Если $r=q$, то моментных тождеств такое же количество, как и параметров для оценки, и мы можем использовать \textbf{оценку метода моментов для панельных данных} $\mathbf{\theta}_{\mathrm{MM}}$, которая является решением 
\begin{align}
\frac{1}{N} \sum^N_{i=1} \mathbf{Z}'_i \mathbf{u}_i (\hat{\bm\theta})=\mathbf{0}.
\label{Eq:23.25}
\end{align}
Используя результаты раздела 6.10.3, посвященного нелинейным системам оценивания, эта оценка асимптотически нормальна и имеет её  ковариационную матрицу можно состоятельно оценить
\begin{align}
\hat{\mathbf{V}} [\hat{\bm\theta}]=\left[ \sum^N_{i=1} \hat{\mathbf{D}}'_i \mathbf{Z}_i \right]^{-1} \sum^N_{i=1} \mathbf{Z}'_i \mathbf{\hat{u}}_i \mathbf{\hat{u}}'_i \mathbf{Z}_i 
\left[ \sum^N_{i=1} \mathbf{Z}'_i \hat{\mathbf{D}}_i  \right]^{-1}
\label{Eq:23.26}
\end{align}
где $\mathbf{\hat{D}}_i=\partial \mathbf{u}_i/ \partial \bm\theta'|_{\hat{\bm\theta}}$ и $\mathbf{\hat{u}}_i=\mathbf{u}_i(\hat{\bm\theta})$. Такая оценка дисперсии дает робастные стандартные ошибки для коротких панелей.

Если $r > q$, то требуется ОММ оценивание, и мы используем \textbf{ОММ оценку для панельных данных} $\hat{\bm\theta}_{\mathrm{GMM}}$, которая минимизирует 
\begin{align}
Q_{N}(\bm\theta)=\left[ \frac{1}{N} \sum^N_{i=1} \mathbf{Z}'_i \mathbf{u}_i (\bm\theta) \right]' \mathbf{W}_N \left[ \frac{1}{N} \sum^N_{i=1} \mathbf{Z}'_i \mathbf{u}_i (\bm\theta) \right],
\label{Eq:23.27}
\end{align}
где $\mathbf{W}_N$ --- это взвешивающая матрица размерности $r \times r$. Асимптотическая ковариационная матрица для этой оценки может быть получена напрямую из результатов для  оценки инструментальных переменных нелинейных систем, представленной в разделе 6.10.4. При моментных тождествах \ref{Eq:23.24} самая эффективная оценка использует $\mathbf{W}_N=[N^{-1} \sum_i \mathbf{Z}'_i \mathbf{\hat{u}}_i 
\mathbf{\hat{u}}'_i \mathbf{Z}_i ]^{-1}$.

Более эффективные оценки возможны при использовании альтернативных моментных тождеств. В частности, если отправной точкой является отдельное условное моментное тождество, то оптимальное безусловное моментное тождество для ОММ оценивания дано в разделе 6.3.7. Этому подходу соответствует оценка обобщенных оценивающих уравнений, описанная ниже. Более подробно это описано в Авери, Хансен, и Хотц (1983) и Брейтунг и Лехнер (1999).


{\centering Пример ОMM оценивания \\}

В качестве особенного примера рассмотрим преобразование <<первые разности>> в применении к модели с мультипликативными фиксированными эффектами. Начнем с условного моментного ограничения \ref{Eq:23.14}. Это приводит к большому количеству безусловных моментных тождеств, одно из которых:
\begin{align}
 \E \left[ \x_{it} \left( y_{it} - \frac{\mathrm g(\x'_{it}\be)}{\mathrm g(\x'_{i,t-1}\be)}\times y_{i,t-1} \right) \right]  = \mathbf{0}, &
& t=1, \dots, T, &
& i=1, \dots, N.
\nonumber 
\end{align}
Предположим, что нам известны данные $(y_{it},\x_{it})$ для $(T+1)$ периодов. Начальный период выпадает в связи с взятием первых разностей. Расположив в столбец $T$ наблюдений для данного $i$, получаем \ref{Eq:23.24} с $\mathbf{Z}'_i=[\x_{i1}, \dots, \x_{iT}]$ и $\mathbf{u}'_i=[u_{i1}, \dots, u_{iT}]$, где $u_{it}=y_{it}-[\mathrm g(\x'_{it}\be)/\mathrm g(\x'_{i,t-1}\be)]y_{i,t-1}$. Здесь $\mathbf{Z}'_i\mathbf{u}_i=\sum_t \x_{it}u_{it}$, поэтому оценка методом моментов $\hat{\be}$ получается из решения
\begin{align}
\sum_{i=1}^N \sum^T_{t=1} \x_{it} \left[ y_{it} - \frac{\mathrm g(\x'_{it}\be)}{\mathrm g(\x'_{i,t-1}\be)} y_{i,t-1} \right]=\mathbf{0}.
\nonumber
\end{align}
Очевидно, что можно использовать дополнительные моментные тождества, например, $\E[\x_{it-1}u_{it}]=\mathbf{0}$, что приводит к сверхиндентифицируемости модели и оценивании с помощью ОММ. Это подробно обсуждалось для линейной модели в разделе 22.2.

{\centering Оценивание  с помощью обобщенных оценивающих уравнений \\}

В модели сквозной регрессии для условного среднего $\E[y_{it}|\x_{it}]=\mathrm g(\x_{it}, \be)$ (см. разедел 23.2.4). Эту модель можно оценить с помощью уже известных ОММ методов. Но мы пойдем дальше и рассмотрим эффективное ОММ оценивание.

Расположив в столбец все $T$ наблюдений для данного $i$ получаем условное моментное тождество
\begin{align}
\E[\mathbf{y}_i-\mathbf{g}_i(\be)|\mathbf{X}_i]=\mathbf{0,}
\label{Eq:23.28}
\end{align}
где $\mathbf{g}_i (\be)=[\mathrm g(\x_{i1},\be), \dots, \mathrm g(\x_{iT},\be)]'$  и $\mathbf{X}_i=[\x_{i1}, \dots, \x_{iT}]'$. Тогда оптимальное условное моментное тождество 
\begin{align}
\E \left[ \frac{\partial \mathbf{g}'_i(\be)}{\partial\be} \{ \mathbf{V}[\mathbf{y}_i|\mathbf{X}_i] \}^{-1} (\mathbf{y}_i - \mathbf{g}_i(\be))\right]=\mathbf{0},
\label{Eq:23.29}
\end{align}
которое получено после применения общего результата раздела 6.3.7. Получаем \textbf{оценку обобщенных оценивающих уравнений} (Generalized Estimating Equations, GEE) $\hat{\be}_{\mathrm{GEE}}$, которая является решением
\begin{align}
\bm\sum^N_{i=1} \frac{\partial \mathbf{g}'_i(\be)}{\partial \be} \Sigma^{-1}_i (\mathbf{y}_i-\mathbf{g}_i(\be))=\mathbf{0},
\label{Eq:23.30} 
\end{align}
где $\bm\Sigma_i$ --- это рабочая ковариационная матрица для $\mathbf{V}[\mathbf{y}_i|\mathbf{X}_i]$. Выражение для асимптотической ковариационной матрицы $\hat{\be}_{\mathrm{GEE}}$ дано в \ref{Eq:23.26} с $\hat{\mathbf{u}}_i=\mathbf{y}_i-\mathbf{g}_i(\hat{\be})$ и $\mathbf{Z}_i=\partial \mathbf{g}'_i(\be)/ \partial\be |_{\hat{\be}} \times \hat{\bm\Sigma}_i$. Для панельных данных эта оценка дисперсии является робастной. Кроме того она робастна к неправильной спецификации $\bm\Sigma_i$.

Согласно Лянг и Цегер (1986) оценка обобщенных оценивающих уравнений широко используется в статистической литературе для обобщенных линейных моделей панельных данных. Разные обобщенные линейные модели соответствуют разным функциям условного среднего $\mathbf{g}_i(\be)$ и действующим ковариационным матрицам $\bm\Sigma_i$.

{\centering Оценивание методом максимального правдоподобия  \\}

Для моделей, основанных на максимальном правдоподобии отправная точка --- совместная плотность для всех $T$ индивидуальных наблюдений, $f(\mathbf{y}_i|\mathbf{X}_i, \bm\theta)$. Для параметрических моделей сквозной регрессии $\bm\theta'= [\bm\beta', \bm\gamma', \bm\eta']$ (см. \ref{Eq:23.18}).

Согласно стандартному подходу, $f(\mathbf{y}_i|\mathbf{X}_i,\bm\theta)=\prod^T_{t=1} f(y_{it}|\x_{it}, \bm\theta)$, где $f(y_{it}|\x_{it}, \bm\theta)$ --- это плотность для $(i,t)$-го наблюдения. Неявное предположение независимости по $t$  для данного $i$ обычно неуместно, особенно для моделей сквозной регрессии, которые не включают случайный эффект, в которых может присутствовать корреляция во времени.  Тем не менее,  оценки $\bm\theta$ состоятельны, даже  если $f(\mathbf{y}_i|\mathbf{X}_i, \bm\theta)$ неправильно специфицирована при правильно специфицированом $f(y_{it}|\x_{it}, \bm\theta)$. Чтобы получать робастные стандартные ошибки для панельных данных для оценки ковариационной матрицы следует использовать сэндвич-форму. Оценка ММП --- это, строго говоря, оценка квази-максимального правдоподобия. Подробное обсуждение см. в разделе 5.7.5. Вообще этот подход является примером статистических выводов с кластеризованными данными (см. раздел 24.5).

Можно использовать более сложную модель для $f(\mathbf{y}_i|\mathbf{X}_i, \bm\theta)$, которая учитывают корреляцию во времени. Однако ненормальные многомерные распределения $\mathbf{y}_i$ сложны для работы. Для обобщенных линейных моделей сквозной регрессии можно использовать оценку обобщенных оценивающих уравнений.

\subsection{Динамические модели}

Динамические модели с индивидуальными эффектами представляют особый интерес, так как они позволяют проводит различие между истинной зависимостью от состояния и кажущейся зависимостью, которая обусловлена ненаблюдаемой гетерогенностью (см. раздел 22.5.1).

Для нелинейных моделей не всегда очевидно, как включать лаговые зависимые переменные в качестве регрессоров, так как некоторые типы данных не всегда объясняются стандартными моделями временных рядов. Это проиллюстрировано в разделе 23.7.4 для модели Пуассона. Когда определена подходящая спецификация, стандартные оценки с фиксированными эффектами будут несостоятельны и в оценках со случайными эффектами нужно учесть начальные условия так, как это было для линейных моделей панельных данных.

{\centering Модели сквозной регрессии \\}

Модели сквозной регрессии игнорируют случайные эффекты и оценивают обычные модели пространственных данных, где в регрессоры включены лаговые зависимые переменные. См. обсуждение в разделе 23.2.4.

{\centering Модели с фиксированными эффектами \\}

Вопросы моделей с фиксированными эффектами близки к представленным в разделе 22.5. Сейчас регрессоры не строго, а слабо экзогенны. Обычные оценки с фиксированными эффектами несостоятельны.

Для моделей с аддитивными эффектами или мультипликативными эффектами можно получить состоятельные оценки, используя преобразование <<взятие первых разностей>> (см. раздел 23.2.2), а также лаги зависимых переменных более высокого порядка в качестве инструментов. Для моделей с аддитивными эффектами это приводит к нелинейной версии оценки Ареллано-Бонда, представленной в разделе 22.5.3. Для мультипликативных эффектов преобразование <<взятие первых разностей>> подробно описано в разделе 23.7.4. Для  динамической логит-модели с фиксированными эффектами см. раздел 23.4.3.

{\centering Параметрические модели со случайными эффектами \\}

Для параметрических моделей со случайными эффектами имеют значение начальные условия лаговых зависимых переменных. Обычно в коротких панелях оценки несостоятельны. При этом несостоятельность уменьшается при увеличении $T$.

Рассмотрим самый простой случай, когда в модели присутствует лаг только первого периода, поэтому в качестве регрессоров выступают $\x_{it}$ и $y_{it-1}$. Плотность случайных эффектов \ref{Eq:23.1} будет $f(y_{it}|y_{it-1}, \x_{it}, \ali, \bm\delta)$ для $t=2, \dots, T$. Однако похожая модель для $y_{i1}$ не может быть использована, так как $y_{i0}$ не наблюдаем. Один из способов --- считать $y_{i1}$ экзогенной переменной, так что мы моделируем условное распределение только для $T-1$ наблюдений $y_{it}, \dots, y_{i2}$. Альтернативный подход предполагает статическую модель для $y_{i1}$, которое зависит от регрессоров $\x_{i1}$ и возможно от предельного эффекта $\ali$. Тогда совместная условная плотность $\mathbf{y}_i$ 
\begin{align}
f(\mathbf{y}_i|\x_{i1}, \dots, \x_{iT}, \ali, \bm\delta, \bm\delta_1, \bm\gamma) \nonumber \\
& =\int \left[ \prod^T_{t=2} f(y_{it}|y_{it-1}, \x_{it}, \ali, \bm\delta) \right] f_1(y_{i1}|\x_{i1}, \ali, \bm\delta_1) \mathrm g(\ali|\bm\gamma) d \ali,
\nonumber
\end{align}
а не \ref{Eq:23.18}, где $f_1(y_{i1}|\x_{i1}, \ali, \bm\delta_1)$ --- это предполагаемая плотность для первого наблюдения.

В анализе чисто временных рядов начальные условия не имеют значения в асимптотике, так как $T \rightarrow \infty$. В коротких панелях, однако, они имеют очень большое значение, так как $T$ маленькое и асимптотика обусловлена тем, что $N \rightarrow \infty$. 

\subsection{Эндогенные регрессоры}

Эндогенные переменные  в нелинейных моделях учитываются схожим образом, как было представлено в главе 22.

Естественный подход --- это панельный ОММ. Для начала определяется условное моментное ограничение $\E[\mathbf{u}_i(\bm\theta)|\mathbf{Z}_i]=\mathbf{0}$ для соответствующим образом определенных остатков $\mathbf{u}_i(\bm\theta)$ и инструментов $\mathbf{Z}_i$. Это приводит к безусловному моментному тождеству \ref{Eq:23.24}, которое лежит в основе ОММ оценивания. Возможными кандидатами на инструменты могут быть в том числе экзогенные регрессоры других периодов, как это было описано в разделе 22.2 и 22.4 для линейной модели.

\section{Пример нелинейной модели панельных данных: Патенты и НИОКР}

Мы будем моделировать зависимость между патентами и затратами на НИОКР, используя американские данные о 346 фирмах для 1975-1979 гг. из Холл, Грилихес, и Хаусман (1986). Зависимая переменная $y_{it}$ --- это количество патентов, определенное как количество поданных заявок на патенты в течение года, которые были в конце концов одобрены. Для простоты мы рассмотрим только одну объясняющую переменную $x_{it}$, реальные затраты на НИОКР в течение года (в ценах 1972 г.).

Рисунок 23.1: Патенты и затраты на НИОКР: сквозная регрессия. Натуральный логарифм количества успешных заявок против натурального логарифма затрат на НИОКР для 346 фирм в 1975-79 гг. Нулевое количество патентов было записано как 0.5.

Очевидно, первая модель --- модель в логарифмах с $\E[\mathrm{ln} y_{it}|x_{it}]=\ali+\beta \mathrm{ln} x_{it}$, так как в таком случае $\beta$ равно эластичности количества патентов по затратам на НИОКР. Эту модель здесь нельзя применять, так как $y_{it}=0$ для рассматриваемого количества наблюдений и $\mathrm{ln 0}$ не определено. Здесь делаем корректировку: перед взятием логарифмов $y_{it}=0$ заменяем на $y_{it}=0.5$.

На рисунке 23.1 изображен график зависимости ln(Patents) от ln(R\& D). Наряду с наблюдениями изображена МНК регрессия (с оцененным коэффициентом наклона 0.834) и кривая непараметрической регрессии. Использовались данные обо всех фирмах. Очевидно, число патентов увеличивается по мере увеличения затрат на НИОКР. С помощью анализа панельных данных, в особенности модели с фиксированными эффектами, можно разделить эту взаимосвязь на пространственную компоненту и временную компоненту. Заметим, что число патентов меняется значительно в зависимости от наблюдения, т.е. фирмы. Среднее число патентов --- 36.3, стандартное отклонение --- 74.5. Число патентов ранжируется от 0 до 608.

Мы оцениваем модель с мультипликативными индивидуальными эффектами для условного среднего с 
\begin{align}
\E[y_{it}|x_{it}, \ali] = \ali \exp (\beta \mathrm{ln} x_{it})=\exp  (\gamma_i+\beta \mathrm{ln} x_{it})
\label{Eq:23.31}
\end{align}
где $\gamma_i=\mathrm{ln} \ali$. Тогда $\beta$ является оценкой эластичности патентов по затратам на НИОКР, так как из \ref{Eq:23.31} следует $\partial \mathrm{ln} \E[y_{it}|x_{it}]/\partial \mathrm{ln} x_{it}=\beta$. В отличие от модели в логарифмах, нулевые значения $y_{it}$ не составляют проблему.

В более сложных параметрических моделях зависимая переменная счетная. Начнем с модели Пуассона:
\begin{align}
y_{it}|x_{it}, \gamma_i \thicksim \mathcal{P}[\exp (\gamma_i + \beta \mathrm{ln} x_{it})].
\label{Eq:23.32}
\end{align}
Эта модель, подробно описанная в разделе 23.7, имеет то же условное среднее для $y_{it}$, как и в \ref{Eq:23.31}.

В таблице \ref{Tab:23.1} представлены оценки для этих данных. Все оценки состоятельны при условии, что условное среднее в \ref{Eq:23.31} со случайным эффектом $\ali$, независимым от $\x_{it}$ и имеет постоянное среднее. Все оценки за исключением последней состоятельны при условии, что $\ali$ --- это фиксированный эффект, который коррелирован с $\x_{it}$. Имеется три вида оценок стандартных ошибок: вычисляемые программой по умолчанию, робастные оценки для панельных данных (где возможно) и оценки бутстрапа (без уточнения). Более подробно для каждой колонки:

\begin{table}[ht]
\caption{{\it Патенты и затраты на НИОКР: Оценки нелинейных моделей панельных данных}$^a$} 
\centering
\begin{tabular}{c|c|c|c|c|c}
\hline \hline
  & \textbf{NLS} & \textbf{Пуассон} & \textbf{GEE} & \textbf{Пуассон-RE} & \textbf{Пуассон-FE} \\ 
\hline 
$\gamma=\mathrm{ln}\alpha$ & 2.529 & 1.712 & 2.068 & 2.313 & --- \\ 
$\beta$ & .509 & .693 & .560 & .349 & -0.038 \\  
Panel с.о.$^\beta$ & (.055) & (.043) & (.033) & (.033) & (.033) \\ 
Бутстреп с.о. & [.054] & [.047] & [.107] & [.119] & {.107} \\ 
Обычные с.о. & {.011} & {.002} & {.004} & {.033} & {.033} \\ 
Сумма $\beta$ & --- & .486 & .460 & .546 & .313 \\ 
N & 1730 & 1730 & 1730 & 1730 & 1620 \\ 
\hline \hline
\multicolumn{6}{p{15cm}}{${}^a$ В таблице представлены оценки нелинейного МНК сквозной регрессии, Пуассона сквозной регрессии, GEE оценка сквозной регрессии, Пуассона со случайными эффектами (RE) и Пуассона с фиксированными эффектами для нелинейной регрессии \ref{Eq:23.31} ln(Patents) на ln(R\&D). В круглых скобках представлены робастные стандартные ошибки для коэффициентов наклона, в квадратных скобках --- стандартные ошибки бутстреп, в фигурных скобках --- обычные стандартные ошибки, которые предполагают независимые и одинаково распределенные ошибки. В предпоследней строчке дана сумма коэффициентов $\beta$ в расширенной модели, которая включает до пяти лагов $\mathrm{ln}(R \& D)$ в качестве регрессоров.} \\
\multicolumn{6}{p{15cm}}{$^b$ с.о. - стандартные ошибки}
\end{tabular} 
\label{Tab:23.1}
\end{table}

\textbf{NLS сквозной регрессии}: Оценки нелинейного МНК в первой колонке получаются в результате оценивания \ref{Eq:23.31} при $\ali=\alpha$. Нелинейный МНК описан в  разделе 5.8. Стандартные ошибки, вычисленные по умолчанию (их значение равно 0.011), в предположении о независимых и одинаково распределенных ошибках намного меньше, чем правильные робастные стандартные ошибки 0.054.

\textbf{Пуассона сквозной регрессии:} Оценки Пуассона во второй колонке --- для модели Пуассона \ref{Eq:23.32} c $\ali=\alpha$ с помощью ММП Пуассона в предположении о независимости по $i$ и $t$. Эластичность равна 0.693 по сравнению с оценкой нелинейного МНК равной 0.509. Стандартные ошибки, вычисленные по умолчанию (равные 0.002), удовлетворяют равенству дисперсии и среднего в распределении Пуассона (см. раздел 20.2.2). После учёта избыточной дисперсии с помощью оценки ковариационной матрицы в сэндвич-форме  (см. также раздел 20.2.2) оценка стандартной ошибки увеличилась до 0.020, что подтверждает необходимость учета избыточной дисперсии в счетных данных. Дополнительный учет корреляции по $t$ для данного $i$ приводит даже к более высоким робастным оценкам стандартных ошибок 0.043.

\textbf{GEE сквозной регрессии:} Оценка обобщенных оценивающих уравнений сквозной регрессии --- это решение \ref{Eq:23.30}, где $\mathrm g(\x_{it}, \be)$ определено в \ref{Eq:23.32} с $\ali=\alpha$. Используемая здесь спецификация действующей матрицы $\bm\Sigma_i$ дана после \ref{Eq:23.55}. Оцененная эластичность равна 0.560 со стандартной ошибкой 0.033, полученной с помощью робастной оценки, обсуждаемой после \ref{Eq:23.30}.

\textbf{Пуассон-RE}: Оценка Пуассона со случайными эффектами предполагает, что $\ali=\mathrm{ln} \gamma_i$ имеет гамма-распределение (см. раздел 23.7.2). Оцененная эластичность равна 0.349 со стандартными ошибками равными 0.033.

\textbf{Пуассон-FE}: Оценка Пуассона с фиксированными эффектами предполагает, что $\ali=\mathrm{ln} \gamma_i$ --- это фиксированный эффект, и он оценивается как в разделе 23.7.3. Оцененная эластичность сейчас отрицательна и равна -0.038, а стандартные ошибки по умолчанию равны 0.033. Для модели Пуассона с фиксированными эффектами исключены фирмы с $\sum_t y_{it}=0$, что привело к потере $22 \times 5=110$ наблюдений.

Есть большая разница между результатами с фиксированными и случайными эффектами, причем фиксированные эффекты более предпочтительны. Оцененная эластичность в модели с фиксированным эффектом имеет неожиданно отрицательный знак из-за того, что модель слишком проста. В частности, затраты на НИОКР влияют на патентную активность с лагом. Если заменить $\beta \mathrm{ln} \x_{it}$ в \ref{Eq:23.31} и \ref{Eq:23.32} на $\sum^5_{l=0} \beta_l \mathrm{ln} \x_{i,t-1}$, то оцененная эластичность будет равна $\sum^5_{l=0} \hat{\beta}_l$, она приведена в предпоследнем ряду таблицы 23.1. Оценка с фиксированными эффектами, равная 0.313, все еще меньше других оценок, но уже ненамного.

\section{Данные бинарного выбора}

Рассмотрим бинарный выбор, в котором $y_{it}$ принимает значения 0 или 1. Например, нам может быть известно, занят ли индивидуум на рынке труда в каждый из наблюдаемых периодов или нет. Ключевым результатом здесь является то, что оценивание фиксированных эффектов возможно только с помощью логит-модели, и невозможно с помощью пробит.

\subsection{Модели бинарного выбора с индивидуальными эффектами}

Естественный способ расширить модель бинарного выбора для пространственных данных (см. раздел 14.3) до модели панельных данных с индивидуальными эффектами --- специфицировать, что $y_{it}$ принимает только значения 0 и 1.
\begin{align}
\P [y_{it}=1|\x_{it}, \be, \ali]= 
\begin{cases}
& F (\ali+\x'_{it} \be) \text{в общем случае} \\
& \Lambda(\ali+\x'_{it}\be) \text{для логит-модели} \\
& \Phi(\ali+\x'_{it}\be) \text{для пробит-модели}
\end{cases}
\label{Eq:23.33}
\end{align}
где $F(\cdot)$ --- это  функция распределения, $\Lambda(\cdot)$  --- логистическая  функция распределения с $\Lambda(\mathrm{z})=e^{\mathrm{z}}/(1+e^{\mathrm{z}})$ и $\Phi(\cdot)$ --- функция стандартного нормального распределения. Зная \ref{Eq:23.33} и предполагая условную независимость, совместная вероятность для $i$-го наблюдения $\mathbf{y}_i=(y_{i1}, \dots, y_{iT})$ будет иметь вид

\begin{align}
f(\mathbf{y}_i|\mathbf{X}_i, \ali, \be)=\prod^T_{t=1} F(\ali+\x'_{it}\be)^{y_{it}} (1-F(\ali+\x'_{it}\be))^{1-y_{it}}.
\label{Eq:23.34}
\end{align}

Для бинарных данных условная вероятность --- это также условное среднее, т.е.
\begin{align}
\E[y_{it}|\ali, \x_{it}]=F(\ali+\x'_{it}\be).
\label{Eq:23.35}
\end{align}
Это одноиндексная модель с индивидуальными эффектами (см. \ref{Eq:23.5}), которая не упрощается ни до модели с аддитивными эффектами, ни до модели с мультипликативными эффектами. Модели с аддитивными или мультипликативными эффектами не являются подходящими, так как они не ограничивают условное среднее и условную вероятность так, чтобы они лежали между 0 и 1.

В моделях бинарного выбора большое значение придается параметрической модели \ref{Eq:23.34}, так как бинарные данные должны иметь распределение Бернулли. Модель условного среднего \ref{Eq:23.35} используется достаточно редко, хотя она применяется в случае эндогенных регресоров. 

\subsection{Модели бинарного выбора со случайными эффектами}

Оценка максимального правдоподобия со случайным эффектом предполагает, что индивидуальные эффекты нормально распределены с $\ali \thicksim \mathcal{N} [0, \sigma^2_{\alpha}]$. Для получения \textbf{оценки максимального правдоподобия со случайным эффектом} $\be$ и $\sigma^2_{\alpha}$ максимизируется логарифм функции правдоподобия $\sum^N_{i=1} \mathrm{ln} f(\mathbf{y}_i|\mathbf{X}_i, \be, \sigma^2_{\alpha})$, где 
\begin{align}
f(\mathbf{y}_i|\mathbf{X}_i, \be, \sigma^2_{\alpha}) = \int f(\mathbf{y}_i|\mathbf{X}_i, \ali, \be) \frac{1}{\sqrt{2 \pi \sigma^2_{\alpha}}} \exp  \left( \frac{-\ali}{2\sigma^2_{\alpha}} \right) ^2 d \ali,
\label{Eq:23.36}
\end{align}
где $f(\mathbf{y}_i|\mathbf{X}_i, \ali, \be)$ дано в \ref{Eq:23.34} с $F=\Lambda$ для логит-модели и $F=\Phi$ для пробит-модели. Не существует явной формулы для интеграла \ref{Eq:23.36}. Обычно его вычисляют с помощью численных методов, используя, например, \textbf{метод Гаусса}.

Если фиксированных эффектов нет, то в качестве альтернативной модели со случайными эффектами  можно использовать модель бинарного выбора сквозной регрессии, в которой $\P  [\mathbf{y}_{it}=1|\x_{it}]=F(\x'_{it}\be)$. Статистические выводы тогда должны основываться на робастных стандартных ошибках (см. раздел 23.2.6). Получить более эффективные оценки можно, используя ОММ подход (см. Авери и др., 1983) или подход обобщенных оценивающих уравнений (см. Лянг и Цегер, 1986).

\subsection{Логит-модель с фиксированными эффектами}

Оценить фиксированные эффекты для логит-модели панельных данных можно, используя оценку условного максимального правдоподобия. Однако этот метод не подходит для других моделей бинарных данных, таких как пробит-модель панельных данных.

После алгебраических преобразований раздела 23.4.5 совместная плотность $\mathbf{y}_i=(y_{i1}, \dots, y_{iT})$ для логит-модели равна
\begin{align}
f(\mathbf{y}_i|\ali, \x_i, \be) = \frac{\exp  \left(\ali \sum_t y_{it} \right) \exp  \left( \left( \sum_t y_{it} \x'_{it} \right) \be \right)}{\prod_t [1+ \exp (\ali+\x'_{it} \be) ]}.
\label{Eq:23.37}
\end{align}
Плотность зависит от $\ali$, которое должно быть устранено из модели. Для наблюдения $i$ есть $\sum_t y_{it}$ исходов равных 1 в $T$ периодах. Определим набор $\mathbf{B}_c=\{\mathbf{d}_i|\sum_t d_{it}=\sum_t y_{it}=c\}$ как набор всех возможных последовательностей нулей и единиц, для которых сумма $T$ бинарных исходов $\sum_t y_{it}=c$. Тогда, если мы включим в условие $\sum_t y_{it}=c$, то, как показано в разделе 23.4.6,  $\ali$ будет устранено и 
\begin{align}
f(\mathbf{y}_i|\sum_t y_{it} = c, \x_i, \be) = \frac{\exp \left(\left(\sum_t y_{it} \x'_{it} \right) \be \right)}{\sum_{\mathbf{d} \in \mathbf{B}_c} \mathbf{exp} \left( \left( \sum_t d_{it} \x'_{it} \right) \be \right)},
\label{Eq:23.38}
\end{align}
согласно результату, полученному в работе Чемберлина (1980). Плотность \ref{Eq:23.38} --- это основа для оценивания условным ММП. Единственная сложность состоит в том, что наборов $\mathbf{B}_c$ и последовательностей внутри них много.

Во-первых, невозможно использовать $\sum_t y_{it}=0$ как условие, так как оно выполняется только в том случае, когда все $y_{it}=0$. Аналогично для  $\sum_t y_{it}=T$. Это может означать большую потерю наблюдений если, например, большинство людей заняты во все периоды времени.

Проиллюстрируем пример, где можно считать условную вероятность. Предположим, что $T=2$ и $\sum_t y_{it}=1$. Тогда возможны две последовательности: ${0, 1}$ и ${1, 0}$, и из условной вероятности в \ref{Eq:23.38} следует, например, что
\begin{align}
\P [y_{i1}=0, y_{i2}=1 | y_{i1}+ y_{i2}=1]
&=\frac{\exp (\x'_{i1}\be}{\exp (\x'_{i1}\be)+\exp (\x'_{i2}\be)} \nonumber \\
& = \frac{\exp ((\x'_{i1}-\x_{i0})'\be}{1+\exp ((\x_{i1}-\x_{i0}'\be)}.
\nonumber
\end{align} 
Если $T=3$, то мы можем поставить условие $\sum_t y_{it}=1$ с возможными последовательностями ${0, 0, 1}$, ${0, 1, 0}$ и ${1, 0, 0}$, либо $\sum_t y_{it}=2$ с возможными последовательностями ${0, 1, 1}$, ${1, 0, 1}$ и ${1, 1, 0}$. Для большого $T$ возможно много последовательностей, и условная вероятность может оказаться очень сложной.

Это условная вероятность --- вероятность из условной логит-модели, где параметры неизменны, а регрессоры меняются в зависимости от выбранной альтернативы. Количество альтернатив меняется в зависимости от индивидуума, где для индивидуума $i$ каждая альтернатива --- это специфическая последовательность нулей и единиц, сумма элементов которой равна $\sum_t y_{it}$. Самый простой путь --- это использовать компьютерный код, созданный специально для данной задачи. Даже в такой случае может быть большое количество альтернатив. Например, если $T=10$ и $\sum_t y_{it}=5$, то у нас 252 альтернативы. Получить состоятельные, но менее эффективные оценки можно, удалив некоторые наблюдения, например, для индивидуумов с большим числом альтернатив из-за большого $\sum_t y_{it}$. Либо можно уменьшить число периодов.

Из-за устранения индивидуальных эффектов $\ali$ интерпретировать коэффициенты регрессии, используя модель первоначальную модель \ref{Eq:23.37}, невозможно. Вместо этого, мы используем условную модель \ref{Eq:23.38}. Например, предположим, что у нас один регрессор и $\beta=0.2$. Тогда, если мы рассматриваем два временных периода и условие $\sum_t y_{it}=1$, то
\begin{align}
\P [y_{i1}=0, y_{i2}=1| y_{i1}+y_{i2}=1]= 
\frac{\exp (\beta(x_{i1}-x_{10}))}{1+\exp (\beta(x_{i1}-x_{10}))}.
\nonumber
\end{align}
Если $x_{i1}$ и $x_{i2}$ различаются на единицу, то условная вероятность этой последовательности равна $\exp (\beta)/[1+\exp  (\beta)]$, если $x_{i1}=x_{i2}$, то вероятность равна одной второй. 

\subsection{Динамические модели бинарного выбора}

Предположим, что у нас есть логит-модель временных рядов, марковский процесс первого порядка, где имеется только один регрессор --- лаг зависимой переменной:
\begin{align}
\P [y_{it}=1|\ali, y_{it-1}]= \frac{\exp (\ali+\gamma y_{it-1})}{1+\exp (\ali+\gamma y_{it-1})}.
\label{Eq:23.39}
\end{align}
Выполнив некоторые алгебраические преобразования раздела 23.4.5, получаем
\begin{align}
f(\mathbf{y}_{it}|y_{i1}, y_{iT}, \sum^{T-1}_{t=2} y_{it}, \gamma)= 
\frac{\exp (\gamma \sum^{T-1}_{t=2} y_{it} y_{it-1})}{\sum_{\mathbf{d} \in \mathbf{C}_i}\exp (\gamma \sum^{T-1}_{t=2} d_{it} d_{it-1})},
\label{Eq:23.40}
\end{align}
где набор $\mathbf{C}_i=\{ \mathbf{d}_i|y_{i1}, y_{iT}, \sum_t d_{it}=\sum_t y_{it} \}$ --- это набор всех возможных последовательностей нулей и единиц, для которой сумма $T$ бинарных исходов равна $\sum_t y_{it}$, первое значение --- $y_{i1}$, последнее значение --- $y_{iT}$.

Оценивание условным ММП, основанным на \ref{Eq:23.40}, даст состоятельные оценки $\gamma$. Минимальное необходимое количество временных периодов --- четыре. Например, если $\mathbf{y}_i$ --- это последовательность $\{ 0, 1, 0, 1\}$, то набор $\mathbf{C}_i$ состоит из последовательностей $\{0, 1, 0, 1\}$ и $\{0, 0, 1, 1\}$. Этот подход предложил Чемберлин (1985). Он рассматривал марковскую модель второго порядка. Чей, Хойнес, и Хислоп (2001) применяли этот метод к данным по административным округам Калифорнии о благосостоянии и выяснили, что учет ненаблюдаемой гетерогенности позволяет выявить истинную зависимость благосостояния.

Предыдущие результаты и обсуждение относились к моделям чистых временных рядов. Оноре и Кирьязиду (2000) предложили метод, который позволяет включать в качестве регрессоров не только лаговые зависимые переменные. Так \ref{Eq:23.39} принимает вид
\begin{align}
\P [y_{it}=1|\ali, y_{it-1}, \x_{it}]=
\frac{\exp (\ali+\x'_{it}\be + \gamma y_{it-1})}{1+\exp (\ali+\x'_{it}\be+\gamma y_{it-1})}.
\label{Eq:23.41}
\end{align}
Рассмотрим четыре временных периода и последовательности с одинаковыми значениями в первом и четвертом периоде, $d_1$ и $d_4$. Тогда вероятность того, что это последовательность $\{d_1, 0, 1, d_4 \}$ сейчас зависит от $\ali$, при условии, что это либо $\{d_1, 0, 1, d_4 \}$, либо $\{d_2, 0, 1, d_4 \}$. Однако зависимость от $\ali$ исчезает, если $x_{3i}=x_{4i}$. Из-за того, что для некоторых наблюдений $x_{3i}=x_{4i}$, особенно в случае с непрерывными данными, Оноре и Кирьязиду (2000) предложили использовать методы ядерного сглаживания с весами, которые зависят от $(x_{3i}-x_{4i})$. Чей и Хислоп (2000) применяли этот и другие методы для динамических моделей бинарного выбора.

\subsection{Модель множественного выбора}

Оценка с фиксированным эффектом может быть расширена до логит-модели множественного выбора, так как эта модель заключает в себе логит-модель бинарного выбора для попарного сравнения альтернатив (см. раздел 15.4.3). Чемберлин (1980) кратко описывает статическую модель, М.-Дж. Ли (2002) представляет более подробное описание. Маньяк (2000), используя динамическую логит-модель с фиксированными эффектами с лаговыми зависимыми в качестве регрессоров, подробно описывает применение этой модели к анализу индивидуального карьерного движения по шести различным ступеням на французском рынке труда. Оноре и Кирьязиду (2000) рассматривают логит-модель множественного выбора.

Для других моделей множественного выбора необходимо использовать случайные эффекты. Эти модели, такие как смешанные логит-модель или пробит-модель множественного выбора, достаточно сложны даже в случае пространственных данных. Более подробное описание см. Трейн (2003).

\subsection{Выводы для логит-модели фиксированных эффектов}

Для простоты опустим индекс $i$. Для логит-модели совместная вероятность  $\mathbf{y}=(y_1, \dots, y_T)$, которая дана в \ref{Eq:23.34}, равна
\begin{align}
f(\mathbf{y}|\bm\alpha)
& =\prod^T_{t=1} \left( \frac{\exp (\alpha+\x'_t \be)}{1+\exp (\alpha+\x'_t\be)} \right) ^{y_t} 
\left( \frac{1}{1+\exp (\alpha+\x'_t\be)}\right)^{1-y_t} 
\label{Eq:23.42}\\
& = \frac{\exp (\sum_t y_t(\alpha+\x'_t \be))}{\prod_t [1+\exp (\alpha+\x'_t \be)]} \nonumber \\
& = \frac{\exp (\alpha \sum_t y_t ) \exp ((\sum_t y_t \x'_t)\be)}{\prod_t [1+\exp (\alpha+\x'_t \be)]}, \nonumber
\end{align}
что в результате дает \ref{Eq:23.37}.

Можно показать, что количество $\sum_t y_t$ может быть достаточной статистикой для $\alpha$. Предположим, у нас есть наблюдение для $\mathbf{y}$, такое что $\sum_t y_t = c$. Определим набор $\mathbf{B}_c = \{ \mathbf{d}| \sum_t d_t = c\}$ как набор всех возможных последовательностей нулей и единиц, для которых сумма $T$ бинарных исходов  равна $c$ при условии $\sum_t y_t = c$. Тогда
\begin{align}
f(\mathbf{y}|\sum_t y_t = c)
& = \frac{\P [\mathbf{y}, \sum_t y_t = c]}{\P [\sum_t y_t = c]}
\label{Eq:23.43} \\
& = \frac{\P [\mathbf{y}]}{\P [\sum_t y_t = c]} \nonumber \\
& = \frac{\P [\mathbf{y}]}{\sum_{\mathbf{d} \in \mathbf{B}_c} \P [\mathbf{d}]} \nonumber \\
& = \frac{\exp ((\sum_t y_t \x'_t) \be )}{\sum_{\mathbf{d} \in \mathbf{B}_c} \exp ((\sum_t d_t \x'_t) \be ) }  \nonumber ,
\end{align}
где для получения первого равенства используется правило Байеса, для второго равенства используется тот факт, что знание $\sum_t y_t$ не несет дополнительной при известном $\mathbf{y}$. В третьем равенстве используется тот факт, что $\P  [\sum_t y_t =c]$ равна сумме вероятностей конкретных комбинаций из нулей и $c$ единиц. В четвертом используется определение $f(\mathbf{y})$ и упрощение, которое базируется на том, что $\sum_t y_t = \sum_t d_t$, когда $\mathbf{d} \in \mathbf{B}_c$.

Сейчас рассмотрим динамическую модель. Заменяя $\x'_t \be$ в \ref{Eq:23.42} на $\gamma y_{t-1}$, получаем 
\begin{align}
f(\mathbf{y})
&= \frac{\exp (\alpha \sum^T_{t=2} y_t) \exp ( \sum^T_{t=2}\gamma y_{t-1} y_t)}{\prod_t [1+\exp (\alpha+\gamma y_{t-1})]} \nonumber \\
&= \frac{\exp (\alpha \sum^T_{t=2} y_t )\exp (\sum^T_{t=2} \gamma y_{t-1} y_t )}{[1+\exp (\alpha)]^{\sum^T_{t=2} (1-y_{t-1})}[1+\exp (\alpha+\gamma)]^{\sum^T_{t=2} y_{t-1}}} \nonumber \\
&= \frac{\exp (\alpha \sum^T_{t=2} y_t) \exp (\sum^T_{t=2} \gamma y_{t-1} y_t)}{[1+\exp (\alpha)]^{(T-1+y_1-y_T+\sum^T_{t=2} y_t)+}[1+\exp (\alpha+\gamma)]^{y_1-y_T+\sum^T_{t=2} y_{t}}}, \nonumber
\end{align}
где во втором равенстве используются некоторые алгебраические преобразования и тот факт, что $y_{t-1}$ принимает значение либо 0, либо 1, а в последнем выражении используется $\sum^T_{t=2} y_{t-1} = y_1 - y_T + \sum^T_{t=2} y_t$. Здесь используются те же алгебраические преобразования, что и в \ref{Eq:23.43}, за исключением того, что помимо условия $\sum^T_{t=2} y_t$ нужно включить в условие  еще $y_1$ и $y_T$ , которые появляются в знаменателе. Мы можем включить в условие $\sum^T_{t=1} y_t$, $y_1$, $y_T$, что будет эквивалентно. Таким образом, получаем
\begin{align}
f(\mathbf{y})=\frac{\exp (\sum^T_{t=2} \gamma y_{t-1} y_t)}{\sum_{\mathbf{d} \in \mathbf{C}_c} \exp (\sum^T_{t=2} \gamma d_{t-1} d_t)},
\nonumber
\end{align}
где $\mathbf{C}=\{ \mathbf{d}| d_1 = y_1, d_T = y_T, \sum^T_{t=1} d_t = \sum^T_{t=1} y_t \}$ --- это набор всех возможных последовательностей нулей и единиц, для которых сумма бинарных исходов $T$ равна $\sum_t y$, первый исход --- $y_1$, последний исход --- $y_T$.

\section{Тобит-модель и модели самоотбора}


Мы рассмотрим цензурированные и урезанные выборки, а также модели самоотбора, когда в нашем распоряжении имеются панельные данные, а не данные пространственного типа.

Анализ сквозной регрессии просто повторяет анализ случая с пространственными данными, лишь с той корректировкой, что нужно использовать робастные стандартные ошибки для панельных данных (см. раздел 23.2.8). Например, см. Грасдал (2001), где рассматриваются модели самоотбора, появившегося в результате истощения панели.

Здесь мы фокусируем внимание на моделях панельных данных с индивидуальными эффектами. Затем можно оценить модели со случайными эффектами при условии выполнения сильного предположения о чисто случайных эффектах. Трудность при оценке моделей со случайными эффектами может состоять лишь в численных вычислениях. Нет простых состоятельных оценок для моделей с фиксированными эффектами, кроме  обычного микроэконометрического случая короткой панели. Более сложные полупараметрические оценки, которые позволяют включать фиксированные эффекты в тобит-модель и обощенную тобит-модель, представлены в разделе 23.8.

\subsection{Модели с цензурированными и урезанными выборками}

Для пространственных данных тобит-модель с цензурированной выборкой представлена в разделе 16.3.1. Версия для панельных данных с индивидуальными эффектами:
\begin{align}
y^*_{it}=\ali+\x'_{it}\be+\e_{it},
\label{Eq:23.44}
\end{align}
где $\e_{it} \thicksim \mathcal{N}[0,\sigma^2_{\e}]$, мы наблюдаем $y_{it}=y^*_{it}$, если $y^*_{it} > 0$ и $y_{it}=0$ или наблюдаем пропуск, если $y^*_{it} \leq 0$. Совместная плотность для $i$-го наблюдения $\mathbf{y}_i=(y_{i1}, \dots, y_{iT})$ можно переписать как
\begin{align}
f(\mathbf{y}_i| \mathbf{X}_i, \ali, \be, \sigma^2_{\e})=\prod^T_{t=1} [\frac{1}{\sigma_{\e}} \phi_{it}]^{d_{it}} [1-\Phi_{it}]^{1-d_{it}},
\label{Eq:23.45}
\end{align}
где $\phi_{it}=\phi((y_{it}-\ali-\x'_{it}\be)/\sigma_{\e})$, $\Phi_{it}=\Phi ((\ali+\x'_{it}\be)/\sigma_{\e})$, и $\phi(\cdot)$ и $\Phi(\cdot)$ обозначают соответственно функцию плотности и функцию распределения стандартной нормальной величины.

Оценка максимального правдоподобия с фиксированным эффектом, основанная на плотности \ref{Eq:23.45}, максимизирует логарифм функции максимального правдоподобия по $\be, \sigma^2_{\e}$ и $\alpha_1, \dots, \alpha_N$. В коротких панелях оценка $\be$ несостоятельна ввиду проблемы второстепенных параметров, и не существует  простого метода взятия разностей или введения условия, который может обеспечить состоятельную оценку. Хекман and МакКарди (1980) применяют оценку максимального правдоподобия с фиксированным эффектом к анализу рынка труда женщин. Хотя они признают несостоятельность оценки, они показывают, что при $T=8$ несостоятельность может быть невелика. Грин (2004a) изучает оценку максимального правдоподобия тобит-модели с фиксированным эффектом с помощью метода Монте Карло.

Оценка со случайным эффектом чаще используется ввиду несостоятельности оценки с фиксированным эффектом. При выполнении предположения, что $\ali \thicksim \mathcal{N} [0, \sigma^2_{\alpha}]$ \textbf{оценка ММП со случайным эффектом} $\be, \sigma^2_{\e}$ и $\sigma^2_{\alpha}$ максимизирует логарифм функции правдоподобия $\sum^N_{i=1} \mathrm{ln} f(\mathbf{y}_i |\mathbf{X}_i, \be, \sigma^2_{\e}, \sigma^2_{\alpha})$, где
\begin{align}
f(\mathbf{y}_i|\mathbf{X}_i, \be, \sigma^2_{\e}, \sigma^2_{\alpha})=
\int f(\mathbf{y}_i | \mathbf{X}_i, \ali, \be, \sigma^2_{\e}) 
\frac{1}{\sqrt{2 \pi \sigma^2_{\alpha}}} \exp  (\frac{-\ali}{2 \sigma^2_{\alpha}})^2 d \ali,
\label{Eq:23.46}
\end{align}
для $f(\mathbf{y}_i |\mathbf{X}_i, \ali, \be, \sigma^2_{\e})$, представленного в \ref{Eq:23.45}. Этот однократный интеграл можно вычислить с помощью \textbf{метода Гаусса}.

Этот подход можно расширить до других моделей, в которых используются цензурированные и урезанные выборки. Например, модель Пуассона со случайными эффектами с цензурированной справа выборкой в разделе 23.7.2 можно использовать, если число событий выше 10 записывается только как 10 или выше.

У полного параметрического подхода есть два недостатка. Во-первых, как и в случае с пространственными данными, зависимость от предположений о распределении намного выше, когда выборка урезана или цензурирована. Во-вторых, предположение о чисто случайных эффектах, независимых от регрессоров, может оказаться чересчур сильным. 

\subsection{Модели самоотбора}

Модели самоотбора могут использоваться для панельных данных по тем же причинам, что и в случае пространственных данных (см. раздел 16.5). Обобщение \textbf{тобит-модели второго типа} в разделе 16.5.1  до линейной модели панельных данных с индивидуальными эффектами $\lambda_i$ и $\delta_i$: 
\begin{align}
y^*_{it} = \ali + \x'_{it} be +\e_{it} 
\label{Eq:23.47} \\
d^*_{it} = \delta_i+\mathbf{z}'_{it} \gamma + v_{it}, \nonumber
\end{align}
где $y_{it}=y^*_{it}$ наблюдается, если $d^*_{it} >0$, и $y_{it}$ не наблюдается в противном случае.

Для формулирования случайных эффектов предполагается, что четыре ненаблюдаемые переменные имеют нормальное распределение. 
Хаусман и Уайз (1979) предложили оценивание ММП, которое включает в себя вычисление двойного интеграла, так как $\ali$ может коррелировать с $\delta_i$, и $\e_{it}$ может коррелировать с $v_{it}$.

Оценка с фиксированными эффектами состоятельна в коротких панелях.  Заметим, однако, что если $d^*_{it}=\delta_i$, то выбор объясняется неизменными во времени характеристиками индивидуума, которые могут быть наблюдаемыми или не наблюдаемыми. В таком случае оценка с фиксированным эффектом в модели $y_{it} = \ali+\x'_{it} \be + \e_{it}$ состоятельна. Модель с фиксированными эффектами учитывает самоотбор выборки в той мере, в которой он зависит характеристик, не меняющихся во времени.

Вербик и Нейман (1992) более подробно обсуждают ключевые предположения, необходимые для состоятельного оценивания этих моделей и предлагают тест на наличие смещения, связанного  самоотбором. Вулдридж (1995) описывает похожий анализ при более слабых предположениях и показывает предположения, которые могут быть не такими ограничительными в некоторых приложениях, и позволяют состоятельно оценить модель с фиксированным эффектом. В работе Велла (1998) можно найти обзор и дополнительные справки.

Методы для учёта самоотбора выборки могут быть расширены до работы с \textbf{истощением панели} (см. раздел 21.8.5), которое приводит к \textbf{смещению в связи с истощением}, если наблюдения зависимой переменной потеряны не случайным образом. Тогда все данные для $i$-го наблюдения ненаблюдаемы, если $d^*_{it} \leq 0$, из-за чего $\mathbf{z}_{it}$ в \ref{Eq:23.47} нужно заменить на переменные других периодов. Пример можно найти у Хаусмана и Уайза (1979), а также у Грасдала (2001). Ссылки на другую литературу можно найти у Бальтаджи (2001) и Хсяо (2003).

\section{Данные о переходах}

Для конкретности возьмем панельные данные о продолжительности благосостояния. Особый интерес представляет измерение индивидуального постоянства в периоды благосостояния, и определение меры, в которой оно объясняется истинной зависимостью от состояния, а не  индивидуальной предрасположенностью к благосостоянию. Так как индивидуальная предрасположенность может частично зависеть от ненаблюдаемых наблюдений, следует использовать модели с индивидуальными эффектами. Для данных о длительности состояний существует необыкновенно широкий набор подходов моделирования, так как возможно несколько типов панельных данных о переходах между состояниями. Здесь мы остановим внимание на моделях с фиксированными эффектами. 

Могут быть доступны данные о том, находится или нет индивидуум в определенном состоянии, например, положении хорошего благосостояния, в несколько различных моментов времени.  В таком случае можно использовать модель бинарного выбора для панельных данных (см. раздел 23.4), такую как динамическая логит-модель с фиксированными эффектами.

Более сложные данные дают нам информацию о продолжительностях индивидуальных состояний. Начнем с \textbf{модели пропорционального риска панельных данных}
\begin{align}
\lambda (t_{ij} | \x_{ij}) = \lambda_j (t_{ij}, \gamma_j) \exp (\x'_{ij} \be) \ali 
\label{Eq:23.48}
\end{align}
где $t_{ij}$ --- продолжительность завершенного периода для $j$-го периода $i$-го индивидуума, $\ali$ --- индивидуальный эффект. Это смешанная модель пропорционального риска, которая обсуждается для данных о продолжительности в главе 18. В ряд условий для непараметрического оценивания модели пропорциональных рисков с данными о продолжительности одного периода (см. раздел 18.3) входит предположение, что $\ali$ распределены независимо от регрессоров. Это позволяет избавиться от фиксированных эффектов. Однако как только нам доступны данные о продолжительности нескольких периодов, $\ali$ может быть фиксированным эффектом, если $\x_{ij}$ не меняется по $j$, как показал  Оноре (1992) (см. раздел 19.4.1). Дальнейшее обсуждение модели \ref{Eq:23.48}, включая динамическую модель длительности состояний с функцией риска для второго периода, зависимого от продолжительности первого, см. в разделе 19.4.1.

Чемберлин (1985) предложил несколько подходов для устранения $\ali$ в различных моделях продолжительности состояний панельных данных. Для модели пропорциональных рисков с базовым риском $\lambda_j(\cdot)$ вероятность того, что второй период будет длиннее, чем первый, не зависит от $\ali$. Для гамма моделей продолжительности состояний можно применять условный ММП, так как гамма --- это распределение из экспоненциального семейства. Для моделей Вейбулла, гамма и лог-нормальной плотность $t_{i1}/t_{i2}$ не зависит от $\ali$ .

Актуальные ссылки и подробные обсуждения, включая чувствительность данных о продолжительности нескольких периодов к цензурированию, см. у Ван ден Берга (2001).

\section{Счетные данные}

Хаусман и др. (1984) представляет модели с фиксированными и случайными эффектами как для моделей Пуассона, так и для отрицательных биномиальных моделей панельных данных. В более современной работе рассматриваются фиксированные эффекты в моделях с мультипликативными эффектами,  которые позволяют получать оценку статической и динамической моделей при относительной слабых предположениях о распределении.

\subsection{Счетные модели с индивидуальными эффектами}

Мы остановимся на модели Пуассона, подробно описанной для пространственных данных в разделе 20.2, хотя панельные версии отрицательных биномиальных моделей тоже кратко рассмотрим.

В \textbf{модели Пуассона с индивидуальными эффектами} 
$y_{it} \thicksim \mathcal{P}[\ali \exp (\x'_{it}\be)]$.
Тогда, предполагая условную независимость, совместная вероятность 
для $i$-го наблюдения $\mathbf{y}_i=(y_{i1}, \dots, y_{iT})$ равна
\begin{align}
f(\mathbf{y}_i|\mathbf{X}_i, \ali, \be)=\prod^T_{t=1} \exp [-\ali \exp (\x'_{it}\be)]
[-\ali \exp (\x'_{it}\be)]^{y_{it}}/y_{it!}.
\label{Eq:23.49}
\end{align}

В менее параметрическом подходе условное среднее моделируется просто как 
\begin{align}
\E[y_{it}|\ali, \x_{it}] = \ali \exp (\x'_{it} \beta) = 
\label{Eq:23.50} \\
= \exp (\gamma_i+\x'_{it}\beta). \nonumber
\end{align}
Это относится и к одноиндексной модели с индивидуальными эффектами, и к модели с мультипликативными эффектами. Так как это мультипликативные эффекты, индивидуальные эффекты $\ali$ могут быть устранены  с помощью вычитания среднего или взятия первых разностей. Заметим, что у модели Пуассона панельных данных \ref{Eq:23.49} условное среднее равно \ref{Eq:23.50}.

\subsection{Счетные модели со случайными эффектами}

Предположение о том, что случайные эффекты имеют гамма-распределение, позволяет получить трактуемое решение для вероятности в модели со случайными эффектами. Предположим, что $\ali$ имеет $\mathcal{G}[\eta, \eta]$ распределение со средним 1 и дисперсией $1/\eta$ и плотностью $\mathrm g(\ali|\eta)=\eta^{\eta} \ali^{\eta-1} e^{-\ali\eta} / \Gamma(\eta)$. Тогда \ref{Eq:23.18} для модели Пуассона \ref{Eq:23.49} будет следующим:
\begin{align}
f(\mathbf{y}_i |\mathbf{X}_i, \be, \eta) = 
\left[ \prod_t \frac{\lambda^{y_{it}}_{it}}{y_{it}!} \right]
\times \left( \frac{\eta}{\sum_t \lambda_{it}+\eta} \right)^{\eta} 
\label{Eq:23.51} \\
\times \left( \sum_t \lambda_{it} \right)^{-\sum_t y_{it}}
\frac{\Gamma (\sum_t y_{it} + \eta)}{\Gamma(\eta)}, \nonumber
\end{align}
где $\lambda_{it}=\exp (\x'_{it}\be)$, а преобразования и выводы даны в разделе 23.7.5. Получившееся условие первого порядка для оценки Пуассона случайных эффектов $\hat{\be}$ можно выразить как
\begin{align}
\sum^N_{i=1} \sum^T_{t=1} \x_{it} \left(y_{it} - \lambda_{it} \frac{\bar{y}_i + \eta/T}{\bar{\lambda}_i + \eta/T} \right) = \mathbf{0},
\label{Eq:23.52}
\end{align}
где $\bar{\lambda}_i = T^{-1} \sum_t \exp (\x'_{it}\be)$.

Ожидаемое значение выражения слева от знака равенства в \ref{Eq:23.52} равно 0, если среднее при условии регрессоров всех периодов равно $\E [y_{it}|\ali, \x_{i1}, \dots, \x_{iT}]=\ali \exp  (\x'_{it} \be)$. Так что несмотря на все параметрические предположения, оценка Пуассона со случайным эффектом состоятельна для $\be$ при выполнении слабого предположения о том, что условное среднее равно \ref{Eq:23.50}, и что регрессоры строго экзогенны. Для вероятности \ref{Eq:23.51} $\E [y_{it}|\x_{it}]=\lambda_{it}$ и $\mathrm{V}[y_{it}|\x_i]=\lambda_{it}+\lambda^2_{it}/\delta$, т.е. избыточная дисперсия имеет вторую форму отрицательного биномиального распределения. Оценка  ковариационной матрицы в сэндвич форме позволит использовать более гибкие модели избыточной дисперсии и условной корреляции. Условия первого порядка для $\eta$, которые мы не приводим, довольно сложны, хотя информационная матрица имеет блочно-диагональный вид по $\be$ и $\eta$.

Для случайных эффектов доступно несколько альтернативных оценок. Во-первых, оценка Пуассона сквозной регрессии игнорирует случайные эффекты и предполагает, что $y_{it} | \x_{it} \thicksim \mathcal{P}[\exp (\x'_{it} \be)]$. Условия первого порядка:
\begin{align}
\sum^N_{i=1} \sum^T_{t=1} \x_{it} (y_{it} - \lambda_{it})= \mathbf{0},
\label{Eq:23.53}
\end{align}
где $\lambda_{it}=\exp (\x'_{it} \be)$. Эта оценка состоятельна, если условное среднее имеет вид \ref{Eq:23.50} c $\E [\ali | \x_{it}]=1$. Поэтому обычная оценка ММП Пуассона для пространственных данных состоятельна, если истинная модель --- это модель с мультипликативными эффектами. Однако, как было показано в разделе 23.3, следует использовать робастные стандартные ошибки. Здесь \ref{Eq:23.26} будет равно
\begin{align}
\hat{V} [\hat{\be}_{pool} ] = 
\left[ \sum_{i,t} \hat{\lambda}_{it} \x_{it} \x'_{it} \right]^{-1}
\sum_{i,t,s} \hat{u}_{it} \hat{u}_{is} \x_{it} \x'_{it}
\left[ \sum_{i,t} \hat{\lambda}_{it} \x_{it} \x'_{it} \right]^{-1},
\label{Eq:23.54}
\end{align}
где $\hat{\lambda}_{it}=\exp (\x'_{it}\hat{\be})$, $\hat{u}_{it}=y_{it}-\hat{\lambda}_{it}$, $\sum_{i,t}$ обозначает $\sum^N_{i=1} \sum^T_{t=1}$, и $\sum_{i,t,s}$ обозначает $\sum^N_{i=1} \sum^T_{t=1} \sum^T_{s=1}$. Альтернативная оценка сквозной регрессии, основанная на \ref{Eq:23.50}, --- это оценка, полученная с помощью нелинейного МНК. В этом случае \ref{Eq:23.53}  будет выглядеть как $\sum_i \sum_t \x_{it} \lambda_{it} (y_{it}-\lambda_{it})=\mathbf{0}$.

Во-вторых, более эффективную оценку сквозной регрессии можно получить, используя  подход обобщенных оценивающих уравнений раздела 23.2.8, в котором предполагается условная корреляция. Общий результат для $\mathrm{g}_{it}=\lambda_{it}=\exp (\x'_{it}\be)$ теперь примет вид
\begin{align}
\sum^N_{i=1} \mathbf{Z}'_i \Sigma^{-1}_i (\mathbf{y}_i-\bm\lambda_i)=\mathbf{0},
\label{Eq:23.55}
\end{align}
где $\mathbf{Z}_i$  --- это матрица размерности $T \times K$ с $t$-м рядом  $\lambda_{it}\x'_{it}$ и $\bm\lambda_i$ --- это вектор размерности $T \times 1$ c $t$-м элементом $\lambda_{it}$. Возможно несколько видов ковариационных матриц $\bm\Sigma_i$ для $\mathrm{V}[\mathbf{y}_i|\mathbf{X}_i]$. Выбор матрицы вида $\bm\Sigma_i=\mathrm{Diag}[\lambda_{it}]$ приводит к оцениванию уравнений \ref{Eq:23.53} с помощью модели Пуассона. Если $\Sigma_{i,tt}=\lambda_{it}$ и $\Sigma_{i,ts}=\lambda_{is}=\phi \sqrt{\lambda_{it}\lambda_{is}}$ для $s \neq t$ , то возможно включить корреляцию по $t$. Речь идёт о  \textbf{равнокоррелированности} или \textbf{перестановочной корреляции}, так как для $s \neq t$ корреляция  равна константе $\phi$.

В-третьих, более эффективное оценивание возможно, если использовать ММП с отрицательной биномиальной моделью, а не моделью Пуассона. Предположим, $y_{it}$ независимая и одинаково распределенная отрицательная биномиальная величина с функцией избыточной дисперсии в форме NB2 с параметрами $\ali\lambda_{it}$ и $\Phi$ (см. раздел 20.4.1). Среднее $y_{it}$ равно $\ali \lambda_{it}/\phi_i$, а дисперсия $(\ali\lambda_{it}/\phi_i) \times (1+\ali/\phi)$. Если $(1+\ali/\phi_i)^{-1}$ --- это случайная величина, имеющая бета-распределение с параметрами $(\eta_1, \eta_2)$, то после некоторых алгебраических преобразований \ref{Eq:23.18} упрощается до 
\begin{align}
f(\mathbf{y}_i|\mathbf{X}_i, \be, \eta) = 
\left( \prod_t \frac{\Gamma (\lambda_{it}+y_{it})!}{\Gamma(\lambda_{it})! |Gamma(y_{it}+1)!} \right)
\label{Eq:23.56} \\
\times \frac{\Gamma(\eta_1+\eta_2)\Gamma (\eta_1+\sum_t \lambda_{it}) \Gamma (\eta_2+\sum_t y_{it})}{\Gamma(\eta_1) \Gamma(\eta_2) \Gamma (\eta_1+\eta_2+\sum_t \lambda_{it} + \sum_t y_{it})}. \nonumber \\
\end{align}
где $\lambda_{it}=\exp (\x'_{it}\be)$. Это база для оценивания параметров $\be$, $\eta_1$ и $\eta_2$ с помощью ММП. Эта модель базируется на более сильных предположениях, чем модель Пуассона со случайными эффектами.

В-четвертых, анализ нельзя ограничивать параметрическими моделями с решением в аналитическом виде для $f(\mathbf{y}_i|\mathbf{X}_i, \be, \eta)$. Крепон и Дюжё (1997a) используют методы симуляционного  максимального правдоподобия для оценки модели преодоления препятствий панельных на счетных данных и модели для панельных счетных данных с избыточным количеством нулей с нормально распределёнными случайными эффектами. 



\subsection{Счетные модели с фиксированными эффектами}

Оценку с фиксированным эффектом для модели Пуассона панельных данных \ref{Eq:23.50} можно вывести несколькими способами.

Во-первых, оценка максимального правдоподобия Пуассона одновременно оценивает $\be$ и $\alpha_1, \dots, \alpha_N$. Логарифм функции правдоподобия, основанной на вероятности \ref{Eq:23.49}, равен
\begin{align}
\mathrm{ln L}(\be, \bm\alpha) 
& = \mathrm{ln} 
\left[ \prod_i \prod_t \{\exp (-\ali \lambda_{it}) (\ali \lambda_{it})^{y_{it}} / y_{it} ! \} \right] 
\label{Eq:23.57} \\
& = \sum_i \left[ -\ali \sum_t \lambda_{it} + \mathrm{ln} \ali \sum_t y_{it} + \sum_t y_{it} \mathrm{ln} \lambda_{it} - \sum_t \mathrm{ln} y_{it}! \right], 
\nonumber 
\end{align}
где $\lambda_{it}=\exp (\x'_{it} \be)$. Взятие производной по $\ali$ и приравнивание ее к нулю дает $\hat{\ali}=\sum_t y_{it}/ \sum_t \lambda_{it}$. Подставим это обратно в \ref{Eq:23.57}, в результате чего получим \textbf{краткую функцию правдоподобия}. Опуская члены, которые не содержат $\be$, получаем
\begin{align}
\mathrm{ln L}_{\mathrm{conc}}(\be) \varpropto 
\sum_i \sum_t \left[ y_{it} \mathrm{ln} \lambda_{it} - y_{it} \mathrm{ln} \left( \sum_s \lambda_{is} \right) \right].
\label{Eq:23.58}
\end{align}
Отсюда следует, что для модели Пуассона с фиксированными эффектами отсутствует проблема второстепенных параметров. Состоятельные оценки $\be$ для фиксированных $T$ и $N \rightarrow \infty$ могут быть получены посредством максимизации $\mathrm{ln L_{conc}}(\be)$ в \ref{Eq:23.58}. При взятии производной \ref{Eq:23.58} по $\be$  получаем условия первого порядка
\begin{align}
\sum_i \sum_t 
\left[ y_{it} \x_{it} - y_{it} \left[ \sum_s \lambda_{is} \x_{is} \right] /
\left[ \sum_s \lambda_{is} \right] \right] = \mathbf{0},
\nonumber
\end{align}
которое можно перезаписать как
\begin{align}
\sum^N_{i=1} \sum^T_{t=1} \x_{it} \left( y_{it} - \frac{\lambda_{it}}{\bar{\lambda}_i}\bar{y}_i \right) = \mathbf{0},
\label{Eq:23.59}
\end{align}
где $\lambda_{it}=\exp (\x'_{it} \be)$  и $\bar{\lambda}_i=T^{-1} \sum_t \exp (\x'_{it} \be)$, см. Бланделл, Гриффит и Виндмайер (1995). Модель Пуассона \ref{Eq:23.49} и линейная модель панельных данных раздела 21.6 необычны тем, что одновременное оценивание $\be$ и $\bm\alpha$  дает состоятельные оценки $\be$ в коротких панелях. Так что \textbf{проблема второстепенных параметров} здесь отсутствует.

Во-вторых, оценивание условным ММП устраняет фиксированные эффекты посредством включения в условие достаточной статистики $\ali$. Для модели Пуассона это $\sum_t y_{it}$. Выполнение алгебраических преобразований, представленных в разделе 23.7.5, дает условную функцию правдоподобия, которая пропорциональна краткой логарифмической функции правдоподобия в \ref{Eq:23.58}. Оценка условного ММП для $\be$ в модели Пуассона с фиксированными эффектами является решением \ref{Eq:23.59}. Это был обычный вывод оценки Пуассона с фиксированными эффектами $\be$, который проделали  Палмгрен (1981) и Хаусман и др. (1984).

В-третьих, выполнение преобразования <<отклонение от среднего>> \ref{Eq:23.14} для модели с мультипликативными эффектами \ref{Eq:23.50} дает $\E [y_{it} - (\lambda_{it} / \bar{\lambda}_i) \bar{y}_i |\x_{i1}, \dots, \x_{iT}]=0$, и поэтому
\begin{align}
\E[\x_{it}(y_{it}-(\lambda_{it}/\bar{\lambda}_i) \bar{y}_i )] = \mathbf{0}.
\label{Eq:23.60}
\end{align}
Используя соответствующие выборочные моментные тождества, получаем оценку $\be$, которая является решением \ref{Eq:23.59}.

Одна и та же оценка была получена тремя различными способами. Из третьего вывода понятно, что важное предположение, необходимое для состоятельности оценки Пуассона  с фиксированным эффектом, состоит в том, что регрессоры строго экзогенны и \ref{Eq:23.50} правильно специфицирована. Статистические выводы должны основываться на робастных стандартных ошибках для панельных данных. В частности, если использовать обычные выводы стандартного и условного ММП, то исходя из первых двух выводов, стандартные ошибки могут быть значительно недооценены из-за того, что в счетных данных не была учтена избыточная дисперсия. Оценка с фиксированным эффектом приводит к потере данных, так как наблюдения $i$, для которых $\sum_t y_{it} = 0$, не дополняет сумму в \ref{Eq:23.59}.

Получить состоятельные оценки $\be$ в присутствии фиксированных эффектов тоже возможно, если использовать особую параметризацию отрицательной биномиальной модели. Хаусман и др. (1984) предполагал, что $y_{it}$ --- это независимая одинаково распределенная NB1 величина с параметрами $\ali \lambda_{it}$ и $\phi_i$, где $\lambda_{it}=\exp (\x'_{it}\be)$, так что $y_{it}$ имеет среднее $\ali \lambda_{it} / \phi$ и дисперсию $(\ali \lambda_{it}/\phi_i) \times (1+\ali /\phi_i)$. Параметры $\ali$ и $\phi_i$ могут быть идентифицируемы вплоть до отношения $\ali/\phi_i$, и эта дробь выпадает из выражения условной совместной плотности для $i$-го наблюдения. Выражение для совместной плотности упрощается до 
\begin{align}
f(y_{i1}, \dots, y_{iT} | \sum_t y_{it})=
& \left( \prod_t \frac{\Gamma(\lambda_{it}+y_{it})}{\Gamma(\lambda_{it}) \Gamma (y_{it} + 1)} \right) 
\label{Eq:23.61} \\
& \times \frac{\Gamma (\sum_t \lambda_{it}) \Gamma (\sum_t y_{it} + 1)}{\Gamma(\sum_t \lambda_{it}+\sum_t y_{it})}.
\nonumber
\end{align}
Целое $\lambda_{it}$  имеет \textbf{отрицательное гипергеометрическое распределение}. Оценка отрицательной биномиальной модели с фиксированными эффектами условного ММП $\be$ максимизирует логарифм функции  правдоподобия, основанную на \ref{Eq:23.61}. Модель Пуассона с фиксированными эффектами более часто используется, так как оценка этой модели будет состоятельно при гораздо более слабых предположениях о распределении.  


\subsection{Динамические счетные модели}


Есть несколько способов учитывать динамику в моделях счетных данных. Чистые модели временных рядов исследуются в Кэмерон и Триведи (1998). Для простоты рассмотрим включение в модель одной лаговой зависимой переменной. Очевидно будет использовать модель $\E [y_t| y_{t-1}, \x_t]=\exp (\gamma y_{t-1}+\x'_t\be)$, но это может привести к <<взрывному поведению>> из-за того, что берется экспонента от $y_{t-1}$. Более стабильную модель можно получить, используя $\exp (\gamma \mathrm{ln} y_{t-1} + \x'_t \be)$, но тогда возникают проблемы при $y_{t-1}=0$. По этой причине используется линейная модель обратной связи $\E [y_t|y_{t-1}, \x_t]=\gamma y_{t-1}+\exp (\x'_t\be)$. Модель Пуассона AR(1) обладает этим свойством и в случае чистых временных рядов имеет корреляционную функцию $\mathrm{Cor}[y_t, y_{t-k}]=\gamma^k$, похожую на корреляционную функцию для модели AR(1) (см. Аль-Ош и Алзаид, 1987).

Таким образом, Бланделл, Гриффит, и Виндмайер (1995, 2002) рассматривали динамическую модель панельных данных с фиксированными эффектами с
\begin{align}
\E [y_{it} | \ali, y_{i,t-1}, \x_{it} ] = \gamma y_{i,t-1} + \ali \exp (\x'_t \be).
\nonumber
\end{align}
Применяя преобразование <<взятие первых разностей>> \ref{Eq:23.15}, получаем условные моментные ограничения
\begin{align}
\E \left[ \frac{\exp (\x'_{i,t-1} \be)}{\exp (\x'_{it}\be)}
(y_{it} - \gamma y_{i,t-1}) - (y-{i,t-1} - \gamma y_{i,t-2}) | y_{i1}, \dots, y_{i,t-2}, \x_{i1}, \dots, \x_{i,t-1} \right] = 0.
\nonumber
\end{align}
Они приведут к большому количеству моментных тождеств (см. в разделе 22.5.3 похожее обсуждение для линейной модели), которые могут служить основой для ОММ оценивания, как в разделе 23.2.6. Крепон и Дюжё (1997b), Монталво (1997), и Бланделл, Гриффит, и Ван Ринен (1995, 1999) используют схожие методы взятия квази-разностей для анализа зависимости количества патентов и затрат на НИОКР.

Бёкенхольт (1999) использует более параметрическую модель, оценивая AR(1) модель Пуассона для целых значений с ненаблюдаемой гетерогенностью, для моделирования которой используется смесь распределений (см. раздел 18.5). 

\subsection{Выводы для моделей Пуассона со случайными и фиксированными эффектами}

Во-первых, рассмотрим модель Пуассона со случайными эффектами с гамма-распределением случайных эффектов. Для простоты опустим индекс $i$ и пусть $\lambda_t=\exp (\x'_t\be)$. Из общей формулы \ref{Eq:23.18} для модели Пуассона \ref{Eq:23.49} и плотности случайных эффектов $\mathrm g(\alpha|\gamma)$ следует, что
\begin{align}
f(y_1, \dots, y_T | \x_t) 
& = \int^{\infty}_0 \left[ \prod_t (e^{-\alpha \lambda_t} (\alpha \lambda_t)^{y_t}/ y_t !) \right]  \mathrm g(\alpha | \gamma) d \alpha 
\nonumber \\
& = \int^{\infty}_0 \left[ \prod_t \lambda_t^{y_t}/y_t ! \right] \left( e^{-\alpha \sum_t \lambda_t} \alpha^{\sum_t y_t} \right) 
\mathrm g(\alpha|\gamma) d \alpha
\nonumber \\
& = \left[ \prod_t \lambda_t^{y_t} /y_t ! \right]
\times \int^{\infty}_0 \left( e^{-\alpha \sum_t \lambda_t} \alpha^{\sum_t y_t} \right) 
\mathrm g(\alpha|\gamma) d \alpha. \nonumber
\end{align}
Для $\mathrm g(\ali|\eta)=\eta^{\eta} \alpha^{\eta-1} e^{\alpha \eta}/\Gamma(\eta)$ используя похожие алгебраические преобразования, что и в разделе 20.4.1, получаем выражение для плотности \ref{Eq:23.51}.

Во-вторых, выведем условную плотность для модели Пуассона с фиксированными эффектами для наблюдений во все временные периоды для данного индивидуума, где для простоты опущен индивидуальный индекс $i$. В общем случае вероятность для $y_1, \dots, y_T$ при $\sum_t y_t$ равна
\begin{align}
f(y_1, \dots, y_T| \sum_t y_t) 
& = f(y_1, \dots, y_T | \sum_t y_t)/f(\sum_t y_t) \nonumber \\
& = f(y_1, \dots, y_T )/f(\sum_t y_t) \nonumber \\
& = \frac{\prod_t (\exp (-\mu_t)\mu_t^{y_t}/y_t !)}{\exp (-\sum_t \mu_t) (\sum_t \mu_t)^{\sum_t y_t}/(\sum_t y_t)!} \nonumber \\
& = \frac{\exp (-\sum_t \mu_t)\prod_t \mu_t^{y_t}/\prod_t y_t !}{\exp (-\sum_t \mu_t) \prod_t (\sum_s \mu_s)^{y_t}/(\sum_t y_t)!} \nonumber \\
& = \frac{(\sum_t y_t)!}{\prod_t y_t !} \times \prod_t \left( \frac{\mu_t}{\sum_s \mu_s} \right)^{y_t}, \nonumber 
\end{align}
где во второй строке используется тот факт, что знание $\sum_t y_t$ ничего не добавляет к знанию $y_1, \dots, y_T$. В  третьем равенстве используется то, что $y_t$ независимо и одинаково распределены с $\mathcal{P}[\mu_t]$ и поэтому $\sum_t y_t$ имеет распределение $\mathcal{P}[\sum_t \mu_t]$. В четвертой и пятой строке выражение упрощается. Условная вероятность --- это вероятность для мультиномиальной модели для $\sum_t y_t$ испытаний, где $t$-й из $T$ исходов появляется в любом испытании с вероятностью $\mu_t / \sum_s \mu_s$. Установив, что $\mu_{it}=\ali \exp (\x'_{it}\be)$, и взяв логарифмы, получаем условное правдоподобие, которое пропорционально функции правдоподобия в краткой форме \ref{Eq:23.58}.

\section{Полупараметрическое оценивание}

В литературе, посвященной \textbf{полупараметрическому оцениванию} панельных данных, главным образом рассматриваются модели ограниченных зависимых переменных, так как, как и для случая панельных данных, предположения о параметрическом распределении становится очень важным, когда используются цензурированные, урезанные выборки или модели самоотбора. Внимание сосредоточено на моделях с фиксированными эффектами. Мы проведем их краткий обзор.

Для данных бинарного выбора Мански (1987) расширил свою оценку, основанную на методе максимального счёта, для модели панельных данных с фиксированными эффектами, представленной в \ref{Eq:23.33}, где функция $F(\cdot)$ уже не специфицирована. Хотя эта оценка состоятельна, она сходится со скоростью меньше, чем $\sqrt{N}$ и не является асимптотически нормальной. 

Для тобит-модели Оноре (1992) расширил подход наименьших абсолютных отклонений для цензурированных выборок, который разработал Пауэл (1986a), до модели панельных данных с фиксированными эффектами \ref{Eq:23.45}, где распределение ошибки $\e_{it}$ не специфицировано. Данные урезаны искусственным образом, так что фиксированный эффект последовательно устранен подходящим преобразованием взятия разностей. Оценка состоятельна со скоростью $\sqrt{N}$ и асимптотически нормальна.

Для панельных данных с самоотбором выборки Кирьязиду (1997) рассматривал тобит-модель типа 2 с фиксированными эффектами, где распределение ошибок $\e_{it}$ и $v_{it}$  не специфицировано. Она представила двухшаговую оценку вида Хекмана. С помощью сглаженной оценки, основанной на максимального счёта Мански (1987), устраняются фиксированные эффекты в уравнении самоотбора, хотя на втором шаге используется довольно сложное преобразование взятия разностей для устранения фиксированных эффектов в результирующем уравнении. Этот подход может быть расширен до обобщенных тобит-моделей. Шарлье, Меленберг, и ван Соест (2001) привели пример применения модели Роя или тобит-модели типа 5 для панельных данных.

Цензурирование часто используется в моделях продолжительности жизни. В разделе 23.6 рассматриваются модели панельных данных с завершенными периодами. Когда для индивидуума наблюдаются и завершенные, и незавершенные периоды, методы частичного правдоподобия не подходят, так как в присутствии не меняющихся во времени фиксированных эффектов цензурирование выборки проводится не независимо. Хоровиц и Ли (2004) предложили состоятельную оценку для  модели пропорциональных рисков \ref{Eq:23.43} с незавершенными периодами, для которых не требуется спецификация базового риска.


\section{Практические замечания}
Как и для случая линейных моделей, если используются панельные данные, как минимум, статистические выводы должны основываться на робастных стандартных ошибках. Вычисление таких ошибок не предусмотрено компьютерными программами для пространственных данных, если не используется опция  кластеризации стандартных ошибок. Кластеризация в этом случае проводится по индивидуальным наблюдениям.

Более эффективное оценивание возможно благодаря использованию моделей, в которых учитывается корреляция во времени. Эконометристы придают особое значение случайным эффектам. Несколько пакетов оценивают модели с нормально распределенными случайными эффектами, используя метод Гаусса для учёта эффектов, а также более специализированные  модели счетных данных со случайными эффектами, имеющие явные аналитические решения. Статистики же придает значение подходу обобщенных оценивающих уравнений для обобщенных линейных моделей, доступному во многих статистических и некоторых эконометрических пакетах.

Эти предшествующие методы дают несостоятельные оценки, если случайные эффекты коррелированы с регрессорами. Поэтому эконометристы уделяют особое внимание использованию фиксированных эффектов. Из-за проблемы второстепенных параметров состоятельные оценки в коротких панелях получаются только у некоторых нелинейных моделей. Доступны эконометрические пакеты для условного ММП оценивания этих моделей, логит-моделей и счетных моделей с фиксированными эффектами. Если модель с фиксированными эффектами недоступна, то нужно использовать модель с более сложными случайными эффектами, а не простейшими независимыми и одинаково распределенными эффектами.

Можно также оценивать динамические модели панельных данных. Они позволяют провести различие между постоянством, вызванным ненаблюдаемой гетерогенностью, и постоянством, вызванным истинной зависимостью от состояния. При оценивании этих моделей может потребоваться написание собственных программ.


\section{Литература}

В этой главе предоставлен обзор обширной и разнообразной литературы. По необходимости пропущено множество деталей. В монографиях о панельных данных, написанных Ареллано (2004), Бальтаджи (2001), Хсяо (2003), и М.-Дж. Ли (2002), рассматриваются модели панельных данных для данных бинарного выбора, цензурированных выборок, а также модели самоотбора. Кэмерон и Триведи (1998), а также М.-Дж. Ли (2002) описали модели панельных данных для счетных данных. Вулдридж (2002) описывает методы для оценки бинарных данных, цензурированных выборок, а также счетных данных. Фармайер и Тутц (1994) и Диггл и др. (1994, 2002) провели обзор статистической литературы для различных обобщенных линейных моделей. Матьяс и Севестр (1995) рассматривают нелинейные модели панельных данных. М.-Дж. Ли (2002) заостряет внимание на ОММ оценивании. Ареллано и Оноре (2001) придают важное значение полупараметрическим методам для нелинейных моделе панельных данных. Куп (2003) рассматривал байесовские методы оценки моделей панельных данных.

\textbf{23.2} Проблему второстепенных параметров обсуждает Ланкастер (2002). Ключевые работы, посвященные методам взятия разностей ---  Чемберлин (1992) and Вулдридж (1997a), а условному ММП --- Андерсен (1970). Для моделей со случайными эффектами Батлер и Моффитт (1982) для устранения нормально распределенных случайных эффектов используют метод Гаусса, в то время как в статистической литературе главное предпочтение отдается подходу, который разработали Лянг и Цегер (1986).

\textbf{23.4} Основные работы, в которых рассматриваются модели с фиксированными эффектами --- Чемберлин (1980) для статических моделей, Чемберлин (1985) для динамических моделей чистых временных рядов, и Оноре и Кирьязиду (2000) для динамических моделей с дополнительными регрессорами. См. также Хсяо (1995).

\textbf{23.5} Исследование моделей выбора панельных данных --- Велла (1998), а также учебники Бальтаджи (2001) и Вулдридж (2002).

\textbf{23.6} Чемберлин (1985) описывает несколько способов устранения фиксированных эффектов в различных моделях длительности состояний. Хорошее обсуждение и ссылки см. в Ван ден Берг (2001, раздел 6). \textbf{Анализ исторических событий} с использованием индивидуальных данных с несколькими промежутками намного сложнее, чем любой анализ панельных данных, так как такие модели являются динамическими по своей природе.

\textbf{23.7} Классическая литература, посвященная моделям счетных панельных данных, --- это Хаусман и др. (1984), а также Бланделл и др. (2002) для динамических моделей.

\textbf{23.8} Исследования полупараметрических методов оценивания моделей панельных данных --- Ареллано и Оноре (2001), а также Л.-Ф. Ли (2001).

{\centering \textbf{Упражнения} \\}

\textbf{23-1} Рассмотрим нелинейную модель панельных данных $y_{it} = \ali+ \exp (\x'_{it}\be)+u_{it}$, где $\be$ --- параметры для оценки, $\ali, i=1, \dots, N$ --- индивидуальные эффекты, $u_{it}$ --- независимые одинаково распределенные ошибки с параметрами $[0, \sigma^2_{\e}]$. Используется короткая панель.
\begin{itemize}
\item[{\bf (a)}] Предположим, что $\ali=0$ для любого $i$. Можно ли состоятельно оценить $\be$? Если да, то напишите формулу или целевую функцию для состоятельной оценки. Если нет, кратко объясните, почему невозможно получить состоятельную оценку для $\be$.
\item[{\bf (b)}] Предположим, что индивидуальные эффекты $\ali$ случайны, независимы и одинаково распределены с параметрами $[0, \sigma^2_{\alpha}]$ независимо от регрессоров. Можно ли состоятельно оценить $\be$? Если да, то напишите формулу или целевую функцию для состоятельной оценки. Если нет, кратко объясните, почему невозможно получить состоятельную оценку для $\be$.
\item[{\bf (с)}] Предположим, что индивидуальные эффекты $\ali$ случайны, но коррелированы с регрессорами. Можно ли состоятельно оценить $\be$? Если да, то напишите формулу или целевую функцию для состоятельной оценки. Если нет, кратко объясните, почему невозможно получить состоятельную оценку для $\be$.
\end{itemize}

\textbf{23-2} (Чемберлин, 1980) Покажите, что оценка ММП в логит-модели бинарного выбора несостоятельна, предел равен $2\beta$ в простой модели, где $T=2$.

\textbf{23-3}  Используйте ту же модель, что и модель Патенты-НИОКР в разделе 23.3. Выбирайте зависимую переменную и модель так, как предложено ниже. В каждом случае оценивайте модели со случайными эффектами и модель с фиксированными эффектами, если это возможно.

\begin{itemize}
\item[{\bf (a)}] Используйте логит-модель с зависимой переменной, показывающей, имеет ли фирма патент.
\item[{\bf (b)}] Используйте тобит-модель с урезанной выборкой, где в качестве зависимой переменной стоит логарифм количества патентов и из выборки удалены фирмы с нулевым количеством патентов.  
\item[{\bf (с)}] Используйте модель Пуассона для количества патентов.
\end{itemize}




\part{Дальнейшие темы}

\noindent В эмпирической работе часто приходится сталкиваться не с одной проблемой с данными, а с несколькими, и работать с ними надо одновременно. К примерам таких проблем можно отнести неслучайность выборки, кластеризацию наблюдений, ошибки измерения и пропущенные данные. Когда они случаются, по отдельности или одновременно, в контексте моделей, обсуждавшихся в частях 4 и 5, могут возникнуть проблемы с оценкой интересующих нас параметров. Три главы в части 6 --- главы 24, 26 и 27 --- анализируют последствия таких проблем и представляют методы борьбы с ними. Методы проиллюстрированы и использованием примеров из более ранних частей книги. Это позволяет связать часть 6 с другими частями книги. 

Глава 24, которая рассматривает особенности данных из сложных опросов, в особенности стратифицированные выборки и кластеризацию, дополняет темы, рассмотренные в главах 3, 5 и 16. Глава 26 посвящена ошибкам измерения в моделях, изученных в главах 4, 14 и 20. Глава 27 является в достаточной степени автономной. Она посвящена проблеме пропущенных данных и восстановлению данных, но использование EM-алгоритма и семплирования по Гиббсу создает точки соприкосновения с главами 10 и 13 соответственно. 

Глава 25 посвящена оценке воздействия. Это широкий термин, который относится к воздействию одной переменной, к примеру образования, на результирующую переменную --- к примеру, на доход. Переменные воздействия могут быть выбраны экзогенно или определены эндогенно. Тема оценки воздействия затрагивает такую проблему, как идентифицируемость влияния на результирующую переменную. Влияние может измеряться предельными эффектами или функциями предельных эффектов. Здесь используются различные методы, включая инструментальные переменные и метод подбора контрольной группы по индексу соответствия. Данная проблема может возникать в контексте любой модели из рассмотренных в частях 4 и 5. Данная глава делает акцент на линейной регрессионной модели, и может быть изучена на раннем этапе знакомства с книгой. Однако, она предполагает, что читатель будет знаком со многими другими темами, рассмотренными в книге, включая инструментальные переменные и модели выбора, поэтому данный вопрос рассматривается только в в последней части книги. 


\chapter{Стратифицированные и кластеризованные выборки}

\section{Введение}

Исследования в области микроэконометрики обычно проводятся на данных, собранных путём опроса выборки из интересующей нас генеральной совокупности. Самое простое статистическое предположение, используемое по отношению к данным опросов --- это \bfseries простой случайный отбор (simple random sampling, SRS) \mdseries, согласно которому все элементы генеральной совокупности имеют равные вероятности попадания в выборку. В таком случае разумно базировать выводы на предположении, что переменные ($y_i,x_i$) не зависят от $i$ и одинаково распределены. Это предположение лежит в основе свойств рассмотренных в этой книге оценок, которые они имеют на малых выборках и в асимптотике, за важным исключением моделей с самоотбором выборки в главе 16. 

На практике, однако, условия простого случайного отбора практически никогда не выполняются для данных опросов. Вместо этого используются другие способы формирования выборки, позволяющие снизить расходы на проведение опроса и повысить точность оценок для тех подгрупп из генеральной совокупности, которые представляют особый интерес. 
К примеру, при проведении опроса домохозяйств генеральная совокупности может быть сперва разделена по географическому признаку на подгруппы, к примеру, отдельные деревни или города, с разными долями в выборке для разных подгрупп. При проведении интервью могут опрашиваться домохозяйства, сконцентрированные, к примеру, в одном квартале. Очевидно, что данные  ($y_i,x_i$) теперь не являются независимыми и одинаково распределенными. Во-первых, распределение  ($y_i,x_i$) может изменяться в зависимости от подгруппы, что противоречит предположению об одинаковом распределении. Во-вторых, так как показатели могут коррелировать для домохозяйств из одной местности, предположение о независимости  ($y_i,x_i$) также нарушается. 

Отсюда следует необходимость адаптации стандартных методов получения распределений оценок, а свойства оценок могут отличаться от полученных в предположении простой случайной выборки. Именно это и является предметом изучения в данной главе. 

Можно выделить следующие последствия для регрессионных моделей. Во-первых, если цель работы --- изучение генеральной совокупности, а не её подгрупп, могут понадобиться взвешенные оценки, чтобы скорректировать коэффициенты на доли подгрупп в выборке. Во-вторых, взвешивание может быть не нужно, если изучается регрессия $y$ на $x$, при условии, что зависимость $y$ от заданного $x$ правильно специфицирована и стратификация проводилась не по зависимой переменной. 
В-третьих, если выборка хотя бы частично определяется значениями зависимой переменной, к примеру, если доля людей с низкими доходами в ней завышена, и доход является зависимой переменной, необходимо использовать взвешенную регрессию. Здесь возможно использование разных процедур оценки, часть из которых была рассмотрена в главе 16 в контексте смещения самоотбора выборки. В-четвертых, кластеризация как минимум приводит к серьезным занижениям в оценках стандартных ошибок, и может даже оказаться причиной несостоятельности оценок параметров, если не учитывать её влияние при помощи методов, похожих на представленные в главе 21 для анализа панельных данных. 

Самый важный вывод для большинства микроэконометрических приложений заключается в том, что, при использовании данных опросов, необходимо учитывать кластеризацию. Кластеризация наблюдений часто встречается как в пространственных, так и в панельных данных и является следствием $(1)$ устройства выборки, $(2)$ устройства эксперимента или $(3)$ метода сбора данных. Примером $(1)$ является комплексный крупномасштабный опрос, в котором используется выборка из географических кластеров с целью уменьшения стоимости проведения исследования. Примером $(2)$ может служить случайный эксперимент, в котором к индивидам из одного места (к примеру, школы или завода) применяется одинаковое воздействие. К примерам $(3)$ можно отнести регрессии по структурным данным, где в число регрессоров входят групповые средние (такие как безработица или налоги на уровне государства), панельные данные и данные по близнецам, даже если в них нет кластеризации домохозяйств. 

Часть 24.2 представляет некоторые концепции и терминологию. Части 24.3 --- 24.5 рассматривают три основных свойства опросных данных: веса в выборке, стратификацию и кластеризацию. В части 24.6 рассматриваются иерархические линейные модели со стратификацией и кластеризацией. Приложение приводится в части 24.7. Комплексные опросы рассматриваются более подробно в части 24.8. 

\section{Формирование выборки}

Формирование выборки для опросов --- хорошо изученная в статистической литературе тема, потому что данные должны быть собраны до проведения анализа, а сам сбор данных может быть очень затратным. Литература на эту тему, как правило, концентрируется на способах получить с минимальными затратами выборку, по которой можно получить несмещенные и достаточно точные оценки параметров генеральной совокупности, особенно средних. 

Структура многоэтапного опроса была описана в разделе 3.2. Текущее обследование населения (Current Population Survey, CPS) США --- отличный пример такого устройства выборки. 

\subsection{Текущее обследование населения}

Текущее обследование населения, CPS --- ежемесячный опрос с охватом в приблизительно 56 тысяч домохозяйств, нацеленный на репрезентативное представление гражданского неинституционального населения в возрасте от 16 лет. Домохозяйства из небольших штатов перепредставлены, чтобы обеспечить надёжность данных на уровне штата. Внутри штатов домохозяйства кластеризованы, чтобы уменьшить расходы на проведение интервью. Они опрашиваются в течение четырех месяцев подряд, не опрашиваются следующие 8 месяцев, и затем опрашиваются ещё 4 месяца. Повторное интервьюирование позволяет снизить издержки, а схема 4-8-4 позволяет проводить анализ во временном разрезе, в том числе исследовать годовые разности. Есть 8 ротационных групп одинакового размера, каждый месяц одна из них вводится в исследование. Рассмотрим устройство одной группы. 

Выделяется 792 страты, каждая из которых соответствует части штата или, в некоторых случаях, всему штату. Все 792 страты делятся на 2007 первичных единиц выборки (Primary Sampling Units, PSU), которые могут представлять собой городской статистический регион (Metropolitan Statistical Area, MSU), пересечение штата и MSU, если MSU покрывает больше одного штата, отдельное графство или несколько смежных графств. Отклонения от этой схемы возможны, когда в PSU попадает мало населения или очень большая территория. В среднем в одной страте 2.5 PSU. Из 792 страт, 432 содержат только одно PSU, в таком случае PSU называют само-преставленным (self-representing) и его всегда включают в опрос. Остальные 360 страт содержат больше одного PSU, и в этом случае выбирается только один PSU из страты с вероятностью, пропорциональной населению по данным на 1990 год. 

Внутри PSU нет промежуточных вторичных единиц выборки. При проведении опроса напрямую отбираются конечные единицы выборки (Ultimate Sampling Units, USU), которые представляют собой географически компактные группы из примерно четырех точек. Вероятность попадания в выборку возрастает, если вероятность выбора данного PSU из своей страты была низкой и, как правило, возрастает, если PSU находится в маленьком штате, чтобы обеспечить перепредставленность штатов с небольшим населением. В этих расчётах Нью-Йорк и Лос-Анджелес рассматриваются как отдельные штаты. Все домохозяйства в USU опрашиваются, кроме случаев, когда в USU входит нетипично большое количество домохозяйств --- тогда случайным образом выбирается подмножество из домохозяйств. 

Текущее обследование населения CPS, устроено таким образом, чтобы обеспечивать само-взвешивание по штатам, то есть, несмотря на использование неслучайной выборки, CPS сохраняет репрезентативность выборок для каждого штата. Однако, невзвешенная выборка не репрезентативна в в рамках всей страны из-за перепредставленности штатов с маленьким населением и из-за того, что не все PSU входят в выборку. 

\subsection{Организация выборки}

Перед тем, как перейти к более детальному рассмотрению устройства выборок для опросов, мы приводим краткий обзор основ организации выборки в отсутствии таких осложнений, как стратификация. 

Обозначим как $z$ вектор переменных, без разделения на зависимые переменные и регрессоры. Мы предполагаем, что $z$ независимы и одинаково распределены с плотностью $f(z)$. Размер генеральной совокупности $N^*$, размер выборки $N$. Выборка --- это $\{ z_i,i=1,\ldots,N \}$, где $i$ обозначает $i$-ый элемент выборки. Стандартное обозначение в литературе, посвященной формированию выборок --- $n$ для размера выборки и $N$ для размера генеральной совокупности. Мы, однако, продолжим использовать $N$ для обозначения размера выборки, потому что необходимость использовать размер генеральной совокупности $N^*$ возникает редко. 

\subsection*{Исчерпывающий отбор}

При \bfseries исчерпывающем отборе\mdseries каждый элемент из генеральной совокупности входит в выборку, соответственно, выборка совпадает с генеральной совокупностью. Такой отбор встречается очень редко при использовании данных на уровне индивидов. Исчерпывающий отбор может иметь место при использовании данных переписей населения, таких, как перепись, проводимая в США каждые 10 лет. Но даже для переписей, более длинные опросники используются при опросе только части населения, исследователи могут предпочесть работы с более удобной для них подвыборкой, да и покрытие переписи на практике не бывает полным. Исчерпывающий отбор более распространен при использовании данных уровня фирм, когда, к примеру, можно выделить и изучить все фирмы отрасли. 

При использовании исчерпывающего отбора может возникнуть дискуссия о том, оправданно ли использование стандартных методов, если выборочные моменты совпадают с моментами по генеральной совокупности. Как правило, методы используются те же самые, просто конечная генеральная совокупность рассматриваются, в свою очередь, как выборка из бесконечной суперсовокупности (superpopulation).

К примеру, предположим, что мы изучаем гендерные различия в оплате труда на рабочем месте, где вся генеральная совокупность состоит из 20 мужчин и 12 женщин, выполняющих сходные задачи. Взяв заработные платы всех мужчин и женщин, так, что выборка совпадает с генеральной совокупностью, мы получили, что средняя заработная плата для мужчин выше, чем для женщин. В таких ситуациях мы обычно проводим стандартные тесты на значимость различий в средней заработной плате вместо того, чтобы со $100\%$ уверенностью заключить, что мужчины получают больше. Это объясняется тем, что мы рассматриваем генеральную совокупность людей, работающих в этом месте, как выборку из суперсовокупности всех рабочих мест или суперсовокупности из этого места в разные моменты времени. 

Исчерпывающий отбор --- дорогая и зачастую ненужная для больших совокупностей процедура, кроме случаев, когда необходимо определить непосредственно размер генеральной совокупности. Поэтому, как правило, используются выборки.  

\subsection*{Простой случайный отбор}

\bfseries Простой случайный отбор\mdseries  --- такой отбор, при котором наблюдения извлекаются из генеральной совокупности случайно и с равной вероятностью. Каждое наблюдение попадает в выборку с вероятностью, равной размеру выборки делённому на размер совокупности, и в пределе имеет такую же плотность $f(z)$. Он называется <<простым>>, потому что более сложные методы как правило также сохраняют элемент случайности. 

\subsection*{Коррекция на конечность выборки}

В эконометрическом анализе обычно предполагается, что при простом случайном отборе отдельные $z$ независимы, поэтому совместная плотность в выборке равна произведению индивидуальных плотностей $f(z_i)$. Это оправданно, если отбор производится из бесконечной совокупности, как в случае, когда мы рассматриваем выборку из суперсовокупности, или если отбор производится из конечной генеральной совокупности с возвращением.

На практике для конечных генеральных совокупностей используется \bfseries выборка без возвращения\mdseries, чтобы одно и то же наблюдение не попадало в выборку дважды. Тогда наблюдения больше не являются независимыми, даже при простом случайном отборе. Чтобы убедиться в этом, заметим, что при простом случайном отборе для любого элемента генеральной совокупности вероятность попасть в выборку равна $N/N^*$. Однако, зная, что данный элемент попал в выборку, вероятность попасть в выборку для любого другого элемента падает до $(N-1)/(N^*-1)$. Условная вероятность отличается от безусловной вероятности. Более формально, можно ввести переменные-индикаторы, показывающие для каждого элемента генеральной совокупности, вошёл ли он в выборку. Эти переменные имеют совместное мультиномиальное распределение со средними $\pi$, дисперсиями $\pi(1-\pi)$ и ковариациями $-\pi(1-\pi)/(N^*-1)$, где $\pi=N/N^*$.

Корреляция между наблюдениями в выборке $\rho = -1/(N^*-1)$, где $\rho$ называют \bfseries межклассовой корреляцией\mdseries. Полагая $z$ скаляром, получаем, что выборочное среднее $\overline{z} = N^{-1} \sum_i z_i$ имеет дисперсию $\V[\overline{z}] = N^{-2} \V[\sum_i z_i]$, которая не упрощается до $N^{-2} \sum_i \V[z_i]$ из-за корреляции между $z_i$. Вычисления, приведённые, к примеру, у Кохрана (1977, c. 23 --- 24) показывают, что 
$$\V[\overline{z}] = (1-f) \frac{S^2}{N},$$
где $f = N/N^*$ --- доля выборки. Результаты, полученные здесь, обычно проще представить в терминах $S^2 = (N^*-1)^{-1}\sum(z_i --- \overline{z})^2$, чем через обычную дисперсию  $\sigma^2 = N^{*-1}\sum(z_i --- \overline{z})^2$. 

Таким образом, для выборки без возвращения из конечной совокупности, дисперсия выборочного среднего равна стандартному $S^2/N$, умноженному на \bfseries коэффициент коррекции на конечность выборки \mdseries $1-f$. Этот коэффициент коррекции присутствует в статистических пакетах, ориентированных на данные опросов. Неучет коэффициента коррекции приводит к переоценке $\V[\overline{z}]$. Для регрессий при использовании данных с возвращением, коррекция на конечность выборки также необходима, однако величина и направление смещения в оценке дисперсии, получаемой МНК, теперь дополнительно зависит от матрицы независимых переменных. 

Коррекция на конечность выборки, как правило, игнорируется в микроэконометрике, и часто это оказывается разумным. К примеру, для опросов домохозяйств размер выборки настолько мал по сравнению с размером генеральной совокупности, что $f = N/N^* \to 0$.

\section{Взвешивание}

Опросы домохозяйств, такие как Текущее обследование населения, CPS, обычно конструируются так, что разные домохозяйства в итоге попадают в выборку с разными вероятностями. Чтобы скорректировать этот эффект, каждому наблюдению в выборке присваивается свой вес. 

Как показано ниже, при экзогенной стратификации веса нужно использовать, если регрессия используется для описания генеральной совокупности и не нужно использовать, если предполагается, что регрессионная модель соответствует истинной структурной модели. 

\subsection{Веса в выборке}

Пусть каждое домохозяйство в генеральной совокупности имеет вероятность $\pi_i$ попадания в выборку и предположим, что (в отличие от простого случайного отбора) эта вероятность не является одинаковой для всех домохозяйств. 

Тогда статистики, такие как средние по всей выборке, придающие одинаковый вес всем наблюдениям, будут завышать вес тех домохозяйств, которые попадают в выборку с большей вероятностью. Это может быть скорректировано взвешиванием с использованием \bfseries выборочных весов\mdseries, обратно пропорциональных вероятности включения в выборку:

\begin{equation}
\label{eq24.1}
w_i \propto 1/\pi_i.
\end{equation}
К примеру, вместо $\overline{z} = N^{-1} \sum_i z_i$ мы можем использовать взвешенное среднее:

$$\overline{z}_W = N^{-1} \sum_i w_iz_i/\sum_iw_i$$
Заметим, что в~(\ref{eq24.1}) важна только пропорциональность. Сумма весов не обязательно должна равняться единице, если мы делим на сумму весов. Для удобства масштабирования часто используется $\sum_iw_i = N^*$. В этом случае вес $w_i$ означает, что данное наблюдение представляет $w_i$ наблюдений из генеральной совокупности. Важно отметить, что при использовании весов нужно соблюдать осторожность. Иногда $w_i$  определяются как $w_i \propto \pi_i$, а некоторые компьютерные пакеты рассчитывают взвешенное среднее как $\sum_i w_iz_i/\sum_i(1/w_i)$. При использовании обратных выборочных весов легко ошибиться и взвесить неправильно. 

При использовании простой случайной выборки размера $N$ из конечной генеральной совокупности размера $N^*$ получаем $\pi_i = 1/N^*$, $w_i$ --- константы и $\overline{z}_W = \overline{z}$. 

Для \bfseries простой стратифицированной выборки \mdseries с простым случайным отбором  внутри страт, предположим, что в страту $s$ входит доля $H_s$ генеральной совокупности размера $N^*$, а в выборку входит $N_s$ наблюдений из этой страты. Тогда  $\pi_i = N_s/H_sN^*$. Соответственно, выборочные веса $w_i \propto H_s/N_s$. 

При \bfseries двухшаговом отборе без стратификации \mdseries пусть $\pi_c$ --- вероятность, что выбран $c$-ый PSU и $\pi_{jc}$ --- вероятность, что в PSU $c$ было выбрано домохозяйство $j$. Тогда выборочные веса $w_i \propto 1/(\pi_c N_c \pi_{jc} N)$, где $N_c$ --- количество опрошенных домохозяйство в PSU $c$ и $N = \sum_c N_c$. Двухшаговый отбор является \bfseries самовзвешивающим \mdseries, если выборочные вероятности на каждом шаге пропорциональны долям в генеральной совокупности, поэтому $\pi_c = N^*_c/N^*$ и $\pi_{jc} = 1/N^*_c$, где $N^*_c$ --- размер генеральной совокупности в $c$-ом PSU. Тогда веса $w_{jc}$ оказываются равными, как и в случае простого случайного отбора, но стандартные ошибки по-прежнему могут нуждаться в коррекции, как будет показано в части 24.8. 

Для Текущего обследования населения, CPS, где перепредставлены домохозяйства из маленьких штатов, может оказаться достаточным использовать $w_i \propto H_s / N_s$, где $s$ обозначает штат. В Текущем обследовании населения это используется в качестве базового веса, после чего ещё проводится коррекция на уровне USU если в USU входит слишком много домохозяйств. Следующая трудность заключается в том, что в страте опрашиваются не все PSU, а опрошенные домохозяйства могут быть нерепрезентативными для своей страты, если изучаемые PSU значительно отличаются от страты в целом. Поэтому осуществляются две дополнительные корректировки. Во-первых, это коррекция на нерепрезентативность расового состава на уровне страты. Во-вторых, веса корректируются так, чтобы выборочные оценки основных подгрупп (сформированных по штату, расе, полу и возрасту) совпадали с независимыми данными о составе населения. Более детальную информацию можно получить с помощью Бюро переписи населения США (U.S. Bureau of Census, 2002). Веса в Текущем обследовании населения, CPS, учитывают отличия состава выборки от состава генеральной совокупности по географическому расположению (штату), расе, полу и возрасту и составлены так, чтобы опрос был репрезентативен на национальном уровне. 

На практике вычисление выборочных весов для многоэтапных опросов может быть очень сложным процессом. Оценка весов может быть неправильной, но даже если они оценены корректно, они могут не принимать во внимание часть показателей, по которым выборка оказывается нерепрезентативной. 

\subsection{Взвешенная регрессия}

Нужно ли использовать взвешенную регрессию в случае, когда известны выборочные веса? Рассмотрим этот вопрос для случая, когда стратификация проводится не по зависимой переменной. Случай, когда стратификация проводится по зависимой переменной, рассмотрен в разделе 24.4. 

Рассмотрим оценку линейной регрессии

\begin{equation}
\label{eq24.2}
y_i = x'_i\beta + u_i.
\end{equation}
по данным опроса с известными выборочными весами $w_i$. В этом случае возможны две оценки: МНК

\begin{equation}
\label{eq24.3}
\hat \beta_{OLS} = (X'X)^{-1}X'y,
\end{equation}
и взвешенного МНК с использованием выборочных весов

\begin{equation}
\label{eq24.4}
\hat \beta_{WLS} = (X'WX)^{-1}X'Wy,
\end{equation}
где $W = \Diag[w_i]$. 

\subsection*{Правильно специфицированное условное математическое ожидание}

Оценка МНК работает, если предполагается, что $\E[u|x] = 0$. Тогда условное математическое ожидание линейно по $x$:

\begin{equation}
\label{eq24.5}
\E[y_i|x_i] = x'_i\beta.
\end{equation}
В этом случае оценка $\beta$ методом наименьших квадратов является состоятельной. Более того, она является эффективной по теореме Гаусса-Маркова, если ошибки $u_i$ гомоскедастичны. Оценка взвешенного МНК $\beta$ также состоятельна, но неэффективна, если ошибки гомоскедастичны (в силу того, что взвешивание в~(\ref{eq24.5}) учитывает нерепрезентативность выборки, но не гетероскедастичность). 

\subsection*{Неправильно специфицированное условное математическое ожидание}

Во многих приложениях~(\ref{eq24.5}) не выполняется. В качестве примера можно привести такие случаи, как пропущенные переменные, ситуации, где $\E[y|x]$ нелинейно по $x$ или где $\E[y_i|x_i] = x'_i \beta_i$ но некоторые компоненты в $\beta_i$ коррелируют с $x_i$. Линейная регрессия по-прежнему может интерпретироваться как наилучший линейный прогноз $y$ для заданного $x$ при квадратичной функции потерь, но требует адаптации для случая нерепрезентативной выборки. 

В выборке $(y_i, x_i)$ являются независимыми и одинаково распределенными. Мы можем записать (см. раздел 4.2)

$$y_i = X'_i \beta^* + u_i,$$
где $\E[u]=0$, $\Cov[x,u] = 0$ и
$$\beta^* = \left( \E[xx'] \right) ^{-1} \E[xy]. $$
Заметим, что мы больше не предполагаем $\E[u|x] = 0$, поэтому возможно, что $\E[y|x] \ne x'\beta$. 

Параметр $\beta^*$ называется \bfseries коэффициентом ценза, \mdseries он был предложен в работе Дюмушеля и Дункана (1983). Это предел по вероятности коэффициента из регрессии $y$ на $x$, который был бы получен при оценке модели на всей генеральной совокупности, а не на нерепрезентативной выборке. 

Если условное математическое ожидание нелинейно по $x$ и выборка непрепрезентативна, оценка метода наименьших квадратов в общем случае не сходится к $\beta^*$, потому что полученные по нерепрезентативной выборке $N^{-1}X'X$ не сходятся к истинным моментам $\E[x'x]$ и, аналогично, нет сходимости для $N^{-1}X'y$. Интуитивно понятно, что если условное среднее нелинейно по $x$, нет оснований верить, что линейные регрессии, оцененные по разным выборкам из одной генеральной совокупности, дадут одинаковые оценки. 

Однако взвешенного МНК с использованием выборочных весов может дать состоятельную оценку $\beta^*$. Если матрица весов W такая, что:

\begin{align}
\label{eq24.6}
&N^{-1}X'WX \overset{p}{\to}  \E[xx'], \nonumber \\
&N^{-1}X'Wy \overset{p}{\to}  \E[xy],
\end{align}
тогда $\hat \beta _{WLS} $ из~(\ref{eq24.4}) сходится к $\beta^*$. 

\subsection*{Простые стратифицированные выборки}

Значимая часть анализа для взвешенного метода наименьших квадратов представлена для случая простых стратифицированных выборок с простым случайным отбором внутри страты. Тогда становится ясным, что~(\ref{eq24.6}) выполняется при $w_i \propto H_s / N_s$, если $i$-ое исследованное домохозяйство находится в $s$-ой страте. 

В литературе также рассматривается возможность наличия разных параметров регрессии внутри страты. Предполагается, что $\E[y_i | x_i]  = x'_i \beta_s$ для домохозяйства $i$ в страте $s$. Целью может быть оценк среднего взвешенного по генеральной совокупности параметра $\beta_W = N^{-1} \sum_s N^*_s \beta_s$. Тогда в общем случае ни МНК, ни взвешенный МНК не сходятся к $\beta_W$, кроме ситуаций, где $\beta_s$ одинаковы для всех страт или независимы и одинаково распределены с постоянным средним по стратам. Важным исключением из этого случая является оценка среднего $y$ (регрессия на константу $x = 1$), когда взвешенное среднее из выборочных средних по стратам является несмещенным относительно среднего по генеральной совокупности. Подробнее см. в разделе 24.4.1 и работы Дюмушель, Дункан (1983), Дитон (1997) или Уллах, Бройниг (1998). 

\subsection*{Нужно ли использовать выборочные веса?}

Предшествующий анализ может быть использован для ответа на вопрос, нужно ли использовать выборочные веса при оценке моделей при \bfseries отсутствии эндогенной стратификации \mdseries. Рассматривается оценка (возможно нелинейных) моделей $\E[y|x]$, но те же рассуждения пременимы и для моделей любого другого показателя, связанного с условным распределением $y$ при условии $x$, к примеру, медианы или плотности. 

Если исследователь использует \bfseries структурный \mdseries или \bfseries аналитический \mdseries подход и предполагает, что модель $\E[y|x]$ правильно специфицирована, то нет необходимости использовать выборочные веса. Результаты анализа могут использоваться для исследования эффектов изменения $x$ на $\E[y|x]$. 

Если же используется \bfseries описательный \mdseries подход или подход, отталкивающийся от данных, нужно использовать выборочные веса. Параметры регрессии в таком случае интерпретируются как коэффициенты ценза. Стоит, однако, помнить, что в для опросов со сложной структурой невозможно получить веса, которые будут соответствовать~(\ref{eq24.6}) столь же точно, как в случае стратифицированных выборок с простым случайным отбором внутри страт. На практике выборочные веса строятся в соответствии с пропорциями отдельных подргупп (по возрасту, полу и расе) в генеральной совокупности. Нет гарантии, что такие веса будут отвечать~(\ref{eq24.6}). 

Некоторые опросы, такие как относительно небольшие лонгитюдные опросы нескольких тысяч домохозяйств, разрабатываются с расчётом на структурное моделирование. Однако, они обычно стараются обеспечить репрезентативность выборки, используя при этом кластеризованный отбор для снижения издержек проведения опроса. Другие опросы, такие как Текущее обследование населения, CPS, делают упор на точность описательных статистик, таких как национальные и региональные оценки уровней безработицы. Здесь разработчики используют подход, близкий к переписям, и явно предпочли бы проводить опросы каждый месяц, если бы это не было так дорого. 

Для любого вида данных микроэконометристы обычно стараются применять \bfseries структурный подход к моделированию. \mdseries В качестве примера, рассмотрим регрессию доходов на уровень образования и социоэкономические характеристики, такие как возраст, пол и раса, но без показателей, характеризующих врожденные способности. 

Большинство эконометристов ограничатся описательной интерпретацией коэффициента при образовании в МНК регрессии из-за эндогенности образования. Интерпретироваться это будет следующим образом: если прочие регрессоры неизменны, один дополнительный год образования связан, но не обазятельно является причиной, с ростом дохода на, скажем, 6\%. Здесь использование выборочных весов в МНК регрессии нужно, чтобы трактовать полученные оценки как характеристики всей генеральной совокупности, а не просто одной, возможно нерепрезентативной, выборки. Даже при том, что никакая интерпретация с точки зрения причинности невозможна, эта оценка может быть полезна для понимания того, как доход меняется с изменением образования при фиксации некоторых основных социоэкономических переменных. В конце концов, суммирование и описание данных --- одна из целей статистики. 

Состоятельная оценка коэффициента при образовании может быть получена при использовании более продвинутых процедур оценивания, таких как инструментальные переменные или методы для панельных данных. В этом случае коэффициенту можно давать интерпретацию с точки зрения причинности. Взвешивание с использованием выборочных весов больше не нужно, тогда как обычное взвешивание для повышения эффективности, если, к примеру, ошибки гетероскедастичны, может быть полезным. 

Может ли модель считаться корректно специфицированной или нет --- важный вопрос. Если она специфирована корректно, взвешенные и невзвешенные оценки, полученные по выборке, будут сходиться по вероятности к одному и тому же пределу, потому что обе являются состоятельными. Это даёт возможность тестировать корректность спецификации тестом Хаусмана, рассматривая разницу между оценками для взвешенной и невзвешенной регрессий. Тест для линейной регрессии предложен в работе Дюмушель, Дункан (1983). 

\subsection{Прогнозирование}

Рассмотрим нелинейную регрессию с правильно специфицированным условным средним, $g(x, \beta)$ и отсутствием эндогенности. Невзвешенная оценка нелинейного МНК даёт состоятельную оценку $\beta$ и может трактоваться с точки зрения причинности. В частности, мы можем использовать $\partial g(x, \hat \beta)/\partial x$ для вычисления эффекта изменения $x$ на единицу на условное среднее. 

Этот эффект изменяется в зависимости от $x$, потому что $g(\cdot)$ нелинейна. Оценка среднего по генеральной совокупности эффекта:
$$
\hat \E\left[ \frac{\partial y}{\partial x}\right] = \sum_{i=1}^N w_i \frac{\partial g(x_i, \hat\beta)}{\partial x_i},
$$
где $w_i$ --- выборочные веса. Аналогично, если оценивать эффект для средних значений регрессоров, лучше использовать среднее взвешенное значение $x$, чем невзвешенное выборочное среднее $x$. 

Даже если полученные невзвешенные оценки параметров являются состоятельными, взвешивание может быть использовано в дальнейшем для вычисления предельных эффектов, если целью является получение эффектов для генеральной совокупности, а не для  выборки. 

\section{Эндогенная стратификация}

Стратификация широко используется, потому что она может повысить точность оценивание, или эквивалентно уменьшить расходы на проведение опроса при заданном уровне точности. К примеру, более точные оценки среднего уровня безработицы в штатах с маленьким населением может быть получено путём увеличения доли бедных штатов в выборке. По тем же причинам в выборках могут быть перепредставлены меньшинства. 

Одно из затруднений, уже рассмотренное в разделе 24.3, заключается в том, что параметры могут изменяться между стратами. К примеру, средний уровень безработицы может быть разным для разных страт. В таком случае используется описательный подход и взвешенные оценки. 

Микроэконометристы часто предпочитают структурный подход и предполагают, что параметры не изменяются от страты к страте. Тогда (см. раздел 24.3) стратификация не вызывает дополнительных затруднений и можно использовать невзвешенную регрессию. Важная оговорка заключается в том, что проблемы всё-таки возникают, если стратификация основывается на значении зависимой переменной. К примеру, если люди с низким доходом перепредставлены и доход является зависимой переменной, оценки коэффициентов становятся несостоятельными. Заметим, что если стратификация проводится по регрессорам, таким как раса, и это приводит к перепредставленности людей с низким непрямым образом, проблем не возникают. Они есть только если стратификация проводится по доходу прямым образом. 

В данном разделе мы определим эндогенную стратификацию и проанализируем возникающие при этом трудности. Потом мы представим оценки, которые будут состоятельными в такой ситуации. Самая простая из них --- это взвешенная оценка, которая может быть использована, если известны вероятности страт как в выборке, так и в генеральной совокупности. Метод представлен в разделе 24.4.5. 

\subsection{Схемы стратификации}

Для данных $z \in \mathcal Z$ страты --- это подмножества $\mathcal Z$. При проведении эконометрического анализа данные обычно разделяются на зависимую переменную $y \in \mathcal Y$, которую, в целях сохранения общности, можно считать вектором, и регроссор, или независимую переменную, $x \in \mathcal X$. Тогда страта $C_s$, $s=1, \dots ,S$ определяется как подмножество $ \mathcal Y \times \mathcal X$. Использованная здесь нотация предложена в работе Имбенс, Ланкастер (1996), где представлены замечательные примеры, воспроизведенные в таблице 24.1. 

Отбор внутри страты предполагается случайным, но некоторые страты могут быть перепредставлены. Из таблицы 24.1 становится ясно, что страты могут суммироваться во множество, большее или меньшее, чем выборочное. Для четвертой и пятой схем стратификация может проводиться только по эндогенным переменным, только по экзогенным или по смеси из них. 

Эконометрическая литература фокусируется на схемах отбора с эндогенным компонентом, потому что в этом случае обычные условные оценки метода максимального правдоподобия оказываются несостоятельными. 

Эндогенная стратификация уже рассматривалась в главе 16. В качестве примера, рассмотрим \bfseries усечённую регрессию\mdseries, где мы наблюдаем $y$ только если $y>0$, поэтому стратификация проводится только по $y$. Тогда для выборочных данных условная плотность $y$ при условии $x$ --- это усеченная в нуле плотность, которая разделяет неусеченную плотность в $\Pr[y > 0|x]$ и тогда:

$$
f^s(y|x,\theta) = \frac{f(y|x,\theta)}{1-F(0|x,\theta)},
$$
где индекс $s$ используется, чтобы отличить \bfseries выборочную плотность \mdseries, от истинной $f(y|x,\theta)$. Как отмечено в главе 16, такая схема отбора имеет тенденцию выбрасывать наблюдения с низкими значениями $y$ при заданном $x$. Предположим $\E[y|x] = \beta_1 + \beta_2 x$ и $\beta_2 > 0$. Тогда для низких значений $x$ будет слишком много относительно высоких значений $y$. Регрессия будет завышать прогнозы $\E[y|x]$ дли низких значений $x$, приводя тем самым к смещению вверх оценки свободного члена $\beta_1$ и занижению коэффициента наклона $\beta_2$. 
 
\begin{table}[h]
\caption{\label{tab:pred} Стратификация со случайным отбором внутри страты}
\begin{center}
\begin{tabular}{lll}
\hline
\hline
 Схема стратификации & Определение & Вид стратификации \\
\hline
Простая случайная & $S = 1, C_1 = \mathcal Y \times \mathcal X$ & Одна страта покрывает все выборочное пространство \\
Чистая экзогенная & $C_s = \mathcal Y \times \mathcal X_s$, где $\mathcal X_s \subset \mathcal X$ &  Стратификация только по регрессорам, \\
& & не по зависимой переменной \\
Чистая эндогенная & $C_s = \mathcal Y_s \times \mathcal X$, где $\mathcal Y_s \subset \mathcal Y$ &  Стратификация только по зависимой переменной, \\
& & не по регрессорам \\
Пополненная выборка & $S = 2, C_1 = \mathcal Y \times \mathcal X,$ & Случайная выборка, дополненная наблюдениями \\
& $C_2 \subset \mathcal Y \times \mathcal X \nonumber$ & из части выборочного пространства \\
Разделенная & $C_s \subset \mathcal Y \times \mathcal X, C_s \cap C_1 = \emptyset,$ & Выборочное пространство разбивается на \\
& $\bigcup_{s=1}^S C_s =  \mathcal Y_s \times \mathcal X.$ & непересекающиеся страты, которые покрывают \\
& & все выборочное пространство \\
\hline
\hline
\end{tabular}
\end{center}
\end{table}

Второй пример --- это \bfseries отбор, основанный на выборе \mdseries для бинарных или мультиномиальных данных, где выборки создаются на основе дискретной переменной $y$. К примеру, если выбор осуществляется между поездкой на работу на автобусе или на машине мы можем перепредставить тех, кто ездит на автобусе, если им пользуется мало людей. Этот пример рассматривается ниже. Он схож на \bfseries исследования случай-контроль \mdseries в медицинской литературе, где, к примеру, полная выборка людей, умерших от болезни ($y=1$) сравнивается с подвыборкой того же размера из всего множества людей, которые не умерли от болезни ($y=0$). Цель --- определить, может один или несколько регрессоров предсказать $y=1$. 

Близкий пример --- это счётные данные о количестве посещений, собираемые путём \bfseries отбора на месте\mdseries, к примеру, в местах отдыха, торговых центрах или в кабинетах врачей. Такие данные усечены, потому что индивиды с $y=0$ не входят в выборку, а посетители, приходящие часто, перепредставлены. Shaw (1998) показывает, что выборочное распределение данных, $f^s(y|x,\theta)$ связано с распределением по генеральной совокупности следующим образом:

\[
f^s(y|x,\theta) = f(y|x,\theta) \frac{y}{\E[y|x,\theta]}.
\]

В этом случае схема отбора очевидно становится эндогенной, несмотря на то, что она не является стратифицированной. 

\subsection{Эндогенность, вызванная стратификацией}

Схемы отбора, такие как схемы со стратификацией, ведут к тому, что функция плотности исследуемой величины в выборке отличается от плотности в генеральной совокупности. При чисто эндогенной стратификации оценке метода максимально правдоподобия остаются состоятельными, потому что условная плотность $y$ при заданном $x$ в выборке не отличается от истинной. Однако, если в стратификации есть эндогенный компонент, эти условные плотности отличаются, что было показано в предыдущих примерах. Теперь мы обсудим этот вопрос более подробно. 

Цель метода максимального правдоподобия --- это получение состоятельных оценок параметров $\theta$ в $f(y|x, \theta)$. В общем, оценки ММП базируются на на вероятности, полученной на основе совместного распределения данных $(y, x)$. На практике часто оказывается достаточно просто сформировать условную вероятность на основе условного распределения $y$ при условии $x$. Это более простой подход может дать состоятельные оценки при предположении, что $x$ экзогенен по отношению к $y$. Тогда совместная плотность представима как:

\begin{equation}
\label{eq24.7}
g(y, x|\theta) = f (y| x, \theta) \times h(x). 
\end{equation}
где параметры плотности $x$ опущено, потому что нет никакой необходимости их оценивать. 

Мы всегда можем написать $g(y, x) = f (y| x) \times h(x)$. Предположение, сделанное в~(\ref{eq24.7}) заключается в том, что параметры $\theta$ появляются в $f(y | x, \theta)$, но не появляются в $h(x)$. В общем случае, вместо~(\ref{eq24.7}) мы могли бы получить:

\begin{equation}
\label{eq24.8}
g(y, x|\theta) = f (y| x, \theta) \times h(x|\theta). 
\end{equation}
Тогда один или больше компонентов $x$ становятся эндогенными по отношению к $y$, потому что теперь появляется обратная связь --- $y$ зависит от $x$, но $x$, в свою очередь, зависит от $y$ через присутствие $\theta$ в $h(x|\theta)$. Классический пример --- это линейные одновременные уравнения. В таких случаях оценка ММП должна базироваться на \bfseries совместном правдоподобии \mdseries: 

\begin{equation}
\label{eq24.9}
\ln \mathrm{L}_{JOINT}(\theta) = \sum_{i=1}^n \ln{f(y_i|x_i, \theta)} + \sum_{i=1}^n \ln h(x_i|\theta). 
\end{equation}
Это даёт состоятельную оценку $\theta$ если, из главы 5:

\begin{equation}
\label{eq24.10}
0 = \E \left[\frac{\partial \ln{g(y, x|\theta)}}{\partial \theta} \right] = \E \left[\frac{\partial \ln{f(y | x,\theta)}}{\partial \theta} \right] + \E \left[ \frac{\partial \ln{h(x|\theta)}}{\partial \theta} \right]
\end{equation}
Условие~(\ref{eq24.10}) выполнено, если плотность $g(y, x|\theta)$ правильно специфицирована и диапазон данных не зависит от $\theta$. Условная оценка метода максимального правдоподобия максимизирует \bfseries условное правдоподобие \mdseries : 
$$
\ln \mathrm{L}_{COND}(\theta) = \sum_{i}\ln f(y_i|x_i, \theta).
$$
Оценка условного ММП состоятельна, если $\E[ \partial \ln f(y|x,\theta) / \partial \theta] = 0$. Это необходимое условие следует из~(\ref{eq24.10}) если $x$ эндогенный, так как~(\ref{eq24.10}) упрощается из-за того, что в этом случае $\partial \ln h(x) / \partial \theta = 0$. Если же $x$ эндогенный, этого упрощения не происходит, так как второй член правой части~(\ref{eq24.10}) не исчезает. Поэтому оценка условного ММП несостоятельна при эндогенном $x$. 

Проблема, которая возникает при стратификации и других схожих схемах отбора заключается в том, что даже если совместная плотность в генеральной совокупности удовлетворяет~(\ref{eq24.7}) и не изменяется по стратам, схемы отбора могут быть устроены так, что что совместная плотность $(y,x)$ в выборке принимает более общую форму:

\begin{equation}
\label{eq24.11}
g^s(y,x|\theta) = f^s(y|x,\theta) \times h^s(x|\theta),
\end{equation}
где индекс $s$ используется для того, чтобы обозначить использование определенной схемы отбора. Тогда условная оценка ММП может быть несостоятельной, даже при том, что она была бы состоятельной в случае простого случайного отбора. 

При \bfseries чистом экзогенном отборе \mdseries единственное отличие между выборочным и истинным распределениями проявляется в частном распределении $x$. Предполагая, что в генеральной совокупности выполнено~(\ref{eq24.7}), получаем, что в выборке:

$$
g^s(y,x|\theta) = f(y|x,\theta) \times h^s(x).
$$
Ясно, что условные оценки ММП будут состоятельными, так как условная плотность по-прежнему $f(y|x,\theta)$ и $\theta$ не присутствует в $h^s(x)$. 

При \bfseries эндогенном отборе \mdseries в выборке выполняется более общий результат~(\ref{eq24.11}), даже если~(\ref{eq24.7}) выполняется в генеральной совокупности. Выборочное и истинное распределения $y$ при условии $x$ могут различаться, с $f^s(y|x,\theta) \ne f(y|x,\theta)$ и возможной зависимостью $h^s(x|\theta)$ от $\theta$.  

\subsection{Эндогенный отбор}

При чистом эндогенном отборе частное распределение $y$ в выборке отличается от распределения в генеральной совокупности. Обозначим как $h(y)$ истинное распределение $y$ и как $h^s(y)$ --- выборочное распределение (совместное, условное и частное распределение мы обозначаем, соответственно, как $g$, $f$, и $h$. Читатель должен понимать, что $h(y)$ отличается от $h(x)$.)

Совместное распределение $x$ и $y$ при чистом эндогенном отборе проще получить, беря условное распределение $x$, чем $y$. Тогда:

\begin{equation}
\label{eq24.12}
g^s(y,x) = f(x|y) h^s(y).
\end{equation}
Упрощение произошло, потому что условное распределение $x$ при условии $y$ неизменно при чистом эндогенном отборе, так что $f^s (x|y) = f(x|y)$. Теперь нам надо выразить $f(x|y)$ в терминах $f(y|x)$. Теперь

\begin{equation}
\label{eq24.13}
f(x|y) = \frac{g(y,x)}{h(y)} = \frac{f(y|x)h(x)}{h(y)}.
\end{equation}
Подставляя~(\ref{eq24.13}) в~(\ref{eq24.12}) получаем

$$
g^s(y,x|\theta) = f(y|x, \theta) \times \frac{h^s(y)}{h(y|\theta)} \times h(x),
$$
где

$$
h(y|\theta) = \int g(y,x|\theta) dx = \int f(y|x,\theta) h(x) dx. 
$$
Условная оценка ММП, использующая только $f(y|x,\theta)$ будет несостоятельной, потому что член $h(y|\theta)$ будет проигнорирован. Вместо этого нужно максимизировать совместное правдоподобие, которое дополнительно включает $h(y|\theta)$. 

\subsection{Эндогенно стратифицированные выборки}

Рассмотрим теперь схемы стратификации, представленные в разделе 24.4.1. Плотность в генеральной совокупности:

$$
g(y, x| \theta) = f(y|x, \theta) h(x). 
$$
Всего есть $S$ страт, где $s$-ая страта --- подмножество $C_s$ множества $\mathcal Y \times \mathcal X$. 

Есть важное различие между вероятностью для наблюдения в генеральной совокупности попасть в $C_s$ и вероятностью попадания в выборку наблюдения из $C_s$. Определим:

\begin{align}
 &H_s = \Pr[\text{Выбрать наблюдение из} \ C_s], \nonumber \\
 &Q_s(\theta) = \Pr[ \text{Случайно выбранное наблюдение из совокупность находится в} \ C_s].
\label{eq24.14}
\end{align}
Здесь $H_s$ определяется устройством выборки, тогда как

\begin{equation}
\label{eq24.15}
Q_s(\theta) = \int_{C_s} f(y|x, \theta) h(x) dy dx. 
\end{equation}
Вероятности страт могут быть известны или неизвестны. Страта считается перепредставленной, если $H_s > Q_s$. 

Мы начнём с получения совместной плотности $s$, $y$ и $x$, где $s$ --- это индикатор для страты, из которой берётся наблюдение. В генеральной совокупности

$$
g(s,y,x|\theta) = Q_s(\theta) g(y,x|s, \theta). 
$$
В выборке, частное распределение индикатора страты отличается от $Q_s$ и

$$
g^s(s,y,x|\theta) = H_s g(y,x|s,\theta) = H_s \frac{f(y|x,\theta)h(x)}{Q_s(\theta)},
$$
где второе равенство выполняется в силу того, что $g(y,x|s)$ равно плотности $g(y,x) = f(y|x) h(x)$, делённой на вероятность попадания в страту $s$ в генеральной совокупности, так что плотность интегрируется по $C_s$ к единице. 

Следовательно, совместная плотность

\begin{equation}
\label{eq24.16}
g^s(s,y,x|\theta) = \frac{H_s}{Q_s(\theta)} f(y|x, \theta) h(x),
\end{equation}
где $Q_s(\theta)$ определяется из~(\ref{eq24.15}). Условная оценка ММП, базирующаяся на истинной условной плотности $f(y|x,\theta)$, будет несостоятельной для $\theta$, потому что она игнорирует член $Q_s(\theta)$, который зависит от $\theta$. 

Можно предложить ряд состоятельных оценок. Здесь мы рассмотрим оценки методом максимального правдоподобия, оценки обобщённого метода моментов и более простые взвешенные оценки, которые могут быть получены при знании вероятностей попадания в страту и для выборки $H_s$ и для генеральной совокупности $Q_s(\theta)$. 

\subsection*{Оценки метода максимального правдоподобия}

Построение ММП-оценки на основе совместной плотности $g^s(s,y,x|\theta)$ в~(\ref{eq24.16}) может быть затруднено, потому что, как следует из~(\ref{eq24.15}), распределение $Q_s(\theta)$ зависит от $h(x)$. Одно из возможных решений заключается в спецификации плотности $h(x)$. Такой подход не используется, потому что эконометристы стараются избегать спецификации распределений регрессоров, даже когда хотят специфицировать условное распределение зависимой переменной. 

Вместо этого используется полупараметрический подход, цель которого --- оценить параметры специфицированной плотности $f(y|x, \theta)$, не специфицируя плотность $h(x)$. Для простоты предположим, что мы знаем вероятности $H_s$. Косслетт (1981a) получил оценку ММП с эндогенной стратификацией, сперва разрешив $x$ быть дискретным с вероятностью $w_i$ того, что случится $x_i$, и максимизируя совместное правдоподобие по $\theta$ и $w_i$, $i = 1, \dots , N$. Условия первого порядка могут быть сведены к концентрированному правдоподобию, в котором будет только $(q+S-1)$ параметров $\theta$ и фукнций $\lambda_s(\theta), s = 1, \dots, S-1$. Далее концентрированное правдоподобие максимизируется по $\theta$ и $\lambda_s$ и получаются те же оценки, что и при максимизации по $\theta$ и $\lambda_s(\theta)$. Затем, так как можно трактовать $\lambda_s$ как параметр, та же процедура может быть использована в случае непрерывных регрессоров. Проблема размерности $q$ и бесконечно-мерная неизвестная плотность $h(x)$ была сведена к размерности $q+S-1$. 

\subsection*{Оценки обобщённого метода моментов}

Замечательные результаты Косслетта (1981a) сложно применить на практике. 

Имбенс (1992) разработал более простую \bfseries оценку ОММ с эндогенной стратификацией, \mdseries которая обладает такой же эффективностью, как и оценка Косслетта. Довольно общее описание и презентация этой оценки дана в Имбенс, Ланкастер (1996) для стратифицированных выборок, полученных мультиномиальным отбором, стандартным стратифицированным отбором, или отбором по вероятности переменных. Совместная плотность --- это опять $g^s(s,y,x|\theta)$ из~(\ref{eq24.16}) и выборочные вероятности страт $H_s$ могут быть неизвестны. ОММ строится на $S-1$ уравнении для $H_s$, $q$ уравнениях для $\theta$, основанных на функции условного правдоподобия $y$ при заданных $s$ и $x$, $S-1$ уравнениях для ограничений на вероятности страт в генеральной совокупности $Q_s(\theta)$ и последнее ограничение необязательно, если есть линейной ограничение на $Q_s(\theta)$, что происходит, к примеру, если страты взаимоисключающи и покрывают всё выборочное пространство. 

\subsection{Взвешенные оценки}

С эндогенной стратификацией легко работать, когда выборочные и истинные вероятности страт $H_s$ и $Q_s(\theta)$, определённые в~(\ref{eq24.14}), известны, хотя оценка в этом случае и не является полностью эффективной. Мы начнём с оценки ММП, прежде чем перейдём к более общим оценкам. 

\subsection*{Взвешенная оценка ММП}

Мански, Лерман (1977) предложили взвешенную оценку максимального правдоподобия (weighted maximum likelihood, WML). Она максимизирует

\begin{equation}
\label{eq24.17}
Q_{WML}(\theta) = \sum_i \frac{Q_i}{H_i} \ln f(y_i|x_i,\theta),
\end{equation}
где $H_i = H_s$ и $Q_i = Q_s$, если $i$-ое наблюдение принадлежит страте $s$. 

Мански, Лерман (1977) назвали эту оценку \bfseries взвешенной оценкой для экзогенного отбора \mdseries (weighted exogenous sampling estimator, WESML), так как~(\ref{eq24.17}) умножает обычный член $\ln f(y_i|x_i,\theta)$ в условном правдоподобии при экзогенном отборе на вес $H_i/Q_i$. Однако, обозначение WESML может привести к путанице, потому что проблема здесь заключается в эндогенности --- просто оказывается, что правильное взвешивание обычной экзогенной оценки может дать состоятельную оценку. 

Несмотря на сходство, целевая функция $Q_{WML}(\theta)$ с формальной точки зрения не является правдоподобием, так как~(\ref{eq24.16}) не предполагает, что выборочная условная плотность $y$ при заданных $x$ и $s$ задаётся $f^s(y|x,\theta) = f(y| x,\theta)^{Q_s/H_s}$. Тем не менее, оценка взвешенного максимального правдоподобия WML является состоятельной. Условия первого порядка для неё:

\begin{equation}
\label{eq24.18}
\sum_i \frac{Q_i}{H_i} \frac{\partial \ln{f(y_i|x_i,\theta)}}{\partial \theta} = 0
\end{equation}
Эта оценка будет состоятельна, если члены суммы имеют нулевые математические ожидания по выборочной плотности $g^s(s,y,x|\theta)$ из~(\ref{eq24.16}). Теперь

\begin{align}
\label{eq24.19}
&\E_s \left[ \frac{Q_s}{H_s} \frac{\partial \ln f(y|x,\theta)}{\partial \theta} \right] \nonumber \\
&=\int \int  \frac{Q_s}{H_s} \frac{\partial \ln f(y|x,\theta)}{\partial \theta} \frac{H_s}{Q_s(\theta)}f(y|x,\theta)h(x)dydx  \nonumber \\
&=\int \int  \frac{\partial \ln f(y|x,\theta)}{\partial \theta} f(y|x,\theta)h(x)dydx \nonumber \\
&=\int \E \left[\frac{\partial \ln f(y|x,\theta)}{\partial \theta} \right] h(x) dx \nonumber \\
&=0,
\end{align}
при обычных условиях, что в генеральной совокупности плотность удовлетворяет $\E[\partial \ln f(y|x,\theta) / \partial \theta] = 0$. Тогда оценка взвешенного максимального правдоподобия состоятельна в присутствии эндогенной стратификации. 

Уравнение для информационной матрицы не выполняется для целевой функции $Q_{WML}(\theta)$ в~(\ref{eq24.17}), так что нам надо использовать сэндвич-форму $N^{-1}A^{-1}BA^{-1}$ для асимптотической дисперсии $\hat \theta_{WML}$, где

\begin{equation}
\label{eq24.20}
A(\theta_0) = \plim \left. \frac{1}{N} \sum_{i=1}^N \frac{Q_i}{H_i} \frac{\partial^2 \ln f(y_i|x_i, \theta)}{\partial \theta \partial \theta'} \right|_{\theta_0}
\end{equation}
и
\begin{equation}
\label{eq24.21}
B(\theta_0) = \plim \left. \frac{1}{N} \sum_{i=1}^N \left( {\frac{Q_i}{H_i}} \right) ^2 \frac{\partial \ln f(y_i|x_i, \theta)}{\partial \theta} \frac{\partial \ln f(y_i|x_i, \theta)}{\partial \theta'}  \right|_{\theta_0}.
\end{equation}

Такая оценка менее эффективна, чем оценка ММП Косслетта или Имбенса, но её достаточно просто реализовать. Но, разумеется, она предполагает знание вероятностей страт. 

\subsection*{Взвешенная М-оценка}

Взвешенная оценка ММП может быть применена к другим оценкам помимо условной ММП-оценки. К примеру, Хаусман, Уайз (1979) рассматривают схожую взвешенную оценку для регрессии методом наименьших квадратов. 

Предположим, что используется простой случайный отбор. Тогда мы минимизируем $\sum_i q(y_i|x_i, \theta)$, с условиями первого порядка $\sum_i \partial q(y_i|x_i, \theta) / \partial \theta = 0$, и предположим, что в генеральной совокупности
$$
\E[\partial q(y|x,\theta) / \partial \theta] = 0,
$$
необходимое условие для состоятельности. Тогда, если отбор --- эндогенно стратифицированный как в разделе 24.2 и в выборке и генеральной совокупности известны вероятности страт $H_s$ и $Q_s$, $\theta$ может быть состоятельно оценена \bfseries взвешенной М-оценкой \mdseries $\hat \theta_W$, которая минимизирует

\begin{equation}
\label{eq24.22}
Q_W(\theta) = \sum \frac{Q_i}{H_i} q(y_i|x_i, \theta). 
\end{equation}
Доказательство состоятельности следует из~(\ref{eq24.18}) и~(\ref{eq24.19}) для оценки взвешенного максимального правдоподобия и дисперсионной матрицы в форме $N^{-1}A^{-1}BA^{-1}$, где $A$ и $B$ определены в~(\ref{eq24.20}) и~(\ref{eq24.21}) с единственным изменением --- заменой $\partial \ln f(y_i|x_i, \theta) / \partial \theta$ на $\partial q(y_i|x_i,\theta) / \partial \theta$. Вулдридж (2001) приводит формальное доказательство. 

Аналогично, для оценки, основанной на $q$ моментов
$$
\E[h(y,x,\theta)] = 0,
$$
при эндогенной стратификации, используем \bfseries взвешенную оценку оценивающих уравнений \mdseries, которая решает
$$
\sum_i \frac{Q_i}{H_i} h(y_i, x_i, \theta) = 0.
$$
Результаты взвешенного ММП применяются с $\partial \ln{f(y_i|x_i, \theta)} / \partial \theta$ заменённой на $h(y_i| x_i, \theta)$. 

Отметим, что веса $Q_i/H_i$ те же самые, что и предложенные в разделе 24.3.2 для оценки параметра ценза при простом экзогенном стратифицированном отборе. Мотивация, однако, отличается. В данном разделе предполагается, что условные моменты корректно специфицированы, так что при экзогенном стратифицированном отборе невзвешенные оценки были бы состоятельными и эффективными. Веса становятся необходимыми, если стратификация эндогенна. 

\section{Кластеризация}

Разделы 24.3 и 24.4, посвященные взвешиванию и стратификации, рассматриваются методы для учета устройства выборки, ведущего к отличиям выборочного распределения от распределения в генеральной совокупности. Предположение о независимости попавших в выборку наблюдений сохранялось. 

В реальности же, данные опросов, как правило, не независимы. Это может возникать в силу использования кластеризованных выборок для уменьшения расходов на проведение опроса, таких, как интервьюирование нескольких домохозяйств в одном квартале. В таких случаях, данные могут коррелировать внутри кластера в силу наличия общего ненаблюдаемого эффекта кластера. Такая зависимость, однако, может возникать и при простом случайном отборе. К примеру, может присутствовать ненаблюдаемый эффект, общий для всех домохозяйств в одном штате. 

Существует несколько методов учёта ненаблюдаемых эффектов внутри кластера. Если внутри-кластерные ненаблюдаемые эффекты не коррелированы с регрессорами, необходимо скорректировать только дисперсии коэффициентов. Если же они коррелированы с регрессорами, оценки параметров регрессии становятся несостоятельными и нужны альтернативные оценки. Анализ также затрудняется тем, что методы могут изменяться в зависимости от того, рассматривается ли большое количество маленьких кластеров или несколько больших. Другие проблемы, такие как взвешивание и стратификация рассмотрены в разделе 24.6. 

Обозначения и модели представлены ниже. Основные различия сводятся к разнице между случайными кластерными эффектами и постоянными кластерными эффектами, схожей с ситуацией для панельных данных. Разные способы оценивания представлены ниже. 

\subsection{Модели с индивидуальными эффектами кластеров}

Рассмотрим оценку линейной регрессии по данным $(y_i,x_i)$, $i = 1, \dots, N$, где $i$ обозначает $i$-ое наблюдение в выборке, к примеру, домохозяйство. 

Проблема заключается в том, что некоторые параметры регрессии могут изменяться в зависимости от кластера $c$, $c = 1, \dots, C$. Пусть $i$-ое наблюдение из выборки --- это $j$-ое наблюдение в $c$-ом вошедшем в выборку кластере. Тогда общая модель для кластеризованных данных:

\begin{equation}
\label{eq24.23}
y_{jc} = x'_{jc} \beta_c + u_{jc}, \qquad j = i, \dots , N_c, \: c = 1, \dots, C.
\end{equation}
где $\Cov[u_{jc}, u_{kc}] \ne 0$, однако $\Cov[u_{jc}, u_{kd}] = 0$ при $c \ne d$. Эта модель учитывает влияние кластеров как через параметры регрессии, которые изменятся в зависимости от кластера, так и через ошибки. которые коррелированы внутри кластера. 

Здесь мы сконцентрируемся на частном случае, на \bfseries модели с индивидуальными эффектами кластеров:\mdseries

\begin{equation}
\label{eq24.24}
y_{jc} = x'_{jc} \beta + \alpha_c + \e_{jc}.
\end{equation}

Здесь только постоянный коэффициент регрессии $\alpha_c$ зависит от кластера, тогда как коэффициенты наклона предполагаются постоянными. В самой простой модели ошибка $\e_{jc}$ предполагается гомоскедастичной

\begin{equation}
\label{eq24.25}
\e_{jc} \sim [0,\sigma^2_{\e}],
\end{equation}

Это предположение может быть ослаблено, чтобы добавить в модель гетероскедастичность и корреляцию внутри кластера. Что более существенно, можно сделать разные предположения относительно $\alpha_c$, что приведёт к двум разным моделям, которые мы сейчас рассмотрим. 

\subsection*{Случайные кластерные эффекты}

В \bfseries моделях со случайными кластерными эффектами \mdseries (cluster-specific random effects, CSRE) константы $\alpha_c$ в~(\ref{eq24.24}) считаются случайными с распределениями, не зависящими от наблюдаемых переменных. В простейшем случае предполагается, что

\begin{equation}
\label{eq24.26}
\alpha_c \sim [0, \sigma^2_{\alpha}].
\end{equation}

Эта модель --- прямой аналог модели со случайными эффектами для панельных данных. Это простая линейная регрессия $y_{jc}$ на $x_{jc}$ с усложнением, заключающимся в том, что ошибки $\alpha_c + \e_{jc}$ коррелированы для наблюдений в одном кластере. МНК-оценки состоятельны, но неэффективны. Корреляция между ошибками приводит к необходимости корректировать стандартные ошибки для МНК-оценок. Оценки обобщенного МНК более эффективны. 

При выполненных предпосылках~(\ref{eq24.25}) и~(\ref{eq24.26}) на $\e_{jc}$ и $\alpha_c$, $\V[\alpha_c + \e_{jc}] = \sigma^2_{\alpha} + \sigma^2_{\e}$ и $\Cov[\alpha_c + \e_{jc}, \alpha_c + \e_{kc}] = \sigma^2_{\alpha}$ для $k \ne j$. Определим \bfseries внутриклассовый коэффициент корреляции: \mdseries

\begin{equation}
\label{eq24.27}
\rho = \Cor[\alpha_c + \e_{jc}, \alpha_c + \e_{kc}] = \frac{\sigma^2_{\alpha}}{\sigma^2_{\alpha}+\sigma^2_{\e}}.
\end{equation}

Есть взаимно-однозначное соответствие между $(\sigma^2_{\alpha}, \sigma^2_{\e})$ и $(\sigma^2, \rho)$, где $\rho$ определяется~(\ref{eq24.27}) и $\sigma^2 = \sigma^2_{\alpha} + \sigma^2_{\e}$. Модель CSRE эквивалентна модели с постоянной внутриклассовой корреляцией. Модель также может иметь Байесовскую интерпретацию, если рассматривать каждое наблюдение как имеющее свою собственную константу $\alpha_{jc}$, которая извлекается из одномерного распределения, и обращаясь к критерию взаимозаменяемости, так что нижний индекс в $\alpha_{jc}$ --- это просто обозначение и не имеет особого значения. Во всех случаях кластеризация имеет ожидаемый эффект, вызывая положительную корреляцию между ошибками внутри кластеров. 

\subsection*{Постоянные кластерные эффекты}

В \bfseries моделях со постоянными кластерными эффектами \mdseries (cluster-specific fixed effects, CSFE) константы $\alpha_c$ из~(\ref{eq24.23}) --- случайные и ненаблюдаемые, как и в CSRE, но могут быть коррелированы с регрессорами. $x_{jc}$ больше не включает константу. 

Эта модель --- прямой аналог модели с постоянными эффектами для панельных данных. Условное математическое ожидание $\E[y_{jc}|x_{jc},\alpha_c] = x'_{jc} \beta + \alpha_c$. Оценка МНК из регрессии $y_{jc}$ на $x_{jc}$ несостоятельна для $\beta$, если пропущенная переменная $\alpha_c$ коррелирована с $x_{jc}$. Состоятельная оценка $\beta$ требует состоятельной оценки $\alpha_c$, которая возможна, если кластеры большие. Если же кластеры маленькие индивидуальные $\alpha_c$ необходимо устранять взятием разностей. 

\subsection*{Сравнение с анализом панельных данных}

Структура и терминология рассматриваемой проблемы, очевидно, очень похожа на случай с анализом панельных данных, рассмотренный в главах 21 и 23. Но в то же время, есть и ряд различий. 

Для панельных данных отдельное наблюдение, такое, как домохозяйство, наблюдается больше, чем один раз, тогда как для случая кластеров оно наблюдается только один раз. В принятой в анализе панельных данных системе обозначений $it$ первый индекс соответствует кластеру, если рассматривается короткая панель, тогда как в кластерных обозначениях $jc$ второй индекс соответствует кластеру. Анализ панельных данных сосредоточен на сбалансированных панелях, тогда как кластеризованные данные обычно несбалансированы, потому что $N_c$ различается для разных кластеров. 

Микроэконометрические методы для панельных данных сосредотачивают внимание на коротких панелях. Это аналогично большому количеству кластеров с маленьким количеством наблюдений на кластер. Тогда $N_c$ маленький, а $C \to \infty$, мы называем их маленькими кластерами. Но нередко встречаются большие кластеры, где $N_c \to \infty$ и $C$ маленький. Для CSFE моделей с большими кластерами надо будет оценить небольшое количество $\alpha_c$ и проблемы со вспомогательными параметрами не возникнут. 

В отличие от панельных данных, может быть неочевидно, как правильно разделить наблюдения по кластерам. К примеру, для Текущего обследования населения CPS кластеризация может рассматриваться как возникающая внутри штатов, внутри страт, внутри PSU или внутри USU. Этот вопрос откладывается до раздела 24.6. Можно ожидать, что внутрикластерная корреляция будет убывать для кластеризации по более агрегированным уровням. Если кластеризация проводится на уровне штатов, кластеры будут большими, тогда как если она проводится на уровне USU они будут маленькими. Более того, возможно, что данные не включают необходимую для кластеризации информацию, такую как страта или USU для наблюдения. 

Аналог динамических, а не статичных панельных моделей --- это модель, где $y_{jc}$ зависит не только от $x_{jc}$, но и от $x_{kc}$, где $k \ne j$. Для кластеризованных данных обычно достаточно специфицировать \bfseries модель со взаимными эффектами \mdseries, которая включает средний по кластеру $\overline{x}_c$, так как порядок наблюдений внутри кластера обычно не важен. 

\section*{Обзор}

Три основных оценки для кластеризации --- это МНК, ОМНК, и оценки within, представленные в разделах 24.5.2 --- 24.5.4. Свойства этих оценок, приведенные в таблице 24.2, могут меняться в зависимости от истинной модели. Прежде всего, если в истинной модели есть постоянные кластерные эффекты, МНК и оценки со случайными эффектами становятся несостоятельными, тогда как оценка within даёт состоятельные оценки, но только коэффициентов для тех регрессоров, которые изменяются внутри кластера. Даже если оценка состоятельна, обычные стандартные ошибки часто требуют коррекции, чтобы учесть кластеризацию и возможную гетероскедастичность, что будет подробнее рассмотрено ниже. 

\begin{table}[h]
\caption{\label{tab:pred} Свойства оценок для разных моделей кластеризации}
\begin{center}
\begin{tabular}{lllc}
\hline
\hline
Раздел & Оценка & Модель кластеризации &  Состоятельность \\
\hline
24.5.2 & МНК & Случайные эффекты & Да \\
  &   & Постоянные эффекты & Нет \\
24.5.3 & ОМНК со случайными эффектами & Случайные эффекты & Да \\
  &   & Постоянные эффекты & Нет \\
24.5.4 & within с постоянными эффектами & Случайные эффекты & Да \\
  &   & Постоянные эффекты & Да \\
\hline
\hline
\end{tabular}
\end{center}
\end{table}

\subsection{Оценки метода наименьших квадратов}

Рассмотрим МНК регрессию

\begin{equation}
\label{eq24.28}
y_{jc} = x'_{jc} \beta + u_{jc}.
\end{equation}

МНК даёт несостоятельные оценки из-за наличия пропущенных переменных, если истинная модель --- CSFE (то есть, $u_{jc} = \alpha_c + \e_{jc}$), где постоянный эффект $\alpha_c$ коррелирован с $x_{jc}$. В этом случае вместо МНК нужно использовать оценку CSFE, рассмотренную в разделе 24.5.4. 

Напротив, в случае CSRE моделей с $\alpha_c$ некоррелированным с $x_{jc}$ МНК остаётся состоятельным. В общем, МНК состоятелен при более широком круге моделей $u_{jc}$, чем CSRE, при условии что $u_{jc}$ некоррелирована с $x_{jc}$. Рассмотрим МНК-оценку для этого случая, обращая внимание на получение корректных стандартных ошибок, зная корреляцию $u_{jc}$ внутри кластера. 

\subsection*{Замечание}

Собирая в~(\ref{eq24.28})  наблюдения внутри кластера в вектор получаем

\begin{equation}
\label{eq24.29}
y_{c} = X_{c} \beta + u_{c},
\end{equation}
где $y_c$ и $u_c$ --- вектора размера $N_c \times 1$ и $X_c$ --- матрица $N_c \times K$. Собирая кластеры в векторы, получаем:

\begin{equation}
\label{eq24.30}
y = X \beta + u,
\end{equation}
где $y$ и $u$ --- вектора размера $N \times 1$ и $X$ --- матрица размера $N \times K$, $N = \sum_c N_c$. 

Три формы представления CSRE модели дают три эквивалентных способа получения МНК оценки модели~(\ref{eq24.28}):

\begin{align}
\label{eq24.31}
 \hat{\beta}_{OLS} &= (X'X)^{-1}X'y \\
&=\left( \sum_{c=1}^C X'_cX_c \right)^{-1}  \sum_{c=1}^C X'_c y_c  \nonumber \\
&=\left( \sum_{c=1}^C \sum_{j=1}^{N_c} x_{jc} x'_{jc} \right)^{-1} \sum_{c=1}^C \sum_{j=1}^{N_c} x_{jc} y_{jc}. \nonumber
\end{align}

Вторая из этих форм может быть очень полезной при предположении о независимости ошибок в разных кластерах. Тогда, как до этого в панельном случае, оценка МНК сходится к

\begin{align}
\label{eq24.32}
\sqrt{N} (\hat{\beta}_{OLS} --- \beta) \overset{d}{\to}  \mathcal N [0, A^{-1} B A^{-1}], 
\end{align}
где
\begin{align}
\label{eq24.33}
& A = \plim N^{-1} \sum_{c=1}^C X'_c X_c, \\
& B = \plim N^{-1} \sum_{c=1}^C X'_c u_c u'_c X_c. \nonumber
\end{align}
Используя независимость $u_c$ по $c$. Разные предположения о свойствах $u_c$ ведут к разным оценкам $B$. 

\subsection*{МНК с робастными к кластерам стандартными ошибками}

При небольшом размере кластеров их число становится большим и можно получить состоятельную оценку $B$ из~(\ref{eq24.33}) заменой $u_c$ на $\hat u_c = y_c --- X_c \hat \beta$. Тогда $\hat \beta_{OLS}$ имеет асимптотическое нормальное распределение с робастной к кластерам ковариационной матрицей

\begin{equation}
\label{eq24.34}
\hat \V [\hat \beta_{OLS}] = \left( \sum_{c=1}^C X'_c X_c \right)^{-1} \sum_{c=1}^C X'_c \hat u_c \hat u'_c X_c \left( \sum_{c=1}^C X'_c X_c \right)^{-1}. 
\end{equation}

Эта формула не налагает ограничений на гетероскедастичность и корреляцию внутри кластеров, потому что на $\V[u_c]$ и, следовательно, на $\V[u_{jc}]$ и $\Cov[u_{jc}, u_{kc}]$ не налагается никаких ограничений. Однако, здесь предполагается, что $N_c$ маленькое и $C \to \infty$. Статистические пакеты часто дают коррекцию на степени свободы. Как правило, оценка~(\ref{eq24.34}) умножается на

$$
dfc = \frac{N-1}{N-K} \times \frac{C}{C-1},
$$
для коррекции оценки $\beta$, потому что на практике число кластеров конечно. 

Что посмотреть, как работает~(\ref{eq24.34}) будем считать регрессоры фиксированными и заметим, что

\begin{align}
B & = \lim N^{-1} \sum_{c=1}^C X'_c \E[u_c u'_c] X_c \nonumber\\
& = \lim N^{-1} \sum_{c=1}^C \sum_{j=1}^{N_c} \sum_{k=1}^{N_c} \E[u_{jc}u'_{kc}] x_{jc} x'_{kc}. \nonumber
\end{align}
Тогда~(\ref{eq24.34}) получается оценкой

\begin{align}
\hat{B} & =  N^{-1} \sum_{c=1}^C X'_c \hat u_c \hat u'_c X_c, \nonumber\\
& =  N^{-1} \sum_{c=1}^C \sum_{j=1}^{N_c} \sum_{k=1}^{N_c} \hat u_{jc} \hat u'_{kc} x_{jc} x'_{kc}. \nonumber
\end{align}

К примеру, рассмотрим оценивание $\E[y]$ с помощью $\overline{y}$. Это регрессия~(\ref{eq24.28}) с $x_{jc}=1$, $\hat \beta_{OLS} = \overline{y}$ и $\hat u_{jc} = y_{jc} --- \overline{y}$. Тогда~(\ref{eq24.34}) приводит к $\hat \V[\overline{y}] = N^{-2} \sum_c (\sum_j (y_{jc} --- \overline{y}))^2$ , вместо оценки $N^{-1} \sum_c \sum_j (y_{jc} --- \overline{y})^2$, которая дополнительно предполагает независимость внутри кластеров. 

\subsection*{Стандартные ошибки в предположении CSRE модели}

Робастные к кластерам оценки~(\ref{eq24.34}) требуют большого числа кластеров. Альтернативные оценки, которые также работают в случае небольшого числа кластеров, могут быть использованы при определенных предпосылках относительно дисперсий и ковариаций ошибки $u_{jc}$. Эти альтернативные оценки также позволяют оценить влияние кластеров на дисперсии оценок. 

В частности, предположимм модель CSRE, заданную~(\ref{eq24.24}) и~(\ref{eq24.26}). Тогда ошибки $u_{jc} = \alpha_c + \e_{jc}$ независимы по $c$, а внутри кластера

$$
\Cov[u_{jc}, u_{kc}]=\begin{cases}
\sigma^2,&\text{$j = k$,}\\
\rho\sigma^2,&\text{$j \ne k$,}
\end{cases}
$$
где коэффициент внутриклассовой корреляции $\rho$ определён в~(\ref{eq24.27}). Тогда

\begin{equation}
\label{eq24.35}
\sum_c = \V[u_c] = \sigma^2 [(1-\rho) I_c + \rho e_c e'_c],
\end{equation}
где $I_c$ --- единичная матрица размера $N_c \times N_c$, $e_c$ --- вектор из единиц размера $N_c \times 1$. 

Зная $\sum_c$  из~(\ref{eq24.35}) и~(\ref{eq24.32}) и~(\ref{eq24.33}), получаем

\begin{equation}
\label{eq24.36}
\V[\hat \beta_{OLS}] = \left( \sum_{c=1}^C X'_c X_c \right)^{-1} \sum_{c=1}^C \sigma^2 X'_c [(1-\rho) I_c + \rho e_c e'_c] X_c \left( \sum_{c=1}^C X'_c X_c \right)^{-1}.
\end{equation}
При постоянной внутриклассовой корреляции, оценка корреляционной матрицы состоятельна как при маленьких, так и при больших кластерах. Очевидные оценки для $\sigma ^2$ и $\rho$:

$$
\hat \sigma^2 = \frac{1}{N-K-1} \sum_{c=1}^C \sum_{j=1}^{N_c} \hat u^2_{jc}
$$
и
$$
\hat \rho = \frac{1}{\sum_c N_c (N_c -1)} \frac{1}{\hat \sigma^2} \sum_{c=1}^C \sum_{j=1}^{N_c} \sum_{k \ne j}^{N_c} \hat u_{jc} \hat u_{kc}.
$$
Оценка $\rho$ содержит много внутрикластерных пар, и состоятельная оценка может быть получена с использованием только их части. Используются пары $\sum_c N_c(N_c --- 1)$, тогда как фактически каждая уникальная внутри-кластерная пара учитывается дважды, так как и $\hat u_{jc} \hat u_{kc}$, и $\hat u_{kc} \hat u_{jc}$ фигурируют в суммах. 

Если кластеры большие, внутрикластерная корреляции может быть разной в разных кластерах. Тогда в~(\ref{eq24.35}) и~(\ref{eq24.36}) $\sigma^2$ и $\rho$ могут быть заменены на $\sigma^2_c$ и $\rho_c$ соответственно. Можно получить их состоятельные оценки:

$$
\hat \sigma^2_c = \frac{1}{N_c-K-1}\sum_{j=1}^{N_c} \hat u^2_{jc}
$$
и
$$
\hat \rho_c = \frac{1}{N_c (N_c -1)} \frac{1}{\hat \sigma^2_c} \sum_{j=1}^{N_c} \sum_{k \ne j}^{N_c} \hat u_{jc} \hat u_{kc}.
$$

\subsection*{Смещение стандартных ошибок, полученных МНК}

Следуя интуиции, можно прийти к выводу, что для кластеризованных данных, стандартная формула для оценки дисперсии метода наименьших квадратов

$$
\V^{Formula}[\hat \beta_{OLS}] = \sigma^2 \left( \sum_{c=1}^C X'_c X_c \right)^{-1},
$$
недооценивает истинную дисперсию оценки МНК в предположении положительной внутрикластерной корреляции, так как каждое дополнительное наблюдение в кластере приносит меньше, чем одну часть независимой информации. Продемонстрируем это смещения для ситуации, когда ошибки описываются CSRE моделью. 

Рассмотрим CSRE модель с одинаковыми регрессорами внутри каждого кластера, так что $x_{jc} = x_c$ и $X_c = e_c x'_c$. Тогда используя $e'_c e_c = N_c$, (\ref{eq24.36}) превращается в

$$
\V[\hat \beta_{OLS}] = \left( \sum_{c=1}^C N_c x_c x'_c \right)^{-1} \sum_{c=1}^C N_c \sigma^2 [1+ \rho(N_c -1)] x_c x'_c \left( \sum_{c=1}^C N_c x'_c x_c \right)^{-1},
$$
результат, представленный Клоэком (1981) и Моултоном (1986).

Теперь перейдем к сбалансированным кластерам. Пусть $M$ --- средний размер кластера, то есть $M = N_c = N/C$. Тогда выражение для дисперсии оценки упрощается до
$$
\V[\hat{\beta}_{OLS}] = [1+\rho (M-1)] \times \sigma^2 \left( M \sum_{c=1}^C x_c x_c'  \right)^{-1},
$$
тогда как формула дисперсии упрощается до $\sigma^2 (M \sum_c x_c x'_c)^{-1}$. Истинные дисперсии --- это произведение множителя

$$
\tau = [1+\rho (M-1)]
$$
на обычную МНК оценку ковариационной матрицы. Даже если $\rho$ маленькие, корректирующий фактор может быть довольно большим. К примеру, при среднем размере кластера $M = 101$ наблюдение, обычные стандартные ошибки, полученные МНК, должны умножаться на $\sqrt{1+100\rho}$. Предполагаемая независимость внутри кластера также может привести к смещению оценки $\sigma^2$, но это не так важно. Для случая сбалансированных кластеров Клоэк показал, что $\E[\sum_c \sum_j \hat u^2_{jc}] = \sigma^2 [N --- K(1+\rho(m-1))]$, так что нам надо корректировать на $[N --- K(1+\rho(m-1))]^{-1}$, а не на $[N-K]^{-1}$. 

На практике некоторые регрессоры могут быть постоянными внутри кластера, а другие могут варьировать. Тогда для случая регрессии со свободным членом и скалярным регрессором (к примеру, $x'_{jc}\beta = \beta_1 + \beta_2 x_{jc}$), Скотт и Холт (1982) показали, что обычная формула для оценки дисперсии свободного члена методом наименьших квадратов должна быть умножена на $1+\rho(M-1)$, как это делалось до этого, но для коэффициента наклона она должна умножаться на меньший коэффициент $1 + \hat \rho_x \rho (M-1)$, где $\hat \rho_x$ может рассматриваться как оценка внутриклассового коэффициента корреляции $x_{jc}$. Для пространственных данных $\hat \rho_x$ сравнительно небольшой, так что основная проблема заключается в стандартных ошибках для инвариантных к кластеру регрессоров. 

Моултон (1986) продемонстрировал в приложении, что смещение в стандартных ошибках при использовании некорректной формулы для МНК оценок дисперсии может быть довольно значительным. Он оценивал модель для логарифма заработной платы, используя пространственные данные Текущего обследования населения CPS с кластеризацией по штатам. В его приложении $N = 18 946$ и $C = 49$. Оцененный коэффициент внутриклассовой корреляции оказался небольшим, $\hat \rho =0.032$. Однако, размер кластеров большой, и если проигнорировать тот факт, что данные несбалансированы и использовать формулы со средним размеров кластера $M = 387$, то $\hat \tau = [1+ \hat \rho (M-1)] = 13.3$. Для постоянных по штату регрессоров истинные стандартные ошибки метода наименьших квадратов должны быть в $\sqrt{13.3} = 3.7$ раза больше полученных обычным путём стандартных ошибок, что является очень большим смещением. Один из способов трактовки этого смещения заключается в том, что для постоянных по штату регрессоров $18 946$ кластеризованных наблюдений дают ту же точность, что и $18946/13.3 = 1425$ независимых наблюдений. Для варьирующих регрессоров смещение будет гораздо меньше, к примеру $[1+ \hat \rho_x \hat \rho(M-1) ] = 2.23$ если $\hat \rho_x = 0.1$. Моултон не приводит результаты для включенных переменных изменяющихся по индивидам. Для постоянных по штату регрессоров, таких как темп роста занятости в штате, скорректированные стандартные ошибки для МНК как правило в 3 --- 4 раза больше некорректных, полученных по стандартной формуле. 

Эти рассуждения сводятся к тому, что присутствует значительное смещение вниз оценок стандартных ошибок для метода наименьших квадратов в случае постоянных по кластеру регрессоров. Для варьирующих по кластеру переменных он также присутствует, но значительно меньше. Постоянные по кластеру регрессоры часто используются при работе с кластеризованными данными, так как обычной практикой является моделировать индивидуальное поведение как зависящее, помимо прочего, от свойств кластера. В этом случае, для того, чтобы сделать статистически корректные выводы, нужно учесть влияние кластеризации. 

\subsection{Индивидуальные для кластеров случайные эффекты}

Если данные описываются моделью со случайными эффектами, то оценка ОМНК в целом более эффективна, чем оценка МНК из предыдущего раздела. При независимости между кластерами оценка ОМНК модели~(\ref{eq24.29}):

\begin{equation}
\label{eq24.37}
\hat \beta_{GLS,RE} = \left( \sum_{c=1}^C X'_c \Sigma_c^{-1} X_c \right)^{-1} \sum_{c=1}^C  X'_c \Sigma_c^{-1} y_c,
\end{equation}
где $ \Sigma_c = \V[u_c]$. Оценка допустимого ОМНК заменяет $ \Sigma_c$ состоятельной оценкой $ \hat \Sigma_c$, тогда, предполагая, что модель~(\ref{eq24.29}) и ковариационная матрица ошибок $ \Sigma_c$ специфицированы правильно, мы получаем:

$$
\V[\hat \beta_{GLS,RE}] = \left( \sum_{c=1}^C X'_c \Sigma_c^{-1} X_c \right)^{-1}.
$$

Для CSRE модели, $ \Sigma_c$ из~(\ref{eq24.35}) может быть состоятельно оценена при помощи $ \hat \Sigma_c$, которая заменяет $\sigma^2$ и $\rho$ из состоятельными оценками из~(\ref{eq24.36}). Как и в похожей модели со случайными эффектами для панельных данных, оценка допустимого ОМНК асимптотически эквивалентна оценке ММП при дополнительных предположениях, что $\alpha_c$ и $\e_{jc}$ нормально распределены. 

CSRE модель привлекательна тем, что оценка ОМНК~(\ref{eq24.37}) может быть просто получена как оценка методом наименьших квадратов трансформированной регрессии

\begin{equation}
\label{eq24.38}
y_{jc} --- \theta_c \overline{y}_c = (x_{jc} -\theta_c \overline{x}_c)' \beta + (\e_{jc} -\theta_c \overline{\e}_c),
\end{equation}
где
\begin{equation}
\label{eq24.39}
\theta_c = 1 --- \frac{\sqrt{1-\rho}}{\sqrt{1+\rho(N_c-1)}} = 1 --- \frac{\sigma_{\e}}{\sqrt{\sigma_{\e}^2 + N_c \sigma_{\alpha}^2}},
\end{equation}
Данный результат доказывается ниже в этом разделе. Для его реализации мы должны заменить $\theta_c$ состоятельной оценкой $\hat \theta_c$. Как и для моделей панельных данных, можно показать, что что обычные стандартные ошибки, полученные МНК в этой регрессии, могут быть использованы, если ошибки $\e_{jc}$ в модели~(\ref{eq24.24}) гомоскедастичны. 

Оценка ОМНК по меньшей мере также эффективна, как оценка МНК, предполагая, что выполнены~(\ref{eq24.24}) и~(\ref{eq24.26}). В частном случае, когда все регрессоры в модели постоянны внутри кластеров, нет выигрыша по эффективности, потому что оценки ОМНК совпадают с МНК (Клоэк, 1981). Скотт и Холт (1982) дают консервативную верхнюю границу для потери эффективности МНК по сравнению с ОМНК:

$$
\frac{\V[c' \hat \beta_{GLS}]}{\V[c' \hat \beta_{OLS}]} \ge 1 --- \left( 1+ \frac{4(1-\rho)[1+\rho(N_0 -1)]}{N_0^2 \rho^2} \right)^{-1}
$$
для произвольного вектора $c$, где $N_0 = \max\{ N_c\}$ --- размер самого большого кластера. Эта граница возрастает по $N_0$ и $\rho$, и даже для $N_0 = 1000$ и $\rho = 0.1$, МНК менее эффективен, чем ОМНК, не больше чем на $22\%$. 

Учитывая то, что выигрыш по эффективности от ОМНК невелик, как правило исследователи фокусируют внимание на МНК с правильными стандартными ошибками, кроме случаев, когда МНК несостоятелен из-за того, что данные описываются CSFE моделью. Основное влияние кластеризации заключается в том, что МНК становится значительно менее эффективным по сравнению со случаем отсутствия кластеризации, как становится ясным из дискуссии о вычислении стандартных ошибок для МНК оценок в разделе 24.5.2. 

Если кластеры велики, CSRE модель может быть ослаблена допущением, что дисперсия и внутриклассовая корреляция могут меняться в зависимости от кластера. Тогда в~(\ref{eq24.35}) для вычисления $\Sigma_c$ мы заменяем $\sigma^2$ и $\rho$ на $\sigma^2_c$ и $\rho_c$ соответственно, используя состоятельные оценки для $\sigma^2_c$ и $\rho_c$, данные после~(\ref{eq24.36}). 

Если кластеры маленькие, можно получить робастные стандартные ошибки, которые не требуют постоянства корреляции ошибок внутри кластеров, аналогично~(\ref{eq24.34}) для МНК. Тогда:

$$
\hat \V [\hat \beta_{GLS,RE}] = \left[ \sum_{c=1}^C X'_c \hat \Sigma_c^{-1} X_c \right]^{-1} \sum_{c=1}^C  X'_c \hat \Sigma_c^{-1/2} \hat u_c \hat u'_c \hat \Sigma_c^{-1/2} X_c \left[ \sum_{c=1}^C X'_c \hat \Sigma_c^{-1} X_c \right]^{-1},
$$
где $\hat u_c = y_c --- X_c \hat \beta_{GLS,RE}$. Эта оценка требует малого $N_c$ и $C \to \infty$ и предполагает независимость оценок в разных кластерах. 

\subsection*{ОМНК, полученный как МНК по трансформированной модели}

Чтобы получить~(\ref{eq24.38}), заметим, что для $\Sigma_c$, определённой в~(\ref{eq24.35}) 

\begin{align}
\Sigma_c^{-1} & = \left[ \sigma^2 [(1-\rho)I_c + \rho e_c e'_c] \right]^{-1} \nonumber\\
& =  \frac{1}{\sigma^2 (1-\rho)} [I_c --- (\rho / \tau_c) e_c e'_c]^{-1}, \nonumber
\end{align}
где $\tau_c = 1 + =\rho (N_c --- 1)$, и, следовательно:

$$
\Sigma_c^{-1/2}=  \frac{1}{\sigma \sqrt{1-\rho}} [I_c --- (\theta_c / N_c) e_c e'_c], 
$$
используя тот факт, что $e$ --- это вектор из единиц размера $M \times 1$, получим

\begin{align}
& [I+aee']^{-1} = I --- [a/(1+aM)] ee', \nonumber\\
& [I+aee']^{-1/2} = I --- M^{-1}[ 1 --- \sqrt{1+aM}]M ee'. \nonumber
\end{align}

Теперь в~(\ref{eq24.37}) $X'_c \Sigma_c^{-1} X_c = \left( \Sigma_c^{-1/2} X_c \right)' \Sigma_c^{-1/2} X_c$, где

\begin{align}
\Sigma_c^{-1/2} X_c & = [I_c --- (\theta_c / N_c) e_c e'_c]X_c \nonumber\\
& = X_c --- \theta_c e_c \overline{x}'_c, \nonumber
\end{align}
где $\overline{x}'_c = N_c^{-1} \sum_j x_{jc}$ и множитель $1 / \sigma \sqrt{ 1 --- \rho}$ игнорируется, потому что он сократится, когда мы будем рассматривать $X'_c \Sigma_c^{-1} y_c$. Отсюда получаем трансформированную модель~(\ref{eq24.38}). 

\subsection{Индивидуальные для кластеров постоянные эффекты}

Основная идея CSFE модели проста: пусть кластерные эффекты входят в условное среднее через постоянный член. Модель выглядит следующим образом:

\begin{equation}
\label{eq24.40}
y_{jc} = \alpha_c + x'_{jc} \beta + \e_{jc}, \qquad j = 1, \dots , N_c, \: c = 1, \dots , C,
\end{equation}
где оценке подлежат $\beta$ и $\alpha_c, \: c = 1, \dots, C$. 

Для оценки CSFE модели необходимо удалить все постоянные по кластеру регрессоры, потому что они не могут быть оценены отдельно от $\alpha_c$. К примеру, если кластеризация проводилась по штату и данные описываются моделью с постоянными эффектами, то влияние постоянных по штату регрессоров, таких как средний уровень безработицы в штате, не может быть идентифицировано. Если требуется всё-таки оценить влияние этих показателей, необходимо использовать МНК или CSRE-оценки. Однако, предварительно нужно использовать тест Хаусмана, аналогичный представленному в главе 21 дял панельных данных, чтобы подтвердить сильное допущение CSRE модели, что $\alpha_c$ некоррелирован с регрессорами. 

Мы будем рассматривать модели в предположении

$$
\e_{jc} \sim [0, \sigma^2_{jc}].
$$
Это разрешает гетероскедастичность неизвестной формы, однако предполагает, что включение кластерного постоянного эффекта $\alpha_c$ достаточно, чтобы учесть любую корреляцию ошибок внутри кластера. Это отличается от ситуации с панельными данными, когда корреляция ошибок во времени даже после включения индивидуальных эффектов допускалась и приводила к более сложным моделям. Однако, если есть желание, можно дополнительно скорректировать оценку стандартных ошибок на корреляцию внутри кластеров методами, схожими с использованными в разделе 24.5.2. 

Главное затруднение при оценке CSFE моделей заключается в необходимости оценивать слишком много $\alpha_c$ в случае большого числа маленьких кластеров. 

\subsection*{Модель с кластерными дамми-переменными}

Для начала рассмотрим случай больших кластеров, когда количество кластеров мало по сравнению с общим размеров выборки. Тогда константы $\alpha_c$ могут быть оценены простым МНК с введением дамми-переменных на каждый кластер. 

Пусть наблюдение $i$ обозначает $j$-ое наблюдение в $c$-ом кластере. Тогда~(\ref{eq24.40}) может быть переписано как \bfseries модель с кластерными дамми-переменными \mdseries

\begin{equation}
\label{eq24.41}
y_i = \sum_{c=1}^C \alpha_c d_{ci} + x'_i \beta + \e_i, \qquad i= 1, \dots , N,
\end{equation}
где $d_{ci}$ --- переменные-индикаторы, равные 1, если $i$-ое наблюдение принадлежит кластеру $c$ и 0 иначе. Таким образом вводятся $C$ кластерных дамми-переменных, таких как дамми-переменные на штат, и, чтобы избежать ловушки дамми-переменных, $x$ не должен содержать постоянного члена. 

МНК оценка этой модели даёт состоятельные оценки как $\alpha_1, \dots, \alpha_C$, так и $\beta$ в предположении, что число кластеров фиксировано и равно $C$ при $N \to \infty$. Можно использовать обычную оценку Эйкера-Уайта для получения робастных в гетероскедастичности стандартных ошибок. 

\subsection*{Оценка within для кластеризованных данных}

В присутствии большого числа маленьких кластеров модель~(\ref{eq24.40}) больше не может оцениваться МНК. Во-первых, могут проблемы с вычислением, потому что число параметров $C+K \to \infty$ в силу того, что $C \to \infty$. Во-вторых, что более важно, в силу того, что число параметров стремится к бесконечности, оценка МНК будет несостоятельна, кроме случаев, когда $N_c \to \infty$. 

Интерес, как правило, лежит в оценивании параметров $\beta$ в~(\ref{eq24.40}), а $\alpha_1 , \dots, \alpha_c$ рассматриваются как несущественные параметры. В таком случае, может оказаться удобным трансформировать данные для удаления фиксированных эффектов. Каждое наблюдение $(y_{jc}, x_{jc})$ заменяется отклонением от кластерной средней $(y_{jc} --- \overline{y}_c, x_{jc} --- \overline{x}_c), \: i = 1, \dots, N_c, \: c = 1, \dots, C$, где $\overline{y}_c = N_c^{-1} \sum_j y_{jc}$ и $\overline{x}_c = N_c^{-1} \sum_j x_{jc}$ --- средние по кластеру. Тогда модель~(\ref{eq24.40}) для $y_{jc}$ предполагает, что

\begin{equation}
\label{eq24.42}
y_{jc} --- \overline{y}_c = (x_{jc} --- \overline{x}_c)' \beta + \e_{jc} --- \overline{\e}_c.
\end{equation}

Применение МНК к трансформированной регрессии~(\ref{eq24.42}) даёт состоятельную оценку $\beta$. Если коэффициенты CSFE также представляют интерес, они могут быть оценены как $\hat \alpha = \overline{y}_c --- \overline{x}'_c \beta$, но эта оценка несостоятельна при маленьких $N_c$. 

Сравнение с главой 21 показывает, что это аналог \bfseries оценки within \mdseries для панельных данных. Как и для панельных данных, оценка $\beta$ из~(\ref{eq24.42}) совпадает с МНК оценкой $\beta$ из модели с кластерными дамми-переменными~(\ref{eq24.41})

Можно также рассмотреть \bfseries оценку between \mdseries, аналогичную оценке для линейных панельных моделей. В этом случае $\overline{y}_c$ регрессируется на $\overline{x}_c, \: c = 1, \dots , N_c$. Из~(\ref{eq24.37}), оценка ОМНК CSRE модели включает регрессию в квази-разностях, где кластерные средние умножаются на $\theta_c$ (определенные в~(\ref{eq24.39})) перед взятием разности. Можно показать, что оценка ОМНК --- это линейная комбинация оценок within и between. Она приближается к оценке within для больших $N_c$, потому что в этом случае $\theta_c \to 1$. Отметим, что оценка within состоятельна при CSRE модели. 

Необходимо проявлять осторожность, трактуя стандартные ошибки, если регрессия оценивалась по скорректированным на среднее наблюдениям. Число степеней свободы для такой регрессии равно $(N-K-C)$, а не $(N-K)$. Если программы пренебрегают этой корректировкой, дисперсию, полученную в результате оценивания, необходимо скорректировать умножением её на $(N-K)/(N-K-C)$, а стандартные ошибки должны быть увеличены на корень из этого числа. 

\subsection{Тесты на наличие кластерных эффектов}

В линейной регрессии тест на наличие кластерных постоянных эффектов при нормальных ошибках --- это просто стандартный $F$-тест с линейными ограничениями $H_0 = \alpha_1 = \alpha_2 = \dots =\alpha_c = 0$ в модели~(\ref{eq24.40}). Этот тест сводится к простому сравнению $R^2$ для двух регрессий --- с кластерными дамми-переменными и без них. 

Для CSRE модели тест на кластерные эффекты --- односторонний тест с нулевой гипотезой $H_0: \sigma^2_{\alpha} = 0$ против $H_1 : \sigma^2_{\alpha} > 0$. Можно также сформулировать эквивалентный тест как $H_0: \rho = 0$ против $H_1 : \rho > 0$, используя определение из~(\ref{eq24.27}). Односторонний LM тест этой гипотезы, рассмотренный Моултоном (1987):

\begin{equation}
\label{eq24.43}
LM = \frac{\sum_c (N_c \overline{u}_c)^2 --- \sum_c \sum_i \hat u_{ic}^2}{\hat \sigma^2 [2(\sum_{\hat c} N_c^2 --- N)]^{1/2}},
\end{equation}
где $\hat \sigma^2 = \sum_c \sum_i \hat u^2_{ic} / N$, $\hat u_{ic}$ обозначает остатки из МНК регрессии $y$ на $x$, и $\overline{u}_c$ --- средний остаток для кластера $c$. 

\subsection{Кластеризация в нелинейных моделях}

Нелинейные модели с кластеризованными данными не получили большого внимания в эконометрической литературе. Однако, есть огромное количество работ по биостатистике, фокусирующих внимание на моделях с бинарным исходом (Пендергаст и др. 1996). Также рассматривались другие модели, такие как пуассоновская регрессия и некоторые модели длительности жизни. Иерархические (многоуровневые) модели также активно использовались, особенно для моделей с бинарным исходом. 

Здесь мы продолжим рассматривать сходства между кластеризованными и панельными данными. Как и в линейном случае, данные $(y_i, x_i)$, $i = 1, \dots, N$, обозначаются как $(y_{jc}, x_{jc})$, $j = 1, \dots, N_c$, $c = 1, \dots, C$. Мы предполагаем независимость по $c$, однако допускаем зависимость наблюдений внутри кластера $c$. 

\subsection*{М-оценки с кластеризацией}

Рассмотрим оценку, являющуюся решением нелинейных уравнений

\begin{equation}
\label{eq24.44}
\sum_{c=1}^{C} \sum_{j=1}^{N_c} h (y_{jc}, x_{jc}, \theta) = 0.
\end{equation}
Часто эти уравнения получаются из задачи максимизации или минимизации целевой функции $\sum_c \sum_j q(y_{jc}, x_{jc}, \theta)$; в таком случае $  h (y_{jc}, x_{jc}, \theta) = \partial q(y_{jc}, x_{jc}, \theta) / \partial \theta$. К примеру, для метода квази-максимального правдоподобия, основанного на произведении частных плотностей $ h (y_{jc}, x_{jc}, \theta) = \partial \ln{ f (y_{jc}| x_{jc}, \theta)} / \partial \theta$. 

Мы предполагаем, что данные кластеризованы, поэтому $\Cov[h_{jc}, h_{kc}] \ne 0$.Однако, мы предполагаем $\E[ h (y_{jc}, x_{jc}, \theta)] = 0$, необходимое условие для состоятельности, что исключает модель с кластерными постоянными эффектами, также представленную ниже. 

Робастная к кластерам дисперсия МНК оценки~(\ref{eq24.34}) легко адаптируется к текущей ситуации заменой $x_{jc} x'_{jc}$ на $\partial h_{jc} / \partial \theta'$ и $x_{jc} \hat u_{jc}$ на $h_{jc} (\hat \theta)$. Тогда $\hat \theta$ асимптотически нормальна с робастной к кластерам ковариационной матрицей

\begin{equation}
\label{eq24.45}
\hat \V[\hat \theta] = \left( \sum_{c=1}^C \sum_{j=1}^{N_c} \left. \frac{\partial h'_{jc}}{\partial \theta} \right| _{\hat \theta} \right)^{-1} \sum_{c=1}^C \sum_{j=1}^{N_c} \sum_{k=1}^{N_c} h_{jc}(\hat \theta) h_{kc} (\hat \theta)' \left( \sum_{c=1}^C \sum_{j=1}^{N_c} \left. \frac{\partial h_{jc}}{\partial \theta'} \right| _{\hat \theta} \right)^{-1}. 
\end{equation}
Некоторые компьютерные программы используют эту оценку как стандартную опцию для многих параметрических нелинейных моделей. 

Популярный пример --- это оценка квази-ММП, основанная на произведении частных плотностей внутри кластера вместо совместной плотности. При наличии зависимости по $j$ внутри кластера $c$ мы должны максимизировать логарифм функции правдоподобия

$$
\ln{L(\theta)} = \sum_{c=1}^{C} \ln{f (y_{1c}, \dots, y_{N_c c}, x_{1c}, \dots, x_{N_c c}, \theta)}.
$$ 
Однако с совместной плотностью достаточно тяжело работать и могут быть проблемы с её получением, потому что  многим одномерным плотностям  соответствует узкий класс  многомерных плотностей. Вместо этого мы можем максимизировать

\begin{align}
Q(\theta) & = \sum_{c=1}^C \ln [f(y_{1c}, x_{1c}, \theta) \times \dots \times f(y_{N_c}, x_{N_c}, \theta)] \nonumber\\
& = \sum_{c=1}^C \sum_{j=1}^{N_c} \ln{f(y_{jc}, x_{jc}, \theta)}. \nonumber
\end{align}
Это больше не является истинной функцией правдоподобия, кроме случаев, когда $y_{jc}$ независимы по $j$, поэтому равенство информационных матриц больше не работает. Предшествовавшие формулы применяются с $h_{jc}(\theta) = \partial \ln{f(y_{jc}, x_{jc}, \theta)} / \partial \theta$ и $\partial h_{jc}(\theta) / \partial \theta = \partial^2 \ln{f(y_{jc}, x_{jc}, \theta)} / \partial \theta \partial \theta'$. 

Это означает, что внутри каждого кластера мы не используем правдоподобие по всем наблюдениям сразу, как в случае независимых наблюдений; вместо этого, мы заменяем его суммой правдоподобий по всем элементам кластера. 

\subsection*{Нелинейные кластерные случайные эффекты}

Достаточно общий подход к работе со кластерными эффектами в нелинейных моделях заключается в рассмотрении оценки, которая минимизирует или максимизирует

\begin{equation}
\label{eq24.46}
Q(\beta, \alpha_1, \dots, \alpha_C) = \sum_{c=1}^C \sum_{j=1}^{N_c} q(y_{jc}, x_{jc}, \beta, \alpha_c),
\end{equation}
где кластерные эффекты входят только через скаляр $\alpha_c, \: c = 1, \dots, C$. 

Простая модель со случайными эффектами предполагает, что $\alpha_c$ --- независимые, одинаково распределенные с параметрами $\delta$. Математическое ожидание по $\alpha_c$ даёт целевую функцию

$$
Q(\beta, \delta) = \sum_{c=1}^C \int \sum_{j=1}^{N_c} q(y_{jc}, x_{jc}, \beta, \alpha_c) f(\alpha_c|\delta) d \alpha_c.
$$
Оценивание может быть сложным, особенно если не существует выражения для интеграла суммы в явной форме. 

Часто оказывается несложно получить математическое ожидание по одному наблюдению $\E_{\alpha_c} [q(y_{jc},$ $x_{jc}, \beta, \alpha_c)] = q^* (y_{jc}, x_{jc}, \beta, \delta)$. Тогда можно применять более простую оценку, которая игнорирует кластеризацию и минимизирует $Q^*(\beta, \delta) = \sum_c \sum_j q^* (y_{jc}, x_{jc}, \beta, \delta)$. Она будет состоятельной, однако стандартные ошибки надо будет скорректировать на кластеризацию~(\ref{eq24.45}). 

К примеру, для счётных данных можно сделать кластерный аналог модели со смесью распределения Пуассона и гамма-распределения для панельных данных. Но пуассоновский квази-ММП, игнорирующий кластеризацию, может по-прежнему использоваться, потому что он состоятельный, однако стандартные ошибки требуют коррекции на кластеризацию. 

Следовательно, несмотря на возможность создания нелинейных моделей со случайными эффектами, зачастую можно оценить их параметры, игнорируя кластеризацию и корректируя стандартные ошибки оценок на кластеризацию. Так что единственная причина, по которой можно оценивать модели с кластерными случайными эффектами, заключается в возможном выигрыше в эффективности. 

\subsection*{Нелинейный кластерные постоянные эффекты}

Нелинейные версии моделей с кластерными постоянными эффектами также максимизируют или минимизируют 

$$
Q(\beta, \alpha_1, \dots, \alpha_C) = \sum_{c=1}^C \sum_{j=1}^{N_c} q(y_{jc}, x_{jc}, \beta, \alpha_c),
$$
как в~(\ref{eq24.43}), но теперь параметры $\alpha_1, \dots, \alpha_C$ оцениваются, а не устраняются интегрированием. 

Для больших кластеров, то есть, маленьких $C$ и $N_c \to \infty$, мы просто оптимизируем $Q(\beta, \alpha_1, \dots, \alpha_C)$ по $\beta$ и $\alpha_1, \dots, \alpha_C$. Предполагая, что $\alpha_1, \dots, \alpha_C$ полностью учитывают кластеризацию, можно проводить оценивание, базируясь на стандартных ошибках, полученных в обычном предположении независимости и одинаковой распределенности. Это нелинейный аналог модели с кластерными дамми-переменными~(\ref{eq24.41}).

Для маленьких кластеров, когда $N_c$ мал и $C \to \infty$, мы сталкиваемся с проблемой в виде слишком большого количества несущественных параметров $\alpha_1, \dots, \alpha_C$. В отличие от случая линейных моделей, для нелинейных моделей в общем случае невозможно элиминировать параметры $\alpha_1, \dots, \alpha_C$ (Холл, Северини, 1998). Однако, в соответствии с главой 23 для панельных данных, это возможно сделать в некоторых случаях. 

К примеру, \bfseries бинарная логит-модель с кластерными постоянными эффектами \mdseries специфицирует

\begin{equation}
\label{eq24.47}
\Pr[y_{jc} = 1] = \frac{1}{1+\exp(-\alpha_c -x'_{jc}\beta)},
\end{equation}
где $x_{jc}$ не может включать постоянный член или постоянные по кластеру наблюдения (иначе модель не будет идентифицироваться). Постоянные эффекты $\alpha_c$ могут быть устранены использованием \bfseries условного ММП \mdseries, с условием на сумму ответов внутри кластера, $\sum_{j=1}^{N_c} y_{jc} = N_c \overline{y}_c$. Совместная условная вероятность для $c$-ого кластера:

\begin{align}
\label{eq24.48}
\Pr[y_{1c}, \dots, y_{N_c c} | N_c \overline{y}_c] = & \frac{\exp(\beta \sum_{j=1}^{N_c} x_{jc} y_{jc})}{\sum_{d \in \widetilde{B_c}} \exp(\beta \sum_{j=1}^{N_c} x_{jc} d_{jc}) }  \nonumber\\
& \times \frac{\Gamma [\sum_{j=1}^{N_c} y_{jc} + 1] \Gamma [N_c --- \sum_{j=1}^{N_c} + 1] }{\Gamma (N_c+1)},
\end{align}
где $\widetilde{B_c} = \{ (d_{1c}, \dots, d_{N_c c}) | d_{nc} = 0 \text{ или } 1, \text{ и } \sum_j d_{jc} = \sum_j y_{jc} \}$. Условное правдоподобие --- это произведение по всем кластерам таких членов, как эти, с исключением из выборки кластеров размера 1. Второй член на правой стороне не зависит от неизвестных параметров и поэтому не может повлиять на максимизацию правдоподобия, поэтому его можно игнорировать при максимизации. Максимизировать такое правдоподобие будет затруднительно, потому что для множества $\widetilde{B_c}$ существуют разные способы выбора $N_c$ исходов, где $y_{jc} = 1$ из $(N_{1_c} + N_{0_c})$ исходов в кластере $c$. К счастью, некоторые современные компьютерные пакеты имеют опцию \bfseries условный логит \mdseries для оценки этой модели. Ковариационная матрица всех неизвестных параметров оценивается как обратная к гессиану логарифма функции правдоподобия. 

В качестве другого примера, рассмотрим \bfseries пуассоновскую модель с кластерными постоянными эффектами: \mdseries

$$
y_{jc} \sim \mathcal P [\mu_{jc} = \alpha_c \exp(x'_{jc} \beta)], \qquad c=1, \dots, C, 
$$
где $\mathcal P[\cdot]$ обозначает распределение Пуассона, а $x_{jc}$ исключает константу и постоянные по кластеру переменные. Это обычная пуассоновская модель, за исключением того, что обычное условное среднее $\exp(x'_{jc} \beta)$ умножается на кластерный постоянный эффект $\alpha_c$. Для этой конкретной модели для удаления параметров $\alpha_c$ можно применить целый ряд подходов, включая условный ММП и концентрированный ММП. Состоятельные оценки параметров $\beta$ могут быть получены из 

$$
\sum_{c=1}^C \sum_{j=1}^{N_c} x_{jc} \left( y_{jc} --- \frac{\overline{y}_c}{\overline{\lambda}_c} \lambda_{jc} \right) =0,
$$
где $\lambda_{jc} = \exp(x'_{jc}\beta)$ и $\overline{y}_c = N_c^{-1} \sum_j y_{jc}$ и $\overline{\lambda}_c = N_c^{-1} \sum_j \lambda_{jc}$ --- кластерные средние. Более подробно эта модель обсуждается в применении к панельным данным в разделе 23.7. 

\subsection{Другие методы работы с кластеризованными данными}

Важное свойство кластеризованных данных заключается в том, что существует зависимость между наблюдениями. Близкая тема --- это \bfseries пространственная корреляция \mdseries (см., к примеру, Анселин (2001), Ли (2004)), где единица наблюдения --- регион, к примеру, штат, и наблюдения в близкорасположенных регионах могут коррелировать. 

Случайные эффекты могут распространены и на коэффициенты наклона, в дополнение к константам. Этот подход представлен в следующем разделе \bfseries иерархические линейные модели\mdseries. Для нелинейных моделей есть сходство с панельными данными, рассмотренными в главе 23. 

Для получения робастных к кластеризации стандартных ошибок в ситуациях, где кластеризация ведёт к внутрикластерной корреляции, но не влияет на состоятельность оценок, может использоваться \bfseries бутстрапирование. \mdseries Интуитивно, нужно строить псевдовыборки с возвращением по кластерам $c$; в таком случае мы хотим маленькие кластеры с $C \to \infty$. На $b$-ой репликации мы извлекаем $C$ кластеров с возвращением и используем все домохозяйства $j$ в этих $C$ кластерах, попавших в псевдовыборку, для оценки $\hat \theta_b$, которая является решением~(\ref{eq24.44}). Можно оценить $\V[\hat \theta]$ применяя обычную формулу для дисперсии к $\hat \theta_1, \dots, \hat \theta_B$, где $B$ --- число итераций бутстрапа. Отметим, что псевдовыборки строятся по кластерам, а не по домохозяйствам, потому что мы считаем кластеры независимы и одинаково распределенными, тогда как внутри кластеров допускаем связь между наблюдениями. 

\section{Иерархические линейные модели}

В разделе 24.5 роль кластеров в модели со случайными эффектами сводилась к влиянию на постоянный член в регрессии. В общем случае, модели со случайными эффектами могут разрешать коэффициентам наклона изменяться в зависимости от кластера. Межкластерная изменчивость части параметров регрессии может быть связана с наблюдаемыми характеристиками кластеров. В силу того, что такие модели требуют нескольких уровней спецификации, они называются \bfseries иерархическими моделями.\mdseries

Стандартный способ работы с кластеризованными данными во многих прикладных статистических дисциплинах --- это использование \bfseries иерархических линейных моделей (hierarchical linear model, HLM) \mdseries, также называемых \bfseries многоуровневыми моделями, \mdseries моделей со случайными коэффициентами, моделей с дисперсией коэффициентов (variance components models), и смешанных линейных моделей или \bfseries моделей со смешанными эффектами. \mdseries Этот класс моделей добавляет дополнительную информацию в спецификацию модели. Мы начнём с рассмотрения модели для индивидов, кластеризованных в группы. Затем модель будет адаптирована к коротким панелям, где наблюдения в разные моменты времени кластеризованы для каждого индивида. 

\subsection{Структура модели}

Иерархическая, или многоуровневая модель --- это модель, которая может быть применена к данным с сгруппированной структурой. К примеру, это данные по индивидам внутри региона, такого как штат или страна, или внутри организации, такой как школа или сообщество, или внутри семьи, если используются данные по близнецам. Панельные данные также относятся к сгруппированным данным. В этом случае повторяемые наблюдения за одним и тем же индивидом сгруппированы внутри индивида. 

Начнём с линейной модели

\begin{equation}
\label{eq24.49}
y_{ij} = x'_{ij} \beta_j + u_{ij}, 
\end{equation}
где, в отличие от предыдущих случаев, $K$ параметров $\beta$ изменяются в зависимости от группы (кластера) $j$. В качестве примера можно привести данные об учениках школ. Тогда $y_{ij}$ --- это результирующая переменная, такая как результат теста $i$-ого ученика $j$-ой школы, и предельный эффект изменения регрессора, к примеру, расы, может меняться в зависимости от школы. Заметим, что в стандартной нотации, применяемой в иерархических линейных моделях, нижние индексы пишутся в другом порядке по сравнению с нотацией, использованной в разделе 24.5, где $y_{cj}$ соответствовал бы результату студента $j$ в школе $c$. 

\bfseries Двухуровневая линейная иерархическая модель \mdseries определяет коэффициенты первого уровня в модели~(\ref{eq24.49}) через линейную функцию от случайного члена и переменных второго уровня, здесь --- характеристик школы. Скалярный параметр $\beta_{kj}$ --- $k$-ый компонент $K \times 1$ вектора $\beta_j$. Тогда $\beta_{kj}$ зависит от вектора характеристик школы $w_k$, принимающего значение $w_{kj}$ для $j$-ой школы:

\begin{equation}
\label{eq24.50}
\beta_{kj} = w'_{kj} \gamma_k + u_{kj}, \qquad k = 1, \dots , K, 
\end{equation}
где первый компонент в $w_{kj}$ обычно константа. Векторизуя по $K$ компонентам $\beta$ получаем:

$$\begin{bmatrix}
\beta_{1j} \\ \vdots \\ \beta_{Kj}
\end{bmatrix} = \begin{bmatrix}
w'_{1j} & 0 & 0 \\ 0 & \ddots & 0 \\ 0 & 0 & w'_{Kj}
\end{bmatrix} \begin{bmatrix}
\gamma_{1} \\ \vdots \\ \gamma_{K}
\end{bmatrix} + \begin{bmatrix}
v_{1j} \\ \vdots \\ v_{Kj}
\end{bmatrix}
$$
или в очевидных матричных обозначениях:

\begin{equation}
\label{eq24.51}
\beta_{j} = W_{j} \gamma + v_j.  
\end{equation}

Модель~(\ref{eq24.50}) достаточно гибкая и включает в себя много других моделей как частные случаи. Сюда входят модели со случайными постоянными членами и случайными коэффициентами наклона, но сюда же входят модели с коэффициентами, зависящими от переменных второго уровня $w_j$. Круг моделей очень широк. 

Если $\beta_{kj} = \gamma_k$, то есть не зависит от регрессоров второго уровня или ненаблюдаемых переменных, то $k$-ый коэффициент первого уровня называется \bfseries постоянным коэффициентом. \mdseries  Если все коэффициенты первого уровня постоянны, модель~(\ref{eq24.49}) упрощается до $y_{ij} = x'_{ij} \gamma + u_{ij}$, в этом случае можно оценивать методом наименьших квадратов. Стоит отметить, что термин <<постоянный коэффициент>> серьезно отличается по смыслу от термина <<постоянный эффект>>, используемого в контексте панельных данных. 

Если $\beta_{kj} = w'_{kj} \gamma_k$, то $k$-ый коэффициент первого уровня называют \bfseries неслучайно варьирующим коэффициентом. \mdseries  Тогда коэффициент становится линейной функцией от характеристик школы. Если все коэффициенты первого уровня в модели постоянны, а константа неслучайно варьирует, модель~(\ref{eq24.49}) сокращается до $y_{ij} = x'_{ij} \beta + w'_{1j} \gamma_1 + u_{ij}$ --- стандартной МНК регрессии результирующей переменной на индивидуальные характеристики и характеристики школы. 

Если $\beta_{kj} = \gamma_k + v_{kj}$, то $k$-ый коэффициент первого уровня называют \bfseries случайно варьирующим коэффициентом. \mdseries Тогда коэффициент становится чисто случайным и не зависит от характеристик школы. Если все коэффициенты первого уровня случайно варьируют, $\beta_j = \gamma + v_j$, модель называют \bfseries моделью с дисперсией коэффициентов \mdseries или моделью со случайными коэффициентами. Если все коэффициенты первого уровня постоянны, кроме случайно варьирующей константы, модель~(\ref{eq24.49}) упрощается до $y_{ij} = x'_{ij} \beta + u_{1j} + u_{ij}$ --- модели со случайной константой. 

На практике некоторые коэффициенты первого уровня могут варьировать как случайно, так и неслучайно, как в общем случае, описанном в~(\ref{eq24.49}). Если только константа первого уровня описывается~(\ref{eq24.49}), а другие коэффициенты первого уровня фиксированы, модель~(\ref{eq24.49}) сокращается до $y_{ij} = x'_{ij} \beta + w'_{1j} \gamma_1 + \upsilon_{1j} + u_{ij}$. Это обычная регрессионная модель с ошибкой, состоящей из двух компонентов. Следовательно, есть корреляция ошибок для респондентов из одной школы. 

Иерархические линейные модели можно расширить путем введения дополнительных уровней. К примеру, отдельные студенты (индекс $i$) могут группироваться в школах (индекс $j$), которые сгруппированы в регионы (индекс $k$). Тогда трехуровневая иерархическая линейная модель специфицируется следующим образом: результат студента $y_{ijk} = z'_{ijk} \pi_{jk} + e_{ijk}$, где параметры $\pi_{jk} = X_{jk} \beta_{k} + u_{jk}$, и в свою очередь $\beta_k = W_j \gamma + w_k$. 

Иерархическую линейную модель можно представить как \bfseries смешанную линейную модель \mdseries, потому что подставляя~(\ref{eq24.50}) в~(\ref{eq24.49}) получаем

\begin{equation}
\label{eq24.52}
y_{ij} = (x'_{ij} W_j) \gamma + x'_{ij} v_j + u_{ij}.
\end{equation}
Цель --- оценить параметр регрессии $\gamma$ и ковариационные матрицы ошибок $u_{ij}$ и $v_j$. Предполагается, что ошибки не зависят от регрессоров, поэтому оценка~(\ref{eq24.52}) как сквозной регрессии методом наименьших квадратов дает состоятельные оценки $\gamma$. Иерархическая линейная модель дает более состоятельные оценки, делая предположения о ковариационных матрицах $u_{ij}$ и $v_j$. 

В простейшем случае $v_{jk}$ предполагаются независимыми и одинаково распределенными $\mathcal N [0, \sigma^2]$ и $v_j$ предполагаются независимыми и одинаково распределенными $\mathcal N [0, \Gamma]$. Тогда модель может быть представлена в следующей форме:

\begin{align}
& y_{ij} \sim \mathcal N [x'_{ij} \beta_j, \sigma^2], \nonumber\\
& y_{ij} \sim \mathcal N [W_j \gamma, \Gamma]. \nonumber
\end{align}
Одними из первых такие модели рассмотрели Линдлей и Смит (1972) в рамках байесовского подхода. Коэффициенты $\gamma$ они называли \bfseries гиперпараметрами \mdseries, при этом гиперпараметры, в свою очередь, могут зависеть от гиперпараметров более высокого уровня. Параметры $\gamma$, $\sigma^2$ и $\Gamma$ могут оценены методом максимального правдоподобия или при помощи байесовских методов. Либо можно использовать вариант ММП, аналогичный представленному для смешанных линейных моделей для панельных данных, представленный в разделе 21.7. Наиболее полно этот вопрос рассматривали Брайк и Рауденбуш (1992, 2002). 

\subsection{Иерархические линейные модели для панельных данных}

В иерархических линейных моделях короткие панели интерпретируются как повторяющиеся измерения для одного индивида. Тогда в двухуровневых иерархических моделях индивидуальный уровень становится вторым, тогда как до этого мы рассматривали его как первый. Модель~(\ref{eq24.28}) превращается в

\begin{equation}
\label{eq24.53}
y_{ti} = x'_{ti} \beta_i + u_{ti},
\end{equation}
где, к примеру, $y_{ti}$ обозначает результат студента $i$ в момент времени $t$, и предельный эффект изменения регрессоров, таких как набор предметов, изученных студентом, меняется в зависимости от студента. Скаляр $\beta_{ki}$, $k$-ый элемент вектора $\beta_i$ размерности $K \times 1$, зависит от вектора индивидуальных характеристик $w_k$, принимающих значение $w_{ki}$ для $i$-ого индивида:

\begin{equation}
\label{eq24.54}
\beta_{ki} = w'_{ki} \gamma_k + v_{ki}, \qquad i = 1, \dots, N.
\end{equation}

Модель с индивидуальными эффектами --- это частный случай преставленной модели, где все коэффициенты первого уровня фиксированы, то есть $\beta_{ki} = \gamma_k$, кроме константы $\beta_{1i}$, которая изменяется в зависимости от индивида (показателя второго уровня). 

Модель с индивидуальными постоянными эффектами возникает, если $\beta_{1i}$ не моделируется, а непосредственно оценивается для каждого индивида. Это крайний случай модели с неслучайно варьирующим коэффициентом, где $\beta_{1i} = w'_{1i} \gamma_1$, а $w_1i$ --- $N \times 1$ вектор переменных-индикаторов, где $l$-ый компонент равен 1, если $i=l$ и равен 0 иначе, так что $\beta_{1i} = \gamma_{1i}$. Иерархические линейные модели не слишком хорошо работают с постоянными эффектами. 

Индивидуальные случайные эффекты возникают, если константа $\beta_{1i}$ является случайно варьирующим коэффициентом, то есть $\beta_{1i} = \gamma_1 + v_{1i}$. Очевидно, можно специфицировать более общую модель со случайными эффектами, где $\beta_{ki}$ будет также зависеть от регрессоров $w_{ki}$. 

Как уже было замечено, иерархическая линейная модель --- это смешанная линейная модель. Для панельного случая аналог~(\ref{eq24.52})

$$
y_{ti} = (x'_{ti} W_i) \gamma + x'_{ti} v_j + u_{ti}. 
$$
Модель со случайными эффектами из главы 21 --- это спецификация $y_{ti} = x'_{ti} \gamma + v_j + u_{ti}$.

Стандартное приложение иерархических моделей --- это модели роста, где результирующая переменная $y_{ti}$ --- это индивидуальный интеллект или рост, который является функцией возраста, и предельный эффект возраста может изменяться в зависимости от индивида. Здесь не только константа, но и коэффициент наклона может изменяться в зависимости от индивида. 

\section{Пример кластеризации: расходы на медицинское обслуживание во Вьетнаме}

В этом разделе мы рассмотрим оценивание моделей при наличии кластеризации, потому что это наиболее часто встречающаяся при проведении микроэконометрических исследований проблема. Используются методы, представленные в разделе 24.5. 

Оцениваются линейные и нелинейные регрессионные модели, основанные на данных индивидуального уровня и уровня домохозяйств из опроса Всемирного Банка Обследование уровня жизни во Вьетнаме, Vietnam Living Standarts Survey (VLSS), проводившегося в 1997 --- 1998 гг. В опросе представлена детальная информация по широкому кругу вопросов о 27700 индивидах в приблизительно 6000 домохозяйств, сгруппированных в 194 коммуны. Тогда коммуна рассматривается как кластер, и предполагается, что переменные могут быть коррелированы внутри коммуны. Средний размер кластера в выборке домохозяйств --- 26, максимальный --- 39, минимальный --- 1. Для иллюстрации линейный и нелинейных моделей будут моделироваться три результирующих переменных. 

Во-первых, мы рассмотрим (лог)линейную модель общих годовых расходов домохозяйства на медицинское обслуживание (LNEXP12M) для домохозяйств с положительными расходами как функцию (логарифма) общих расходов домохозяйства (HHEXP), контролируя на несколько стандартных социодемографических переменных, получив кривую Энгеля для расходов на здравоохранение. Интерес представляет коэффициент при общих расходах домохозяйства, являющийся оценкой эластичности спроса домохозяйства на медицинское обслуживание по доходу. 

Во-вторых, мы используем информацию по индивидам для оценки кластеризованных моделей счетных данных того типа расходов на здравоохранение, на который приходится большая доля агрегированных частных расходов на медицинское обслуживание. При моделировании этих переменных мы контролируем на текущий статус здоровья индивида, доход домохозяйства, наличие медицинской страховки и ряд демографических характеристик, таких как возраст, пол, семейный статус и уровень образования главы домохозяйства. Информация о здоровье индивида ограничена наличием болезней (переменная ILLNESS) и травм (переменная INJURY) в период проведения опроса, продолжительностью болезни и количеством дней ограниченной дееспособности. Больше всего нас интересуют коэффициенты при доходе и индикаторе наличия страховки. 

В таблице 24.3 представлены определения и описательные статистики переменных, использованных в этих примерах. 

В обоих случаях основной вопрос заключается в следующем: каково влияние кластеризации на оценки эластичности и как оцененная эластичность изменяется в зависимости от используемых статистических предположений, моделей и оценок? 

\subsection{Обсуждение результатов}

В таблице 24.4 представлены результаты для случаев МНК регрессии, HC $t$-отношения, постоянных и случайных эффектов. Использование устойчивой к гетероскедастичности оценки ковариационной матрицы, не учитывающей кластеризацию, приводит к относительно слабым изменениям в оценке стандартных ошибок. Однако, использование устойчивой к кластеризации оценки ковариационной матрицы~(\ref{eq24.34}) приводит к существенным изменениям в стандартных ошибках. Значение $t$-статистики для эластичности расходов падает с 16.01 до 12.68. Все $t$-статистики становятся меньше, а $t$-статистики для переменных SEX и HHSIZE падают ниже 1.96. Как и ожидалось, игнорирование внутрикластерной корреляции приводит к раздутию $t$-статистик методом наименьших квадратов. 

$F$-тест на одновременное равенство всех постоянных эффектов нулю отвергает нулевую гипотезу. Результаты для модели с фиксированными эффектами сходны с полученными ранее, но $t$-статистики стали ещё меньше. Точечная оценка эластичности по доходу теперь 0.60 против 0.67 для МНК. Однако, никаких заметных отличий в результатах с точки зрения роли и влияния отдельных переменных не наблюдается. 

$\chi^2(1)$ тест на равенство нулю случайной составляющей в константе, основанный на~(\ref{eq24.43}), отвергает нулевую гипотезу, показывая, что модель со случайными эффектами является значимым улучшением стандартной модели. Однако, использование RE модели также не приводит к существенным изменениям оценок влияния отдельных переменных. Как и ожидалось, результаты, полученные доступным ОМНК и в модели со случайными эффектами (ОМНК) оказались очень схожими. Небольшие различия обусловлены различиями в оценках, использованных при ОМНК преобразовании. Оценки ДОМНК основываются на $\hat \rho = 0.12$, полученной усреднением 100 разных оценок $\rho$, полученных при помощи 100 репликаций остатков из МНК. 

\begin{table}[h]
\caption{\label{tab:pred} Расходы на медицину во Вьетнаме: описание данных}
\begin{center}
\begin{tabular}{p{3cm} p{10cm} c c}
\hline
\hline
Данные по & Определение & Среднее &  Стандартное\\
домохозяйствам & & & отклонение \\
\hline
LNEXP12M & Общие расходы домохозяйства на медицинское обслуживание за год & 6.31 & 1.59 \\
AGE & Возраст главы домохозяйства & 48.01 & 13.77 \\
SEX & Равен 1, если глава домохозяйства женщина, 0 иначе & 0.27 & 0.44 \\
HHSIZE & Размер домохозяйства & 4.73 & 1.96 \\
URBAN & Равен 1 для городских домохозяйств, 0 иначе & 0.29 & 0.45 \\
EDUC & Количество лет образования главы домохозяйства & 7.09 & 4.41 \\
HHEXP & Общие номинальные расходы домохозяйства (вьетнамские донги, в ценах 1998) & 15273 & 13020 \\
Индивидуальные данные & & &\\
\hline
PHARVIS & Число посещений аптеки & 0.51 & 1.31 \\
LNMEDEXP (>0) & логарифм общих расходов на медицину для тех, у кого они положительны (вьетнамские донги, в ценах 1998)  & 2.14 & 1.08 \\
AGE & Возраст в годах & 29.7 & 9.6 \\
SEX & Равен 1, если респондент --- мужчина & 0.51 & 0.49 \\
MARRIED & Равен 1 для состоящих в браке респондентов & 0.40 & 0.49 \\
EDUC & Уровень полученного образования & 3.38 & 1.94 \\
ILLNESS & Число болезеней, перенесенных за последние 12 месяцев & 0.62 & 0.90 \\
INJURY & Равен 1, если были травмы в период наблюдения & 0.62 & 0.90 \\
ILLDAYS & Число дней, которые человек был болен & 2.80 & 5.45 \\
ACTDAYS & Число дней ограниченной активности & 0.06 & 1.11 \\
INSURANCE & Равен 1, если у респондента есть медицинская страховка & 0.16 & 0.37 \\
MEDEXP (>0) & Медицинские расходы при условии, что они положительны & 21.04 & 208 \\
MEDEXP & Медицинские расходы (вьетнамские донги, в ценах 1998) & 6.13 & 112.75 \\
\hline
\hline
\end{tabular}
\end{center}
\end{table}

Абсолютные различия между результатами  моделей со случайными и фиксированными эффектами сравнительно невелики. Неформальное сравнение не позволяет сказать, какая модель работает значимо лучше; однако, тест Хаусмана показывает, что есть статистически значимые различия между двумя наборами оценок. 

Подводя итог, полученные результаты говорят о том, что необходимо учитывать внутрикластерную корреляцию, однако то, как мы это сделаем, не имеет большого значения. 

Теперь рассмотрим результаты, полученные для счетной переменной --- количества посещений аптеки индивидами (PHARVIS) --- при помощи пуассоновской модели. Это очень интересный показатель, потому что высокая доля медицинских расходов во Вьетнаме приходится на лекарства, которые люди покупают непосредственно в аптеках и принимают самостоятельно, без консультаций с врачом. Считается, что такая форма лечения по качеству хуже лечения, проводимого профессиональным врачом. Во Вьетнаме только часть населения, как правило, высокооплачиваемые работники государственного и частного секторов, могут купить себе медицинскую страховку, которая дает им право на лечение в государственных больницах и получение необходимых лекарств. Из таблицы 24.3 видно, что 16\% населения имеют такую страховку. 


\begin{sidewaystable}[!htbp]
\caption{\label{tab:pred} Расходы на медицину во Вьетнаме: FE и RE модели для положительных расходов}
\begin{minipage}{\textwidth}
\begin{center}
\begin{tabular}{lcccccccccc}
\hline
\hline
& \multicolumn{4}{c}{МНК} & \multicolumn{2}{c}{ДОМНК} & \multicolumn{2}{c}{FE} & \multicolumn{2}{c}{RE (ОМНК)} \\
\cmidrule(r){2-5} \cmidrule(r){6-7} \cmidrule(r){8-9} \cmidrule(r){10-11}
Переменная\footnote{$R^2_W$ --- $R^2$ для within-регрессии; $R^2_B$ --- $R^2$ для between-регрессии; $R^2$ --- общий $R^2$} & Коэф. & OLS & $|t|$-Гет. & $|t|$-Класт. & Коэф. & $|t|$ & Коэф. & $|t|$ & Коэф. & $|t|$ \\
\hline
LNHHEXP & 0.670 & 16.01 & 15.76 & 12.68 & 0.620 & 14.14 & 0.603 & 11.61 & 0.626 & 13.39 \\
AGE & 0.010 & 6.39 & 6.36 & 5.46 & 0.011 & 6.96 & 0.011 & 6.93 & 0.011 & 6.85 \\
SEX & 0.097 & 1.88 & 1.88 & 1.64 & 0.108 & 2.13 & 0.112 & 2.17 & 0.106 & 2.09 \\
HHSIZE & 0.028 & 2.19 & 2.15 & 1.89 & 0.014 & 1.06 & 0.010 & 0.76 & 0.015 & 1.17 \\
FARM & 0.134 & 2.73 & 2.72 & 2.22 & 0.088 & 1.58 & 0.069 & 1.14 & 0.092 & 1.69\\
EDUC & $-0.090$ & 7.36 & 7.07 & 6.03 & $-0.61$ & 4.73 & $-0.51$ & 3.76 & $-0.063$ & 4.92 \\
CONS & $-0.510$ & 1.34 & 1.34 & 1.09 & $-0.051$ & 0.30 & $-0.051$ & 0.08 & $-0.166$ & 0.40 \\
$R^2$ & \multicolumn{2}{c}{0.088} &  &  &  &  &  &  &  &  \\
$R^2_W$ (>0) & \multicolumn{8}{c}{ } & \multicolumn{2}{c}{0.051} \\
$R^2_B$ & \multicolumn{8}{c}{ } & \multicolumn{2}{c}{0.288} \\
$\rho$ & \multicolumn{4}{c}{ } & \multicolumn{2}{c}{0.12} & \multicolumn{4}{c}{ }\\
$\frac{\sigma^2_{\alpha}}{\sigma^2_{\alpha} + \sigma^2}$ & \multicolumn{8}{c}{ } & \multicolumn{2}{c}{0.093}\\
$F(193, 4806)$ & \multicolumn{6}{c}{ } & \multicolumn{2}{c}{3.49} & \multicolumn{2}{c}{ } \\
$\chi^2(1)$ & \multicolumn{8}{c}{ } & \multicolumn{2}{c}{432.75} \\
Хаусман $\chi^2(6)$ & \multicolumn{6}{c}{ } & \multicolumn{4}{c}{17.89} \\
N & \multicolumn{4}{c}{5006} & \multicolumn{2}{c}{4977} & \multicolumn{4}{c}{ } \\
\hline
\hline
\end{tabular}
\end{center}
\end{minipage}
\end{sidewaystable}


 
\begin{table}
\caption{\label{tab:pred} Число посещений аптек}
\begin{center}
\begin{tabular}{lccccccccccc}
\hline
\hline
Число & 0 & 1 & 2 & 3 & 4 & 5 & 6 & 7 & 8 & 9 & 10+ \\
\hline
PHARVIS & 20639 & 3827 & 1716 & 776 & 359 & 174 & 64 & 43 & 16 & 4 & 115 \\
PHARVIS (доля) & 0.744 & 0.137 & 0.062 & 0.028 & 0.013 & 0.006 & 0.002 & 0.001 & 0.000 & 0.000 & 0.004 \\
\hline
\hline
\end{tabular}
\end{center}
\end{table}

В таблице 24.5 приведено наблюдаемое частотное распределение переменной PHARVIS. Около $26\%$ респондентов по меньшей мере один раз за период наблюдения посещали аптеку и около $95\%$ делали это не более трех раз. 


\begin{table}[h]
\caption{\label{tab:pred} RE и FE модели для числа посещений аптек}
\begin{center}
\begin{tabular}{ p{2.5cm} cc cc cc cc}
\hline
 & \multicolumn{2}{c}{Пуассон} & Уст. к гет. & Уст. кластер & \multicolumn{2}{c}{FE Пуассон} & \multicolumn{2}{c}{RE Пуассон}\\
\cmidrule(r){2-3} \cmidrule(r){4-4} \cmidrule(r){5-5} \cmidrule(r){6-7} \cmidrule(r){8-9}
Переменная & Коэф. & $|t|$ & $|t|$ & $|t|$ & Коэф. & $|t|$ & Коэф. & $|t|$ \\
\hline
CONS & $-1.637$ & 35.78 & 18.81 & 12.25 & ---  & --- & 1.318 & 19.41 \\
LNHHEXP & 0.078 & 5.68 & 3.08 & 1.90 & $-0.114$ & 6.01 & $-0.095$ & 4.95 \\
INSURANCE & $-0.245$ & 9.57 & 5.68 & 4.29 & $-0.163$ & $6.17$ & $-0.178$ & $6.44$ \\
SEX & 0.084 & 4.96 & 2.76 & 2.73 & 0.098 & 5.75 & 0.099 & 571 \\
AGE & 0.024 & 2.38 & 1.27 & 1.06 & 0.003 & 0.32 & 0.005 & 0.55 \\
MARRIED & 0.124 & 5.92 & 2.96 & 2.78 & 0.164 & 7.59 & 0.158 & 7.38 \\
ILLDAYS & 0.042 & 40.00 & 14.91 & 12.91 & 0.046 & 40.14 & 0.46 & 40.18 \\
ACTDAYS & 0.008 & 1.71 & 0.43 & 0.45 & 0.025 & 4.53 & 0.024 & 4.35 \\
INJURY & 0.171 & 2.30 & 0.84 & 0.85 & 0.144 & 1.80 & 0.143 & 1.80 \\
ILLNESS & 0.562 & 87.15 & 24.60 & 21.81 & 0,584 & 73.45 & 0.585 & 74.16 \\
EDUC & $-0.52$ & 11.10 & 6.47 & 3.92 & $-0.24$ & 4.18 & $-0.026$ & 4.61\\
$-\ln L$ & & \multicolumn{2}{c}{25281} &  & \multicolumn{2}{c}{22446} & \multicolumn{2}{c}{23419} \\
N & & \multicolumn{2}{c}{27765} & & \multicolumn{2}{c}{27671} & \multicolumn{2}{c}{27765} \\
\hline
\hline
\end{tabular}
\end{center}
\end{table}


В таблице 24.6 представлены результаты оценивания для нескольких вариантов пуассоновской регрессии, аналогичных приведенными в таблице 24.4 для линейной регрессии. В первом столбце даны оценки ММП, во втором --- обычные нескоректирвоанные $t$-статистики. Далее приводятся робастные к гетероскедастичности $t$-статистики. Они заметно меньше, иногда более чем вдвое, чем нескорректированные. В четвертом столбце находятся скорректированные на кластеризацию $t$-отношения, основанные на~(\ref{eq24.45}). Они заметно меньше, чем полученные другими способами статистики, что говорит о наличии значительной внутрикластерной корреляции. Средний размер кластера больше 140, поэтому даже небольшая внутрикластерная корреляция может сильно <<раздуть>> $t$-статистики. 

После этого мы рассматриваем моделирование внутрикластерной корреляции при помощи моделей со случайными и фиксированными эффектами. Модель с фиксированными эффектами оценивается при помощи условного ММП. Кластеры, где недостаточно внутрикластерной изменчивости, исключаются из рассмотрения. Полученные коэффициенты кардинально отличаются от полученных для пуассоновской регрессии при помощи метода максимального правдоподобия. Во-первых, коэффициент при $\ln(HHEXP)$ из значимого положительного становится значимым отрицательным. Это означает, что стандартная регрессия определяет посещение аптеки как нормальный товар, а регрессия с фиксированными эффектами --- как инфериорный; то есть, люди меньше занимаются самолечением по мере роста дохода. Это может быть объяснено тем, что постоянные эффекты учитывают влияние пропущенных переменных, коррелированных с зависимой переменной. К пропущенным переменным можно отнести количество и качество альтернативных медицинских услуг, доступных жителям коммуны. Они могут серьезно изменяться в зависимости от географического расположения и экономического статуса коммуны. 

Последние два столбца таблицы 24.6 показывают результаты, полученные по модели со случайными эффектами. Здесь предполагается, что константа в пуассоновской модели изменяется в зависимости от кластера и случайна, при этом все константы извлекаются из одного одномерного распределения --- гамма-распределения с единичным средним. Такая формулировка привлекает тем, что не требует дополнительных условий. Пуассоновская модель со случайными эффектами с постоянным членом, подчиняющимся гамма-распределению, была предложена Хаусманом и другими (1984); её функцию правдоподобия можно представить аналитически и адаптировать для случая кластеризованных данных. Оценки, полученные при помощи моделей со случайными эффектами, качественно похожи на полученные по модели с фиксированными эффектами. Однако, оценка коэффициента при доходе --- одной из ключевых переменных --- серьезно отличается от полученной для простой пуассоновской регрессии. 

Этот пример показывает, что внутрикластерная корреляция может влиять не только на эффективность, но и на сами оценки. 

\section{Комплексные опросы}

В предыдущих разделах наше внимание было сосредоточено на стратификации, взвешивании и кластеризации по отдельности. Здесь мы сосредоточимся на комплексных опросах, которые используют многоуровневые стратифицированные выборки с кластеризацией. Такие опросы фокусируются на описании генеральной совокупности в условиях, когда её параметры могут изменяться в зависимости от страты. Тогда используются взвешенные оценки и они же рассматриваются как оценки параметров генеральной совокупности. Цель --- получение состоятельной оценки дисперсии взвешенной оценки с учетом кластеризации, которая может быть сложнее, чем рассмотренная в разделе 24.5. 

\subsection{Оценка дисперсии в комплексных опросах}

Рассмотрим следующую ситуацию. $i$-ое наблюдение в выборке --- это домохозяйство $j$, находящееся в кластере $c$ из страты $s$. К примеру, зависимая переменная обозначается $y_{scj}$, хотя, строго говоря, наблюдение $(s, c, j)$ надо представлять как $(s, c_s, j_{C_s})$. Данные --- это наборы $(y_{scj}, x_{scj}, w_{scj})$, где $w_{scj}$ --- это выборочные веса, обратно пропорциональные вероятности попадания наблюдения в выборку. Нижние индексы упорядочены по уровню дезагрегирования, в обратном порядке по сравнению с нотацией из раздела 24.5. 

Внутри страты используется двухшаговый или многошаговый отбор, поэтому домохозяйства попадают в выборку в результате по меньшей мере двух последовательных отборов. На первом шаге случайным образом выбирается подмножество  из всех PSU внутри страты. На втором --- выбирается подмножество из домохозяйств, входящих в выбранные PSU. На этом этапе возможен кластеризованный отбор. Также возможны дальнейшие отборы при использовании SSU. 

\subsection*{Дисперсия линейной статистики}

Начнем с рассмотрения оценки дисперсии линейной статистики, которую можно просуммировать по стратам, PSU и домохозяйствам:

$$
\hat u = \sum_{s=1}^S \sum_{c=1}^{C_s} \sum_{j=1}^{N_{cs}} u_{scj} = \sum_{s=1}^S \sum_{c=1}^{C_s} u_{sc}, 
$$
где $u_{sc}$ --- это сумма по PSU, то есть
$$
u_{sc} = \sum_{j=1}^{N_{cs}} u_{scj}.
$$
Примеры $u_{scj}$, такие, как взвешенное среднее или взвешенная регрессия, приведены ниже. Дисперсия $u$:

$$
\V[u] = \sum_{s=1}^S \sum_{c=1}^{C_s} \V[u_{sc}] = \sum_{s=1}^S C_s \sigma^2_s, 
$$
В предположении, что $u_{sc}$ независимы по стратам и одинаково распределены по PSU с общей дисперсией $ \sigma^2_s$. Если $u_{sc}$ независимы по $c$ и одинаково распределены, можно использовать обычную несмещенную оценку дисперсии $ \sigma^2_s$, так что $\hat  \sigma^2_s = (C_s --- 1)^{-1} \sum_c (u_{sc} --- \overline{u}_s)^2$. Тогда 

\begin{equation}
\label{eq24.55}
\V[\hat u] = \sum_{s=1}^S \frac{C_s}{C_s --- 1}  \sum_{c=1}^{C_s} (u_{sc} --- \overline{u}_s)^2, 
\end{equation}
где $\overline{u}_s = C_s^{-1} \sum_c u_{sc}$ --- среднее по страте сумм по PSU. 

Эта оценка допускает кластеризацию внутри PSU, потому что 

\begin{align}
\sum_{c=1}^{C_s} (u_{sc} --- \overline{u}_s)^2 & = \sum_{c=1}^{C_s} \left(  \sum_{j=1}^{N_{cs}} u_{scj} ---  \overline{u}_s \right)^2 \nonumber\\
& =\sum_{c=1}^{C_s} \sum_{j=1}^{N_{cs}} (u_{scj} --- \overline{u}_s)^2 + \sum_{c=1}^{C_s} \sum_{j=1}^{N_{cs}} \sum_{k \ne j}^{N_{cs}} (u_{scj} --- \overline{u}_s)(u_{sck} --- \overline{u}_s). \nonumber
\end{align}
Первая сумма проистекает из дисперсии при SRS. Вторая сумма будет положительной при кластеризованном отборе и тем самым увеличит дисперсию. По поводу вида отбора внутри страты и типа кластеризации не делается никаких предположений. К примеру, (\ref{eq24.55}) дает правильные стандартные ошибки даже при трехшаговом отборе с дальнейшим отбором внутри SSU. 

Оценка~(\ref{eq24.55}) требует, чтобы из каждой страты извлекалось по меньшей мере два PSU. Если извлекается только один PSU, одна из возможностей --- это включить страту с одним PSU в другую страту, которая а-приори считается похожей. Это допустимо при $C_s \ge 2$ то есть, если на одну страту приходится по меньшей мере два PSU. Это приведет к завышению оценки $\V[u]$, потому что различия в средних по стратам привносят смещение вверх. \footnote{Это метод не применим к Текущему обследованию населения, CPS, потому что во многие страты входит только один PSU, а для остальных только один PSU попадает в опрос. Для борьбы с этим формируются псевдо-страты и используются методы репликации для получения новых выборок из псевдо-страт. См. Бюро переписи населения США (U.S. Bureau of Census, 2002)}

На практике PSU часто отбираются без возвращения, поэтому в $u_{sc}$ присутствует определенная зависимость. Тогда~(\ref{eq24.55}) переоценивает $\V[u]$, аналогично ситуации из раздела 24.2.3. В этом случае используются более сложные формулы. 

\subsection*{Дисперсия взвешенного среднего}

Среднее по генеральной совокупности оценивается как отношение суммы взвешенных $y_{scj}$, назовем её $\hat y$, к сумме выборочных весов, $\hat w$. Тогда

$$
\overline{y}_W = \hat y / \hat w = \sum_{s=1}^S \sum_{c=1}^{C_s} \sum_{j=1}^{N_{cs}} w_{scj} y_{scj} \bigl/  \sum_{s=1}^S \sum_{c=1}^{C_s} \sum_{j=1}^{N_{cs}} w_{scj}.
$$

Если выборочные веса известны, получаем
$$
\overline{y}_W = \sum_{s=1}^S \sum_{c=1}^{C_s} \sum_{j=1}^{N_{cs}} w^*_{scj} y_{scj},
$$
где $w^*_{scj} =w_{scj} / \hat w$ и $\V[\overline{y}_W]$ может быть рассчитана при помощи~(\ref{eq24.55}) с $u_{scj} = w^*_{scj} y_{scj}$. 

Если выборочные веса рассматриваются как неизвестные, можно использовать \bfseries дельта-метод \mdseries или \bfseries  метод линеаризации \mdseries для получения $\V[\hat y / \hat w]$ как функции от $\V[\hat y]$, $\V[\hat w]$ и $\Cov[\hat y, \hat w]$. Первые два показателя могут быть оценены при помощи~(\ref{eq24.55}) с $u_{scj} = w_{scj} y_{scj}$ и $u_{scj} = w_{scj}$. Третий может быть оценен, если заменить $(u_{sc} --- \overline{u}_s)^2$ в~(\ref{eq24.55}) на $(u_{sc} --- \overline{u}_s) (v_{sc} --- \overline{v}_s)$ и принять $u_{scj} = w_{scj} y_{scj}$ и $v_{scj} = w_{scj}$. Это пример оценки-соотношения. 

Для нелинейных статистик, таких как оценки-соотношения, в литературе предлагаются другие методы, основанные на техниках \bfseries джекнайф \mdseries и \bfseries повторяющихся сбалансированных репликациях. \mdseries Из-за нелинейности оценки дисперсии больше не являются несмещенными, но остаются состоятельными при количестве страт $S \to \infty$ (см. Кревски и Рао, 1981). Некоторые результаты при фиксированном S и $\sum_{c=1}^{C_s} N_{cs} \to \infty$ приведены в Волтер (1985). Можно также использовать \bfseries бутстрап\mdseries, но тут требуется осторожность. Подробнее см. Рао и Ву (1988) и Шао и Ту (1995). 


\subsection*{Дисперсия оценки взвешенного МНК}

Оценка $\hat \beta_W$ взвешенного МНК параметров регрессии (см. раздел 24.3) является решением

$$
\sum_{s=1}^S \sum_{c=1}^{C_s} \sum_{j=1}^{N_{cs}} w_{scj} x_{scj} (y_{scj} --- x'_{scj} \hat \beta_W) = 0.
$$

Обычными алгебраическими преобразованиями получаем
$$
\hat \beta_W --- \beta = \left( \sum_{s=1}^S \sum_{c=1}^{C_s} \sum_{j=1}^{N_{cs}} w_{scj} x_{scj} x'_{scj} \right)^{-1} \times \sum_{s=1}^S \sum_{c=1}^{C_s} \sum_{j=1}^{N_{cs}} w_{scj} (y_{scj} --- x'_{scj} \hat \beta_W).
$$
Соответственно, получаем дисперсию $\beta$ в сэндвич-форме $\V[\hat \beta] = A^{-1} B A^{-1}$, где B --- дисперсия второй тройной сумму, которую можно оценить при помощи~(\ref{eq24.55}) с $u_{scj} = w_{scj} x_{scj} (y_{scj} --- x'_{scj} \hat \beta_W)$. 

\subsection*{Дисперсия взвешенной М-оценки}

В достаточно общем случае рассматривается взвешенная М-оценка $\hat \theta_W$, которая является решением

$$
\sum_{s=1}^S \sum_{c=1}^{C_s} \sum_{j=1}^{N_{cs}} w_{scj} h( y_{scj}, x_{scj}, \hat \theta_W) = 0
$$
Примеры включают линейную регрессию с $h_{scj} = x_{scj} (y_{scj} --- x'_{scj} \beta)$ и квази-ММП с $h_{scj} = \partial \ln f (y_{scj} | x_{scj}, \theta) / \partial \theta$. 

Предполагая состоятельную оценку $\theta$, которая требует $\E[h(y_{scj}, x_{scj}, \theta)] = 0$, мы можем использовать разложение в ряд Тейлора первого порядка и получить

$$
\sqrt{N} (\hat \theta_W --- \theta) \overset{d}{\to} \mathcal N \left[  0, A^{-1} B A'^{-1} \right],
$$
где
$$
A = \plim N^{-1} \sum_{s=1}^S \sum_{c=1}^{C_s} \sum_{j=1}^{N_{cs}} w_{scj} \frac{\partial h(y_{scj}, x_{scj}, \theta)}{\partial \theta'}
$$
и
$$
B = \plim N^{-1} \sum_{s=1}^S \sum_{c=1}^{C_s} \sum_{j=1}^{N_{cs}} \sum_{k=1}^{N_{cs}} w_{scj} w_{sck} h(y_{scj}, x_{scj}, \theta)  \frac{\partial h(y_{sck}, x_{sck}, \theta)}{\partial \theta'},
$$
где выражение для $B$ предполагает независимость $h_{scj}$ по стратам и кластерам, но разрешает зависимость внутри кластера. Оценивание $A$ очевидно. Для $B$ используется~(\ref{eq24.55}) с $u_{scj} = w_{scj} h_{scj}$, поэтому

$$
B = \sum_{s=1}^S \frac{C_s}{C_s --- 1} \sum_{c=1}^{C_s} [\overline{z}_{sc} --- \overline{z}_s]^2, 
$$
где $ \overline{z}_{sc} = \sum_{j=1}^{N_{cs}} w_{scj} h (y_{scj}, x_{scj}, \theta)$ и $\overline{z}_{s} = C_s^{-1} \sum_{c=1}^{C_s} \overline{z}_{s}$. 

\subsection*{Эндогенная стратификация}

Саката (1998) распространил эти результаты на случай эндогенного отбора. Он рассматривает параметры ценза и разрабатывает асимптотическую теорию, предполагая, что количество страт $S \to \infty$. Результаты аналогичны представленным в предыдущем разделе. 

\section{Практические соображения}

В микроэкономических исследованиях обычным является использование структурного подхода. При отсутствии эндогенной стратификации используются невзвешенные оценки. Главная проблема заключается в поиске правильных стандартных ошибок при наличии кластеризации. При случайных кластерных эффектах потери в эффективности от игнорирования кластеризации невелики. В некоторых пакетах могут быть встроены робастные к кластеризации оценки стандартных ошибок (не путать с робастными к гетероскедастичности), которые подходят для случайных кластерных эффектов и большого количества кластеров. CSRE и CSFE модели могут быть реализованы при помощи МНК, при условии (в случае CSFE), что кластеров не слишком много. В качестве альтернативы можно проводить оценивание при помощи процедур для панельных данных, если они поддерживают несбалансированные панели. Как и в случае в панельными данными, большинство исследователей, не занимающихся глубоко эконометрикой, могут быть удовлетворены моделями со случайными эффектами, но для получения состоятельных оценок иногда всё же требуется использовать постоянные эффекты. 

Если применяется описательный подход и параметры изменяются в зависимости от страты, необходимо использовать взвешивание. Можно использовать взвешивание для МНК оценок, но его необходимо комбинировать с робастными к кластеризации стандартными ошибками. В некоторых пакетах есть модули для оценки опросных данных, которые позволяют получить робастные к кластеризации стандартные ошибки методами, представленными в разделе 24.6. В пакете SUDAAN представлены многие методы, рассмотренные в этой главе для линейных и основных нелинейных моделей. 

\section{Библиографические заметки}

\begin{enumerate}
\item[$24.2 --- 24.3$] Есть множество литературы, посвященной отбору. Классические ссылки на выборочные опросы включают Киш (1965) и Кохран (1997, первое издание в 1953). В работе Скиннера (1989) приведен хороший обзор, у Гровза (1989) представлено достаточно нетехническое описание подходов к опросам, используемых во многих общественных науках, и при этом рассмотрено много полезных на практике вопросов. Для полноты мы представили это литературу по отбору при формировании опросов, хотя эконометрические исследования редко используют методы, представленные в разделе 24.8. Есть немного эконометрических работ, за исключением Падни (1989) и Дитона (1997) и главы в книге Уллаха и Бройнига (1998). 
\item[$24.4$] Больше всего внимания в эконометрической литературе было уделено учету эндогенной стратификации. Литературы много и мы ограничиваемся простым обзором. Подробнее эти вопросы разбирает Амэмия (1985), где приведено много ссылок, включая Мански и Лерман (1977) для моделей дискретного выбора и Хаусман и Уайз (1979) для моделей с самоотбором выборки. Простые взвешенные оценки как правило применимы, хотя и неэффективны. Имбенс и Ланкастер (1996) рассматривают способ реализации на практике эффективной оценки при заданной спецификации условной плотности. 
\item[$24.5$] Для микроэконометрических приложений учет кластеризации очень важен. Работы Клока (1981) и Моултона (1986, 1990) сыграли ключевую роль в осознании этой проблемы. Дэвис (2002) рассматривает общий подход к моделям с многоуровневыми ошибками. Гробард и Корн (1994) хорошо разбирают линейный регрессионный анализ в приложении к кластеризованным данным. Они рассматривают модели как со случайными, так и с постоянными эффектами, с упором на предположения, которые должны быть выполнены, чтобы оценки модели со случайными эффектами оставались применимыми. Пендергаст и другие (1996) приводят обзор методов анализа кластеризованных бинарных данных. Из-за того, что средний член на правой стороне уравнения~(\ref{eq24.34}) включает усреднение по кластерам, точность этой оценки зависит от количества кластеров. Последствия использования робастных к кластеризации оценок ковариационной матрицы при малом числе кластеров остается интересной темой для исследования (Доналд и Ланг, 2001; Ангрист и Лэви, 2002). Вулдридж (2003) приводит обзор. 
\item[$24.6$] Иерархические линейные модели оцень активно используются в общественных науках. Брайк и Рауденбуш (2002) приводят хороший обзор, покрывающий ситуации бинарных, порядковых, счетных и мультиномиальных зависимых переменных как с точки зрения вероятности, так и с точки зрения байесовского подхода. 
\item[$24.7$] Дитон (1997) рассматривает ряд вопросов, связанных с оценкой моделей по кластеризованным данным опросов о стандартах жизни (Living Standarts Surveys), проводимых Всемирным Банком в развивающихся странах. 
\item[$24.8$] Во многих стандартных статистических пакетах (к примерах, STATA и SUDAAN) встроены процедуры для оценки моделей с постоянными и случайными кластерными эффектами для линейных и нелинейных моделей по сквозным и панельным данным. 
\end{enumerate}

\subsection*{Упражнения}

\begin{enumerate}
\item[$24 --- 1$]
\begin{enumerate}
\item Проверьте выражение для $\sum_c$, приведенное в~(\ref{eq24.25}).
\item Докажите состоятельность для оценок $\hat \sigma ^2$ и $\hat \rho$ в CSRE модели.
\item Рассмотрите смещение в стандартных ошибках для CSRE модели по сбалансированным кластерам. Покажите, что в этом случае $\E [\sum_c \sum_j \hat u^2_{cj}] = \sigma^2 [N --- K (1 + \rho (m-1))]$. 
\end{enumerate}

\item[$24 --- 2$] (адаптировано из Гринвальда, 1983) Рассмотрите линейную регрессионную модель $y = X \beta + u$, где $\E[u] = 0$ и $\E [uu'] = \sigma^2 \Omega^* = \Omega$. При помощи стандартного результата для МНК оценки $\hat \beta = (X'X)^{-1} X'y$ (см. раздел 4.4) можно получить правильное выражение для $\V[\hat \beta]$ как $V_2 = (X'X)^{-1} (X' \Omega X)^{-1} (X'X)^{-1}$, тогда как $V_1 = \hat \sigma^2 (X'X)^{-1}$ с $\hat \sigma^2 = \hat u' \hat u / (N-K)$ некорректна при $\Omega \ne I$. 
\begin{enumerate}
\item Покажите, что смещение $V_1$ можно задать как $B = B_1 + B_2$, где $B_2 = (X'X)^{-1} X' (\Omega --- \sigma^2 I) X (X'X)^{-1}$ и $B_1 = (N --- K)^{-1} \tr\{B_2 (X'X)\} (X'X)^{-1}$ (Гринвальд называет $B_2$ <<прямым смещением>>)
\item Оцените два члена для специального случая $X'X = I_K$. Покажите, что $B \to B_2$ при $N \to \infty$. 
\end{enumerate}

\item[$24 --- 3$] Рассмотрите робастную к кластеризации МНК оценку дисперсии~(\ref{eq24.33}). Предположите, что есть два уровня кластеризации. К примеру, в терминах эмпирических приложений из этой главы, кластеризация может быть на уровне семьи и, затем, коммуны, если несколько членов одной семьи из одной коммуны включены в выборку. Как будет модифицироваться формула при наличии двух уровней кластеризации?

\item[$24 --- 4$] Для этого упражнения используйте $50\%$ выборки из Обследование уровня жизни во Вьетнаме, VLSS. Определите $y = 1$, если индивид по меньшей мере один раз посещал аптеку (PHARVIS) и $y = 0$ иначе. Этот пример предполагает, что у Вас есть доступ к программе, работающей с кластеризованными данными. 
\begin{enumerate}
\item Используя те же объясняющие переменные, что были использованы в пуассоновской модели в разделе 24.7, оцените бинарную логит-модель методом максимального правдоподобия, используя как стандартную оценку дисперсии, так и робастную в сэндвич-форме. 
\item Переоцените модель из пункта $(a)$, используя робастные к кластеризации стандартные ошибки. Объясните разницу между робастными стандартными ошибками из пунктов $(a)$ и $(b)$. 
\item Используйте переменную <<commune>> в качестве идентификатора кластера. Переоцените логит-модель при помощи моделей с постоянными и случайными кластерными эффектами. Сравните оценки и стандартные ошибки коэффициентов при LNHHEXP и INSURANCE. Влияет ли кластеризация в данных на выводы о значимости этих переменных? 
\end{enumerate}

\end{enumerate}






\chapter{Оценка эффектов воздействия}

\section{Введение}

Оценка воздействия (treatment evaluation) рассматривает вопросы измерения эффекта вмешательств того или иного рода на интересующие нас показатели, при этом как вмешательство, так и результирующий показатель могут определяться достаточно широко и применяться в различных контекстах. Сам подход и часть терминологии пришли из медицинских наук, где под вмешательством часто понимается использование того или иного типа лечения. Проведя лечение, можно измерять его эффект в сравнении, к примеру, с отсутствием лечения или другим его видом. В экономических приложениях воздействия и вмешательства как правило имеют сходный смысл. 

К примерам воздействия в экономическом контексте можно отнести участие в программах профессионального обучения, членство в профсоюзе, получение трансфертного платежа в рамках социальной программы, изменения в правилах получения социальных пособий или осуществления финансовых трансакций, изменение экономических стимулов и так далее; эти вопросы разобраны в работах Моффитта (1992), Фридлэндера, Гринберга, и
Роббинса (1997), и Хекмана, Лалонда, и Смита (1999). Если осуществляемое воздействие может быть разной интенсивности или типа, используется термин  \bfseries <<множественные воздействия>>. \mdseries По сравнению со случаем единичного воздействия эта ситуация не создает дополнительных затруднений, но позволяет более гибко определять базу для сравнения. 

Результат в данном случае заключается в изменении экономических показателей индивида при изменении его экономического статуса или внешних условий. Чаще всего результирующая переменная $y$ является непрерывной, и переменная воздействия $D$ --- бинарной, равной 1, если воздействие было и 0 в противном случае. Пример вмешательства из экономики труда --- это профессиональное обучение, которое может влиять на зарплату работников, прошедших обучение. В общем, однако, и результат, и воздействие, могут быть как непрерывными, так и дискретными. Детали анализа могут изменяться, но общая идея будет применима ко всем ситуациям. Для простоты, мы сосредоточимся на случае непрерывной результирующей переменной и бинарного воздействия. В дальнейшем мы расширим анализ на другие полезные на практике случаи. 

Очевидно, что оценка воздействия очень полезна для построения государственной политики, потому что успешные воздействия могут получать более широкое распространение, а в существующие программы могут вноситься улучшения. Хекман и Смит (1998) рассмотрели связь между несколькими наиболее часто используемыми измерителями эффекта воздействия и традиционным анализом выгод и издержек. 

Стандартная проблема анализа воздействия заключается в предпосылке о наличии причинной связи между воздействием и результатом. В каноническом примере одиночного воздействия мы наблюдаем $(y_i, x_i, D_i), \; i = 1, \dots, N$ и изучаем влияние гипотетического изменения $D$ на $y$ при фиксированном $x$. Такая ситуация рассматривается в модели с потенциальным результатом, уже рассмотренной в Главе 2, где интересующая нас результирующая переменная сравнивается на объектах, подвергшихся и не подвергшихся воздействию. Следовательно, это похоже на ситуацию с пропущенными данными, где можно изучать причинно-следственные связи и выдвигать гипотезы. Нас интересует, как результат среднего индивида, не подвергшегося воздействию, поменялся бы если бы он подвергся воздействию. То есть, нас интересует величина $\Delta y / \Delta D$ --- эффект этого воздействия. Здесь причинность в терминах ceteris paribus --- при прочих равных. 

Чем отличается эта глава от предыдущих, где мы тоже рассматривали идентификацию и оценку моделей? Здесь меняется объект нашего внимания, из-за чего возникают отличия. Главное отличие проистекает из того, что мы ориентируемся на целое семейство мер эффективности воздействия. Эти меры являются функциям от параметров и данных, и они позволяют проверять связанные с проводимой политикой гипотезы. Важный и интересный результат заключается в том, что не все меры могут быть построены при наличии данных и оценок. Выбор оценки и типа используемых данных налагает ограничения на гипотезы, которые можно проверить, и, следовательно, на эффекты, которые можно состоятельно оценить. 

Помимо в этого, в литературе также делается упор на преимуществах оценки с использованием минимума функциональных форм и ограничений (к примеру, полупараметрических оценок). Причиной тому является желание получить значимые для политики результаты, по возможности не зависящие от сильных предположений. Допустимость использования полупараметрических методов оценки воздействия в линейных моделях показать проще, чем в случае использования нелинейных моделей с ограничениями на зависимые переменные. 

В разделе 25.2 рассмотрены предположения, связанные с оцениванием. Раздел 25.3 рассматривает меры эффектов воздействия, которые обычно используются при идентификации и оценке. В разделе 25.4 рассматриваются оценки при помощи сопоставления (matching) и мер склонности (propensity score). Оценка воздействия методом разность разностей (differences-in-differences), использующаяся при изучении событий с использованием квази-экспериментальных данных, рассмотрена в разделе 25.5. Продолжая работать с квази-экспериментальными данными, мы рассмотрим разрывные регрессии (regression discontinuity) в разделе 25.6 и оценки методом инструментальных переменных в разделе 25.7. Рассматриваются преимущественно линейные модели. В разделе 25.8 приводятся примеры применения предложенных методов на практике. 

\section{Структура и предположения}

Методы оценивания эффектов воздействия опираются на предположения, позволяющие идентифицировать причинные эффекты, так же, как и, к примеру, линейная модель одновременных уравнений опирается на предположения того же рода (см. Главу 2). В этом разделе мы рассмотрим предположения, позволяющие оценки сопоставления и мер склонности, предложенные в разделе 25.4. 

Для начала рассмотрим структуру для получения причинных параметров при оценке воздействия. 

\subsection{Условия исследования эффектов воздействия}

В плане условий, начнем со случайного назначения объектов, на которые будет оказано воздействие, в рамках социального эксперимента, описанного в разделе 3.3. Пусть есть целевая совокупность для воздействия и $N$ обозначает количество случайно отобранных индивидов, на которых будет оказано воздействие. Тогда $N_C = N - N_T$ --- число индивидов, не подвергнутых воздействию, которые служат контрольной группой. 

Случайное назначение предполагает, что процесс назначения игнорирует возможное влияние воздействия на результат. К примеру, никто не попадает в выборку только потому, что потенциальная выгода от его включения будет большой и никто не игнорируется потому, что выгода от его включения невелика. Обозначим $(y_i, x_i, D_i; \; i = 1, \dots, N)$ вектор наблюдений со скалярной результирующей переменной $y$, вектором наблюдаемых переменных $x$ и бинарным индикатором воздействия $D$. Для простоты, предположим, что все, кому назначено воздействие, получают его, а все кому не назначено --- не получают. Результирующая переменная индивида, подвергшегося воздействию, обозначается $y_1$, не подвергшегося --- $y_0$. После проведения эксперимента и сбора данных, мы оцениваем эффект этого воздействия. Самый естественный способ измерения эффекта воздействия --- это построить показатель, который будет сравнивать средние результаты в подвергшихся и не подвергшихся воздействию группах. 

С одним важным отличием, та же самая схема может применяться и к данным, полученных в ходе наблюдения. Отличие заключается в том, что теперь нет механизма случайного назначения, возможно, потому что индивиды сами решают, подвергаться им воздействию или нет, возможно по другим причинам. 

Надо заметить, что большая часть работ в области оцени воздействия имеют характер частичного равновесия, то есть, они предполагают отсутствие эффектов общего равновесия. Под этим мы подразумеваем, что эффекты воздействия невелики и не влияют на переменные, которые мы считаем экзогенными. Это предположение не выполняется, если мы рассматриваем программу, затрагивающую целый сектор, являющийся важной частью национальной экономики. К примеру, введение всеобщего медицинского страхования может повлиять на весь сектор, связанный с оказанием медицинских услуг, что может затруднить применение методов, рассматриваемых в этой главе. 

Есть несколько проблем, которые могут возникнуть при оценке эффектов воздействия. Есть также небольшие различия в интерпретациях, возникающие из-за различий в предположениях, используемых для получения оценок. Следовательно, мы начнем с рассмотрения этих предположений. 

\subsection{Предположение об условной независимости}

Необходимо ввести ряд предположений, чтобы сравнения результатов в двух группах имели какой-то смысл. Для начала мы перечислим и объясним их, а затем используем в обсуждении идентифицируемости отдельных эффектов воздействия. 

Важным предположением является \bfseries предположение об условной независимости \mdseries, согласно которому, результаты и воздействие независимы при заданных $x$, то есть

\begin{equation}
\label{eq25.1}
y_0, y_1 \perp D | x. 
\end{equation}
Это означает, что участие в программе не зависит от результата после учета изменчивости результата, обусловленной различиями в $x$. Правильно примененное случайное назначение обеспечивает выполнение этого требования. При полностью случайном назначении мы можем сделать даже более сильное предположение:

\begin{equation}
\label{eq25.2}
y_0, y_1 \perp D, 
\end{equation}
потому что назначение будет случайным на всем пространстве $(y, x)$. Более часто применяемое предположение~(\ref{eq25.1}) может быть использовано для идентификации параметров воздействия, потому что, согласно ему, воздействие и результат независимы, если мы учли влияние регрессоров $x$, некоторые из которых могут быть связаны с $D$. 

Предположение об условной независимости достаточно широко и предполагает следующее 


\begin{align}
\label{eq25.3}
& F(y_j | x, D = 1) = F(y_j | x, D = 0) = F(y_j | x), \; j = 0,1 ,  \\
& F(u_j | x, D = 1) = F(u_j | x, D = 0) = F(u_j | x), \; j = 0,1, \nonumber
\end{align}
где $u$ --- ошибка в модели регрессии. Это означает, что решение об участии не влияет на \bfseries распределение возможных результатов \mdseries. 

Чтобы оценить последствия этого предположения, предположим, что $\E[y|x, D]$ линейно, то есть, уравнение результат-участие:

\begin{equation}
\label{eq25.4}
y = x' \beta + \alpha D + u, 
\end{equation}
где $\E[u|D] = \E [y - x' \beta - \alpha D | D] = 0$. Следовательно, $D$ может трактоваться как экзогенная переменная, и тогда в модели не будет смещения из-за одновременности или самоотбора. При стандартных предположениях на $x$, возможно получить состоятельные оценки параметров регрессии. 

Можно рассмотреть предположение, которое будет слабее, чем~(\ref{eq25.1}):

\begin{equation}
\label{eq25.5}
y_0 \perp D | x,
\end{equation}
которое предполагает условную независимость участия и $y_0$. Это предположение используется для установления идентифицируемости среднего по совокупности \bfseries эффекта воздействия на подвергшуюся воздействию группу (treatment effect on the treated, ATET) \mdseries, как будет показано ниже. 

В литературе встречаются другие названия для предположения~(\ref{eq25.5}). Имбенс (2005) ссылается на него как на \bfseries предположение о несмешиваемости (unconfoundedness assumption) \mdseries, Рубин называет его \bfseries предположением об игнорируемости (ignorability assumption) \mdseries (Рубин, 1978, Вулдридж, 2001). Его введение приводит к тому, что при включении $x$ в регрессию в модели нет смещения из-за пропущенных переменных, и, следовательно, нет смешивания. Это предположение эквивалентно тому, что назначение воздействия игнорирует результаты, следовательно, на него можно ссылаться и как на предположение об игнорируемости. 

Это предположение необходимо, если переменная воздействия рассматривается как экзогенная, что очень важно для простоты оценивания. Иначе нужно использовать модели с самоотбором выборки или методы работы с инструментальными переменными для учета эндогенных переменных воздействия, эти методы разобраны в разделе 25.4. 

\subsection{Предположение о пересечении}

Второе предположение, называемое \bfseries предположением о пересечении \mdseries или \bfseries о сопоставлении \mdseries (overlap, matching), необходимо для идентификации некоторых мер воздействия. Согласно этому предположению

\begin{equation}
\label{eq25.6}
0 < \Pr [D = 1| x] < 1, 
\end{equation}
Это предположение гарантирует, что для каждого значения $x$ мы наблюдаем как подвергшиеся воздействию случаи, так и не подвергшиеся. В этом смысле у нас есть пересечение подвергшихся и не подвергшихся воздействию выборок. Каждому подвергшемуся воздействию индивиду сопоставляется неподвергшийся с таким же $x$. Если это предположение не выполнено, мы можем иметь индивидов с одинаковым вектором $x$, целиком подвергнутых воздействию, и других, с другим $x$, --- целиком не подвергнутых. Это требование не обязательно для идентификации параметра воздействия только для подвергнутой воздействию группы. В этом случае для идентификации эффекта воздействия для случайно отобранного индивида, нам требуется аналогичный ему индивид, не подвергавшийся воздействию. Тогда достаточно условия $\Pr [D = 1|x] < 1$. 

\subsection{Предположение об условном среднем}

Третье предположение --- это \bfseries предположение о независимости условного среднего: \mdseries

\begin{equation}
\label{eq25.7}
\E[y_0 | D = 1, x] = \E [y_0 | D = 0, x] = \E [y_0|x], 
\end{equation}
согласно которому $y_0$ не влияет на участие. 

\subsection{Меры склонности}

Когда индивиды, подвергающиеся воздействию, выбираются не случайно, а в зависимости от вектора наблюдаемых характеристик $x$, как это бывает при использовании данных наблюдения или когда воздействие назначается индивидам в зависимости от их наблюдаемых характеристик (таких как возраст, пол или социоэкономический статус), очень полезным оказывается использование \bfseries мер склонности. \mdseries Эта мера условной вероятности участия в программе при заданном $x$ обозначается $p(x)$, где

\begin{equation}
\label{eq25.8}
p(x) = \Pr [D = 1| X = x].
\end{equation}
Мера склонности может быть рассчитана по данным $(D_i, x_i)$ при помощи параметрических или полупараметрических методов, рассмотренных в Главе 14 (к примеру, при помощи логит-регрессии). 

Очень важную роль в оценке воздействия играет предположение, называемое \bfseries балансирующим условием, \mdseries согласно которому

\begin{equation}
\label{eq25.9}
D \perp x | p (x).
\end{equation}
Иначе говоря, для индивидов с одинаковой мерой склонности назначение на воздействие случайно и не зависит от вектора $x$. Выполнение балансирующего условия  можно тестировать. 

\begin{table}[h!]
\caption{\label{tab: } Эффекты воздействия --- обозначения}
\begin{center}
\begin{tabular}{ll}
\hline
\hline
Обозначение & Определение \\
\hline
$y_1$ & Результат группы, подвергнутой воздействию (экспериментальной) \\
$y_0$ & Результат группы, не подвергнутой воздействию (контрольной) \\
$p(x)$ & Мера склонности \\
$N_T$ & Число подвергнутых воздействию индивидов в выборке \\
\hline
\hline
\end{tabular}
\end{center}
\end{table}

Есть очень полезный результат, касающийся условной независимости $p(x)$, полученный в работе Розенбаума и Рубина (1983):

\begin{equation}
\label{eq25.10}
y_0, y_1 \perp D | x \Rightarrow y_0, y_1 \perp D | p(x).
\end{equation}
То есть, выполнение предположения об условной независимости при заданном $x$ приводит к условной независимости при заданном $p(x)$, то есть, независимости $y_0, y_1$ и $D$ при заданном $p(x)$. 

Чтобы получить этот результат, заметим, что 

\begin{align}
\Pr[D=1 | y_0, y_1, p(x)] & =  \E[D | y_0, y_1, p(x)] \nonumber \\
& =  \E[ \E[D | y_0, y_1, p(x), x] | y_0, y_1, p(x)] \nonumber \\
& =  \E[ \E[D | y_0, y_1, x] | y_0, y_1, p(x)] \nonumber \\
& =  \E[ \E[D | x] | y_0, y_1, p(x)] \nonumber \\
& =  \E[ p(x) | y_0, y_1, p(x)] \nonumber \\
& = p(x). \nonumber
\end{align}
Здесь вторая и третья строки следуют из закона повторных ожиданий. Четвертая строка использует условную независимость. Интуиция за этим результатом заключается в том, что $p(x)$ --- функция от $x$ и, в определенном смысле, содержит меньше информации, чем $x$. Поэтому условная независимость при заданном $p(x)$ следует из независимости при заданных $x$. Из-за условия на $x$ мы избавляемся от корреляции между $x$ и $D$, так же как беря условие на меру склонности, мы устраняем корреляцию между $x$ и $D$. Поэтому регрессия, сходная с~(\ref{eq25.4}) имеет вид:
\begin{align}
\label{eq25.12}
y & = x' \beta + \alpha p(x) + u \\
& = x' \beta + \alpha \hat p(x) + ( u + \alpha (p(x) - \hat p(x))).
\end{align}
Во второй строке неизвестная $p(x)$ заменяется на выборочную оценку, что приводит к добавлению выборочной ошибки к регрессионной ошибке. Достоинства и недостатки этого подхода будут рассмотрены ниже. В таблице 25.1 приводятся основные обозначения. 

\section{Эффекты воздействия и смещение самоотбора}

Мы начинаем с представления двух широко используемых способов измерения эффектов воздействия --- способа, усредняющего по всем индивидам и способа, усредняющего только по подвергшимся воздействию. После этого мы обсудим роль самоотбора при изучении воздействия. Методы, представленные в разделах 25.4 --- 25.6 предполагают, что эффекты самоотбора прямо зависят только от измеряемых наблюдаемых характеристик индивида, таких как возраст. Если помимо этого зависят от ненаблюдаемых показателей, необходимо использовать методы из Главы 16. Данный раздел включает обсуждение вопросов, связанных с самоотбором. 

\subsection{Два основных параметра: ATE и ATET}

Определим $\Delta$ как разницу в результате в экспериментальной и контрольной группах:

\begin{equation}
\label{eq25.13}
\Delta = y_1 - y_0,
\end{equation}
где, при желании, можно добавить зависимость от $x$. Подчеркнем, что $\Delta$ не наблюдается явно, потому что ни один индивид на наблюдается в двух состояниях одновременно. Полученные по генеральной совокупности \bfseries средний эффект воздействия (average treatment effect, ATE)  \mdseries и \bfseries средний эффект воздействия подвергнутой воздействию группы (average treatment effect on the treated, ATET) \mdseries определяются:

\begin{equation}
\label{eq25.14}
ATE = \E[\Delta],
\end{equation}
\begin{equation}
\label{eq25.15}
ATET = \E [\Delta | D = 1],
\end{equation}
с выборочными аналогами:
\begin{equation}
\label{eq25.16}
\widehat{ATE} = \frac{1}{N} \sum_{i=1}^N [\Delta_i],
\end{equation}
\begin{equation}
\label{eq25.17}
\widehat{ATET} = \frac{1}{N_T} \sum_{i=1}^{N_T} [\Delta_i | D_i = 1],
\end{equation}
где $N_T = \sum_{i=1}^N D_i$. В каждом из этих двух случаев, вычисления очевидны, если есть возможность получить $\Delta_i$. Процедура затрудняется тем, что в формулах присутствует ненаблюдаемый компонент, который необходимо оценить, что требует введения дополнительных предположений. 

ATE релевантна, когда воздействие может быть применено к любому индивиду и гипотетически можно исследовать выигрыш от воздействия для любого случайно выбранного члена генеральной совокупности. ATET используется, когда мы хотим оценить средний выигрыш именно для тех, кого подвергли воздействию. Подробнее см. в работе Хекмана и Вытлацила (2002). 

Чтобы понять проблему оценки воздействия, рассмотрим средний выигрыш от участия при заданных характеристиках $x$. Тогда

\begin{align}
\label{eq25.18}
ATE & =  \E[ \Delta | X=x ]  \\
& =  \E[ y_1 - y_0 | X = x ] \nonumber \\
& =  \E[ y_1 | X=x ] - \E[ y_0 | X=x ] \nonumber \\
& =  \E[ y_1 | x, D = 1] - \E[ y_0 | x , D = 0 ], \nonumber
\end{align}
где последнее равенство использует предположение об условной независимости~(\ref{eq25.1}). 

Имея выборку для участников, можно оценить $\E [y_1 | D = 1, x]$. Однако, $\E[y_0 | x, D = 0]$ не наблюдается, потому что это мера для участников, которую они имели бы если бы они не участвовали, а наблюдать одновременно одних и тех индивидов как участников и неучастников невозможно. Для использования ATE мы должны найти оценку для второго члена. 

По определению~(\ref{eq25.18})

\begin{align}
\label{eq25.20}
ATE & =  \E[ y_1 | x, D = 1] - \E[ y_0 | x , D = 0 ]  \\
& =  \mu_1(x) - \mu_0(x) + \E[ u_1 | x, D = 1] - \E[ u_0 | x , D = 0 ]  \nonumber \\
& =  \mu_1(x) - \mu_0(x) + \E[ u_1 | x] - \E[ u_0 | x ] \nonumber \\
& =  \mu_1(x) - \mu_0(x),
\end{align}
где первый член в правой части уравнения на первой строчке может быть оценен при помощи данных об участниках программы, а второй член не наблюдается. Следующие три строчки получаются применением условной независимости и предположения об условном среднем при использовании спецификации $y_1 = \mu_1(x) + u_1$ для подвергающихся воздействию и $y_0 = \mu_0 (x) + u_0$ для не подвергающихся. Второй член из последней строки требует только независимости средних, а не полной условной независимости. 

\subsection{Отбор и смещение самоотбора}

Проблема описанного выше метода заключается в том, что $\E [y_0 | x, D = 1]$ не наблюдается. Решение этой проблемы частично зависит от типа доступных данных. В социальных экспериментах в качестве прокси для сравнения используются подходящие участники, не вошедшие в экспериментальную группу. Если данные получаются в процессе наблюдения, группа для сравнения получается из того же источника, что и экспериментальная группа, или из других баз данных, в итоге всё сводится к получению функции от $\E[y_0 | x, D = 0]$, которая может быть оценена при помощи данных о не-участниках. Простота вычисления для данных, полученных в ходе хорошо построенных и проведенных социальных экспериментов, противопоставляется другим проблемам, таким как \bfseries смещение выборки \mdseries и \bfseries смещение от замещения \mdseries (рассмотрено в Главе 3). 

Предположим, что для подвергающихся воздействию участников результат принимает вид:

\begin{align}
\label{eq25.22}
y_1 & =  \E[ y_1 | x]  + u_1  \\
& =  \mu_1(x) + u_1
\end{align}
а для не-участников:

\begin{align}
\label{eq25.24}
y_0 & =  \E[ y_0 | x]  + u_0  \\
& =  \mu_0(x) + u_0
\end{align}
Заметим, что эта спецификация устроена как регрессия с переключением (аналогично модели Роя, рассмотренной в разделе 16.7) в том смысле, что здесь участники и не-участники имеют разные функции условного среднего, $\mu_1(x)$ и $\mu_0(x)$, которые записываются в более общих терминах, чем требуется для чисто линейных моделей. Мы предполагаем, что $\E [u_1 | X] = \E[u_0 | X] = 0$, однако $\E [u_1 | x, D]$ и $\E [u_0 | x, D]$ не обязательно равны нулю. 

Более распространенная, но более узкая, спецификация имеет вид

\begin{equation}
\label{eq25.25}
\mu_1 (x) = \mu_0 (x) + \alpha D, 
\end{equation}
в которой для группы, подвергающейся воздействию, вводится дополнительная константа $\alpha$, но коэффициенты наклона не зависят от воздействия. 

\begin{table}[!h]
\caption{\label{tab:} Измерение эффектов воздействия: ATE и ATET}
\begin{center}
\begin{tabular}{lll}
\hline
\hline
Мера & Эффект воздействия & Частный случай (25.25) \\
\hline
ATE при заданном x & $\E [\Delta | x] = \mu_1(x) - \mu_0(x) $ & $\E[\Delta | x] = \alpha$ \\
ATET с x и эффектом отбора & $\E [\Delta | x, D = 1] $ & $\E[\Delta | x, D = 1]$ \\
& $ = \mu_1(x) - \mu_0(x)$ & $ = \alpha + \E [u_1 - u_0 | x, D = 1]$ \\
& $ + \E[u_1 - u_0 | x, D = 1]$ & \\

Дополнительный выигрыш & $\E [u_1 - u_0 | x, D = 1] $ & $\E [u_1 - u_0 | x, D = 1]$ \\
индивида с $x$ & & \\
Среднее смещение самоотбора & $\E [u_0 | x, D = 1]$ & $\E [u_0 | x, D = 1]$ \\
&  $ - \E [u_0 | x, D = 0]$ & $ - \E [u_0 | x, D = 0]$ \\
\hline
\hline
\end{tabular}
\end{center}
\end{table}

Наблюдаемый результат $y$ записывается как

\begin{equation}
\label{eq25.26}
y = D y_1 + (1-D) y_0. 
\end{equation}

Комбинируя эти уравнения, получаем

\begin{align}
\label{eq25.27}
y & = D (\mu_1 (x) + u_1) + (1-D)(\mu_0(x) + u_0) \nonumber \\
& =  \mu_0 (x) + D(\mu_1 (x) - \mu_0 (x) + u_1 - u_0) + u_0. 
\end{align}
Из-за того, что $D = 1$ или $0$, второй член в регрессии <<переключается>>. Второй член в~(\ref{eq25.27}) измеряет выгоду от участия; первая составляющая $\mu_1(x) - \mu_0(x)$ измеряет средний выигрыш для участника с характеристиками $x$ и вторая составляющая $(u_1 - u_0)$ --- это выигрыш, индивидуальный для каждого участника. Вторая компонента может наблюдаться участником, но не наблюдаться исследователем. 

Выражения для ATE и ATET приводятся в таблице 25.2, в общем случае и для спецификации~(\ref{eq25.25}). 

Среднее смещение самоотбора --- это разница между участниками программы и не-участниками в базовом состоянии. Эта разница не может быть приписана воздействию. Специальный случай --- это $\E [u_1 - u_0 | x, D = 1] = 0$, который возникает, если нет ненаблюдаемых компонент в выигрыше от участия или лучшая индивидуальная оценка $u_1 - u_0$ равна нулю. 

Смещение самоотбора возникает, когда переменная воздействия коррелирована с ошибкой в уравнении для результата. Эта корреляция может быть вызвана наличием пропущенных переменных, которые влияют на $D$ и $y$. Тогда компонента регрессионной ошибке, соответствующая пропущенной переменной, будет коррелирована  с $D$ --- случай \bfseries отбора по наблюдаемым показателям. \mdseries Другой источник смещения --- это ненаблюдаемые факторы, влияющие на $D$ и $y$. Это случай \bfseries отбора по ненаблюдаемым показателям. \mdseries Предположение об условной независимости устраняет проблемы, связанные с пропущенными переменными. 

\subsection{Отбор по наблюдаемым показателям}

При использовании данных, полученных в ходе наблюдения, проблема отбора по наблюдаемым показателям решается при помощи регрессионных методов и методов сопоставления. В следующих разделах данной главы эти методы разбираются в деталях. Прежде, чем перейти к ним, отметим, что в качестве примера здесь можно привести двухчастную модель из раздела 16.4. В этом же разделе мы рассмотрим другой очевидный метод. 

 \bfseries Оценка управляющей функции \mdseries нужна для того, чтобы учесть возможную корреляцию наблюдаемых переменных $z$, определяющих $D$, с результатами. Рассмотрим конкретный случай с уравнением для результата:

\begin{equation}
\label{eq25.28}
y_i = x'_i \beta + \alpha D_i + u_i
\end{equation}
с такой ошибкой, что 

$$
\E [u_i | x_i, D_i] = \E [u_i | x_i, D_i, z_i]. 
$$
В случае отбора по наблюдаемым показателям возможно, что $\E[u_i|z_i] \ne 0$. Тогда

\begin{equation}
\label{eq25.29}
\E [y_i | x_i, D_i, z_i] = x'_i \beta + \alpha D_i + \E [u_i | x_i, z_i],
\end{equation}
что мотивирует использование оценок управляющей функции, основанной на МНК/ОМНК оценивании уравнения. Основная идея заключается в том, что нужно добавить в уравнение для результата все наблюдаемые переменные, которые могут быть коррелированы с $u_i$ и оценить полученное уравнение. В этом случае

\begin{equation}
\label{eq25.30}
y_i = C'_i \delta + \alpha D_i + \{ u_i - \E[u_i | D_i, C_i] \},
\end{equation}
где $C_i$ включается в себя все переменные, входящие в $x$ или в $z$. Наличие $z$ в регрессии устраняет возможную корреляцию между $u$ и $z$. Заметим, что при отборе по ненаблюдаемым показателям, вызванном наличием общих ненаблюдаемых факторов, влияющих на $D$ и $u$, мы по-прежнему сталкиваемся с возможностью возникновения проблем с идентификацией. 

Эта оценка была использована в работе Хекмана и Хотца (1989), там же предложены альтернативные варианты простых оценок метода управляющих функций. 

\subsection{Отбор по ненаблюдаемым показателям}

Теперь рассмотрим частный линейный случай, где решение об участии в программе воздействия эндогенно. Это пример из известного класса моделей с <<эндогенной дамми-переменной>>. С эмпирической точки зрения модель очень важна при работе с данными, полученными в ходе наблюдений, потому что в таком случае есть несколько причин отказаться от ограничивающих предположений $y_0, y_1 \perp D | x$ или $\E[u|x, D] = 0$. Нарушение предположения об условной независимости приводит к тому, что простая регрессия методом наименьших квадратов не может идентифицировать ATE, в силу чего возникает необходимость использовать другую стратегию идентификации. 

Основные элементы в стратегии идентификации, которые мы будем обсуждать, похожи на случай моделей с самоотбором. Подход требует достаточно сильных предположений и полностью параметрический. В рассмотренном специальном случае, спецификация аналогична модели Роя. Берутся линейные условные средние в уравнениях для результирующей переменной. В модели также присутствует уравнение для решения об участия для переменной $D_i$. Тогда:

\begin{align}
\label{eq25.31}
& y_{1i} = x'_i \beta_1 + u_{1i}, \\
& y_{0i} = x'_i \beta_0 + u_{0i}, \nonumber \\
& D^*_i = z'_i \gamma + \e_i, \nonumber
\end{align}
где $D_i^*$ --- латентная переменная, такая, что

\begin{equation}
\label{eq25.32}
D_i=\begin{cases}
1, \text{ если }D_i^*>0,\\
0, \text{ если }D_i^* \le 0,
\end{cases}
\end{equation}
предполагается, что $\E [u_1 | x, z] = \E[u_0 | x, z ] = 0$. 

Часть переменных из $z$ могут присутствовать и в $x$, но предполагается, что по меньшей мере один компонент $z$, обозначенный $z_1$, уникален. То есть, существует по меньшей мере одна независимая переменная, определяющая $D$. Поэтому можно рассматривать $z_1$ как инструментальную переменную, коррелированную с эндогенной переменной $D$, и связанную с результирующими переменными $y_0$ и $y_1$ только через $D$. 

Далее, предполагается, что $(u_{1i}, u_{0i}, \e_i)$ имеют совместное нормальное распределение с нулевым средним и ковариационной матрицей $\Sigma$:

\begin{equation}
\label{eq25.33}
\Sigma = \begin{bmatrix}
\sigma_{11} & \sigma_{10} & \sigma_{1 \e} \\
\sigma_{10} & \sigma_{00} & \sigma_{0 \e} \\
\sigma_{1 \e} & \sigma_{0 \e} & 1 \\
\end{bmatrix}.
\end{equation}
Ненулевые ковариации $\sigma_{1 \e}$ и $\sigma_{0 \e}$ отражают эндогенность переменной воздействия. Параметр $\sigma_{10}$ отражает ковариацию между результатами. Из-за того, что мы не можем наблюдать одного индивида в обоих состояний, этот параметр не может быть оценен и обычно считается равным нулю. Дисперсия $\e$ принимается за 1 для удобства оценивания. 

При использовании такой полностью параметрической спецификации, модель может быть оценена методом максимального правдоподобия или двухшаговой полупараметрической процедурой. Эти вопросы были рассмотрены в Главе 16. Оставляя в стороне нюансы оценивания, мы сосредоточимся на измерении эффекта воздействия. 

Выгода от участия, ATET, задается следующим образом:

\begin{equation}
\label{eq25.34}
y_{1i} - \E [y_{0i} | D_i = 1] = y_{1i} - x'_i \beta_0 + \sigma_{0 \e} \frac{\phi(z'_i \gamma)}{1 - \Phi (z'_i \gamma)}.
\end{equation}
Это можно переписать в форме:

\begin{equation}
\label{eq25.35}
\E [y_{1i} | D_i = 1] - \E [y_{0i} | D_i = 1] = x'_i (\beta_1 - \beta_0) + (\sigma_{0 \e} - \sigma_{1 \e}) \frac{\phi(z'_i \gamma)}{\Phi (z'_i \gamma)},
\end{equation}
где $(\alpha_{0 \e} - \alpha_{1 \e}) \phi(z'_i \gamma)/\Phi (z'_i \gamma)$ --- эффект отбора, см. раздел 16.7.1. 

В частном случае, где $x'_i \beta_0 = x'_i \beta_1$ и дамми на воздействие входит в уравнение для $y_1$ линейно  с коэффициентом $\alpha$, средний эффект программы равен

\begin{equation}
\label{eq25.36}
\E [y_{i} | D_i = 1] - \E [y_{i} | D_i = 0] = \alpha + \text{ эффект отбора}.
\end{equation}

Для некоторых выборок такая стратегия оценки может быть неустойчивой. К примеру, экспериментальная и контрольная группы могут быть слишком разными, предположение о многомерной нормальности может не выполняться, или инструментальная переменная $z_1$ может быть слабой или коррелированной с ошибкой модели для результата. 

Эти соображения приводят к необходимости использовать альтернативные процедуры оценивания, представленные в этой главе. Они, как правило, предполагают отбор только по наблюдаемым переменным, хотя в разделе 25.7 представлены методы с инструментальными переменными, применимые и при отборе по ненаблюдаемым показателям. 

\section{Оценки при помощи сопоставления и мер склонности}

Когда данные получены в процессе наблюдения, возможности влиять на выбор участников, как при проведении эксперимента, нет. Поэтому нет возможности рассчитать аналог ATE как среднюю разницу между результатами в двух группах, нет базы для сравнения. Её можно заменить на данные, полученные, возможно, из другого источника, но где есть индивиды со сходными наблюдаемыми характеристиками, $x$, не подвергавшиеся воздействию, и сопоставить с имеющимися индивидами, подвергавшимися изучаемому воздействию. 

Средний результат для контрольной группы можно использовать как средний результат для экспериментальной группы в гипотетической ситуации, где она бы не подвергаются воздействию. Этот подход разрешает проблему с оцениванием путем ввода дополнительного предположения: что отбор не зависит от результата для не-участников при условии на $x$. Чтобы применять его на практике, необходимо определить критерии сопоставления. 

\subsection{Предположения для оценки эффектов воздействия}

Оценка эффектов воздействия при помощи сопоставления может использоваться, когда отбор осуществляется только на основе наблюдаемых показателей. Помимо этого, предполагается выполнение \bfseries условия пересечения \mdseries ~(\ref{eq25.6}), согласно которому для каждого заданного $x$ есть положительная вероятность не-участия. Это гарантирует, что для каждого участника с характеристиками $x$ мы можем найти не-участника с такими же характеристиками. Грубо говоря, контрольная и экспериментальная группы имеют сравнимые наблюдаемые характеристики. Для получения хороших сопоставлений требуется выполнение условия пересечения. Помимо этого, ключевую роль играет то, что ненаблюдаемые характеристики не влияют на назначение воздействия и результат. 

При оценке регрессии происходит импутация пропущенных потенциальных результатов. Если $D_i = 1$, $y_{0,i}$ определяются при помощи оценки условной регрессии $\hat \mu_0 (x_i)$. Оценки сопоставления определяет пропущенные значения по методу <<ближайших соседей>>, которые определяются при помощи подходящей метрики на основе наблюдаемых характеристик. Это дает определенные основания для проведения аналогий между оценкой сопоставления и непараметрическими методами, основанными на ближайших соседях. Оценка сопоставления как правило оценивает разницу средних, а дисперсия оценки вычисляется одним из способов вычисления дисперсии разности. 

Сопоставление является очень привлекательным методом, если: $(1)$ нам доступен большой набор переменных $x$, $(2)$ есть много потенциальных регрессоров и $(3)$ предметом изучения является ATET. Для оцениванивания также требуется предположение об отсутствии эффектов общего равновесия, или \bfseries предположения о стабильности величины воздействия (stable unit treatment value assumption, SUTVA), \mdseries согласно которому воздействие не имеет непрямых эффектов на наблюдения из контрольной группы. Оценка сопоставления, однако, не требует предположения, что эффект воздействия входит в условное среднее линейно. Первый шаг заключается в поиске ближайших соответствий для каждого наблюдения, и он же позволяет понять, доступно ли для сравнения достаточно контрольных наблюдений. В отличие от регрессионного подхода, здесь меньше опасности экстраполировать  за пределами диапазона значений данных. 

Предположим, что всем наблюдениям из экспериментальной группы нашлись сопоставления в терминах всех наблюдаемых объясняющих переменных. Тогда, в определенном смысле, мы учитываем все различия между экспериментальной и контрольной группами. Зная результаты $y_{1i}$ и $y_{0i}$ для экспериментальной и контрольной групп соответственно, можно получить средний эффект воздействия:

\begin{align}
\label{eq25.37}
\E & [y_{1i} | D_i = 1] - \E [y_{0i} | D_i = 0]  \\
& = \E [y_{1i} - y_{0i} | D_i = 1] + \{  \E [y_{0i} | D_i = 1] -  \E [y_{0i} | D_i = 0] \}. \nonumber
\end{align}
Первый член во второй строчке --- это ATET, а второй --- это <<смещение>>, он будет нулевым, если назначение в экспериментальную и контрольную группы было случайным. В таком случае всё, что требуется для получения оценки ATET --- это оценить среднюю разность результатов. 

В более реалистичном случае мы будем использовать наблюдаемые объясняющие переменные $x_i$. Предполагается, что сюда входят переменные, которые влияют на отбор в экспериментальную группу. Если контрольная и экспериментальная группы сопоставляются по каждому сочетанию объясняющих переменных, можно легко вычислить различия в результатах для каждого наблюдения из экспериментальной группы и каждого $x_i$. Средняя разность результатов по всем подвергнутым воздействию индивидам и по всем $x_i$ будет средним эффектом воздействия. Формально, в этом случае (см. Ангрист и Крюгер, 2000, с. 1316) эффект воздействия примет вид

\begin{align}
\label{eq25.38}
\E [y_{1i} - y_{0i} | D_i = 1] &= \E[ \{ \E[y_{1i} | x_i, D_i = 1] - \E[y_{0i} | x_i, D_i = 0] \} | D_i = 1 ]  \\
& = \E [\Delta_x | D_i = 1], \nonumber
\end{align}
где $\Delta_x = \E [y_{1i} | x_i, D_i = 1] - \E[y_{0i} | x_i, D_i = 0] $. 

Если переменные $x$ --- дискретные, оценка сопоставления определяется как взвешенная сумма

\begin{equation}
\label{eq25.39}
\E [y_{1i} - y_{0i} | D_i = 1] = \sum_x \Delta_x \Pr [x_i = x | D_i = 1],
\end{equation}
где $\Pr [x_i = x | D_i = 1]$ --- вероятность $x_i$ при $D_i = 1$. Ангрист и Крюгер (2000) рассматривают некоторые аспекты этой оценки. 

\subsection{Точное сопоставление}

Процедура заключается в сопоставлении индивидов из экспериментальной и контрольной выборок на основании их наблюдаемых характеристик $x$. 

\bfseries Точно сопоставление \mdseries применимо, когда вектор объясняющих переменных дискретный и выборка содержит много наблюдений для каждого отдельного значения $x$. 

Если вектор объясняющих переменных имеет большую размерность или если среди переменных есть непрерывные, точно сопоставление двух групп невозможно. Тогда применяются методы \bfseries неточного сопоставления. \mdseries Неточное сопоставления преобразует $x$ в показатель с меньшей размерностью, непрерывный или дискретный, обычно используется скаляр $f(x)$. 

\subsection{Меры склонности}

Использование мер склонности (Розенбаум и Рубин, 1983) --- популярный метод неточного сопоставления. Вместо сопоставления регрессоров он сопоставляет меры склонности. Точные сопоставления по-прежнему невозможны, поэтому для сравнения используются те объекты, которые оказались достаточно близко к исследуемому. 

\bfseries Мера склонности \mdseries --- условная вероятность получить воздействие при заданном $x$, обозначаемая $p(x)$, была предложена Розенбаум и Рубин (1983). Как отмечено в разделе 25.2.5, если данные позволяют осуществлять сопоставление на основе $x$, они позволяют также и сопоставлять на основе мер склонности. 

Меры склонности как правило оцениваются при помощи параметрических моделей, таких как логит или пробит, но, в принципе, могут быть оценены и непараметрически. 

\subsection*{Сопоставление про помощи мер склонности}

При использовании мер склонности мы учитываем влияние регрессоров, учитывая влияние определенной функции от них --- условной вероятности получить воздействие, $\Pr [ D_i = 1 | x_i ]$ . То есть, сопоставление проводится по мере склонности. Она может быть легко рассчитана с помощью, к примеру, логит-регрессии. Более того, можно учитывать влияние лагированных переменных, добавляя их в вектор объясняющих переменных. Если смещение самоотбора можно устранить учетом $x_i$, его можно устранить и учетом меры склонности. Зачастую проще брать условия на меры склонности, чем на многомерный вектор $x$. Дехеджа и Ваба (1998) приводят эмпирическую иллюстрацию, основанную на данных, использованных в работе Лалонда (1986).

\subsection*{Особенности реализации}

Использование мер склонности требует хороших моделей для оценки самих мер. Цель при этом --- получить состоятельную оценку вероятности участия, а не получить оценки параметров в функции для меры склонности. Более точные оценки меры склонности, скорее всего, будут получены при помощи гибких параметрических или непараметрических моделей. 

При сопоставлении на основе $p(x_i)$ важны три вопроса: $(1)$ сопоставлять с возвращением или без возвращения; $(2)$ сколько наблюдений использовать для сопоставления и $(3)$ какой метод сопоставления выбрать. 

Сопоставление без возвращения означает, что любое наблюдение в контрольной группе сопоставляется не более чем одному наблюдению из экспериментальной группы, с которым они наиболее близки, тогда как при сопоставлении с возвращением могут быть множественные сопоставления. При сопоставлении без возвращения, слишком маленький размер контрольной выборки может привести к тому, что сопоставления будут не очень близкими в терминах $p(x)$, что повышает смещение оценки. 

Вопрос выбора количества наблюдений в контрольной выборке для сопоставления --- это выбор между смещением и дисперсией. При использовании одного, самого близкого, сопоставления уменьшается смещение, но при увеличении числа сопоставляемых наблюдений уменьшается дисперсия, тогда как смещение растет, потому что дополнительные наблюдения хуже описывают изучаемые объекты. Частичным решением этой проблемы может быть использование определенного радиуса вокруг $p(x)$ для наблюдения из экспериментальной выборки, внутри которого наблюдения сопоставляются (а за его пределами --- нет). Другими словами, мы используем только лучшие совпадения. Это называется \bfseries <<циркульное (caliper) сопоставление>>. \mdseries

Хекман и др. (1997, 1998) изучали качество оценок сопоставления на экспериментальных данных  полученных при реализации Закона о партнерстве по подготовке кадров (Job Training Partnership Act, JTPA) в сочетании с контрольными группами, взятыми из трех источников. Качество данных играет ключевую роль в устойчивости оценок эффектов воздействия с помощью методов сопоставления. Наилучшие результаты достигаются тогда, когда данные для контрольной и экспериментальной групп устроены сопоставимо, когда обе группы находятся на одном рынке труда и когда мера склонности может быть оценена при помощи большого набора регрессоров. 

Вопрос чувствительности результатов к выбору метода не имеет простого решения. Результат может меняться для разных выборок, в зависимости от уровня пересечения между контрольной и экспериментальной выборками. Если две группы похожи в том смысле, что присутствует значительное пересечение их мер склонности, и если контрольная группа велика, сопоставления найти проще и можно использовать сопоставление с без возвращения. Если же контрольная группа невелика и отличается от экспериментальной, хорошие по качеству совпадения быстро закончатся и возможности использовать всю экспериментальную выборку не будет. Эта проблема особенно вероятна,  если используется сопоставление без возвращения. 

Дехеджа и Ваба (2002) применяют методы используя данные полученные в ходе Национальной программы поддержки занятости (National Supported Work Program) --- это хороший и поучительный пример. Мы исследуем и иллюстрируем вопросы применения описанных методов в разделе 25.8 на данных Дехеджа и Ваба. 

\subsection{Измерение эффектов воздействия}

Обозначим группу для сравнения для подвергнутого воздействию наблюдения $i$ с характеристиками $x_i$ как множество $A_j (x) = \{ j | x_j \in c(x_i) \}$, где $c(x_i)$ --- окрестность точки $x_i$. Обозначим $N_C$ --- количество наблюдений в группе для сравнения и $w(i, j)$ --- вес $j$-ого наблюдения в сопоставлении для $i$-ого наблюдения, $\sum_j w(i, j) = 1$. Тогда \bfseries  общая формула \mdseries для оценки сопоставления ATET:

\begin{equation}
\label{eq25.40}
\Delta^M = \frac{1}{N_T} \sum_{i \in \{ D=1 \}} [y_{1,i} - \sum_j w(i,j) y_{0,j}],
\end{equation}
где $0 < w(i,j) \le 1$ и $\{ D =1 \}$ --- множество индивидов из экспериментальной группы, а $j$ --- элемент из множества сопоставленных наблюдений. Разные оценки сопоставления получаются изменением весов $w(i,j)$. 

\subsection*{Методы сопоставления}

Простое сопоставление сравнивает наблюдения с одинаковыми дискретными $x$

\begin{equation}
\label{eq25.41}
\Delta^M = \sum_k w_k [ \overline{y}_{1,k} - \overline{y}_{0,k} ],
\end{equation}
где $\overline{y}_{1}$ --- средний результат для экспериментальной группы, $\overline{y}_{0}$ --- средний результат для контрольной группы и $w_k$ --- вес $k$-ого наблюдения (то есть, доля наблюдений в точке $k$). 

Частный случай (Дехеджа и Ваба, 2002):

\begin{equation}
\label{eq25.42}
\frac{1}{N_T} \sum_i \left( y_i - \frac{1}{N_{C,i}} \sum_{j \in \{ D = 0\}} y_j \right) ,
\end{equation}
где $N_T$ --- размер экспериментальной группы ($D = 1$) и $N_{C,i}$ --- размер сравнительной группы, соответствующей $i$-ому наблюдению. 

Метод \bfseries сопоставления по ближайшим соседям \mdseries  для каждого $i$ из экспериментальной группы выбирает набор $A_i (x) = \{ j | \min_j \| x_i - x_j \| \}$, где $\| \; \|$  обозначает Евклидово расстояние между векторами. Если $w(i,j) = 1$ в~(\ref{eq25.40}) при $j \in A_i (x)$ и 0 иначе, то полученная спецификация использует только одно наблюдение для построения сравнительной группы. 

Другая оценка получается при помощи \bfseries ядерного сопоставления \mdseries, где

$$
w(i,j) = \frac{K (x_j - x_i)}{\sum_{j=1}^{N_{C,i}} K (x_j - x_i)},
$$
где $K$ --- ядерная функция, рассмотренная в разделе 9.3. 

Преимущество этих методов заключается в том, что они не требуют предположений о функциональной форме уравнений для результирующей переменной при оценке ATET и могут оценивать её для отдельных значений $x$. Недостаток их заключается в том, что при большой размерности $x$ число сопоставлений может оказаться очень низким. В таких случаях есть основания использовать сопоставления, основанные на скалярной метрике. \bfseries Сопоставление при помощи мер склонности, \mdseries рассмотренное выше, является примером таких методов. 

Сопоставление по ближайшим соседям и ядерное сопоставление могут быть также определены в терминах мер склонности. К примеру, для сопоставления по ближайшим соседям можно определить множество для сопоставления как $A_i (p(x)) = \{ p_j | \min_j \| p_i - p_j\| \}$. 

\bfseries Стратифицированное \mdseries или \bfseries интервальное сопоставление \mdseries основано на разделении диапазона значений меры склонности на интервалы, так что внутри каждого интервала наблюдения из контрольной и экспериментальной групп имеют, в среднем, одинаковую меру склонности. Можно использовать те же блоки, которые были идентифицированы алгоритмом при оценке мер склонности. Затем мы вычисляем разность между средними результатами для двух групп. ATET в этом случае --- взвешенное среднее этих разностей, при этом веса определяются распределением наблюдений из экспериментальной группы по блокам. Один из недостатков этого метода заключается в том, что он не учитывает наблюдения из тех блоков, где нет наблюдений из контрольной либо из экспериментальной группы. 

Обозначим как $b$ блоки, определенные на интервалах значений меры склонности. Тогда эффект воздействия внутри $b$-ого блока определяется как

$$
ATET_b^S = (N_b^T)^{-1} \sum_{i \in I(b)} Y_{1i} - (N_b^C)^{-1} \sum_{j \in I(b)} Y_{0j},
$$
где $I(b)$ --- множество наблюдений из блока $b$, $N_b^T$ --- количество наблюдений из экспериментальной группы в блоке $b$, $N_b^C$ --- количество наблюдений из контрольной группы в блоке $b$. Тогда эффект воздействия на основе стратификации определяется как

\begin{equation}
\label{eq25.43}
ATET^S = \sum_{b=1}^B ATET_b^S \times \left[ \sum_{i \in I(b)} D_i / \sum D_i \right], 
\end{equation}
где член в квадратных скобках --- это вес каждого блока, определяемый как соответствующая ему доля наблюдений из экспериментальной группы, и $B$ --- общее число блоков. 

При \bfseries радиальном сопоставлении \mdseries множество $A_i (p(x)) = \{ p_j \bigl| \; \|p_i - p_j \| < r \}$ получается на основе мер склонности. Это означает, что все наблюдения из контрольной выборки с мерами склонности внутри радиуса $r$ сопоставляются $i$-ому наблюдению из экспериментальной группы. 

Можно выразить ATE и ATET в терминах $p(x)$, предполагая условие пересечения $0 < p(x) < 1$. Тогда

\begin{equation}
\label{eq25.44}
ATE = \E \left[  \frac{(D - p(x)) y}{p(x)(1 - p(x))}  \right],
\end{equation}
\begin{equation}
\label{eq25.45}
ATET = \E \left[  \frac{(D - p(x)) y}{\Pr[D = 1](1 - p(x))}  \right];
\end{equation}

последний результат получен Дехеджа (1997).

Из этих результатов получаются следующие выводы:
\begin{align}
\label{eq25.46}
y & = (1-D)y_0 + Dy_1  \nonumber \\
& = y_0 + D(y_1 - y_0), \nonumber \\
(D - p(x))y & = (D - p(x))(y_0 + D(y_1 - y_0))  \nonumber \\
& = Dy_1 - p(x) y_0 - D p(x) y_1 + D p(x) y_0 \nonumber \\
& = Dy_1 - p(x) (1-D) y_0 - D p(x) y_1.
\end{align}

Далее, беря математические ожидания и заметив, что $\E[D|x] = p(x)$ получаем

\begin{align}
\label{eq25.47}
\E[(D - p(x)) y | x] & = p(x) \E[y_1] - p(x)(1 - p(x)) \E[y_0] - p^2(x) \E[y_1]  \\
& = p(x)\E[y_1 - p(x)y_1] - p(x)(1-p(x))\E[y_0] \nonumber \\
& = p(x)(1-p(x)) \E[y_1 - y_0],  \nonumber 
\end{align}
откуда следует, что 

$$
ATE = \E[y_1 - y_0] = \E \left[ \frac{(D - p(x)) y}{ p(x) (1 - p(x))} \right].
$$

Чтобы получить результат Дехеджа:

\begin{align}
\label{eq25.48}
\E \left[ \frac{(D - p(x)) y}{1 - p(x)} \right] & = \E[p(x) \E [\mu_1(x) - \mu_0 (x)]]  \\
& = \E[D (y_1 - y_0)] \nonumber \\
& = \E[D (y_1 - y_0) | D = 1] \; \Pr [D = 1],  \nonumber
\end{align}
где первая строка следует из~(\ref{eq25.47}), вторая получается при применении предположения об условной независимости, а последняя выражает совместное математическое ожидание как произведение частного и условного ожиданий, что приводит к

$$
ATET = \frac{\E [D (y_1 - y_0)]}{\Pr [D = 1]}.
$$

Используя~(\ref{eq25.44}) и~(\ref{eq25.45}), можно получить состоятельные оценки на основе выборки размера $N$:

\begin{equation}
\label{eq25.49}
\widehat{ATE} = \frac1N \sum_{i=1}^N \left[  \frac{(D_i - \hat p(x_i)) y_i}{\hat p(x_i)(1 - \hat p(x_i))}  \right],
\end{equation}
\begin{equation}
\label{eq25.50}
\widehat{ATET} = \left(  \frac1N \sum_{i=1}^N D_i  \right)^{-1}  \sum_{i=1}^N \left[ \frac1N \frac{(D_i - \hat p(x_i)) y_i}{(1 - \hat p(x_i))}  \right],
\end{equation}
где $N^{-1} \sum_{i=1}^N D_i$ --- состоятельная оценка $\Pr[D=1]$, 

\subsection{Дисперсия ATET на основе $x$ и $p(x)$}

При предположениях идентифицируемости из раздела 25.2, $\widehat \Delta_x$ и $\widehat \Delta_{p(x)}$ определяются как 

\begin{align}
\widehat \Delta_x & =\frac1{N_T} \sum [y_{1i} - \widehat \E [y_0 | D = 0, x = x_i]] \nonumber \\
& = \frac1{N_T} \sum_{i \in \{ D = 1 \}} [y_{1i} - \sum_{j \in A_i(x)} w_{ij} y_{0,j}]  \nonumber 
\end{align}
и 
\begin{align}
\widehat \Delta_{p(x)} & =\frac1{N_T} \sum [y_{1i} - \widehat \E [y_0 | D = 0, p(x) = p(x_i)]] \nonumber \\
& = \frac1{N_T} \sum_{i \in \{ D = 1 \}} [y_{1i} - \sum_{j \in A_i(p(x))} w_{ij} y_{0,j}],  \nonumber 
\end{align}
где $i$ --- индекс экспериментальной группы, $w_{ij} - 1/N_{c,i}$ и $N_{c,i}$ --- число наблюдений в сравнительной группе для $i$-ого наблюдения. Обе --- состоятельные оценки ATET, $\E [y_1 - y_0 | D = 1 , x]$, первая основана на $x$, вторая --- на $p(x)$. Вопрос, встающий на практике, --- что лучше с точки зрения эффективности, учет различий в мере склонности или в $x$. Хан (1998), Хекман и др. (1998) и другие показали, что нельзя точно сравнить эти две оценки в терминах асимптотической дисперсии, даже если мы предположим, что $p(x_i)$ известны, что на практике невозможно при использовании данных, полученных в ходе наблюдения. 

Запишем асимптотические дисперсии для двух случаев следующим образом:
\begin{align}
\V[\widehat \Delta_x] & = \E [\V[y_1 | D = 1, x] | D = 1] + \V [\E[y_1 - y_0 | D = 1, x] | D = 1], \nonumber \\
\V[\widehat \Delta_{p(x)}] & = \E [\V[y_1 | D = 1, p(x)] | D = 1] + \V [\E[y_1 - y_0 | D = 1, p(x)] | D = 1], \nonumber 
\end{align}
где мы используем декомпозицию дисперсии, приведенную в разделе A.8. В общем случае $x$ --- лучший предиктор, чем $p(x)$, поэтому
\begin{align}
\E[\V[y_1 | D = 1, x] | D = 1] & \le \E [\V[y_1 | D = 1, p(x)] | D = 1] , \nonumber \\
\V [\E[y_1 - y_0 | D = 1, x] | D = 1] & \ge  \V [\E[y_1 - y_0 | D = 1, p(x)] | D = 1], \nonumber 
\end{align}
потому что при фиксированном $x$ происходит меньшая потеря информации, что при фиксировании $p(x)$, который является функцией от $x$. Поэтому второе сравнение говорит в пользу использования мер склонности, тогда как первое --- в пользу $x$. 

Полезные с практической точки зрения разъяснения и  компьютерные программы для вычисления ATET приведены Бекером и Ичино (2002).

\section{Оценка методом разность разностей}

В главах 2 и 3 обсуждались концепции \bfseries естественного эксперимента \mdseries и \bfseries квазиэксперимента, \mdseries в которых переменная воздействия претерпевает изменение, которое может рассматриваться как экзогенная изменчивость переменной воздействия. Экспериментальная группа может сравниваться с не подвергнутой воздействию контрольной группой. 

В некоторых случаях исследователь располагает данными об экспериментальной и контрольной группах как до, так и после эксперимента. Тогда для $i$-ого наблюдения из экспериментальной группы изменение в результате оценивается как $[y_{ia} - y_{ib} | D_{ia} = 1]$ и для контрольной группы $[y_{ia} - y_{ib} | D_{ia} = 0]$. Тогда оценка {\it{разности разностей}} равна $[y_{ia} - y_{ib} | D_{ia} = 1] - [y_{ia} - y_{ib} | D_{ia} = 0]$, где нижние индексы $a$ и $b$ обозначают состояния после (англ. after) и до (англ. before) эксперимента. Она формирует базу для оценки эффекта воздействия. Этот метод был представлен в разделах 3.4.2 и 22.6. 

Рассмотрим модель с фиксированным эффектом $\phi_i$ и сдвигом $\delta_t$, где результаты до и после воздействия задаются, соответственно:

\begin{equation}
\label{eq25.51}
y_{it,0}= \phi_i + \delta_t + \e_{it}, 
\end{equation}
\begin{equation}
\label{eq25.52}
y_{it,1}= y_{it,0} + \alpha, 
\end{equation}
так, что 
\begin{align}
\label{eq25.53}
y_{it} & = (1 - D_{it}) y_{it,0} + D_{it} y_{it, 1} \\
&  =  \phi_i + \delta_t + \alpha D_{it} + \e_{it}. \nonumber 
\end{align}

В этих уравнениях $t = a, b$; ~(\ref{eq25.51}) --- для группы, не подвергшейся воздейстию и~(\ref{eq25.52}) для группы, подвергшейся воздействию. При использовании формулировок <<до>> и <<после>>, мы получаем эффект воздействия:

\begin{align}
\label{eq25.54}
\alpha & = \E [y_{ia} - y_{ib} | D_{ia} = 1] - \E [y_{ia} - y_{ib} | D_{ia} = 0] \\
&  =  \{ \E [y_{ia} | D_{ia} = 1] - \E [y_{ia}| D_{ia} = 0] \} \nonumber  \\
& \quad -  \{ \E[y_{ib} | D_{ia} = 1] - \E [y_{ib}| D_{ia} = 0] \}, \nonumber 
\end{align}
где после взятия разностей исключаются фиксированный эффект $\alpha$ и сдвиг $\delta_t$. 

Есть альтернативы взятию разностей. Одна из них --- это явно учитывать различие в результатах до воздействия для двух групп при помощи регрессии. 

К примеру, заменяя $\phi_i$ в~(\ref{eq25.51}) на $x_i' \beta + \gamma y_{ib}$, получаем
\begin{align}
\label{eq25.55}
& y_{ia,0} = x_i' \beta + \gamma y_{ib} + \delta_a + \e_{ia} \\
& y_{ia,1} = x_i' \beta + \gamma y_{ib} + \delta_a + \alpha D_{ia} \e_{ia}. \nonumber 
\end{align}
Оценки $\alpha$ получаются регрессированием результатов после воздействия на результаты до воздействия, $x_i$ и $D_{ia}$. Интерпретация $\alpha$ как параметра причинности основана на предположении, что после учета $x$ и $y_b$ разница в результате после воздействия для двух групп полностью обусловлена эффектом воздействия. Фиксированный эффект задается в линейной форме, тогда как стратегия сопоставления может основываться на более слабых предположениях. 

Наши предыдущие результаты могли быть основаны на квазиэкспериментальных данных. К примеру, можно сравнивать людей в одном штате с одним законом с людьми в другом штате с другими законами, тогда приходится учитывать влияние штата. Новый элемент --- это добавление данных перед экспериментом. При предположении, что в обоих штатах одинаковый параметр сдвига, можно использовать оценки разности разностей для устранения эффектов штата, которые иначе приходилось бы учитывать. 

\section{Разрывный дизайн}

Идентификация эффекта воздействия иногда может быть облегчена естественным экспериментом или при использовании данных, полученных в квазиэкспериментальных условиях. Разрывный дизайн (Regression discontinuity design) --- пример квазиэкспериментального дизайна, где вероятность получения воздействия --- разрывная функция от одной или более переменных. Такое устройство данных возникает в условиях, когда воздействие запускается по административному или организационному правилу. К примеру, Ангрист и Лейви (1999) изучают эффект размера класса на успеваемость студентов, используя данные, полученные во время действия правила Маймонида, согласно которому класс должен быть разделен при достижении определенного порогового размера. Ван дер Клаув (2003) оценивает эффект финансовой помощи на решение студентов об обучении в колледже, используя информацию об административном правиле, согласно которому финансовая помощь увязывается с результатами студента --- оценкой SAT и средним баллом. Этим эконометрическим приложениям предшествовала работа Тислтуэйт и Кемпбелла (1960), исследовавшая влияние студенческих стипендий на карьерные устремления, использовавшая тот факт, что поощрения назначались только, если результаты студента превосходили определенный порог; см. также Трочим (1984). Обсуждение здесь следует  работе Ван дер Клаува (2003).

\subsection{Разрывный механизм назначения воздействия}

При разрывном дизайне, у нас есть дополнительная информация о правиле отбора. Известно, что назначение воздействия зависит, по меньшей мере частично, от величины наблюдаемой непрерывной переменной по сравнению с пороговым значением так, что соответствующая вероятность получить воздействие, мера склонности --- разрывная функция от этой переменной с разрывом в пороговой точке. График 25.1 иллюстрирует выборку, сгенерированную разрывным механизмом. 

\vspace{3cm}
График 25.1

Regression Discontinuity Example --- Пример разрывной регрессии

Selection variable S --- Переменная отбора S

Actual data --- Фактические данные

Outcome y --- Зависимая переменная y

No treat (low) --- Нет воздействия (низкие значения)

Treat (high) --- Есть воздействие (высокие значения)


При простейшем виде разрывного дизайна, называемом \bfseries четкий разрывный дизайн, \mdseries индивиды назначаются в экспериментальную и контрольную группы только на основании наблюдаемой непрерывной переменной $S$, называемой переменной отбора или переменной назначения. Те, кто находится ниже порогового значения $\overline{S}$ не получают воздействие и входят в контрольную группу, те же, кто находится выше порога, получают воздействие ($D = 1$). То есть, назначение  воздействия осуществляется на основе известного правила: $D_i = \boldsymbol{1} [S_i \ge \overline{S}]$. На графике 25.2 четкий разрывный дизайн показан сплошной линией (см. Ван дер Клаув, 2003).

При четком разрывном дизайне

\begin{equation}
\label{eq25.56}
\E[u | D, S] = \E [u | S],
\end{equation}
где $u$ --- ошибка в уравнении для результата. Из-за того, что $S$ --- единственная переменная, систематически влияющая на $D$, $S$ будет учитывать любую корреляцию между $D$ и $u$. 

При $D_i = D(S_i) = \boldsymbol{1} [S_i \ge \overline{S}]$, из-за зависимости между $D_i$ и $u_i$ МНК оценка $\alpha$ будет несостоятельной. Как было замечено выше, один из подходов к оценке эффектов воздействия в этом случае заключается в спецификации и включении условного среднего $\E[u|D,S]$ как <<регрессора>> в уравнение для результата. Тогда

\begin{equation}
\label{eq25.57}
y_i = \beta + \alpha D_i + k (S_i) + \e_i ,
\end{equation}
где $\e_i = y_i - \E [y_i | D_i, S_i ] $. Если $k (S)$ правильно специфицирована регрессия дает состоятельные оценки $\alpha$. 

Если $k(S)$ линейна, $\alpha$ будет оцениваться как расстояние между двумя параллельными  регрессионными прямыми в пороговой точке, в данном случае --- как разность постоянных членов. Это несмещенная оценка общего эффекта воздействия для линейной  функции $k(S)$. 

В более общем случае для изменяющихся эффектов воздействия, когда коэффициент при $D$ есть $\E[\alpha_i | \overline{S}]$, или локальных ATE (local ATE, LATE), рассмотренных в разделе 25.7.1, функция $k(S)$ соответствует величине $\E[u|S] + (\E[\alpha_i | S] - \E[\alpha_i | \overline{S} ]) \boldsymbol{1} [S_i \ge \overline{S}]$, где $\boldsymbol{1} [S_i \ge \overline{S}] = 1$, если условие в скобках выполнено. Неправильная спецификация $k(S)$ приводит к несостоятельности, поэтому имеет смысл попробовать полупараметрическую спецификацию, к примеру, $k(S) = \sum_{j=1}^J \eta_j S^j$, где $J$ может быть определена подходящим методом. 

Переменная $S$ может быть связана с результатом $y$, что автоматически приведет к связи $(y, D)$, даже если нет никакой причинной связи между этими двумя показателями. В противоположность этому случаю, при случайном назначении такой связи удается избежать. 

Тогда как случайное назначение делает экспериментальную и контрольную группы схожими по всем показателям, кроме получения воздействия, четкий разрывный дизайн делает их разными, по крайней мере с точки зрения переменной $S$. Это нарушает предположение о \bfseries <<строгой игнорируемости>> \mdseries Розенбаума и Рубина (1983), которое также требует предположения о пересечении, $0 < \Pr [D = 1 | S] < 1$, тогда как в моделях с четким разрывным дизайном $\Pr [D = 1|S] \in [0,1]$. 

\subsection{Идентификация и оценка при четком разрывном дизайне}

Интуитивно понятно, что набор индивидов из маленькой окрестности пороговой точки будет похож на случайный эксперимент в пороговой точке, потому что у них практически одинаковые значения $S$. Индивиды, находящиеся чуть ниже порога, будут очень похожи на тех, кто расположился чуть выше. Сравнение средних значений $y$ для этих двух групп даст оценку среднего эффекта воздействия. 

Увеличение интервала вокруг порога приведет к смещению оценки эффекта воздействия, особенно если сама переменная назначения связана с результирующей переменной при фиксированном статусе воздействия. Если можно сделать предположение о функциональной форме этой связи, можно <<использовать больше наблюдений и экстраполировать вниз и вверх от пороговой точки и получить результат, который показал бы рандомизированный эксперимент. Эта двойная экстраполяция, в сочетании с использованием <<рандомизированного эксперимента>> вокруг пороговой точки --- основная идея разрывного анализа>> (Ван дер Клаув, 2003, стр. 1258).

Заметим, что при таком разрывном дизайне

\begin{equation}
\label{eq25.58}
\lim_{S \downarrow \overline{S}} \E[y|S] - \lim_{S \uparrow \overline{S}} \E[y|S] = \alpha + \lim_{S \downarrow \overline{S}} \E[u|S] - \lim_{S \uparrow \overline{S}} \E[u|S].
\end{equation}

Более формальный способ предположить, что в отсутствии воздействия, индивиды в малой окрестности $\overline{S}$ будут иметь схожие средние результаты, выглядит следующим образом:

\bfseries Предположение A1. \mdseries Условное среднее $\E [u|S]$ непрерывно в точке $\overline{S}$. 

\bfseries Предположение A2. \mdseries Средний эффект воздействия $\E [\alpha_i|S]$ непрерывен справа в точке $\overline{S}$:

\begin{equation}
\label{eq25.59}
y_i = \beta + \alpha D_i + k(S_i) + \e_i ,
\end{equation}
где $\e_i = y_i - \E [y_i | D_i , S_i ]$. 

Тогда получается результат из~(\ref{eq25.58}). 

\subsection{Нечеткий разрывный дизайн}

Здесь назначение воздействия зависит от переменной отбора стохастически. Известно, что мера склонности $\Pr[D = 1|S]$ имеет разрыв в точке $\overline{S}$. Возможным последствием неточного назначения рядом с пороговой точкой является нечеткий разрывный дизайн, когда значения $S$ вокруг порога попадают как в экспериментальную, так и в контрольную группы. В качестве альтернативы, назначение может быть основано на дополнительных переменных, наблюдаемых организатором, назначающим воздействие, но ненаблюдаемых исследователем. Тогда, в отличие от четкого разрывного дизайна, отбор при \bfseries нечетком разрывном дизайне (fuzzy RD design) \mdseries зависит как от наблюдаемых, так и от ненаблюдаемых переменных. На графике 25.2 нечеткий разрывный дизайн показан пунктирной линией. 

Мы по-прежнему можем использовать разрыв в правиле отбора для идентификации эффекта воздействия при предположении A1. Если $\E [u|S]$ непрерывно в $\overline{S}$, то $\lim_{S \downarrow \overline{S}} \E[y|S] - \lim_{S \uparrow \overline{S}} \E[y|S] = \alpha [\lim_{S \downarrow \overline{S}} \E[D|S] - \lim_{S \uparrow \overline{S}} \E[D|S] ]$. Следовательно, эффект воздействия $\alpha$ может быть оценен как

\begin{equation}
\label{eq25.60}
\frac{\lim_{S \downarrow \overline{S}} \E[y|S] - \lim_{S \uparrow \overline{S}} \E[y|S]}{\lim_{S \downarrow \overline{S}} \E[D|S] - \lim_{S \uparrow \overline{S}} \E[D|S]},
\end{equation}
где знаменатель $\lim_{S \downarrow \overline{S}} \E[D|S] - \lim_{S \uparrow \overline{S}} \E[D|S] \ne 0$ из-за разрыва $\E[D|S]$ в точке $\overline{S}$. 

В случае \bfseries гетерогенных эффектов воздействия \mdseries нам требуются дополнительные предположения

\bfseries Предположение A2*. \mdseries Средний эффект воздействия $\E [\alpha_i | S]$ непрерывен в точке $\overline{S}$.

\bfseries Предположение A3. \mdseries $D_i$ не зависит от $\alpha_i$ при фиксированном $S$ в окрестности $\overline{S}$:

\begin{equation}
\label{eq25.61}
y_i = \beta + \alpha \E[D_i|S_i] + k (S_i) + \epsilon_i,
\end{equation}
где $\e_i = y_i - \E [y_i | D_i , S_i ] $ и $k(S_i)$ --- спецификация $\E [u_i | S_i ]$. 

\subsection{Двухшаговая оценка}

Если $\Cov [D,u] \ne 0$, МНК оценка $\alpha$ будет смещенной. Однако, можно получить состоятельную оценку. Рассмотрим

\begin{equation}
\label{eq25.62}
y_i = \beta + \alpha \E[D_i|S_i] + k (S_i) + \e_i,
\end{equation}
где $\e_i = y_i - \E [y_i | D_i , S_i ] $ и $k(S_i)$ --- спецификация $\E [u_i | S_i ]$. 

Шаг 1: Специфицируем функциональную форму мер склонности для нечеткого разрывного дизайна как

\begin{equation}
\label{eq25.63}
\E [D_i | S_i] = f (S_i) + \gamma \boldsymbol{1} [S_i \ge \overline{S}],
\end{equation}
где $f (S_i)$ --- некая непрерывная функция от $S$, которая непрерывна в точке $\overline{S}$. Специфицируя функциональную форму $f$ (или оценивая $f$ полу- или непараметрически) можно оценить $\gamma$, величину разрыва мер функции меры склонности в точке $\overline{S}$. 

Шаг 2: Оцениваем уравнение результата с управляющей функцией и $D_i$ замененным на полученную на первом шаге оценку $\E [D_i | S_i ] = \Pr [D_i = 1 | S_i]$. Эта оценка будет разрывной по $S$, тогда как управляющая функция для $k(S)$ будет непрерывной по $S$ в $\overline{S}$. При правильной спецификации $f(S_i)$ и $k(S_i)$ двухшаговая процедура будет состоятельной. 

\vspace{3cm}
График 25.2

Sharp and Fuzzy RD Designs -- Четкий и нечеткий разрывный дизайн

Sharp Design -- четкий дизайн

Fuzzy design -- нечеткий дизайн

Propensity score $\Pr[D=1|S]$ -- мера склонности $\Pr[D = 1|S]$

Selection variable S --- Переменная отбора S

Рисунок 25.2. Дизайн разрывной регрессии. Четкий дизайн (сплошная линия) и нечеткий дизайн (пунктирная линия) назначения воздействия 


\section{Метод инструментальных переменных}

В последние годы методы инструментальных переменных набирают популярность как альтернативы ММП и другим строго параметрическим методам (Ангрист, Имбенс, и Рубин, 1996). Они выглядят очень привлекательно в моделях с \bfseries отбором по ненаблюдаемым переменным \mdseries (см. раздел 25.3.4). Во многих случаях такая модель состоит из линейного уравнения для непрерывной результирующей переменной, для которой специфицированы условное среднее и структура дисперсии, без дополнительных предположений о форме распределения. В самом частом случае рассматривается непрерывная результирующая переменная, вектор регрессоров $x$ и одна эндогенная дамми-переменная воздействия ($D$), которая отвечает за решение об участии в программе воздействия. Это уравнение называется уравнением участия или уравнением отбора. В более общем случае, можно рассматривать ограниченный или дискретный результат и множество переменных воздействия. 

Обсуждение ниже перекликается с обсуждением моделей отбора и инструментальных переменных в других частях книги. Подход инструментальных переменных дает нам возможность рассматривать другой, <<локальный>>, вариант параметра ATE. 

\subsection{Локальный ATE (LATE)}

Рассмотрим простую линейную модель. Результирующая переменная --- линейная функция от наблюдаемых переменных $x$ и индикатора участия $D$:

\begin{equation}
\label{eq25.64}
y_i = x'_i \beta + \alpha D_i + u_i,
\end{equation}
и решение об участии зависит от единственной переменной $z$, которая является инструментом:

\begin{equation}
\label{eq25.65}
D_i^* = \gamma_0 + \gamma_1 z_i + v_i,
\end{equation}
где $D_i^*$  --- латентная переменная для наблюдаемой переменной $D_i$, задаваемой как

\begin{equation}
\label{eq25.66}
D_i=\begin{cases}
0, \text{ если }D_i^* \le 0, \\
1, \text{ если }D_i^*>0,
\end{cases}
\end{equation}

Вводится два предположения:

\begin{enumerate}
\item Существует переменная $z$, которая присутствует в уравнении для $D$, но не присутствует в уравнении для $y$, Она может быть непрерывной или дискретной, в частном случае ---  бинарной. Исключение регрессоров $x$ из уравнения участия --- это упрощение. Одновременное присутствие $z$ в уравнении участия и его исключение из уравнения для результирующей переменной называется \bfseries исключающим ограничением (exclusion restriction). \mdseries Эта струкрура модели знакома по Главе 16 о моделях с самоотбором. 
\item $\Cov [z,v] = \Cov [u,z] = \Cov [x,u] = 0$ и 
$$
\Cov [D, z] \ne 0.
$$
Вместе с первым предположением, это предположение означает, что $y$ зависит от $z$ только через $D$, и $D$ зависит от $z$ нетривиальным образом. Поэтому мы используем обозначение $D(z)$, чтобы подчеркнуть зависимость $D$ от $z$.
\end{enumerate}

При этих предположениях IV оценивание~(\ref{eq25.64}) даёт состоятельные оценки $(\beta, \alpha)$. Пусть $z' = z + \delta, \; \delta \ne 0$. Отметив также, что $\E [D | x, D(z)] = \Pr [D(z) = 1]$ и взяв ожидания, получим:

\begin{align}
\E [y | x, D(z) ] & = x' \beta + \alpha \Pr[D(z) = 1],  \nonumber \\
\E [y | x, D(z') ] & = x' \beta + \alpha \Pr[D(z') = 1], \nonumber 
\end{align}
где, после вычитания, мы имеем

$$
\E [y | x, z' ] - \E [y | x, z ] = \alpha [ \Pr[D(z') = 1] -  \Pr[D(z) = 1] ].
$$

Решая это уравнение относительно $\alpha$ получаем выражение для \bfseries локального среднего эффекта воздействия (local average treatment effect LATE), \mdseries рассмотренного Имбенсом и Ангристом (1994):

\begin{align}
\label{eq25.67}
\alpha_{LATE} & = \frac{\E [y | x, z' ] - \E [y | x, z ]}{\Pr[D(z') = 1] -  \Pr[D(z) = 1]'} \\
& = \frac{ \int_{R(x)} \left[ \E [y | x, z' ] - \E [y | x, z ] \right] dF(x|x \in R(x))}{ \int_{R(x)} \left[ \Pr[D(z') = 1] -  \Pr[D(z) = 1] \right] dF(x|x \in R(x))' } \nonumber \\
& = \frac{\E [y | z' ] - \E [y | z ]}{\Pr[D(z') = 1] -  \Pr[D(z) = 1]} , \nonumber
\end{align}
где во второй строке стоит усреднение по $x$, чей носитель обозначен как $R(x)$. Это выражение определено при $\Pr[D(z') = 1] -  \Pr[D(z) = 1] \ne 0$. Выборочный аналог этого выражения --- это отношение средней разности между экспериментальной и контрольной группами к изменению доли подвергнутых воздействию индивидов в силу изменения $z$. Это оценка, полученная с помощью метода инструментальных переменных. Используя асимптотическую нормальность оценок метода инструментальных переменных, мы можем получить доверительные интервалы для параметра LATE. 

Приставка <<локальный>> в LATE появляется из-за того, что он измеряет эффект воздействия на тех индивидах, которые приняли решение об участии в программе в силу изменений $z$. LATE зависит от конкретных значений $z$, использованных для оценки воздействия и от выбора инструментов. Группа <<переходящих>>  может быть нерепрезентативной для всей совокупности подвергнутых воздействию индивидов, не говоря уже об общей генеральной совокупности. Помимо этого, LATE может быть не информативным с точки зрения последствий крупных изменений политики и программ в силу того, что они могут влиять на инструменты, отличные от тех, которые мы наблюдаем. 

Для бинарных инструментов LATE и оценки метода инструментальных переменных эквивалентны, как показано в работе Ангриста и др. (1996, стр. 447). Если же в уравнении участия есть более одного инструмента, как при наличии переидентифицирующих ограничений, оценка LATE для каждого инструмента в общем случае будет отличаться. Однако, можно построить взвешенное среднее. 

Дальнейший анализ применим, когда эффект воздействия не изменяется в зависимости от индивида. Если же эффекты воздействия \bfseries гетерогенны \mdseries, возможны сложности с оценкой влияния $z$: наблюдаемая изменчивость вызвана разницей в $z$ или в $\alpha$? При гетерогенности идиосинкратический компонент эффекта воздействия

$$
u_{i,1} = u_{i,0} + D_i (\alpha_i (x_i) - \alpha (x_i)),
$$
является функцией от $\alpha_i (x_i) - \alpha (x_i)$, см.~(\ref{eq25.27}). Тогда предыдущих предположений недостаточно для вычисления ATE и ATET. Решением этой проблемы может выступать добавление \bfseries предположения о монотонности \mdseries в качестве дополнительного условия идентификации. Согласно ему, инструмент влияет на уровень участия монотонно, так что если в среднем участие более вероятно при $Z = w$, чем при $Z = z$, то любой, кто участвовал бы при $Z = z$ обязательно должен участвовать при $Z = w$. 

\subsection{Связь с другими мерами}

Оценка метода инструментальных переменных $\alpha$ совпадает с полученной при помощи двухшагового МНК, где первым шагом оценивается вероятность получения воздействия, $\E[D = 1 | x, z]$, а затем строится регрессия $y$ на $x$ и оцененную вероятностью, в предположении, конечно же, что эффект воздействия аддитивен. Рассмотрим частный случай оценки метода инструментальных переменных , где $x$ --- скаляр, равный единице, а $z$ ---- скалярная дамми-переменная, которая обозначает право на участие в программе воздействия: $z=1$ означает наличие права, $z=0$ --- отсутствие. 

Мы можем разделить генеральную совокупность на четыре категории: \bfseries конформисты (compliers,C)\mdseries, \bfseries всегда участвующие (always-takers, A)\mdseries, \bfseries никогда не участвующие (never-takers, N) \mdseries и \bfseries нарушители (defiers, D)\mdseries. Конформисты участвуют только когда имеют на это право, всегда участвующие --- всегда, вне зависимости от наличия права, никогда не участвующие --- никогда, вне зависимости от прав, а нарушители не получают воздействие, когда имеют на это право и получают, когда не имеют. Предположим, что нарушителей нет, и будем рассматривать другие три категории. 

\bfseriesОценка Вальда \mdseries эффекта воздействия определяется как:

\begin{equation}
\label{eq25.68}
TE_{WALD} = \frac{\E[y_i | z_i = 1] - \E[y_i | z_i = 0] }{\E[D_i | z_i = 1] - \E[D_i | z_i = 0]},
\end{equation}
где числитель, выраженный как взвешенное среднее эффектов воздействия по трем категориям с весами, равными вероятностям попасть в каждую категорию, равен:

\begin{align}
\Pr & [C]\{ \E[y_i | z_i = 1, C] - \E[y_i | z_i = 0, C] \}  \nonumber \\
& + \Pr[A]\{ \E[y_i | z_i = 1, A] - \E[y_i | z_i = 0, A] \} \nonumber \\
& + \Pr[N]\{ \E[y_i | z_i = 1, N] - \E[y_i | z_i = 0, N] \} \nonumber \\
& = \Pr[C]\{ \E[y_i | z_i = 1, C] - \E[y_i | z_i = 0, C] \}. \nonumber 
\end{align}
Результат в последней строчке получается из-за того, что члены, соответствующие всегда участвующим и никогда не участвующим равны нулю. Знаменатель в~(\ref{eq25.68}) равен вероятности конформизма, $\Pr[C]$. Следовательно

\begin{equation}
\label{eq25.69}
TE_{WALD} =\ E[y_{1,i} | z_i = 1, C] - \E[y_{0,i} | z_i = 0, C].
\end{equation}
Если мы сравним $TE_{WALD}$ с LATE, то обнаружим, что LATE --- это мера эффекта воздействия для подгруппы тех, кто на границе решения участвовать --- не участвовать, которых мы называем конформистами. 

В эмпирических экономических приложениях очень популярна концепция предельного влияния изменения непрерывной переменной, рассчитываемая как частная производная. Для дискретных переменных существует дискретный аналог. \bfseries Предельный эффект воздействия (marginal treatment effect (MTE)) \mdseries при условии на $x$ определяется как

\begin{equation}
\label{eq25.70}
MTE = \left. \frac{\partial \E[y|x,z]}{\partial \Pr [D = 1|x, Z]} \right|_{Z=z} .
\end{equation}

Хекман и Вытлацил (2002) показали, что ATE, ATET и LATE являются средними MTE, взятыми по разным подмножествам $Z$. ATE --- ожидаемая величина MTE по всему множеству возможных значений $z$, включая те, в которых уровень участия нулевой или равен единице. ATET исключает те $z$, где нет участия. LATE --- это средний MTE по тем $z$, где уровни участия различаются. 

\subsection{Оценка модели с гетерогенными эффектами воздействия с помощью метода инструментальных переменных}

Теперь рассмотрим модель, допускающую отбор по ненаблюдаемым показателям и гетерогенные эффекты воздействия. Используется контекст линейной модели с эндогенной переменной воздействия со случайным коэффициентом, см. Bjorklund и Моффитт (1987). Такая модель, которая обосновывается тем, что эффекты воздействия могут быть непостоянными по наблюдениям из экспериментальной группы, была рассмотрена Вулдридж (1997) и Хекман и Вытлацил (1998).

Мы записываем модель как систему одновременных уравнений, где результирующая переменная $y_1$ зависит от переменной воздействия $y_2$. Для простоты рассматривается непрерывная переменная воздействия $y_2$. Обозначая инструмент $z$ и экзогенную переменную $x_i$ получаем модель:

\begin{align}
\label{eq25.71}
y_{1,i} & = (\alpha + v_i) y_{2i} + x'_i \beta_1 + \e_i  \\
& = \alpha y_{2i} + x'_i \beta_1 + \e_i + v_i y_{2i} \nonumber \\
& = v_i \overline{y}_2 \alpha y_{2i} + x'_i \beta_1 + w_i, \nonumber \\
\label{eq25.72}
y_{2i} & = \gamma z_i + x'_i \beta_2 + \eta_i ,
\end{align}
где $w_i = \e_i + v_i (y_{2i} - \overline{y}_2 )$. Предельная реакция $y_1$ на изменение $y_2$ равна $(\alpha + v_i)$ и изменяется в зависимости от индивида, таким образом в модели допускаются \bfseries гетерогенные эффекты воздействия. \mdseries

Предположим, что $\E[\e_i | x_i , y_{2i}] = \E[v_i | x_i , y_{2i}] = 0$. Тогда $ \E[\e_i + v_i y_{2i}| x_i , y_{2i}] = 0$ и $\V[\e_i + v_i y_{2i}| x_i , y_{2i}]$ зависит от $x_i$ и следовательно гетероскедастична. Тогда МНК оценка $(\alpha, \beta_1)$ состоятельна, но не эффективна. Это следует из предполагаемой экзогенности $y_2$. 

Теперь рассмотрим случай, где переменная воздействия эндогенна. Делаются следующие предположения:

\begin{equation}
\label{eq25.73}
\E[\e_i | x_i , z_i ] = \E [\eta_i | x_i , z_i] = \E [v_i | x_i , z_i ] = 0,
\end{equation}
\begin{equation}
\label{eq25.74}
\E[\e_i^2 | x_i , z_i ] = \sigma_{\e}^2 ; \E [v_i^2 | x_i , z_i] = \sigma_v^2 ; \E [\eta_i^2 | x_i , z_i ] = \sigma_{\eta}^2.
\end{equation}
Эндогенность возникает из-за возможности корреляции между $v$ и $\eta$. В частности, предположим, что $\E[v_i | \eta_i] = \rho \eta_i$. Это выполняется, если $(v,\eta)$ имеют совместное нормально распределение. При таких предположениях $z$ --- валидный инструмент и $x$ --- экзогенный. Исключение $z$ из уравнения для $y_1$ --- идентифицирующее ограничение. Поэтому естественно выглядит оценка инструментальными переменными с инструментами $(z,x)$ уравнения~(\ref{eq25.71}). Заметим, однако, что для получения состоятельной оценки требуется $\E[w_i | x_i, z_i] = 0$. Первая компонента $w_i$, $\e_i$, не коррелирована с $z_i$ по предположению; вторая компонента $w_i$ --- это $v_i(y_{2i} - \overline{y}_2)$, которая, как может показаться, должна быть коррелирована с $z_i$, от которого зависит $y_{2i}$. Если это так, оценка метода инструментальных переменных будет несостоятельна. Однако, можно показать, что оценка метода инструментальных переменных будет состоятельной при введенных выше предположениях. Ключевой шаг здесь --- это показать, что $\E[v_i y_{2i} | z_i] = \E[v_i y_{2i}]$, как было продемонстрировано в Вулдридж (1997) применением закона итеративных ожиданий; тогда

\begin{align}
\label{eq25.75}
\E[vy_2|z] & = \E[\E[vy_2|z,\eta] | z]  \\
& = \E[y_2 \E[v|z,\eta] | z] = \E[\rho \eta y_2 | z] \nonumber \\
& = \rho E[\eta^2 | z] = \rho \sigma_{\eta}^2 = \E[v y_2]. \nonumber
\end{align}

Несмотря на то что оценка метода инструментальных переменных состоятельна при введенных предположениях, она не эффективна из-за гетероскедастичности ошибок. Поэтому необходимо использовать устойчивые к гетероскедастичности стандартные ошибки. Наконец, мы не рассмотрели вопросы чувствительности оцененных эффектов воздействия к выбору инструментов при гетерогенных эффектах воздействия. 

\subsection{Эндогенное воздействие в нелинейных моделях}

Рассмотрим теперь как анализ, проведенный в разделах 25.3 и 25.7, поменялся бы если бы результатом программы профессионального переобучения была занятость, а не доход, или продолжительность пребывания на одном рабочем месте. Либо предположим, что после обучения значительная часть участников остается без работы и имеет нулевые доходы; тогда выборка --- это смесь индивидов с нулевыми и ненулевыми доходами и, следовательно, не является нормальной. Как нужно расширить представленные методы, чтобы работать с нелинейностью и ненормальностью?

Спецификация и оценка нелинейных, ненормальных моделей воздействия и результата с самоотбором --- частая проблема в микроэконометрике. Как и в линейных моделях, основное внимание в таких моделях уделяется влиянию эндогенной переменной воздействия на экономический результат. Спецификация модели включает уравнение для результата со структурно-причинной интерпретацией и другие уравнения, которые моделируют процесс генерации переменных воздействия. Есть два широких подхода к этой проблеме: параметрический, который опирается на основанные на правдоподобии методы (включая Байесовские) и полупараметрический, основанный на ОММ или линеаризованных методах инструментальных переменных. 

Типичная ситуация иллюстрируется следующими несколькими примерами. В экономике труда, Bingley и Walker (2001) исследовали влияние продолжительности безработицы мужей на решения их жен об участии на рынке труда. Здесь переменная воздействия принимает неотрицательные значения и может быть цензурированной или усеченной. Pitt и Rosenzweig (1990) исследовали влияние эндогенного состояния здоровья младенцев на деятельность их матерей в течение дня; здесь переменная воздействия дискретна и результат непрерывен. Carrasco (2001) исследует влияние рождения детей на уровень участия в рабочей силе женщин. В моделях воздействия-результата, связанных с фертильностью, Jensen (1999) исследует влияние использования контрацептивов, дискретной переменной, на время между рождениями детей, ограниченную зависимую переменную. Olsen и Farkas (1989) исследуют влияние рождения детей на риск исключения из школы. В экономике здоровья, Kenkel и Terza (2001) изучают влияние рекомендаций врача (дискретная переменная) на потребление алкоголя (непрерывная и неотрицательная). Gowrisankaran and Town (1999) рассматривают влияние выбора госпиталя на риск смерти в госпитале. В экономике здоровья часто изучается влияние выбора медицинской страховки  на использование медицинских услуг, иногда измеряемое как расходы, а иногда --- как число отдельных потребленных услуг --- посещений врача или госпитализаций (Deb и Trivedi 1997). Terza (1998) и van Ophem (2000) моделируют влияние наличия у домохозяйства автомобиля на количество путешествий. Много привести и множество других примеров. 

У этих моделей есть множество статистических свойств. Во-первых, как воздействие, так и результат нелинейны и ненормальны: мультиномиальны, счетны, дискретны или цензурированы. Во-вторых, в каждой модели воздействие эндогенно. Наконец, у исследователей часто есть хорошие априорные причины выбрать те или иные частные распределения для воздействий и результатов. Однако, переход от частных распределений к совместной модели воздействия и результата --- важный шаг, на котором могут возникнуть проблемы, когда используются ненормальные многомерные распределения. Часто у частных распределений нет (или есть, но очень ограниченные) удобных многомерных аналогов (к примеру, счетных моделей и моделей длительности). В других случаях, воздействие и результат берутся из разных семейств распределений (к примеру, мультиномиальное воздействие и уровень риска в качестве результата), так что аналитические многомерные распределения часто просто не существуют. Из-за специализированной природы приложений в этой области, эта тема здесь больше не рассматривается. 

\section{Пример: влияние профессиональной подготовки на доходы}

Проводившийся в 1970-ые проект по национальной поддержке работы (National Supported Work, NSW), измерял влияние профессиональной подготовки на доходы при помощи рандомизированного эксперимента, который назначал некоторым индивидам профессиональную подготовку (экспериментальная группа), а других оставлял без подготовки (контрольная группа). Эффект подготовки затем измерялся прямым сравнением выборочных средних доходов после подготовки для экспериментальной и контрольной групп. 

Как обсуждалось в Главе 3, рандомизированные эксперименты достаточно редки в общественных науках. Более часто используется выборка, полученная в ходе наблюдения, часть индивидов из которой получают воздействие, а часть --- нет. Сравнение подвергнутых воздействию с неподвергнутыми должно учитывать различия в наблюдаемых характеристиках и, возможно, в ненаблюдаемых характеристиках. 

Чтобы определить адекватность стандартных микроэконометрических методов для данных, полученных в ходе наблюдения, Лалонд (1986) сравнил результаты экспериментальной группы из NSW с результатами контрольных групп, полученных из двух национальных опросов. Полученные результаты значительно отличались от полученных при помощи экспериментальной и контрольной выборки из NSW. Он пришел к выводу, что полученные в ходе наблюдения данные недостаточно надежны. 

Дехеджа и Ваба (1999, 2002) провели повторный анализ части данных Лалонд при помощи альтернативных методов сопоставления и получили результаты, близкие к полученным по экспериментальным данным. В этом разделе мы используем данные Дехеджа и Ваба (1999), чтобы продемонстрировать приложение методов, представленных в разделах 25.2 --- 25.5, которые учитывают только отбор по наблюдаемым показателям. 

\subsection{Данные Дехеджа и Ваба}

В экспериментальную выборку вошли 185 мужчин, прошедших профессиональную подготовку в 1976 --- 1977 гг. Контрольная группа состоит из 2490 мужчин-глав домохозяйств в возрасте до 55, не находяхся на пенсии, извлеченных из PSID. Дехеджа и Ваба (1999) называют эти две выборки подвыборкой RE74 (из экспериментальной группы NSW) и выборкой PSID-1 (не прошедших подготовку). Переменная-индикатор воздействия$D$ определяется как $D = 1$, если индивид прошел подготовку (тогда наблюдение попадает в экспериментальную выборку) и $D = 0$, если индивид не прошел подготовку (тогда наблюдение в контрольной выборке). 

Описательные статистики основных переменных приведены в таблице 25.3. Экспериментальная группа заметно отличается от контрольной, с большей пропорцией темнокожих (84\%), с образованием ниже университетского (71\%) и безработные в предшествовавший воздействию 1975 год (71\%). Оценки эффекта воздействия должны учитывать эти различия. 

\subsection{Управляющие функции}

Разные оценки влияние подготовки на доходы даны в таблице 25.4. 

Интересующая нас результирующая переменная --- это доходы после подготовки, RE78. Одна из возможных мер эффеекта воздействия - это средняя разница по переменной RE78 между прошедшими подготовку из NSW и не прошедшими из PSID, полученная оценка: $\$ 6 349 - \$ 21 554 = - \$ 15 205$. Такая оценка называется оценкой \bfseries сравнения экспериментальной и контрольной групп, \mdseries потому что она дублирует анализ в условиях эксперимента. Она также может быть рассчитана как коэффициент при переменной воздействия $D$ в МНК регрессии RE78 на константу и $D$ по объединенной (экспериментальной и контрольной) выборке. 

\begin{table}
\caption{\label{tab: } Влияние подготовки: Выборочные средние в выборках}
\begin{minipage}{17.5cm}
\begin{center}
\begin{tabular}{llcc}
\hline
\hline
Данные по & Определение & Эксперимент. &  Контрол. \\
домохозяйствам\footnote{Взяты те же данные, что и в таблице 1 в Дехеджа и Ваба (1999). Экспериментальная группа --- это подвыборока RE74 из NSW. Контрольная группа --- это выборка из PSID-1 с мужчинами --- главами домохозяйств до 55 лет, не находящимися на пенсии. Воздействие происходило в 1976 --- 1977 гг.} & & выборка & выборка\\
\hline
AGE & Возраст в годах & 25.82 & 34.85 \\
EDUC & Образование в годах & 10.35 & 12.12 \\
NODEGREE & 1, если EDUC < 12 & 0.71 & 0.31 \\
BLACK & 1, если темнокожий & 0.84 & 0.25 \\
HISP & 1, если латиноамериканец & 0.06 & 0.03 \\
MARR & 1, если женат & 0.19 & 0.87 \\
U74 & 1, если безработный в 1974 & 0.60 & 0.10 \\
U75 & 1, если безработный в 1975 & 0.71 & 0.09 \\
RE74 & Реальный доход в 1974 (в 1982\$) & 2096 & 19429 \\
RE75 & Реальный доход в 1975 (в 1982\$) & 1532 & 19063 \\
RE78 & Реальный доход в 1978 (в 1982\$) & 6349 & 21554 \\
D & 1, если участвовал в программе подготовки & 1.00 & 0.00 \\
Sample size &  & 185 & 2490 \\
\hline
\hline
\end{tabular}
\end{center}
\end{minipage}
\end{table}

Такая оценка воздействия неверна, потому что отражает в основном различия в типах индивидов в двух выборках --- индивиды из контрольной группы не очень подходят для сравнения. Можно учесть разницу, включив характеристики до воздействия в качестве регрессоров и оценивая МНК

\begin{equation}
\label{eq25.76}
RE78_i = x'_i \beta + \alpha D_i+ u_i, \qquad i = 1, \dots, 2675. 
\end{equation}
Это приводит к значительно меньшем оценкам эффекта воздействия $\widehat{\alpha} = \$ 218$ если в качестве регрессоров, как и в Дехеджа и Ваба, использовать постоянный член, AGE, AGESQ, EDUC, NODEGREE, BLACK, HISP, RE74 и RE75. Этот подход в разделе 25.3.3 называется \bfseries оценкой управляющей функции. \mdseries 

\subsection{Разность разностей}

Второй подход --- это \bfseries сравнение до-после, \mdseries которое смотрит на разницу доходов после воздействия RE78 и доходов до воздействия RE75. Используя средние доходы экспериментальной группы получаем оценку $\$ 6 349 - \$ 1 532 = \$ 4 817$. 

Это оценка также может быть неточной из-за того, что она отражает все изменения за рассматриваемый период времени, такие как изменения в экономике, а не только подготовку. \bfseries Оценка разности разностей, \mdseries  рассмотренная в разделе 25.5, дополнительно вычисляет похожую величину для контрольной группы, $\$ 21 554 - \$ 19 063 = \$ 2 491$, и использует эту меру изменений дохода за период, произошедших не в силу воздействия, так что изменение только в силу воздействия равно $\$ 4 817 - \$ 2 491 = \$ 2 326$. 

\begin{table}[h]
\caption{\label{} Влияние подготовки: Разные оценки эффекта воздействия}
\begin{minipage}{17.5cm}
\begin{center}
\begin{tabular}{p{7cm}lcc}
\hline
\hline
Метод & Определение & Оценка &  Ст. ошибка\footnote{Стандартные ошибки превых четырех оценок рассчитаны при помощи устойчивый к гетероскедастичности процедур из соответствующей МНК регрессии.} \\
\hline
Сравнение экспериментальной и контрольной групп & $\overline{RE78}_{D=1} - \overline{RE78}_{D=0} $ & -15 205 & 656 \\
Оценка управляющей функции & $\widehat{\alpha}$ из МНК регрессии~(\ref{eq25.76}) & 218 & 768\\
Сравнение до-после & $\overline{RE78}_{D=1} - \overline{RE75}_{D=1} $ & 4817 & 625 \\
Разность разностей & $\widehat{\alpha}$ из МНК регрессии~(\ref{eq25.77}) & 2326 & 749 \\
Мера склонности & см. Раздел 25.8.4 & 995 & --- \\
\hline
\hline
\end{tabular}
\end{center}
\end{minipage}
\end{table}

Можно показать, что оценка разностей в разностях эквивалентна оценке $\alpha$ из МНК регрессии

\begin{equation}
\label{eq25.77}
RE_{it} = \phi + \delta D78_{it} + \gamma \alpha D_i + \alpha D78_{it} \times D_i + u_{i}, \qquad i = 1, \dots, 2675, t = 75, \; 78.
\end{equation}

Здесь $RE_{i, 75}$ обозначает доходы в период до воздействия и $RE_{i,78}$ --- доходы в период после воздействия, так что при оценке регрессии используется 5350 наблюдений. Переменная-индикатор $D78_{it}$ равна 1 для периода после воздействия, индикатор $D_i$ равен 1, если индивид находится в экспериментальной группе, показатель $ D78_{it} \times D_i $ равен 1 для индивидов из экспериментальной выборки в период после воздействия. 

В более общем случае, константу $\phi$ в~(\ref{eq25.77}) можно заменить на $x'_{it} \beta$. Это ничего не меняет в этом примере, где регрессоры не изменяются во времени, так что $x_{it} = x_i$. Этот метод может быть применен к повторяющимся структурным (cross-section) данным (см. Раздел 22.6.2), потому что он не требует, чтобы индивиды из двух групп наблюдались в оба периода времени (1975 и 1978). 

\subsection{Простая оценка меры склонности}

Третий подход сравнивает результат $RE78$ для индивида из экспериментальной выборки с прогнозом $RE78$ для того же индивида в случае, если бы он не получал воздействие. Первая оценка сравнения экспериментальной и контрольной групп \$15205 --- это крайне упрощенный случай данного подхода, который использует в качестве прогноза $RE78$ среднее значение показателя для контрольной группы (\$21554). Можно получить более точные оценки при помощи регрессионных моделей. К примеру, модель~(\ref{eq25.76}) специфицирует $\E[RE78|x]$ равным $x'\beta + \alpha$ для подвергнутых воздействию и $x' \beta$ для не подвергнутых. Это налагает ограничения как на влияние регрессоров $x$, так и на эффект воздействия, которые, при заданном $x$, предполагаются постоянными по индивидам. 

В литература, посвященной эффектам воздействия, выделяются подходы, которые не опираются на такие сильные предположения. Самый очевидный подход --- это сравнение индивидов из экспериментальной и контрольной групп с одинаковыми значениями $x$, но на практике такое \bfseries сопоставление по регрессорам \mdseries невозможно, если релевантными считаются несколько регрессоров, принимающих большой диапазон значений. 

\vspace{3cm}
График 25.3: Влияние обучения: график доходов после воздействия от меры склонности для двух выборок. Включены только наблюдения с близкой мерой склонности. Наблюдения с доходами больше \$20000 не изображены на графике, однако включены в непараметрическую регрессию. 

Вместо этого, при предположениях из разделов 25.3 и 25.4, может оказаться достаточным \bfseries сопоставления по мере склонности, \mdseries определяемой как условная вероятность получения воздействия $\Pr [D = 1 | x]$. В этом примере, при оценке логит-модели мы используем только данные для 1975 года:

\begin{equation}
\label{eq25.78}
\Pr [D_i = 1 | x_i] = \Lambda (x'_i \beta), \qquad i = 1, \dots, 2675,
\end{equation}

Где, из раздела 14.2, $\Lambda (z) = e^z / (1+e^z)$, и в качестве регрессоров, как и в Дехеджа и Ваба (1999), взяты AGE, AGESQ, EDUC, EDUCSQ, NODEGREE, BLACK, HISP, MARR, RE74, RE75, RE74SQ, RE75SQ и $U74*BLACK$. 

График 25.3 показывает доходы доохды после воздействия RE78 в зависимости от меры склонности, отдельно для экспериментальной и контрольной групп. Рассматривая только меру склонности (ось $x$), становится ясно, что большая часть наблюдений в контрольной выборке имеет очень низкую меру склонности, как и ожидалось, учитывая то, что (как видно из таблицы 25.3) в экспериментальной выборке завышена доля черных, безработных и людей с низким уровнем образования. 

Обращаясь к доходу после воздействия RE78 (ось $y$), видно, что эффект воздействия оценивается как разность между данным индивидом из экспериментальной группы ($D = 1$) и индивидом из контрольной группы ($D = 0$) с такой же (спрогнозированной) мерой склонности. Каждый график в 25.3 показывает также оцененную непараметрическую регрессию RE78 на меру склонности. Эффект воздействия меньше одной тысячи долларов на большей части диапазона значений меры склонности, однако он положительный и заметно больше по значению для значений меры склонности около 0.8. 

Есть много способов реализации подхода, основанного на сравнении индивидов с похожими мерами склонности с последующим усреднением по всем индивидам из экспериментальной выборки. Одна из возможных стратегий заключается в сопоставлении подвергнутых воздействию индивидов с индивидами из контрольной группы с самой близкой мерой склонности. В Разделе 25.4.4 этот подход назывался сопоставлением по ближайшим соседям. Более простая стратегия заключается в стратификации данных по мере склонности, $p(x)$, с использованием среднего по страте значения RE78 контрольной группы в качестве оценки результата без воздействия. К примеру, если наблюдение из экспериментальной группы имеет меру склонности $p(x) = 0.35$, то гипотетической мерой склонности для случая не-участия может быть средняя $p(x)$ для наблюдений из контрольной группы с $0.3 < p(x) < 0.4$. Общий эффект тогда $\sum_s w_s (\overline{RE78}_{s,D=1} - \overline{RE78}_{s,D=0})$, где $\overline{RE78}_{s,D=1}$ и $\overline{RE78}_{s,D=1}$ --- средние по страте $s$ значения RE78 для, соответственно, экспериментальных и контрольных наблюдений, а веса $w_s$ равны доле экспериментальных наблюдений в каждой страте. Простая схема стратификации может использовать, к примеру, 10 страт одинакового размера с $0.0 < p(x) \leq 0.1$, $0.1 < p(x) \leq 0.2$ и так далее. Такой подход называется сопоставлением со стратификацией, он был рассмотрен в разделе 25.4.4. Эта процедура может использоваться в случаях, когда меры склонности для двух выборок накладываются друг на друга, см. раздел 25.4.3. Здесь мера склонности изменяется от 0.0005 до 0.9420 для экспериментальной группы и от 0.0000 до 0.9371 для контрольной группы, поэтому из наблюдения исключается 1423 наблюдения из контрольной группы и 8 из экспериментальной. Получившаяся оценка эффекта \$995 приведена в таблице 25.4. 

\subsection{Сопоставление при помощи мер склонности}

Как упоминалось в разделе 25.4, другие стратегии сопоставления включают в себя радиальное и ядерное сопоставление, которые также достаточно несложно реализовать. Оставшаяся часть данной главы рассматривает эти и другие подходы, с упором на методы, использующие меры склонности. 

\subsection*{Оцененные меры склонности}

Оцененные меры склонности получаются при использовании двух разных спецификаций логит-модели, из Дехеджа и Ваба (1999) и Дехеджа и Ваба (2002) соответственно. Спецификации для мер склонности приведены внизу таблицы 25.6. Единственное отличие от Дехеджа и Ваба (1999, 2002) заключается в том, что константа включена в логит-модели. Оценки коэффициентов, не приведенные в целях экономии места, имеют ожидаемые знаки. 

\subsection*{Алгоритмы сопоставления и балансировка}

Важный на практике вопрос заключается в выборе алгоритма сопоставления, основанного на мерах склонности, который соответствовал бы условию балансировки~(\ref{eq25.9}). Дехеджа и Ваба (2002, стр. 161) приводят алгоритм, который начинается с экономной логит-модели для оценки $p(x)$. Алгоритм работает следующим образом. Данные сортируются по $\widehat p(x)$. Наблюдения стратифицируются так, что внутри страты $\widehat p(x)$ для экспериментальной и контрольной групп близки. К примеру, в начале может использоваться грубая сетка с одинаковой длины границами. Внутри каждой страты необходимо протестировать равенство средних в двух группах для каждой объясняющей переменной. Если нет статистически значимых различий, регрессоры можно считать сбалансированными между экспериментальной и контрольной группами и алгоритм можно остановить. Если для какой-то страты нет баланса, для \bfseries несбалансированной страты \mdseries нужно использовать более тонкую сетку. Если наблюдается много несбалансированных страт, изначальная логит-модель переоценивается с другой спецификацией, которая включает произведения показателей и члены более высоких порядков. 

\begin{table}[h]
\caption{Влияние подготовки: Распределение мер склонности для экспериментальной и контрольной групп по спецификации Дехеджа и Ваба (1999)}
\begin{center}
\begin{minipage}{12cm}
\begin{tabular}{cccc}
\hline
\hline
Минимум $\widehat p(x)$ \footnote{Для второй строки, к примеру, мера склонности лежит между 0.10 и 0.20 для 10 экспериментальных и 56 контрольных индивидов} & Экспериментальная & Контрольная &  Общая \\
\hline
0.000364 & 9 & 960 & 969 \\
0.10 & 10 & 56 & 66 \\
0.20 & 14 & 33 & 47 \\
0.40 & 24 & 22 & 46 \\
0.60 & 33 & 7 & 40 \\
0.80 & 95 & 8 & 103 \\
Общая & 185 & 1086 & 1271 \\
\hline
\hline
\end{tabular}
\end{minipage}
\end{center}
\end{table}
Используя программное обеспечение из Бекер и Ичино (2002), мы применили алгоритм Дехеджа и Ваба (2002) для расчета мер склонности. Во всех случаях, расчет мер склонности был ограничен областью пересечения мер склонности тестированием \bfseries условия балансировки \mdseries по тем наблюдениям, для которых меры склонности лежат на пересечении областей значения мер склонности для контрольной и экспериментальной выборок. Это ограничение серьезно уменьшает размер выборки. Размер контрольной группы упал с 2490 наблюдений до 1086 при использовании спецификации из Дехеджа и Ваба (2002). 

В таблице 25.5 показано число наблюдений в каждой из выборок по интервалам значений меры склонности после балансировки. Эти результаты отличаются от полученных Дехеджа и Ваба (2002), потому что они исключают контрольные наблюдения из NSW-PSID не на основании пересечения областей значения, вместо этого они исключают те наблюдения, для которых оцененная мера склонности меньше, чем минимальная из мер склонности для экспериментальных наблюдений. Таблицы показывают, что доля экспериментальных наблюдений очень низка для первых интервалов по сравнению с последующими. 

Похожее упражнение для спецификации из Дехеджа и Ваба (1999) даёт схожие результаты. В контрольной группе 1146 наблюдений. Границы для интервалов $\widehat p(x)$: 0.0006526, 0.05, 0.10, 0.20, 0.40, 0.60 и 0.80. 

\subsection*{Оценка ATET методами сопоставления}

Результаты для разных методов сопоставления представлены в таблице 25.6. Оценка ATET методом ближайших соседей для спецификации Дехеджа и Ваба (2002) равна \$2385, для Дехеджа и Ваба (1999) приблизительно \$560. Стратификация и ядерное сопоставление также дают смешанные результаты, оценки ATET изменяются от \$1452 до \$2156. 

Для сравнения, оценки ATET из Дехеджа и Ваба (2002) приведены в таблице 25.7. Отметим также, что базовая оценка эффекта воздействия равна \$1794. Она получена регрессией RE78 на $D$ для версии выборки NSW из Дехеджа и Ваба (2002), в которую входят и участники и не-участники. Оценки ATET из этой таблице заметно отличаются от полученных в Дехеджа и Ваба (2002), равно как и от базовых, экспериментальных оценок. Для спецификации Дехеджа и Ваба (2002), оценка ближайшего соседа очень близка к базовой и даже лучше результатов Дехеджа и Ваба (2002) из-за уменьшения смещения. 

\begin{table}[h]
\caption{Влияние подготовки: оценки ATET}
\begin{center}
\begin{minipage}{17cm} 
\begin{tabular}{lccccc}
\hline
\hline
Процедура сопоставления & Кол-во в  & Кол-во в & ATET & Ст. ошибка & \% в \$1794 \footnote{$ATET/1794 \cdot 100$.} \\
& эксперимент. & контрол. & &\\
\hline
\multicolumn{6}{l}{Спецификация Дехеджа и Ваба (2002) \footnote{Логит-модель: $\Pr[treat = 1]$ = h(CONSTANT, CONSTANT, AGE, $AGE^2$, EDU, $EDU^2$, MARRIED, NODEGREE, BLACK, HISPANIC, RE74, $RE74^2$, RE75, U74, U75, U74*HISPANIC).}}\\
Ближайший сосед & 185 & 53 & 2385 & 1209 \footnote{Бутстрапированные стандартные ошибки, 200 репликаций.} & 133 \\
Радиус, $r = 0.001$ & 54 & 517 & $-7815$ & 1118 \footnote{Аналитические стандартные ошибки.} & $-436$ \\
Радиус, $r = 0.0001$ & 24 & 92 & $-9333$ & 2282 \textcolor{red}{$^d$} & $-520$ \\
Радиус, $r = 0.00001$ & 15 & 19 & $-2200$ & 2986 \textcolor{red}{$^d$} & $-123$ \\
Стратификация & 185 & 1086 & 1452 & 1041 \textcolor{red}{$^c$} & 81 \\
Ядерная & 185 & 1058 & 1309 & 975 \textcolor{red}{$^c$} & 73 \\
\multicolumn{6}{l}{Спецификация Дехеджа и Ваба (2002) \footnote{Логит-модель: $\Pr[treat = 1]$ = h(CONSTANT, AGE, $AGE^2$, EDU, $EDU^2$, MARRIED, NODEGREE, BLACK,
HISPANIC, RE74, $RE74^2$, RE75, $RE75^2$, $RE74*RE75$, $U74*BLACK$).}}\\
Ближайший сосед & 185 & 57 & 560 & 1098 \textcolor{red}{$^c$} & 31 \\
Радиус, $r = 0.001$ & 57 & 583 & $-9358$ & 997 \textcolor{red}{$^d$} & $-522$ \\
Радиус, $r = 0.0001$ & 27 & 76 & $-7847$ & 2066 \textcolor{red}{$^d$} & $-437$ \\
Радиус, $r = 0.00001$ & 16 & 13 & 223 & 4551 \textcolor{red}{$^d$} & 12 \\
Стратификация & 185 & 1146 & 2156 & 814 \textcolor{red}{$^c$} & 120 \\
Ядерная & 185 & 1146 & 1518 & 890 \textcolor{red}{$^c$} & 85 \\
\hline
\hline
\end{tabular}
\end{minipage}
\end{center}
\end{table}

Для стратифицированных и ядерных оценок, смещение больше. Для оценки радиальным сопоставлением, это смещение хуже, потому что дает отрицательные оценки эффекта воздействия вместо положительных, которые Дехеджа и Ваба (2002) получили при циркульном сопоставлении. Разница между нашим радиальным сопоставлением и циркульным сопоставлением Дехеджа и Ваба (2002) заключается в том, что при циркульном сопоставлении объект, не имеющий аналогов внутри заданного радиуса, сопоставляется с ближайшим соседом за пределами радиуса.  В нашем же случае, в сходной ситуации такой объект просто игнорируется. Полученные различия демонстрируют чувствительность оценок сопоставления к предположениям. 

Робастность оценок ATET при разных спецификациях может быть оценена в терминах соотношения ATET и базовой оценки, приведенного в последней колонке таблицы 25.6. За исключением сопоставления со стратификацией, это отношение очень сильно меняется в зависимости от спецификации. К примеру, оценка ближайшего соседа составляет 133\% базовой оценки в спецификации Дехеджа и Ваба (2002), но только 31\% в спецификации Дехеджа и Ваба (1999). Аналогично, оценки ATET чувствительны к использованной мере склонности, кроме случая ядерной оценки. 

\begin{table}[h]
\caption{Влияние подготовки: оценки ATET Дехеджа и Ваба (2002)}
\begin{center}
\begin{tabular}{lcc}
\hline
\hline
Процедура сопоставления & ATET &  Ст. ошибка \\
\hline
Ближайший сосед & 1890 & 1202 \\
Радиус, $r = 0.001$ & 1824 & 1187 \\
Радиус, $r = 0.0001$ & 1973 & 1191 \\
Радиус, $r = 0.00005$ & 1928 & 1196 \\
Радиус, $r = 0.00001$ & 1893 & 1198 \\
\hline
\hline
\end{tabular}
\end{center}
\end{table}

Качество работы методов сопоставления зависит от выбора модели меры склонности для экспериментальной и контрольной групп (Дехеджа and Ваба, 2002). Однако, очевидно, что есть связь между методами и моделями меры склонности. 

\section{Библиографические замечания}

Ранние экономические приложения методов сопоставления и разности разностей встречаются в Ashenfelter (1978) и Ashenfelter и Card (1985). Оценка воздействия --- очень популярная и быстро развивающая область в современной эконометрике. 
\begin{enumerate}
\item[25.2] Ангрист et al. (1996) приводят полезные сравнения концепций и терминологии в медицинской и эконометрической литературе.  
\item[25.3] Хекман и Robb (1985) рассматривают оценку эффекта программ для разных видов данных в присутствии самоотбора. См. также  Bjorklund и Моффитт (1987). Хекман и Хотц (1989) также говорят, что необходимо проверить результаты несколькими тестами на спецификацию чтобы проверить их робастность и оценить влияние смещения самоотбора. К примеру, они предлагают использовать множественные сравнительные группы для оценки чувствительности результатов, полученных на основе одной контрольной группы. Большая часть их ранних работ использует параметрический подход. В более новых также применяются непараметрические методы. 
\item[25.4] Хекман, Ichimura и Todd (1997) и Хекман et al. (1998) изучают и применяют оценки сопоставления. Важный результат, касающийся условий на меру склонности, дан в Розенбаум и Рубин’s (1983, теорема 2). Эффективное оценивание ATE при помощи оцененных мер склонности проанализировано в Hirano, Имбенс и Ridder (2003). Дехеджа и Ваба (2002) применяют методы сопоставления мер склонности для варианта данных Лалонд (1986). Экспериментальные данные сопоставляются с наблюдениями из CPS и PSID. Смит и Todd (2004) повторно анализируют данные Дехеджа and Ваба при помощи разных вариантов оценок на основе мер склонности. Они подчеркивают смещения, связанные с альтернативными оценками мер склонности и отмечают важность использования высококачественных данных для минимизации смещения. Бекер и Ичино (2002) приводят обзор нескольких оценок на основе сопоставления мер склонности. Они также предоставляют набор программ для STATA с иллюстрациями, который можно использовать для оценки ATET. В выпуске Quarterly Journal of Economics за февраль 2004 есть подборка по эконометрике сопоставления. 
\item[25.6] Хан, Todd, and Ван дер Клаув (2001) анализируют идентификацию эффектов воздействия для RD-модели при слабых предположениях.  
\item[25.7] Имбенс и Ангрист (1994) анализируют свойства оценок LATE. Ангрист et al.(1996) обсуждают использование методов инструментальных переменных и делают сравнение с оценкой LATE. За статьей следует оживленная дискуссия, в которой приводятся разные мнения об оценках метода инструментальных переменных и связи с другой литературой, см. также Хекман (1997). Ангрист (2001) обсуждает некоторые простые стратегии для работы с эндогенными дамми в нелинейных моделях с ненормальными результатами. За этой статьей следует обсуждение и комментарии, которые анализируют достоинства и недостатки линеаризованного метода инструментальных переменных. Нет соглашения по поводу того, какой из подходов выглядит наиболее многообещающе. Хекман, Tobias, и Вытлацил (2003) рассматривают оценки эффектов воздействия в рамках моделей с латентными переменными. Vella and Verbeek (1999) сравнивают метод инструментальных переменных с использованием управляющих функций, включающих коррекцию на смещение самоотбора. 
\end{enumerate}

\section*{Упражнения}

\begin{enumerate}
\item[25-1] (Адаптировано из Хекман, 1996) Рассмотрите модель воздействие-результат $y = x' \beta +\alpha d + \e$, где $d$ --- бинарный индикатор, равный 1, если воздействие было назначено (назначение случайно), и 0 в противном случае.
\begin{enumerate}
\item Является ли рандомизированное воздействие достаточным условием для оценки $\alpha$? 
\item Является ли рандомизированное воздействие достаточным условием для оценки $\alpha$ и $\beta$? 
\end{enumerate}

\item[25-2] В предыдущей задаче рандомизация относится к воздействию. Теперь рассмотрим рандомизированное право на получение воздействия. Пусть $e = 1$ обозначает, что индивид имеет право (значение присваивается случайно) и $e = 0$ --- что не имеет. Покажите, что в этом случае при $\Pr [d = 1 | x] \ne 0$, эффект воздействия задается $\E [y | e = 1, x] - \E [y | e = 0, x]/\Pr[d = 1|x]$.

\item[25-3] Рассмотрим модель с нелинейным результатом воздействия $\E [y|x,d] = \exp (x'\beta + \alpha d)$, где $d$ --- это бинарный индикатор воздействия. Предположим, что нам доступны состоятельные оценки $(\beta, \alpha)$ и оценка ковариационной матрицы $\widehat \V [\widehat{\beta}, \widehat{\alpha}]$. Пусть оценка асимптотически нормальна. Наметьте бутстраповский или Монте-Карло алгоритм для оценки ATE и его асимптотической дисперсии, если даны $(x_i , d_i), \; i = 1, \dots, N$. 

\item[25-4] Рассмотрим модель с нелинейным результатом воздействия $\E [\ln y|x,d] = x'\beta + \alpha d$, где $d$ --- это бинарный индикатор воздействия. Предположим, что нам доступны состоятельные оценки $(\beta, \alpha)$ и оценка ковариационной матрицы $\widehat \V [\widehat{\beta}, \widehat{\alpha}]$. Пусть нас интересует оценка ATE в терминах $y$, а не $\ln y$. Предложите процедуру оценивания и проанализируйте её состоятельность. 

\item[25-5] В этой главе в эмпирической иллюстрации использовалась контрольная группа из PSID и экспериментальная из NSW. Дехеджа и Ваба (2002) использовали две контрольные группы. Вторая была основана на CPS. В этом упражнении Вас попросят повторить некоторые вычисления, представленные здесь, при помощи контрольной группы из CPS, а не из PSID. 
\begin{enumerate}
\item Сделайте таблицу, аналогичную 25.3. Сравните группу из NSW с группой из CPS по возрасту, расовой принадлежности, образованию и доходам до воздействия.  
\item Различия между экспериментальной и контрольной группами могут рассматриваться при помощи оцененной меры склонности, как это было сделано в разделе 25.8. Используя подход из раздела 25.8.4, оцените меру склонности для общей выборки NSW-CPS, включая объясняющие переменные в модель линейно и в более высоких порядках, как в Дехеджа и Ваба (2002). Игнорируя те контрольные наблюдения, для которых мера склонности меньше минимума по экспериментальной группе, сравните меры склонности в двух выборках при помощи гистограмм. Прокомментируйте качество сопоставления для разных интервалов значений меры склонности. 
\item Используя методы сопоставления, описанные и реализованные в разделах 25.8.4 и 25.8.5 (особенно метод ближайшего соседа, стратификацию или интервальное сопоставление, ядерное сопоставление и радиальное сопоставление), постройте таблицу, аналогичную таблице 25.6. Прокомментируйте полученные оценки ATET и сравните их с полученными по контрольной выборке из PSID. 
\end{enumerate}

\end{enumerate}






\chapter{Модели ошибок измерения}
%%%% Comments %%%%

\section{Введение}
Проблемы, связанные с ошибками измерений в эконометрике встречаются повсеместно. В микроэконометрике основным источником  ошибок измерения являются неправильные ответы на вопросы анкет, неправильное кодирование правильных ответов и использование правильно измеренных переменных в качестве прокси для теоретически подходящей, но ненаблюдаемой переменной (например, использование наблюдаемого дохода как прокси для <<нормального дохода>>.  Вопросы, связанные с конфиденциальной информацией, могут привести к неполным или неверным ответам. Иными словами, ошибки измерений порождаются ненаблюдаемыми \emph{(или латентными)} переменными, когда они заменяются прокси-переменными.

Приведём несколько примеров. Рассмотрим задачу тестирования наличия дискриминации по половому признаку в рамках исследования доходов. Очевидный подход --- построить регрессию переменной доходов на категориальную переменную пола индивида, учитывая квалификацию, возраст, опыт работы и т.д. Однако наиболее подходящей переменной может быть индивидуальная производительность труда на рабочем месте, которая не может быть измерена напрямую и вместо которой может быть использована прокси. Таким образом, влияние ошибки измерений на выводы относительно гендерной дискриминации, является важной проблемой. Исследования индивидуального спроса на товары и услуги используют такие понятия, как <<экономические издержки>> \emph{(economic cost)} и <<полная цена услуги>> \emph{(full price of a service)}. Тем не менее, такие переменные редко напрямую могут быть посчитаны по доступным данным, так что для оценивания моделей должны быть сконструированы их заменители.  Неизбежно появляются ошибки измерения.

В этой книге нет фактически ни одной модели, защищённой от проблем ошибок измерений. Бинарные  эндогенные и экзогенные переменные являются потенциально источником ошибок классификации; данные по переходам из одного состояния в другое и дискретные данные, взятые из ретроспективных обследований подвержены ошибкам памяти; данные по относительно однозначным переменным, таким как почасовая заработная плата и расходы домохозяйств, могут искажаться из-за преднамеренного завышения или неверно сообщенной информации. В отличие от агрегированных данных, когда агрегирование может нивелировать некоторые ошибки измерений, данные на уровне отдельных индивидов эти ошибки сохраняют.

В первой части этой Главы изучаются последствия ошибок измерений и стратегии оценивания для ликвидации этих последствий. Рассматриваются как линейные, так и нелинейные модели. Хотя более реалистичным было бы решать проблему ошибок измерений, принимая во внимание, что она возникает одновременно с рядом других, более удобно для представления предположить, что это единственная проблема, которая стоит перед эконометристом.

Вообще говоря, последствия ошибок измерения заключаются в неправильной идентификации интересующих параметров. Решение такой задачи сложное. Можно просто не включать действительно подходящую переменную, можно заменить её на прокси для истинного значения. Существуют, по крайней мере, две причины, по которым не стоит впадать в крайности. Первая заключается в том, что невключение в модель основной интересующей переменной ведёт к серьёзному смещению, что является замещением одной проблемы другой при той же невозможности идентификации параметров. Вторая причина такова: в линейной регрессии использование прокси вместо латентной переменной даёт меньшее асимптотическое смещение, чем невключение переменной в модель, если ошибки наблюдения случайны и независимы от истинных регрессоров (МакКоллум, 1972). Игнорирование переменной даёт оценки низкого качества.  Тем не менее, использование прокси не решает проблему несмещенности, хотя и приводит к меньшему смещению.

Основная идея, лежащая в основе решения задачи борьбы с ошибками измерений, --- чтобы восстановить параметр при латентной переменной и  идентифицировать саму модель, необходимо располагать большей информацией в виде дополнительных предположений об ошибках измерений или получить дополнительные данные и использовать эту информацию после введения правдоподобных допущений. Такой подход является довольно известным. Тем не менее, когда дополнительная информация недоступна, эконометрические модели представляют собой хорошую альтернативу.

Ошибки измерений вызывают очень серьёзные последствия, так как во многих случаях они ведут к тому, что параметры регрессии не могут быть идентифицированы. Например, Кард (2001) проанализировал эмпирические исследования коэффициента при переменной образования в регрессии доходов и обнаружил, что, как правило, результаты смещены в сторону занижения на 25--35\%.  Конкретные масштабы последствий ошибок измерений зависят от функциональной формы модели, от того, какую форму принимают ошибки (аддитивную или мультипликативную) и от структуры анализируемых данных. Решение проблемы, возникшей по причине ошибок измерений, обычно требует введения в модель дополнительной информации, либо в форме данных, либо в форме предположений.

Удобно построить рассмотрение ошибок измерений, разбив его на части: случаи линейных моделей, нелинейных моделей и специальные случаи. Разделы 26.2 и 26.3 посвящены линейной регрессии. В Разделе 26.4 речь идёт о нелинейной регрессии. В Разделе 26.5 приводятся несколько примеров использования метода Монте-Карло.  Необходимые в анализе линейных регрессий идеи применимы и в случае нелинейных моделей. В любом случае более ясные результаты обычно доступны в моделях специального вида.

\section{Ошибки измерений в линейной регрессии}

Ошибки измерений регрессоров или {\bf ошибки в переменных}  --- важная тема, так как это причина несостоятельности МНК-оценок, даже если ошибки измерений имеют нулевое математическое ожидание. Принято говорить, что ошибки измерений регрессоров ведут к смещению оценок, но мы пользуемся более сильным термином --- несостоятельность --- то есть смещение сохраняется даже при стремлении размера выборки к бесконечности.

Спектр моделей ошибок измерений широк, они описывают ситуации, когда ошибкам измерений подвержены объясняющие переменные --- регрессоры, зависимые переменные или и те, и другие. В работе Хаусмана (2001) эти случаи названы <<проблемы справа>> и <<проблемы слева>>. В случае ошибки в регрессорах, когда мы имеем классическую модель ошибки переменной,  интерес представляет взаимосвязь зависимой переменной $y$ и объясняющих переменных $(\mathbf{W}, \mathbf{X^*})$, где $\mathbf{W}$ измерена без ошибок, а $\mathbf{X^*}$ --- ненаблюдаемая, но прокси для неё, переменная $\mathbf{X}$, доступна. Вопрос исследования ставится тогда следующим образом: является ли оценка взаимосвязи $\mathbf{y}$ и $(\mathbf{W}, \mathbf{X})$ достаточным основанием для заключений относительно $\mathbf{X^*}$.


В статистической литературе  принято разделять  функциональный и структурный подходы к моделированию ошибок измерения. Если $\mathbf{X^*}$ отражает истинные ненаблюдаемые регрессоры, то, согласно функциональному подходу, они буду представлены как неизвестные фиксированные константы. В структурном подходе они будут считаться случайными. Кэррол, Рупперт и Стефански (1995) далее выделяют \emph{функциональное моделирование}, в котором относительно регрессоров $\mathbf{X}$ сделаны лишь минимальные предположения, вне зависимости от того, фиксированные они или случайные, и \emph{структурное моделирование}, которое предполагает параметрические предположения о распределении $\mathbf{X}$. Функциональные модели ошибок измерений являются примерами моделей с бесконечным количеством мешающих параметров, что порождает известные сложности в применении метода максимального правдоподобия (которые обсуждались в Главе, посвященной анализу панельных данных). Этому моменту нечасто уделяется внимание в эконометрической литературе.

Масштабы несостоятельности на практике могут быть существенны. В исследованиях детерминант доходов индивидов тема проблемы ошибок измерений, как и методов её выявления, является очень популярной

\subsection{Классическая модель ошибок измерений} 

Стандартная модель ошибок измерений имеет непрерывную зависимую переменную $y$, которая является линейной функцией от $K$ истинных регрессоров $\mathbf{x^*}$. Аддитивная ошибка измерений $y$ не порождает никаких проблем, если она не коррелирована с регрессорами, так как она может быть просто объединена с ошибкой регрессии. Если бы $\mathbf{x^*}$ наблюдались, то параметры могут быть состоятельно оценены с помощью МНК-регрессии $y$ на $\mathbf{x^*}$,
\[
y_i=\mathbf{x^*}_i'+u_i,
\]
Где $u_i$ --- независимые одинаково распределённые случайные величины с параметрами $[0,\sigma^2]$. Вместо этого мы наблюдаем $\mathbf{x}\neq \mathbf{x^*}$ и строим регрессию $y$ на $\mathbf{x}$, а не на $\mathbf{x^*}$. Взаимосвязь между истинными и реально наблюдаемыми регрессорами задаётся как

\begin{equation}
\mathbf{x}_i=\mathbf{x}_i^*+\mathbf{v}_i, i=1,\ldots,N,
\end{equation}

где аддитивно заданная ошибка измерений предполагается распределённой следующим образом:
\begin{equation}
\mathbf{v}_i \sim [\mathbf{0},\mathbf{\Sigma_{vv}}].
\end{equation}

Для ненаблюдаемых истинных регрессоров предполагается нулевое математическое ожидание, то есть мы трактуем их как отклонения от среднего с ковариационной матрицей 
\begin{equation}
\Var\mathbf{([x^*_i]}=\mathbf{\Sigma_{x^*x^*}}.
\end{equation}
Отметим, что $\mathbf{x}$ есть несмещённая оценка $\mathbf{x^*}$, так как математическое ожидание ошибки измерений $\mathbf{v}$ предполагается равным нулю. Ошибка измерений, согласно предпосылкам модели, независима от $\mathbf{x^*}$ и ошибки регрессии $u$,
\begin{equation}
\Expect(\mathbf{v_i}|\mathbf{x^*_i})=\Expect[\mathbf{v_i}|u_i]=0.
\end{equation}

\subsection{Несостоятельность оценок метода наименьших квадратов}  

Чтобы рассмотреть последствия ошибок измерений, удобно представить процесс порождающий данные для классической модели ошибок измерений в матричном виде как 
\begin{align}
\mathbf{y}&=\mathbf{X^*}\beta+\mathbf{u}, \\
\mathbf{X}&=\mathbf{X^*}+\mathbf{V}, \notag
\end{align}
где $u$, ошибка регрессии, удовлетворяет условиям $\Expect(\mathbf{u}|\mathbf{X^*})=\mathbf{0}$ и $\Expect(\mathbf{uu'}|\mathbf{X^*})=\sigma^2\mathbf{I}_N$. Подставляя второе выражение в первое, получаем
\begin{equation}
\mathbf{y}=\mathbf{X}\beta+(\mathbf{u}-\mathbf{V}\beta).
\end{equation}
МНК-регрессия $\mathbf{y}$ на $\mathbf{X}$ ведёт к получению несостоятельных оценок $\mathbf{\beta}$, так как ошибка $(\mathbf{u}-\mathbf{V\beta})$ коррелирована с регрессором $\mathbf{X}$ через ошибку измерений $\mathbf{V}$.
Формально мы имеем
\begin{align*}
\plim N^{-1}\mathbf{X'}(\mathbf{u}-\mathbf{V\beta})&=\plim N^{-1}(\mathbf{X^*}+\mathbf{V})'(\mathbf{u}-\mathbf{V\beta}) \\
&=-\mathbf{\Sigma_{vv}}\mathbf{\beta} \\
&\neq\mathbf{0},
\end{align*}
где мы воспользовались тем, что $N^{-1}\mathbf{V'V}=N^{-1}\sum\limits_i\mathbf{v}_i\mathbf{v}'_i$ и $\mathbf{v}_i$ независимы и одинаково распределены с параметрами $[\mathbf{0}, \mathbf{\Sigma_{vv}}]$. Это и есть основной источник несостоятельности. Теперь 
\begin{align*}
\plim N^{-1}\mathbf{X'X}&=\plim N^{-1}(\mathbf{X^*+V})'(\mathbf{X^*+V}) \\
&=\mathbf{\Sigma_{x^*x^*}}+\mathbf{\Sigma_{vv}},
\end{align*}
где мы воспользовались тем, что $\mathbf{x_i^*}$ --- независимые одинаково распределённые случайные величины с нулевым средним и ковариационной матрицей $\Var\mathbf{[x^*_i]}=\mathbf{\Sigma_{x^*x^*}}$. Также
\begin{align*}
\plim N^{-1}\mathbf{X'y}&=\plim N^{-1}(\mathbf{X^*+V})'(\mathbf{X^*\beta+u}) \\
&=\mathbf{\Sigma_{x^*x^*}}+\mathbf{\Sigma_{vv}},
\end{align*}
так что, согласно теореме Слуцкого (Приложение А, Теорема А.3.), мы получаем
\begin{align}
\plim \mathbf{\widehat {\beta}}&={\plim N^{-1}\mathbf{X'X}}^{-1}\plim N^{-1}\mathbf{X'y} \\
&={\mathbf{\Sigma_{xx}}}^{-1}(\mathbf{\Sigma_{xx}}-\mathbf{\Sigma_{vv}})\mathbf{\beta} \notag \\
&=\mathbf{\beta}-{(\mathbf{\Sigma_{x^*x^*}}+\mathbf{\Sigma_{vv}})}^{-1}\mathbf{\Sigma_{vv}}\mathbf{\beta}. \notag 
\end{align}

Очевидно, что несостоятельность МНК появляется из-за ошибок измерений и предположения $\mathbf{\Sigma_{vv}}\neq\mathbf{0}$.

Для последующего изложения отметим, что, если мы имеем состоятельную оценку ковариационной матрицы $\mathbf{\Sigma_{vv}}$, обозначенной как $\mathbf{S_{vv}}$, и если $(\mathbf{X'X}-\mathbf{S_{vv}})$ положительно определена, то можно вычислить оценку скорректированным МНК $\mathbf{\widehat {\beta_a}}={(\mathbf{X'X}-\mathbf{S_{vv}})}^{-1}\mathbf{X'y}$. Эта формула может быть использована для изучения влияния дисперсии ошибки измерений на оценки метода наименьших квадратов.

\subsection{Ошибки измерений и скалярный регрессор}

Частный случай этой модели, который обычно рассматривается в учебниках, подразумевает наличие одного истинного или ненаблюдаемого регрессора $x^*$ с дисперсией $\sigma^2_x$, наблюдаемой переменной $x$, ошибки измерений $v$ с нулевым средним и дисперсией $\sigma^2_v$. Таким образом, мы имеем уравнение регрессии $y=\beta x^*+u$, где $\Expect[u|x^*]=0$, $\Var[u|x^*]=\sigma^2_u$,  и $\Cov[v,u]=0$, но в регрессии вместо $x^*$ используется наблюдаемая величина $x$. 
В этом случае (26.7) может быть представлено как
\begin{align}
\plim\widehat {\beta}&=\frac{\sigma^2_{x^*}}{\sigma^2_{x^*}+\sigma^2_v}\beta \\
&=\frac{1}{1+\sigma^2_v / \sigma^2_{x^*}}\beta \notag \\
&=\beta[1-s/(1+s)], \notag 
\end{align}

где $s=\sigma^2_v / \sigma^2_{x^*}$, что носит название отношения {\bf шум-сигнал}, а компонента ${(1+s)}^{-1}$ называется {\bf отношением стабильности}. Оценка коэффициента $\widehat {\beta}$ асимптотически смещена к нулю, при этом смещение зависит  от отношения шум-сигнал. Эта величина ещё носит название {\bf смещения затухания} \emph{(attenuation bias)}. Такая терминология интуитивно понятна, т.к. оценки предельного воздействия изменения $\mathbf{x^*}$ на изменение $y$, полученные исследователем, затухают, уменьшаются из-за появления ошибки наблюдений регрессора $x^*$. 
  
Отметим также, что
\[
\Var[y|x]=\sigma^2_u+\frac{{\beta}^2\sigma^2_v\sigma^2_{x^*}}{\sigma^2_{x^*}+\sigma^2_v}\geqslant \sigma^2_u.
\]
Из этого следует, что ошибка измерений не только занижает оценки коэффициентов, но и увеличивает дисперсию ошибки регрессии. И уменьшение дисперсии ошибок измерений определённо приведёт к снижению дисперсии ошибок регрессии.

Если бы в описанное выше уравнение регрессии был включен свободный член, смещенной была бы его МНК-оценка $\bar{y} - \widehat{\beta}\bar{x}$,  где $(\bar y, \bar x)$ --- выборочные средние значения, которые всё ещё представляют собой несмещенные оценки средних значений всей генеральной совокупности. Крэгг (1994) предложил термин {\bf <<смещение заражения>>} \emph{(contamination bias)} для описания влияния ошибок измерения на другие параметры в регрессионном уравнении.

В качестве примера рассмотрим регрессию логарифма почасовой заработной платы на количество лет образования. Предположим, что продолжительность образования $x^*$ измерена с ошибкой, также предположим, что стандартное отклонение истинной величины продолжительности образования равно 2, а стандартное отклонение ошибок измерений равно 1, так что $\sigma^2_{x^*}=4$, $\sigma^2_v=1$, и $\sigma^2_x=5$. Тогда $\plim\widehat{\beta}=0.8 \times \beta$. МНК-оценка коэффициента наклона 0.04 тогда означает, что один дополнительный год обучения в школе соответствует 5\%-ому увеличению заработной платы, а не 4\%-ому.

\subsection{Обобщения} 

При обобщении этого простого, но изящного результата, исследователи часто задаются вопросом, является ли затухание общей чертой всех моделей ошибок измерения и что если затуханию подвергаются все параметры? Хотя результат не обязательно распространять на более общие модели, он может послужить хорошей отправной точкой. Хаусман (Hausman, 2001) назвал смещение затухания, вызванное ошибками измерения, <<железным законом эконометрики>>. 

Если ошибки измерений предполагается некоррелированной с истинной ненаблюдаемой переменной, такие ошибки называются <<классическими>>. Несмотря на удобство, эта предпосылка иногда не выполняется. На самом деле, в некоторых случаях она и не может выполняться. Например, если $x$ является бинарной переменной, ошибки измерений будут ошибками классификации. Если в результате неправильной классификации 0 будет измерен как 1 и наоборот, то ошибки измерений должны будут коррелировать с истинными значениями.

Когда в модель включается больше одного регрессор, обозначим $\mathbf{X^*}=[\mathbf{x^* Z}]$. Как и в предыдущем случае, мы предполагаем, что только один регрессор наблюдается с ошибкой измерения, то есть $x=x^*+v$. Тогда выражение для оценки коэффициента при $x$ методом наименьших квадратов будет следующим:
\begin{equation}
\plim\widehat{\beta}_{x|\mathbf{Z}}=\beta \left[ 1-\frac{\sigma^2_v}{\sigma^2_{x^*}(1-R^2_{x^*,\mathbf{Z}})+\sigma^2_v}\right],
\end{equation}
где $R^2_{x^*,\mathbf{Z}}$ представляет собой $R^2$ во вспомогательной регрессии $\mathbf{x^*}$ на $\mathbf{Z}$. Формула (26.9) представляет собой другой вариант формулы (26.8), только здесь мы вместо дисперсии $x^*$ мы считаем дисперсию после исключения влияния $\mathbf{Z}$ на $\mathbf{x^*}$. И снова  МНК-оценки смещены к нулю, хотя и меньше, чем в случае одного регрессора. Коэффициенты при регрессорах, измеренных без ошибок, также несостоятельны, при этом направление смещения зависит от $\mathbf{\Sigma_{x^*x^*}}$ (Леви, 1973). Этот эффект может быть снова назван смещением заражения. Смещение затухания в рассмотренном случае сильно зависит от предпосылки об аддитивности ошибок измерений. 

Когда более чем один регрессор измерен с ошибками, априори определить направление смещения не получается, хотя в любой рассматриваемой задаче оно может быть выведено, исходя из знания о $\mathbf{\Sigma_{x^*x^*}}$ и $\mathbf{\Sigma_{vv}}$. В большинстве работ предполагается, что ошибки измерения существуют только для одного регрессора, в этом случае оценка смещена к нулю. Интуиция, основанная на рассмотренных выше примерах, позволяет заключить, что если ошибки измерений разных регрессоров независимы, тогда каждая из них будет способствовать затуханию своего коэффициента регрессии, и к росту условной дисперсии. Крэгг (1994) анализировал множественную регрессионную модель с ошибками измерений и изучал взаимосвязь между смещениями разных регрессоров.

\subsection{Ошибки измерений в линейных моделях панельных данных}
 
Эффект от ошибок измерений в регрессорах становится более сложным в случае панельных данных.

Рассмотрим сквозную модель для панельных данных $y_{it}=\beta x^*_{it}+u_{it}$, где мы наблюдаем $x_{it}=x^*_{it}+v_{it}$, при этом для упрощения рассмотрим скалярный регрессор. Полученные ранее результаты по-прежнему верны, если мы рассматриваем пространственный срез данных. Тем не менее, если мы оцениваем модель, используя наблюдения за несколько лет, нам необходимо скорректировать  предыдущие результаты, так как регрессор $x^*_{it}$ будет скорее всего положительно коррелирован, нежели  будет иметь независимые наблюдения по $t$ для определённого $i$. Например, если мы строим регрессию в первых разностях
\begin{align*}
\Delta y_{it}&=\beta \Delta x^*_{it}+ \Delta u_{it} \\
&= \beta \Delta x_{it}+\Delta u_{it}-\beta \Delta v_{it}
\end{align*}
(см. Главу 21.6) и определив параметр $\rho = Cor[x^*_{it}, x^*_{i,t-1}]$, получим
\begin{align*}
\plim\widehat{\beta}&=\beta + \left( \plim \frac{1}{N} \sum \limits^{N}_{i=1}(\Delta x_{it})^2 \right)^{-1} \left( \plim \frac{1}{N} \sum \limits^{N}_{i=1}(\Delta x_{it} \Delta u_{it}-\beta \Delta x_{it} \Delta v_{it}) \right) \\
&= \beta - \frac{2\beta \sigma^2_v}{2(1-\rho)\sigma^2_{x^*}+2\sigma^2_v} \\
&=\beta-\frac{\beta \sigma^2_v}{(1-\rho)\sigma^2_{x^*}+\sigma^2_v},
\end{align*}
используя равенства$\Var[\Delta v_{it}]=2\Var[v_{it}]$ и $\Var[\Delta x^*_{it}]=2(1-\rho)\Var[x^*_{it}]$.

Масштаб несостоятельности увеличивается по сравнению с пространственной выборкой, если $\rho>0$. Более того, при $\rho\rightarrow 1$, что вероятно в случае панельных данных, несостоятельность становится очень сильной. Она может быть снижена путём взятия разностей, которые отстоят друг от друга по времени на $m>1$ шагов, так как $\Corr[x^*_{it}, x^*_{i,t-m}]$ будет убывать с ростом $m$.

\section{Стратегии идентификации} 

Принято считать, что модели с ошибками измерения не могут быть идентифицированы без введения дополнительных предпосылок. Это утверждение может быть интерпретировано в контексте частного  случая парной регрессии следующим образом. Одно  значение оценки параметра $\widehat{\beta}$, или точнее, его предела по вероятности, будет соответствовать бесконечному количестве комбинаций параметра $\beta$ и отношения шум-сигнал $s$. А при введении дополнительных предпосылок в модель можно будет исключить некоторые комбинации упомянутых параметров, потенциально соответствующие имеющимся данным. Если дополнительные ограничения позволяют получить единственное решение, модель считается точно идентифицированной. Если введённых ограничений больше, чем достаточно, чтобы однозначно идентифицировать параметры, модель будет переидентифицированной.

Общая стратегия идентификации моделей ошибок измерений заключается в получении границ на значения, а не точечных оценок интересующих параметров, если нет дополнительной априорной информации о данных. Если доступны дополнительные данные или информация об ошибках измерений, то осуществимыми становятся и другие стратегии идентификации, такие как метод инструментальных переменных или метод моментов. 
Наличие дополнительной информации об ошибках измерения --- это широкое понятие, которое включает одну из самых старых стратегий идентификации --- использование инструментальных переменных, которые выступают связующим звеном между истинными ненаблюдаемыми переменными и соответствующие им переменными. Например, использование дополнительной информации может привести к состоятельной оценке коэффициента затухания, $\frac{\sigma^2_{x^*}}{\sigma^2_{x^*}+\sigma^2_{v}}$, приводя к возможности скорректировать оценки на это смещение. Наконец, дублированные данные или контрольные данные могут быть доступны, и использованы как информация о моментах ошибок измерений. Эти возможности будут подробнее разобраны далее.

\subsection{Ограничения на параметры регрессии}
 
Обратимся снова к множественной регрессии из Главы 26.2. Рассмотренная в том случае модель требует положительной полуопределённости матриц $\mathbf{\Sigma_{x^*x^*}}$, $\mathbf{\Sigma_{vv}}$ и $\sigma^2$. Вместе с условиями ортогональности они могут быть использованы для построения ограничений на интервал, в котором должны лежать коэффициенты. Клеппер и Лимер (1984), а также Вансбик и Мейер (2000) рассматривали эту проблему в общей её постановке. Наиболее доступный частный случай граничного  подхода –-- построение обратной регрессии, описанное далее.

\subsection*{Обратная регрессия} 

В простой парной регрессии с переменными $(x, y)$ {\bf прямой} называют регрессию $y$ на $x$, тогда как {\bf обратной} --– регрессию $x$ на $y$. В общем случае множественной регрессии с $K$ независимыми переменными существует единственная прямая регрессия и $K$ обратных. Каждая обратная регрессия содержит в левой части экзогенную переменную, измеренную с ошибками, а в правой части --- другие экзогенные переменные и $y$. В случае парной регрессии с ошибками измерений легко показать, что оценки коэффициентов наклона в прямой и обратной постановке определяют верхнюю и нижнюю границы для истинного значения коэффициента. Этот результат может быть использован при анализе эффектов ошибок измерения. Лимер (1978) приводит отличное обсуждение логики  обратной регрессии.
 
Для начала приведём обоснование использования обратной регрессии в приложении к простой парной регрессии с ошибками измерения:
\begin{align}
y&=\beta x^*+u, \\
x&=x^*+v, \notag
\end{align}
где $u$ --– ошибка регрессии  и $v$ --– ошибка измерений, которая представляет собой отклонение наблюдаемых значений $x$ от истинных значений переменной $x^*$, которая включена в регрессию. Будем предполагать, что $u \sim \cN [0,\sigma^2_u]$ и $v \sim \cN [0,\sigma^2_v]$.

В соответствии со структурным подходом Солари (1969) и Лимера (1978), рассмотрим $x^*$ как  неизвестный параметр в функции правдоподобия. Общая функция правдоподобия для имеющихся данных $(\mathbf{y}, \mathbf{x})$ выглядит как
\begin{align}
L(\mathbf{x^*}, \beta, \sigma^2_v)\propto ( \sigma^2_u)^{N/2} \exp \left[ -\frac{1}{2\sigma^2_u}(\mathbf{y-\beta x})'(\mathbf{y-\beta x}) \right] \notag \\
\times ( \sigma^2_v)^{N/2} \exp \left[ -\frac{1}{2\sigma^2_v}(\mathbf{x^*-\beta x})'(\mathbf{x^*-\beta x}) \right].
\end{align}
Эта функция не определена в точках, удовлетворяющим условиям $\sigma^2_u=0$ и $\mathbf{x^*}=\mathbf{x}$, или условиям $\sigma^2_v=0$ и $\mathbf{y}=\mathbf{\beta x^*}$. Если мы просто минимизируем  функцию правдоподобия там где она определена с учетом ограничений, мы получим два скалярных параметра регрессии, $\widehat{\beta}_D=\mathbf{y'x/x'x}$ для прямой регрессии и $\widehat{\beta}_R=\mathbf{y'x/y'y}$ для обратной. Интуитивно, если $\mathbf{x}$ измерен без ошибок, то $\mathbf{y}$ случаен, а $\mathbf{x}$ --- нет, тогда прямая регрессия имеет осмысленную интерпретацию в терминах условных ожиданий. Если же  $\mathbf{x}$ --- стохастический (измерен с ошибкой), то условное ожидание $\Expect[\mathbf{x|y}]$ имеет смысл, так как система из двух уравнений сводится к уравнению $x=(1/\beta)y-u/\beta+v$. Таким образом, обратная регрессия даёт МНК-оценку $\widehat{1/\beta}$. Напрямую можно проверить, что
\begin{align}
r^2_{xy} \widehat{\beta}_R&=\widehat{\beta}_D, \\
\widehat{\beta}_D &< \beta < \widehat{\beta}_R, \notag
\end{align}
где $r^2_{xy}$ --- квадрат выборочной корреляции между $x$ и $y$; границы свидетельствуют о том, что $\widehat{\beta}_D$ --- заниженная оценка $\beta$ и $\widehat{\beta}_R$ --- завышенная оценка. Отметим, что этот интервал может быть очень широким для микроэкономических данных, где почти всегда $r^2_{xy}<0.5$ и даже величина $r^2_{xy}<0.1$ довольно распространена.

Лимер (1978) предложил модель, в которой $(y, x*)$ имеют двумерное нормальное распределение со средним $(\beta \bar{x}^*, \bar{x}^*)$ и ковариационной матрицей 
\begin{equation}
\mathbf{\Sigma}=
\begin{bmatrix}
\sigma^2_u+\beta^2\sigma^2_{x^*} & \beta\sigma^2_{x^*} \\ \beta\sigma^2_{x^*} & \sigma^2_{x^*}+\sigma^2_v
\end{bmatrix} .
\end{equation}
Он показал (Лимер, 1978, стр. 239--240), что для этой модели функция правдоподобия достигает максимума при любом значении $\widehat{\beta}_D$ из промежутка от оценки из прямой регрессии $\beta$ до оценки из обратной регрессии $\widehat{\beta}_R$.

Из упомянутого анализа следует, что хотя $\beta$ и не идентифицируется, можно всё же получить состоятельные границы для искомого значения. Здесь успешно применяется {\bf метод идентификации границ коэффициентов}. Этот результат может быть легко распространён на случай множественной регрессии, а которой только один регрессор измерен с ошибками (Боллингер, 2003). Клеппер и Лимер (1984) рассмотрели обобщение случая множественной регрессии с $K$ независимыми переменными, предположив, что все они могут иметь ошибки измерений. В этом случае будет одна прямая регрессия и $K$ обратных. После оценивания каждая обратная регрессия перенормируется чтобы коэффициент при $y$ в левой части оказался равным единице. Пусть $\mathbf{\widehat{\beta}_{D}}$ --- вектор оценок из прямой регрессии и $\mathbf{\widehat{\beta}_{R,j}}$ --- вектор оценок из $j$-ой обратной регрессии. Согласно результатам Клеппера и Лимера (1984), если коэффициенты из прямой и обратных регрессий лежат в одном ортанте, то множество возможных значений $\mathbf{\beta}$ --- это выпуклая оболочка множества коэффициентов для прямой и обратной регрессии; т.е. $\mathbf{\beta} \in \{ \mathbf{\widehat{\beta}|\widehat{\beta}=\lambda_D \widehat{\beta}_D}+\lambda_1 \widehat{\beta}_{R,1}+ \dots \lambda_k \widehat{\beta}_{R,K} \}$, где $\lambda$ --- неотрицательные веса, в сумме дающие единицу. Наименьший коэффициент из прямой и обратных регрессий является нижней границей интервала, а наибольшие --- верхней границей. Таких границ не существует, если значения коэффициента меняют знак.

В дополнение к работе Клеппера и Лимера (1984) можно упомянуть несколько работ, в которых этот подход применятся на практике. Грин (1983) и Голдбергер (1984) использовали метод обратных регрессий для измерения масштабов дискриминации в контексте размера заработной платы. Боллингер (2003) измерял разницу в доходах белого и черного населения на основе модели зарплаты и человеческого капитала. Боллингер (1996) обратился к методу границ при построении  регрессии на категориальную дамми-переменную, при этом имели место ошибки классификации.

\subsection{Идентификация с помощью инструментальных переменных} 

Одним решением проблемы идентификации можно назвать введение одного или более моментных ограничений, которые позволяют идентифицировать модель. Моментное ограничение, как правило, состоит в том, что существует некоторая инструментальная переменная коррелированная или имеет причинно-следственную связь с ошибочно измеренной переменной. Более того, она не коррелирует и не имеет причинно-следственных связей с зависимой переменной модели. Добавляя это ограничение в исходную модель, мы теоретически получаем решение проблемы идентификации.

Исторически, оценивание методом инструментальных переменных было предложено как возможный путь  решения проблемы ошибок измерения в линейных моделях (Райерсол, 1941; Дарбин, 1954). Подход с использованием инструментальных переменных также обоснован, когда одна или более переменная в правой части уравнения является эндогенной, а значит коррелированной с ошибкой регрессии. Модель линейных одновременных уравнений и линейная модель ошибок измерений сходны по структуре, и, таким образом, использование оценок метода инструментальных переменных в случае возникновения ошибок измерений вполне естественно.

Снова рассмотрим  линейную модель с инструментальными переменными из Разделов 4.8 и 6.4, где $\mathbf{y=X\beta+u}$ и $\Expect[\mathbf{u|X}] \neq \mathbf{0}$. Мы можем использовать двухшаговую МНК оценку, если доступен набор валидных инструментов $\mathbf{Z}$, таких что $\dim[\mathbf{Z}] \geqslant \dim[\mathbf{X}]$.

С помощью теста Хаусмана на эндогенность в регрессорах можно проверить наличие ошибок измерений, см. Раздел 8.3. Возможны несколько вариантов теста, один из которых подробнее рассмотрен в Разделе 8.4.

Основная проблема применения оценок метода инструментальных переменных заключается в сложности нахождения валидных инструментов на практике. Хорошие инструменты удовлетворяют двум требованиям: некоррелированность с ошибкой регрессии (для состоятельности) и высокая корреляция с заменяемыми переменными (для эффективности). Такие инструменты не так просто найти. Хотя в идеале можно создать валидные инструменты проанализировав  взаимосвязи регрессоров и ошибки модели, как правило, обычно на практике используются ad-hoc методы. В отличие от подхода, подразумевающего полное описание системы, ad-hoc метод проще и требуется для него меньше. Отметим, что требования к валидности инструментов не порождают никакой процедуры выбора инструментов. Технические требования могут быть удовлетворены, если переменная не имеет причинно-следственной связи с изучаемым явлением. Нужно найти переменную, сильно коррелированную с регрессором(-ами) и некоррелированную с ошибкой регрессии. Несколько интересных примеров применения этой идеи можно встретить в литературе; см., например, работу Ангриста (1990). Даже если такая переменная найдена, её использование может давать  спорные и противоречивые результаты.

В качестве примера предположим, что у нас имеется несколько возможных инструментов для пространственной регрессии доходов на образование. Во-первых, если доступны данные о родных братья или сёстрах, то их уровень образования может быть использован как инструмент, так как уровни образования братьев и сестёр предполагаются коррелированными. Состоятельность оценок метода инструментальных переменных тогда основывается на некоррелированности ошибки измерений $v$ и любой ошибкой измерений уровня образования родных братьев и сестёр. Во-вторых, также могут быть использованы такие переменные, связанные с уровнем образования, как уровень образования родителей или их доход. Использование более широкого множества инструментальных переменных может быть связано с риском использования слабо коррелирующих с $x$ переменных, что ведёт к неточности и возможно плохим оценкам на малых выборках. В-третьих, более, чем один вопрос об уровне образования может быть задан респонденту в рамках опроса, или же информация об образовании может быть доступна из опросов прошлых лет, если речь идёт об исследовании данных панельной структуры. Такие инструменты, скорее всего, будут сильно коррелировать с $x$, но в этом случае сложно поверить в выполнение предпосылки о некоррелированности между ошибками измерений в $x$ и $z$.
 
Лаги переменных часто используются как инструменты, но и они будут иметь ошибки измерений, так что использование метода инструментальных переменных оправдано только в случае, если отсутствует автокорреляция между ошибками измерений.

Последствия от ошибок измерений могут иметь больший размах в случае анализа панельных данных. Так как в этом случае измерения $x^*_{it}$ берутся за несколько периодов, то оценивание методом инструментальных переменных может быть использовано для получения состоятельных оценок параметров в предположении о некоррелированности ошибок измерений во времени. См. Хсяо (1986, стр. 63-65).

\subsection{Идентификация с помощью дополнительных моментных ограничений} 

Предположения о распределении ошибок регрессии и ошибок измерения $(u, v)$ могут обеспечить возможность идентификации модели. Есть один важный случай, в котором использование информации или допущений относительно распределения ненаблюдаемого истинного значения измеряемой переменной помогает при идентификации. Предположения о нормальности совместного закона распределения $(y, x, x^*)$ вместе с предположением о то, что ошибки регрессии и ошибки измерений являются независимыми одинаково распределёнными нормальными случайными величинами  $v\sim \mathcal{N}[0, \, \sigma^2_v]$ и   $u\sim \mathcal{N}[0, \, \sigma^2_u]$, недостаточно для идентифицируемости модели ошибок измерения. Тем не менее, предпосылка о том, что первые четыре момента $(x^*, u, v)$ существуют и третий момента каждой величины и перекрёстные моменты третьего порядка --- ненулевые,  достаточна для идентификации, что будет продемонстрировано ниже. Данная предпосылка говорит о ненормальности совместного распределения.

Снова построим модель вида (26.10)
\begin{align*}
y&=\beta x^* + u, \\
x&=x^*+v,
\end{align*}
с приведенной формой  $y=\beta x+ \epsilon$, где $\epsilon = u- \beta v$, и которая оценивается с помощью инструментальных переменных. Однако теперь мы добавляем новую информацию: распределение $x^*$ ненормально в том смысле, что асимметрия и эксцесс отличны от нормального Крэгг (1997); Даженэ and Даженэ, (1997); Вансбик и Мейер (2000). Эти предпосылки приводят к следующим шести условиям:
\begin{align*}
&\Expect [(xy)x]=\beta \Expect [x^{*3}], &\ &\Expect [(xy)u]=0, \\
&\Expect [(x^2)x]=\Expect [x^{*3}]+\Expect [v^3],&\ &\Expect [(x^2)u]=-\beta \Expect [v^3], \\
&\Expect [(y^2)x]=\beta^2 \Expect [x^{*3}],&\ &\Expect [(y^2)u]=-\beta \Expect [\epsilon^3],
\end{align*}
Первая строчка означает, что искусственно созданная переменная $x_i y_i$ будет являться валидным инструментом, если $\Expect[x^{*3}_i] \neq 0$.  Из второй строчки видно, что $x^2_i$ будет валидным инструментом, если  $\Expect[x^{*3}_i] \neq 0$, но  $\Expect[v^3_i] = 0$, то есть, если $x^*$ не распределён нормально, но $v$ имеет симметричное распределение. Действительно, чем больше коэффициент асимметрии, тем лучше инструмент. Тем не менее, так как переменная $x^*$ --- ненаблюдаемая, то любые заключения относительно неё должны базироваться на $x$. В последней строчке содержится утверждение, что $y_i^2$ будет валидным инструментом, если третий момент $x^*$ не будет равным нулю при том, что третий момент $\epsilon$ будет нулевым.

При использовании этих моментных условий метод инструментальных переменных может применяться для получения состоятельных оценок параметров модели. Этот пример показывает, как дополнительные предположения о моментах распределения могут помочь для создания хороших инструментов даже в условиях, когда кроме $(y_i, x_i)$ нет доступных данных.

\subsection{Дублированные данные} 

Альтернативное решение возможно, если может быть оценена дисперсия ошибок измерений. В этом случае идея состоит в том, что мы можем скорректировать матрицу выборочных вторых моментов регрессоров $\mathbf{X'X}$ на величину, зависящую от дисперсии и ковариаций ошибок измерений. Обратим внимание, что мы не пытаемся скорректировать сами наблюдения. Вместо этого корректируются выборочные моменты, так как оценка  является функцией от выборочный моментов. Основная идея также обобщается до более сложных моделей.

Когда ковариационная матрица ошибок измерений $\mathbf{\Sigma_{vv}}$ известна, состоятельная оценка коэффициента $\mathbf{\beta}$ может быть получена с помощью формулы
\begin{equation}
\mathbf{\tilde{\beta}}=(\mathbf{X'X}-N \mathbf{\Sigma_{vv}})^{-1} \mathbf{X'y},
\end{equation}
где $N$ –-- размер выборки. Эта оценка состоятельна в силу того, что
\begin{align*}
\mathbf{\tilde{\beta}}&=\plim(N^{-1}\mathbf{X'X}-N \mathbf{\Sigma_{vv}})^{-1} \plim N^{-1} \mathbf{X'y}  \\
&= (\mathbf{\Sigma_{x^*x^*}} + \mathbf{\Sigma_{vv}} - \mathbf{\Sigma_{vv}})^{-1} \mathbf{\Sigma_{x^*x^*} \beta} \\
&= \mathbf{\beta},
\end{align*}
где предел $\plim N^{-1} \mathbf{X'y}=\mathbf{\Sigma_{x^*x^*} \beta}$ получен с использованием равенства $\mathbf{X}=\mathbf{X^*}+\mathbf{V}$ и $\mathbf{y}=\mathbf{X} \beta + (\mathbf{u}-\mathbf{V \beta})$. Для подробного описания способов оценивания $\mathbf{\Sigma}_{vv}$ в практических приложениях можно обратиться к работе Крашинского (2004).

{\bf Дублирование (повторение, реплицирование) данных} –-- это ситуация, в которой доступна несмещенная оценка ненаблюдаемой переменной $\mathbf{X^*}$. Предположим, что ошибка измерений аддитивна и имеется наблюдаемая величина $\mathbf{X}$:
\[
\mathbf{X}=\mathbf{X^*}+\mathbf{V}.
\]
Если $\mathbf{X}$ является несмещенной оценкой $\mathbf{X^*}$, то $\Expect[\mathbf{V|X^*}]=\mathbf{0}$. Если данные дублированы, то это означает, что мы имеем, по крайней мере, два доступных наблюдения $\mathbf{X}$. Также это означает, что с использованием повторных измерений $\mathbf{X}$ мы может получить оценки моментов $\mathbf{V}$, предполагая, что ошибки повторных измерений  не коррелированы.

Предположим, что в нашем распоряжении две скалярных величины (реплики) $X(1)$ и $X(2)$, такие что $X_{(j)}=X^*+V_{(j)}, \, j=1, \, 2$. Тогда $\Var[V_{(j)}]=\Expect[X^2_{(j)}]-\Expect[X_{(1)}X_{(2)}]$, которая может быть оценена, исходя из выборочного среднего, как $N^{-1} \Sigma_i [X^2_{(j),i}-X_{(1),i}X_{(2),i}]$. В этом случае параметры регрессии могут быть оценены  по формуле (26.14). 

В качестве примера предположим, что мы хотим предсказать средний  балл (GPA) за первый год обучения в колледже, используя результаты за экзамен SAT в старших классах (тест на проверку академических способностей). Известно, что наблюдаемые баллы за SAT для одного и того же человека различаются в разных попытках сдачи экзамена. Положим, что $x^*$ отражает истинный балл за SAT, а $x_1$ и $x_2$ отражают наблюдаемые результаты двух разных экзаменов. Тогда $x_1=x^*+v_1$, $x_2=x^*+v_2$, и предполагается, что $v_1$ и $v_2$ независимы и имеют одинаковую дисперсию $\sigma^2_v$. Из этого следует, что $\Cov[x_1, x_2]=\sigma^2_{x^*}$, $\Var[x_1]=\Var[x_2]=\sigma^2_{x^*}+\sigma^2_v$ и $\Corr^2[x_1,x_2]=\sigma^2_{x^*} / (\sigma^2_{x^*}+\sigma^2_v)$. Согласно исследованиям, тест имеет надёжность 0.9, что означает, что корреляция между результатами двух сдач теста равна 0.9 и квадрат корреляции равен 0.81. Таким образом, $\sigma^2_{x^*} / (\sigma^2_{x^*}+\sigma^2_v)=0.81$. Из (26.8) следует, что $\plim \widehat{\beta}=0.81 \times \beta$, то есть, учет ошибок измерений балла за SAT помогает лучше предсказать результаты первого года обучения (GPA), чем МНК.

\subsection{Верифицирующие данные} 
Иногда верифицирующая выборка формируется для дополнительной проверки правильности исходных ответов. Хотя {\bf верифицирующая выборка} (validation sample) и принадлежит исследуемой совокупности, она может быть получена из другого независимого источника. Например, пациенты могут заполнять анкету об оказанных им медицинских услугах, и те, кто оказывают эти услуги, могут ответить на вопросы в качестве верифицирующей информации. Другой пример касается работников, которые могут предоставить информацию о чем-то, тогда как верифицирована она может быть на основе информации, полученной от работодателей. Один из лучших примеров в экономике –-- верификация на базе Панельного исследования динамики доходов (Panel Study of Income Dynamics, PSID) Баунда и др. (1994).

Пусть $\mathbf{X}$ –-- матрица регрессоров, измеренных с ошибками, размерности $N \times K$  и пусть $X_v$ –-- матрица контрольных измерений размерности $M \times K$. Мы можем использовать верифицирующие данные, построив регрессию $X_v$ на $\mathbf{X}$ и сгенерировав <<предсказанные>> значения $\mathbf{X[X'X]}^{-1} \mathbf{X'X}_v$, которые послужат заменой искаженной матрице $\mathbf{X}$. Для нелинейных моделей используется более сложный подход, см. Ли и Сепански (1995).

Использование сгенерированных регрессоров в интересующей исследователя регрессии  может быть полезной стратегией, если предсказания взяты из хорошо подогнанной регрессии. Сгенерированные регрессоры являются оценками истинных значений и, таким образом, связаны с неопределённостью оценивания. Эта неопределённость должна быть учтена при оценивании дисперсии оценок коэффициентов. Соответствующая теория описана в Разделе 6.8.

\section{Ошибки измерений в нелинейных моделях} 

Нелинейные модели, что должно быть достаточно понятно, включают в себя огромный спектр моделей. Получение общих результатов, таких как смещение затухания, для широкого класса моделей, представляет собой непростую задачу. Довольно часто общие результаты получают при упрощающих  предположениях, в то время как особые результаты можно получить для частных случаев. Таким образом, неудивительно, что развитие темы нелинейных моделей в литературе порождает много процедур и подходов, специфических и применимых к определённого рода моделям. Например, в случае моделей бинарного выбора с ошибками измерений в левой части естественно обращать внимание на проблему неверной классификации; в случае счетных моделей  с ошибками измерений в левой части также естественно концентрироваться на проблеме занижения и завышения показателей. Мотивированный такими сложностями, Хсяо (1992) предложил переключить внимание от поиска решений для общего вида моделей на более специальные типы задач. Однако в поиске специфических результатов есть опасность узости и потери части общих результатов. Таким образом, мы начинаем рассмотрение с некоторых общих результатов.

\subsection{Идентификация с помощью инструментальных переменных} 

Общая техника для линейных моделей ошибок переменных --– метод инструментальных переменных. Для нелинейной по независимым переменным регрессионной модели было показано Амэмией (1985), что оценка метода инструментальных переменных в общем случае несостоятельна, кроме случая, когда выполнена предпосылка о сжимающейся (shrinking) ковариационной матрице.

Простое представление описанной выше идеи основано на регрессионном уравнении
\begin{equation}
y=\beta_0 +\beta_1 f(x^*)+ \epsilon,
\end{equation}
где $f(x^*)$ --- гладкая, дифференцируемая ограниченная функция скалярного регрессора $x^*$, измеренного без ошибок. Наблюдаемая переменная определяется как $x=x^*+v$, где $v$ --– ошибка измерений. Заменяя переменную $x^*$ и используя разложение в ряд Тейлора функции $f(x-v)$ в окрестности точки $x$, получаем
\begin{equation}
y=\beta_0 +\beta_1 f(x)+ \epsilon - \beta_1 f^{(1)}(x)v + \beta_1 \sum \limits^{\infty}_{j=2} f^{(j)}(x)(-v)^j/j!,
\end{equation}
где $f^{(j)}(\cdot)$ --- производная $j$-ого порядка функции $f(\cdot)$. В случае квадратичной функции $f(x)=x^2+\gamma x$, имеем $f^{(1)}(x)=2x+ \gamma$, $f^{(1)}(x)=2$ и $f^{(j)}(x)=0, \, j>2$. Тогда
\begin{align}
y&=\beta_0 + \beta_1 (x^2+ \gamma x) + \epsilon - \beta_1 (2x+ \gamma)v + \beta_1 2v^1 /2 \notag \\
&= \beta_0 + \beta_1 x^2 + \beta_1 \gamma x + (\epsilon- \beta_1 xv - \beta_1 \gamma v + \beta_1 v^2),
\end{align}
Следовательно, валидные инструменты должны  коррелировать с $x^2$ и $x$, но не коррелировать с $u=(\epsilon- \beta_1 xv - \beta_1 \gamma v + \beta_1 v^2)$. Понятно, что недостаточно, чтобы $v$ и $\epsilon$ по отдельности не коррелировали с инструментами. Это означает, что инструментальная переменная для $f(x)$ должна удовлетворять более жестким требованиям, чем в линейном случае.

В более общем случае, используя аппроксимацию Тейлора, Амэмия показал, что инструментальные переменные не дают состоятельных оценок в нелинейных моделях ошибок переменных, так как остаточный член включает в себя как ошибку измерений, так и наблюдаемую зараженную ошибками переменную. Таким образом, невозможно найти инструментальную переменную, которая бы сильно коррелировала с наблюдаемой переменной и не коррелировала с остаточным членом. Более того, с практической точки зрения, непросто найти подтверждение валидности инструмента при оценивании из-за ограниченности информации о латентной переменной $(x^*)$ и ошибках измерений.
 
\subsection{Идентификация с помощью дублированных данных} 

Имея в виду сложности применения метода инструментальных переменных, приведём ещё два существующих альтернативных подхода.

Первый подразумевает введение очень жестких предположений об условном распределении ненаблюдаемой переменной $x^*$ при фиксированном наблюдаемом $x$. Такие предположения, усиленные техническими условиями, позволяют идентифицировать параметры модели. Такой подход можно встретить у Амэмии (1985) и Хсяо (1989).

Второй подход предусматривает возможность получения большого количества измерений каждого ненаблюдаемого $x^*$, обозначенного как $x_{(j)}$. Тогда среднее по реплицированным измерениям для каждого $x^*$ служит заменой для ненаблюдаемого регрессора. Мы получаем состоятельные оценки нелинейной модели, так как ковариация ошибок измерений сокращается до нуля при увеличении числа реплик, см. работу Амэмия (1985). К сожалению, в эконометрике этот способ редко воплотим.

Так как не существует типичной структуры нелинейной модели ошибок измерения, которая может быть использована для идентификации и оценивания регрессионной модели, рассмотрим несколько специальных нелинейных регрессий.

Хаусман, Ньюи и Пауэлл (1995) анализировали полиномиальные кривые Энгла на основе данных Исследования потребительских расходов (Consumer Expenditure Survey). Их полиномиальная функция была линейна по параметрам. Они доказали, что, при условиях регулярности, и инструментальные переменные, и дополнительные измерения могут быть использованы для получения состоятельных и асимптотически нормальных оценок. Применительно к этому случаю, ближайшие квартальные данные использовались и как дублирующая информация и как инструментальные переменные. Далее они предположили, что нелинейная функция в общем может быть аппроксимирована полиномиальной функцией. Тем не менее, они признали, что инструментальные переменные не могут быть введены в этом случае, так что требуются дополнительные измерения истинных регрессоров.

Ли (2002) предложил общий двухшаговый подход к проблеме нелинейных моделей ошибок переменных, который основан на дублированных измерениях. В качестве первого шага, на базе эмпирических характеристических функций и обратного преобразования Фурье получают непараметрические оценки для условной функции плотности латентных переменных. После получения оценки плотности строятся полупараметрические оценки нелинейного метода наименьших квадратов с использованием критерия минимального расстояния. Он установил состоятельность этих оценок. Эти оценки также робастны в том смысле, что не требуют никакого знания о функциональной форме латентных переменных. Подход Ли может быть применён в ситуации любых нелинейных моделей ошибок переменных, если доступны дублированные измерения. Тем не менее, асимптотическое распределение оценки не установлено.

\subsection{Ошибки измерений в зависимых переменных} 
В линейной регрессионной модели ошибки измерений в зависимой переменной увеличивают стандартные ошибки параметров регрессии, но не ведут к несостоятельности оценок. В нелинейной модели возникают ещё ряд последствий.

В одном из классов моделей предполагается наличие неверной классификации наблюдений в моделях качественного выбора. Эта идея породила много работ, связанных с ошибкой из-за декларирования неверных данных.

\subsection*{Модели дискретного выбора} 

Потерба и Саммерс (1995) в своём исследовании влияния страхования от безработицы на длительность безработицы с использованием данных Текущего обследования населения (Current Population Survey, CPS) обобщили вероятностную модель, чтобы учесть неправильную классификацию индивидов при смене статуса на рынке труда. Конкретнее, они фокусировались на возможных ошибках отнесения к одному из трёх классов: занятные, безработные и не входящие в рабочую силу. Они анализировали множественную логит-модель с учетом некоторых особенностей данных: предполагалось, что все безработные индивиды правильно ответили на вопрос о статусе занятости в первый месяц исследования. Их результаты показали, что страхование от безработицы увеличивает длительность безработицы, и что учитывание неправильного отнесения в группы по статусу на рынке труда усиливает  влияние страхования безработицы на её длительность. Тем не менее, их модель основана на предпосылке о том, что вероятность предоставления неправильных ответов фиксирована и не коррелирует с индивидуальными характеристиками, которая, с чем согласны сами авторы, на практике маловероятно выполняется. Хотя авторы утверждают, что оценки параметров состоятельные, Хаусман, Абревая и Скотт-Мортон (1998) спорят с ними, утверждая, что стандартные ошибки несостоятельно оценены из-за игнорирования изменчивости вероятности ошибки и не блочно-диагональной формы информационной матрицы.

Хаусман и др. (1998) предложили параметрический метод оценивания модели бинарного выбора с ошибками классификации. Тем не менее, их параметрический метод требует знания закона распределения ошибок. Они указывают на то, что оценки параметром могут быть несостоятельными, если предпосылка о типе распределения не выполняется. Также они предлагают двухшаговый полупараметрический метод. Основное условие для идентификации в этой модели, которое, как они показали, слабее условия для параметрического метода, состоит в том, что ожидаемое значение наблюдаемой зависимой переменной есть возрастающая функция от некоторого индекса. В сравнении с подходом Потерба и Саммерса (1995), их подход робастен в том смысле, что вероятность неверной классификации является функцией от индивидуальных характеристик. Используя CPS и PSID, они показали, что имеет место серьезные ошибки классификации в переменной смены работы.

Кляйн и Шерман (1997) разработали {\bf орбит-модель} (с чертами модели упорядоченного выбора и тобит-модели) для оценки возможного спроса на новый видео продукт. Они выявили, что потенциальные потребители завышают спрос. Орбит-модель представляет собой двухшаговую процедуру с оцениванием параметров стандартной тобит-модели для реального будущего спроса на первом шаге и с оцениванием функции связи между текущим предполагаемым спросом и реальным будущим спросом. Далее они доказывают состоятельность и асимптотическую нормальность оценок орбит-модели. Несмотря на это, идентификация модели требует выполнения предпосылки о том, что предполагаемый нулевой спрос действительно будет нулевым в будущем. Это может быть достаточно сильным предположением.
 
Хсяо и Сан (1999) использовали данные обследования спроса на сложную электронику. Они показывали, что респонденты могут давать смещенную информацию относительно спроса. Они предложили модель случайных ответов и модель одностороннего смещения для преувеличений.  В их модели  различные параметрические вероятности соответствуют истинному и альтернативному выбору.  Логит или пробит функции используются для истинных выявленных предпочтений. Они определили, что <<имеет место значительное смещение в ответах, и на анализируемом рынке ставки и эластичности цен являются более скромными, чем оценки, полученные на основе  предположений о сообщении потребителями своих истинных предпочтений>>.

\subsection*{Регрессии дискретных переменных} 
В нелинейной регрессии счетных переменных Камерон и Триведи (1998) предложили подход для моделирования  случайного занижения значений. Их модель порождает обобщенную пуассоновскую и отрицательную биномиальную модели для счетных данных, путем рассмотрения индикатора записи события.  Говоря точнее, для каждого наступления события вводится бернуллиевская случайная величина, являющееся индикатором того, что событие будет зафиксировано. При положительной вероятности того, что событие не будет зафиксировано,  распределение записанных событий будет иметь меньшее среднее и дисперсию, чем распределение действительно наступавших событий. Далее они исследуют оценку модели, полученную методом максимального правдоподобия, квази-обобщенным методом псевдо-максимального правдоподобия и методом моментов. Основываясь на методе Монте-Карло, они определили, что использование метода максимального правдоподобия хорошо в случаях выборок, превосходящих 500 наблюдений.

Джордан и др. (1997) применили метод ошибок в переменных в модели пуассоновской регрессии. В исследовании смертности от рака желудка в пяти регионах Японии они отметили, что регрессор (например, уровень ликопена в плазме) неизвестен, а при его оценке возникают случайные ошибки. Принимая предпосылку о том, что ошибка измерений нормально распределена, они применяют байесовский подход и получают апостериорное распределение параметров, используя сэмплирование Гиббса. Результаты говорят о том, что скорректированная модель даёт более точные оценки параметров, даже когда выборка мала.

\subsection{Пуассоновская регрессия с ошибками измерений в независимых переменных} 
Теперь мы более детально рассмотрим один  пример нелинейной регрессионной модели с аддитивными ошибками измерений в независимых переменных. Этот пример иллюстрирует и последствия от такого рода ошибок измерений, и доступные стратегии оценивания.

Гуо и Ли (2002) показали, что ошибки измерений в регрессорах в общем случае ведут к завышенной дисперсии наблюдаемых данных. Они также показали с использованием Монте-Карло симуляций, что смещение возникает в случае, если завышенная дисперсия, вызванная ошибками измерений, неверно смоделирована как возникшая вследствие ненаблюдаемой неоднородности. Таким образом, из наличия завышенной дисперсии не обязательно следует вывод о ненаблюдаемой гетерогенности.

Стефански (1989) и Накамура (1990) предложили {\bf скорректированную скоринговую оценку} \emph{(corrected score estimator)}, которая является состоятельной в случае наличия ошибок измерений. В частности, Накамура (1990) нашел в явном виде скоринговую функцию  для случая, когда ошибки измерений распределены нормально и доступны дублированные наблюдения. Гуо и Ли (2002) обобщили результаты Накамуры (Nakamura, 1990).

\subsection*{Ошибки наблюдений и завышенная дисперсия} 
В этом Разделе мы рассмотрим пуассоновскую регрессионную модель, в которой дискретная случайная переменная $y$ распределена на закону Пуассона с параметром $\mu = \exp(\mathbf{x^{*\prime} \beta})$, где $\mathbf{\beta}$ --- вектор параметров размерности $K \times 1$. Как известно, в пуассоновской регрессионной модели  на дисперсию наложено требование:
\begin{equation}
\Expect[y| \mathbf{x^*}]=\Var[y \mid  \mathbf{x^*}].
\end{equation} 
Если ошибка измерений имеет аддитивную форму, то
\[
\mathbf{x}=\mathbf{x^*}+ \epsilon,
\]
где $\epsilon$ предполагается независимой от ненаблюдаемой латентной переменной $\mathbf{x^*}$, обладающей нулевым средним и ковариационной матрицей $\mathbf{\Sigma_{\epsilon}}$. Это замечание включает в рассмотрение и случай, когда все или некоторые объясняющие переменные измерены с ошибками.

Ошибки измерений увеличивают дисперсию, см работу Чешера (1991). Это применимо к пуассоновской регрессии в том смысле, что, хотя (26.18) выполняется при фиксированном $\mathbf{x^*}$, однако результат меняется, если фиксировать  $\mathbf{x}$. Вместо этого мы получаем $\Expect[y| \mathbf{x}]<\Var[y| \mathbf{x^*}]$, отчасти потому, что $\Expect[y| \mathbf{x^*}] \neq \Expect[y| \mathbf{x}]$  и $\Var[y| \mathbf{x^*} \neq \Var[y| \mathbf{x^*}]$.

Если $g(\mathbf{x^*|x})$ обозначает условную плотность $\mathbf{x^*}$ при заданном $\mathbf{x}$, тогда, как Гуо и Ли показали, 
\begin{align}
\Expect[y| \mathbf{x}] &= \int \Expect[y| \mathbf{x^*}]g(\mathbf{x^*|x})\, d\mathbf{x^*} \notag \\
&= \int \Expect[y^2| \mathbf{x^*}]g(\mathbf{x^*|x})\, d\mathbf{x^*} - \int (\Expect[y| \mathbf{x^*}])^2 g(\mathbf{x^*|x})\, d\mathbf{x^*},
\end{align}
И, используя (26.18), условная дисперсия $y$ при заданном $\mathbf{x}$ будет следующей:
\begin{equation}
\Var[y| \mathbf{x^*}]= \int \Expect[y^2| \mathbf{x^*}]g(\mathbf{x^*|x})\, d\mathbf{x^*} -  \left[ \int \Expect[y| \mathbf{x^*}]g(\mathbf{x^*|x})\, d\mathbf{x^*} \right]^2.
\end{equation}

Сравнение (26.19) и (26.20) показывает, что первая компонента в скобках в (26.19) совпадает с первой компонентой в (26.20). Используя этот факт, Гуо и Ли  показали, что
\begin{equation}
 \left[ \int \Expect[y| \mathbf{x^*}]g(\mathbf{x^*|x})\, d\mathbf{x^*} \right]^2 \leqslant \int (\Expect[y| \mathbf{x^*}])^2 g(\mathbf{x^*|x})\, d\mathbf{x^*},
\end{equation}
что может быть интерпретировано, как завышение дисперсии вследствие наличия ошибок измерений.

\subsection*{Оценивание модели ошибок в переменных} 
Когда $\mathbf{x}$ искажен ошибками измерений, оценивание методом максимального правдоподобия или нелинейным методом наименьших квадратов, основанными на наблюдениях $(y, \mathbf{x})$ не  дает состоятельных оценок. Замена  регрессора $\mathbf{x^*}$ на $x$ называется <<наивной>> моделью.

Есть две темы для размышлений. Во-первых, почему ММП даёт несостоятельные оценки, когда появляются ошибки измерений? Во-вторых, возможно ли вообще получить состоятельные оценки? Ответ на второй вопрос –-- <<да>>, если мы применяем {\bf метод скорректированной скоринговой функции} для обобщенных линейных моделей, согласно работам Стефански (1989) и Накамуры (1990). 

Идея, лежащая в основе метода скорректированной скоринговой функции заключается в том, что условное распределение скорректированной оценки коэффициента при фиксированной независимой переменной $\mathbf{x^*}$ и зависимой переменной $y$, центрировано относительно ММП-оценки, что обеспечивает состоятельность оценки истинного значения интересующего параметра.

\subsection*{Несостоятельные и состоятельные оценки} 
Пусть есть выборка $N$ наблюдений $(y_i, \mathbf{x}^*_i), \, i=1, \dots,N$ , которые имеют распределение Пуассона с вероятностью
\[
\Prob[Y_i=y_i|\mathbf{x}^*_i]=\frac{e^{-\mu_i(\mathbf{\beta_0})}\mu_i(\mathbf{\beta_0})^{y_i}}{y_i!},
\]
где $\mu_i(\beta_0)=\exp(\mathbf{x}^{*\prime}_i \mathbf{\beta_0})$. При заданных наблюдениях $(y_i, \mathbf{x}^*_i), \, i=1, \dots, N$ ММП-оценки коэффициентов $\mathbf{\widehat{\beta}}$ состоятельные, если предел по вероятности среднего значения логарифма функции правдоподобия
\begin{align}
\plim N^{-1}\ln L(\mathbf{\beta})&=N^{-1} \sum \limits_i \{ -e^{\mathbf{x}^{*\prime}_i \mathbf{\beta}} + y_i \mathbf{x}^{*\prime}_i \mathbf{\beta} - \ln y_i! \} \notag \\
&= \Expect_{y,\mathbf{x^*}}[-e^{\mathbf{x^{*\prime}} \mathbf{\beta}} + y \mathbf{x^{*\prime}} \mathbf{\beta} - \ln y! ]
\end{align}
достигает максимума при $\mathbf{\beta} = \mathbf{\beta_0}$.

Предположим, что мы наблюдаем $\mathbf{x}_i$ вместо $\mathbf{x}^*_i$, где $\mathbf{x}_i=\mathbf{x}^*_i+\mathbf{\epsilon}_i$ и $\epsilon_i \sim \mathcal{N}[\mathbf{0}, \, \mathbf{\Sigma_{\epsilon}}]$ независимы от $\mathbf{x}^*_i$. Тогда $y_i| \mathbf{x}_i$ не распределено по Пуассону. Если, несмотря на это, использовать {\bf  <<наивную>> модель Пуассона}, полученная оценка $\mathbf{\tilde{\beta}}$ максимизирует
\begin{equation}
Q(\mathbf{\beta})= N^{-1} \sum \limits_i \{ -e^{\mathbf{x}^{*\prime}_i \mathbf{\beta}} + y_i \mathbf{x}^{*\prime}_i \mathbf{\beta} - \ln y_i! \}.
\end{equation}
Этот логарифм неверно специфицированной функции правдоподобия сходится к функции
\begin{equation}
\plim Q(\mathbf{\beta})=\Expect_{y,\mathbf{x}^*}[-e^{\mathbf{x}^{*\prime} \mathbf{\beta}} + y \mathbf{x^{*\prime}} \mathbf{\beta} - \ln y! ] +  \Expect_{\mathbf{x^*}} [-e^{\mathbf{x^{*\prime}} \mathbf{\beta}}] (\Expect_{\epsilon}[e^{\mathbf{\epsilon}' \mathbf{\beta}}]-1),
\end{equation}
которая, в общем случае, не достигает максимума при $\mathbf{\beta}= \mathbf{\beta_0}$. Поэтому $\mathbf{\tilde{\beta}}$ не является состоятельной оценкой $\mathbf{\beta}_0$.

Подходящая модификация этой целевой функции обеспечивает состоятельные оценки. Из уравнений (26.22) и (26.24) следует, что
\[
\{ \plim Q(\mathbf{\beta})-\Expect_{\mathbf{x^*}} [-e^{\mathbf{x^*}' \mathbf{\beta}}](\Expect_{\epsilon}[e^{\mathbf{\epsilon}' \mathbf{\beta}}]-1) \} = \plim N^{-1}\ln L(\mathbf{\beta}).
\]
Поэтому можно максимизировать целевую функцию
\[
Q^+(\mathbf{\beta})=N^{-1} \sum \limits_i \{ -e^{\mathbf{x}^{*\prime}_i \mathbf{\beta}} + y_i \mathbf{x}^{*\prime}_i \mathbf{\beta} - \ln y_i! \} - \Expect_{\mathbf{x^*}} [-e^{\mathbf{x^{*\prime}} \mathbf{\beta}}](\Expect_{\epsilon}[e^{\mathbf{\epsilon}' \mathbf{\beta}}]-1),
\]
так как $Q^+(\mathbf{\beta})$ стремится к $-N^{-1} \sum \limits_i \ln L(\mathbf{\beta})$. Теперь, учитывая независимость $x^*$ и $\mathbf{\epsilon}$,
\[
\Expect_{\mathbf{x^*}}[ -e^{\mathbf{x^*}'_i \mathbf{\beta}}]\Expect_{\epsilon}[e^{\mathbf{\epsilon}' \mathbf{\beta}}] = \Expect_{\mathbf{x^*},\epsilon}[e^{(\mathbf{x^*}+\mathbf{\epsilon})' \mathbf{\beta}}]=-\Expect_{\mathbf{x}}[e^{\mathbf{x}'_i \mathbf{\beta}}],
\]
что состоятельно оценивается как $-N^{-1} \sum \limits_i -e^{\mathbf{x}_i' \mathbf{\beta}}$. После  упрощения следует, что максимизация $Q^+(\mathbf{\beta})$ эквивалентна максимизации
\begin{equation}
Q^{++}(\mathbf{\beta})=N^{-1} \sum \limits_i \{ y_i \mathbf{x}^*_i \mathbf{\beta} - \ln y! \} - \Expect_{\mathbf{x^*}} [e^{\mathbf{x^*}' \mathbf{\beta}}].
\end{equation}
Это даёт состоятельную оценку $\mathbf{\beta}_0$. Использование этого метода требует подходящей оценки $\Expect_{\mathbf{x}^*}[ e^{\mathbf{x}^{*\prime} \mathbf{\beta}}]$, которая возможна, если доступны дублированные данные. Если распределение объясняющих переменных специфицировано с помощью неизвестных параметров, то эти параметры могут быть оценены с помощью реплицированных измерений. Таким образом, можно получить оценку $\Expect_{\mathbf{x}^*}[ e^{\mathbf{x}^{*\prime} \mathbf{\beta}}]$.

Оценка $\mathbf{\beta_C}$, которая максимизирует (26.25) была названа {\bf скорректированной скоринговой оценкой} Гуо и Ли (2002), так как это корень уравнения  скорректированной скоринговой функции $\sum_i (y_i \mathbf{x}_i- \Expect_{\mathbf{x^*}}[\mathbf{x^*}e^{\mathbf{x^*}'\mathbf{\beta}}])=\mathbf{0}$. Гуо и Ли также показали асимптотическую нормальность этой оценки. Асимптотическая оценка ковариационной матрицы равна $\widehat{\Var}[\widehat{\mathbf{\beta}}]=N^{-1} \widehat{\mathbf{A}}^{-1}\widehat{\mathbf{B}}\widehat{\mathbf{A}}^{-1}$, где 
\begin{align*}
\widehat{\mathbf{A}}&=\Expect_{\mathbf{x^*}}[e^{\mathbf{x}^{*\prime} \widehat{\mathbf{\beta_C}}} \mathbf{x^*} \mathbf{x^*}'], \\
\widehat{\mathbf{B}}&=N^{-1} \sum \limits_i (y_i \mathbf{x}_i - \Expect_{\mathbf{x^*}}[e^{\mathbf{x}^{*\prime} \widehat{\mathbf{\beta_C}}} \mathbf{x^*}]) (y_i \mathbf{x}_i - \Expect_{\mathbf{x^*}}[e^{\mathbf{x}^{*\prime} \widehat{\mathbf{\beta_C}}} \mathbf{x^*}])'.
\end{align*}

Накамура (1990) ввёл сильную предпосылку о том, что ошибки измерений $\mathbf{\varepsilon}$ имеют нормальное распределение $\mathcal{N}[\mathbf{0}, \, \mathbf{\Omega}]$. Тогда
\[
\exp (\mathbf{x^*}' \mathbf{\beta}) = \Expect_{\mathbf{x|x^*}} \left[ \exp \left(\mathbf{x}' \mathbf{\beta} - (\mathbf{\beta' \Omega \beta} /2) \right) \right].
\]
Согласно закону повторных ожиданий,
\[
\Expect_{\mathbf{x^*}} [\exp (\mathbf{x^*}' \mathbf{\beta})] = \Expect_{\mathbf{x}} \left[ \exp \left(\mathbf{x}' \mathbf{\beta} - (\mathbf{\beta' \Omega \beta} /2) \right) \right],
\]
Что может быть состоятельно оценено, как $N^{-1} \sum \limits_i [ \exp (\mathbf{x}' \mathbf{\beta} - (\mathbf{\beta' \Omega \beta} /2) )]$. Следовательно, для $Q(\mathbf{\beta})$ в (26.23) предел по вероятности, приведённый в (26.24), сводится к
\[
\plim Q(\mathbf{\beta}) = N^{-1} \sum \limits_i \left[ y_i \mathbf{x}'_i \mathbf{\beta} - \ln y_! - \exp \left(\mathbf{x}' \mathbf{\beta} - (\mathbf{\beta' \Omega \beta} /2) \right) \right].
\]
Это логарифм {\bf скорректированной функции правдоподобия}, предложенный Накамурой (1990). Максимизация по $\mathbf{\beta}$ позволяет получить состоятельные оценки $\mathbf{\beta_0}$.

Подход Накамуры  напоминает один из способов оценивания линейной регрессии с ошибками измерений (см. (26.14)) при известной оценке ковариационной матрицы ошибок измерений. Как и в том случае, для максимизации скорректированной скоринговой функции Накамуры требуется знание о $\mathbf{\Omega}$, ковариационной матрице ошибок измерений. Она может быть получена из дублированных данных. Тем не менее, если регрессоры преимущественно дискретны, то предпосылка о нормальности ошибок измерений неправдоподобна. В таких случаях подход Гуо и Ли выглядит более привлекательным.

Для случая множественности регрессоров $\mathbf{x^*}$, вычисление $\Expect [\exp (\mathbf{x^*}' \mathbf{\beta})]$ непросто, даже если известно распределение $\mathbf{x^*}$, так как требуется работа с многомерными интегралами. Методы, основанные на симуляции Ли (2002) обеспечивают возможное разрешение этой проблемы.

Применение некоторых других нелинейных моделей ошибок в переменных также требует реплицированных наблюдений; см. например, работы Хсяо (1992) и Хаусмана, Ньюи и Пауэлла (1995). Панельные данные могут обеспечить дублированные наблюдения на уровне индивидов. Например, рассмотрим случай, когда имеется скалярный регрессор $x^*$, для которого имеются две реплики измерений $x$, так как $x_{ij} = x_i + \epsilon_{ij}$ для $i=1,\dots,N$ и $j=1, \, 2$. Тогда оценка $\sigma^2_{\epsilon}$, полученная методом моментов, будет равна $\widehat{\sigma}^2_{\epsilon} = \sum \limits_i (x^2_{i1}+x^2_{i2} - 2x_{i1}x_{i2})/2N$. Таким образом, и среднее значение, и дисперсия $\mathbf{x^*}$ могут быть оценены.
 
\section{Пример симуляции смещения затухания} 
Аналитические результаты для линейной модели приведены в Разделы 26.2, но результаты в нелинейных моделях получить намного сложнее. Здесь мы приведём два примера симуляции, один из которых --- для логит модели, другой --- для линейной в логарифмах модели, которые иллюстрируют смещение затухания в нелинейной регрессии с ошибками измерений в регрессорах.

В первом примере данные порождены логит-моделью, где
\begin{align*}
y^* &=\alpha^* + \beta^* + \epsilon, \\
x^* &\sim \mathcal{U}[0, \, 1], \, \epsilon \sim \text{logistic}, \\
y &= 
\begin{cases}
0, \text{если} y^* &\leqslant0 ,\\
1, \text{если} y^* &>0.
\end{cases}
\end{align*}
Усложнением является то, что $x*$ измерен с ошибками, так что
\begin{align*}
x &=x^*+v, \\
v &\sim \mathcal{N}[0, \, \sigma^2_v].
\end{align*}
Так как $x^* \sim \mathcal{U}[0, \, 1]$ , он имеет дисперсию $\sigma^2_{x^*}=1/12$, и отношение шум--сигнал равно $s=12\sigma^2_v$. Была оценена логит-регрессия $y$ на $x^*$.

Чтобы провести симуляцию, мы строим логит регрессию $y$ на $x$ для шести различных значений отношения шум--сигнал, включая 0, что будет являться базой для сравнения. Размер выборки --- 1000, количество проведённых симуляций --- 100. 

В Таблице 26.1 отражены средние значения $(\widehat{\alpha}, \, \widehat{\beta})$ при 100 симуляциях, где $\widehat{\alpha}$ и $\widehat{\beta}$ --- оценки свободного члена и коэффициента наклона из логит регрессии $y$ на $x$, а не правильной регрессии $y$ на $x^*$, для выборки из $N=1000$ наблюдений и шести различных значений $\sigma^2_v$, которые дают шесть различных значений отношения шум--сигнал $s$. Первая колонка со значением $s=0$ является базовой. Напомним, что для линейной МНК модели такой же постановки  мультипликативное смещение оценки коэффициента наклона равно $1/(1+s)$, или 0.96, 0.8, 0.5, 0.2 и 0.1 соответственно. В этом примере  направление смещений то же, но для случая логит-модели они сравнительно больше.


\begin{table}[h!]
\caption{\label{tab:26.1 } Смещение затухания в логит-модели с ошибками измерения}
\begin{center}
\begin{tabular}{lcccccc}
\hline
\hline
Отношение шум-сигнал & 0 & 0.04 & 0.25 & 1 & 4 & 9 \\
\hline
Среднее $\hat{\alpha}$ &   0.785 & 1.062 & 1.406 & 1.548 & 1.570 & 1.596  \\
Среднее $\hat{\beta}$ &    1.799 & 1.224 & 0.446 & 0.125 & 0.037 & 0.012 \\
\hline
\hline
\end{tabular}
\end{center}
\end{table}

\begin{table}[h!]
\caption{\label{tab:26.2 } Смещение затухания в нелинейной регресии с ошибками измерения}
\begin{center}
\begin{tabular}{lcccccc}
\hline
\hline
$\sigma^2_x/\sigma^2_{x^*}$ & 0.00025 & 0.0025 & 0.025 & 0.25 & 2.5 & 25 \\
\hline
Среднее $\hat{\beta}$ &   0.393 & 0.383 & 0.341 & 0.217 & 0.063 & 0.020  \\
\hline
\end{tabular}
\end{center}
\end{table}

Второй пример –-- парная линейная в логарифмах мультипликативная регрессионная модель с $\alpha=4$ и $\beta=0.4$ и аддитивными ошибками измерений обеих переменных. В этом случае постановка следующая:
\begin{align*}
y^* &= 4{x^*}^{0.4}u, \, u \sim \mathcal{N}[10, \, 0.0001], \\
x^* &= 100+ \mathcal{U}[0, \, 1], \\
y &= y^*+ \epsilon_y, \, \epsilon_y \sim \mathcal{N}[0, \, \sigma^2_y], \\
x &= x^*+ \epsilon_x, \, \epsilon_x \sim \mathcal{N}[0, \, \sigma^2_x].
\end{align*}
В симуляции используется выборка размером 1000, количество симуляций --- 100. Мы меняем значение дисперсии $x^*$ от эксперимента к эксперименту, получая в результате следующие значения $\sigma^2_{x} / \sigma^2_{x^*}$: 0.001, 0.01, 0.1, 1,5, 10, 50, 100, 1000 и 5000.

В верхней строке Таблицы 26.2 приведены средние значения оценок коэффициента наклона в разных экспериментах, в которых менялось значение отношения шум--сигнал. Снова видно, что имеет место смещение затухания.

Оба примера дают результаты, соответствующие <<железному закону эконометрики>>.

\section{Библиографические заметки} 

Работа Вансбика и Мейера (2000) является самой современной и полной написанной работой по ошибкам измерений с точки зрения эконометрики. В ней содержится глубокий анализ большего числа тем, рассмотренных в этой Главе с упором на линейные модели. Авторы также включили несколько глав, в которых ошибки измерения связаны с факторными моделями, моделями латентной переменной и моделями структурных уравнений. Говоря о результатах, авторы избегают фразы <<это может быть показано>>, чтобы подробно их изложить. Также с эконометрической позиции, Хаусман (2000) приводит обзор последних результатов, полученных в работах его коллег и его самого. Баунд, Браун и Матьовец (2001) исследовали вопросы ошибок измерений на рынке труда.

Проблема ошибок измерений хорошо представлена в статистической литературе. Полезна ссылка на работу Фуллера (1987), в частности, интересен его разбор подхода ортогональных регрессий, который применим в случае, если известно отношение шум-сигнал. Хотя наш анализ линейных моделей стандартен для эконометрической литературы, читателю также стоит ознакомиться с альтернативной моделью ошибок Берксона, в которой ненаблюдаемая истинная переменная предполагается константой, но несовершенство измерений является источником ошибок, а также с неоклассической моделью ошибок измерений, рассмотренной Ангристом и Крюгером (1999). Мадански (1959) приводит обзор более ранних результатов и подходов. Смотрите также работу Стефански (2000).

\begin{enumerate}
\item Модели панельных данных с ошибками измерений анализируются в работе Бьорна (1992).
\item Интересная тема обратной регрессии рассматривалась Голдбергером (1984) и Грином (1983) в их комментариях к работе Конвея и Робертса (1983). Лимер (1978) привёл глубокий анализ обратных регрессий с байесовских позиций. Хан и Хаусман (2002) использовали идею обратных регрессий для спецификации теста на валидность инструментов в рамках проблемы ошибок измерений. Эта задача связана с тем, что доступные инструменты могут быть слабыми, что проводит к плохим оценкам. Идея Хана-Хаусмана состоит в получении оценок инструментальных переменных в прямой регрессии, где возникают ошибки измерений в правой части уравнения. Обратная регрессия имеет те же ошибки, но в левой части. Эта регрессия также оценивается с помощью инструментальных переменных, причем тех же, что использовались в прямой регрессии.
\item Литература по теме ошибок измерений в нелинейных моделях более разнородна. Работа Амэмия (1985) особенно полезна эконометристам. Со статистической точки зрения, Кэррол и др. (1995) рассматривали нелинейные модели, в частности класс обобщенных линейных моделей, с аддитивными ошибками измерений в регрессорах, используя различные методы, включая ряд подходов, в которых могут применяться реплицированные данные, если они доступны. Ли, Триведи и Гуо (2003) развивали и использовали модель ошибок измерений, в которых с ошибками измерена счетная переменная.
\end{enumerate}


\section*{Упражнения} 
\begin{enumerate}
\item Пусть имеет место  смещение затухания \emph{(attenuation bias)} для параметра наклона парной модели ошибок переменных (Уравнение 26.9 в Разделе 26.2.3). Расширим эту модель, включив константу.
\begin{enumerate}
\item Найдите аналогичный результат  смещения из-за ошибки измерений для константы.
\item	Найдите аналогичный результат при идентификации границ для МНК-оценки свободного члена, сходного с Уравнением (26.12) в Разделе 26.3.1.
\end{enumerate}
\item (Адаптировано из работы Боллингера, 2003) Пусть есть линейная множественная регрессионная модель со скалярным регрессором $x$, который измерен с ошибками, и вектором регрессоров $z$, которые свободны от ошибок измерений.
\begin{enumerate}
\item Сохраняя предпосылки относительно ошибок измерений для парной модели ошибок переменных, обобщите результат смещения затухания и результат при идентификации границ на этот случай.
\item Проверьте, что новые результаты содержат в себе парную регрессию как частный случай.
\end{enumerate}
\item (Адаптировано из работы Вансбика и Мейера, 2000) Пусть есть квадратичная регрессионная модель $\mathbf{y}=\alpha + \beta\mathbf{x^*}+\gamma\mathbf{{x^*}^2}+\epsilon$, где регрессор $\mathbf{x^*}=\mathbf{x}+\mathbf{v}$ с $\mathbf{x}$ --– наблюдаемым и $\mathbf{v}$ --– ошибкой измерений. Предположим, что $(\mathbf
x^*, \, \epsilon, \, \mathbf{v})$ попарно не коррелированы и нормально распределены с нулевым средним.
\begin{enumerate}
\item	Сравните смещение МНК-оценок $\beta$ и $\gamma$.
\item	Идентифицируема ли модель? Сравните последний результат с результатом оценивания парной линейной модели ошибок переменных.
\end{enumerate}
\item В литературе, посвященной, мобильности, связывающей поколения \emph{(intergenerational mobility)}, используется следующая модель (Солон, 1992; Циммерман, 1992):
\begin{equation}
Y^{son}_i=\alpha + \beta Y^{father}_i + \epsilon^{son}_i,
\end{equation}
где $\epsilon_i$ независимы и нормальны $\mathcal{N}[0,\, \sigma^2]$. Здесь $Y$ --– мера перманентного статуса (например, перманентный доход) и $\beta$ измеряет степень возвращения к среднему экономическому статусу. Положим, что этот перманентный статус не наблюдается. Вместо него, текущий статус $Y_{it}$ наблюдается с $Y_{it}=Y_i+ \gamma X_{it}+ w_{it}$, так что $Y_{it}$  состоит из индивидуального фиксированного эффекта $Y_{i}$, называемого  перманентным статусом, систематических факторов $X_{it}$ и переходной ошибки $w_{it}$. Пусть $\widehat{\gamma}$ отражает посчитанный на основе МНК коэффициент и пусть
\[
Y_{it}-\widehat{\gamma}X_{it}= Y_i+ (\gamma - \widehat{\gamma})X_{it} + w_{it} = Y_i + v_{it}.
\]
\begin{enumerate}
\item Пусть $Y^{father}_i=T^{-1} \sum \limits^{T}_{t=1}Y^{father}_{it}$ отражает средний статус отца, который используется как независимая переменная, --- прокси для ненаблюдаемого перманентного статуса в (26.26). Пусть $\widehat{\beta}_{avg}$ обозначает соответствующий коэффициент регрессии. Покажите, что $\plim \widehat{\beta}_{avg}=\beta P_Y$, где $P_Y=n\sigma^2_Y/(\sigma^2_Y+T^{-1} \sigma^2_{\epsilon})$.
\item Предположим, что переходная составляющая доходов отца следует авторегрессионной схеме, $v^{father}_{it}=\rho v^{father}_{it}+\xi_{it}$, где $\xi \sim \text{iid} \mathcal{N}[0,\, \sigma^2_{\xi}]$, $i=1,\dots ,T$. Покажите, что теперь $\plim \widehat{\beta}_{avg}=\beta P^*_Y$, где $P_Y=n\sigma^2_Y/(\sigma^2_Y+T^{-1} V)$ и $V=\sigma^2_{\xi}[T(1-\rho^2)]^{-1}[(1+2\rho\{ T-(1-\rho^T)/(1-\rho) \} /T(1-\rho)]$.
\end{enumerate}
\end{enumerate} 




\chapter{Пропущенные данные и восстановление данных}
\section{Введение}
Проблема {\bf пропусков в данных}, по которым проводится исследование известна давно и возникает из-за отсутствия ответов или частичных ответов на вопросы обследований. Причины того, что респонденты не отвечают, включают нежелание раскрывать информацию, о которой их спрашивают, сложность вспомнить события, произошедшие в прошлом, а также незнание правильного ответа на вопрос. {\bf Восстановления} \emph{(imputation)} представляет собой процесс оценки или предсказания пропущенных наблюдений.

В этой Главе мы имеем дело с регрессией, построенной на основе  данных $(y_i, \, \mathbf{x}_i), \, i=1,\dots, N$. Для некоторых наблюдений часть элементов $\mathbf{x}_i$ или  $(y_i, \, \mathbf{x}_i)$ пропущены. Возникает ряд вопросов. Когда мы можем проводить анализ только по полной выборке, и когда мы должны попробовать заполнить пропуски в наблюдениях? Какие методы восстановления пропусков могут быть применены? Когда получены значения, которыми могут быть заполнены пропуски, какими должны будут стать процесс оценивания и выводы?

Если набор данных содержит пропущенные наблюдения, и если эти пропуски могут быть заполнены с помощью некоторой статистической процедуры,  то есть возможность воспользоваться преимуществом большей по объему и более репрезентативной выборки, и в идеальных условиях получить более точные выводы. Природа издержек оценивания пропущенных наблюдений заключается в необходимости введения предпосылок (зачастую несоответствующих действительности) для возможности использования прокси для отсутствующих значений, а также в наличии ошибок аппроксимации, свойственных таким процедурам. Далее, статистические выводы, которые делаются после восстановления пропусков, усложняются, так как должны быть учтены ошибки аппроксимации, возникающие при восстановлении. 

Пропуски в данных, как результат отсутствия ответов на вопросы обследований, а также истощения панели, возникают часто. Пополнение пропусков может быть осуществлено как агентствами, которые создают и обеспечивают доступность баз данных обследований, так и теми, кто использует данные для моделирования. В первом случае агентство может обладать более обширной информацией, включая конфиденциальную, которая может быть полезной при заполнении пропусков. Во втором случае исследователь, строящий модели, может иметь специальную  модель, которая может быть использована при восстановлении. В обоих случаях возможно восстановление пропущенных наблюдений, основанное на некоторой модели.

\vspace{3cm}
Рисунок 27.1. Пропуски в данных. Примеры пропусков в регрессорах.

A: Univariate missing data pattern --- Пропуски в одном регрессоре

B: Special Pattern of missing data on x1 and x2 --- Частный случай пропусков в $x_1$ и $x_2$

C: General pattern of missing data --- Общий случай пропусков в регрессорах



Интересный пример пропущенных наблюдений возникает в контексте Обследование Финансового Положения Потребителей (Survey of Consumer Finances) см. работу Кенникелла (1998). Из-за деликатности самого сюжета финансового положения потребителя, обследование содержит большое количество пропусков в вопросах о доходах и благосостоянии. Аналитики Федеральной резервной системы США (U.S. Federal reserve) занимались развитием и внедрением комплекса алгоритмов заполнения пропусков в данных для случаев непрерывных и дискретных переменных, используя как открытую и доступную информацию обследований дохода и благосостояния, так и конфиденциальную информацию из переписи.

Рисунок 27.1 отражает несколько возможных примеров пропусков значений регрессоров. База данных содержит скалярную зависимую переменную $y$ и три регрессора: $x_1$, $x_2$ и $x_3$ для каждого наблюдения, которые могут быть записаны как $(\mathbf{y, \, x_1, \, x_2, \, x_3})$. В панели A есть полная информация по показателям $(\mathbf{y, \, x_2, \, x_3})$, тогда так часть данных по $\mathbf{x_1}$ отсутствует. В панели B полная информация есть по $(\mathbf{y, \, x_3})$, но нет части данных по $(\mathbf{x_1, \, x_2})$, так что значения по ним никогда не наблюдаются одновременно. В панели C представлен пример пропусков в данных общего вида по всем трём  регрессорам, но без какого-то определённого порядка этих пропусков.  

Наиболее простая форма работы с пропусками в переменных --– удалить их и анализировать урезанную выборку, состоящую из <<полных>> \emph{(<<complete>>)} данных. Например, в случае панели A, полной будет являться выборка из переменных $(\mathbf{y, \, x_1, \, x_2, \, x_3})$, сформированная по всем доступным значениях $\mathbf{x_1}$ и соответствующим им значениям $(\mathbf{y, \, x_2, \, x_3})$. В случае панели B, тем не менее, согласно такому подходу, не останется ни одного доступного наблюдения, если не исключать $(\mathbf{x_1, \, x_2})$ из рассмотрения. В панели C полная выборка будет сформирована при удалении всех наблюдений, для которых хотя бы по одной трёх независимых переменных есть пропуски.
Описанная выше процедура носит название полного удаления наблюдений с пропусками (\emph{listwise deletion}). К ней часто прибегают и она, как правило, является опцией <<по умолчанию>> в статистических пакетах. Это не всегда безобидно, последствия зависят от устройства пропусков, и сделанные на основе этих исследований выводы могут быть серьезно искажены. Конечно, в общем, выкидывание данных означает потерю информации, что снижает эффективность оценивание. Таким образом, при условии, что пропуски в данных могут быть заполнены без внесения искажений, полное удаление наблюдений с пропусками представляется неверным путём. В этой Главе будут рассмотрены альтернативные подходы и рамки их применимости.

В целом, есть два подхода к вопросу восстановления пропусков в данных, первый {\bf основан на моделях} \emph{(model-based)}, второй --- нет. Современная точка зрения склоняется к подходу, основанному на моделях. В нём используется построение моделей для заполнения пропусков, а затем в анализе участвует вся выборка с целью получения лучших оценок параметров модели.  Это процесс носит итерационный характер. Возможно восстановление как пропусков по одной переменной, так и по многим. Основная идея современного подхода --- рассмотрение пропущенных значений как случайных переменных, а затем замещение их многократными случайными выборками, полученными из рассматриваемого распределения. Процесс этот называется множественным восстановлением данных \emph{(multiple imputation)}. Симуляционные методы могут использоваться для аппроксимации такого распределения.

Эта тема оправдывает отдельную короткую вводную главу, так как заполнение пропусков в данных является важным моментом микроэконометрических исследований. Анализируемые данные неизбежно содержат пропуски, и наиболее распространённой практикой восстановления пропущенных значений является метод полное удаление наблюдений с пропусками. Но доступны и более совершенные методы. Важным предостережением является тот факт, что, тем не менее, любые методы работы с пропусками в данных основываются на предпосылках, которые в ряде случаев могут быть слишком жесткими.

Большая часть этой Главы посвящена основанным на моделях подходам. Раздел 27.2 представляет собой введение в терминологию и предпосылки, которые, как правило, имеют место в литературе на тему пропущенных данных и их восстановления. Раздел 27.3 содержит общее рассмотрение методов работы с пропусками в данных, которые не предполагают построения моделей. Раздел 27.4 начинается с первого подхода, основанного на моделях, метода максимального правдоподобия. Раздел 27.5 посвящен регрессионному подходу и методам EM-типа. В Разделах 27.6 и 27.7 представлены подходы восстановления пропущенных данных на основе байесовского подхода и MCMC. Раздел 27.8 иллюстрирует описанное выше примером. В Разделах 27.6--27.8 приводится удачное  применение байесовских методов из Главы 13.

\section{Предположения при работе с пропущенными данными}

Некоторая базовая терминология и формальные определения, широко используемые в литературе о проблеме восстановления пропущенных значений переменных, принадлежат Рубину (1976), который ввёл два основных механизма пропусков: случайный пропуск (missing at random) и {\bf полностью случайный пропуск} \emph{(missing completely at random)}, которые служат базовыми. 

Постановка Рубина включает $\mathbf{Y}$, матрицу размеров $N \times p$, состоящую из полного набора данных, часть их которых может не наблюдаться. Обозначим $\mathbf{Y}_{obs}$ наблюдаемую часть и $\mathbf{Y}_{mis}$ --- ненаблюдаемую (пропущенную) часть наблюдений. В контексте регрессионной модели матрица $\mathbf{Y}$ содержит и регрессоры, и зависимые переменные. Таким образом, этот анализ подразумевает работу с пропусками общего вида. Обозначим $\mathbf{R}$ матрицу размерности $N \times p$, состоящую из переменных-индикаторов, элементы которой равны нулю или единице в зависимости от того, наблюдаемо или нет соответствующее значение в матрице $\mathbf{Y}$.

Для регрессии с одной зависимой переменной $\mathbf{Y}$ включает наблюдения по зависимой переменной  $\mathbf{y}$ и $(p-1)$-му регрессору $\mathbf{X}$. Вероятность того, что $x_{ki}$, наблюдение $i$ переменной $x_k$, пропущено, может быть (i) независима от её фактического  значения, (ii) зависима от фактического значения, (iii) зависима от $x_{kj}, \, j \neq i$, или (iv) зависима от $x_{lj}, \, j \neq i, \, l \neq k$.

Далее более подробно рассматриваются предпосылки относительно структуры пропусков.

\subsection{Случайные пропуски} 
Предположим, что $x_i \, (i = 1, \dots ,N)$  есть наблюдений одной переменной в исследуемой базе данных. Предпосылка {\bf случайности пропуска} \emph{(missing at random --- MAR)} означает, что наличие пропуска значения $x_i$ не зависит от самого значения, но может зависеть от других значений $x_j \, (j \neq i)$. Формально,
\begin{align}
x_i \, \text{ удовлетворяет MAR} & \Rightarrow \Prob[x_i \text{ пропущено } \, | \, x_i, \, x_j \, \forall \, j \neq i] \\
&= \Prob[x_i \text{ пропущено } \, | \, x_j \, \forall \, j \neq i]. \notag
\end{align}
После учета остальных значений $x$ вероятность пропуска $x_i$ не связана со значением $x_i$.

Согласно более формальному определению Рубина (1979), предпосылка MAR  подразумевает, что вероятностная модель для индикаторной переменной $\mathbf{R}$ не зависит от $\mathbf{Y}_{mis}$, то есть,
\begin{align*}
\Prob[\mathbf{R} \, | \, \mathbf{Y}_{obs}, \, \mathbf{Y}_{mis}, \, \mathbf{\psi}]=\Prob[\mathbf{R} \, | \, \mathbf{Y}_{obs}, \, \mathbf{\psi}],
\end{align*}
где $\mathbf{\psi}$ --- вектор параметров, которые определяют механизм пропуска.

Если предпосылка MAR выполнена, то при использовании метода максимального правдоподобия по полным наблюдениям не возникает смещения из-за пропущенных значений, хотя оценки могут быть неэффективными. Если MAR-предпосылка не выполняется, то, вероятность пропусков зависит от ненаблюдаемых значений переменных. MAR-ограничение не тестируемо, так как пропущенные наблюдения неизвестны. В силу того, что MAR является сильной предпосылкой, желательным является  анализ чувствительности, основанный на разных предпосылках о пропусках.

Отдельным вопросом является вопрос о том, действительно ли характер пропусков в чистом виде случаен. На практике мы должны ожидать, что наблюдения, пропущенные в кластерах данных, в смысле Главы 24, могут быть коррелированы. Тем не менее, этот вопрос не относится к смещенности от наличия пропусков, связанных со значениями переменной.

\subsection{Полностью случайные пропуски} 

{\bf Полностью случайные пропуски} \emph{(Missing completely at random --- MCAR)} представляет собой частный случай MAR. Это означает, что $\mathbf{Y}_{obs}$ является просто случайной выборкой всех потенциально наблюдаемых значений переменных Шефер (1997).

Снова обозначим с помощью $x_i$ наблюдение за переменной в исследуемом наборе данных. Тогда данные имеют MCAR-пропуски, если вероятность пропуска в $x_i$ не зависит ни от собственного значения, ни от других значений переменных в выборке. Формально,
\begin{align}
x_i \, \text{ удовлетворяет MCAR } & \Rightarrow \Prob[x_i \text{ пропущено } \, | \, x_i, \, x_j \, \forall \, j \neq i] \\
&= \Prob[x_i \text{ пропущено }]. \notag
\end{align}
Например, MCAR нарушается, если (a) те, кто не сообщает доход, в среднем старше тех, кто сообщает, или (b) пропущены, как правило, меньшие значения.

Для случаев (i)--(iv), упомянутых в начале этого Раздела, случай (i) удовлетворяет обоим условиям MAR и MCAR, случаи (iii) и (iv) удовлетворяют MAR, случай (ii) не удовлетворяет MAR.

MCAR подразумевает, что наблюдаемые значения являются случайной подвыборкой потенциальной полной выборки. Если предпосылки справедливы, то игнорирование неполноты выборки (наличия пропусков в данных) не приведёт к смещённым результатам.

Подводя итоги, отметим, что нарушение MCAR приводит к смещению самоотбора. MAR является более слабой предпосылкой, которая также помогает при заполнении пропусков в данных, так как она предполагает, что механизм потери данных зависит только от наблюдаемых значений.

\subsection{Игнорируемые и неигнорируемые пропуски} 

Механизм потери данных называют {\bf игнорируемым}, если (a) набор данных удовлетворяет предпосылке MAR и (b) параметры процесса порождающего пропуски, $\mathbf{\psi}$, не связаны с параметрами $\mathbf{\theta}$, которые мы хотим оценивать.

Это условие, схожее со {\bf слабой экзогенностью}, которая обсуждалась в Главе 2, подразумевает, что параметры $\mathbf{\theta}$ модели отличаются от параметров $\mathbf{\psi}$ механизма пропусков. Таким образом, если пропущенные данные игнорируемы, то нет необходимости в моделировании механизма пропущенных данных, как обязательной части исследовательского процесса. MAR и <<игнорируемость>> часто рассматриваются как эквивалентные при выполнении предпосылки о соблюдении условия (b) игнорируемости, что часто оправдано, см. работу Эллисона (2002).

{\bf Неигнорируемый} механизм появления пропущенных значений наблюдений возникает, если MAR-предпосылка нарушается для $(y, \, x)$. Случай, когда предпосылка MAR нарушена только для $x$ сюда не относится. В случае неигнорируемых пропусков необходимо моделировать процесс порождающий пропуски вместе с построением основной модели для получения состоятельных оценок параметров $\mathbf{\theta}$. Чтобы избежать возможного смещения выборки, должно быть использовано соответствующее оценивание, такое как, например, двухшаговая процедура Хекмана (см. Главу 16).

Литература по восстановлению пропусков в данных акцентируется на игнорируемых пропусках. Если  данные удовлетворяют предпосылке MCAR, то пропуски не вызывают проблем, кроме потери эффективности оценок, которая может быть уменьшена с помощью восстановления пропусков. Если ситуация иная, и данные удовлетворяет только предпосылке MAR, то методы восстановления пропущенных значений  должны быть использованы для  получения состоятельных оценок, а не только для увеличения  эффективности.

\section{Работа с пропусками  без применения моделей} 

Если никакие модели не могут быть использованы, то можно просто анализировать доступные данные или прибегнуть к анализу после восстановления пропусков без применения моделей.

\subsection{Использование только доступных данных}
Полное удаление наблюдений с пропусками или анализ полной выборки  означает исключение наблюдений, которые имеют пропущенные значения по одной или нескольким переменным из исследуемого набора. При MCAR-предпосылке, выборка, оставшаяся после такого исключения, является по-прежнему случайной выборкой из генеральной совокупности, так что полученные при этом оценки будут состоятельными. Тем не менее, стандартные ошибки будут выше, так как использовано меньше информации. Если набор регрессоров велик, то общий эффект от полного удаления наблюдений с пропусками может привести к существенным потерям в общем числе наблюдений. Это может подтолкнуть к  исключению из анализа переменных с высокой долей пропущенных значений, но результаты, полученные после такой операции, вероятно, будут неверными. 

% опечатка в английском тексте в следующем абзаце. MCAR вместо MAR верно.
Если MCAR-предпосылка не выполняется и данные удовлетворяют только MAR, то оценки будут смещёнными. Таким образом, полное удаление наблюдений с пропусками не робастно к отклонениям от MCAR. Однако полное удаление робастно к отклонениям от MСAR относительно независимых 
переменных в регрессионном анализе, если вероятность пропуска значений любого из регрессоров не зависит от значений зависимой переменной. Обобщая сказанное, можно сказать, что полное удаление наблюдений с пропусками допустимо, если случаи неполноты данных из-за пропущенных значений составляют небольшой процент, например, около 5\% или меньше, от общего числа (Шефер, 1996). Важно отметить, что выборка после полного удаления наблюдений с пропусками в этом случае репрезентативна. 


{\bf Попарное удаление} наблюдений  или анализ доступных данных \emph{(available-case analysis)} часто полагается более предпочтительным методом, чем полное удаление наблюдений с пропусками. Идея этого метода состоит в использовании всех доступных пар наблюдений $(x_{1i}, \, x_{2i})$ при оценивании выборочных моментов пары $(x_1, \, x_2)$ и в использовании всех наблюдений по отдельной переменной  при оценивании её моментов. Таким образом, в линейной регрессии, под попарным удалением мы будем понимать оценивание $(\mathbf{X'X})$ и $(\mathbf{X'y})$ с использованием всех возможных пар регрессоров, тогда как под полным удалением мы понимаем оценивание тех же матриц, но после удаления всех всех наблюдений с пропусками. Понятно, что мы теряем меньше информации при попарном исключении. Идея состоит в том, чтобы использовать максимум информации для оценивания индивидуальных описательных статистик, таких как средние и ковариации, а затем в использовании этих статистик для вычисления оценок параметров регрессии.

Есть два важных ограничения применения попарного исключения: (1) Вычисленные обычным методом стандартные ошибки и тестовые статистики будут смещенными и (2) полученная в результате матрица ковариаций регрессоров $(\mathbf{X'X})$ может не оказаться  положительно определённой.

\subsection{Восстановление данных без использования моделей} 

Существует ряд  ad hoc или слабо аргументированных процедур, которые часто реализованы в статистических пакетах.

{\bf Заполнение средним} \emph{(mean imputation)} подразумевает замещение пропуска средним из доступных значений. Эта процедура не меняет среднего значения, но оказывает влияние на распределение данных. Очевидно, плотность в середине распределения будет увеличена. Также она будет влиять на ковариации и корреляции с другими переменными.

{\bf Метод карточной колоды} подразумевает замену пропущенных значений случайно выбранными из доступных наблюдаемых значений этой переменной, он схож с процедурой бутстрэпа. Этот метод сохраняет частное распределение переменной, но искажает ковариации и корреляции между переменными. 

При построении регрессии ни один из этих двух весьма распространённых подходов не является привлекательным несмотря на их простоту.

\section{Функция правдоподобия по наблюдаемым данным} 

Современный подход  состоит в восстановлении пропущенных значений с помощью однократной или нескольких повторных выборок из  оцененного распределения, основанного на заданной модели наблюдаемых данных и модели механизма появления пропусков в данных. Байесовские варианты этой процедуры используют выборку из апостериорного распределения, при построении которого учитываются и метод максимального правдоподобия, и априорное распределение параметров.

Первый важный вопрос --- роль, которую играет механизм проявления пропусков в процедуре их восстановления, и особенно, можно ли его игнорировать.

Обозначим $\mathbf{\theta}$ параметры процесса порождающего $\mathbf{Y}=(\mathbf{Y}_{obs}, \, \mathbf{Y}_{mis})$ и с помощью $\mathbf{\psi}$ --- параметры механизма пропусков. Для удобства записи предположим, что $(\mathbf{Y}_{obs}, \, \mathbf{Y}_{mis})$ --– непрерывные переменные. Тогда совместное распределение $(\mathbf{R}, \, \mathbf{Y}_{mis})$ имеет вид 
\begin{align}
\Prob[\mathbf{R}, \, \mathbf{Y}_{obs} | \mathbf{\theta}, \, \mathbf{\psi}] &= \int \Prob[\mathbf{R}, \, \mathbf{Y}_{obs}, \, \mathbf{Y}_{mis}| \, \mathbf{\theta}, \, \mathbf{\psi}]\mathbf{dY}_{mis} \\
&= \int \Prob[\mathbf{R}| \, \mathbf{Y}_{obs}, \, \mathbf{Y}_{mis}, \, \mathbf{\psi}] \Prob[\mathbf{Y}_{obs}, \, \mathbf{Y}_{mis}| \, \mathbf{\theta}]\mathbf{dY}_{mis} \notag \\
&= \Prob[\mathbf{R}| \, \mathbf{Y}_{obs}, \, \mathbf{\psi}] \int \Prob [\mathbf{Y}_{obs}, \, \mathbf{Y}_{mis}| \, \mathbf{\theta}]\mathbf{dY}_{mis} \notag \\
&=\Prob[\mathbf{R}| \, \mathbf{Y}_{obs}, \, \mathbf{\psi}] \Prob[\mathbf{Y}_{obs}| \, \mathbf{\theta}]. \notag
\end{align}
В первом равенство  формула совместной плотности $(\mathbf{R}, \, \mathbf{Y}_{obs})$ получается путем интегрирования (усреднения) по $\mathbf{Y}_{mis}$ совместной плотности для всех данных и $\mathbf{R}$. Вторая строка разлагает совместную плотность в произведение условных плотностей, условие берется по $\mathbf{Y}_{obs}$ и $\mathbf{Y}_{mis}$. В третьей строке разделяются механизм появления пропусков и механизм порождающий наблюдаемые данные, что возможно при выполнении предпосылки MAR. Последняя строка означает, то $\mathbf{\theta}$ и $\mathbf{\psi}$ являются несвязанными параметрами, и, таким образом, выводы относительно $\mathbf{\theta}$ могут быть сделаны баз учета механизма появления пропусков и только на основе $\mathbf{Y}_{obs}$. 

{\bf Функция правдоподобия по наблюдаемым данным} \emph{(observed-data likelihood)} пропорциональна последнему сомножителю в четвёртой строке:
\begin{equation}
L[\mathbf{\theta}|\mathbf{Y}_{obs}] \propto \Prob[\mathbf{Y}_{obs}|\mathbf{\theta}]
\end{equation}
В ней задействованы только наблюдаемые значения $\mathbf{Y}_{obs}$, хотя параметры $\mathbf{\theta}$ появляются в процессе порождающем  все наблюдения (наблюдаемые и ненаблюдаемые). Как в Главе 13, коэффициент пропорциональности опущен в формуле (27.4).

В рамках MAR-предпосылки {\bf совместная апостериорная плотность}  параметров $(\mathbf{\theta}, \, \mathbf{\psi})$ представляется как произведение $\Prob[\mathbf{R}, \, \mathbf{Y}_{obs}|\mathbf{\theta}, \, \mathbf{\psi}]$ и совместной априорной плотности $\pi (\mathbf{\theta}, \, \mathbf{\psi})$:
\begin{align}
\Prob[\mathbf{\theta}, \, \mathbf{\psi}|\mathbf{Y}_{obs}, \, \mathbf{R}] &= k \Prob[\mathbf{R}, \, \mathbf{Y}_{obs}|\mathbf{\theta}, \, \mathbf{\psi}] \pi (\mathbf{\theta}, \, \mathbf{\psi}) \\
&\propto \Prob[\mathbf{R}|\mathbf{Y}_{obs}, \, \mathbf{\psi}] \Prob[\mathbf{Y}_{obs}|\mathbf{\theta}]\pi (\mathbf{\theta}, \, \mathbf{\psi}) \notag \\
&\propto \Prob[\mathbf{R}|\mathbf{Y}_{obs}, \, \mathbf{\psi}] \Prob[\mathbf{Y}_{obs}|\mathbf{\theta}]\pi_{\theta} (\mathbf{\theta})\pi_{\psi} (\mathbf{\psi}), \notag
\end{align}
где $k$ в первой строке есть коэффициент пропорциональности, не зависящий от $(\mathbf{\theta}, \, \mathbf{\psi})$. Во второй строке используется разложение на сомножители из (27.3), а в третьей строке использована предпосылка о независимости априорных распределений $\mathbf{\theta}$ и $\mathbf{\psi}$.

Так как основной интерес для нас представляют параметры $\mathbf{\theta}$, выразим частную апостериорную плотность для $\mathbf{\theta}$, проинтегрировав по $\mathbf{\psi}$  совместную апостериорную плотность. Мы получаем {\bf апостериорную плотность по наблюдаемым данным} \emph{(observed-data posterior)}
\begin{align}
\Prob[\mathbf{\theta}|\mathbf{Y}_{obs}, \, \mathbf{R}] &= \int \Prob[\mathbf{\theta}, \, \mathbf{\psi}|\mathbf{Y}_{obs}, \, \mathbf{R}]d\mathbf{\psi} \\
&\propto \Prob[\mathbf{Y}_{obs}|\mathbf{\theta}]\pi_{\theta} (\mathbf{\theta}) \int \Prob[\mathbf{R}|\mathbf{Y}_{obs}, \, \mathbf{\psi}]\pi_{\psi} (\mathbf{\psi}) \notag \\
&\propto L [\mathbf{\theta}|\mathbf{Y}_{obs}]\pi_{\theta} (\mathbf{\theta}), \notag 
\end{align}
где во второй строке разделяются $\mathbf{\theta}$ и $\mathbf{\psi}$, а в последней строке интеграл включен в константу пропорциональности. Таким образом, выражение в последней строке не включает $\mathbf{\psi}$ и независимо от механизма появления пропусков в $\mathbf{R}$.

\section{Восстановление пропусков на основе регрессии} 

В этом Разделе мы рассмотрим восстановление пропусков на основе метода наименьших квадратов. Основной компонент этого подхода --- использование EM-алгоритма, ранее введённого и описанного в Разделе 10.3.7. 

EM-алгоритм состоит из двух шагов: подсчета ожидания и максимизации. Структура EM-алгоритма тесно связана с байесовским подходом MCMC и методами восстановления данных. Таким образом, вместо представления полностью готового к применению метода для работы с пропущенными значениями переменных, мы приведём пример, иллюстрирующий использования современных техник множественного восстановления пропусков в данных и показывающий основные свойства такого подхода.

\subsection{Пример линейной регрессии с пропущенными значениями зависимой переменной} 

На практике встречаются пропуски значений как зависимой (эндогенной) переменной, так и объясняющих переменных. Мы рассмотрим пример регрессии, в котором пропущены значения зависимой переменной,
\begin{align}
\begin{bmatrix}
\mathbf{y}_1 \\ \mathbf{y}_{mis} 
\end{bmatrix} 
= 
\begin{bmatrix}
\mathbf{X}_1 \\ \mathbf{X}_2 
\end{bmatrix} 
\mathbf{\beta}
+
\begin{bmatrix}
\mathbf{u}_1 \\ \mathbf{u}_2 
\end{bmatrix}
,
\end{align}
где $\Expect[\mathbf{u|X}]= \mathbf{0}$ и $\Expect[\mathbf{uu'|X}]= \sigma^2\mathbf{I_N}$. Сложность заключается в том, что блок наблюдений зависимой переменной $\mathbf{y}$, обозначенный как $\mathbf{y}_{mis}$, пропущен. Мы предполагаем, что доступные полные наблюдения являются случайной выборкой из генеральной совокупности, так что пропущенные данные удовлетворяют предпосылке MAR, но не MCAR.

При предпосылке MAR и $N_1>K$, первый блок из $N_1$ наблюдений может быть использован для получения состоятельных оценок вектора параметров размерности $K$, состоящего из $\mathbf{\beta}$  и $\sigma^2$. Оценки метода максимального правдоподобия $(\mathbf{\beta}, \, \sigma^2$  при гауссовском распределении ошибки буду равны $\widehat{\mathbf{\beta}}=[\mathbf{X'_1 X_1}]^{-1}\mathbf{X'_1 y_1}$ и $s^2=(\mathbf{y_1-X_1}\widehat{\mathbf{\beta}})'(\mathbf{y_1-X_1}\widehat{\mathbf{\beta}})/N_1$. Согласно стандартной теории и при предположениях о нормальности, $\widehat{\mathbf{\beta}}|\text{ данные } \sim \mathcal{N}[\mathbf{\beta}, \sigma^2[\mathbf{X'_1 X_1}]^{-1}]$ и $s^2 / \sigma^2 | \widehat{\mathbf{\beta}} \sim (N_1-K)\chi^2_{N_1-K}$.

Для начала рассмотрим процедуру наивного единичного восстановления пропущенных значений. Прогноз для $\mathbf{y}_{mis}$ при известном $\mathbf{X_2}$, обозначаемый как $\widehat{\mathbf{y}}_{mis}$, равен $\mathbf{X_2} \widehat{\mathbf{\beta}}$, где $\widehat{\mathbf{\beta}}$ --- описанная выше оценка, полученная на основе лишь первых $N_1$ наблюдений. Тогда
\begin{align}
\widehat{\Expect}[\mathbf{y}_{mis}|\mathbf{X_2}&=\widehat{\mathbf{y}}_{mis}=\mathbf{X_2} \widehat{\mathbf{\beta}}, \\
\widehat{\Var}[\mathbf{y}_{mis}]&\equiv \widehat{\Var}[\widehat{\mathbf{y}}|\mathbf{X_2}]=s^2(\mathbf{I}_{N_2}+\mathbf{X_2}[\mathbf{X'_1 X_1}]^{-1}]\mathbf{X'_2}). \notag
\end{align}
где $s^2\mathbf{I}_{N_2}$ является оценкой $\Var[\mathbf{u}_2]$.

В наивном методе генерируется $N_2$ предсказанных значений $\mathbf{y}_{mis}$, а затем применяются стандартные регрессионные методы для полной восстановленной выборки из $N=N_1+N_2$ наблюдений.

Два шага в наивном методе соответствуют двум шагам в {\bf EM-алгоритме}. Шаг, на котором предсказываются значения, соответствует {\bf E-шагу}, второй этап применения метода наименьших квадратов к расширенной выборке --- это {\bf M-шаг}.



Тем не менее, у этого решения есть недостатки. Во-первых, рассмотрим шаг пополнения выборки. Так как сгенерированные значения $\mathbf{y}_{mis}$ \emph{в точности} в плоскости МНК-прогнозов, то дополнение выборки значениями $(\widehat{\mathbf{y}}_{mis}, \, \mathbf{X_2})$ для получения новой оценки $\mathbf{\widehat{\beta}_A}$ не изменяют прошлого значения $\widehat{\mathbf{\beta}}$:
\begin{align*}
\mathbf{\widehat{\beta}_A}&= [\mathbf{X'_1 X_1}+\mathbf{X'_2 X_2}]^{-1}[\mathbf{X'_1 y_1}+\mathbf{X'}_2 \widehat{\mathbf{y}}_{mis}] \\
&=[\mathbf{X'_1 X_1}+\mathbf{X'_2 X_2}]^{-1}[\mathbf{X'_1 X_1}\widehat{\mathbf{\beta}}+\mathbf{X'_2 X_2}\widehat{\mathbf{\beta}}] \\
&= \widehat{\mathbf{\beta}}.
\end{align*}

Во-вторых, оценка $\sigma^2$, полученная по стандартной формуле из остатков регрессии по расширенной выборке даёт значение, которое слишком мало, так как дополнительные $N_2$ остатка равны нуля по построению,
\begin{align}
s^2_A &= (\mathbf{y-X}\widehat{\mathbf{\beta}}_A)'(\mathbf{y-X}\widehat{\mathbf{\beta}}_A)/N \\
&= (\mathbf{y_1-X_1}\widehat{\mathbf{\beta}})'(\mathbf{y_1-X_1}\widehat{\mathbf{\beta}})/N<s^2 \notag
\end{align}
где в $s^2$ используется верное деление на $N_1$, а не на $N$.

Наконец, из выражения для выборочной дисперсии $\widehat{\mathbf{y}}_{mis}$ можно сделать вывод, что сгенерированные предсказания гетероскедастичны, в отличие от $\mathbf{y}_1$, и, таким образом дисперсия $\widehat{\mathbf{\beta}}_A$ не может быть оценена по обычной формуле метода наименьших квадратов. Наблюдения $\widehat{\mathbf{y}}_{mis}$ получены из распределения с другой дисперсией. В наивном методе не принимается во внимание неопределённость относительно оценок $\widehat{\mathbf{y}}_{mis}$.

Чтобы решить эти проблемы, необходимо модифицировать метод. Во-первых, при оценивании $\widehat{\mathbf{y}}_{mis}$ должна учитываться неопределённость относительно $\widehat{\mathbf{\beta}}$. Это может быть сделано путём корректировки $\widehat{\mathbf{y}}_{mis}$ и добавки некоторого <<шума>> в генерируемые предсказания, так, чтобы оценки пропущенных значений больше походили на полученные из (оценённого или условного) распределения $\mathbf{y}_1$. На шаге стандартизации может быть использован факт того, что оценка $\Var[\widehat{\mathbf{y}}_{mis}]$, $\widehat{\mathbf{\Var}}$, доступна из (27.8). Следовательно, компоненты преобразованной переменной $\widehat{\mathbf{\Var}}^{-1/2}\widehat{\mathbf{y}}_{mis}$ будут иметь единичную дисперсию. Чтобы сымитировать распределение $y_1$, мы можем сгенерировать выборку Монте-Карло из распределения $\mathcal{N}[0, \, s^2]$ и умножить их на $\widehat{\mathbf{\Var}}^{-1/2}\widehat{\mathbf{y}}_{mis}$.

Пересмотренный алгоритм:
\begin{enumerate}
\item	Оценить $\widehat{\mathbf{\beta}}$, используя $N_1$ полных наблюдений, как и раньше.
\item	Сгенерировать значения $\widehat{\mathbf{y}}_{mis}=\mathbf{X}_2\widehat{\mathbf{\beta}}$.
\item	Сгенерировать скорректированные значения $\widehat{\mathbf{y}}_{mis}^a=(\widehat{\mathbf{\Var}}^{-1/2}\widehat{\mathbf{y}}_{mis}) \odot \mathbf{u}_m$, где $\mathbf{u}_m$ сгенерированы  методом Монте-Карло из распределения $\mathcal{N}[0, \, s^2]$ и $\odot$ обозначает поэлементное перемножение.
\item	Используя расширенную выборку, получить исправленное значение оценки $\widehat{\mathbf{\beta}}$.
\item	Повторить этапы 1--4, где для этапа 1 используется исправленное значение оценки $\widehat{\mathbf{\beta}}$.
\end{enumerate}

Шаги данного алгоритма,  также являющегося EM-алгоритмом, повторяются, пока не будет достигнута сходимость, а именно пока изменение в коэффициентах или изменение в сумме квадратов остатков МНК-регрессии не станут меньше заданного порога.

Чтобы связать эту тему с дальнейшим обсуждением, мы дадим алгоритму другую интерпретацию. На шаге 3 генерируется случайная величина из условного распределения $\mathbf{y}$ при заданном $\mathbf{\beta}$, а на шаге 4 генерируется случайная величина из условного распределения $\mathbf{\beta}$ при заданных $s^2$ и $\mathbf{X}$. Этот подход может быть развит далее  путём добавления ещё одного шага, на котором генерируется случайная величина из распределения $s^2$. Мы не будет проходить по всем этапам этого подхода, так как они становятся более понятными в ходе дальнейшего рассмотрения восстановления пропусков в данных.

Альтернативные модели пропущенных значений зависимой переменной были рассмотрены в Главе 16. Они не используют MAR-предпосылку и специфицируют неигнорируемые пропуски. В том случае  рассмотренный выше EM-алгоритм ведёт к несостоятельным оценкам. Цензурированная тобит-модель специфицирует, что пропуски имеют место  для наблюдений с $\mathbf{x'\beta}+u \leqslant 0$, и состоятельными являются тобит-оценки метода максимального правдоподобия (см. Раздел 16.3). Амэмия (1985, стр. 376-376) подробно рассматривает EM-алгоритм для тобит-модели.

\section{Пополнение данных и алгоритм MCMC}
Общая структура байесовского подхода к  данным с пропусками состоит в использовании следующего типа итерационного алгоритма, который включает шаги восстановления пропусков и предсказания. 
{\bf Шаг восстановления пропусков} \emph{(imputation step, I-step)} состоит в генерировании случайного значения из условного предсказанного распределения $\mathbf{Y}_{mis}$. Имея оценки $r$-ого шага,
\begin{equation}
\mathbf{Y}_{mis}^{(r+1)} \sim \Prob[\mathbf{Y}_{mis}|\mathbf{Y}_{obs}, \, \mathbf{\theta}^{(r)}].
\end{equation}
Данное выражение означает, что  случайно генерируется значение $\mathbf{Y}_{mis}^{(r+1)}$ на основе прогнозного условного распределения $\mathbf{Y}_{mis}$ при известном текущем значении оценки $\mathbf{\theta}^{(r)}$ и наблюдаемых значениях $\mathbf{Y}_{obs}$. Стоит отметить, что $\mathbf{Y}_{mis}$ в общем случае представляет собой матрицу, так что описанное выше относится  в принципе к генерации нескольких случайных значений.

{\bf Шаг предсказания} \emph{(prediction step, P-step)} осуществляется путём генерирования значений из апостериорного распределения для полной выборки
\begin{equation}
\mathbf{\theta}^{(r+1)} \sim \Prob[\mathbf{\theta}|\mathbf{Y}_{obs}, \, \mathbf{Y}_{mis}^{(r+1)}].
\end{equation}
Таким образом, $\mathbf{Y}_{obs}$ пополняются восстановленными значениями $\mathbf{Y}_{mis}^{(r+1)}$, сгенерированными из прогнозного распределения $\mathbf{Y}_{mis}$, а случайное значение генерируется из апостериорного распределения $\mathbf{\theta}$. Шаги (27.10) и (27.11) затем могут быть повторены.

Последовательное генерирование случайных значений согласно двум описанным шагам образует марковскую цепь. Этот процесс, который очень похож на EM-алгоритм, по сути является сэмплированием Гиббса из Раздела 13.5.2, но в литературе о пропусках в данных его относят к {\bf методам пополнения выборки}. При выполнении соответствующих условий и согласно теореме из Раздела 13.5.1, случайные значения будут сходиться к стационарному распределению при достаточно большой длине цепи $r$. Последний член цепи дает нам единственное восстановленное значение $\mathbf{Y}_{mis}$. Таким образом мы можем считать, что  $\mathbf{\theta}^{(r)}$ сгенерировано примерно согласно распределения $\Prob[\mathbf{\theta}|\mathbf{Y}_{obs}]$ и $\mathbf{Y}_{mis}^{(r+1)}$ --- примерно согласно распределению $\Prob[\mathbf{Y}_{mis}|\mathbf{Y}_{obs}]$. В любом случае MCMC цепь должна иметь значительную длину для обеспечения уверенности в том, что сходимость действительно имеет место для восстанавливаемых значений. Эта проблема была предметом рассмотрения Главы 13.

После того, как цепь сойдётся, мы получим решение одновременно двух задач --- восстановления пропущенных значений, основанного на модели, специфицированной для данных, а также оценивания модели с использованием как наблюдаемых, так и восстановленных значений переменных. После достижения сходимости мы будем иметь все необходимые данные для расчетов апостериорных моментов $\mathbf{\theta}$ и любых интересующих нас функций от $\mathbf{\theta}$, а также $\mathbf{Y}$, используя идеи из Главы 13.

В качестве примера рассматриваемой процедуры мы снова приведём  регрессию для случая пропущенных значений из предыдущей Раздела. Шаги MCMC-алгоритма следующие:

\begin{enumerate}
\item	На основе наблюдаемых данных вычислить $\widehat{\mathbf{\beta}}=[\mathbf{X'_1 X_1}]^{-1}\mathbf{X'_1 y_1}$ и $\widehat{\mathbf{u}}=(\mathbf{y_1-X_1}\widehat{\mathbf{\beta}})$.
\item	Сгенерировать $\sigma^2$ как $\widehat{\mathbf{u}}'\widehat{\mathbf{u}}$, делённое на случайное значение, полученное из распределения $\chi^2_{N_1-K}$.
\item	Сгенерировать $\mathbf{\beta}|\sigma^2 \sim \mathcal{N}[\widehat{\mathbf{\beta}}, \sigma^2[\mathbf{X'_1 X_1}]^{-1}]$.
\item	Сгенерировать $\mathbf{y}_{mis} \sim \mathcal{N}[\mathbf{X}_2\widehat{\mathbf{\beta}}, \sigma^2]$.
\item	Используя $\mathbf{y}$ вместо $\mathbf{y}_1$ и $\mathbf{X}$ вместо $\mathbf{X}_1$, повторить шаги с 1 по 4 после соответствующих корректировок.
\end{enumerate}

Обоснуем шаг 2. При неинформативном априорном распределении $(\mathbf{\beta}, \sigma^2)$, условное апостериорное распределение $\widehat{\mathbf{u}}'\widehat{\mathbf{u}}/ \sigma^2$ --- это  $\chi^2_{N_1-K}$-распределение, если использованы только наблюдаемые значения. После пополнения выборки используется $\chi^2_{N-K}$-распределение. Для обоснования шага 4 заметим,  что при неинформативном априорном распределении, условное апостериорное распределение $\beta$ --- это $\mathcal{N}[\widehat{\mathbf{\beta}}, \sigma^2[\mathbf{X'_1 X_1}]^{-1}]$. После пополнения выборки оно меняется до $\mathcal{N}[\widehat{\mathbf{\beta}}, \sigma^2[\mathbf{X' X}]^{-1}]$. На шаге 4 восстанавливаются пропущенные значения с использованием условной прогнозной плотности $\mathcal{N}[\mathbf{X}_2\widehat{\mathbf{\beta}}, \sigma^2]$. Эти шаги можно модифицировать, если мы используем, например информативное нормальное-гамма априорное распределение для $(\mathbf{\beta}, \sigma^2)$. Условные апостериорные законы распределения для этого случая приведены в Разделе 13.3.



\section{Множественное восстановление пропусков}

Анализ, проведённый в предыдущем Разделе, показывает, как  MCMC цепь помогает получить однократное восстановление пропуска. Тем не менее, однократное восстановление не позволяет адекватно учесть неопределённость, возникающую в случае пропущенных данных. Это ключевой аргумент в пользу применения процедуры множественного восстановления данных. Условное прогнозное распределение $\mathbf{Y}_{mis}|\mathbf{Y}_{obs}, \, \mathbf{\theta}$ получается путём усреднения по наблюдаемым данным апостериорного распределения $\mathbf{\theta}$:
\[
\Prob[\mathbf{Y}_{mis}|\mathbf{Y}_{obs}] = \int \Prob[\mathbf{Y}_{mis}|\mathbf{Y}_{obs}, \, \mathbf{\theta}] \Prob[\mathbf{\theta}|\mathbf{Y}_{obs}]\mathbf{d\theta}.
\]
Правильное множественное восстановление с точки зрения байесовского подхода отражает неопределённость относительно $\mathbf{Y}_{mis}$, при имеющейся неопределённости относительно  коэффициентов модели.

После {\bf множественного восстановления} пропущенные значения $\mathbf{Y}_{mis}$ заменяются восстановленными значениями $\mathbf{Y}_{mis}^{(1)}, \, \mathbf{Y}_{mis}^{(2)}, \, \mathbf{Y}_{mis}^{(3)}\, \dots, \, \mathbf{Y}_{mis}^{(m)}$. Каждый из восстановленных полных наборов данных затем анализируется как изначально бывший полным. Результаты $m$ раз проведённого анализа будут показывать изменчивость, отражающую неопределённость, возникшую из-за пропусков в данных. С $m$ различными наборами данных возникает вопрос о том, как определять нужное значение $m$ и как объединить $m$ наборов оценок параметров и ковариационных матриц. Мы освещаем оба вопроса, используя существующие результаты, но без подробных обоснований.

При объединении результатов, полученных с помощью множественного восстановления пропусков, основным является следующий результат, верный для произвольной статистики $Q$:
\begin{equation}
\Prob[Q|\mathbf{Y}_{obs}] = 
\int \Prob[Q|\mathbf{Y}_{obs}, \, \mathbf{Y}_{mis}] 
\Prob[\mathbf{Y}_{mis}|\mathbf{Y}_{obs}]\mathbf{d}\mathbf{Y}_{mis}
\end{equation}
Это результат говорит, что настоящее апостериорное распределение $Q$ можно получить путём усреднения апостериорного распределения $Q$ по полным данным. Это означает усреднение  по результатам множественного восстановления пропущенных значений (Рубин, 1996).

Из уравнения (27.12) следует, что финальная оценка $Q$ определяется с помощью закона повторных ожиданий
\begin{equation}
\Expect[Q|\mathbf{Y}_{obs}]=\Expect[\Expect[Q|\mathbf{Y}_{obs}, \, \mathbf{Y}_{mis}]|\mathbf{Y}_{obs}].
\end{equation}
Взяв среднее арифметическое всех $Q_r$, считаемых по полным восстановленным многократно выборкам, мы получим апостериорное среднее значение $Q$.

Финальная дисперсия $Q$ задается формулой
\begin{equation}
\Var[Q|\mathbf{Y}_{obs}]=\Expect[\Var[Q|\mathbf{Y}_{obs}, \, \mathbf{Y}_{mis}]|\mathbf{Y}_{obs}]+ \Var[\Expect[Q|\mathbf{Y}_{obs}, \, \mathbf{Y}_{mis}]|\mathbf{Y}_{obs}],
\end{equation}
здесь используется разложение дисперсии, приведённое в Разделе А.8.

Рубин (1996) также предложил следующие правила для учета информации о моментах, сформулированной в терминах скалярных параметров. Допустим, что для произвольного скалярного параметра, $\widehat{Q}_r$ --- точечная оценка $r$-ого из всех восстановления и $\widehat{U}_r$ --- оценка дисперсии. Тогда определим средние значения точечных оценок и оценок дисперсии соответственно как 
\begin{align}
\bar{Q}&=m^{-1} \sum \limits^{m}_{r=1}\widehat{Q}_r, \\
\bar{Q}&=m^{-1} \sum \limits^{m}_{r=1}\widehat{Q}_r, \\
\end{align}
и ковариацию между оценками разных восстановлений как
\begin{equation}
B=(m-1)^{-1} \sum \limits^{m}_{r=1}(\widehat{Q}_r - \bar{Q})^2
\end{equation}
тогда общая дисперсия равна
\begin{equation}
T=\bar{U}+(1+m^{-1})B.
\end{equation}


\begin{table}[h]
\begin{center}
\caption{\label{tab:27.1} Относительная эффективность множественного восстановления}
\begin{tabular}[t]{lccc}
\hline
\hline
Число & \multicolumn{2}{c}{Пропущено наблюдений ($\lambda$)} \\
восстановлений ($m$) & 10\% & 30\% & 50\% \\
\hline
3  &  0.967 & 0.909 & 0.857 \\
10 &  0.990 & 0.970 & 0.952 \\
20 &  0.995 & 0.985 & 0.975 \\
\hline
\hline
\end{tabular}
\end{center}
\end{table}




Результаты (27.15 и 27.16) следуют из (27.13), уравнение (27.18) следует из (27.14). Шефер (1997) приводит результаты для комбинирования $P$-значений и статистик отношения правдоподобия и дает дополнительные ссылки.

После восстановления пропусков выводы относительно отдельных коэффициентов или групп коэффициентов можно строить на основе финальных оценок, так как стандартная центральная предельная теорема и связанные с ней результаты для больших выборок  могут быть распространены и на этот случай.

Следующая формула отражает меру относительной эффективности $m$ множественных восстановлений:
\begin{equation}
reff=(1+(\lambda/m))^{-1},
\end{equation}
где $\lambda$ --- доля пропущенных наблюдений. Эффективность измеряется относительно ситуации отсутствия пропусков. Арифметические вычисления в Таблице 27.1 показывают, что уже при трёх восстановлениях эффективность может быть равна 97\% при 10\% пропущенных значений и 86\% при 50\% пропущенных значений. Десять и более раз восстановленные данные дают относительную эффективность, превосходящую 95\% при 50\% пропусков. Таким образом, как было отмечено Шефером (1997), количество восстановлений не обязательно должно быть высоким.

\section{Пример восстановления пропусков с помощью MCMC} 

В этом Разделе приведено две иллюстрации  методов восстановления пропусков в данных: метод полного удаления наблюдений с пропусками и метод восстановления пропусков с помощью среднего значения, которые не требуют построения моделей, (см. Раздел 27.2), а также основанный на модели метод пополнения выборки с использованием MCMC алгоритма (см. Раздел 27.6). Пропущенными являются только значения  регрессоров, при этом механизм появления пропусков удовлетворяет предпосылке MAR.

Первый пример --- это  простая множественная регрессия, второй --- логит-регрессию. Для ясности и простоты мы используем искусственно сгенерированные данные с известным процессом, порождающи данные.

\subsection{Линейная регрессия с пропущенными значениями регрессоров} 



В нашем примере линейной регрессии данные порождаются процессом:
\begin{equation}
y_i=\beta_0+\beta_1 x_{1i}+\beta_2 x_{2i}+u_i, \, i=1, \, 2, \dots , \, N,
\end{equation}
с $u_i|x_{1i}, \, x_{2i} \sim \mathcal{N}[0, \, \sigma^2]$ и $(x_{1i}, \, x_{2i})$ имеющими двумерное нормальное распределение
\begin{equation}
\begin{bmatrix}
x_{1i} \\ x_{2i}
\end{bmatrix}
\sim \mathcal{N}
\begin{bmatrix}
\begin{bmatrix} 0 \\ 0 \end{bmatrix}, \begin{bmatrix} 1 & \rho \\ \rho & 1 \end{bmatrix}
\end{bmatrix},
\end{equation}
так что $x_{2i}|x_{1i} \sim \mathcal{N}[\rho x_{1i}, \, 1-\rho^2]$. Также мы зададим $\mathbf{\beta'}=\begin{bmatrix} 1 & 1 & 1 \end{bmatrix}, \, n=1000$, и рассмотрим две доли случайно пропущенных значений $x_1$ и $x_2$ --- 10\% или 25\%. Для любого $i$, могут быть пропущены $x_1$ или $x_2$ или сразу оба регрессора. Мы также используем два разных значения $\rho$, 0.36 и 0.64.

Для марковской цепи 500 первых итераций используются в качестве прожига. Вычисления для марковской цепи были проведены с помощью алгоритма SAS MI Proc, в котором используется неинформативное априорное распределение. Для демонстрационных целей число восстановлений было взято равным 10, но длина цепи после прожига варьировалась от 10 до 10000. Proc MI комбинирует результаты множественных восстановлений, используя Уравнения (27.15)—(27.18).



\begin{table}[h]
\begin{center}
\caption{\label{tab:27.2} Восстановление пропущенных данных: линейная регрессия с 10\% пропущенных данных и высокой корреляций, MCMC алгоритм}
\begin{tabular}[t]{lccccccc}
\hline
\hline
 & Нет пропущенных & Полное & Замещение & \multicolumn{4}{c}{Длина марковской цепи} \\
 & данных & удаление & средним  & 10 & 1000 & 5000 & 10000 \\
\hline
$\hat{\beta}_0$  &  0.919 & 0.913 & 0.899 & 0.910 & 0.911 & 0.909 & 0.903 \\
& (0.104) & (0.113)& (0.105) & (0.102) & (0.101) & (0.103) & (0.101) \\
$\hat{\beta}_1$  &  1.097 & 1.067 & 1.053 & 1.196 & 1.205 & 1.199 & 1.199 \\
& (0.138) & (0.151) & (0.141) & (0.148) & (0.155) & (0.144) & (0.147) \\
$\hat{\beta}_2$  &  1.000 & 1.072 & 1.112 & 1.042 & 1.051 & 1.041 & 1.055 \\
& (0.132) & (0.145) & (0.135) & (0.140) & (0.146) & (0.143) & (0.146) \\
$R^2$ & 0.240 & 0.254 & 0.226 & & & & \\
\hline
\hline
\end{tabular}
\end{center}
\end{table}


\begin{table}[h]
\begin{center}
\caption{\label{tab:27.3} Восстановление пропущенных данных: линейная регрессия с 25\% пропущенных данных и высокой корреляций, MCMC алгоритм}
\begin{tabular}[t]{lccccccc}
\hline
\hline
 & Нет пропущенных & Полное & Замещение & \multicolumn{4}{c}{Длина марковской цепи} \\
 & данных & удаление & средним  & 10 & 1000 & 5000 & 10000 \\
\hline
$\hat{\beta}_0$  &  0.919 & 0.863 & 0.984 & 0.899 & 0.898 & 0.925 &  0.900 \\
& (0.104) & (0.167) & (0.108) & (0.108) & (0.105) & (0.111) & (0.110) \\
$\hat{\beta}_1$  &  1.097 &  1.048 & 1.062 & 1.028 & 1.047 & 1.082 & 0.987 \\
&  (0.138) & (0.167) & (0.150) & (0.152) & (0.166) & (0.161) & (0.155) \\
$\hat{\beta}_2$  &  1.000 & 1.129 &  1.156 & 1.071 & 1.085 &  1.024 & 1.124 \\
&  (0.132) & (0.161) & (0.148) & (0.152) & (0.144) & (0.172) & (0.152)  \\
$R^2$ & 0.240 & 0.268 & 0.203 & & & & \\
\hline
\hline
\end{tabular}
\end{center}
\end{table}

              
      
      




Таблицы 27.2 и 27.3 содержат результаты для высоких $\rho$, а также низких и высоких долей пропусков. Между методами не наблюдается большой разницы. Так как выполнена MAR-предпосылка, то точечные оценки после полного удаления наблюдений с пропусками близки к оценкам по всей выборки.  Но, как и ожидалось, стандартные ошибки после полного удаления выше. При замещении средним точечные оценки $\beta_2$ отличаются больше, но наблюдаемые отличия находится в пределах выборочной ошибки. Оказалось, что в обоих случаях марковская цепь достигает стационарности довольно быстро, между результатами для 10 и 10000 итераций разница невелика. Вероятно, это происходит из-за того, что длина прожига составляет 500 итераций, что может быть больше, чем нужно для этого относительно простого случая.

В Таблице 27.4 симуляционная процедура повторена для <<наихудшего>> сценария: низкого значения $\rho$ и 25\% пропусков. Расхождение между точечными оценками по всей выборке, после полного удаления наблюдений с пропусками и после замещения средним оказалось в общем относительно больше, чем для случаев MCMС. Тем не менее, даже в этом случае отличия от оценок по всей выборке на самом деле не очень велико. Ещё раз мы увидели, что особые преимущества от использования длинной марковской цепи в этом примере выявлены не были.



\begin{table}[h]
\begin{center}
\caption{\label{tab:27.4} Восстановление пропущенных данных: линейная регрессия с 10\% пропущенных данных и низкой корреляций, MCMC алгоритм}
\begin{tabular}[t]{lccccccc}
\hline
\hline
 & Нет пропущенных & Полное & Замещение & \multicolumn{4}{c}{Длина марковской цепи} \\
 & данных & удаление & средним  & 10 & 1000 & 5000 & 10000 \\
\hline
$\hat{\beta}_0$  &  1.121 & 1.162 & 1.142 & 1.149 & 1.155 & 1.154 & 1.141 \\
& (0.099) & (0.130) & (0.103) & (0.104) & (0.103) & (0.104) & (0.101) \\
$\hat{\beta}_1$  & 1.099 & 0.930 & 1.052 & 1.026 & 1.020 & 1.004 & 1.044  \\
& (0.107) & (0.134) & (0.121) & (0.127) & (0.128) & (0.124) & (0.124) \\
$\hat{\beta}_2$  &  1.102 & 1.122 & 1.215 & 1.130 & 1.157 & 1.137 & 1.151 \\
&  (0.107) & (0.134) & (0.124) & (0.128) & (0.129) & (0.129) & (0.119) \\
$R^2$ & 0.243 & 0.235 & 0.186 & & & & \\
\hline
\hline
\end{tabular}
\end{center}
\end{table}


\begin{table}[h]
\begin{center}
\caption{\label{tab:27.5} Восстановление пропущенных данных: логит-регрессия с 10\% пропущенных данных и высокой корреляций, MCMC алгоритм}
\begin{tabular}[t]{lccccccc}
\hline
\hline
 & Нет пропущенных & Полное & Замещение & \multicolumn{4}{c}{Длина марковской цепи} \\
 & данных & удаление & средним  & 10 & 1000 & 5000 & 10000 \\
\hline
$\hat{\beta}_0$  &  -0.447  & -0.498  & -0.439  & -0.527   &  -0.534   & -0.531  & -0.539   \\
                 &  (0.070) & (0.078) & (0.070) &  (0.073) &   (0.073) & (0.072) & (0.073)  \\
$\hat{\beta}_1$  &  -0.597  & -0.658  & -0.602  & -0.620   &  -0.673   & -0.681  & -0.675    \\
                 &  (0.096) & (0.108) & (0.098) &  (0.106) &   (0.102) & (0.101) & (0.103)  \\
$\hat{\beta}_2$  &  -0.444  & -0.474  & -0.523  & -0.597   &  -0.540   & -0.536  & -0.553   \\
                 &  (0.092) & (0.103) & (0.094) &  (0.107) &   (0.103) & (0.099) & (0.101)  \\
\hline
\hline
\end{tabular}
\end{center}
\end{table}







\subsection{Логит-регрессия с пропущенными значениями регрессоров} 
Далее мы рассмотрим пример нелинейной модели с пропущенными значениями регрессоров, используя симулированные данные. В этом симуляционном примере мы продолжим работать с данными, устройство которых было описано ранее, но заменим зависимую переменную на дискретную бинарную. Сначала изменим процесс порождающий данные из примера линейной регрессии, теперь $y=y^*$ --- латентная переменная. То есть:
\begin{equation}
y_i=\beta_0+\beta_1 x_{1i}+\beta_2 x_{2i}+u_i, \, i=1, \, 2, \dots , \, N.
\end{equation}
Тогда бинарная переменная $y_i$ формируется согласно  правилу:



\begin{equation}
y_i=
\begin{cases}
1, \text{ если } y_i^*>0, \\
0, \text{ если } y_i^*\leqslant 0
\end{cases}
\end{equation}



Мы будем моделировать вероятность того, что $y_i=0$, используя логит-модель, хоть подобный процесс порождающий данные подходит и для пробит-модели. Как было рассмотрено в Разделе 14.4.1, в логит-модели идентифицируется вектор параметров $\mathbf{\beta}/\sigma$, где дисперсия равна $\sigma^2=\pi^2/3$. Если все компоненты $\mathbf{\beta}$ равны единице, логит-модель даёт оценки истинных значений параметров, равные приблизительно -0.551. Оценивание с помощью MCMC осуществляется, как и раньше, с использованием неинформативного априорного распределения.

В Таблице 27.5 представлен <<благоприятный>> случай с 10\% пропусков в данных и высокой корреляции между $x_1$ и $x_2$, а в Таблице 27.6 представлен <<менее удачный>> случай с 25\% пропусков и низкой корреляцией между $x_1$ и $x_2$.

В первом случае даже при отсутствии пропусков оценка $\widehat{\beta}_2$ значительно отличается от своего ожидаемого значения. Оценки методом MCMC несколько изменяются, когда длина марковской цепи возрастает с 10 до 1000. Тем не менее, при дальнейшем увеличении длины цепи имеет место лишь небольшое изменение в точечных оценках, что можно интерпретировать как довод в пользу сходимости цепи к стационарному распределению.

Для второго примера с менее удачным процессом порождающим данные результаты представлены в Таблице 27.6. Основная разница состоит в том, что расхождение ожидаемых точечных оценок и оценённых значений несколько больше, чем в предыдущем случае. Тем не менее, говоря в общем, результаты множественного восстановления пропусков в случае логистической регрессии схожи с результатами для линейной регрессии.



\section{Практические соображения}

Основной вывод, касающийся методов анализа, о которых шла речь в этой Главе, состоит в том, что множественное восстановление данных имеет преимущество перед однократным. Более того, подходы, основанные на моделях, более адекватные, чем механические подходы, такие как замена средним или метод карточной колоды. Тем не менее, во многих случаях практического применения процедур восстановления данных  реализация алгоритма MCMC может быть существенно труднее, чем относительно простые примеры, рассмотренные в последнем Разделе.

Следует различать множественное восстановление пропусков, где конечным результатом являются сами данные, и восстановлением, в котором конечный результат состоит из оценок коэффициентов для получения выводов. Хотя обе процедуры могут быть основаны на моделях, вторая может использовать более сложные эконометрические модели. Примеры этого можно найти в работах Браунстоуна и Валетта (1996), Штайнбринкнера (1999), Кенникела (1998), а также Дэйви, Шанахана и Шефера (2001).

Даже когда первостепенно именно восстановление пропущенных значений, без подробного моделирования проблема может быть далеко не простой. Например, в своём исследовании 1995 года <<Обследования Финансового Положения Потребителей>> (Survey of Consumer Finances) Кинникел (1998, стр. 5) отмечает, что:

[Когда] исследование содержит большое количество переменных, имеет место значительная или частичная потеря области данных, характер пропущенной информации неоднороден, законы распределения многих переменных скошены, данные имеют сложную структуру, [тогда] анализ в условиях отсутствия восстановления пропусков будет очень трудной задачей. Более того, любой, кто использует публично доступные наборы данных, будет испытывать недостаток в ключевой информации, которая может оказаться важной для понимания закона распределения пропущенных значений. Таким образом, даже из соображений эффективности, стоит задуматься о  восстановлении пропущенных данных.

Несмотря на сложность проблемы, Кенникелл воспользовался процедурами восстановления пропусков, сходными с рассмотренными в этой Главе.

Штайнбринкнер (1999), также столкнувшийся с проблемой пропусков в данных, в которой полное удаление данных с пропусками <<оставляет эконометриста со слишком маленьким набором данных, чтобы оценивать интересующую модель>>, развил двухшаговую симуляционную процедуру, основанную на максимальном правдоподобии, для оценивания совместного распределения пропущенных значений и для оценивания моделей длительности первого периода обучения.

Для относительно простых случаев может быть использовано такое программное обеспечение, как пакет SAS Proc MI. S-Plus и SOLAS также могут быть полезны. Хороший обзор компьютерных программных пакетов можно найти в работе  Хортона и Липшица (2001). За дополнительной информацией стоит посмотреть Интернет-сайты с соответствующей тематикой.

Большинство методов анализа, на которых строится эта Глава, основаны на предпосылке об игнорируемости механизма появления пропусков в данных. С эконометрической точки зрения, это может быть значительным упрощением. Например, посмотрите работу Лилларда, Смита и Уэлша (1986), которые критиковали метод карточной колоды для восстановления данных о зарплатах. Что можно сделать, если механизм не игнорируем? Как отмечено в Разделе 27.4, неигнорируемый механизм появления пропусков в данных подразумевает, что параметры $\mathbf{\beta}$ и $\mathbf{\psi}$ взаимосвязаны. Тогда нужно специфицировать в явном виде этот механизм, как в случае моделей самоотбора  и моделей смещения из-за истощения выборки (см. Главу 16 и Раздел 23.5.2). Шефер (1997, стр. 28) приводит ссылки на литературу по этой теме.

\section{Библиографические заметки}
Среди важных ранних работ следует отметить работы Литтла и Рубина (1987) и Рубина (1987). Эллисон (2002) приводит относительно нетехническое, но понятное введение в проблему пропущенных значений переменных и соответствующую литературу. Рубин (1996) приводит обзор литературы с исторической точки зрения. Шефер (1997) приводит более полный анализ, который включает в себя случаи категориальных переменных, смешенных дискретно-непрерывных данных и данных комплексных обследований.
\begin{enumerate}
\item В работе Менга (2000) можно найти исторический обзор механизмов появления пропусков.
\item Литтл (1988, 1992) приводит хороший обзор литературы о линейных регрессиях с пропусками в регрессорах, затрагивающей как случаи подходов, основанных на моделях, так и не требующие их.
\end{enumerate}

\section*{Упражнения} 

\begin{enumerate}
\item Пусть есть некоторая регрессионная модель, линейная или нелинейная, с зависимой переменной $y$ и экзогенными переменными $\mathbf{x}$, а также независимыми одинаково распределёнными ошибками $\mathbf{\epsilon}$. Покажите, что, если вероятность пропущенных значений $\mathbf{x}$ независима от $y$, то регрессия, основанная на полном удалении наблюдений с пропусками, обеспечивает состоятельную оценку условной функции среднего. [Подсказка: покажите, что условное распределение $y$ при заданном $\mathbf{x}$ не искажается при наличии пропусков.]

\item (Адаптировано из Гурьеру и Монфора, 1981). Пусть есть регрессионная модель $\mathbf{y}=\beta_1\mathbf{x}+\mathbf{Z\beta_2} + \mathbf{u}$, где $\mathbf{y}$ --– вектор размерности $N \times 1$, $\mathbf{Z}$ --– матрица размерности $N \times K$, а $\mathbf{x}$ --– скалярный регрессор, вектор размерности $N \times 1$, некоторые компоненты которого пропущены. Предположим, что наблюдения пропущены случайно и $\Expect[\mathbf{u}|\mathbf{x, \, Z}]=\mathbf{0}$ и $\Expect[\mathbf{uu'}|\mathbf{x, \, Z}]=\sigma^2\mathbf{I}_N$. И $\mathbf{Z}$, и $\mathbf{y}$ полностью наблюдаемы. 
Следующий механизм предложен для работы с пропущенными значениями. Будем использовать линейную регрессионную модель, связывающую $\mathbf{x}$ и $\mathbf{Z}$, $\mathbf{x}=\mathbf{Z\gamma}+ \mathbf{\epsilon}$, где $\Expect[\mathbf{\epsilon}| \, \mathbf{Z}]=\mathbf{0}$ и $\Expect[\mathbf{\epsilon \epsilon'}| \, \mathbf{Z}]=\sigma^2_{\epsilon}\mathbf{I}_N$. Тогда обозначим $\widehat{\mathbf{\gamma}}=[\mathbf{Z'_c Z_c}]^{-1}\mathbf{Z'_c x_c}$, где индекс c относится к <<полным данным>>. Восстановим пропущенные значения по формуле $\widehat{\mathbf{x}}_m=\mathbf{Z_m}[\mathbf{Z'_c Z_c}]^{-1}\mathbf{Z'_c x_c}$, где $\mathbf{x}_m$ относится к пропущенным наблюдениям и $\mathbf{Z_m}$ –-- к соответствующим значениям $\mathbf{Z}$. Исходная регрессия тогда переоценивается с использованием полного набора из $N$ наблюдений после восстановления пропусков в $\mathbf{x}$ с помощью восстановленных значений.
\begin{enumerate}
\item	Объясните, почему МНК-оценки, основанные на полных и восстановленных данных могут быть смещенными в случае малой выборки.
\item	Какие дополнительные предпосылки нужны для того, чтобы доказать состоятельность МНК-оценок, основанных на полных и восстановленных данных?
\item	Будет ли МНК-оценка эффективной?
\end{enumerate}

\item Рассмотрим утверждение о том, что при оценивании модели после восстановления пропусков в данных  оценки точности оценок скорее всего будут завышены, если не будет сделано никакой корректировки на наличие шага восстановления. Другими словами, восстановленные данные могут быть рассмотрены как сгенерированные переменные и, таким образом, с ними возникают проблемы как при двухшаговогом оценивании, о котором шла речь в Разделе 6.6. Объясните, будет ли корректировка, связанная с восстановлением пропущенных данных, необходима в асимптотике.

\end{enumerate}









\appendix

%%% ссылки: def:A1, th:A12 etc



\chapter{Асимптотическая теория}

\section{Введение}

В этом приложении мы рассмотрим поведение \textbf{последовательности случайных величин} $b_N$ при $N$, стремящемся к бесконечности.

В приложениях индекс $N$ --- это размер выборки, а последовательность $b_N$ --- это оценка, например, $\hat{\beta}$ или $\hat{\theta}$, или один из элементов оценки, например, $N^{-1}\sum_i x_i^2$ или $N^{-1}\sum_i x_i u_i$ в случае МНК с одной объясняющей переменной и без константы, или тестовая статистика.

Для теории оценивания существенно важными являются два аспекта поведения \textbf{последовательности} $b_N$ при $N\to \infty$. Во-первых, мы рассматриваем \textbf{сходимость по вероятности} последовательности $b_N$ к пределу $b$, константе или случайной величине, близкой к $b_N$ в вероятностном смысле, который мы уточним. 
Во-вторых, если предел $b$ --- это случайная величина, то мы изучаем \textbf{предельное распределение}, при этом может потребоваться масштабирование исходной последовательности.

Обычно оценки являются функциями от \textbf{средних арифметических} или \textbf{сумм}. В таком случае свойства оценки проще всего получить, используя результаты о среднем арифметическом, в частности, \textbf{закон больших чисел} и \textbf{центральные предельные теоремы}. Мы используем обозначение $\bar{X}_N=N^{-1}\sum_i X_i$, где $X_i$ --- это произвольные усредняемые случайные величины. Важно отличать это обозначение от обозначения $\mathbf{x}_i$, используемого для вектора объясняющей переменной. Например, для МНК с одним регрессором без константы мы будем применять закон больших чисел к среднему от величин $X_i=x_i^2$ и центральную предельную теорему к среднему от величин $X_i=x_i u_i$.

Таблица A.1. охватывает все определения и теоремы, упоминаемые в этом приложении. Они формулируются без доказательств, но с некоторыми комментариями. Основное внимание уделено асимптотически нормальным оценкам, которые возникают чаще всего при работе с пространственными данными. Дополнительные результаты нужны в случае непараметрического оценивания, параметрического оценивания, если множество возможных значений данных зависит от параметров, и для анализа временных рядов при наличии единичного корня. 

\begin{table}[h]
\caption{\label{tab:pred} Асимптотическая теория: определения и теоремы}
\begin{tabular}{cclc}
\hline 
\hline
Определение & Теорема & Название & Уравнение \\ 
\hline 
A.1 &  & Сходимость по вероятности & (A.1) \\ 
A.2 &   & Состоятельность & (A.2) \\ 
 & A.3 & Теорема Слуцкого & (A.3) \\ 
A.4 &  & Сходимость в среднеквадратичном & (A.4) \\ 
 & A.5 & Неравенство Чебышева & (A.5) \\ 
A.6 &  & Сходимость почти наверное & (A.6) \\ 
A.7 &  & Закон больших чисел & (A.7) \\ 
 & A.8 & Усиленный закон больших чисел &  \\ 
 & A.9 & Закон больших чисел Маркова &  \\ 
A.10 &  & Сходимость по распределению & (A.9) \\ 
 & A.11 & Теорема о непрерывном отображении & (A.10) \\ 
  & A.12 & Теорема о преобразованиях &  (A.11) \\ 
A.13 &  & Центральная предельная теорема & (A.13)  \\ 
 & A.14 & ЦПТ Линберга-Леви & \\
 & A.15 & ЦПТ Ляпунова & \\
 & A.16 & Теорема Крамера-Вольда & \\
 & A.17 & Теорема о нормальности предела произведения & (A.15) \\
A.18 & & Асимптотическое распределение & (A.16) \\
A.19 & & Асимптотическая дисперсия & (A.17) \\
A.20 & & Оценка асимптотической дисперсии & (A.18) \\
A.21 & & Асимптотическая эффективность & (A.19) \\
A.22 & & Стохастический порядок малости & \\
\hline
\hline
\end{tabular} 
\end{table}

Первое ключевое понятие, сходимость по вероятности, представлено в разделе A.2. Законы больших чисел представлены в разделе A.3. Другое ключевое понятие, сходимость по распределению, обсуждается в разделе A.4. Сходимость к нормальному распределению с использованием центральных предельных теорем представлена в разделе A.5. Дальнейшие результаты и терминология для предельных многомерных нормальных распределений представлены в разделе A.6. Стохастический порядок малости, часто используемое обозначение в асимптотической теории, представлен в разделе A.7. В разделе A.8. представлены некоторые полезные свойства математического ожидания.


\section{Сходимость по вероятности}

С силу случайности выборки даже при бесконечном её размере мы никогда не можем быть уверены, что последовательность $b_N$ окажется ближе небольшого наперёд заданного расстояния от своего предела. В роли последовательности $b_N$ часто выступает последовательность оценок $\hat{\theta}$, зачастую чтобы подчеркнуть, что это не одна оценка, а именно последовательность, используют обозначение $\hat{\theta}_N$. Однако мы можем быть почти уверены. Разные точные формулировки этой почти уверенности соответствуют разным типам сходимости последовательности случайных величин. Один из наиболее часто используемых типов --- это сходимость по вероятности.

\subsection{Сходимость по вероятности}

Напомним, что неслучайная последовательность действительных чисел $\{a_N\}$ сходится к числу $a$, если для любого $\e>0$ существует число $N^*=N^*(\e)$, такое что, для всех $N>N^*$ выполнено условие
\[
|a_N-a|<\e
\]
Например, если $a_N=2+3/N$, то предел последовательности равен $a=2$ так как $|a_N-a|=|2+3/N-2|=|3/N|<\e$ для всех $N>N^*=3/\e$.

Когда мы имеем последовательность случайных величин, мы не можем быть уверены в том, что оказались на расстоянии ближе чем на $\e$ от предела в силу случайности. Вместо это мы требуем, чтобы вероятность оказаться на расстоянии ближе чем на $\e$ от предела была сколь угодно близка к единице. Таким образом мы требуем, чтобы для любого $\e > 0$
\[
\lim_{N\to\infty} \P[|b_N-b|<\e]=1
\]

\begin{definition}[Сходимость по вероятности] Последовательность случайных величин $\{b_N\}$ \textbf{сходится по вероятности} к числу $b$, если для любого $\e>0$ и $\delta>0$, существует $N^*=N^*(\e,\delta)$, такое что для любых $N>N^*$ выполнено условие
\label{def:A1}
\begin{equation}
\P[|b_N-b|<\e]>1-\delta
\end{equation}
\end{definition}

Мы используем обозначение $\plim b_N=b$, где \verb|plim| расшифровывается как \textbf{probability limit}, т.е. предел по вероятности, также используется обозначение $b_N \overset{p}{\to}b$.

Заметим, что величина $b$ может быть константой или случайной величиной. Сходимость действительных чисел является частным случаем сходимости по вероятности.

Определение \ref{def:A1} подходит для последовательностей скалярных случайных величин. Обобщение на случай \textbf{векторных случайных величин}, таких как векторных оценок параметров, является несложным. Мы можем либо применить определение к каждому элементу вектора $\mathbf{b}_N$, либо заменить выражение $|b_N-b|$ скаляром $(\mathbf{b}_N-\mathbf{b})'(\mathbf{b}_N-\mathbf{b})=(b_{1N}-b_1)^2+\ldots+(b_{KN}-b_K)^2$ или его квадратным корнем $||\mathbf{b}_N-\mathbf{b}||$. 

Когда последовательность $\{\mathbf{b}_N\}$ является последовательностью оценок параметра $\hat{\mathbf{\theta}}$, имеется следующий аналог несмещённости для большого размера выборки.

\begin{definition}[Состоятельность]
Оценка $\hat{\mathbf{\theta}}$ является \textbf{состоятельной} оценкой для параметра $\mathbf{\theta}_0$ если 
\begin{equation}
\plim \hat{\mathbf{\theta}} = \mathbf{\theta}_0
\end{equation}
\end{definition}

Индекс 0 у $\mathbf{\theta}$ объясняется в разделе 5.2.3. Заметим, что из несмещённости не следует состоятельность. Несмещённость означает только, что ожидаемое значение $\hat{\mathbf{\theta}}$ равно $\mathbf{\theta}_0$, и допускает разброс вокруг $\mathbf{\theta}_0$, который необязательно исчезает, когда размер выборки стремится к бесконечности. Также состоятельная оценка не обязана быть несмещённой. Например, прибавление $1/N$ к несмещённой и состоятельной оценке превратит её в смещённую, но состоятельную.

Хотя последовательность вектора случайных величин $\{\mathbf{b}_N\}$ может сходится к случайному вектору $\mathbf{b}$, во многих эконометрических приложениях $\{\mathbf{b}_N\}$ сходится к вектору констант. Например, мы хотели бы, чтобы оценка неизвестного параметра, сходилась бы к самому параметру. Стоит обратить внимание, что некоторые из следующих результатов верны, только если предельное значение $\mathbf{b}$ является константой.

\begin{theorem}[Теорема Слуцкого]
\label{th:A3}
Пусть $\mathbf{b}_N$ --- конечномерный вектор случайных величин и $g(\cdot)$ --- действительная функция, непрерывная в точке $\mathbf{b}$. Тогда
\begin{equation}
\mathbf{b}_N\overset{p}{\to}\mathbf{b} \Rightarrow g(\mathbf{b}_N)\overset{p}{\to}g(\mathbf{b})
\end{equation}
\end{theorem}


Доказательство приводится в Amemiya (1985, стр. 79). Ruud (2000) формулирует похожую теорему (см. также Rao, 1973, стр. 124), в которой предел $\mathbf{b}$ может быть случайной величиной, при этом функция $g(\cdot)$ должна быть везде непрерывной. Заметим, что некоторые авторы называют теоремой Слуцкого теорему \ref{th:A12} ниже.

Теорема \ref{th:A3}  --- это одна из главный причин, по которой большинство результатов в эконометрике формулируются асимптотически, а не для конечных выборок. Утверждение теоремы является очень удобным свойством, которое не является верным для математических ожиданий. Например, из равенства $\plim$ $(b_{1N},b_{2N})=(b_1,b_2)$ следует, что $\plim$ $(b_{1N}b_{2N})=b_1b_2$, в то время как в общем случае $\E[b_{1N}b_{2N}]$ не равно $\E[b_{1}]\E[b_{2}]$.



\subsection{Другие виды сходимости}

Часто бывает легко определить наличие другого вида сходимости, из которой будет следовать сходимость по вероятности. Эти виды сходимости приведены для полноты изложения. Законы больших чисел, изложенные в следующем разделе, используются гораздо чаще.

\begin{definition}[Сходимость в среднем квадратичном] Последовательность случайных величин $\{b_N\}$  \textbf{сходится в среднем квадратичном} к случайной величине $b$, если 
\begin{equation}
\lim_{N\to\infty} \E[(b_N-b)^2] = 0
\end{equation}
\end{definition}

Мы используем обозначение $b_N \overset{m}{\to}b$. Сходимость в среднем квадратичном полезна тем, что из $b_N\overset{m}{\to} b$ следует сходимость по вероятности $b_N\overset{p}{\to}b$ (см. Rao, 1973, стр. 110), кроме того, её обычно легче доказать. При этом требуется существование дисперсии величин $b_N$. Если $\E[b_N]=b$, тогда остается только доказать, что дисперсия $b_N$ стремится к нулю при $N\to\infty$. Если величины $b_N$ являются смещенными оценками для $b$, тогда мы требуем, чтобы сумма дисперсии и возведенного в квадрат смещения стремилась к нулю.

Другой результат, используемый при доказательстве сходимости по вероятности, --- это неравенство Чебышева.

\begin{theorem}[Неравенство Чебышева]
\label{th:A5}
Для любой случайной величины $Z$ со средним $\mu$ и с дисперсией $\sigma^2$, для любого $k > 0$,
\begin{equation}
\P[(Z-\mu)^2>k]\leq \sigma^2/k
\end{equation}
\end{theorem}

Доказательство можно найти в Rao (1973, стр. 95). В обобщенном неравенстве Чебышева величина $(Z-\mu)^2$ из теоремы \ref{th:A5} заменена на произвольную неотрицательную функцию $g(Z)$, и оно приобретает вид $\P[g(Z)>k]\leq \E[g(Z)]/k$ для любых $k>0$, см. Amemiay (1985, стр. 87).

Теорема \ref{th:A5} может быть использована для доказательства сходимости по вероятности, при этом $Z$ заменяется на $b_N$. Для  теоремы требуются математическое ожидание и дисперсия $b_N$, которые легко можно получить для оценок, рассчитываемых с помощью среднего арифметического независимых случайных величин. Впрочем, в этом случае можно воспользоваться более простым способом --- напрямую применить закон больших чисел к среднему и получить предел по вероятности.

Концептуально более сложное понятие сходимости --- сходимость почти наверное.

\begin{definition}[Сходимость почти наверное] Последовательность случайных величин $\{b_N\}$ \textbf{сходится почти наверное } к $b$, если
\begin{equation}
\P[\lim_{N\to\infty} b_N = b] = 1
\end{equation}
\end{definition}
Сходимость почти наверное обозначается $b_N \overset{as}{\to} b$. Из сходимости почти наверное следует сходимость по вероятности (см. Rao, 1973, стр. 111). Сходимость по вероятности допускает более вольное поведение $b_N$, чем сходимость почти наверное.

Сходимость почти наверное также называется \textbf{сильной состоятельностью} в отличие от сходимости по вероятности, называемой \textbf{слабой состоятельностью}. Сходимость по вероятности легче для понимания и достаточна для большинства эконометрических приложений.



\section{Законы больших чисел}

Законы больших чисел --- это теоремы о сходимости по вероятности или о сходимости почти наверное для частного случая, когда последовательность оценок $\{b_N\}$ является последовательностью выборочных средних, т.е. $b_N=\bar{X}_N$, где

\begin{equation}
\label{eq:A7}
\bar{X}_N=\frac{1}{N}\sum_{i=1}^N X_i
\end{equation}

Здесь $X_i$ обозначает произвольную случайную величину, это обозначение может использоваться не только для регрессора.

С помощью закона больших чисел легче доказывать сходимость по вероятности, чем используя $(\delta,\e)$-язык определения \ref{def:A1} или чем используя другие виды сходимости.

\begin{definition}[Закон больших чисел]  \textbf{Слабый закон больших чисел} (ЗБЧ) накладывает условия на отдельные величины $X_i$ в выражении $\bar{X}_N$, при которых
\begin{equation}
\label{eq:A8}
(\bar{X}_N - \E[X_N]) \overset{p}{\to} 0
\end{equation}
\textbf{Усиленный закон больших чисел} подразумевает сходимость почти наверное.
\end{definition}

Бывает полезно представлять, что ЗБЧ означает сходимость $\bar{X}_N$ к своему математическому ожиданию. Строго говоря, ЗБЧ означает лишь более слабое условие, что $\bar{X}_N$ сходится к \textit{предельному значению своего математического ожидания}, т.к. из \ref{eq:A8}  следует, что

\[
\plim \bar{X}_N = \lim \E[\bar{X}_N]
\]

Если у $X_i$ одинаковое математическое ожидание $\mu$, то ЗБЧ означает, что $\plim \bar{X}_N = \mu$.

Две наиболее популярные формулировки закона больших чисел таковы:

\begin{theorem}[ЗБЧ Колмогорова] 
\label{th:A8}
Пусть $\{X_i\}$ независимы и одинаково распределены. Если $\E[X_i]=\mu$ и $\E[|X_i|]<\infty$, то $(\bar{X}_N - \E[X_N]) \overset{as}{\to} 0$.
\end{theorem}

\begin{theorem}[ЗБЧ Маркова] 
\label{th:A9}
Пусть $\{X_i\}$ независимы и необязательно одинаково распределены с $\E[X_i]=\mu_i$  и $V[X_i]=\sigma^2_i$. Если для некоторого $\delta>0$ выполнено условие $\sum_{i=1}^{\infty} (\E[|X_i-\mu_i|^{1+\delta}]/i^{1+\delta}) < \infty$, то $(\bar{X}_N - \E[X_N]) \overset{as}{\to} 0$.
\end{theorem}

Формулировку данных теорем можно найти в книге White (2001a, стр. 32 и стр. 35), доказательство --- в работе Rao (1973, стр. 114-116). Оба закона относятся к сходимости почти наверное, из которой следует сходимость по вероятности. Rao (1973) называет теорему \ref{th:A8} вторым ЗБЧ Колмогорова и приводит теорему \ref{th:A9} для частного случая $\delta=1$, называя её первым ЗБЧ Колмогорова.

ЗБЧ Колмогорова допускает даже бесконечную дисперсию $X_i$ за счёт требования одинаковой распределённости. Заключение теоремы, поэтому, можно упростить до $\bar{X}_N \overset{as}{\to} \mu$, где $\mu=E[X]$. Существует слабая версия этого закона, называемая теоремой Хинчина, утверждающая, что для последовательности независимых одинаково распределённых $\{X_i\}$, у которых существует $\E[X]$, средние $\bar{X}_N$ сходятся по вероятности. Формулировки Хинчина достаточно для большинства эконометрических приложений.

ЗБЧ Маркова не требует одинаковой распределённости, но требует существования момента с порядком выше первого. Естественный выбор $\delta$ --- это $\delta=1$. В этом случае требуется существование дисперсии и дополнительное условие $\sum_{i=1}^{\infty} (\sigma^2_i/i^2) < \infty$. Дисперсии могут меняться и даже расти с $i$ при условии, что они растут не слишком быстро, т.е. сумма $\sum_{i=1}^{\infty} (\sigma^2_i/i^2)$ остаётся конечной. Например, дополнительное условие выполнено, если $\sigma^2_i=\sigma^2$, т.к. $\sum_{i=1}^{\infty} 1/i^2$ сходится, и не выполнено, если $\sigma^2_i=i\sigma^2$, т.к. $\sum_{i=1}^{\infty} 1/i$ расходится.

Во многих микроэконометрических задачах, включая регрессии со \textit{стратифицированной выборкой} или с \textit{фиксированными регрессорами}, нужна более сложная версия ЗБЧ --- версия Маркова.

ЗБЧ удобны тем, что накладывают условия на отдельные величины $X_i$, а не на последовательность средних $\bar{X}_N$. Именно с помощью ЗБЧ эконометристы чаще всего доказывают сходимость по вероятности. Связано это с тем, что большинство оценок и тестовых статистик являются функциями от средних арифметических исходных данных или ненаблюдаемых величин.

\section{Сходимость по распределению}

При $N\to\infty$ распределение состоятельной оценки $\hat{\theta}$ становится \textit{вырожденным} и сосредотачивается в точке $\theta_0$. Чтобы получить \textit{невырожденное} распределение при $N\to\infty$ необходимо \textit{отмасштабировать} оценку $\hat{\theta}$. Часто в качестве масштабирующего множителя подходит $\sqrt{N}$, в этом случае мы рассматриваем последовательность случайных величин $b_N=\sqrt{N}(\hat{\theta}-\theta_0)$.

В общем случаем $N$-ая случайная величина в последовательности $b_N$ имеет чрезвычайно сложную функцию распределения $F_N$. Как и любая последовательность функций, последовательность $F_N$ функций распределения может сходится к предельной функции. Здесь подразумевается сходимость в естественном смысле.

\begin{definition}[Сходимость по распределению] 
\label{def:A10}
Последовательность случайных величин $\{b_N\}$ \textbf{сходится по распределению} к случайной величине $b$ если
\begin{equation}
\label{eq:A9}
\lim_{N\to\infty} F_N = F
\end{equation}
в точках непрерывности функции $F$. Здесь $F_N$ --- функция распределения величины $b_N$, $F$ --- функция распределения величины $b$, сходимость --- поточечная.
\end{definition}

Мы используем обозначение $\b_N \overset{d}{\to}b$, а функцию $F$ мы называем \textbf{предельным распределением} последовательности $\{b_N\}$.

Из сходимости по вероятности следует сходимость по распределению, т.е. из $\b_N \overset{p}{\to}b$ следует $\b_N \overset{d}{\to}b$ (см. Rao, 1973, стр. 122).

В общем случае обратное неверно. Например, пусть $b_N=X_N$, $N$-ая копия независимых и одинаково распределённых величин $X \sim \mathcal{N}[\mu,\sigma^2]$. В этом случаем $\b_N \overset{d}{\to}b\sim \mathcal{N}[\mu,\sigma^2]$, но дисперсия $(b_N-b)$ не уменьшается с $N\to \infty$, а значит $b_N$ не сходится по вероятности к $b$.

В частном случаем, когда $b$ --- константа, тем не менее, из $\b_N \overset{d}{\to}b$ следует $\b_N \overset{p}{\to}b$ (см. Rao, 1973, стр. 120). В этом случае предельное распределение является вырожденным, сосредоточенным в точке $b$.

Чтобы обобщить предельное распределение на случай \textbf{векторных случайных величин}, просто определим $F_N$ и $F$ соответственно как функции распределения векторов $\mathbf{b}_N$ и $\mathbf{b}$.

\begin{theorem}[Теорема о непрерывном отображении]
\label{th:A11}
Пусть $\mathbf{b}_N$ --- последовательность конечномерных векторных случайных величин и $g(\cdot)$ --- непрерывная вещественная функция. Тогда
\begin{equation}
\mathbf{b}_N \overset{d}{\to} \mathbf{b} \Rightarrow g(\mathbf{b}_N) \overset{d}{\to} g(\mathbf{b})
\end{equation}
\end{theorem}

Доказательство можно найти в Rao (1973, стр. 124). Теорема \ref{th:A11} --- это аналог теоремы \ref{th:A3} для случая сходимости по распределению.

Следующая теорема описывает последствия умножения, сложения и деления последовательности с пределом по распределению на последовательность, сходящуюся по вероятности к константе.

\begin{theorem}[Теорема о преобразованиях]
\label{th:A12}
Если $a_N \overset{d}{\to} a$ и $b_N \overset{p}{\to} b$, где $a$ --- случайная величина, а $b$ --- константа, то

\begin{enumerate}
\item $a_N+b_N \overset{d}{\to} a+ b$ 
\item $a_Nb_N \overset{d}{\to} a b$ 
\item $a_N/b_N \overset{d}{\to} a/ b$ если $\P[b=0]=0$
\end{enumerate}
\end{theorem}

Доказательство можно найти в Rao (1973, стр. 122). Теорему \ref{th:A12} также называют \textit{теоремой Крамера}. Она также называется теоремой Слуцкого. Мы называем теоремой Слуцкого теорему \ref{th:A3}.

Теорема \ref{th:A12} очень полезна тем, что позволяет найти отдельно предельное распределение последовательности $a_N$ и предел по вероятности последовательности $b_N$, вместо того, чтобы исследовать совместное поведение пары $a_N$ и $b_N$. Из-за своей важности результат с умножением последовательностей даже имеет специальное название \textit{правило произведения}.

\section{Центральная предельная теорема}

Центральные предельные теоремы --- это теоремы \textit{о сходимости по распределению} для последовательности $\{b_N\}$  \textit{выборочных средних}. Центральная предельная теорема --- это простой способ получить предельное распределение последовательности $\{b_N\}$, гораздо более простой, чем прямое использование формулы \ref{eq:A9}.

Из ЗБЧ следует, что выборочное среднее сходится к константе $\lim \E[\bar{X}_N]$, т.е. к вырожденному распределению. Поэтому мы отмасштабируем величину $(\bar{X}_N-\E[\bar{X}_N])$ на её стандартное отклонение, чтобы построить величину с единичной дисперсией, которая могла бы сходиться к невырожденному распределению.

\begin{theorem}[Центральная предельная теорема]
Пусть 
\begin{equation}
Z_N=\frac{\bar{X}_N-\E[\bar{X}_N]}{\sqrt{\V[\bar{X}_N]}}
\end{equation}
где $\bar{X}_N$ --- выборочное среднее. \textbf{Центральная предельная теорема} (ЦПТ) формулирует условия на отдельные слагаемые $X_i$ в $\bar{X}_N$ при которых
\begin{equation}
Z_N \overset{d}{\to} \mathcal{N} [0,1]
\end{equation}
т.е. при которых $Z_N$ сходится по распределению к стандартному нормальному.
\end{theorem}

По построению у величины $Z_N$ ожидание равно $0$, а дисперсия равна $1$, поэтому остаётся доказать только нормальность. Формальные доказательства ЦПТ используют характеристическую функцию, обобщение производящей функции моментов $Z_N$. При этом доказывается что с $N\to \infty$ производящие функции сходятся к производящей функции для стандартной нормальной случайной величины.

Заметим, что если $\bar{X}_N$ удовлетворяет условиям ЦПТ, то и $h(N)\bar{X}_N$ также удовлетворяет условиям ЦПТ, например, для $h(N)=\sqrt{N}$, так как

\[
Z_N=\frac{h(N)\bar{X}_N-\E[h(N)\bar{X}_N]}{\sqrt{\V[h(N)\bar{X}_N]}}
\]

Во многих приложениях удобно применять центральную предельную теорему к нормализованным $\sqrt{N}\bar{X}_N=N^{-1/2}\sum_{i=1}^{N} X_i$, т.к. $\V[\sqrt{N}\bar{X}_N]$ конечна.

Примеры центральных теорем включают следующие:

\begin{theorem}[ЦПТ Линдеберга-Леви]
\label{th:A14}
Если величины $\{X_i\}$ независимы и одинаково распределены с $\E[X_i]=\mu$ и $\V[X_i]=\sigma^2$, то $Z_N \overset{d}{\to} \mathcal{N}[0,1]$.
\end{theorem}

Доказательство можно найти в Rao (1973, стр. 127).

Эту версию ЦПТ можно найти в большинстве вводных учебников по статистике и она полезна в случае независимых и одинаково распределённых слагаемых. В этом случае выражение для $Z_N$ можно упростить до
\[
Z_N=\frac{\bar{X}_N-\mu}{\sigma/\sqrt{N}}.
\]

Заметим, что в случае независимых и одинаково распределённых случайных величин существования $\mu$ достаточно, чтобы $\bar{X}_N \overset{p}{\to} \mu$, однако чтобы получить предельное нормальное распределение, требуется дополнительная предпосылка о существовании $\sigma^2$.

В таких приложениях как, например, МНК с фиксированными регрессорами, предпосылка о независимости и одинаковой распределённости не подходит. В этом случае можно применить ЦПТ к независимым и неодинаково распределённым $\{X_i\}$, при этом требуются дополнительные предпосылки.

\begin{theorem}[ЦПТ Ляпунова]
\label{th:A15}
Пусть $\{X_i\}$ независимы с $\E[X_i]=\mu_i$ и $\V[X_i]=\sigma^2_i$. Если для некоторого $\delta>0$ выполнено условие 
\[
\lim_{N\to\infty} \frac{\sum_{i=1}^N \E[|X_i-\mu_i|^{2+\delta}]}{\left( \sum_{i=1}^N \sigma_i^2 \right)^{(2+\delta)/2}} = 0
\]
то $Z_N \overset{d}{\to} \mathcal{N}[0,1]$.
\end{theorem}

Данный вариант ЦПТ Ляпунова доказан в White (2001a, стр. 119). Rao (1973, стр. 128) рассматривает частный случай $\delta=1$.

Основным дополнительным требованием ЦПТ Ляпунова является существование абсолютного момента порядка большего, чем два. Следует отметить также и другие дополнительные предположения по сравнению со случаем одинаково распределённых случайных величин. Для неодинаково распределённых $X_i$ 

\[
Z_N=\frac{\sum_{i=1}^N X_i - \sum_{i=1}^N \mu_i}{\sqrt{\sum_{i=1}^N \sigma^2_i}}
\]

Теоремы \ref{th:A14} и \ref{th:A15} являются частными случаями более общей ЦПТ Линдеберга-Феллера (см. Rao, 1973, стр. 128). У ЦПТ Линдеберга-Феллера есть дополнительная предпосылка, которую трудно проверять.

В большинстве микроэконометрических приложений таких, как \textit{стратифицированные выборки} или \textit{фиксированные регрессоры}, используется более сложная версия ЦПТ --- версия Ляпунова.

\section{Многомерное нормальное предельное распределение}

В этом разделе мы сосредоточимся на распространённых в микроэконометрике оценках с многомерным нормальным предельным распределением.

\subsection{Многомерное нормальное предельное распределение}

Рассмотрённые ранее ЦПТ относились к скалярным последовательностям случайных величин. Они могут быть обобщены на случай векторных случайных величин с помощью следующего утверждения:

\begin{theorem}[Теорема Крамера-Вольда]
Пусть $\{\mathbf{b}_N\}$ --- последовательность случайных векторов размера $k\times 1$. Если последовательность $\mathbf{\lambda}'\mathbf{b}_N$ сходится к нормальному распределению для любого вектора констант $\mathbf{\lambda}$ размера $k\times 1$, то последовательность $\mathbf{b}_N$ сходится к многомерному нормальному распределению.
\end{theorem}

Rao (1973, стр. 128) приводит более общий результат, неограниченный нормальным распределением.

Достоинство данного результата в том, что если $\mathbf{b}_N$ --- это вектор средних арифметических, то $\mathbf{\lambda}'\mathbf{b}_N=\lambda_1 b_{1N}+\ldots + \lambda_k b_{kN}$ будет скаляром и к нему будет применима скалярная ЦПТ из предыдущего раздела. Это приведёт к тому, что

\[
\frac{\mathbf{\lambda}'\mathbf{b}_N-\mathbf{\lambda}'\mathbf{\mu}_N}
{\sqrt{\mathbf{\lambda}'\mathbf{\V}_N \mathbf{\lambda}}} \overset{d}{\to} \mathcal{N}[0,1],
\]
где $\mathbf{\mu}_N=\E[\mathbf{b}_N]$ и $\mathbf{\V}_N=\V[\mathbf{b}_N]$. В этом случае мы получаем, что

\begin{equation}
\label{eq:A14}
\mathbf{\V}^{-1/2}(\mathbf{b}_N-\mathbf{\mu}_N)\overset{d}{\to}
\mathcal{N}[\mathbf{0},\mathbf{I}].	
\end{equation}

Этот результат объясняется дальше в подразделе A.6.3.

\subsection{Линейное преобразование}

Микроэконометрические оценки часто могут быть представлены в виде $\sqrt{N}(\hat{\mathbf{\theta}}_N-\mathbf{\theta}_0)=\mathbf{H}_N\mathbf{a}_N$, где $\plim \mathbf{H}_N$ существует и $\mathbf{a}_N$ имеет предельное нормальное распределение. Распределение этого произведения, или линейного преобразования для $\mathbf{a}_N$, может быть получено с помощью части два теоремы \ref{th:A12} (теоремы о преобразовании). Мы переформулируем это утверждение в том виде, в котором оно часто встречается:

\begin{theorem}[Теорема о нормальности предела произведения]
\label{th:A17}
Если вектор $\mathbf{a}_N \overset{d}{\to} \mathcal{N}[\mathbf{\mu},\mathbf{A}]$ и матрица $\mathbf{H}_N\overset{p}{\to} \mathbf{H}$, где $\mathbf{H}$ положительно определена, то
\begin{equation}
\label{eq:A15}
\mathbf{H}_N\mathbf{a}_N \overset{d}{\to} \mathcal{N}
[\mathbf{H}\mathbf{\mu},\mathbf{H}\mathbf{A}\mathbf{H}'].
\end{equation}
\end{theorem}

Теорема \ref{th:A17} может быть непосредственно применена к оценке. Например, МНК оценка

\[
\sqrt{N}(\hat{\mathbf{\beta}}-\mathbf{\beta}_0)=
\left(
\frac{1}{N}\mathbf{X}'\mathbf{X}
\right)^{-1}
\frac{1}{\sqrt{N}}\mathbf{X}'\mathbf{u}
\]
рассматривается как произведение $\mathbf{H}_N=(N^{-1}\mathbf{X}'\mathbf{X})^{-1}$ и $\mathbf{a}_N=N^{-1/2}\mathbf{X}'\mathbf{u}$. Поэтому мы ищем предел по вероятности для $\mathbf{H}_N$ и предельное распределение $\mathbf{a}_N$.

Теорема \ref{th:A17} может быть использована для оправдания \textit{замены} ковариационной матрицы предельного распределения на состоятельную оценку без замены предельного распределения. Если мы доказали, что 

\[
\sqrt{N}(\hat{\mathbf{\theta}}-\mathbf{\theta}_0) \overset{d}{\to}
\mathcal{N}[\mathbf{0},\mathbf{B}],
\]
то из теоремы \ref{th:A17} следует, что
\[
\mathbf{B}_N^{-1/2}\sqrt{N}(\hat{\mathbf{\theta}}-\mathbf{\theta}_0) \overset{d}{\to}
\mathcal{N}[\mathbf{0},\mathbf{I}]
\]
для любой положительно определенной состоятельной оценки $\mathbf{B}_N$ матрицы $\mathbf{B}$.


\subsection{Предельная ковариационная матрица} % (fold)

Собственно ЦПТ позволяет сформулировать довольно запутанный результат \ref{eq:A14}. Домножив на $\bV_N^{1/2}$ и применив теорему \ref{th:A17}, мы получим более удобную формулировку:

\[
\bb_N-\bmu_N \dto \cN[\mathbf{0},\bV],
\]
где $\bV=\plim \bV_N$ и мы предполагаем, что $\bb_N$ и $\bV_N$ правильным образом отмасштабированы, т.е. $\bV$ существует и положительно определена.

Разные авторы по-разному определяют \textbf{предельную ковариационную матрицу} $\bV$. 

Общее определение --- просто

\[
\bV = \plim \bV_N.
\]

Это наиболее общий способ представить данный результат, именно в этом виде он используется в данной книге. В случае фиксированных регрессоров результат упрощается до $\bV = \lim \bV_N$.

В микроэконометрических приложениях матрица $\bV_N$ часто оказывается средним арифметическим некоторых матриц, скажем,

\[
\bV_N=\frac{1}{N}\sum_{i=1}^{N} \bS_i,
\]
где $\bS_i$ --- это квадратные матрицы, зависящие от параметров и данных, относящихся к $i$-му наблюдению. При независимости по $i$, как правило, можно применить закон больших чисел и получить, что $\bV_N-\E[\bV_N]\pto 0$. Тогда оказывается, что

\[
\bV = \lim \E[\bV_N] = \lim \frac{1}{N} \sumton \E[\bS_i].
\]

Выражения такого вида использует Amemiya (1985).

Если $\bS_i$ независимы и одинаково распределены, то $\E[\bS_i]=\E[\bS]$ одинаково для всех наблюдений. Поэтому простая случайная выборка приводит к простому выражению 

\[
\bV = \E[\bS],
\]
этим видом, например, пользуются Newey и McFadden (1994) и Wooldridge (2002).

Например, рассмотрим оценку МНК с гомоскедастичной ошибкой. Здесь $\sqrt{N}(\mathbf{\hat{\beta}}-\mathbf{\beta}_0) \dto \cN [\mathbf{0},\sigma^2 \mathbf{M}^{-1}_{\mathbf{XX}}]$. Матрицу $\mxx=\plim N^{-1} \sum_i \mathbf{x}_i\mathbf{x}_i'$ можно представить в виде $\mxx = \lim N^{-1} \sum_i \E[\mathbf{x}_i\mathbf{x}_i']$, если применим закон больших чисел, а $\mxx=\E[\bx\bx']$ при простой случайной выборке.

Матрица $\bV$ может принимать сложные формы, например, она может принять сэндвич форму $\bA\bB\bA'$. Тогда указанные идеи применяются к каждой компоненте. Например, $\bB = \plim \bB_N$ может быть представлен в виде $\bB = \lim \E[\bB_N]$ или в виде $\bB=\E[\bS]$ при случайной выборке, если $\bB=N^{-1}\sum_i \bS_i$.

\subsection{Асимптотическое распределение и дисперсия}

При выводе предельного распределения оценки мы работаем с последовательностью $\b_N=\sqrt{N}(\hat{\theta}-\theta_0)$, чтобы обеспечить ненулевую дисперсию $b_N$ при $N\to\infty$. Если предельное распределение $b_N$ оказывается нормальным, то многие авторы говорят, что оценка $b_N$ асимптотически нормальна, и называют предел ковариационной матрицы \textbf{асимптотической ковариационной матрицей} $b_N$.

Иногда бывает удобно выразить результат в терминах распределения и ковариационной матрицы самой оценки $\hat{\mathbf{\theta}}$.

\begin{definition}[Асимптотическое распределение $\hat{\btheta}$]
\label{def:A18}
Если 
\begin{equation}
\label{eq:A16}
\sqrt{N}(\hat{\btheta}-\btheta_0)\dto \cN[\mathbf{0},\bB],	
\end{equation}
то мы говорим, что \textit{при большой выборке} оценка $\hat{\btheta}$ \textbf{асимптотически нормально распределена}, и обозначаем это
\begin{equation}
\label{eq:A17}
	\hat{\btheta} \sim \cN [\btheta_0,N^{-1}\bB]
\end{equation}
где слова <<при большой выборке>> означают, что $N$ достаточно велико, чтобы \ref{eq:A16} было хорошим приближением, но не настолько велико, что дисперсия в \ref{eq:A17} занулилась.
\end{definition}

Формула \ref{eq:A17} следует из \ref{eq:A16}, так как деление случайной величины на $\sqrt{N}$ приводит к делению дисперсии на $N$.

Удобно неявно подразумевать асимптотическую нормальность и использовать следующую терминологию:

\begin{definition}[Асимптотическая ковариационная матрица $\hat{\btheta}$]
\label{def:A19}
Если \ref{eq:A16} выполнено, мы говорим, что \textbf{асимптотическая ковариационная матрица} оценки $\hbtheta$ это:
\begin{equation}
	\label{eq:A18}
	\V[\hbtheta]=N^{-1}\bB
\end{equation}	
\end{definition}

\begin{definition}[Оцененная асимптотическая ковариационная матрица $\hat{\btheta}$]
\label{def:A20}
Если \ref{eq:A16} выполнено, мы говорим, что \textbf{оценённая асимптотическая ковариационная матрица} оценки $\hbtheta$ это:
\begin{equation}
	\label{eq:A19}
	\hat{\V}[\hbtheta]=N^{-1}\hat{\bB},
\end{equation}	
где $\hat{\bB}$ --- это состоятельная оценка матрицы $\bB$.
\end{definition}

Некоторые авторы используют обозначения $\mathrm{Avar}[\hbtheta]$ и $\widehat{\mathrm{Avar}}[\hbtheta]$ в определениях \ref{def:A19} и \ref{def:A20}, чтобы избежать потенциальной путаницы с дисперсией $\V[\cdot]$. В данной книге очень малое количество оценок имеют явное выражение для дисперсии в конечных выборках, поэтому $\V[\hbtheta]$ обозначает асимптотическую ковариационную матрицу оценки.


Пример к определениям \ref{def:A18}-\ref{def:A20}. Если $\{X_i\}$ независимы и одинаково распределены $[\mu, \sigma^2]$, то из ЦПТ Линдеберга-Леви следует, что $\sqrt{N}(\bar{X}_N-\mu)/\sigma \dto \cN [ 0,1]$, или, по другому говоря, $\sqrt{N}\bar{X}_N \dto \cN [\mu,\sigma^2]$. Мы говорим, что асимптотически $\bar{X}_N \sim \cN[\mu,\sigma^2/N]$, асимптотическая дисперсия $\bar{X}_N$ равна $\sigma^2/N$, оценка асимптотической дисперсии $\bar{X}_N$ равна $s^2/N$, где $s^2$  --- это состоятельная оценка для $\sigma^2$. В качестве $s^2$ подойдет, например, $s^2=\sum_i (X_i - \bar{X}_N)^2/(N-1)$.

\subsection{Асимптотическая эффективность}

Для конечной выборки по неравенству Крамера-Рао нижняя граница для ковариационной матрицы несмещённой оценки равна $-(\E[\partial^2 \ln L_N/\partial \btheta\btheta' |_{\btheta_0}])^{-1}$. Этот результат можно обобщить на случай асимптотически нормальных оценок.

\begin{definition}[Асимптотическая эффективность]
\label{def:A21}
	Состоятельная асимптотически нормальная оценка $\hbtheta$ параметра $\btheta$ называется \textbf{асимптотически эффективной}, если её асимптотическая ковариационная матрица совпадает с нижней границей Крамера-Рао. 
\end{definition}

\section{Стохастический порядок малости}

Для описания скорости сходимости удобно использовать обозначения $(O,o)$, О-большое и о-малое.

Последовательность неслучайных чисел $a_N$ --- это $O(g(N))$, если $\lim (a_N/g(N))$ конечен и ненулевой, и $o(g(N))$, если $\lim (a_N/g(N))$ нулевой. Другими словами, последовательность $a_N$ --- это $O(g(N))$, если она того же порядка малости, что и функция $g(N)$, и $o(g(N))$, если она более высокого порядка малости, чем $g(N)$. Например, последовательность $(3/N)+(5/N)^2$ --- это $O(1/N)$ или $O(N^{-1})$, т.к. при больших $N$ она пропорциональна $N^{-1}$. Рассматриваемая последовательность будет $o(N^{-1/2})$, но больше, чем $o(N^{-1})$.

Эти обозначения можно обобщить до \textbf{стохастического порядка малости} последовательности случайных величин. Мы получаем обозначения $(O_p,o_p)$.

\begin{definition}[Стохастический порядок малости]
\label{def:A22}
	Последовательность случайных величин $b_N$ является $O_p(g(N))$ если
	\[
	0<\plim \frac{b_N}{g(N)}<\infty
	\]
	и является $o_p(g(N))$ если 
	\[
	\plim \frac{b_N}{g(N)}=0.
	\]
\end{definition}

Чаще всего $g(N)=N^{-c}$ для некоторой константы $c\geq 0$. Состоятельная оценка $\hat{\theta}$ параметра $\theta_0$ может быть записана в виде $\hat{\theta}=\theta_0+o_p(1)$, т.к. она равна $\theta_0$ плюс слагаемое, сходящееся к нулю по вероятности. Оценка $\hat{\theta}$, которая также является \textit{корень из $N$-состоятельной}, может быть записана в виде $\hat{\theta}=\theta_0+O_p(N^{-1/2})$, т.к. в этом случае $N^{1/2}(\hat{\theta}-\theta_0)=O_p(1)$.


\section{Прочие результаты}

Этот раздел содержит ключевые результаты об условном математическом ожидании, верные для конечных выборок. А также теоремы о возможности перемены порядка математического ожидания и преобразования.

\begin{theorem}[Закон повторного ожидания]
Если  $X$ и $Y$ --- случайные величины, то:
\[
\E[Y]=\E_X[\E_{Y|X}[Y|X]],
\]
где $\E[\cdot]$ --- безусловное математическое ожидание $Y$, $\E_X[\cdot]$ --- безусловное математическое ожидание по отношению к частной функции распределения $X$, и $\E_{Y|X}[\cdot|X]$ --- условное математическое ожидание $Y$ при известном $X$.	
 \end{theorem} 

Этот результат означает, что, взяв сначала условное ожидание $Y$ при известном $X$, а затем взяв ожидание по $X$, мы получим безусловное математическое ожидание $Y$. См. Rao (1973, стр. 97) для доказательства. Например, если $\E[u|\bx]=0$, то $\E[u]=\E_{\bx}[\E[u|\bx]]=\E_{\bx}[0]=0$.

\begin{theorem}[Разложение дисперсии]
Если  $X$ и $Y$ --- случайные величины, то:
\[
\V[Y]=\E_X[\V_{Y|X}[Y|X]]+\V_X[\E_{Y|X}[Y|X]],
\]
где $\V[Y]$ --- безусловная дисперсия $Y$, $\E_X[\cdot]$ --- безусловное математическое ожидание по отношению к частному функции распределения $X$, $\V_{Y|X}[Y|X]$ обозначает условную дисперсию $Y$ при известном $X$, $\V_X[\cdot]$ обозначает дисперсию по отношению к безусловному распределению $X$, и $\E_{Y|X}[\cdot|X]$ --- условное математическое ожидание $Y$ при известном $X$.	
\end{theorem}

Словами, безусловная дисперсия $Y$ равна сумме (1) ожидаемого значения (по $X$) условной дисперсии и (2) дисперсии (по $X$) условного математического ожидания. Простой способ запомнить это состоит в том, чтобы увидеть, что безусловная дисперсия равна $\E\V$ плюс $\V\E$. Доказательство можно найти в Rao (1973, стр. 97).

\begin{theorem}[Неравенство Йенсена]
Если у случайной величины $Z$ существует ожидание $\E[Z]$ и $g(\cdot)$ --- \textbf{выпуклая функция}, то
\[
g(\E[Z]) \leq \E[g(Z)].
\]
Если, наоборот, функция $g(\cdot)$ является \textbf{вогнутой}, то
\[
g(\E[Z]) \geq \E[g(Z)].
\]
\end{theorem}
Результат доказывается в Rao (1973, стр. 58) и очень полезен для нелинейных моделей. Он подчёркивает разницу между поведением среднего индивида и средним поведением. Например, предположим, что адекватна экспоненциальная модель с $\E[y|\bx]=\exp(\bx'\mathbf{\beta})$. Экспоненциальная функция вогнута, поэтому из неравенства Йенсена следует, что $\exp(\E[\bx'\bbeta])\geq \E[\exp(\bx'\bbeta]$. Условное среднее для индивида со средними характеристиками, равными $\bx=\E[x]$, превосходит безусловное среднее $\E[y]=\E[\E[y|\bx]]=\E[\exp(\bx'\bbeta)]$.

\section{Библиографические заметки}

Классический источник доказательств --- Rao (1973, стр. 108-130), который мы цитируем, где возможно. Многие результаты взяты из работ Amemiya (1985, глава 3) и White (2001a).

Учебники магистерского уровня такие, как Greene (2003), приводят обзор ключевых результатов. Более продвинутые книги Davidson и MacKinnon (1993), Hendry (1995), Ruud (2000) и Wooldridge (2002) излагают результаты на таком же уровне, как в этой книге, или более подробно. Davidson (1994) излагает длинный курс стохастической теории для эконометристов. Как уже отмечалось, терминология может отличаться в разных источниках, особенно это касается теорем Слуцкого и Крамера.


\chapter{Псевдо-случайные величины}

В этом приложении мы приводим функции плотности или вероятности и два первых момента наиболее распространённых одномерных распределений. Также мы предъявляем алгоритмы для генерации псевдо-случайных выборок из этих распределений.

\begin{table}[h]
\caption{\label{tab:pred} Функции плотности непрерывных случайных величин и их моменты}
\begin{tabular}{cp{6cm}p{4cm}c}
\hline 
\hline
Случайная величина & Ф. плотности $f(x)$ & Среднее, дисперсия & Ограничения\\ 
\hline 
Равномерная, $\mathcal{U}[a;b]$ & $1/(b-a)$ & $\frac{a+b}{2}$, $\frac{(a-b)^2}{12}$ & $b>a$ \\ 
Нормальная, $\mathcal{N}[\mu,\sigma^2]$ & $\frac{1}{\sigma \sqrt{2\pi}}e^{-\frac{(x-\mu)^2}{2\sigma^2}}$ & $\mu$, $\sigma^2$ & $\sigma^2>0$ \\ 
Экспоненциальная, $\mathcal{E}[\lambda]$ & $\lambda e^{-\lambda x}$ & $1/\lambda$, $1/\lambda^2$ & $\lambda>0$ \\ 
Гамма, $\mathcal{G}[a,b]$ & $\frac{1}{\Gamma(a)b^a}x^{a-1}e^{-x/b}$ & $ab$, $ab^2$ & $a,b>0$ \\ 
Бета, $\mathcal{B}[a,b]$ & $\frac{\Gamma(a+b)}{\Gamma(a)\Gamma(b)}x^{a-1}(1-x)^{b-1}$ & $\frac{a}{a+b}$, $\frac{ab}{(a+b)^2(a+b+1)}$ & $a,b>0$ \\ 
Логистическая, $\mathcal{L}[a,b]$ & $e^{-\frac{x-a}{b}}/[b(1+\exp(-\frac{x-a}{b}))^2]$ & $a$, $(b\pi)^2/3$ & $b>0, a \in \mathbb{R}$ \\ 
Хи-квадрат, $\chi^2(n)$ & $\frac{x^{n/2-1}\exp(-x/2)}{\Gamma(n/2)2^{n/2}}$ & $n$, $2n$ & \\ 
t, $t(v)$ & $\frac{\Gamma((v+1)/2)}{\Gamma(v/2)\sqrt{v\pi}}(1+x^2/v)^{-(v+1)/2}$ & $0$, $\frac{v}{v-2}$ при $v>2$ & $v\in\mathbb{N}$ \\ 
F, $F(w,v)$ & $\frac{\Gamma((v+w)/2)(v/w)^{v/2}}{\Gamma(v/2)\Gamma(w/2)}\times$

$x^{w/2-1}(x+v/w)^{-(v+w)/2}$ & $\frac{v}{v-2}$ при $v>2$  

$\frac{2v^2(v+w-2)}{w(v-4)(v-2)^2}$ при $v>4$ & $v,w\in\mathbb{N}$\\
\hline
\hline
\end{tabular} 
\end{table}

\begin{table}[h]
\caption{\label{tab:pred} Алгоритмы генерации непрерывных случайных величин}
\begin{tabular}{p{5cm}p{2cm}p{9cm}}
\hline 
\hline
Случайная величина & Диапазон 

значений & Алгоритм генерации \\
\hline
Равномерная, $\mathcal{U}[a;b]$ & $x \in [a;b]$ & Сгенерить $r$ из $\mathcal{U}[0;1]$ 

Присвоить $x=a+(b-a)r$ \\
Нормальная, $\mathcal{N}[\mu,\sigma^2]$ & $ x_1,x_2 \in \mathbb{R}$ &  Сгенерить $r_1$, $r_2$ из $\mathcal{U}[0;1]$ 

Присвоить:

\hspace{0.5cm} $x_1=\mu+\sigma\sqrt{-2\ln r_1}\cos(2\pi r_2)$ 

\hspace{0.5cm} $x_2=\mu+\sigma\sqrt{-2\ln r_1}\sin(2\pi r_2)$ \\ 
Экспоненциальная, $\mathcal{E}[\lambda]$ & $x \geq 0$  & Сгенерить $r$ из $\mathcal{U}[0;1]$ 

Присвоить $x=-\frac{\ln r}{\lambda}$ \\
Гамма, $\mathcal{G}[a,b]$ & $x \geq 0$ & Способ i (если $a \in \mathbb{N}$):

Сгенерить $y_1$, $y_2$, \ldots, $y_n$ из $\mathcal{E}[\lambda]$ 

Присвоить $x=\sum_{i=1}^a y_i$

Способ ii (если $a \notin \mathbb{N}$):

Представить $a$ в виде $a=m+q$, где $m\in \mathbb{N}$, $0<q<1$

Сгенерить $y_1$ из $\mathcal{B}[q,1-q]$

Сгенерить $y_2$ из $\mathcal{E}[1]$

Сгенерить $r_1$, $r_2$, \ldots, $r_n$ из $\mathcal{U}[0;1]$

Присвоить $x=-\frac{1}{\lambda}[\sum_{i=1}^m \ln r_i - y_1y_2]$ \\
Бета, $\mathcal{B}[a,b]$ & $x \in [0;1] $ & Способ i ($a,b \in \mathbb{N}$): 

Присвоить произвольное $k$

Сгенерить $y_1$ из $\mathcal{G}[k,a]$

Сгенерить $y_2$ из $\mathcal{G}[k,b]$

Присвоить $x=y_1/(y_1+y_2)$

Способ ii ($a \notin \mathbb{N}$ или $b \notin \mathbb{N}$): 

Присвоить $r_1=1$ и $r_2=1$

Повторять пока $(r_1^{1/a}+r_2^{1/b}) > 1$:

\hspace{0.5cm} Сгенерить $r_1$, $r_2$ из $\mathcal{U}[0;1]$

Присвоить $x=r_1^{1/a}/(r_1^{1/a}+r_2^{1/b})$ \\

Логистическая, $\mathcal{L}[a,b]$ & $x\in \mathbb{R}$ & Сгенерить $r$ из $\mathcal{U}[0;1]$ 

Присвоить $x=a+b\ln \frac{r}{1-r}$ \\ 
Хи-квадрат, $\chi^2(n)$ & $x\geq 0$ & Сгенерить $y_1$, $y_2$, \ldots, $y_n$ из $\mathcal{N}[0;1]$

Присвоить $x=\sum_{i=1}^n y_i$\\ 
t, $t(v)$ & $x\in \mathbb{R}$ & Сгенерить $y_1$ из $\mathcal{N}[0;1]$

Сгенерить $y_2$ из $\chi^2(v)$

Присвоить $x=y_1/\sqrt{y_2/v}$ \\ 
F, $F(w,v)$ & $x \geq 0$ & Сгенерить $y_1$ из $\chi^2(w)$

Сгенерить $y_2$ из $\chi^2(v)$

Присвоить $x=(y_1/w)/(y_2/v)$ \\
\hline
\hline 
\end{tabular}
\end{table}


\begin{table}[h]
\caption{\label{tab:pred}  Функции плотности непрерывных случайных величин и их моменты}
\begin{tabular}{p{5cm}ccp{3cm}}
\hline 
\hline
Случайная величина & Вероятность $f(x)$ & Среднее, дисперсия & Ограничения\\ 
\hline 
Биномиальная, $Bi[n,p]$ & $\binom{n}{x}p^x(1-p)^{n-x}$ & $np$, $np(1-p)$ & $n\in \mathbb{N}$, $p\in[0;1]$ \\
Пуассоновская, $\mathcal{P}[\lambda]$ & $e^{-\lambda}\lambda^x/x!$ & $\lambda$, $\lambda$ & $\lambda>0$ \\
Отрицательная биномиальная, $NB[n,p]$ & $\binom{n+x-1}{x}p^n(1-p)^x$ & $n(1-p)/p$, $n(1-p)/p^2$ & $0<p<1$, 

$n>1$\\
\hline
\hline
\end{tabular} 
\end{table}


\begin{table}[h]
\caption{\label{tab:pred} Алгоритмы генерации дискретных случайных величин}
\begin{tabular}{p{5cm}p{4cm}p{7cm}}
\hline
\hline 
Случайная величина & Диапазон значений & Алгоритм генерации \\
\hline
Биномиальная, $Bi[n,p]$ & $x=0,1,\ldots,n$ & Присвоить $x = 0$

Повторить $n$ раз:

\hspace{0.5cm}Сгенерить $r$ равномерно на $[0;1]$ 

\hspace{0.5cm}Если $r\leq p$, то присвоить $x = x+1$

Вывести $x$ \\

Пуассоновская, $\mathcal{P}[\lambda]$ & $x=0,1,\ldots$ & Присвоить $x = 0$ и $t = 0$

Повторять пока $t<\lambda$:

\hspace{0.5cm}Сгенерить экспоненциальную $y$  

\hspace{0.5cm}Присвоить $t=t+y$ и $x=x+1$

Вывести $x$ \\
Отрицательная биномиальная, $NB[n,p]$ & $x=0,1,\ldots$ & Сгенерить $\lambda$ из $\mathcal{G}(n,\frac{1-p}{p})$

Сгенерить $x$ из $\mathcal{P}(\lambda)$ 

Вывести $x$ \\
\hline
\hline
\end{tabular}
\end{table}




\end{document}
